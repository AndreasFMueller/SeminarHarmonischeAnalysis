%
% themen.tex -- slide template
%
% (c) 2021 Prof Dr Andreas Müller, OST Ostschweizer Fachhochschule
%
\bgroup
\begin{frame}[t]
\setlength{\abovedisplayskip}{5pt}
\setlength{\belowdisplayskip}{5pt}
\frametitle{Template}
\vspace{-20pt}
\begin{columns}[t,onlytextwidth]
\begin{column}{0.48\textwidth}
\begin{enumerate}
\item<2-> Gravitationswellensignale
\item<3-> JWST Phasing
\item<4-> CT-Bilder rekonstruieren
\item<5-> Sonogramm
\item<6-> Unschärferelation
\item<7-> Quoternionische Fourier-Transformation
\item<8-> Mellin-Transformation
\item<9-> Vorhersage der Gezeiten mit Analogcomputern
\item<10-> Kann man Fourier lernen?
\item<11-> Bluestein-FFT-Algorithmus
\end{enumerate}
\end{column}
\begin{column}{0.48\textwidth}
\begin{enumerate}
\setcounter{enumi}{10}
\item<12-> FFTW
\item<13-> Ptolemäus und Fourier
\item<14-> Milankovic-Zyklen
\item<15-> Spektrale Methoden
\item<16-> Autotune
\item<17-> Wavelets als orthogonale Funktionenfamilie
\item<18-> $C^*$-Algebren und Gelfand-Transformation
\end{enumerate}
\end{column}
\end{columns}
\end{frame}
\egroup
