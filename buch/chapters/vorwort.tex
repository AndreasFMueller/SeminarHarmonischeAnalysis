%
% vorwort.tex -- Vorwort zum Buch zum Seminar
%
% (c) 2019 Prof Dr Andreas Mueller, Hochschule Rapperswil
%
\chapter*{Vorwort}
\lhead{Vorwort}
\rhead{}
Dieses Buch entstand im Rahmen des Mathematischen Seminars
im Frühjahrssemester 2023 an der Ostschweizer Fachhochschule in Rapperswil.
Die Teilnehmer, Studierende der Studiengänge für Elektrotechnik
und Bauingenieurwesen
der OST, erarbeiteten nach einer Einführung in das Themengebiet jeweils
einzelne Aspekte des Gebietes in Form einer Seminararbeit, über
deren Resultate sie auch in einem Vortrag informierten. 

Im Frühjahr 2023 war die harmonische Analysis das Thema des Seminars.
Ausgangspunkt war dabei die Idee, dass das Skalarprodukt von Funktionen
die ``Ähnlichkeit'' zweier Funktionen misst.
Damit lässt sich sowohl die klassische Fourier-Theorie ebenso motivieren
wie auch Verallgemeinerungen auf viele Situationen, in denen Funktionen
durch Approximation aus einer orthonormierten Funktionenfamilie gewonnen
werden.
Diese Vorgehensweise liegt einer grossen Zahl interessanter mathematischer
Methoden zu Grunde und führt auf eine breite Palette nützlicher
Konzepte.
Ausser den klassischen Integraltransformationen treten daher in diesem
Buch auch neuere Transformationen wie die Radon- oder M\'endez-Transformation
und ihre Anwendungen auf.

In einigen Arbeiten wurde auch Code zur Demonstration der 
besprochenen Methoden und Resultate geschrieben, soweit
möglich und sinnvoll wurde dieser Code im Github-Repository
\index{Github-Repository}%
dieses Kurses%
\footnote{\url{https://github.com/AndreasFMueller/SeminarHarmonischeAnalysis.git}}
\cite{buch:repo}
abgelegt.
Im genannten Repository findet sich auch der Source-Code dieses
Skriptes, es wird hier unter einer Creative Commons Lizenz
zur Verfügung gestellt.

