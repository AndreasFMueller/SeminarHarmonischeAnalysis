%
% 2-trigo.tex -- trigonometrische Funktionen als Beispiel
%
% (c) 2023 Prof Dr Andreas Müller, OST Ostschweizer Fachschule
%
\section{Trigonometrische Funktionen
\label{buch:orthofkt:section:trigo}}
\kopfrechts{Trigonometrische Funktionen}
Es ist Joseph Fouriers Verdienst zu erkennen, dass nicht allzu pathologische
periodische Funktionen immer als Reihen von trigonometrischen
Funktionen ausgedrückt werden können.
Die Koeffizienten können als Skalarprodukte berechnet werden,
dies ist der Zugang, den wir aus heutiger Sicht wählen würden.
Fourier selbst hatte jedoch auch eine abenteuerliche Methode,
die Monome durch die trigonometrischen Funktionen auszudrücken.
Beliebige Funktionen können dann durch Polynome approximiert
werden, deren Terme wieder durch die trigonometrischen Reihen
für die Monome ersetzt werden können.
Hier möchten wir von der Sicht der orthogonalen Funktionen ausgehen
und müssen daher die Orthogonalitätseigenschaften der trigonometrischen
Funktionen untersuchen.

%
% Reelle trigonometrische Funktionen
%
\subsection{Reelle trigonometrische Funktionen}
Die Funktionen
\[
C_k(t) = \cos kt
\qquad\text{und}\qquad
S_k(t) = \sin kt
\]
sind $2\pi$-periodisch für jede ganze Zahl $k\in\mathbb{Z}$.
Natürlich ist $C_k(t)=C_{-k}(t)$ und $S_{-k}(t)=-S_k(t)$, so
dass nur natürliche Werte von $k$ untersucht werden müssen.
Ausserdem ist $S_0(t)=\sin 0t=0$,
die Funktion $S_0(t)$ kann also ignoriert werden.
Die Funktion $C_0(t)=\cos 0t=1$ ist die Konstante.

$2\pi$-periodische Funktionen sind eindeutig bestimmt durch die
Werte im Intervall $[-\pi,\pi]$, es müssen also nur Funktionen in
diesem kleineren Intervall approximiert werden können.
Das naheliegende Skalarprodukt ist daher
\[
\langle f,g\rangle
=
\int_{-\pi}^\pi f(t)g(t)\,dt.
\]

%
% Orthogonalität
%
\subsubsection{Orthogonalität}
Die Skalarprodukte der Funktionen $C_k$ und $S_k$ können mit Hilfe
der trigonometrischen Identitäten
\begin{align*}
C_k(t)C_l(t)
&=
\frac12\bigl(\cos(k-l)t+\cos(k+l)t\bigr)
\\
S_k(t)S_l(t)
&=
\frac12\bigl(\cos(k-l)t-\cos(k+l)t\bigr)
\\
C_k(t)S_l(t)
&=
\frac12\bigl(\sin(k+l)t-\sin(k-l)t\bigr)
\end{align*}
direkt berechnet werden.
Für $k\ne l$ gilt
\[
\renewcommand\arraycolsep{2pt}
\renewcommand\arraystretch{2.2}
\begin{array}{lclclcl}
\langle C_k,C_l\rangle
&=&
\displaystyle
\int_{-\pi}^\pi \cos kt\cos lt\,dt
&=&
\displaystyle
\frac12
\int_{-\pi}^\pi \cos(k-l)t+\cos(k+l)t\,dt
&=&
\displaystyle
\frac12\biggl[
\frac{\sin(k-l)t}{k-l} + \frac{\sin(k+l)t}{k+l}
\biggr]_{-\pi}^\pi,
\\
\langle S_k,S_l\rangle
&=&
\displaystyle
\int_{-\pi}^\pi \sin kt\sin lt\,dt
&=&
\displaystyle
\frac12
\int_{-\pi}^\pi \cos(k-l)t-\cos(k+l)t\,dt
&=&
\displaystyle
\frac12
\biggl[
\frac{\sin(k-l)}{k-l}-\frac{\sin(k+l)t}{k+l}
\biggr]_{-\pi}^\pi.
\end{array}
\]
Die Nullstellen der Sinus-Funktione sind ganzzahlige Vielfache von $\pi$,
daher sind $\langle C_k,C_l\rangle=0$ und $\langle S_k,S_l\rangle = 0$.
Da die Funktion $\cos kt \sin lt$ ungerade ist, folgt auch
\[
\langle C_k,S_l\rangle
=
\int_{-\pi}^\pi \cos kt\sin lt\,dt
=
0,
\]
ganz unabhängig davon, ob $k\ne l$ oder nicht.
Damit ist gezeigt, dass die trigonometrischen Funktionen orthogonal sind.

%
% Normierung
%
\subsubsection{Normierung}
Die trigonometrischen Funktionen sind nicht normiert.
Zur Berechnung der Norm werden die Formeln 
\[
\sin^2\vartheta = \frac12(1-\cos2\vartheta)
\qquad\text{und}\qquad
\cos^2\vartheta = \frac12(1+\cos2\vartheta)
\]
die Quadrate der trigonometrischen Funktionen benötigt.
Damit werden die Normen von $C_k$ und $S_k$ für $k>0$:
\begin{align*}
\|C_k\|^2
&=
\int_{-\pi}^\pi \cos^2kt\,dt
=
\frac12\int_{-\pi}^\pi 1+\cos 2kt\,dt
=
\pi
\\
\|S_k\|^2
&=
\int_{-\pi}^\pi \sin^2kt\,dt
=
\frac12\int_{-\pi}^\pi 1-\cos 2kt\,dt
=
\pi.
\end{align*}
Die Funktion $C_0(t)=1$ ist etwas speziell, sie hat die Norm
\[
\|C_0\|
=
\int_{-\pi}^\pi C_0(t)^2\,dt
=
\int_{-\pi}^\pi \,dt
=
2\pi.
\]
Damit können wir jetzt eine Hilbert-Basis zusammenstellen:
\[
\begin{aligned}
c_0(t) &= \frac{1}{\!\sqrt{2\pi}},\\
c_k(t) &= \frac{1}{\!\sqrt{\pi}} C_k(t) = \frac{1}{\!\sqrt{\pi}} \cos kt,\\
s_k(t) &= \frac{1}{\!\sqrt{\pi}} S_k(t) = \frac{1}{\!\sqrt{\pi}} \sin kt.\\
\end{aligned}
\]

%
% Fourier-Reihen
%
\subsubsection{Fourier-Reihen}
Die Funktionen $c_0(t)$, $c_k(t)$ und $s_k(t)$ bilden eine Hilbert-Basis.
Jede $2\pi$-periodische Funktion $f(t)\in L^2$ hat daher eine in
$L^2([-\pi,\pi])$ konvergente Reihenentwicklung der Form
\[
f(t)
=
\langle c_0,f\rangle c_0(t)
+
\sum_{k=1}^\infty
(
\langle c_k,f\rangle c_k(t)
+
\langle s_k,f\rangle s_k(t)
).
\]
Wir schreiben die einzelnen Terme aus um die Verbindung zu der in der
Theorie der Fourier-Reihen üblichen Notation herzustellen.
Der konstante Term ist
\[
\langle c_0,f\rangle c_0(t)
=
\int_{-\pi}^\pi \frac{1}{\!\sqrt{2\pi}} f(t)\,dt \frac{1}{\!\sqrt{2\pi}}
=
\frac12 \cdot \frac1{\pi} \int_{-\pi}^\pi f(t)\,dt
=
\frac12 a_0
\]
Für die Cosinus- und Sinus-Terme erhalten wir 
\begin{align*}
\langle c_k,f\rangle c_k(t)
&=
\frac{1}{\!\sqrt{\pi}}
\int_{-\pi}^\pi \cos kt f(t)\,dt \cdot \frac{1}{\!\sqrt{\pi}}\cos kt
=
\frac{1}{\pi} \cos kt\,f(t)\,dt\cdot \cos kt
=
a_k \cos kt
\\
\langle s_k,f\rangle s_k(t)
&=
\frac{1}{\!\sqrt{\pi}}
\int_{-\pi}^\pi \int_{-\pi}^\pi \sin kt f(t)\,dt \cdot \frac{1}{\!\sqrt{\pi}}\sin kt
=
\frac{1}{\pi} \int_{-\pi}^\pi \sin kt\,f(t)\,dt\cdot \sin kt
=
b_k \sin kt.
\end{align*}
Die Funktion $f(t)$ hat daher die $L^2$-konvergente Fourier-Reihe
\[
f(t) = \frac{a_0}{2} + \sum_{k=1}^\infty (a_k \cos kt + b_k \sin kt)
\qquad\text{mit}\qquad
\left\{
\quad
\begin{aligned}
a_k
&=
\frac{1}{\pi}
\int_{-\pi}^\pi f(t) \cos kt\,dt
&&\text{für $k\ge 0$,}
\\
b_k
&=
\frac{1}{\pi}
\int_{-\pi}^\pi f(t) \sin kt\,dt
&&\text{für $k> 0$.}
\end{aligned}
\right.
\]

%
% Andere Intervalle
%
\subsubsection{Andere Intervalle}
Die Untersuchungen dieses Abschnittes bis jetzt gingen von der 
Periode $2\pi$ aus.
Für ein Periodenintervall der Länge $2l$ können wir die Substitution
$x=(l/\pi)t$ verwenden.
Aus der $2l$-periodischen Funktion $f(x)$ entsteht dann die $2\pi$-periodische
Funktion $t\mapsto f(lt/\pi)$ mit der Fourier-Reihe
\[
f(x)
=
\frac{a_0}{2}
+
\sum_{k=1}^\infty
\biggl(a_k\cos\frac{k\pi x}{l} + b_k\sin\frac{k\pi x}{l}\biggr)
\qquad\text{mit}\qquad
\left\{\quad
\begin{aligned}
a_k
&=
\frac{1}{\pi}
\int_{-\pi}^\pi f(tl/\pi)\cos kt\,dt,
\\
b_k
&=
\frac{1}{\pi}
\int_{-\pi}^\pi f(tl/\pi)\sin kt\,dt.
\end{aligned}
\right.
\]
Indem man die Variablentransformation unter Verwendung von
$dt = (\pi/l)dx$ auch in den Integralen durchführt,
erhält man die Darstellung
\begin{align*}
a_k
&=
\frac{1}{\pi}\int_{-l}^l f(x)\cos\biggl(\frac{k\pi x}l\biggr)\cdot\frac{\pi}{l}\,dx
=
\frac{1}{l}\int_{-l}^l f(x) \cos\frac{k\pi x}{l} \,dx
\\
b_k
&=
\frac{1}{\pi}\int_{-l}^l f(x)\sin\biggl(\frac{k\pi x}l\biggr)\,\frac{\pi}{l}\,dx
=
\frac{1}{l}\int_{-l}^l f(x) \sin\frac{k\pi x}{l} \,dx
\end{align*}
Dasselbe Resultat kann man erhalten, indem man wie vorhin die
orthogonalen, $2l$-periodischen Funktionen
\[
\begin{aligned}
C^{l}_k(x) &= \cos\frac{k\pi x}{l} &&\text{für $k\ge 0$}
\\
S^{l}_k(x) &= \sin\frac{k\pi x}{l} &&\text{für $k>0$}
\end{aligned}
\]
bezüglich des Skalarproduktes
\[
\langle f,g\rangle_l
=
\int_{-l}^l f(x)g(x)\,dx
\]
orthonormiert.

%
% Komplexe Exponentialfunktionen
%
\subsection{Komplexe Exponentialfunktionen}
Die eulersche Formel
\[
e^{it} = \cos t + i\sin t
\]
stellt einen Zusammenhang zwischen den trigonometrischen Funktionen
und der komplexen Exponentialfunktion her.
Die Funktionen
\[
e_k(t) = e^{ikt} = \cos kt + i\sin kt\qquad\text{für $k\in\mathbb{Z}$}
\]
sind $2\pi$-periodisch.
Im Gegensatz zur Situation bei den trigonometrischen Funktionen können
wir hier nicht auf die negativen Indizes $k<0$ verzichten.
Für eine reelle Funktion ist lässt sich die Fourierreihe unter Verwendung
der Identitäten
\[
\cos kt
=
\frac{e_k(t) + e_{-k}(t)}{2}
=
\frac{e_k(t) + e_{-k}(t)}2
\qquad\text{und}\qquad
\sin kt
=
\frac{e_k(t) - e_{-k}(t)}{2i}
=
i\frac{-e_{k}(t)+e_k(t)}2
\]
als komplexe Reihe 
\begin{align*}
f(t)
&=
\frac{a_0}2
+\sum_{k=1}^\infty (a_k\cos kt + b_k \sin kt)
\\
&=
\frac{a_0}2 e_0(t)
+
\frac12
\sum_{k=1}^\infty
\bigl(
a_k(e_k(t)+e_{-k}(t))
+
ib_k(-e_k(t)+e_{-k}(t))
\bigr)
\\
&=
\frac{a_0}2e_0(t)
+
\sum_{k=1}^\infty (a_k-ib_k)e_k(t)
+
\sum_{k=1}^\infty (a_k+ib_k)e_{-k}(t)
\\
&=\sum_{k\in\mathbb{Z}} c_ke_k(t)
\quad\text{mit}\quad
\left\{\quad
\begin{aligned}
c_k    &= \frac{a_k-ib_k}2&&\text{für $k>0$}\\
c_0    &= \frac{a_0}2     &&                \\
c_{-k} &= \frac{a_k+ib_k}2&&\text{für $k>0$}\\
\end{aligned}
\right.
\end{align*}
schreiben.
Die Koeffizienten $c_k$ erfüllen also $c_{-k}=\overline{c_k}$
und können direkt aus der Theorie der reellen Fourier-Reihen
gewonnen werden.

%
% komplexes Skalarprodukt und Orthogonalität
%
\subsubsection{Komplexes Skalarprodukt und Orthogonalität}
Da die Funktionen $e_k(t)$ komplexertig sind, kann man damit
auch komplexwertige Funktionen approximieren. 
Für eine komplexwertige Funktion kann man nicht mehr erwarten, dass die
Koeffizienten $c_{-k}=\overline{c_k}$ erfüllen.
Damit wird die obige Rechnung zur Bestimmung der Koffizienten
viel komplizierter.
Stattdessen möchten wir ein Skalarprodukt für komplexwertige Funktionen
verwenden und eien Basis von komplexwertigen Funktionen, die 
bezüglich dieses Skalarproduktes orthonormiert sind.
Dann kann man die Koeffizienten $c_k$ wieder mit Hilfe des komplexen
Skalarproduktes ermitteln.

Wir versuchen es daher mit dem auf der Hand liegenden komplexen
Skalarprodukt
\begin{equation}
\langle f,g\rangle
=
\int_{-\pi}^\pi \overline{f(x)}g(x)\,dx
\label{buch:orthfunkt:komplexskalar}
\end{equation}
und berechnen die Skalarprodukte für $k\ne l$
\[
\langle e_k,e_l\rangle
=
\int_{-\pi}^\pi \overline{e^{ikt}} e^{ilt}\,dt
=
\int_{-\pi}^\pi e^{i(l-k)t}\,dt
=
\biggl[
\frac{1}{i(l-k)} e^{i(l-k)t}
\biggr]_{-\pi}^\pi
=
\frac{1}{i(l-k)}
[ e_{l-k} ]_{-\pi}^\pi
=
0
\]
da die Funktionen $e_k(t)$ $2\pi$-periodisch sind.

%
% Normierung
%
\subsubsection{Normierung}
Für das komplexe Skalarprodukt~\eqref{buch:orthfunkt:komplexskalar}
ist die Norm der Funktionen $e_k(t)$
\begin{align*}
\|e_k\|^2
&=
\int_{-\pi}^\pi |e_k(t)|^2\,dt
=
\int_{-\pi}^\pi dt
=
2\pi.
\end{align*}
Die Funktionen sind also nicht normiert.
Wir könnten die Funktionen anpassen, aber da wir uns beim Skalarprodukt
noch nicht wirklich festgelegt haben, können wir auch versuchen, letzteres
derart anzupassen, dass die Funktionen $e_k$ normiert sind.
Die Motivation dafür ist, dass die Funktionen andere Eigenschaften haben,
die sich später als nützlich erweisen werden, insbesondere die 
Potenzgesetze, die auf $e_k(s+t)=e^{ik(s+t)}=e^{iks}e^{its}=e_k(s)e_k(t)$
führen.

Um aus $e_k$ eine orthonormierte Funktionenfamilie zu machen, 
setzen wir
\begin{equation}
\langle f,g\rangle
=
\frac{1}{2\pi}
\int_{-\pi}^\pi \overline{f(t)} g(t)\,dt.
\end{equation}
Mit diesem Skalarprodukt wird die Norm $\|e_k\|=1$.
Da die Funktionen $e_k(t)$ eine Hilbert-Basis bilden, kann jetzt jede
$2\pi$-periodische komplexe Funktion $f(t)\in L^2$ durch die komplexe
Fourier-Reihe
\begin{align*}
f(t)
&=
\sum_{k\in\mathbb{Z}} c_k e^{ikt}
\qquad\text{mit}\quad
c_k
=
\langle e_k,f\rangle
=
\frac{1}{2\pi}
\int_{-\pi}^{\pi}  \overline{e^{ikt}} f(t)\,dt
=
\frac{1}{2\pi}
\int_{-\pi}^{\pi}  e^{-ikt} f(t)\,dt
\end{align*}
approximieren.

%
% Symmetrieeigenschaften
%
\subsection{Symmetrieeigenschaften
\label{buch:trigo:subsection:symmetrie}}
Die trigonometrischen Funktionen und die komplexen Exponentialfunktionen
haben eine zusätzliche Symmetrieeigenschaft, die sie gegenüber den früher
untersuchten orthogonalen Polynomen auszeichnet.

%
% Phasenverschiebung
%
\subsubsection{Phasenverschiebung}
Wir untersuchen, wie sich eine periodische Funktion unter einer
Translation um $\delta$ ändert.
Sei $T_\delta$ der Verschiebungsoperator definiert durch
\[
T_\delta
\colon 
f\mapsto T_\delta f
\qquad\text{mit}\quad
(T_\delta f)(t) = f(t+\delta).
\]
Es ist klar, dass der Operator $T_\delta$ ein linearer Operator ist.

Wendet man $T_\delta$ auf die Funktion $e_k(t)$ ergibt sich
\begin{align*}
(T_\delta e_k)(t)
&=
e_k(t+\delta)
=
e^{ik(t\delta)}
=
e^{ik\delta} e^{ikt}
=
e^{ik\delta} e_k(t)
\intertext{oder}
T_\delta e_k &= e^{ik\delta} e_k,
\end{align*}
die Funktion ist also eine Eigenfunktion des Operators $T_\delta$
zum Eigenwert $e^{ik\delta}$.

Für die trigonometrischen Funktionen ist die Situation etwas komplizierter,
man bekommt
\begin{align*}
(T_\delta C_k)(t)
&=
\cos k(t+\delta)
=
\cos kt \cos k\delta - \sin kt \sin k\delta
=
\cos k\delta C_k(t) - \sin k\delta S_k(t)
\\
(T_\delta S_k)(t)
&=
\sin k(t+\delta)
=
\cos kt\sin k\delta
+
\sin kt\cos k\delta
=
\sin k\delta C_k(t)
+
\cos k\delta S_k(t),
\end{align*}
die man auch in Matrixform als
\[
T_\delta
\begin{pmatrix}
C_k\\S_k
\end{pmatrix}
=
\begin{pmatrix}
\cos k\delta & -\sin k\delta \\
\sin k\delta &  \cos k\delta
\end{pmatrix}
\begin{pmatrix}
C_k\\S_k
\end{pmatrix}
\]
Die Funktionen $C_k$ und $S_k$ sind also nicht mehr Eigenfunktionen,
aber sie spannen einen zweidimensionalen invarianten Unterraum auf.

%
% Ableitung
%
\subsubsection{Ableitung}
Die Symmetrieeigenschaften äussern sich auch in den Ableitungen
der Funktionen.
Die Ableitungsfunktion ist der Grenzwert
\[
Df
=
f'
=
\lim_{\delta\to 0}
\frac{T_\delta f - f}{\delta}
=
\frac{d}{d\delta} T_\delta f.
\]
Für die komplexen Exponentialfunktionen $e_k$ finden wir
\[
De_k
=
e_k'
=
\frac{d}{d\delta} T_\delta e_k
=
\frac{d}{d\delta} e^{ik\delta} e_k
=
ik e_k.
\]
Die Funktionen $e_k$ sind also Eigenfunktionen des Ableitungsoperators
$D=d/dt$ zum Eigenwert $ik$.

Für die reellen trigonometrischen Funktionen ist dies nicht möglich,
die Ableitung von $c_k$ ist $Dc_k=-ks_k$ und
die Ableitung von $s_k$ ist $Ds_k=kc_k$.
Die Funktionen sind also nicht Eigenfunktionen des Ableitungsoperators.
Für die zweite Ableitung $D^2$ gilt jedoch
\begin{align*}
D^2c_k &= D(-ks_k) = -k^2c_k
\\
D^2s_k &= Dkc_k = -k^2s_k,
\end{align*}
die Funktionen $s_k$ und $c_k$ sind also Eigenfunktionen der zweiten
Ableitung zum Eigenwert $-k^2$.




