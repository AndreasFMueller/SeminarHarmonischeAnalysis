%
% chapter.tex -- Orthogonale Funktionen
%
% (c) 2022 Prof Dr Andreas Müller, Hochschule Rapperswil
%
% !TeX spellcheck = de_CH
\chapter{Orthogonale Funktionen
\label{buch:chapter:orthofunkt}}
\kopflinks{Orthogonale Funktionen}
Kapitel~\ref{buch:chapter:skalarprodukte} hat gezeigt, dass Funktionen
besonders leicht durch eine Familie von Funktionen dargestellt werden 
können, wenn die Funktionenfamilie aus orthonormierten Funktionen
bezüglich eines Skalarproduktes besteht.
Eine solche Familie von Funktionen kann im Prinzip mit Hilfe des
Gram-Schmidtschen Orthogonalisierungsverfahrens gefunden werden.
Dies wird im Abschnitt~\ref{buch:orthofkt:section:orthopol} für
Polynome durchgerechnet.
Diese Funktionenfamilien sind zwar nützlich und ermöglichen mit
der Grauss-Quadratur ein sehr effizientes numerisches
Integrationsverfahren, sie haben aber kaum andere nützliche Eigenschaften.

Die trigonometrischen Funktionen werden in
Abschnitt~\ref{buch:orthofkt:section:trigo} als orthogonal nachgewiesen,
diese Funktionen haben die zusätzliche Eigenschaft, dass Sie eine
Symmetrieeigenschaft haben.
Die Translation der Funktionen $t\mapsto \sin nt$ oder
$t\mapsto\cos nt$um einen Phasenwinkel ergibt eine Funktion
$\sin(nt+\delta)$ oder $\cos(nt+\delta)$, die sich wieder durch
$\sin nt$ und $\cos nt$ ausdrücken lässt.

Noch einfacher wird die Situation für die komplexen Exponentialfunktionen
$t\mapsto e^{int}$.
Die Verschiebung $e^{i(nt+\delta)}=e^{i\delta}e^{int}$ ist ganz
offensichlich ein Vielfaches der Ursprünglichen Funktion, sie ist
ein Eigenvektor des Verschiebungsoperators.
Aus der linearen Algebra ist bekannt, dass symmetrische Matrizen 
orthogonale Eigenvektoren haben.
Die Verallgemeinerung auf Funktionenräume liefert einen Mechanismus,
der automatisch orthogonale Funktionenfamilien liefert.
Die trigonometrischen Funktionen sind nur ein Spezialfall dieses
Mechanismus.

Das Sturm-Liouville-Problem handelt von der Lösung eines Typs
von Differentialgleichung zweiter Ordnung, die unter anderem auf
die trigonometrischen Funktionen führt.
Andere Spezialfälle reproduzieren die orthogonalen Polynome
von Abschnitt~\ref{buch:orthofkt:section:orthopol}.

Die Idee, orthogonale Funktionen als Eigenfunktionen eines
Operators zu finden, funktioniert auch für mehrere unabhängige
Variablen.
Elliptische partielle Differentialgleichungen können sehr oft
in eine Form gebracht werden, die eine orthogonale Funktionenfamilie
hervorbringt, welche dann verwendet werden kann, um Lösungen
linear zu kombinieren.

%
% 1-orthopol.tex -- trigonometrische Funktionen als Beispiel
%
% (c) 2023 Prof Dr Andreas Müller, OST Ostschweizer Fachschule
%
\section{Orthogonale Polynome
\label{buch:orthofkt:section:orthopol}}
\kopfrechts{Orthogonale Polynome}
Das Orthogonalisierungsverfahren von Gram-Schmidt kann dazu verwendet
werden, Familien von orthogonalen Polynomen zu konstruieren.

%
% Das Orthogonalisierungsverfahren von Gram-Schmidt
%
\subsection{Das Orthogonalisierungsverfahren von Gram-Schmidt}
Das Orthogonalisierungsverfahren von Gram-Schmidt geht aus von einer
Basis $a_1,a_2,\dots,a_n\subset V$ eines Vektorraums mit einem
Skalarprodukt, welches wie in Kapitel~\ref{buch:chapter:skalarprodukt}
mit $\langle\;\,,\;\rangle$ bezeichnet wird.
Aus den Vektoren $a_i$ können jetzt die orthogonalen Vektoren $b_i$
konstruiert werden, die die zusätzliche Eigenschaft haben, dass der
von $b_i$ immer im von den Vektoren $a_1,\dots,a_i$ aufgespannten
Unterraum von $V$ liegt.
Diese Bedingungen legen nur die Richtung von $b_i$ fest. 
Auch die Forderung, dass die Norm $\|b_i\|=1$ sein soll, lässt noch
zwei Möglichkeiten $\pm b_i$ offen.
Durch die Forderung, dass in der Linearkombination
\[
b_i = \alpha_i a_i + \alpha_{i-1} a_{i-1} + \dots + \alpha_1 a_1
\]
der Koeffiziente $\alpha_i >0$ sein muss.

Der Vektor $b_1$ muss also ein positives Vielfaches von $a_1$ sein:
\begin{align}
b_1
&=
\frac{a_1}{\|a_1\|}.
\notag
\intertext{Die folgenden Vektoren müssen um Vielfache der bereits
gefundenen Vektoren $b_k$ mit $k<i$ korrigiert werden, damit das
Skalarprodukt verschwindet.
Ausserdem müssen Sie normiert werden, was mit der Formel}
b_i
&=
\frac{
a_i - \langle b_{i-1},a_i\rangle b_{i-1} - \langle b_{i-2},a_i\rangle b_{i-2}
- \dots - \langle b_1,a_i\rangle b_1
}{
\|
a_i - \langle b_{i-1},a_i\rangle b_{i-1} - \langle b_{i-2},a_i\rangle b_{i-2}
- \dots - \langle b_1,a_i\rangle b_1
\|
}.
\label{buch:orthofkt:gs}
\end{align}
Der Koeffizient von $a_i$ in \eqref{buch:orthofkt:gs} ist
\begin{equation}
\frac{1}{
\|
a_i - \langle b_{i-1},a_i\rangle b_{i-1} - \langle b_{i-2},a_i\rangle b_{i-2}
- \dots - \langle b_1,a_i\rangle b_1
\|
}
> 0,
\label{buch:orthfkt:gsnenner}
\end{equation}
wie gefordert.

Der Nenner von \eqref{buch:orthfkt:gsnenner} kann nicht verschwinden.
Wäre der Nenner $=0$, dann könnte
\[
a_i
=
\langle b_{i-1},a_i\rangle b_{i-1} + \langle b_{i-2},a_i\rangle b_{i-2}
+ \dots + \langle b_1,a_i\rangle b_1
\]
als Linearkombination der Vektoren $b_1,\dots,b_{i-1}$ geschrieben
werden.
Diese liegen aber alle im von $a_1,\dots,a_{i-1}$ aufgespannten Raum,
es müsste also auch $a_i$ in diesem Raum liegen.
Dies widerspricht der Voraussetzung, dass die Vektoren eine Basis bilden.

%
% Hilbertbasis eines Prähilbertraumes
%
\subsubsection{Hilbertbasis eines Prähilbertraumes}
Das Orthogonalisierungsverfahren von Gram-Schmidt kann dazu verwendet
werden, eine Hilbertbasis in einem Hilbertraum $H$ zu konstruieren.
Ausgangspunkt ist wieder eine Folge von Vektoren $a_i\in H$.
Damit der Nenner von \eqref{buch:orthfkt:gsnenner} nicht verschwindet,
muss verlangt werden, das $a_i$ nicht im Raum aufgespannt von den
Vektoren $a_1,\dots,a_{i-1}$ liegen darf.
Dies garantiert, dass \eqref{buch:orthofkt:gs} eine Folge $b_i$ hervorbringt,
mit der Eigenschaft, dass jeder Vektor $a_i$ sich eindeutig durch die
$b_i$ ausdrücken lässt.
Wenn sich jeder Vektor in $H$ beliebig gut durch eine Folge von
Linearkombinationen der $a_i$ approximieren lässt, dann gilt dies auch
für die $b_i$.
Somit produziert das Orthogonalisierungsverfahren von Gram-Schmidt
automatisch eine Hilbertbasis.

%
% Legendre Polynome
%
\subsection{Legendre-Polynome}
Wir betrachten stetige Funktionen in $C([-1,1])$ auf dem Intervall
$[-1,1]$ mit reellen Werten und dem Skalarprodukt
\[
\langle f,g\rangle
=
\int_{-1}^1 f(x)g(x)\,dx.
\]
$C([-1,1])$ ist ein Prähilbertraum.

Um mit dem Verfahren von Gram-Schmidt eine Familie von orthogonalen
Funktionen in $C([-1,1])$ zu produzieren, wird zunächst eine Familie
von nicht notwendigerweise orthogonalen Funktionen benötigt, mit der
sich jede Funktion in $C([-1,1])$ beliebig genau approximieren
lässt.
Dies ist gemäss dem folgenden Satz mit Polnomen möglich.

\begin{satz}[Weierstrass]
Jeder stetige Funktion auf dem Intervall $[a,b]$ lässt sich beliebig
genau durch Polynome approximieren: ist $f\colon [a,b]\to\mathbb{R}$
eine stetige Funktion, dann gibt es Polynome $p_n(x)$, die gleichmässig
gegen $f(x)$ konvergieren, d.~h.~\(
\|p_n-f\| \to 0
\)
für $n\to\infty$.
\end{satz}

Da die Polynome gleichmässig gegen $f(x)$ konvergieren, folgt für die
$L^2$-Norm von $L^2([a,b])$
\[
\| f-p_n \|_2^2
=
\int_a^b |f(x)-p_n(x)|^2\,dx
\le 
\int_a^b \varepsilon^2 \,dx
=
\varepsilon |b-a|.
\]
Die Folge der approximierenden Polynome $p_n(x)$ konvergiert also
auch in der $L^2$-Norm.

Dies zeigt, dass als Ausgangsfunktionen für den Gram-Schmidt-Prozess
die Monome
\begin{equation}
a_0(x)=1=x^0,\;
a_1(x)=x,\;
a_2(x)=x^2,\;
a_3(x)=x^3,\;
\ldots\;
a_n(x)=x^n,\;
a_{n+1}(x)=x^{n+1},\;
\label{buch:orthfkt:monome}
\end{equation}
verwendet werden können, die alle linear unabhängig sind.
Der von den Funktionen $a_0,\dots,a_n$ erzeugte Unterraum von
$C([a,b])$ besteht aus den Polynomen vom Grad $\le n$.
Die Vektoren $b_n$ sind daher ebenfalls Vektoren vom Grad $n$.

Wir führen jetzt den Gram-Schmidt-Prozess für die Monome
\eqref{buch:orthfkt:monome}
und das $L^2$-Skalarprodukt auf dem Intervall $[-1,1]$ durch.
Wir bezeichnen das Polynome vom Grad $n$, welches erzeugt wird mit $p_n(x)$.

Die Funktion $p_0$ hat die Norm
\[
\|a_0\|^2
=
\int_{-1}^1 a_0(x)^2\,dx
=
\int_{-1}^1 \,dx
=
2
\qquad\Rightarrow\qquad
\|a_0\| = \!\sqrt{2}.
\]
Somit ist
\[
p_0(x) = \frac{a_0(x)}{\|a_0\|} = \frac{1}{\!\sqrt{2}}.
\]

Der Vektor $b_1$ wird berechnet durch
\[
b_1
=
\frac{
a_1-\langle b_0,a_1\rangle b_0
}{
\|
a_1-\langle b_0,a_1\rangle b_0
\|
},
\]
wir müssen daher zunächst das Skalarprodukt
\[
\langle b_0,a_1\rangle
=
\int_{-1}^1 b_0(x)a_1(x)\,dx
=
\int_{-1}^1 \frac{1}{\!\sqrt{2}} x\,dx
=
\frac{1}{\!\sqrt{2}} \biggl[\frac12x^2\biggr]_{-1}^1=0
\]
berechnen.
Es folgt dass $a_1$ bereits orthogonal zu $b_0$ ist, er muss also
nur noch normiert werden.
Dazu müssen wir die Norm
\[
\| a_1\|^2
=
\int_{-1}^1 a_1(x)^2\,dx
=
\int_{-1}^1 x^2\,dx
=
\biggl[\frac13x^3\biggr]_{-1}^1
=
\frac{2}{3}
\qquad\Rightarrow\qquad
\|a_1\|
=
\!\sqrt{\frac{2}{3}}
\]
verwenden und bekommen
\[
p_1(x) = \!\sqrt{\frac{3}{2}} x.
\]

Die dritte Funktion $p_2(x)$ entsteht durch Orthonormierung des
Monoms $x^2$, dazu sind zunächst die Skalarprodukte
\begin{align*}
\langle b_0,a_2\rangle
&=
\int_{-1}^1 \frac{1}{\!\sqrt{2}}x^2\,dx
=
\frac{1}{\!\sqrt{2}} [\frac13x^3]_{-1}^1
=
\frac{1}{\!\sqrt{2}} \frac{2}{3}
=
\frac{\!\sqrt{2}}{3},
\\
\langle b_1,a_2\rangle
&=
\int_{-1}^1 \!\sqrt{\frac{3}{2}} x \cdot x^2\,dx
= 
0.
\end{align*}
Die Funktion $p_2(x)$ wird daher erhalten durch Normierung des Polynoms
\[
x^2 - 
\frac{\!\sqrt{2}}{3} b_0(x)
=
x^2-
\frac{\!\sqrt{2}}{3}\cdot\frac{1}{\!\sqrt{2}}
=
x^2-\frac13.
\]
Dazu muss die Norm von $x^2-\frac13$ bestimmt werden, sie ist
\begin{align*}
\|x^2-\frac13\|^2
&=
\int_{-1}^1 \biggl(x^2-\frac13\biggr)^2\,dx
=
\int_{-1}^1 x^4-\frac23x^2+\frac{4}{9}\,dx
\\
&=
\biggl[\frac{x^5}{5} -\frac{2x^3}{9}+\frac{4x}{9}\biggr]_{-1}^1
=
\frac{2}{5} - \frac{4}{9} + \frac{8}{9}
=
\frac{18-20+40}{45}
=
\frac{38}{45}
\\
\Rightarrow\qquad
\|x^2-\frac13\|
&=
\!\sqrt{\frac{45}{38}}
\end{align*}
und somit
\[
p_2(x)
= 
\!\sqrt{\frac{38}{45}} \biggl(x^2-\frac13\biggr).
\]

Dieser Prozess lässt sich beliebig weiterführen, die Rechnung wird
allerdings immer aufwendiger.
Ein praktischeres Verfahren ist dringend nötig und wird sich
aus der Sturm-Liouville-Theorie von
Abschnitt~\ref{buch:orthofkt:section:sturmliouville} ergeben.
Etwas vereinfacht werden kann er durch die Beobachtung, dass die
Polynome $p_{2n}(x)$ immer gerade sind und die Polynome $p_{2n+1}(x)$
immer ungerade, so dass die Skalarprodukte 
$\langle p_{2k},p_{2l+1}\rangle=0$ sind, weil das Produkt
$p_{2k}(x)p_{2l+1}(x)$ eine ungerade Funktion ist und daher auf
dem symmetrischen Intervall $[-1,1]$ verschwindendes Integral hat.

Weiter sind in den Polynomen $p_n(x)$ die Normierungsfaktoren mit
ihren komplizierten Wurzeln eher unangenehm.
Da zum Beispiel in der Gauss-Quadratur als Anwendung der Polynome
nur die Nullstellen von Interesse sind, werden sie oft in einer
anderen Normierung gemäss der folgenden Definition tabelliert.

\begin{definition}
Die {\em Legendre-Polynome} $P_n(x)$ sind bezüglich des $L^2$-Skalarproduktes 
auf $[-1,1]$ orthogonale Polynome mit positivem Leitkoeffizienten und
$P_n(1)=1$.
\end{definition}

Aus obiger Konstruktion ist klar, dass $P_n(x) = p_n(x)/p_n(1)$.
Für die oben ausgerechneten ersten drei Polynome finden wir daher die
Legendre-Polynome
\begin{align*}
P_0(x) &= 1 \\
P_1(x) &= x \\
P_2(x) &= \frac12(3x^2-1).
\end{align*}


%
% Andere Skalarprodukte
%
\subsection{Andere Skalarprodukte}
Das gewöhnliche $L^2$-Skalarprodukt ist nur eines von vielen möglichen
Skalarprodukten für Funktionen auf dem Intervall $(-1,1)$.
Jeder Wahl eines Skalarproduktes führt auf einen andere Funktionenfamilie.

\subsubsection{Tschebyscheff-Polynome}
Das Skalarprodukt
\begin{equation}
\langle f,g\rangle_{\text{T}}
=
\int_{-1}^1
f(x)g(x)\frac{1}{\!\sqrt{1-x^2}}\, dx
\label{buch:orthfkt:tschebyscheff}
\end{equation}
führt auf eine Familie $t_n(x)$ von orthonormierten Polynome.
Da ähnlich wie bei den Legendre-Polynomen oft nur die Nullstellen
wichtig sind, wird auch hier die Normierung $T_n(x) = t_n(x)/t_n(1)$
verwendet, was die
sogenannten Tschebyscheff-Polynome
\(
T_n(x)
\)
ergibt,
die durch Gleichung
\[
T_n(\cos \vartheta) = \cos n\vartheta
\]
gegeben sind.

Tatsächlich kann man direkt nachrechnen, dass die Polynome $T_n$ und $T_m$
mit $n\ne m$ orthogonal sind.
Dazu berechnen wir
\begin{align*}
\langle T_n,T_m\rangle
&=
\int_{-1}^1 T_n(x) T_m(x)\,dx
\intertext{und verwenden die Substitution $x=\cos\vartheta$ mit
$dx/d\vartheta=-\sin\vartheta$.}
&=
\int_{\pi}^0
T_n(\cos\vartheta) T_m(\cos\vartheta)
\!\sqrt{1-\cos^2\vartheta}
\, \frac{-d\vartheta}{-\sin\vartheta}
\\
&=
\int_0^{\pi}
\cos n\vartheta \cos m\vartheta \,d\vartheta
\\
&=
\int_0^\pi
\frac12(\cos(n-m)\vartheta + \cos(n+m)\vartheta)
\,d\vartheta
=
\frac12
\biggl[
\frac{\sin(n-m)\vartheta}{n-m}+\frac{\sin(n+m)\vartheta}{n+m}
\biggr]_0^{\pi}
=
0,
\end{align*}
da Sinus für ganzzahlige Vielfache von $\pi$ verschwindet.

%
% Gewichtsfunktionen
%
\subsubsection{Gewichtsfunktionen}
Sei $w\colon(-1,1)\to \mathbb{R}$ eine stetige Funktion mit
nichtnegativen Werten, dann kann das zugehörige Skalarprodukt
\[
\langle f,g\rangle_w
=
\int_{-1}^1 f(x)g(x)\,w(x)\,dx
\]
definiert werden.
Die Funktion $w(x)$ heisst die {\em Gewichtsfunktion} des
Skalarproduktes.
\index{Gewichtsfunktion}%

Das Skalarprodukt 
\eqref{buch:orthfkt:tschebyscheff},
welches auf die Tschebyscheff-Polynome geführt hat, gehört zur 
Gewichtsfunktion
\[
w_T(x)
= 
\frac{1}{\!\sqrt{1-x^2}}.
\]

\begin{definition}
Die {\em Jacobi-Gewichtsfunktion} $w^{(\alpha,\beta)}(x)$ ist 
\[
w^{(\alpha,\beta)}
\colon
(-1,1) \to \mathbb{R}
:
x\mapsto (1-x)^\alpha(1+x)^\beta.
\]
Das zugehörige Skalarprodukt wird auch
\[
\langle f,g\rangle_{w^{(\alpha,\beta)}}
=
\langle f,g\rangle_{(\alpha,\beta)}
=
\int_{-1}^1 f(x)g(x) w^{(\alpha,\beta)}\,dx
=
\int_{-1}^1 f(x)g(x) (1-x)^\alpha(1+x)^\beta\,dx
\]
geschrieben.
\end{definition}

Die Familie orthogonaler Polynome, die sich zu diesem Skalarprodukt
ergeben, heissen {\em Jacobi-Polynome}.
Die Gewichtsfunktion der Tschebyscheff-Polynome ist ein Spezialfall,
nämlich
\[
w_T(x)
=
\frac{1}{\!\sqrt{1-x^2}}
=
\frac{1}{\!\sqrt{(1-x)(1-x)}}
=
(1-x)^{-\frac12}(1+x)^{-\frac12}
=
w^{(-\frac12,-\frac12)}(x).
\]
Die $(-\frac12,-\frac12)$-Jacobi-Polynome sind also die
Tschebyscheff-Polynome.

%
% Hermite-Polynome
%
\subsubsection{Hermite-Polynome}
Auch für unendliche Intervalle lassen sich Gewichtsfunktionen und
Familien von orthogonalen Funktionen angeben.
Da Polynome für $x\to\infty$ beliebig schnell anwachsen können,
muss die Gewichtsfunktion exponentiell schnell gegen $0$ gehen, damit
das Integral beschränkt bleibt.

Für $\mathbb{R}$ als Definitionsbereich der Polynome kann man die
{\em Hermite-Gewichtsfunktion}
\[
w(x) = e^{-\frac{x^2}2}
\]
verwenden, sie führt auf die {\em Hermite-Polynome}.

%
% Laguerre-Polynome
%
\subsubsection{Laguerre-Polynome}
Für den Definitionsbereich $\mathbb{R}_{>0}$ kann die
Laguerre-Gewichtsfunktion
\[
w_{\text{Laguerre}}(x)
=
e^{-x}
\]
verwendet werden, die auf die sogenannten {\em Laguerre-Polynome}
führt.
Die Theorie der Laguerre- und Hermite-Polynome wird ausführlich
in \cite{buch:mathsem-spezfunk} behandelt.


%
% 2-trigo.tex -- trigonometrische Funktionen als Beispiel
%
% (c) 2023 Prof Dr Andreas Müller, OST Ostschweizer Fachschule
%
\section{Trigonometrische Funktionen
\label{buch:orthofkt:section:trigo}}
\kopfrechts{Trigonometrische Funktionen}
Es ist Joseph Fouriers Verdienst zu erkennen, dass nicht allzu pathologische
periodische Funktionen immer als Reihen von trigonometrischen
Funktionen ausgedrückt werden können.
Die Koeffizienten können als Skalarprodukte berechnet werden,
dies ist der Zugang, den wir aus heutiger Sicht wählen würden.
Fourier selbst hatte jedoch auch eine abenteuerliche Methode,
die Monome durch die trigonometrischen Funktionen auszudrücken.
Beliebige Funktionen können dann durch Polynome approximiert
werden, deren Terme wieder durch die trigonometrischen Reihen
für die Monome ersetzt werden können.
Hier möchten wir von der Sicht der orthogonalen Funktionen ausgehen
und müssen daher die Orthogonalitätseigenschaften der trigonometrischen
Funktionen untersuchen.

%
% Reelle trigonometrische Funktionen
%
\subsection{Reelle trigonometrische Funktionen}
Die Funktionen
\[
C_k(t) = \cos kt
\qquad\text{und}\qquad
S_k(t) = \sin kt
\]
sind $2\pi$-periodisch für jede ganze Zahl $k\in\mathbb{Z}$.
Natürlich ist $C_k(t)=C_{-k}(t)$ und $S_{-k}(t)=-S_k(t)$, so
dass nur natürliche Werte von $k$ untersucht werden müssen.
Ausserdem ist $S_0(t)=\sin 0t=0$,
die Funktion $S_0(t)$ kann also ignoriert werden.
Die Funktion $C_0(t)=\cos 0t=1$ ist die Konstante.

$2\pi$-periodische Funktionen sind eindeutig bestimmt durch die
Werte im Intervall $[-\pi,\pi]$, es müssen also nur Funktionen in
diesem kleineren Intervall approximiert werden können.
Das naheliegende Skalarprodukt ist daher
\[
\langle f,g\rangle
=
\int_{-\pi}^\pi f(t)g(t)\,dt.
\]

%
% Orthogonalität
%
\subsubsection{Orthogonalität}
Die Skalarprodukte der Funktionen $C_k$ und $S_k$ können mit Hilfe
der trigonometrischen Identitäten
\begin{align*}
C_k(t)C_l(t)
&=
\frac12\bigl(\cos(k-l)t+\cos(k+l)t\bigr)
\\
S_k(t)S_l(t)
&=
\frac12\bigl(\cos(k-l)t-\cos(k+l)t\bigr)
\\
C_k(t)S_l(t)
&=
\frac12\bigl(\sin(k+l)t-\sin(k-l)t\bigr)
\end{align*}
direkt berechnet werden.
Für $k\ne l$ gilt
\[
\renewcommand\arraycolsep{2pt}
\renewcommand\arraystretch{2.2}
\begin{array}{lclclcl}
\langle C_k,C_l\rangle
&=&
\displaystyle
\int_{-\pi}^\pi \cos kt\cos lt\,dt
&=&
\displaystyle
\frac12
\int_{-\pi}^\pi \cos(k-l)t+\cos(k+l)t\,dt
&=&
\displaystyle
\frac12\biggl[
\frac{\sin(k-l)t}{k-l} + \frac{\sin(k+l)t}{k+l}
\biggr]_{-\pi}^\pi,
\\
\langle S_k,S_l\rangle
&=&
\displaystyle
\int_{-\pi}^\pi \sin kt\sin lt\,dt
&=&
\displaystyle
\frac12
\int_{-\pi}^\pi \cos(k-l)t-\cos(k+l)t\,dt
&=&
\displaystyle
\frac12
\biggl[
\frac{\sin(k-l)}{k-l}-\frac{\sin(k+l)t}{k+l}
\biggr]_{-\pi}^\pi.
\end{array}
\]
Die Nullstellen der Sinus-Funktione sind ganzzahlige Vielfache von $\pi$,
daher sind $\langle C_k,C_l\rangle=0$ und $\langle S_k,S_l\rangle = 0$.
Da die Funktion $\cos kt \sin lt$ ungerade ist, folgt auch
\[
\langle C_k,S_l\rangle
=
\int_{-\pi}^\pi \cos kt\sin lt\,dt
=
0,
\]
ganz unabhängig davon, ob $k\ne l$ oder nicht.
Damit ist gezeigt, dass die trigonometrischen Funktionen orthogonal sind.

%
% Normierung
%
\subsubsection{Normierung}
Die trigonometrischen Funktionen sind nicht normiert.
Zur Berechnung der Norm werden die Formeln 
\[
\sin^2\vartheta = \frac12(1-\cos2\vartheta)
\qquad\text{und}\qquad
\cos^2\vartheta = \frac12(1+\cos2\vartheta)
\]
die Quadrate der trigonometrischen Funktionen benötigt.
Damit werden die Normen von $C_k$ und $S_k$ für $k>0$:
\begin{align*}
\|C_k\|^2
&=
\int_{-\pi}^\pi \cos^2kt\,dt
=
\frac12\int_{-\pi}^\pi 1+\cos 2kt\,dt
=
\pi
\\
\|S_k\|^2
&=
\int_{-\pi}^\pi \sin^2kt\,dt
=
\frac12\int_{-\pi}^\pi 1-\cos 2kt\,dt
=
\pi.
\end{align*}
Die Funktion $C_0(t)=1$ ist etwas speziell, sie hat die Norm
\[
\|C_0\|
=
\int_{-\pi}^\pi C_0(t)^2\,dt
=
\int_{-\pi}^\pi \,dt
=
2\pi.
\]
Damit können wir jetzt eine Hilbert-Basis zusammenstellen:
\[
\begin{aligned}
c_0(t) &= \frac{1}{\!\sqrt{2\pi}},\\
c_k(t) &= \frac{1}{\!\sqrt{\pi}} C_k(t) = \frac{1}{\!\sqrt{\pi}} \cos kt,\\
s_k(t) &= \frac{1}{\!\sqrt{\pi}} S_k(t) = \frac{1}{\!\sqrt{\pi}} \sin kt.\\
\end{aligned}
\]

%
% Fourier-Reihen
%
\subsubsection{Fourier-Reihen}
Die Funktionen $c_0(t)$, $c_k(t)$ und $s_k(t)$ bilden eine Hilbert-Basis.
Jede $2\pi$-periodische Funktion $f(t)\in L^2$ hat daher eine in
$L^2([-\pi,\pi])$ konvergente Reihenentwicklung der Form
\[
f(t)
=
\langle c_0,f\rangle c_0(t)
+
\sum_{k=1}^\infty
(
\langle c_k,f\rangle c_k(t)
+
\langle s_k,f\rangle s_k(t)
).
\]
Wir schreiben die einzelnen Terme aus, um die Verbindung zu der in der
Theorie der Fourier-Reihen üblichen Notation herzustellen.
Der konstante Term ist
\[
\langle c_0,f\rangle c_0(t)
=
\int_{-\pi}^\pi \frac{1}{\!\sqrt{2\pi}} f(t)\,dt \frac{1}{\!\sqrt{2\pi}}
=
\frac12 \cdot \frac1{\pi} \int_{-\pi}^\pi f(t)\,dt
=
\frac12 a_0
\]
Für die Cosinus- und Sinus-Terme erhalten wir 
\begin{align*}
\langle c_k,f\rangle c_k(t)
&=
\frac{1}{\!\sqrt{\pi}}
\int_{-\pi}^\pi \cos kt f(t)\,dt \cdot \frac{1}{\!\sqrt{\pi}}\cos kt
=
\frac{1}{\pi} \cos kt\,f(t)\,dt\cdot \cos kt
=
a_k \cos kt
\\
\langle s_k,f\rangle s_k(t)
&=
\frac{1}{\!\sqrt{\pi}}
\int_{-\pi}^\pi \int_{-\pi}^\pi \sin kt f(t)\,dt \cdot \frac{1}{\!\sqrt{\pi}}\sin kt
=
\frac{1}{\pi} \int_{-\pi}^\pi \sin kt\,f(t)\,dt\cdot \sin kt
=
b_k \sin kt.
\end{align*}
Die Funktion $f(t)$ hat daher die $L^2$-konvergente Fourier-Reihe
\[
f(t) = \frac{a_0}{2} + \sum_{k=1}^\infty (a_k \cos kt + b_k \sin kt)
\qquad\text{mit}\qquad
\left\{
\quad
\begin{aligned}
a_k
&=
\frac{1}{\pi}
\int_{-\pi}^\pi f(t) \cos kt\,dt
&&\text{für $k\ge 0$,}
\\
b_k
&=
\frac{1}{\pi}
\int_{-\pi}^\pi f(t) \sin kt\,dt
&&\text{für $k> 0$.}
\end{aligned}
\right.
\]

%
% Parseval-Gleichung
%
\subsubsection{Parseval-Gleichung}
Da die Abbildung einer Funktion auf die Folge der Fourier-Koeffizienten
eine Isometrie in den Raum $l^2$ ist, folgt die {\em Parseval-Gleichung}
\index{Parseval-Gleichung}
\begin{equation}
\| f\|^2
=
\frac{a_0^2}{2}
+
\sum_{k=1}^\infty(|a_k|^2 + |b_k|^2).
\label{buch:orthofkt:trigo:eqn:parseval}
\end{equation}

%
% Andere Intervalle
%
\subsubsection{Andere Intervalle}
Die Untersuchungen dieses Abschnittes bis jetzt gingen von der 
Periode $2\pi$ aus.
Für ein Periodenintervall der Länge $2l$ können wir die Substitution
$x=(l/\pi)t$ verwenden.
Aus der $2l$-periodischen Funktion $f(x)$ entsteht dann die $2\pi$-periodische
Funktion $t\mapsto f(lt/\pi)$ mit der Fourier-Reihe
\[
f(x)
=
\frac{a_0}{2}
+
\sum_{k=1}^\infty
\biggl(a_k\cos\frac{k\pi x}{l} + b_k\sin\frac{k\pi x}{l}\biggr)
\qquad\text{mit}\qquad
\left\{\quad
\begin{aligned}
a_k
&=
\frac{1}{\pi}
\int_{-\pi}^\pi f(tl/\pi)\cos kt\,dt,
\\
b_k
&=
\frac{1}{\pi}
\int_{-\pi}^\pi f(tl/\pi)\sin kt\,dt.
\end{aligned}
\right.
\]
Indem man die Variablentransformation unter Verwendung von
$t=x\pi/l$ und $dt = (\pi/l)dx$ auch in den Integralen durchführt,
erhält man die Darstellung
\begin{align*}
a_k
&=
\frac{1}{\pi}\int_{-l}^l f(x)\cos\biggl(\frac{k\pi x}l\biggr)\cdot\frac{\pi}{l}\,dx
=
\frac{1}{l}\int_{-l}^l f(x) \cos\frac{k\pi x}{l} \,dx
\\
b_k
&=
\frac{1}{\pi}\int_{-l}^l f(x)\sin\biggl(\frac{k\pi x}l\biggr)\,\frac{\pi}{l}\,dx
=
\frac{1}{l}\int_{-l}^l f(x) \sin\frac{k\pi x}{l} \,dx.
\end{align*}
Dasselbe Resultat kann man erhalten, indem man wie vorhin die
orthogonalen, $2l$-periodischen Funktionen
\[
\begin{aligned}
C^{l}_k(x) &= \cos\frac{k\pi x}{l} &&\text{für $k\ge 0$}
\\
S^{l}_k(x) &= \sin\frac{k\pi x}{l} &&\text{für $k>0$}
\end{aligned}
\]
bezüglich des Skalarproduktes
\[
\langle f,g\rangle_l
=
\int_{-l}^l f(x)g(x)\,dx
\]
orthonormiert.

%
% Komplexe Exponentialfunktionen
%
\subsection{Komplexe Exponentialfunktionen}
Die eulersche Formel
\[
e^{it} = \cos t + i\sin t
\]
stellt einen Zusammenhang zwischen den trigonometrischen Funktionen
und der komplexen Exponentialfunktion her.
Die Funktionen
\[
e_k(t) = e^{ikt} = \cos kt + i\sin kt\qquad\text{für $k\in\mathbb{Z}$}
\]
sind $2\pi$-periodisch.
Im Gegensatz zur Situation bei den trigonometrischen Funktionen können
wir hier nicht auf die negativen Indizes $k<0$ verzichten.
Für eine reelle Funktion lässt sich die Fourier-Reihe unter Verwendung
der Identitäten
\[
\cos kt
=
\frac{e_k(t) + e_{-k}(t)}{2}
=
\frac{e_k(t) + e_{-k}(t)}2
\qquad\text{und}\qquad
\sin kt
=
\frac{e_k(t) - e_{-k}(t)}{2i}
=
i\frac{-e_{k}(t)+e_k(t)}2
\]
als komplexe Reihe 
\begin{align*}
f(t)
&=
\frac{a_0}2
+\sum_{k=1}^\infty (a_k\cos kt + b_k \sin kt)
\\
&=
\frac{a_0}2 e_0(t)
+
\frac12
\sum_{k=1}^\infty
\bigl(
a_k(e_k(t)+e_{-k}(t))
+
ib_k(-e_k(t)+e_{-k}(t))
\bigr)
\\
&=
\frac{a_0}2e_0(t)
+
\sum_{k=1}^\infty (a_k-ib_k)e_k(t)
+
\sum_{k=1}^\infty (a_k+ib_k)e_{-k}(t)
\\
&=\sum_{k\in\mathbb{Z}} c_ke_k(t)
\quad\text{mit}\quad
\left\{\quad
\begin{aligned}
c_k    &= \frac{a_k-ib_k}2&&\text{für $k>0$}\\
c_0    &= \frac{a_0}2     &&                \\
c_{-k} &= \frac{a_k+ib_k}2&&\text{für $k>0$}\\
\end{aligned}
\right.
\end{align*}
schreiben.
Die Koeffizienten $c_k$ erfüllen also $c_{-k}=\overline{c_k}$
und können direkt aus der Theorie der reellen Fourier-Reihen
gewonnen werden.

%
% komplexes Skalarprodukt und Orthogonalität
%
\subsubsection{Komplexes Skalarprodukt und Orthogonalität}
Da die Funktionen $e_k(t)$ komplexwertig sind, kann man damit
auch komplexwertige Funktionen approximieren. 
Für eine komplexwertige Funktion kann man nicht mehr erwarten, dass die
Koeffizienten $c_{-k}=\overline{c_k}$ erfüllen.
Damit wird die obige Rechnung zur Bestimmung der Koffizienten
viel komplizierter.
Stattdessen möchten wir ein Skalarprodukt für komplexwertige Funktionen
und eine Basis von komplexwertigen Funktionen verwenden, die 
bezüglich dieses Skalarproduktes orthonormiert sind.
Dann kann man die Koeffizienten $c_k$ wieder mit Hilfe des komplexen
Skalarproduktes ermitteln.

Wir versuchen es daher mit dem auf der Hand liegenden komplexen
Skalarprodukt
\begin{equation}
\langle f,g\rangle
=
\int_{-\pi}^\pi \overline{f(x)}g(x)\,dx
\label{buch:orthfunkt:komplexskalar}
\end{equation}
und berechnen die Skalarprodukte für $k\ne l$
\[
\langle e_k,e_l\rangle
=
\int_{-\pi}^\pi \overline{e^{ikt}} e^{ilt}\,dt
=
\int_{-\pi}^\pi e^{i(l-k)t}\,dt
=
\biggl[
\frac{1}{i(l-k)} e^{i(l-k)t}
\biggr]_{-\pi}^\pi
=
\frac{1}{i(l-k)}
[ e_{l-k} ]_{-\pi}^\pi
=
0
\]
da die Funktionen $e_k(t)$ periodisch mit Periode $2\pi$ sind.

%
% Normierung
%
\subsubsection{Normierung}
Für das komplexe Skalarprodukt~\eqref{buch:orthfunkt:komplexskalar}
ist die Norm der Funktionen $e_k(t)$
\begin{align*}
\|e_k\|^2
&=
\int_{-\pi}^\pi |e_k(t)|^2\,dt
=
\int_{-\pi}^\pi dt
=
2\pi.
\end{align*}
Die Funktionen sind also nicht normiert.
Wir könnten die Funktionen anpassen, aber da wir uns beim Skalarprodukt
noch nicht wirklich festgelegt haben, können wir auch versuchen, letzteres
derart anzupassen, dass die Funktionen $e_k$ normiert sind.
Die Motivation dafür ist, dass die Funktionen andere Eigenschaften haben,
die sich später als nützlich erweisen werden, insbesondere die 
Potenzgesetze, die auf $e_k(s+t)=e^{ik(s+t)}=e^{iks}e^{its}=e_k(s)e_k(t)$
führen.

Um aus $e_k$ eine orthonormierte Funktionenfamilie zu machen, 
setzen wir
\begin{equation}
\langle f,g\rangle
=
\frac{1}{2\pi}
\int_{-\pi}^\pi \overline{f(t)} g(t)\,dt.
\end{equation}
Mit diesem Skalarprodukt wird die Norm $\|e_k\|=1$.
Da die Funktionen $e_k(t)$ eine Hilbert-Basis bilden, kann jetzt jede
$2\pi$-periodische komplexe Funktion $f(t)\in L^2$ durch die komplexe
Fourier-Reihe
\begin{align*}
f(t)
&=
\sum_{k\in\mathbb{Z}} c_k e^{ikt}
\qquad\text{mit}\quad
c_k
=
\langle e_k,f\rangle
=
\frac{1}{2\pi}
\int_{-\pi}^{\pi}  \overline{e^{ikt}} f(t)\,dt
=
\frac{1}{2\pi}
\int_{-\pi}^{\pi}  e^{-ikt} f(t)\,dt
\end{align*}
approximieren.

%
% Symmetrieeigenschaften
%
\subsection{Symmetrieeigenschaften
\label{buch:trigo:subsection:symmetrie}}
Die trigonometrischen Funktionen und die komplexen Exponentialfunktionen
haben eine zusätzliche Symmetrieeigenschaft, die sie gegenüber den früher
untersuchten orthogonalen Polynomen auszeichnet.

%
% Phasenverschiebung
%
\subsubsection{Phasenverschiebung}
Wir untersuchen, wie sich eine periodische Funktion unter einer
Translation um $\delta$ ändert.
Sei $T_\delta$ der Verschiebungsoperator definiert durch
\[
T_\delta
\colon 
f\mapsto T_\delta f
\qquad\text{mit}\quad
(T_\delta f)(t) = f(t+\delta).
\]
Es ist klar, dass der Operator $T_\delta$ ein linearer Operator ist.

Wendet man $T_\delta$ auf die Funktion $e_k(t)$ ergibt sich
\begin{align*}
(T_\delta e_k)(t)
&=
e_k(t+\delta)
=
e^{ik(t+\delta)}
=
e^{ik\delta} e^{ikt}
=
e^{ik\delta} e_k(t)
\intertext{oder}
T_\delta e_k &= e^{ik\delta} e_k,
\end{align*}
die Funktion ist also eine Eigenfunktion des Operators $T_\delta$
zum Eigenwert $e^{ik\delta}$.

Für die trigonometrischen Funktionen ist die Situation etwas komplizierter,
man bekommt
\begin{align*}
(T_\delta C_k)(t)
&=
\cos k(t+\delta)
=
\cos kt \cos k\delta - \sin kt \sin k\delta
=
\cos k\delta \cdot C_k(t) - \sin k\delta \cdot S_k(t)
\\
(T_\delta S_k)(t)
&=
\sin k(t+\delta)
=
\cos kt\sin k\delta
+
\sin kt\cos k\delta
=
\sin k\delta \cdot C_k(t)
+
\cos k\delta \cdot S_k(t),
\end{align*}
die man auch in Matrixform als
\[
T_\delta
\begin{pmatrix}
C_k\\S_k
\end{pmatrix}
=
\begin{pmatrix}
\cos k\delta & -\sin k\delta \\
\sin k\delta & \phantom{-} \cos k\delta
\end{pmatrix}
\begin{pmatrix}
C_k\\S_k
\end{pmatrix}
\]
Die Funktionen $C_k$ und $S_k$ sind also nicht mehr Eigenfunktionen,
aber sie spannen einen zweidimensionalen invarianten Unterraum auf.

%
% Ableitung
%
\subsubsection{Ableitung}
Die Symmetrieeigenschaften äussern sich auch in den Ableitungen
der Funktionen.
Die Ableitungsfunktion ist der Grenzwert
\index{Ableitungsfunktion}%
\[
Df
=
f'
=
\lim_{\delta\to 0}
\frac{T_\delta f - f}{\delta}
=
\frac{d}{d\delta} T_\delta f.
\]
Für die komplexen Exponentialfunktionen $e_k$ finden wir
\[
De_k
=
e_k'
=
\frac{d}{d\delta} T_\delta e_k
=
\frac{d}{d\delta} e^{ik\delta} e_k
=
ik e_k.
\]
Die Funktionen $e_k$ sind also Eigenfunktionen des Ableitungsoperators
$D=d/dt$ zum Eigenwert $ik$.

Für die reellen trigonometrischen Funktionen ist dies nicht möglich,
die Ableitung von $c_k$ ist $Dc_k=-ks_k$ und
die Ableitung von $s_k$ ist $Ds_k=kc_k$.
Die Funktionen sind also nicht Eigenfunktionen des Ableitungsoperators.
Für die zweite Ableitung $D^2$ gilt jedoch
\begin{align*}
D^2c_k &= D(-ks_k) = -k^2c_k
\\
D^2s_k &= Dkc_k = -k^2s_k,
\end{align*}
die Funktionen $s_k$ und $c_k$ sind also Eigenfunktionen der zweiten
Ableitung zum Eigenwert $-k^2$.





%
% 3-sa.tex -- selbstadjungierte Operatoren
%
% (c) 2022 Prof Dr Andreas Müller, OST Ostschweizer Fachhochschule
%
\section{Selbstadjungierte Operatoren
\label{buch:orthofkt:section:sa}}
\kopfrechts{Selbstadjungierte Operatoren}
In der linearen Algebra lernt man, dass die Eigenvektoren
symmetrischer Matrizen zu verschiedenen Eigenwerten orthogonal
sind.
Abschnitt~\ref{buch:trigo:subsection:symmetrie}
zeigt, dass die trigonometrischen Funktionen und die
komplexen Exponentialfunktionen als Eigenfunktionen eines
Operators auftreten können.
In diesem Abschnitt soll gezeigt werden, dass diese Situation
keine Ausnahme ist und zu anderen orthogonalen Funktionenfamilien
führen kann, wenn man die Begriffe der linearen Algebra auf
die Situation des Hilbert-Raumes ausweiten kann.

%
% Der adjungierte Operator
%
\subsection{Der adjungierte Operator}
In einem Hilbert-Raum lässt sich zu jedem beschränkten linearen
Operator der sogenannte adjungierte Operator konstruieren, der
bis auf Konjugation der Transponierten einer Matrix entspricht.

%
% Operatornorm
%
\subsubsection{Operatornorm}
Sei $A:H\to H$ eine lineare Abbildung oder {\em Operator} auf dem
Hilbert-Raum $H$.
Damit die Abbildung stetig ist, müssen kleine Vektoren $v$ unter der
Wirkung $A$ einigermassen klein bleiben.
Dies bedeutet, dass es eine Zahl $M$ gibt derart, dass $\|Ax\| \le M\|x\|$.
Man nennt dies auch einen {\em beschränkten Operator}.
\index{Operator!beschränkt}%
Die kleinstmögliche solche obere Schranke ist
die sogenannte {\em Norm}
\index{Norm eines Operators}%
\[
\|A\|
=
\sup_{x\in H\setminus\{0\}} \frac{\|Ax\|}{\|x\|}
<
\infty
\]
des Operators $A$.

In einem Hilbert-Raum kann man die Norm auch mit der Cauchy-Schwarz-Ungleichung
charakterisieren.
$\|A\|$ ist die kleinste Zahl $M$, für die 
\begin{equation}
|\langle Ax,y\rangle| \le M
\label{buch:orthofkt:sa:eqn:normcauchyschwarz}
\end{equation}
für alle Vektoren $x$ und $y$ der Länge $1$ ist.
Dies ist gleichbedeutend mit
\begin{equation}
\|A\|
=
\sup_{x,y\in H, \|x\|=1, \|y\|=1}
\langle Ax,y\rangle.
\label{buch:orthofkt:sa:eqn:normcauchyschwarzsup}
\end{equation}
Diese Beschreibung der Norm wird im nächsten Abschnitt nützlich sein zur
Bestimmung der Norm des adjungierten Operators.

%
% Der Operator A^*
%
\subsubsection{Der Operator $A^*$}
Sei jetzt $x\in H$ ein Vektor.
Die Funktion
\[
l
\colon
y\mapsto \langle Ay,x\rangle
\]
ist linear und stetig, die Norm ist
\[
|l(y)|
=
|\langle x, Ay\rangle|
\le
\|x\|
\cdot
\|Ay\|
\le
\|x\| \cdot \|A\| \cdot \|y\|
.
\]
Nach dem Darstellungssatz~\ref{buch:skalarprodukt:hilbertraum:satz:riesz}
von Riesz folgt, dass es einen Vektor $v$ derart, dass
\[
l(y) = \langle v,y\rangle.
\]
Der Vektor $v$ wird auch mit $A^*x$ bezeichnet.
Die Abbildung $x\mapsto A^*x$ ist wieder ein linearer Operator, denn
für Vektoren $x_1,x_2\in H$ und $\lambda\in\mathbb{R}$ gilt
\begin{align*}
\langle A^*(x_1+x_2),y\rangle
&=
\langle x_1+x_2,Ay\rangle
\\
&=
\langle x_1,Ay\rangle
+
\langle x_2,Ay\rangle
\\
&=
\langle A^*x_1,y\rangle
+
\langle A^*x_2,y\rangle
\\
&=
\langle A^*x_1+A^*x_2,y\rangle
&&\Rightarrow&
A^*(x_1+x_2)&=A^*x_1 + A^*x_2
\\
\langle A^*\lambda x,y\rangle
&=
\langle \lambda x,Ay\rangle
\\
&=
\lambda \langle x,Ay\rangle
\\
&=
\lambda \langle A^*x,y\rangle
&&\Rightarrow&
A^*\lambda x&=\lambda A^*x.
\end{align*}
Mit Hilfe von
\eqref{buch:orthofkt:sa:eqn:normcauchyschwarzsup}
kann auch die Norm von $A^*$ bestimmt werden:
\[
\|A^*\|
=
\sup_{x,y\in H, \|x\|=\|y\|=1} \langle A^*x,y\rangle
=
\sup_{x,y\in H, \|x\|=\|y\|=1} \langle x,Ay\rangle
=
\|A\|.
\]
Der adjungierte Operator ist als ein linearer Operator mit
der gleichen Norm.

\begin{definition}
Der Operator $A^*$ heisst der zu $A$ {\em adjungierte} Operator.
\index{adjungierter Operator}%
\index{Operator!adjungiert}%
Ein Operator $A$ heisst {\em selbstadjungiert}, wenn $A^*=A$ oder
\index{selbstadjungiert}%
$\langle Ax,y\rangle = \langle x,Ay\rangle$ für alle $x,y\in H$.
\end{definition}


%
% Eigenwerte von selbstadjungierten Operatoren
%
\subsection{Eigenwerte von selbstadjungierten Operatoren}
Die Eigenschaften der Eigenvektoren symmetrischer Matrizen lassen
sich jetzt ohne Änderung der Beweise übertragen.

\begin{satz}
Die Eigenwerte eines selbstadjungierten Operators sind reell.
\end{satz}

\begin{proof}[Beweis]
Sei $A$ ein selbstadjungierter Operator auf dem Hilbert-Raum $H$ und
$x$ ein Eigenvektor zum Eigenwert $\lambda$, also $Ax=\lambda x$
und damit
\begin{equation}
\langle x,Ax\rangle
=
\langle x,\lambda x\rangle
=
\lambda \langle x,x\rangle.
\label{buch:orthofkt:sa:reell1}
\end{equation}
Andererseits gilt
\begin{equation}
\langle x,Ax\rangle
=
\langle A^*x,x\rangle
=
\langle Ax,x\rangle
=
\langle \lambda x,x\rangle
=
\overline{\lambda}
\langle x,x\rangle.
\label{buch:orthofkt:sa:reell2}
\end{equation}
Zusammen mit~\eqref{buch:orthofkt:sa:reell1} folgt jetzt
\[
\lambda \langle x,x\rangle = \overline{\lambda} \langle x,x\rangle.
\]
Da $x\ne 0$ ist $0\ne \|x\|^2 =\langle x,x\rangle$, darf man durch
$\|x\|^2$ teilen und erhält 
\[
\lambda = \overline{\lambda},
\]
folglich $\lambda$ ist reell.
\end{proof}

\begin{satz}
Eigenvektoren eines selbstadjungierten Operators zu verschiedenen Eigenwerten
sind orthogonal.
\end{satz}

\begin{proof}[Beweis]
Seien $x_1$ und $x_2$ Eigenvektoren des selbstadjungierten Operators $A$
zu den Eigenwerten $\lambda_1\ne \lambda_2$.
Da $A$ selbstadjungiert ist, gilt
$\langle Ax_1,x_2\rangle=\langle x_1,Ax_2\rangle$.
Wir berechnen diese beiden Skalarprodukte mit Hilfe der Eigenvektoreigenschaft:
\[
\renewcommand{\arraycolsep}{3pt}
\begin{array}{cclcl}
\langle Ax_1,x_2\rangle
&=&
\langle \lambda_1x_1,x_2\rangle
&=&
\lambda_1 \langle x_1,x_2\rangle
\\
          \|&&&&
\\
\langle x_1,Ax_2\rangle
&=&
\langle x_1,\lambda_2x_2\rangle
&=&
\lambda_2 \langle x_1,x_2\rangle.
\end{array}
\]
Die Differenz dieser beiden Terme ist
\[
0
=
(\lambda_1-\lambda_2)\langle x_1,x_2\rangle.
\]
Da $\lambda_1\ne \lambda_2$ ist, ist der Klammerausdruck nicht $0$, daher muss
das Skalarprodukt $\langle x_1,x_2\rangle=0$ sein, die beiden Eigenvektoren
sind orthogonal.
\end{proof}

\begin{beispiel}
Der Vektorraum $H=\mathbb{R}^n$ mit dem Standardskalarprodukt ist ein
endlichdimensionaler Hilbert-Raum.
Ein Operator $A\colon H\to H$ wird unter Verwendung einer Basis
$\{b_1,\dots,b_n\}$ durch die Matrix mit den Matrix-Elementen
\[
a_{i\!j}
=
\langle b_i,Ab_j\rangle
=
\langle Ab_i,b_j\rangle
=
\langle b_j,Ab_i\rangle
=
a_{ji}
\]
beschrieben, die Matrix ist symmetrisch.
\end{beispiel}

\begin{beispiel}
Für den endlichdimensionalen komplexen Vektorraum $H=\mathbb{C}^n$ mit dem
Standardskalarprodukt
\[
\langle x,y\rangle
=
\sum_{i=1}^n  \overline{x}_i y_i
\]
ist ein Hilbert-Raum.
Ein selbstadjungierter Operator hat in einer Basis $b_1,\dots,b_n$
die Matrixelemente
\[
a_{i\!j}
=
\langle b_i,Ab_j\rangle
=
\langle Ab_i,b_j\rangle
=
\overline{
\langle b_j,Ab_i\rangle
}
=
\overline{a_{ji}}.
\]
Die Matrix von $A$ ist also nicht mehr symmetrisch, sondern konjugiert
symmetrisch oder hermitesch (siehe
Definition~\ref{buch:skalarprodukt:definition:def:hermitesch}).
\end{beispiel}

\begin{beispiel}
\label{buch:orthfkt:sa:beispiel:D2}
Wir betrachten den Prähilbertraum
\[
H
=
\{f\in C^{\infty}(\mathbb{R})
\mid
\text{$f$ ist $2\pi$-periodisch}
\}
\]
der $2\pi$-periodischen, beliebig oft stetig differenzierbaren
stetig differenzierbaren Funktionen auf $\mathbb{R}$ mit dem
Skalarprodukt
\[
\langle f,g\rangle
=
\int_{-\pi}^\pi
f(x)g(x)\,dx
\]
und den Operator
\[
D^2
\colon
H_0 \to H_0
:
f\mapsto f''.
\]
Wir berechnen die Skalarprodukte
\begin{align}
\langle D^2f,g\rangle
&=
\int_{-\pi}^\pi f''(x)g(y)\,dx
=
\biggl[f'(x)g(x)\biggr]_{-\pi}^\pi
-
\int_{-\pi}^\pi f'(x)g'(x)\,dx
\label{buch:orthfunkt:sa:eqn:D2fg}
\\
\text{und}\qquad
\langle f,D^2g\rangle
&=
\int_{-\pi}^\pi f(x)g''(y)\,dx
=
\biggl[f(x)g'(x)\biggr]_{-\pi}^\pi
-
\int_{-\pi}^\pi f'(x)g'(x)\,dx.
\label{buch:orthfunkt:sa:eqn:fD2g}
\end{align}
Da $f(x)$ und $g(x)$ $2\pi$-periodisch sind, ist
$f'(\pi)g(\pi)=f'(-\pi)g(-\pi)$, der erste Term
in~\eqref{buch:orthfunkt:sa:eqn:D2fg} verschwindet daher.
Dasselbe passiert auch in \eqref{buch:orthfunkt:sa:eqn:fD2g}.
Es bleibt
\[
\langle D^2f,g\rangle
=
-
\int_{-\pi}^\pi f'(x)g'(x)\,dx.
=
\langle f,D^2g\rangle,
\]
der Operator $D^2$ ist daher selbstadjungiert.
\end{beispiel}

Das letzte Beispiel zeigt zusammen mit den Erkenntnissen aus dem
Abschnitt~\ref{buch:trigo:subsection:symmetrie}, dass die
Eigenfunktionen eines selbstadjungierten Operators gute Kandidaten
für eine harmonische Analysis sind, die sich besonderes gut
zur Lösung von Problemen eignen, die mit dem Operator $D^2$ der
zweiten Ableitung zusammenhängen.





%
% 4-sturmliouville.tex -- Sturm-Liouville-Probleme
%
% (c) 2022 Prof Dr Andreas Müller, OST Ostschweizer Fachhochschule
%
\section{Sturm-Liouville-Problem
\label{buch:orthofkt:section:sturmliouville}}
\kopfrechts{Sturm-Liouville-Problem}
Das Beispiel~\ref{buch:orthfkt:sa:beispiel:D2} zeigt, dass
Eigenfunktionen eines Differentialoperators zu interessanten
Familien orthogonaler Funktionken führen können.
Die daraus abgeleitete harmonische Analysis kann zur Lösung
gewisser Differentialgleichungen genutzt werden.
Ziel dieses Abschnitts ist zu zeigen, dass das Beispiel auf
eine wesentlich grössere Klasse von Differentialgleichungen
erweitert werden kann.

%
% Der Sturm-Liouville-Differentialoperator
%
\subsection{Der Sturm-Liouville-Differentialgoperator}
Sie $a<b$ und seien zwei Funktionen
$p(x)\in C^2([a,b])$, $q(x)\in C([a,b])$ gegeben.
Wir definieren den Sturm-Liouville-Operator
\[
L = \frac{d}{dx} p(x) \frac{d}{dx} + q(x),
\]
der die Funktion $y(x)$ auf
\begin{align*}
(Ly)(x)
&=
\frac{d}{dx}p(x)\frac{d}{dx}y(x) + q(x)y(x)
\\
&=
\frac{d}{dx}p(x)y'(x) + q(x)y(x)
\\
&=
p'(x)y'(x)+p(x)y''(x)+q(x)y(x)
\end{align*}
abbildet.
Die Funktion $q(x)$ wirkt also durch punktweise Multiplikation.

Ableitungen kommen nur im ersten Teil vor, den wir mit
\[
L_0
=
\frac{d}{dx}p(x)\frac{d}{dx}
\]
bezeichnen.

%
% Der Sturm-Liouville-Operator ist selbstadjungiert
%
\subsection{Der Sturm-Liouville-Operator ist selbstadjungiert}
Wir betrachten die Frage der Selbstadjungiertheit des
Operators $L$ bezüglich des Standardskalarproduktes
\[
\langle f,g\rangle
=
\int_a^b f(x)g(x)\,dx.
\]
Der Multiplikationsoperator mit der Funktion $q(x)$ ist ganz
offensichtlich selbstadjungiert, es gilt nämlich
\[
\langle qf,g\rangle
=
\int_a^b (q(x)f(x))g(x)\,dx
=
\int_a^b f(x)(q(x)g(x))\,dx
=
\langle f,qg\rangle
\]
für alle Funktionen $f$ und $g$, für die die Integrale definiert sind.

Der Operator $L_0$ hat ähnliche Eigenschaften wie die zweite
Ableitung im Beispiel~\ref{buch:orthfkt:sa:beispiel:D2}.
Dazu berechnen wir die Skalarprodukte
\begin{align}
\langle Lf,g\rangle
&=
\int_a^b (Lf)(x)g(x)\,dx
=
\int_a^b \frac{d}{dx}\biggl(q(x)\frac{d}{dx}f(x)\biggr) g(x)\,dx
\\
&=
\biggl[ q(x)f'(x) g(x) \biggr]_a^b
-
\int_a^b q(x)f'(x)g'(x)\,dx
\label{buch:orthfkt:sturmliouville:Lfg}
\\
\langle f,Lg\rangle
&=
\int_a^b f(x) (Lg)(x)\,dx
=
\int_a^b f(x)\frac{d}{dx}\biggl(q(x)\frac{d}{dx}g(x)\biggr)\,dx
\\
&=
\biggl[ f(x) q(x)g'(x)\biggr]_a^b
-
\int_a^b f'(x)q(x)g'(x)\,dx.
\label{buch:orthfkt:sturmliouville:fLg}
\end{align}
Die beiden Integrale stimmen überein, aber der erste Term ist
offensichtlich verschieden.
$L$ ist also im allgemeinen nicht selbstadjungiert.

%
% Randbedingungen
%
\subsection{Randbedingungen}
Aus den Gleichungen
\eqref{buch:orthfkt:sturmliouville:Lfg}
und
\eqref{buch:orthfkt:sturmliouville:fLg}
kann man Bedingungen an die Funktionen ableiten, die überhaupt
zugelassen werden sollen, die den Operator $L_0$ zu einem
selbstadjungierten Operator machen.
Die Funktionen müssen die Bedingungen
\begin{align}
\biggl[ p(x)f'(x) g(x) \biggr]_a^b
&=
\biggl[ p(x)f(x) g'(x) \biggr]_a^b
\notag
\\
p(b)f'(b)g(b)
-
p(a)f'(a)g(a) 
&=
p(b)f(a)g'(b)
-
p(a)f(a)g'(a)
\notag
\intertext{oder}
p(b)
\bigl(
f'(b)g(b)-f(b)g'(b)
\bigr)
&=
p(a)
\bigl(
f'(a)g(b)
-
f(a)g'(a)
\bigr)
\notag
\intertext{erfüllen.
Mit der Determinante lässt sich das noch etwas übersichtlicher
schreiben als}
\left|\begin{matrix}
p(b)f'(b)&p(b)g'(b) \\
f(b) &g(b)
\end{matrix}\right|
=
\left|\begin{matrix}
p(a)f'(a)&p(a)g'(a) \\
f(a) &g(a)
\end{matrix}\right|.
\label{buch:orthfkt:sturmliouville:det}
\end{align}

Randbedingungen können immer nur für einen Randpunkt gefordert
werden, wir müssen also den Wert der Determinante festlegen.
Angenommen, wir fordern, dass die Determinante einen Wert
$d\ne 0$ annimmt. Setzen wir die Funktione $\lambda f$ und $g$ ein,
ensteht die Determinante
\[
\left|\begin{matrix}
\lambda p(a)f'(a)&p(a)g'(a) \\
\lambda f(a) &g(a)
\end{matrix}\right|
=
\lambda
\left|\begin{matrix}
p(a)f'(a)&p(a)g'(a) \\
f(a) &g(a)
\end{matrix}\right|
=
\lambda t \ne t
\]
für $\lambda \ne 1$.
Eine solche Forderung führt also auf eine Menge von zulässigen Funktionen,
die nicht mehr ein Vektorraum ist.
Es bleibt daher nur die Möglichkeit, dass die Determinante $=0$ sein muss.

Die Determinante 
\eqref{buch:orthfkt:sturmliouville:det}
ist genau dann Null, wenn die Spalten der Matrix linear abhängig sind.
Insbesondere spannen Sie den gleiche Gerade in $\mathbb{R}^2$ auf.
Es gibt daher Vektoren
\[
\begin{pmatrix}
h_a\\
k_a
\end{pmatrix}
\qquad\text{und}\qquad
\begin{pmatrix}
h_b\\
k_b
\end{pmatrix}
\]
die auf den Vektoren
\[
\begin{pmatrix}
p(a)f'(a)\\
f(a)
\end{pmatrix}
\qquad\text{und}\qquad
\begin{pmatrix}
p(a)g'(a)\\
g(a)
\end{pmatrix}
\]
senkrecht stehen.
Dies ist gleichbedeutend mit der Randbedingung
\begin{align*}
k_af(a) + h_ap(a)f'(a)&=0
\\
k_bf(b) + h_bp(b)f'(b)&=0.
\end{align*}
Wir fassen die Resultate im folgenden Satz zusammen.

\begin{satz}
Seien $h_a, h_b, k_a, k_b\in \mathbb{R}$ gegeben.
Dann ist der Operator $L_0$ ist selbstadjungiert im Prähilbertraum
\begin{equation}
H(h_a,h_b,k_a,k_b)
=
\left\{
f \in C^1([a,b])
\;
\left|
\;
\begin{aligned}
&\int_a^b |f(x)|^2\,dx < \infty,
\int_a^b |p(x)| |f'(x)|^2\,dx < \infty,
\\
&
\begin{aligned}
k_af(a) + h_ap(a)f'(a) &= 0 \\
k_af(b) + h_bp(b)f'(b) &= 0 
\end{aligned}
\end{aligned}
\right.
\right\}
\label{buch:orthfkt:sturmliouvill:ph}
\end{equation}
mit dem Skalarprodukte
\[
\langle f,g\rangle = \int_a^b f(x)g(x)\,dx.
\]
\end{satz}

%
% Das verallgemeinerte Eigenwertproblem
%
\subsection{Das Eigenwertproblem}
Zu Zahlen $h_a$, $h_b$, $k_a$, $k_b$ ist der Operator $L_0$ im 
Prählibertraum~\eqref{buch:orthfkt:sturmliouvill:ph}
selbstadjungiert.
Wir erwarten, dass die Eigenvektoren dieses Differentialoperators eine
Hilbertbasis für die Analyse und Synthese von Funktionen in diesem
Prählibertraum liefern.
In diesem Abschnitt soll gezeigt werden, wie sich damit 
Hilbertbasen für erweiterte Differentialgleichungen 

%
% Das Eigenwertproblem für den Operator $L$
%
\subsubsection{Das Eigenwertproblem für den Operator $L$}
Wir betrachten jetzt den Operator
\[
L
=
\frac{d}{dx}p(x)\frac{d}{dx} + q(x)
=
L_0 + q(x).
\]
Es wurde bereits gezeigt, dass $q(x)$ immer selbstadjungiert ist,
und dass $L_0$ in einem Prähilbertraum der Form
Prählibertraum~\eqref{buch:orthfkt:sturmliouvill:ph}
selbstadjungiert wird.
Eine Funktion $y\in H(h_a,h_b,k_a,k_b)$ ist eine Eigenfunktion des
Operators $L$ zum Eigenwert $\lambda$, wenn sie die Differentialgleichung
zweiter Ordnung
\begin{align*}
\frac{d}{dx} p(x) y'(x) + q(x)y(x) = \lambda y(x)
\intertext{oder}
 p(x) y''(x) + p'(x)y'(x) + q(x)y(x) = \lambda y(x)
\end{align*}
erfüllt.
Für $H(h_a,h_b,k_a,k_b)$ kann eine Hilbertbasis aus Eigenfunktionen
gefunden werden.

%
% Das verallgemeinerte Eigenwertproblem für symmetrische Matrizen
%
\subsubsection{Das verallgemeinerte Eigenwertproblem für symmetrische Matrizen}

%
% Das verallgemeinerte Sturm-Liouville-Eigenwertproblem
%
\subsubsection{Das verallgemeinerte Sturm-Liouville-Eigenwertproblem}
Das verallgemeinerte Sturm-Liouville-Eigenwertproblem ist die Aufgabe,
eine Lösung $y(x)$ der Differentialgleichung
\[
\frac{d}{dx}p(x)\frac{d}{dx} y(x)
+ q(x)y(x)
=
\lambda w(x) y(x)
\]
zu finden.
Das Problem unterscheidet sich von dem vorher untersuchten Problem durch
den zusätzlichen Faktor $w(x)>0$, der auch Gewichtsfunktion heisst.
Das Eigenwertproblem auf die bereits studierte Situation zurückgeführt
werden, indem das Skalarprodukt mit der Gewichtsfunktion $w(x)$ verwendet
wird.

\begin{definition}
Gegeben ist $p(x)\in C^1([a,b])$, $q(x),w(x)\in C([a,b])$ und $w(x)>0$
in $(a,b)$.
Der {\em allgemeine Sturm-Liouville-Operator} ist
\[
L
=
\frac{1}{w(x)}
\biggl(
\frac{d}{dx}p(x)\frac{d}{dx}
+
q(x)
\biggr).
\]
Das {\em verallgemeinerte Sturm-Liouville-Eigenwertproblem} ist die Aufgabe,
eine Lösung $y(x)$ der Gleichung
\(
Ly(x) = \lambda y(x)
\)
zu finden.
\end{definition}

\begin{satz}
Der Operator $L$ ist selbstadjungiert im Prähilbertraum
\[
H(h_a,h_b,k_a,k_b,w)
=
\left\{
f \in C^1([a,b])
\;
\left|
\;
\begin{aligned}
&\int_a^b |f(x)|^2w(x)\,dx < \infty,
\int_a^b |p(x)| |f'(x)|^2w(x)\,dx < \infty,
\\
&
\begin{aligned}
k_af(a) + h_ap(a)f'(a) &= 0 \\
k_af(b) + h_bp(b)f'(b) &= 0 
\end{aligned}
\end{aligned}
\right.
\right\}
\]
mit dem Skalarprodukt $\langle \;\,,\;\rangle_w$ mit der Gewichtsfunktion
$w(x)$.
\end{satz}

\begin{proof}[Beweis]
Wir müssen nachrechnen, dass der Operator $L$ selbstadfjungiert ist.
Dazu seien $f,g\in H(h_a,h_b,k_a,k_b,w)$, wir berechnen die
Skalarprodukte
\begin{align*}
\langle Lf,g\rangle_w
&=
\int_a^b (Lf)(x)g(x)w(x)\,dx
=
\int_a^b
\biggl(
\frac{d}{dx}p(x)\frac{d}{dx}f(x) + 
q(x) f(x)
\biggr) g(x)\,dx
=
\langle L_0f,g\rangle + \langle qf,g\rangle
\\
\langle f,Lg\rangle_w
&=
\int_a^b f(x)(Lg)(x)w(x)\,dx
=
\int_a^b
f(x)
\biggl(
\frac{d}{dx}p(x)\frac{d}{dx}g(x) + 
q(x) g(x)
\biggr) \,dx
=
\langle f,L_0g\rangle + \langle f,qg\rangle
\end{align*}
Da der Operator $L_0$ und der Multiplikationsoperator mit $q(x)$ bezügich
des Standardskalarproduktes $\langle\;\,,\;\rangle$ selbstadjungiert sind,
sind die beiden Ausdrücke gleich.
Damit ist gezeigt, dass $L$ bezüglich $\langle\;\,,\;\rangle_w$
selbstadjungiert ist.
\end{proof}

%
% 5-pde.tex -- PDE
%
% (c) 2022 Prof Dr Andreas Müller, OST Ostschweizer Fachhochschule
%
\section{Partielle Differentialoperatoren
\label{buch:orthofkt:section:pde}}
\kopfrechts{Partielle Differentialoperatoren}
Selbstadjungierte partielle Differentialoperatoren auf einem Gebiet 
$\Omega$ führen auf natürlich Art und Weise zu einer verallgemeinerten
Theorie der harmonischen Analysis, die auch zur Lösung der
partiellen Differentialgleichung verwendet werden kann.
Die Vorgehensweise wird zunächst in
Abschnitt~\ref{buch:orthfkt:pde:subsection:pdeharm}
allgemein skizziert und anschliessend für ein paar interessante
Gebiete und Differentialoperatoren durchgeführt.

%
% Partielle Differentialgleichungen udn harmonische Analysis
%
\subsection{Partielle Differentialgleichung und harmonische Analysis
\label{buch:orthfkt:pde:subsection:pdeharm}}
In diesem Abschnitt soll illustriert werden, wie sich aus Aufgabenstellungen
aus der Theorie der partiellen Differentialgleichungen 

%
% Partielle Differentialoperatoren
%
\subsubsection{Lineare partielle Differentialoperatoren}
Ein linearer partieller Differentialoperator ist eine lineare
Funktion der partiellen Ableitungen einer Funktion, die auf einem
Gebiet definiert ist.

\begin{definition}
Ein Gebiet $\Omega$ ist eine offene Teilmenge von $\mathbb{R}^n$.
Ein partieller Differentialoperator der Ordnung $r$ ist eine Linearkombination
\[
L
=
\sum_{k_1+\dots+k_n\le r}
a_{k_1\dots k_n}(x)
\frac{\partial^{k_1+\dots+k_n}}{\partial x_1^{k_1}\dots\partial x_n^{k_n}}.
\]
\end{definition}

\begin{beispiel}
Der Laplace-Operator auf einem beliebigen Gebiet $\Omega\subset\mathbb{R}^n$
ist definiert als
\[
\Delta
=
\frac{\partial^2}{\partial x_1^2}
+
\dots
+
\frac{\partial^2}{\partial x_n^2}
\]
ist ein partieller Differentialoperator zweiter Ordnung.
\end{beispiel}

%
% Skalarprodukt
%
\subsubsection{Skalarprodukt}
Da $\Omega$ eine offene Teilmenge von $\mathbb{R}^n$ ist, lässt sich
sofort ein Skalarprodukt von Funktionen auf $\Omega$ definieren.

\begin{definition}
Ist $\Omega\subset \mathbb{R}^n$ ein Gebiet in $\mathbb{R}^n$, dann ist
\begin{equation}
\langle u,v\rangle_\Omega
=
\int_{\Omega}  \overline{u(x)} v(x)\,dx
\label{buch:orthofkt:pde:eqn:skalarprodukt}
\end{equation}
ein Skalarprodukt für Funktionen auf $\Omega$.
\end{definition}

Mit einer Gewichtsfunktion $w\colon \Omega \to \mathbb{R}^+$ lässt sich
das Skalarprodukt noch etwas verallgemeinern, es wird zu
\[
\langle u,v\rangle_{\Omega,w}
=
\int_{\Omega} \overline{u(x)}v(x)\,w(x)\,dx.
\]
Diese Erweiterung ist auch deshalb nötig, weil bei einem Wechsel
der Koordinaten nach $x_i=x_i(x_1',\dots,x_n')$, das Integral
\eqref{buch:orthofkt:pde:eqn:skalarprodukt}
den zusätzlichen Faktor der Funktionaldeterminante 
\[
\langle u,v\rangle_{\Omega}
=
\int_{\Omega} \overline{u(x)} v(x)\,dx 
=
\int_{\Omega'} \overline{u(x')} v(x')
\det\frac{\partial(x_1,\dots,x_n)}{\partial(x_1',\dots,x_n')}
\,dx'
\]
erhält, wobei $\Omega'$ das Gebiet in $\mathbb{R}^n$ ist, welches
durch die Koordinatenfunktionen $x_i(x_1',\dots,x_n')$ bijektiv auf
$\Omega$ abgebildet wird.


%
% Randbedingungen
%
\subsubsection{Randbedingungen}
Wie beim Sturm-Liouville-Problem entsteht ein interessantes 
erst dadurch, dass zusätzlich Randbedingungen formuliert werden.
Der Operator $L=d^2/dx^2$ auf dem Intervall $(a,b)\subset\mathbb{R}$
ist zum Beispiel nur selbstadjungiert, wenn man ihn auf Funktionen
einschränkt, die gewisse Randbedingungen erfüllen.
$L$ ist der Sturm-Liouville-Operator von Abschnitt
\ref{buch:orthofkt:sturmliouville:subsection:operator}
mit $p(x)=1$ und $q(x)=0$.
In Abschnitt~\ref{buch:orthofkt:sturmliouville:subsection:operator}
wurde gezeigt, dass der Operator $L$ nur selbstadjungiert bezüglich
des $L^2$-Skalarproduktes wird, wenn Randbedinungen der Form
\begin{equation}
h_a u(a) + k_a u'(a) =  0
\qquad\text{und}\qquad
h_b u(b) + k_a u'(b) =  0
\end{equation}
gefordert werden.
Die Wahl $h_a=h_b=1$ und $k_a=k_b=0$ hat zum Beispiel zur Folge,
dass die Funktionen die Randwerte $u(a)=u(b)=0$ haben müssen.
Die Wahl $h_a=h_b=0$ und $k_a=k_b=1$ hat dagegen zur Folge,
dass die Ableitungen der Funktionen in den Endpunkten $=0$ sein
müssen.
Für partielle Differentialoperatoren und die 

\begin{definition}[Dirichlet-Randbedingungen]
Sei $\Omega\subset\mathbb{R}^n$ ein Gebiet mit Rand $\partial\Omega$
und $f\colon \partial\Omega\mathbb R$ eine Funktion auf dem Rand.
Man sagt, eine Funktion $u\colon\overline{\Omega}\to\mathbb{R}$ erfüllt 
{\em Dirichlet-Randbedingungen}, wenn
\index{Randbedingung!Dirichlet-}%
\index{Dirichlet-Randbedingung}%
\(
u(x) = f(x)
\)
für $x\in \partial\Omega$ ist.
\end{definition}

Dirichlet-Randbedinungen entsprechen also der Vorgabe von Werten
oder dem Fall $h_a=h_b=0$ und $k_a=k_b=0$.

\begin{definition}
Sei $\Omega\in\mathbb{R}^n$ ein Gebiet mit glattem Rand und
$n(x)$ der Einheitsnormalenvektor im Punkt $x\in\partial\Omega$
auf dem Rand.
Die {\em Normalableitung} einer Funktion
\index{Normalableitung}%
$u\colon\overline{\Omega}\to \mathbb{R}$ auf dem Rand ist die
Richtungsableitung
\[
\frac{\partial u}{\partial n}
=
\frac{d}{dt} u(x+tn(x))\bigg|_{t=0}.
\]
Man sagt, die Funktion $u$ erfüllt eine Neumann-Randbedingung, wenn
\index{Neumann-Randbedingung}%
\index{Randbedingung!Neumann-}%
\[
\frac{\partial u}{\partial n}(x) = g(x)
\]
erfüllt ist.
\end{definition}

Neumann-Randbedingungen entsprechen also der Vorgabe der Ableitungen
oder dem Fall $h_a=h_b=0$ und $k_a=k_b=0$.
Die Dirichlet- und Neumann-Randbedingungen können auch in sogenannte
gemischte Randbedingungen zusammengefasst werden, indem 
\index{Randbedingung!gemischt}%
\index{gemischte Randbedingung}%
\begin{equation}
h(x) u(x) + k(x) \frac{\partial u}{\partial n}(x) = g(x)
\label{buch:orthofunkt:pde:eqn:gemischterandbedingung}
\end{equation}
für  $x \in\partial \Omega$ für auf dem Rand definierte Funktion
$h(x)$ und $k(x)$ gefordert wird.

%
% Homogene und inhomogene Differentialgleichungen und Randbedingungen
%
\subsubsection{Homogene und inhomogene Differentialgleichungen und
Randbedingungen}
Eine partielle Differentialgleichung ist eine Differentialgleichung
der Form
\begin{equation}
L u = v,
\label{buch:orthofunkt:pde:eqn:dgl}
\end{equation}
wobei $L$ ein Differentialoperator auf dem Gebiet $\Omega$ ist und
$h$ eine Funktion auf $\Omega$.
Eine Lösung der partiellen Differentialgleichung ist eine Funktion
$u\colon\overline{\Omega}\to \mathbb{R}$ derart, dass die Gleichung
\eqref{buch:orthofunkt:pde:eqn:dgl} in $\Omega$ gilt.
Ohne zusätzliche Randbedingungen wird die Lösung der Gleichung
im Allgemeinen nicht eindeutig sein.
Sind $u_1$ und $u_2$ Lösungen der Differentialgleichung
\eqref{buch:orthofunkt:pde:eqn:dgl}, dann ist
\[
L(u_1-u_2)
=
Lu_1-Lu_2
=
v-v
=0,
\]
die Differenz $u_1-u_2$ ist also eine Lösung der zugehörigen
{\em homogenen} Differentialgleichung.
\index{homogen}%
Eine Lösung $u_p$ von 
\eqref{buch:orthofunkt:pde:eqn:dgl} heisst auch eine
{\em partikuläre Lösung}.

Die beiden Lösungen können sich durch die Werte der Funktionen
oder der Normalableitungen auf dem Rand unterscheiden.
Es ist also im allgemeinen nötig, eine Randbedingung vorzugeben.
Eine partikuläre Lösung $u_p$ erfüllt die Randbedingung möglicherweise
nicht.
Eine Lösung $u$, die auch die Randbedingung erfüllt, unterschiedet sich
von $u_p$ um 
$\tilde{u} = u-u_p$ und erfüllt die Differentialgleichung
\[
L\tilde{u}
=
Lu - Lu_p
=
f - f
=
0
\]
und die Randbedingung
\begin{align}
h(x)\tilde{u}(x) + k(x)\frac{\partial \tilde{u}}{\partial x} (x)
&=
h(x)u(x) + k(x)\frac{\partial u}{\partial n}(x)
-
h(x)u_p(x) - k(x)\frac{\partial u_p}{\partial n}(x)
\notag
\\
&=
g(x)
-
h(x)u_p(x) - k(x)\frac{\partial u_p}{\partial n}(x).
\label{buch:orthofunkt:pde:eqn:utilderand}
\end{align}
Um die Lösung $\tilde{u}$ zu finden, reicht es daher, eine Lösung der
homogenen Differentialgleichung $L\tilde{u}=0$ zu finden, die die
Randbedingungen \eqref{buch:orthofunkt:pde:eqn:utilderand}
erfüllt.
Das Problem ist damit reduziert auf eine homogene Differentialgleichung
mit möglicherweise inhomogenen Randbedingungen.

Durch Vorgabe einer Randbedingung werden die möglichen Lösungen
eingeschränkt, es ist aber immer noch möglich, dass die Lösung
nicht eindeutig ist.
Seien also $u_1$ und $u_2$ Lösungen der Differentialgleichung mit der 
Randbedingung~\eqref{buch:orthofunkt:pde:eqn:gemischterandbedingung},
dann ist die Differenz $u=u_1-u_2$ eine Lösung der homogenen
Differentialgleichung $Lu=0$ und die Randbedingungen sind
\[
h(x) u(x) + k(x) \frac{\partial u}{\partial n}(x)
=
h(x) u_1(x) + k(x) \frac{\partial u_1}{\partial n}(x)
-
h(x) u_2(x) + k(x) \frac{\partial u_2}{\partial n}(x)
=
g(x)-g(x)=0.
\]
Die Differenz erfüllt also die homogene Differentialgleichung
{\em und} zusätzlich homogene Randbedingungen.
Die Frage nach der Eindeutigkeit der Lösung der partiellen
Differentialgleichung mit den gegebenen Randbedingungen wird also
beantwortet von der Lösungsmenge der homogenen Differentialgleichung
mit homogenen Randbedingungen.

%
% Harmonische Analysis
%
\subsubsection{Harmonische Analysis}
Die im Fall des Eigenwertproblems für den Sturm-Liouville-Operator
beschriebene Situation tritt sehr häufig auch bei partiellen
Differentialgleichungen auf.
Wir erwarten daher, dass zu einem partiellen Differentialoperator $L$
auf einem Gebiet $\Omega$ jeweils eine orthogonale Funktionenfamilie
von Eigenfunktionen  $u_n(x)$ des Operators mit Eigenwerten $\lambda_n$
gehört, die zur Synthese beliebiger Funktionen auf $\Omega$ verwendet
werden kann.

Für sogenannte elliptische partielle Differentialoperatoren zweiter
Ordnung lassen sich dann die Lösungsfunktionen der Gleichungen
\begin{align*}
Lu &= \varrho                            &&\text{Poisson-Gleichung}      \\
\intertext{oder für den Fall, dass $u(t,x)$ zusätzlich von der Zeit abhängt}
\frac{\partial u}{\partial t}     &= Lu  &&\text{Wärmeleitungsgleichung} \\
\frac{\partial^2 u}{\partial t^2} &= LU  &&\text{Wellengleichung}
\end{align*}
als Reihe 
geeigneten Randbedingungen lassen sich die Lösungsfunktionen dann in 
der Form
\begin{equation}
u(x)
=
\sum_{n=0}^\infty a_n u_n(x)
\label{buch:orthofkt:eqn:unreihe}
\end{equation}
entwickeln, wobei die Koeffizienten $a_n$ 
im Falle der Wärmeleitungslgeichung und der Wellengleichung
Funktionen der Zeit sind.
Eingesetzt in die Differentialgleichung ergeben sich dann für die
Koeffizienten $a_n$ die Gleichungenn
\begin{align}
\lambda_n a_n  &=  b_n      &&\text{Poisson-Gleichung} 
\notag
\\
\dot{a}_n &= \lambda_n a_n  &&\text{Wärmeleitungsgleichung}
\label{buch:orthfkt:eqn:waermeleitung:ode}
\\
\ddot{a}_n &= \lambda_n a_n &&\text{Wellengleichung,}
\label{buch:orthfkt:eqn:wellen:ode}
\end{align}
wobei die $b_n$ die Koeffizienten der Entwicklung von $\varrho$ nach
den Funktionen $u_n(x)$ ist.
Im Falle der Wärmeleitungsgleichung wird das Problem also reduziert
auf die Lösung einer gewöhnlichen Differentialgleichung.

Zusätzlich zu den Randbedingungen auf dem Rand des Gebietes $\Omega$,
auf dem der Differentialoperator $L$ definiert ist, braucht die
Wärmeleitungsgleichung auch noch eine Randbedingung der Form
\[
u(0,x) = f(x)
\]
hinzu, die durch Anwendung der Entwicklung~\eqref{buch:orthofkt:eqn:unreihe}
zur Anfangsbedinung 
\[
a_n(0) = f_n
\]
für \eqref{buch:orthfkt:eqn:waermeleitung:ode} wird, wobei $f_n$ die
Koeffizienten der Entwicklung der Funktion $f(x)$ nach den Eigenfunktionen
$u_n$ sind.
Für die Wellengleichung \eqref{buch:orthfkt:eqn:wellen:ode} wird 
auch noch eine Neumann-Randbedingung der Form
\[
\frac{\partial u}{\partial t}(0,x) = g(x)
\]
nötig, die zu einer Anfangsbedingung für die Ableitungen
\[
\dot{a}_n(0) = g_n
\]
für die gewöhnliche Differentialgleichung \eqref{buch:orthfkt:eqn:wellen:ode}
wird.

Die Bestimmung der Entwicklungskoeffizienten erfolgt jeweils mit Hilfe
der Skalarprodukte
\[
b_n = \langle u_n, \varrho\rangle,
\qquad
f_n = \langle u_n, f\rangle,
\qquad
g_n = \langle u_n, g\rangle
\]
gewonnen ist.

%
% Fourier-Theorie
%
\subsection{Fourier-Theorie
\label{buch:orthofkt:subsection:fourier-theorie}}
Die Funktionen der Fourier-Theorie entstehen als Eigenfunktionen des
Differentialoperators
\begin{equation}
D^2 = \frac{d^2}{dx^2}
\end{equation}
der zweiten Ableitung auf dem Intervall $[0,l]$.
Das Skalarprodukt ist das übliche $L^2$-Skalarprodukt.

\subsubsection{$D^2$ ist selbstadjungiert}
Seien $f$ und $g$ zwei zweimal stetig differenzierbare Funktionen,
wir berechnen das Skalarprodukt $\langle D^2f,g\rangle$ und
$\langle f,D^2g\rangle$:
\begin{align*}
\langle D^2f,g\rangle
&=
\int_0^{l}
f''(x) g(x)\,dx
=
\left[ f'(x) g(x) \right]_0^{l}
-
\int_0^{l} f'(x)g'(x)\,dx
\\
\langle f,D^2g\rangle
&=
\int_0^{l} f(x)g''(x)\,dx
=
\left[ f(x) g'(x) \right]_0^{l}
-
\int_0^{l} f'(x)g'(x)\,dx
\end{align*}
Die Integrale auf der rechten Seite stimmen überein.
Der Operator $D^2$ ist also genau dann selbstadjungiert, wenn die anderen
Terme auch übereinstimmen, wenn also
\[
\left[ f'(x) g(x) \right]_0^{l}
=
\left[ f(x) g'(x) \right]_0^{l}.
\]
Ausgeschrieben ist dies
\begin{align}
f'(l)g(l) - f'(0)g(0)
&=
f(l)g'(l) - f(0)g'(0)
\notag
\\
\Rightarrow\qquad
f'(l)g(l) - f(l)g'(l)
&=
f'(0)g(0)-f(0)g'(0)
\notag
\intertext{oder mit Determinanten geschrieben}
\biggl|\begin{matrix}
f'(l) & g'(l) \\
f(l)  & g(l)
\end{matrix}\biggr|
&=
\biggl|\begin{matrix}
f'(0) & g'(0) \\
f(0)  & g(0)
\end{matrix}\biggr|
\label{buch:orthofkt:pde:eqn:detfg}
\end{align}
Bei der Diskussion des Sturm-Liouville-Problems haben wir die beiden
Intervallenden unabhängig voneinander betrachtet und zwei unabhängige
Randbedingungen gefunden, welche erzwingen, dass 
\eqref{buch:orthofkt:pde:eqn:detfg} erfüllt ist.
Es gibt aber noch eine weitere Lösung, nämlich zu verlangen, dass
die Funktionswerte und Ableitung für $x=0$ und $x=l$ übereinstimmen.
Eine Funktion mit $f(0)=f(l)$ und $f'(0)=f'(l)$ kann periodisch
auf ganz $\mathbb{R}$ ausgedehnt werden.
Wegen $f(0)=f(l)$ ist die ausgedehnte Funktion stetig auf ganz $\mathbb{R}$
und wegen $f'(0)=f'(l)$ ist sogar die Ableitung stetig.

%
% Eigenfunktionen
%
\subsubsection{Eigenfunktionen}
Eigenfunktionen des Differentialoperators $D^2$ sind Lösungen der
Differentialgleichung
\[
y'' = \lambda y.
\]
Die übliche Lösungsmethode liefert Lösungen
\[
y(x) = e^{\pm\!\sqrt{\lambda} x} \quad
\text{ für $\lambda > 0$ und }
\qquad
y(x)
=
\begin{cases}
\cos\sqrt{-\lambda} x \\
\sin\sqrt{-\lambda} x
\end{cases}
\quad\text{für $\lambda < 0$.}
\]
Natürlich sind auch Linearkombinationen wieder Eigenfunktionen.

Die Funktionen erfüllen aber die Randbedingungen nicht.
Zum Beispiel ist es unmöglich, die periodischen Randbedingungen
mit den Exponentialfunktionen zu erfüllen, denn dazu müsste
$1=e^{0}=e^{\pm\!\sqrt{\lambda}l}$ sein, was nur für $\lambda=0$
gilt.

%
% Randbedingungen für \lambda < 0
%
\subsubsection{Randbedingungen für $\lambda<0$}
Je nach gewählten Randbedingungen sind die Eigenfunktionen des Operators
$D^2$, die auch die Randbedingungen erfüllen, verschieden.
Die allgemeine Theorie garantiert aber dass die Eigenfunktionen die
Form einer Linearkombination
\begin{equation}
y(x)
=
A\cos \sqrt{-\lambda}x
+
B\sin \sqrt{-\lambda}x
\label{buch:orthofkt:pde:periodisch}
\end{equation}
haben und untereinander orthogonal sind.
Wir untersuchen die Bedingungen an $A$ und $B$, die sich aus verschiedenen
Randbedingungen ergeben.

Homogene Dirichlet-Randbedingungen werden erfüllt, wenn $y(0)=y(l)=0$ ist.
In Folgenden bestimmen wir $A$ und $B$ für $l=\pi$.
Wegen
\[
y(0)
=
A\cos \sqrt{-\lambda}0
+
B\sin \sqrt{-\lambda}0
=
A
=
0
\]
folgt, dass eine Eigenfunktion eine Sinus-Funktion sein muss.
Die Randbedingung bei $x=\pi$ wird nur erfüllt, wenn
\[
y(\pi)
=
B\sin\sqrt{-\lambda}\pi
=
0
\qquad
\pi\sqrt{-\lambda}
=
k\pi,\; k\in\mathbb{Z},
\]
weil die Nullstellen der Sinusfunktion Vielfache von $\pi$ sind.
Es folgt, dass $\lambda = -k^2$ und dass die Funktionen $\sin kx$
die Eigenfunktionen dazu sind.

Homogene Neumann-Randbedingungen werden erfüllt, wenn $y'(0)=y'(l)=0$,
wir untersuchen wieder den Fall $l=\pi$.
Aus
\[
y'(x)
=
-A\sqrt{-\lambda}\sin\sqrt{-\lambda} x 
+
B\sqrt{-\lambda}\cos\sqrt{-\lambda} x 
\]
folgt für $x=0$, dass
\[
y'(0)
=
B\sqrt{-\lambda}
\qquad\Rightarrow\qquad B=0.
\]
Am rechten Rand $x=\pi$ folgt dann
\[
y'(\pi)
=
-A\sqrt{-\lambda}\sin\sqrt{-\lambda}\pi
=
0
\qquad\Rightarrow\qquad
\sqrt{-\lambda}\pi = k\pi,\; k\in\mathbb{Z}
\qquad\Rightarrow\qquad
\lambda=-k^2,\;k\in\mathbb{Z}.
\]
In diesem Fall sind also die Kosinusfunktionen die Eigenfunktionen.

Als Beispiel für eine gemischte Randbedingung betrachten wir den
Fall $y(0)=0$ und $y'(\pi)=0$.
Zunächst folgt aus der Diskussion der Dirichlet-Randbedingung, dass
$y(x)=A\sin\sqrt{-\lambda}x$ sein muss.
Am rechten Rand folgt dann
\[
y'(\pi)
=
A\sqrt{-\lambda}\cos\sqrt{-\lambda}\pi
=
0
\quad\Rightarrow\quad
\sqrt{-\lambda}\pi = (k+{\textstyle\frac12})\pi,\;k\in\mathbb{Z}
\]
und $\lambda = -(k+{\textstyle\frac12})^2$
Die Eigenfunktionen sind wieder Sinusfunktionen, aber mit anderen
Frequenzen.

Die Beispiele zeigen, dass die Eigenfunktionen entweder Sinus-
oder Kosinusfunktionen sind.
Mit diesen Funktionen lassen sich beliebige Funktionen approximieren,
die die Randbedingungen erfüllen.

%
% Periodische Randbedingungen für \lambda < 0
%
\subsubsection{Periodische Randbedingungen für $\lambda < 0$}
Beliebige periodische Funktionen auf dem Intervall $[0,2\pi]$ lassen sich
mit den im vorangegangenen Abschnitt gefundenen Funktionen nicht
approximieren, dazu müssen periodische Randbedingungen verlangt werden.
Die Funktion $y(x)$ von
\eqref{buch:orthofkt:pde:periodisch}
erfüllt periodische Randbedingungen für die Werte und Ableitungen, wenn
\begin{align*}
y(0) &= y(2\pi) 
&&\Rightarrow&
A &= A\cos\sqrt{-\lambda}2\pi + B\sin\sqrt{-\lambda}2\pi
\\
y'(0)&=y'(2\pi)
&&\Rightarrow&
\sqrt{-\lambda}B
&=
-
A\sqrt{-\lambda}\sin\sqrt{-\lambda}2\pi
+
B\sqrt{-\lambda}\cos\sqrt{-\lambda}2\pi.
\end{align*}
Die Koeffizienten $A$ und $B$ sind Lösungen des linearen Gleichungssystems
\[
\renewcommand{\arraycolsep}{2pt}
\begin{array}{rcrcr}
\cos2\pi\sqrt{-\lambda} \cdot A
&+&
\sin2\pi\sqrt{-\lambda} \cdot B
&=& A
\\
-\sin2\pi\sqrt{-\lambda} \cdot A
&+&
\cos2\pi\sqrt{-\lambda} \cdot B
&=& B
\end{array}
\Leftrightarrow
\begin{array}{rcrcl}
(\cos2\pi\sqrt{-\lambda}-1) \cdot A
&+&
\sin2\pi\sqrt{-\lambda} \cdot B
&=&
0
\\
-\sin2\pi\sqrt{-\lambda} \cdot A
&+&
(\cos2\pi\sqrt{-\lambda}-1) \cdot B
&=& 0.
\end{array}
\]
Als homogenes lineares Gleichungssystems kann es nur dann eine nichttriviale
Lösung geben, wenn die Determinante der Koeffizientenmatrix verschwindet,
wenn also
\begin{align*}
0
&=
\biggl|\begin{matrix}
\cos 2\pi\sqrt{-\lambda}-1 & \sin 2\pi\sqrt{-\lambda}   \\
-\sin 2\pi\sqrt{-\lambda}  & \cos 2\pi\sqrt{-\lambda}-1
\end{matrix}\biggr|
\\
&=
\cos^2 2\pi\sqrt{-\lambda} -2\cos2\pi\sqrt{-\lambda}+1
+
\sin^22\pi\sqrt{-\lambda}
\\
&=
2(1-\cos2\pi\sqrt{-\lambda})
\qquad\Rightarrow\qquad
2\pi\sqrt{-\lambda} = 2k\pi,\;k\in\mathbb{Z},
\end{align*}
da die Maxima der Kosinusfunktion bei den Vielfachen von $2\pi$ liegen.
Somit sind die möglichen Eigenwerte $\lambda=-k^2$.

Die Matrix des Gleichungssystems zur Bestimmung der Koeffizienten $A$ und $B$
wird für $\lambda=-k^2$ zur Nullmatrix, somit sind beliebige Koeffizienten
$A$ und $B$ möglich.
Jede Linearkombination von $\cos kx$ und $\sin kx$ ist Eigenfunktion
des Operators $D^2$ mit Eigenwerte $-k^2$ mit periodischen Randbedingungen.
Dies sind die Basisfunktionen, die für die Fourier-Theorie verwendet
werden.

%
% Randbedingungen für \lambda > 0
%
\subsubsection{Randbedingungen für $\lambda > 0$}
Im Fall $\lambda >0$ muss eine Eigenfunktion die Form
\[
y(x)
=
Ae^{\sqrt{\lambda}x}
+
Be^{-\sqrt{\lambda}x}
\quad\text{mit}\quad
y'(x)
=
A\sqrt{\lambda}e^{\sqrt{\lambda}x}
-
B\sqrt{\lambda}e^{-\sqrt{\lambda}x}
\]
haben.
Die Diskussion der Eigenfunktionen wird etwas einfacher auf einem
symmetrischen Intervall $[-l,l]$.

Homogene Dirichlet-Randbedingungen werden erfüllt, wenn
\begin{align*}
0&=y(\phantom{-}l) = Ae^{l\sqrt{\lambda}} + Be^{-l\sqrt{\lambda}}
\\
0&=y(         - l) = Ae^{-l\sqrt{\lambda}} + Be^{l\sqrt{\lambda}}
\end{align*}
Dies ist ein homogenes lineares Gleichungssystem, es hat nur dann eine
nichttriviale Lösung, wenn die Determinante nicht verschwindet, wenn also
\[
0
=
\biggl|
\begin{matrix}
e^{l\sqrt{\lambda}}  & e^{-l\sqrt{\lambda}} \\
e^{-l\sqrt{\lambda}} & e^{l\sqrt{\lambda}}
\end{matrix}
\biggr|
=
e^{2l\sqrt{\lambda}} - e^{-2l\sqrt{\lambda}}
0
e^{2l\sqrt{\lambda}} ( 1 - e^{-4l\sqrt{\lambda}}).
\]
Die Exponentialfunktion hat keine reellen Nullstellen, daher
muss der Klammerausdruck verschwinden.
Dies ist aber nur für $\lambda=0$ möglich.
Homogene Randbedingungen lassen sich also für $\lambda>0$ nicht erfüllen.

Für homogene Neumann-Randbedingungen sind die zu erfüllenden Gleichungen
\begin{align*}
0&= y'(\phantom{-}l)
=
A\sqrt{\lambda} e^{l\sqrt{\lambda}} - B\sqrt{\lambda}e^{-l\sqrt{\lambda}}
\\
0&= y'(         - l) =
A\sqrt{\lambda} e^{-l\sqrt{\lambda}} - B\sqrt{\lambda}e^{l\sqrt{\lambda}}.
\end{align*}
Die Determinante dieses homogenen linearen Gleichungssystems ist
\[
\biggl|\begin{matrix}
\sqrt{\lambda} e^{l\sqrt{\lambda}}  & -\sqrt{\lambda}e^{-l\sqrt{\lambda}}
\\
\sqrt{\lambda} e^{-l\sqrt{\lambda}} & -\sqrt{\lambda}e^{l\sqrt{\lambda}}
\end{matrix}\biggr|
=
-\lambda
\biggl|\begin{matrix}
e^{l\sqrt{\lambda}}  & e^{-l\sqrt{\lambda}}
\\
e^{-l\sqrt{\lambda}} & e^{l\sqrt{\lambda}}
\end{matrix}\biggr|
=
-\lambda
(
e^{2l\sqrt{\lambda}} - e^{-2l\lambda}
).
\]
Wie für Dirichlet-Randbedingungen folgt, dass auch diese Randbedingungen
nicht erfüllt werden können.

Wir versuchen auch noch periodische Randbedingungen zu erfüllen, also
die Gleichungen
\[
\begin{aligned}
y(-l) &= y(l)\\
y'(-l)&=y'(l)
\end{aligned}
\quad\Rightarrow\quad
\renewcommand{\arraycolsep}{2pt}
\begin{array}{rcrcrcr}
e^{-l\sqrt{\lambda}}A
&+&
e^{l\sqrt{\lambda}}B
&=&
e^{l\sqrt{\lambda}}A
&+&
e^{-l\sqrt{\lambda}}B
\phantom{.}
\\
\sqrt{\lambda}e^{-l\sqrt{\lambda}}A
&-&
\sqrt{\lambda}e^{l\sqrt{\lambda}}B
&=&
\sqrt{\lambda}e^{l\sqrt{\lambda}}A
&-&
\sqrt{\lambda}e^{-l\sqrt{\lambda}}B.
\end{array}
\]
Bringt man alles auf eine Seite, entsteht ein Gleichungssystem mit der
Koeffizientenmatrix
\[
\begin{pmatrix}
-e^{l\sqrt{\lambda}} + e^{-l\sqrt{\lambda}}
	& e^{l\sqrt{\lambda}} - e^{-l\sqrt{\lambda}}
\\
\sqrt{\lambda}(
-
e^{l\sqrt{\lambda}}
+
e^{-l\sqrt{\lambda}}
)
	& \sqrt{\lambda}(-e^{l\sqrt{\lambda}} + e^{-l\sqrt{\lambda}})
\end{pmatrix}
=
\begin{pmatrix*}[r]
-              2\sinh(l\sqrt{\lambda})&               2\sinh(l\sqrt{\lambda})\\
-\sqrt{\lambda}2\sinh(l\sqrt{\lambda})&-\sqrt{\lambda}2\sinh(l\sqrt{\lambda})
\end{pmatrix*}
\]
mit der Determinante
\[
4\sqrt{\lambda}
\bigl(
\sinh^2(l\!\sqrt{\lambda})
+
\sinh^2(l\!\sqrt{\lambda})
\bigr)
=
8\sqrt{\lambda} \sinh^2(l\sqrt{\lambda}),
\]
die nur verschwindet, wenn $l=0$ ist.
Somit lassen sich auch periodische Randbedingungen für $\lambda > 0$
nicht erfüllen.

%
% Funktionen auf einem Rechteck
%
\subsection{Funktionen auf einem Rechteck}
In diesem Abschnitt betrachten wir den Operator 
\[
\Delta
=
\frac{\partial^2}{\partial x^2}
+
\frac{\partial^2}{\partial y^2}
\]
auf einem Rechteck $\Omega = (0,l_x)\times (0,l_y)$.
Das Skalarprodukt ist wieder das übliche $L^2$-Skalarprodukt.


%
% $\Delta$ ist selbstadjungiert
%
\subsubsection{$\Delta$ ist selbstadjungiert}
Seien $f$ und $g$ zweimal stetig differenzierbare Funktionen und berechnen
das Skalarprodukt
\begin{align}
\langle \Delta f, g\rangle
&=
\int_0^{l_x}\int_0^{l_y} \Delta f(x,y) g(x,y) \,dx\,dy
\notag
\\
&=
\int_0^{l_x}\int_0^{l_y} \frac{\partial^2 f}{\partial x^2}(x,y)\,g(x,y)\,dx\,dy
+
\int_0^{l_x}\int_0^{l_y} \frac{\partial^2 f}{\partial y^2}(x,y)\,g(x,y)\,dx\,dy
\notag
\\
&=
\int_0^{l_x}\int_0^{l_y} \frac{\partial^2 f}{\partial x^2}(x,y)\,g(x,y)\,dx\,dy
+
\int_0^{l_y}\int_0^{l_x} \frac{\partial^2 f}{\partial y^2}(x,y)\,g(x,y)\,dy\,dx
\notag
\\
&=\phantom{+}
\int_0^{l_y}
\left[ \frac{\partial f}{\partial x}(x,y)\, g(x,y)\right]_0^{l_x}
\,dy
-
\int_0^{l_y}\int_0^{l_x}
\frac{\partial f}{\partial x}(x,y)\, \frac{\partial g}{\partial x}(x,y)
\,dx\,dy
\notag
\\
&\phantom{=}+
\int_0^{l_x}
\left[ \frac{\partial f}{\partial y}(x,y)\, g(x,y)\right]_0^{l_x}
\,dx
-
\int_0^{l_x}\int_0^{l_y}
\frac{\partial f}{\partial y}(x,y)\, \frac{\partial g}{\partial y}(x,y)
\,dy\,dx
\notag
\\
&=\phantom{+}
\int_0^{l_x} \frac{\partial f}{\partial y}(x,l_y)\, g(x,l_y) \,dx
-
\int_0^{l_x} \frac{\partial f}{\partial y}(x,0)\, g(x,0) \,dx
\label{buch:orthofkt:pdf:eqn:rechteck:r1}
\\
&\phantom{=}+
\int_0^{l_y} \frac{\partial f}{\partial x}(l_x,y)\, g(l_x,y) \,dy
-
\int_0^{l_y} \frac{\partial f}{\partial x}(0,y)\, g(0,y) \,dy
\label{buch:orthofkt:pdf:eqn:rechteck:r2}
\\
&\phantom{=}
-
\int_0^{l_x}
\int_0^{l_y}
\frac{\partial f}{\partial x} \frac{\partial g}{\partial x}
+
\frac{\partial f}{\partial y} \frac{\partial g}{\partial y}
\,dy\,dx.
\label{buch:orthofkt:pdf:eqn:rechteck:total}
\end{align}
Die ersten vier Integrale auf der rechten Seite hängen nur ab
von den Werten der Funktionen und der Ableitungen auf dem Rand.
Erfüllen die Funktionen homogene Dirichlet-Randbedingungen, dann
ist der Integrand immer 0 und auf der rechten Seite bleibt nur
der Term \eqref{buch:orthofkt:pdf:eqn:rechteck:total}.
%\[
%-\int_0^{l_x} \int_0^{l_y}
%\frac{\partial f}{\partial x} \frac{\partial g}{\partial x}
%+
%\frac{\partial f}{\partial y} \frac{\partial g}{\partial y}
%\,dy\,dx.
%\]
Der gleiche Term bleibt auch bei der Berechnung von
$\langle f,\Delta g\rangle$
stehen, so dass für Funktionen, die homogene Dirichlet-Randbedingungen
erfüllen, der Operator $\Delta$ selbstadjungiert ist.

Die Integrale
\eqref{buch:orthofkt:pdf:eqn:rechteck:r1}
über $x$ sind Integrale entlang der zur $x$-Achse parallelen Teile des
Randes von $\Omega$.
Die Ableitung $\partial f/\partial y$ ist die Ableitung in Richtung
senkrecht auf den Rand des Gebietes, die Normalableitung.
In den Integralen
\eqref{buch:orthofkt:pdf:eqn:rechteck:r2}
über $y$ sind Integrale entlang der zur $y$-Achse parallelen Teile
tritt die Ableitung $\partial f/\partial x$ auf, die die Normalableitung
ist.
Erfüllen die Funktionen homogene Neumann-Randbedingungen, dann sind
diese Normalableitungen $=0$ und die Integrale
\eqref{buch:orthofkt:pdf:eqn:rechteck:r1}
und
\eqref{buch:orthofkt:pdf:eqn:rechteck:r2}
verschwinden.
Auch in diesem Fall ist $\Delta$ auf den Funktionen, die homogene
Neumann-Randbedingungen erfüllen, selbstadjungiert.

%
% Eigenfunktionen
%
\subsubsection{Eigenfunktionen}
Eigenfunktionen des Laplace-Operators auf einem Rechteck sind
Lösungen der Differentialgleichung $\Delta f=\lambda f$.
Nimmt man an, dass die Funktion $f$ als Produkte zweier Funktionen
$X(x)$ und $Y(y)$ geschrieben werden kann, kann man den Ansatz
$f(x,y) = X(x)Y(y)$ in die Differentialgleichung einsetzen und erhält
\[
X''(x)Y(y) + X(x)Y''(y) = \lambda X(x)Y(y).
\]
Ausserhalb der Nullstellen von $X(x)$ und $Y(y)$ kann man durch das
Produkt teilen und bekommt
\[
\frac{X''(x)}{X(x)}
+
\frac{Y''(y)}{Y(y)}
=
\lambda
\qquad\Rightarrow\qquad
\frac{X''(x)}{X(x)}
=
\lambda
-
\frac{Y''(y)}{Y(y)}.
\]
Die linke Seite der letzten Gleichung hängt nur von $x$ ab,
die rechte Seite hängt nur von $y$ ab.
Dies bedeutet, dass beide Seiten konstant sind.
Es gibt also eine Konstante $\mu$ derart, dass
\begin{align*}
\frac{X''(x)}{X(x)}&= \mu
&&\Rightarrow
&
X''(x)&=\mu X(x)
&&\text{für $x\in (0,l_x)$}
\\
\frac{Y''(y)}{Y(y)}&=\lambda-\mu
&&\Rightarrow
&
Y''(y)&=(\lambda-\mu)Y(y)
&&\text{für $y\in (0,l_y)$}
\end{align*}
Damit ist das Problem auf das einimensionale Problem von
Abschnitt~ \ref{buch:orthofkt:subsection:fourier-theorie}
reduziert.
Für Dirichlet-Randbedingungen sind die Funktionen
\[
X(x) = \sin\frac{\pi k_x}{l_x} x
\quad\text{und}\quad
Y(y) = \sin\frac{\pi k_y}{l_y} y
\]
zu verwenden, die zugehörigen Eigenwerte sind
\[
\lambda
=
-\pi^2\biggl(
\frac{k_{x\mathstrut}^2\mathstrut}{l_x^2}
+
\frac{k_{y\mathstrut}^2\mathstrut}{l_y^2}
\biggr),
\quad
k_x,k_y\in\mathbb{Z}.
\]
Auf die gleiche Art und Weise können
auch für homogene Neumann-Randbedingungen Lösungen mit 
den Kosinusfunktionen gefunden werden.

%
% Periodische Randbedingungen
%
\subsubsection{Perdiodische Randbedingungen}
Die Integrale 
\eqref{buch:orthofkt:pdf:eqn:rechteck:r1}
und
\eqref{buch:orthofkt:pdf:eqn:rechteck:r2}
können aber auch zum verschwinden gebracht werden, wenn man sicherstellt,
wenn die Funktionswerte und Normalableitungen auf den Rändern des Gebietes
übereinstimmen.
Die Bedingungen
\begin{align*}
f(x,l_y) &= f(x,0)
&&\text{und}&
\frac{\partial f}{\partial x}(x,l_y) &= \frac{\partial f}{\partial x}(x,0)
\\
f(l_x,y) &= f(0,y)
&&\text{und}&
\frac{\partial f}{\partial y}(l_x,y) &= \frac{\partial f}{\partial y}(0,y).
\intertext{Aus den  ersten beiden Gleichung folgt ausserdem, dass}
\frac{\partial f}{\partial x}(x,0) &=
\frac{\partial f}{\partial x}(x,l_y)
&&\text{und}&
\frac{\partial f}{\partial y}(0,y) &=
\frac{\partial f}{\partial y}(l_x,y).
\end{align*}
Eine solche Funktion kann also $l_x$-periodisch in $x$ und $l_y$-periodisch
in $y$ auf ganz $\mathbb{R}$ erweitert werden, so dass die erweiterte Funktion
und ihre Ableitungen immer noch stetig sind.

%
% Funktionen in einem Kreisgebiet
%
\subsection{Funktionen in einem Kreisgebiet}
Ein Kreisgebiet $\Omega = \{(x,y)\mid x^2+y^2< r^2\}$ ist am einfachsten
mit Polarkoordinaten zu beschreiben.
Der Laplace-Operator in Polarkoordinaten ist
\[
\Delta 
=
\frac1r \frac{\partial}{\partial r} r \frac{\partial}{\partial r}
+
\frac{1}{r^2} \frac{\partial}{\partial\varphi}.
\]
Funktionen auf dem Kreisgebiet sind Funktion $f(r,\varphi)$ müssen
\[
f(0,\varphi) = f(0,0)
\qquad\text{und}\qquad
f(r,\varphi) = f(r,\varphi+2\pi)
\]
erfüllt sein, da die Funktion sonst nicht stetig sein kann.
Damit die Funktion auch im Nullpunkt differenzierbar ist, müssen die
Ableitungen für $r=0$ zusammenpassen.
Die Richtungsableitungen in $x$- und $y$-Richtung sind
\[
\frac{\partial f}{\partial r}(0,0)
\qquad\text{und}\qquad
\frac{\partial f}{\partial r}(0,{\textstyle\frac{\pi}2}).
\]
In jeder anderen Richtung ist die Richtungsableitung
\begin{equation}
\frac{\partial f}{\partial r}(0,\varphi)
=
\begin{pmatrix}
\frac{\partial f}{\partial r}(0,0)\\
\frac{\partial f}{\partial r}(0,{\textstyle\frac{\pi}2})\\
\end{pmatrix}
\cdot
\begin{pmatrix}
\cos\varphi\\
\sin\varphi
\end{pmatrix}
=
\frac{\partial f}{\partial r}(0,0)\cos\varphi
+
\frac{\partial f}{\partial r}(0,{\textstyle\frac{\pi}2})\sin\varphi.
\label{buch:orthofkt:eqn:phibed}
\end{equation}
Die Gleichung
\eqref{buch:orthofkt:eqn:phibed}
ist eine Bedingung, die Funktion $f$ erfüllen muss, damit $f$
differenzierbar ist.

%
% \Delta ist selbstadjungiert
%
\subsubsection{$\Delta$ ist selbstadjungiert}
Seien $f$ und $g$ auf dem Kreisgebiet definiert, dann können wir damit
das Skalarprodukt
\begin{align}
\langle \Delta f, g\rangle
&=
\int_0^R \int_0^{2\pi}  \Delta f(r,\varphi) g(r,\varphi)\,r\,d\varphi\,dr
\notag
\\
&=
\int_0^R \int_0^{2\pi}
\biggl(
\frac1r \frac{\partial}{\partial r}r\frac{\partial f}{\partial r}(r,\varphi)
+
\frac1{r^2}\frac{\partial^2 f}{\partial\varphi^2}(r,\varphi)
\biggr)
g(r,\varphi)
\, r\,d\varphi\,dr
\notag
\\
&=
\int_0^{2\pi}
\int_0^R
\frac{\partial}{\partial r}r\frac{\partial f}{\partial r}(r,\varphi)
g(r,\varphi)
\,dr
\,
d\varphi
+
\int_0^R
\frac1{r}
\int_0^{2\pi}
\frac{\partial^2 f}{\partial\varphi^2}(r,\varphi)
g(r,\varphi)
\,d\varphi
\,dr
\notag
\\
&=
\phantom{+}
\int_0^{2\pi}
\biggl[
\frac{\partial f}{\partial r}(r,\varphi)\, g(r,\varphi)
\biggr]_0^R
\,r
\,d\varphi
-
\int_0^R
\int_0^{2\pi}
\frac{\partial f}{\partial r}
\frac{\partial g}{\partial r}
\,d\varphi
\,dr
\label{buch:orthofkt:pde:eqn:kreisdr}
\\
&\phantom{=}
+
\int_{0}^R
\frac1r
\biggl[
\frac{\partial f}{\partial \varphi}(r,\varphi)
g(r,\varphi)
\biggr]_0^{2\pi}
\,dr
-
\int_{0}^R
\frac1r
\int_0^{2\pi}
\frac{\partial f}{\partial\varphi}(r,\varphi)
\frac{\partial g}{\partial\varphi}(r,\varphi)
\,d\varphi
\,dr
\label{buch:orthofkt:pde:eqn:kreisdphi}
\end{align}
berechnen.
Das erste Integral in 
\eqref{buch:orthofkt:pde:eqn:kreisdphi}
verschwindet, weil $f$ und $g$ $2\pi$-periodisch sind in $\varphi$.
Das erste Integral in 
\eqref{buch:orthofkt:pde:eqn:kreisdr}
hängt nur von den Werten von $g$ und der Ableitung von $f$ in $r$-Richtung
auf dem Rand des Gebietes ab.
Die doppelten Integrale sind symmetrisch in $f$ und $g$.
Daraus kann man wieder folgern, dass $\Delta$ auf Funktionen mit
einer homogenen Dirichlet- oder einer homogenen Neumann-Randbedingung
selbstadjungiert ist.

%
% Eigenfunktionen
%
\subsubsection{Eigenfunktionen}
Mit dem Ansatz $f(r,\varphi) = R(r) \Phi(\varphi)$ kann man versuchen,
Lösungen der partiellen Differentialgleichung $\Delta f=\lambda f$
zu finden.
Einsetzen in die Differentialgleichung führt auf
\[
\frac1r \frac{d}{dr} rR'(r) \Phi(\varphi)
+
\frac1{r^2} \Phi''(\varphi)
=
\lambda R(r)\Phi(\varphi).
\]
Auch hier können die Funktionen $R(r)$ und $\Phi(\varphi)$ mittels
Division durch $f(r,\varphi)$ getrennt werden.
Aus
\[
\frac1r \frac{R'(r) + rR''(r)}{R(r)}
+
\frac{1}{r^2}\frac{\Phi''(\varphi)}{\Phi(\varphi)}
=
\lambda
\]
werden die beiden Gleichungen
\[
\frac{r^2R''(r) + rR'(r)}{R(r)}
-
\lambda r^2
=
-\frac{\Phi''(\varphi)}{\Phi(\varphi)}
\quad\Rightarrow\quad
\left\{
\begin{aligned}
r^2R''(r) + rR'(r) +(m^2 - \lambda r^2)R(r) &= 0
\\
\Phi''(\varphi)&=-m^2 \Phi(\varphi)
\end{aligned}
\right.
\]
Die zweite Gleichung hat die trigonometrischen Funktionen
als Lösungen.
Die erste Gleichung ist etwas schwieriger zu lösen.
Sie kann zum Beispiel mit der Potenzreihenmethode gelöst werden, die
Lösungsfunktionen sind die Bessel-Funktionen.

%
% Funktionen auf einer Kugeloberfläche
%
\subsection{Funktionen auf einer Kugeloberfläche}
Die Wellenausbreitung auf der Erdoberfläche kann mit Funktionen
auf einer Kugeloberfläche modelliert werden.
Die Wellengleichung verwendet den Laplace-Operator, der in Kugelkoordinaten
die Form
\begin{equation}
\Delta
=
\frac{1}{r^2}
\frac{\partial}{\partial r}
r^2
\frac{\partial}{\partial r}
+
\frac{1}{r^2\sin\vartheta}
\frac{\partial}{\partial\vartheta}
\biggl(\sin\vartheta\frac{\partial}{\partial\vartheta}\biggr)
+
\frac{1}{r^2\sin^2\vartheta}
\frac{\partial^2}{\partial\varphi^2}
\label{buch:orthofkt:pde:laplacekugel}
\end{equation}
hat.
Die Abhängigkeit von $r$ wird auf der Kugeloberfläche nicht benötigt,
so dass sich der Operator vereinfacht auf
\[
\Delta
=
\frac{1}{\sin\vartheta}
\frac{\partial}{\partial\vartheta}
\biggl(\sin\vartheta\frac{\partial}{\partial\vartheta}\biggr)
+
\frac{1}{\sin^2\vartheta}
\frac{\partial^2}{\partial\varphi^2}.
\]
Als Skalarprodukt wird das gewöhnliche $L^2$-Skalarprodukt
\[
\langle f,g\rangle
=
\int_{0}^{2\pi}
\int_0^\pi
f(\vartheta,\varphi)
g(\vartheta,\varphi)
\sin \vartheta
\,d\vartheta
\,d\varphi
\]
verwendet.

%
% \Delta ist selbstadjungiert
%
\subsubsection{$\Delta$ ist selbstadjungiert}
Der Laplace-Operator ist auch für Funktionen auf der Kugeloberfläche
selbstadjungiert.
Da die Kugeloberfläche keinen Rand hat, sind die einzige Bedingungen an
die Funktionen, dass sie auf der Kugeloberfläche differenzierbar sind.
Dazu müssen sie in $\varphi$ $2\pi$-periodisch sein und 
$f(0,\varphi)$ und $f(\pi,\varphi)$ müssen konstant sein.
Damit die Ableitungen an den Polen der Kugel existieren, müssen die
Bedingungen
\[
\frac{\partial f}{\partial\vartheta}(\vartheta,\varphi)
=
\frac{\partial f}{\partial\vartheta}(\vartheta,0)
\cos\varphi
+
\frac{\partial f}{\partial\vartheta}(\vartheta,{\textstyle\frac{\pi}2})
\sin\varphi
\]
erfüllen.
\begin{align}
\langle \Delta f, g\rangle
&=
\int_{0}^{2\pi}
\int_0^\pi
\Delta
f(\vartheta,\varphi)
g(\vartheta,\varphi)
\sin \vartheta
\,d\vartheta
\,d\varphi
\notag
\\
&=
\int_{0}^{2\pi}
\int_0^\pi
\biggl(
\frac{1}{\sin\vartheta}\frac{\partial}{\partial\vartheta}
\biggl(
\sin\vartheta\frac{\partial f}{\partial\vartheta}
\biggr)
+
\frac{1}{\sin^2\vartheta}
\frac{\partial^2 f}{\partial\varphi^2}
\biggr)
g(\vartheta,\varphi)
\sin \vartheta
\,d\vartheta
\,d\varphi
\notag
\\
&=
\int_{0}^{2\pi}
\int_0^\pi
\frac{\partial}{\partial\vartheta}
\biggl(
\sin\vartheta\frac{\partial f}{\partial\vartheta}
\biggr)
\,d\vartheta
\,d\varphi
+
\int_{0}^{2\pi}
\int_0^\pi
\frac{1}{\sin^2\vartheta}
\frac{\partial^2 f}{\partial\varphi^2}
g(\vartheta,\varphi)
\sin \vartheta
\,d\vartheta
\,d\varphi
\notag
\\
&=\phantom{+}
\int_0^{2\pi}
\biggl[
\sin\vartheta\frac{\partial f}{\partial\vartheta}(\vartheta,\varphi)
g(\vartheta,\varphi)
\biggr]_0^\pi
\,d\varphi
-
\int_0^{2\pi}
\int_0^\pi
\frac{\partial f}{\partial\vartheta}(\vartheta,\varphi)
\frac{\partial g}{\partial\vartheta}(\vartheta,\varphi)
\sin\vartheta
\,d\vartheta
\,d\varphi
\label{buch:orthofkt:pde:deltakugel1}
\\
&\phantom{=}+
\int_0^\pi \frac{1}{\sin\vartheta}
\biggl[
\frac{\partial f}{\partial\varphi}(\vartheta,\varphi)\, g(\vartheta,\varphi)
\biggr]_0^{2\pi}
\,d\vartheta
-
\int_0^\pi
\int_{0}^{2\pi}
\frac{\partial f}{\partial\varphi}
\frac{\partial g}{\partial\varphi}
\sin\vartheta
\,d\varphi
\,d\vartheta
\label{buch:orthofkt:pde:deltakugel2}
\end{align}
Das erste Integral in
\eqref{buch:orthofkt:pde:deltakugel2} verschwindet, weil die
Funktion und die Ableitungen $2\pi$-periodisch in $\varphi$
sind.
Das erste Integral in
\eqref{buch:orthofkt:pde:deltakugel1}
kann man durch die Beobachtung vereinfachen, dass für
$\vartheta=0$ und $\vartheta=\pi$.
die Funktionen konstant sein müssen.
Das erste Integral in \eqref{buch:orthofkt:pde:deltakugel1}
verschwindet, weil $\sin0=\sin\vartheta=0$.
Die verbleibenden doppelten Integrale in \eqref{buch:orthofkt:pde:deltakugel1}
und \eqref{buch:orthofkt:pde:deltakugel2} sind symmetrisch in $f$ und $g$.
Somit ist der Laplace-Operator auf beliebigen auf der Kugeloberfläche
differenzierbaren Funktionen selbstadjungiert.
Aus der allgemeinen Theorie folgt somit, dass die Eigenfunktionen orthogonal
sind.

%
% Lösungen der Legendre-Differentialgleichung
%
\subsubsection{Lösungen der Legendre-Differentialgleichung}
Die Eigenfunktionen des Laplace-Operators auf der Kugeloberfläche
kann wieder mit einem Ansatz
$f(\vartheta,\varphi)=\Theta(\vartheta)\Phi(\varphi)$
gefunden werden.
Einsetzen in die Differentialgleichung gibt
\begin{align*}
\frac{1}{\sin\vartheta}
\frac{\partial}{\partial\vartheta}\bigl(
\sin\vartheta \Theta'(\vartheta)\Phi(\varphi)
\bigr)
+
\frac{1}{\sin^2\vartheta} \Phi''(\varphi)\Theta(\vartheta)
&=
\lambda
\Theta(\vartheta)\Phi(\varphi)
\\
\frac{1}{\sin\vartheta}
\bigl(
\cos\vartheta \Theta'(\vartheta)
+
\sin\vartheta\Theta''(\vartheta)
\bigr)
\Phi(\varphi)
+
\frac{1}{\sin^2\vartheta} \Phi''(\varphi)\Theta(\vartheta)
&=
\lambda
\Theta(\vartheta)\Phi(\varphi).
\intertext{Division durch $f(\vartheta,\varphi)$ und Multiplikation
mit $\sin^2\vartheta$ ergibt}
\sin\vartheta
\frac{
\cos\vartheta \Theta'(\vartheta) +\sin\vartheta\Theta''(\vartheta)
}{
\Theta(\vartheta)
}
+
\frac{\Phi''(\varphi)}{\Phi(\varphi)}
&=
\lambda\sin^2\vartheta,
\intertext{in der man $\vartheta$ auf die linke Seite und $\varphi$ auf
die rechte Seite verschieben kann}
\sin\vartheta
\frac{
\cos\vartheta\Theta'(\vartheta)
+
\sin\vartheta\Theta''(\vartheta)
-\lambda\sin\vartheta \Theta(\vartheta)
}{
\Theta(\vartheta)
}
&=
-\frac{\Phi''(\varphi)}{\Phi(\varphi)}
=
m^2
\end{align*}
Für die Funktionen $\Phi(\varphi)$ sind wieder beliebige Linearkombinationen
der trigonometrischen Funktionen $\cos m\varphi$ und $\sin m\varphi$
die Lösungen.
Alternativ können sie auch als komplexe Funktionen
$\Phi_m(\varphi) = e^{im\varphi}$ geschrieben werden.

Die Funktion $\Theta(\vartheta)$ erfüllt die Differentialgleichung
\[
\sin^2\vartheta \Theta''(\vartheta)
+
\sin\vartheta\cos\vartheta
\Theta'(\vartheta)
-
\lambda\sin^2\vartheta
\Theta(\vartheta)
=
m^2 \Theta(\vartheta).
\]
Division durch $\sin^2\vartheta$ kann die Gleichung etwas vereinfachen:
\begin{equation}
\Theta''(\vartheta)
+
\frac{\cos\vartheta}{\sin\vartheta} \Theta'(\vartheta)
+\biggl(
-
\lambda
-
\frac{m^2}{\sin^2\vartheta} 
\biggr)
\Theta(\vartheta)
=
0
\label{buch:orthfkt:pde:dglthetaphi}
\end{equation}
Diese Gleichung ist bekannt als die {\em assoziierte
Legendre-Differentialgleichung}.
Man kann zeigen, dass sie $\lambda=-l(l+1)$ mit $l\in\mathbb{N}$
als Eigenwerte hat.

\begin{satz}
\label{buch:orthofkt:pde:satz:kugel}
Der Laplace-Operator auf der Kugeloberfläche hat die Form
\eqref{buch:orthofkt:pde:laplacekugel}.
Seine Eigenfunktionen sind von der Form
$Y_l^m(\vartheta,\varphi)=e^{im\varphi}\Theta_l^m(\vartheta)$,
mit $l\in \mathbb{N}$ und ganzzahligen $m$ mit $-l\le m\le l$.
Die Funktionen $\Theta_l^m(\vartheta)$ sind Lösungen der
Differentialgleichung \eqref{buch:orthfkt:pde:dglthetaphi}, wobei
der Eigenwert $\lambda = -l(l+1)$ ist.
\end{satz}

Die Berechnung der Funktionen $\Theta_l^m(\vartheta)$ wird viel
einfacher in Abhängigkeit von der Koordinaten $z=\cos\vartheta$
anstelle von $\vartheta$.

%
% Koordinatentransformation z = cos theta
%
\subsubsection{Koordinatentransformation $z=\cos\vartheta$}
Die Parametrisierung mit $\vartheta$
in \eqref{buch:orthfkt:pde:dglthetaphi}
ist etwas unhandlich, daher soll der Parameter
$\vartheta$ durch $z=\cos\vartheta$ und
$\sin\vartheta = \sqrt{1-z^2}$
ersetzt werden.
Die Funktion $\Theta(\vartheta)$ ist dann eine Funktion $P(z)$ mit
$P(\cos\vartheta)=\Theta(\vartheta)$.
Die Ableitungen von $\Theta$ können durch Ableitungen von $P$
ausgedrückt werden wie folgt.
Zunächst sind die Ableitungen nach $\vartheta$:
\begin{align*}
\Theta'(\vartheta)
&=
\frac{d}{d\vartheta} P(\cos\vartheta)
=
-P'(\cos\vartheta) \sin\vartheta
\\
&=
-
\sqrt{1-z^2}
P'(z)
\\
\Theta''(\vartheta)
&=
\frac{d}{d\vartheta} 
\Theta'(\vartheta)
=
-\frac{d}{d\vartheta}\bigl(P'(\cos\vartheta) \sin\vartheta\bigr)
=
P''(\cos\vartheta)\sin^2\vartheta
-
P'(\cos\vartheta)\cos\vartheta
\\
&=
(1-z^2)
P''(z)
-
z
P'(z)
\end{align*}
Damit kann man die Differentialgleichung
\eqref{buch:orthfkt:pde:dglthetaphi}
in die Form
\[
(1-z^2)P''(z)
-
zP'(z)
-
\frac{z}{\sqrt{1-z^2}}\sqrt{1-z^2}P'(z)
-
\lambda P(z)
-
\frac{m^2}{1-z^2}P(z)
=
0
\]
für die Funktion $P(z)$ finden.
\begin{equation}
(1-z^2)P''(z) - 2zP'(z)
+
\biggl(
-
\lambda
-
\frac{m^2}{1-z^2}
\biggr)
P(z)
=
0
\label{buch:orthofkt:pde:eqn:legendreassocz}
\end{equation}
Diese Differentialgleichung kann auch in der Form des
Sturm-Liouville-Operators
\[
L
=
\frac{d}{dz}
(1-z^2)
\frac{d}{dz}
+
\biggl(-\lambda-\frac{m^2}{1-z^2}\biggr)
\]
geschrieben werden.
Solche Operatoren haben wir früher studiert.

%
% Eigenfunktionen
%
\subsubsection{Eigenfunktionen}
Die Lösungen der assozierten Legendre-Differentialgleichung
in der Form 
\[
(1-z^2)P''(z)
-2z P'(z) +
\biggl(
-l(l+1)-\frac{m^2}{1-z^2}
\biggr)P(z)
=
0
\]
können mit Polynomen gelöst werden.
Sie heissen die {\em assoziierten Legendre-Polynome} $P_l^m(z)$.
Eine ausführliche Diskussion dieser Lösungen kann in
\cite{buch:mathsem-spezfunk}
gefunden werden.

Mit Hilfe der assoziierten Legendre-Polynomen können die Eigenfunktion
von $\Delta$ als
\[
e^{im\varphi}
P_l^m(z)
\qquad\text{mit $-l\le m\le l$}
\]
geschrieben werden.





\uebungsabschnitt

\aufgabetoplevel{chapters/020-orthofkt/uebungsaufgaben}
\begin{uebungsaufgaben}
\uebungsaufgabe{201}
%\uebungsaufgabe{102}
%\uebungsaufgabe{103}
%\uebungsaufgabe{104}
\end{uebungsaufgaben}
\enduebungsabschnitt
