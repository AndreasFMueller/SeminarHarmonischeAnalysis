%
% 4-sturmliouville.tex -- Sturm-Liouville-Probleme
%
% (c) 2022 Prof Dr Andreas Müller, OST Ostschweizer Fachhochschule
%
\section{Sturm-Liouville-Problem
\label{buch:orthofkt:section:sturmliouville}}
\kopfrechts{Sturm-Liouville-Problem}
Das Beispiel~\ref{buch:orthfkt:sa:beispiel:D2} zeigt, dass
Eigenfunktionen eines Differentialoperators zu interessanten
Familien orthogonaler Funktionken führen können.
Die daraus abgeleitete harmonische Analysis kann zur Lösung
gewisser Differentialgleichungen genutzt werden.
Ziel dieses Abschnitts ist zu zeigen, dass das Beispiel auf
eine wesentlich grössere Klasse von Differentialgleichungen
erweitert werden kann.

%
% Der Sturm-Liouville-Differentialoperator
%
\subsection{Der Sturm-Liouville-Differentialgoperator}
Sie $a<b$ und seien zwei Funktionen
$p(x)\in C^2([a,b])$, $q(x)\in C([a,b])$ gegeben.
Wir definieren den Sturm-Liouville-Operator
\[
L = \frac{d}{dx} p(x) \frac{d}{dx} + q(x),
\]
der die Funktion $y(x)$ auf
\begin{align*}
(Ly)(x)
&=
\frac{d}{dx}p(x)\frac{d}{dx}y(x) + q(x)y(x)
\\
&=
\frac{d}{dx}p(x)y'(x) + q(x)y(x)
\\
&=
p'(x)y'(x)+p(x)y''(x)+q(x)y(x)
\end{align*}
abbildet.
Die Funktion $q(x)$ wirkt also durch punktweise Multiplikation.

Ableitungen kommen nur im ersten Teil vor, den wir mit
\[
L_0
=
\frac{d}{dx}p(x)\frac{d}{dx}
\]
bezeichnen.

%
% Der Sturm-Liouville-Operator ist selbstadjungiert
%
\subsection{Der Sturm-Liouville-Operator ist selbstadjungiert}
Wir betrachten die Frage der Selbstadjungiertheit des
Operators $L$ bezüglich des Standardskalarproduktes
\[
\langle f,g\rangle
=
\int_a^b f(x)g(x)\,dx.
\]
Der Multiplikationsoperator mit der Funktion $q(x)$ ist ganz
offensichtlich selbstadjungiert, es gilt nämlich
\[
\langle qf,g\rangle
=
\int_a^b (q(x)f(x))g(x)\,dx
=
\int_a^b f(x)(q(x)g(x))\,dx
=
\langle f,qg\rangle
\]
für alle Funktionen $f$ und $g$, für die die Integrale definiert sind.

Der Operator $L_0$ hat ähnliche Eigenschaften wie die zweite
Ableitung im Beispiel~\ref{buch:orthfkt:sa:beispiel:D2}.
Dazu berechnen wir die Skalarprodukte
\begin{align}
\langle Lf,g\rangle
&=
\int_a^b (Lf)(x)g(x)\,dx
=
\int_a^b \frac{d}{dx}\biggl(q(x)\frac{d}{dx}f(x)\biggr) g(x)\,dx
\\
&=
\biggl[ q(x)f'(x) g(x) \biggr]_a^b
-
\int_a^b q(x)f'(x)g'(x)\,dx
\label{buch:orthfkt:sturmliouville:Lfg}
\\
\langle f,Lg\rangle
&=
\int_a^b f(x) (Lg)(x)\,dx
=
\int_a^b f(x)\frac{d}{dx}\biggl(q(x)\frac{d}{dx}g(x)\biggr)\,dx
\\
&=
\biggl[ f(x) q(x)g'(x)\biggr]_a^b
-
\int_a^b f'(x)q(x)g'(x)\,dx.
\label{buch:orthfkt:sturmliouville:fLg}
\end{align}
Die beiden Integrale stimmen überein, aber der erste Term ist
offensichtlich verschieden.
$L$ ist also im allgemeinen nicht selbstadjungiert.

%
% Randbedingungen
%
\subsection{Randbedingungen}
Aus den Gleichungen
\eqref{buch:orthfkt:sturmliouville:Lfg}
und
\eqref{buch:orthfkt:sturmliouville:fLg}
kann man Bedingungen an die Funktionen ableiten, die überhaupt
zugelassen werden sollen, die den Operator $L_0$ zu einem
selbstadjungierten Operator machen.
Die Funktionen müssen die Bedingungen
\begin{align}
\biggl[ p(x)f'(x) g(x) \biggr]_a^b
&=
\biggl[ p(x)f(x) g'(x) \biggr]_a^b
\notag
\\
p(b)f'(b)g(b)
-
p(a)f'(a)g(a) 
&=
p(b)f(a)g'(b)
-
p(a)f(a)g'(a)
\notag
\intertext{oder}
p(b)
\bigl(
f'(b)g(b)-f(b)g'(b)
\bigr)
&=
p(a)
\bigl(
f'(a)g(b)
-
f(a)g'(a)
\bigr)
\notag
\intertext{erfüllen.
Mit der Determinante lässt sich das noch etwas übersichtlicher
schreiben als}
\left|\begin{matrix}
p(b)f'(b)&p(b)g'(b) \\
f(b) &g(b)
\end{matrix}\right|
=
\left|\begin{matrix}
p(a)f'(a)&p(a)g'(a) \\
f(a) &g(a)
\end{matrix}\right|.
\label{buch:orthfkt:sturmliouville:det}
\end{align}

Randbedingungen können immer nur für einen Randpunkt gefordert
werden, wir müssen also den Wert der Determinante festlegen.
Angenommen, wir fordern, dass die Determinante einen Wert
$d\ne 0$ annimmt. Setzen wir die Funktione $\lambda f$ und $g$ ein,
ensteht die Determinante
\[
\left|\begin{matrix}
\lambda p(a)f'(a)&p(a)g'(a) \\
\lambda f(a) &g(a)
\end{matrix}\right|
=
\lambda
\left|\begin{matrix}
p(a)f'(a)&p(a)g'(a) \\
f(a) &g(a)
\end{matrix}\right|
=
\lambda t \ne t
\]
für $\lambda \ne 1$.
Eine solche Forderung führt also auf eine Menge von zulässigen Funktionen,
die nicht mehr ein Vektorraum ist.
Es bleibt daher nur die Möglichkeit, dass die Determinante $=0$ sein muss.

Die Determinante 
\eqref{buch:orthfkt:sturmliouville:det}
ist genau dann Null, wenn die Spalten der Matrix linear abhängig sind.
Insbesondere spannen Sie den gleiche Gerade in $\mathbb{R}^2$ auf.
Es gibt daher Vektoren
\[
\begin{pmatrix}
h_a\\
k_a
\end{pmatrix}
\qquad\text{und}\qquad
\begin{pmatrix}
h_b\\
k_b
\end{pmatrix}
\]
die auf den Vektoren
\[
\begin{pmatrix}
p(a)f'(a)\\
f(a)
\end{pmatrix}
\qquad\text{und}\qquad
\begin{pmatrix}
p(a)g'(a)\\
g(a)
\end{pmatrix}
\]
senkrecht stehen.
Dies ist gleichbedeutend mit der Randbedingung
\begin{align*}
k_af(a) + h_ap(a)f'(a)&=0
\\
k_bf(b) + h_bp(b)f'(b)&=0.
\end{align*}
Wir fassen die Resultate im folgenden Satz zusammen.

\begin{satz}
Seien $h_a, h_b, k_a, k_b\in \mathbb{R}$ gegeben.
Dann ist der Operator $L_0$ ist selbstadjungiert im Prähilbertraum
\begin{equation}
H(h_a,h_b,k_a,k_b)
=
\left\{
f \in C^1([a,b])
\;
\left|
\;
\begin{aligned}
&\int_a^b |f(x)|^2\,dx < \infty,
\int_a^b |p(x)| |f'(x)|^2\,dx < \infty,
\\
&
\begin{aligned}
k_af(a) + h_ap(a)f'(a) &= 0 \\
k_af(b) + h_bp(b)f'(b) &= 0 
\end{aligned}
\end{aligned}
\right.
\right\}
\label{buch:orthfkt:sturmliouvill:ph}
\end{equation}
mit dem Skalarprodukte
\[
\langle f,g\rangle = \int_a^b f(x)g(x)\,dx.
\]
\end{satz}

%
% Das verallgemeinerte Eigenwertproblem
%
\subsection{Das Eigenwertproblem}
Zu Zahlen $h_a$, $h_b$, $k_a$, $k_b$ ist der Operator $L_0$ im 
Prählibertraum~\eqref{buch:orthfkt:sturmliouvill:ph}
selbstadjungiert.
Wir erwarten, dass die Eigenvektoren dieses Differentialoperators eine
Hilbertbasis für die Analyse und Synthese von Funktionen in diesem
Prählibertraum liefern.
In diesem Abschnitt soll gezeigt werden, wie sich damit 
Hilbertbasen für erweiterte Differentialgleichungen 

%
% Das Eigenwertproblem für den Operator $L$
%
\subsubsection{Das Eigenwertproblem für den Operator $L$}
Wir betrachten jetzt den Operator
\[
L
=
\frac{d}{dx}p(x)\frac{d}{dx} + q(x)
=
L_0 + q(x).
\]
Es wurde bereits gezeigt, dass $q(x)$ immer selbstadjungiert ist,
und dass $L_0$ in einem Prähilbertraum der Form
Prählibertraum~\eqref{buch:orthfkt:sturmliouvill:ph}
selbstadjungiert wird.
Eine Funktion $y\in H(h_a,h_b,k_a,k_b)$ ist eine Eigenfunktion des
Operators $L$ zum Eigenwert $\lambda$, wenn sie die Differentialgleichung
zweiter Ordnung
\begin{align*}
\frac{d}{dx} p(x) y'(x) + q(x)y(x) = \lambda y(x)
\intertext{oder}
 p(x) y''(x) + p'(x)y'(x) + q(x)y(x) = \lambda y(x)
\end{align*}
erfüllt.
Für $H(h_a,h_b,k_a,k_b)$ kann eine Hilbertbasis aus Eigenfunktionen
gefunden werden.

%
% Das verallgemeinerte Eigenwertproblem für symmetrische Matrizen
%
\subsubsection{Das verallgemeinerte Eigenwertproblem für symmetrische Matrizen}

%
% Das verallgemeinerte Sturm-Liouville-Eigenwertproblem
%
\subsubsection{Das verallgemeinerte Sturm-Liouville-Eigenwertproblem}
Das verallgemeinerte Sturm-Liouville-Eigenwertproblem ist die Aufgabe,
eine Lösung $y(x)$ der Differentialgleichung
\[
\frac{d}{dx}p(x)\frac{d}{dx} y(x)
+ q(x)y(x)
=
\lambda w(x) y(x)
\]
zu finden.
Das Problem unterscheidet sich von dem vorher untersuchten Problem durch
den zusätzlichen Faktor $w(x)>0$, der auch Gewichtsfunktion heisst.
Das Eigenwertproblem auf die bereits studierte Situation zurückgeführt
werden, indem das Skalarprodukt mit der Gewichtsfunktion $w(x)$ verwendet
wird.

\begin{definition}
Gegeben ist $p(x)\in C^1([a,b])$, $q(x),w(x)\in C([a,b])$ und $w(x)>0$
in $(a,b)$.
Der {\em allgemeine Sturm-Liouville-Operator} ist
\[
L
=
\frac{1}{w(x)}
\biggl(
\frac{d}{dx}p(x)\frac{d}{dx}
+
q(x)
\biggr).
\]
Das {\em verallgemeinerte Sturm-Liouville-Eigenwertproblem} ist die Aufgabe,
eine Lösung $y(x)$ der Gleichung
\(
Ly(x) = \lambda y(x)
\)
zu finden.
\end{definition}

\begin{satz}
Der Operator $L$ ist selbstadjungiert im Prähilbertraum
\[
H(h_a,h_b,k_a,k_b,w)
=
\left\{
f \in C^1([a,b])
\;
\left|
\;
\begin{aligned}
&\int_a^b |f(x)|^2w(x)\,dx < \infty,
\int_a^b |p(x)| |f'(x)|^2w(x)\,dx < \infty,
\\
&
\begin{aligned}
k_af(a) + h_ap(a)f'(a) &= 0 \\
k_af(b) + h_bp(b)f'(b) &= 0 
\end{aligned}
\end{aligned}
\right.
\right\}
\]
mit dem Skalarprodukt $\langle \;\,,\;\rangle_w$ mit der Gewichtsfunktion
$w(x)$.
\end{satz}

\begin{proof}[Beweis]
Wir müssen nachrechnen, dass der Operator $L$ selbstadfjungiert ist.
Dazu seien $f,g\in H(h_a,h_b,k_a,k_b,w)$, wir berechnen die
Skalarprodukte
\begin{align*}
\langle Lf,g\rangle_w
&=
\int_a^b (Lf)(x)g(x)w(x)\,dx
=
\int_a^b
\biggl(
\frac{d}{dx}p(x)\frac{d}{dx}f(x) + 
q(x) f(x)
\biggr) g(x)\,dx
=
\langle L_0f,g\rangle + \langle qf,g\rangle
\\
\langle f,Lg\rangle_w
&=
\int_a^b f(x)(Lg)(x)w(x)\,dx
=
\int_a^b
f(x)
\biggl(
\frac{d}{dx}p(x)\frac{d}{dx}g(x) + 
q(x) g(x)
\biggr) \,dx
=
\langle f,L_0g\rangle + \langle f,qg\rangle
\end{align*}
Da der Operator $L_0$ und der Multiplikationsoperator mit $q(x)$ bezügich
des Standardskalarproduktes $\langle\;\,,\;\rangle$ selbstadjungiert sind,
sind die beiden Ausdrücke gleich.
Damit ist gezeigt, dass $L$ bezüglich $langle\;\,,\;\rangle_w$
selbstadjungiert ist.
\end{proof}
