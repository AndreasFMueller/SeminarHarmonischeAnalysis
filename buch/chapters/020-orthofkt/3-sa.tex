%
% 3-sa.tex -- selbstadjungierte Operatoren
%
% (c) 2022 Prof Dr Andreas Müller, OST Ostschweizer Fachhochschule
%
\section{Selbstadjungierte Operatoren
\label{buch:orthofkt:section:sa}}
\kopfrechts{Selbstadjungierte Operatoren}
In der linearen Algebra lernt man, dass die Eigenvektoren
symmetrischer Matrizen zu verschiedenen Eigenwerten orthogonal
sind.
Abschnitt~\ref{buch:trigo:subsection:symmetrie}
zeigt, dass die trigonometrischen Funktionen und die
komplexen Exponentialfunktionen als Eigenfunktionen eines
Operators auftreten können.
In diesem Abschnitt soll gezeigt werden, dass diese Situation
keine Ausnahme ist und zu anderen orthogonalen Funktionenfamilien
führen kann, wenn man die Begriffe der linearen Algebra auf
die Situation des Hilbert-Raumes ausweiten kann.


%
% Der adjungierte Operator
%
\subsection{Der adjungierte Operator}
Sei $A:H\to H$ eine lineare Abbildung oder {\em Operator} auf dem
Hilbert-Raum $H$.
Damit die Abbildung stetig ist, muss die Norm des Operators
\[
\|A\|
=
\sup_{x\in H\setminus\{0\}} \frac{\|Hx\|}{\|x\|}
<
\infty
\]
sein.
Sei jetzt $x\in H$ ein Vektor.
Die Funktion
\[
l
\colon
y\mapsto \langle Ay,x\rangle
\]
ist linear und stetig, die Norm ist
\[
|l(y)|
=
|\langle x, Ay\rangle|
\le
\|x\|
\cdot
\|Ay\|
\le
\|x\| \cdot \|A\| \cdot \|y\|
.
\]
Nach dem Darstellungssatz von Riesz folgt, dass es einen Vektor 
$A^*x$ gibt derart, dass
\[
l(y) = \langle A^*x,y\rangle.
\]
Die Abbildung $x\mapsto A^*x$ ist wieder ein linearer und beschränkter
Operator.
Die Linearität folgt aus der Linearität des Skalarproduktes, die Norm
von $A^*$ ist
\[
\|A^*x\|^2
=
\]

\begin{definition}
Der Operator $A^*$ heisst der zu $A$ {\em adjungierte} Operator.
Ein Operator $A$ heisst selbstadjungiert, wenn $A^*=A$ oder
$\langle Ax,y\rangle = \langle x,Ay\rangle$ für alle $x,y\in H$.
\end{definition}


%
% Eigenwerte von selbstadjungierten Operatoren
%
\subsection{Eigenwerte von selbstadjungierten Operatoren}
Die Eigenschaften der Eigenvektoren symmetrischer Matrizen lassen
sich jetzt ohne Änderung der Beweise übertragen.

\begin{satz}
Die Eigenwerte eines selbstadjungierten Operators sind reell.
\end{satz}

\begin{proof}[Beweis]
Sei $A$ ein selbstadjungierter Operator auf dem Hilbert-Raum und
$x$ ein Eigenvektor zum Eigenwert $\lambda$, also $Ax=\lambda x$
und damit
\begin{equation}
\langle x,Ax\rangle
=
\langle x,\lambda x\rangle
=
\lambda \langle x,x\rangle
\label{buch:orthofkt:sa:reell1}
\end{equation}
Andererseits gilt
\begin{equation}
\langle x,Ax\rangle
=
\langle A^*x,x\rangle
=
\langle Ax,x\rangle
=
\langle \lambda x,x\rangle
=
\overline{\lambda}
\langle x,x\rangle.
\label{buch:orthofkt:sa:reell2}
\end{equation}
Zusammen mit~\eqref{buch:orthofkt:sa:reell1} folgt jetzt
\[
\lambda \langle x,x\rangle = \overline{\lambda} \langle x,x\rangle.
\]
Da $x\ne 0$ ist $0\ne \|x\|^2 =\langle x,x\rangle$, darf man durch
$\|x\|^2$ teilen und erhält 
\[
\lambda = \overline{\lambda},
\]
$\lambda$ ist reell.
\end{proof}

\begin{satz}
Eigenvektoren eines selbstadjungierten Operators zu verschiedenen Eigenwerten
sind orthogonal.
\end{satz}

\begin{proof}[Beweis]
Seien $x_1$ und $x_2$ Eigenvektoren des selbstadjungierten Operators $A$
zu den Eigenwerten $\lambda_1\ne \lambda_2$.
Da $A$ selbstadjungiert ist, gilt
$\langle Ax_1,x_2\rangle=\langle x_1,Ax_2\rangle$.
Wir berechnen diese beiden Skalarprodukte mit Hilfe der Eigenvektoreigenschaft:
\[
\begin{array}{cclcl}
\langle Ax_1,x_2\rangle
&=&
\langle \lambda_1x_1,x_2\rangle
&=&
\lambda_1 \langle x_1,x_2\rangle
\\
          \|&&&&
\\
\langle x_1,Ax_2\rangle
&=&
\langle x_1,\lambda_2x_2\rangle
&=&
\lambda_2 \langle x_1,x_2\rangle
\end{array}
\]
Die Differenz dieser beiden Terme ist
\[
0
=
(\lambda_1-\lambda_2)\langle x_1,x_2\rangle.
\]
Da $\lambda_1\ne \lambda_2$ ist der Klammerausdruck nicht $0$, daher muss
das Skalarprodukt $\langle x_1,x_2\rangle=0$ sein, die beiden Eigenvektoren
sind orthogonal.
\end{proof}

\begin{beispiel}
Der Vektorraum $H=\mathbb{R}^n$ mit dem Standardskalarprodukt ist ein
endlichdimensionaler Hilbertraum.
Ein Operator $A\colon H\to H$ wird unter Verwendung einer Basis
$\{b_1,\dots,b_n\}$ durch die Matrix mit den Matrix-Elementen
\[
a_{ij}
=
\langle b_i,Ab_j\rangle
=
\langle Ab_i,b_j\rangle
=
\langle b_j,Ab_i\rangle
=
a_{ji}
\]
beschrieben, die Matrix ist symmetrisch.
\end{beispiel}

\begin{beispiel}
Für den endlichdimensionalen komplexen Vektorraum $H=\mathbb{C}^n$ mit dem
Standardskalarprodukt
\[
\langle x,y\rangle
=
\sum_{i=1}^n  \overline{x}_i y_i
\]
ist ein Hilbertraum.
Ein selbstadjungierter Operator hat in einer Basis $b_1,\dots,b_n$
die Matrixelemente
\[
a_{ij}
=
\langle b_i,Ab_j\rangle
=
\langle Ab_i,b_j\rangle
=
\overline{
\langle b_j,Ab_i\rangle
}
=
\overline{a_{ji}}.
\]
Die Matrix von $A$ ist also nicht mehr symmetrisch, sondern konjugiert
symmetrisch.
\end{beispiel}

\begin{definition}
Eine Matrix $A$ heisst hermitesch, wenn
$\transpose{A}=\overline{A}$.
\end{definition}

\begin{beispiel}
Wir betrachten den Prähilbertraum
\[
H
=
\{f\in C^{\infty}(\mathbb{R})
\mid
\text{$f$ ist $2\pi$-periodisch}
\}
\]
der $2\pi$-periodischen, beliebig oft stetig differenzierbaren
stetig differezierbaren Funktionen auf $\mathbb{R}$ mit dem
Skalarprodukt
\[
\langle f,g\rangle
=
\int_{-\pi}^\pi
f(x)g(x)\,dx
\]
und den Operator
\[
D^2
\colon
H_0 \to H_0
:
f\mapsto f''.
\]
Wir berechnene die Skalarprodukte
\begin{align}
\langle D^2f,g\rangle
&=
\int_{-\pi}^\pi f''(x)g(y)\,dx
=
\biggl[f'(x)g(x)\biggr]_{-\pi}^\pi
-
\int_{-\pi}^\pi f'(x)g'(x)\,dx,
\label{buch:orthfunkt:sa:eqn:D2fg}
\\
\langle f,D^2g\rangle
&=
\int_{-\pi}^\pi f(x)g''(y)\,dx
=
\biggl[f(x)g'(x)\biggr]_{-\pi}^\pi
-
\int_{-\pi}^\pi f'(x)g'(x)\,dx.
\label{buch:orthfunkt:sa:eqn:fD2g}
\end{align}
Da $f(x)$ und $g(x)$ $2\pi$-periodisch sind, ist
$f'(\pi)g(\pi)=f'(-\pi)g(-\pi)$, der erste Term
in~\eqref{buch:orthfunkt:sa:eqn:D2fg} verschwindet daher.
Dasselbe passiert auch in \eqref{buch:orthfunkt:sa:eqn:fD2g}.
Es bleibt
\[
\langle D^2f,g\rangle
=
-
\int_{-\pi}^\pi f'(x)g'(x)\,dx.
=
\langle f,D^2g\rangle,
\]
der Operator $D^2$ ist daher selbstadjungiert.
\end{beispiel}

Das letzte Beispiel zeigt zusammen mit den Erkenntnissen aus dem
Abschnitt~\ref{buch:trigo:subsection:symmetrie}, dass die
Eigenfunktionen eines selbstadjungierten Operators gute Kandidaten
für eine harmonische Analysis sind, die sich besonderes gut
zur Lösung von Problemen eignen, die mit dem Operator $D^2$
zusammenhängen.




