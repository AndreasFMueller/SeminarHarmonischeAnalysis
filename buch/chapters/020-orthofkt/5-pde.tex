%
% 5-pde.tex -- PDE
%
% (c) 2022 Prof Dr Andreas Müller, OST Ostschweizer Fachhochschule
%
\section{Partielle Differentialoperatoren
\label{buch:orthofkt:section:pde}}
\kopfrechts{Partielle Differentialoperatoren}
Selbstadjungierte partielle Differentialoperatoren auf einem Gebiet 
$\Omega$ führen auf natürlich Art und Weise zu einer verallgemeinerten
Theorie der harmonischen Analysis, die auch zur Lösung der
partiellen Differentialgleichung verwendet werden kann.
Die Vorgehensweise wird zunächst in
Abschnitt~\ref{buch:orthfkt:pde:subsection:pdeharm}
allgemein skizziert und anschliessend für ein paar interessante
Gebiete und Differentialoperatoren durchgeführt.

%
% Partielle Differentialgleichungen udn harmonische Analysis
%
\subsection{Partielle Differentialgleichung und harmonische Analysis
\label{buch:orthfkt:pde:subsection:pdeharm}}
In diesem Abschnitt soll illustriert werden, wie sich aus Aufgabenstellungen
aus der Theorie der partiellen Differentialgleichungen 

%
% Partielle Differentialoperatoren
%
\subsubsection{Lineare partielle Differentialoperatoren}
Ein linearer partieller Differentialoperator ist eine lineare
Funktion der partiellen Ableitungen einer Funktion, die auf einem
Gebiet definiert ist.

\begin{definition}
Ein Gebiet $\Omega$ ist eine offene Teilmenge von $\mathbb{R}^n$.
Ein partieller Differentialoperator der Ordnung $r$ ist eine Linearkombination
\[
L
=
\sum_{k_1+\dots+k_n\le r}
a_{k_1\dots k_n}(x)
\frac{\partial^{k_1+\dots+k_n}}{\partial x_1^{k_1}\dots\partial x_n^{k_n}}.
\]
\end{definition}

\begin{beispiel}
Der Laplace-Operator auf einem beliebigen Gebiet $\Omega\subset\mathbb{R}^n$
ist definiert als
\[
\Delta
=
\frac{\partial^2}{\partial x_1^2}
+
\dots
+
\frac{\partial^2}{\partial x_n^2}
\]
ist ein partieller Differentialoperator zweiter Ordnung.
\end{beispiel}

%
% Skalarprodukt
%
\subsubsection{Skalarprodukt}
Da $\Omega$ eine offene Teilmenge von $\mathbb{R}^n$ ist, lässt sich
sofort ein Skalarprodukt von Funktionen auf $\Omega$ definieren.

\begin{definition}
Ist $\Omega\subset \mathbb{R}^n$ ein Gebiet in $\mathbb{R}^n$, dann ist
\begin{equation}
\langle u,v\rangle_\Omega
=
\int_{\Omega}  \overline{u(x)} v(x)\,dx
\label{buch:orthofkt:pde:eqn:skalarprodukt}
\end{equation}
ein Skalarprodukt für Funktionen auf $\Omega$.
\end{definition}

Mit einer Gewichtsfunktion $w\colon \Omega \to \mathbb{R}^+$ lässt sich
das Skalarprodukt noch etwas verallgemeinern, es wird zu
\[
\langle u,v\rangle_{\Omega,w}
=
\int_{\Omega} \overline{u(x)}v(x)\,w(x)\,dx.
\]
Diese Erweiterung ist auch deshalb nötig, weil bei einem Wechsel
der Koordinaten nach $x_i=x_i(x_1',\dots,x_n')$, das Integral
\eqref{buch:orthofkt:pde:eqn:skalarprodukt}
den zusätzlichen Faktor der Funktionaldeterminante 
\[
\langle u,v\rangle_{\Omega}
=
\int_{\Omega} \overline{u(x)} v(x)\,dx 
=
\int_{\Omega'} \overline{u(x')} v(x')
\det\frac{\partial(x_1,\dots,x_n)}{\partial(x_1',\dots,x_n')}
\,dx'
\]
erhält, wobei $\Omega'$ das Gebiet in $\mathbb{R}^n$ ist, welches
durch die Koordinatenfunktionen $x_i(x_1',\dots,x_n')$ bijektiv auf
$\Omega$ abgebildet wird.


%
% Randbedingungen
%
\subsubsection{Randbedingungen}
Wie beim Sturm-Liouville-Problem entsteht ein interessantes 
erst dadurch, dass zusätzlich Randbedingungen formuliert werden.
Der Operator $L=d^2/dx^2$ auf dem Intervall $(a,b)\subset\mathbb{R}$
ist zum Beispiel nur selbstadjungiert, wenn man ihn auf Funktionen
einschränkt, die gewisse Randbedingungen erfüllen.
$L$ ist der Sturm-Liouville-Operator von Abschnitt
\ref{buch:orthofkt:sturmliouville:subsection:operator}
mit $p(x)=1$ und $q(x)=0$.
In Abschnitt~\ref{buch:orthofkt:sturmliouville:subsection:operator}
wurde gezeigt, dass der Operator $L$ nur selbstadjungiert bezüglich
des $L^2$-Skalarproduktes wird, wenn Randbedinungen der Form
\begin{equation}
h_a u(a) + k_a u'(a) =  0
\qquad\text{und}\qquad
h_b u(b) + k_a u'(b) =  0
\end{equation}
gefordert werden.
Die Wahl $h_a=h_b=1$ und $k_a=k_b=0$ hat zum Beispiel zur Folge,
dass die Funktionen die Randwerte $u(a)=u(b)=0$ haben müssen.
Die Wahl $h_a=h_b=0$ und $k_a=k_b=1$ hat dagegen zur Folge,
dass die Ableitungen der Funktionen in den Endpunkten $=0$ sein
müssen.
Für partielle Differentialoperatoren und die 

\begin{definition}[Dirichlet-Randbedingungen]
Sei $\Omega\subset\mathbb{R}^n$ ein Gebiet mit Rand $\partial\Omega$
und $f\colon \partial\Omega\mathbb R$ eine Funktion auf dem Rand.
Man sagt, eine Funktion $u\colon\overline{\Omega}\to\mathbb{R}$ erfüllt 
{\em Dirichlet-Randbedingungen}, wenn
\index{Randbedingung!Dirichlet-}%
\index{Dirichlet-Randbedingung}%
\(
u(x) = f(x)
\)
für $x\in \partial\Omega$ ist.
\end{definition}

Dirichlet-Randbedinungen entsprechen also der Vorgabe von Werten
oder dem Fall $h_a=h_b=0$ und $k_a=k_b=0$.

\begin{definition}
Sei $\Omega\in\mathbb{R}^n$ ein Gebiet mit glattem Rand und
$n(x)$ der Einheitsnormalenvektor im Punkt $x\in\partial\Omega$
auf dem Rand.
Die {\em Normalableitung} einer Funktion
\index{Normalableitung}%
$u\colon\overline{\Omega}\to \mathbb{R}$ auf dem Rand ist die
Richtungsableitung
\[
\frac{\partial u}{\partial n}
=
\frac{d}{dt} u(x+tn(x))\bigg|_{t=0}.
\]
Man sagt, die Funktion $u$ erfüllt eine Neumann-Randbedingung, wenn
\index{Neumann-Randbedingung}%
\index{Randbedingung!Neumann-}%
\[
\frac{\partial u}{\partial n}(x) = g(x)
\]
erfüllt ist.
\end{definition}

Neumann-Randbedingungen entsprechen also der Vorgabe der Ableitungen
oder dem Fall $h_a=h_b=0$ und $k_a=k_b=0$.
Die Dirichlet- und Neumann-Randbedingungen können auch in sogenannte
gemischte Randbedingungen zusammengefasst werden, indem 
\index{Randbedingung!gemischt}%
\index{gemischte Randbedingung}%
\begin{equation}
h(x) u(x) + k(x) \frac{\partial u}{\partial n}(x) = g(x)
\label{buch:orthofunkt:pde:eqn:gemischterandbedingung}
\end{equation}
für  $x \in\partial \Omega$ für auf dem Rand definierte Funktion
$h(x)$ und $k(x)$ gefordert wird.

%
% Homogene und inhomogene Differentialgleichungen und Randbedingungen
%
\subsubsection{Homogene und inhomogene Differentialgleichungen und
Randbedingungen}
Eine partielle Differentialgleichung ist eine Differentialgleichung
der Form
\begin{equation}
L u = v,
\label{buch:orthofunkt:pde:eqn:dgl}
\end{equation}
wobei $L$ ein Differentialoperator auf dem Gebiet $\Omega$ ist und
$h$ eine Funktion auf $\Omega$.
Eine Lösung der partiellen Differentialgleichung ist eine Funktion
$u\colon\overline{\Omega}\to \mathbb{R}$ derart, dass die Gleichung
\eqref{buch:orthofunkt:pde:eqn:dgl} in $\Omega$ gilt.
Ohne zusätzliche Randbedingungen wird die Lösung der Gleichung
im Allgemeinen nicht eindeutig sein.
Sind $u_1$ und $u_2$ Lösungen der Differentialgleichung
\eqref{buch:orthofunkt:pde:eqn:dgl}, dann ist
\[
L(u_1-u_2)
=
Lu_1-Lu_2
=
v-v
=0,
\]
die Differenz $u_1-u_2$ ist also eine Lösung der zugehörigen
{\em homogenen} Differentialgleichung.
\index{homogen}%
Eine Lösung $u_p$ von 
\eqref{buch:orthofunkt:pde:eqn:dgl} heisst auch eine
{\em partikuläre Lösung}.

Die beiden Lösungen können sich durch die Werte der Funktionen
oder der Normalableitungen auf dem Rand unterscheiden.
Es ist also im allgemeinen nötig, eine Randbedingung vorzugeben.
Eine partikuläre Lösung $u_p$ erfüllt die Randbedingung möglicherweise
nicht.
Eine Lösung $u$, die auch die Randbedingung erfüllt, unterschiedet sich
von $u_p$ um 
$\tilde{u} = u-u_p$ und erfüllt die Differentialgleichung
\[
L\tilde{u}
=
Lu - Lu_p
=
f - f
=
0
\]
und die Randbedingung
\begin{align}
h(x)\tilde{u}(x) + k(x)\frac{\partial \tilde{u}}{\partial x} (x)
&=
h(x)u(x) + k(x)\frac{\partial u}{\partial n}(x)
-
h(x)u_p(x) - k(x)\frac{\partial u_p}{\partial n}(x)
\notag
\\
&=
g(x)
-
h(x)u_p(x) - k(x)\frac{\partial u_p}{\partial n}(x).
\label{buch:orthofunkt:pde:eqn:utilderand}
\end{align}
Um die Lösung $\tilde{u}$ zu finden, reicht es daher, eine Lösung der
homogenen Differentialgleichung $L\tilde{u}=0$ zu finden, die die
Randbedingungen \eqref{buch:orthofunkt:pde:eqn:utilderand}
erfüllt.
Das Problem ist damit reduziert auf eine homogene Differentialgleichung
mit möglicherweise inhomogenen Randbedingungen.

Durch Vorgabe einer Randbedingung werden die möglichen Lösungen
eingeschränkt, es ist aber immer noch möglich, dass die Lösung
nicht eindeutig ist.
Seien also $u_1$ und $u_2$ Lösungen der Differentialgleichung mit der 
Randbedingung~\eqref{buch:orthofunkt:pde:eqn:gemischterandbedingung},
dann ist die Differenz $u=u_1-u_2$ eine Lösung der homogenen
Differentialgleichung $Lu=0$ und die Randbedingungen sind
\[
h(x) u(x) + k(x) \frac{\partial u}{\partial n}(x)
=
h(x) u_1(x) + k(x) \frac{\partial u_1}{\partial n}(x)
-
h(x) u_2(x) + k(x) \frac{\partial u_2}{\partial n}(x)
=
g(x)-g(x)=0.
\]
Die Differenz erfüllt also die homogene Differentialgleichung
{\em und} zusätzlich homogene Randbedingungen.
Die Frage nach der Eindeutigkeit der Lösung der partiellen
Differentialgleichung mit den gegebenen Randbedingungen wird also
beantwortet von der Lösungsmenge der homogenen Differentialgleichung
mit homogenen Randbedingungen.

%
% Harmonische Analysis
%
\subsubsection{Harmonische Analysis}
Die im Fall des Eigenwertproblems für den Sturm-Liouville-Operator
beschriebene Situation tritt sehr häufig auch bei partiellen
Differentialgleichungen auf.
Wir erwarten daher, dass zu einem partiellen Differentialoperator $L$
auf einem Gebiet $\Omega$ jeweils eine orthogonale Funktionenfamilie
von Eigenfunktionen  $u_n(x)$ des Operators mit Eigenwerten $\lambda_n$
gehört, die zur Synthese beliebiger Funktionen auf $\Omega$ verwendet
werden kann.

Für sogenannte elliptische partielle Differentialoperatoren zweiter
Ordnung lassen sich dann die Lösungsfunktionen der Gleichungen
\begin{align*}
Lu &= \varrho                            &&\text{Poisson-Gleichung}      \\
\intertext{oder für den Fall, dass $u(t,x)$ zusätzlich von der Zeit abhängt}
\frac{\partial u}{\partial t}     &= Lu  &&\text{Wärmeleitungsgleichung} \\
\frac{\partial^2 u}{\partial t^2} &= LU  &&\text{Wellengleichung}
\end{align*}
als Reihe 
geeigneten Randbedingungen lassen sich die Lösungsfunktionen dann in 
der Form
\begin{equation}
u(x)
=
\sum_{n=0}^\infty a_n u_n(x)
\label{buch:orthofkt:eqn:unreihe}
\end{equation}
entwickeln, wobei die Koeffizienten $a_n$ 
im Falle der Wärmeleitungslgeichung und der Wellengleichung
Funktionen der Zeit sind.
Eingesetzt in die Differentialgleichung ergeben sich dann für die
Koeffizienten $a_n$ die Gleichungenn
\begin{align}
\lambda_n a_n  &=  b_n      &&\text{Poisson-Gleichung} 
\notag
\\
\dot{a}_n &= \lambda_n a_n  &&\text{Wärmeleitungsgleichung}
\label{buch:orthfkt:eqn:waermeleitung:ode}
\\
\ddot{a}_n &= \lambda_n a_n &&\text{Wellengleichung,}
\label{buch:orthfkt:eqn:wellen:ode}
\end{align}
wobei die $b_n$ die Koeffizienten der Entwicklung von $\varrho$ nach
den Funktionen $u_n(x)$ ist.
Im Falle der Wärmeleitungsgleichung wird das Problem also reduziert
auf die Lösung einer gewöhnlichen Differentialgleichung.

Zusätzlich zu den Randbedingungen auf dem Rand des Gebietes $\Omega$,
auf dem der Differentialoperator $L$ definiert ist, braucht die
Wärmeleitungsgleichung auch noch eine Randbedingung der Form
\[
u(0,x) = f(x)
\]
hinzu, die durch Anwendung der Entwicklung~\eqref{buch:orthofkt:eqn:unreihe}
zur Anfangsbedinung 
\[
a_n(0) = f_n
\]
für \eqref{buch:orthfkt:eqn:waermeleitung:ode} wird, wobei $f_n$ die
Koeffizienten der Entwicklung der Funktion $f(x)$ nach den Eigenfunktionen
$u_n$ sind.
Für die Wellengleichung \eqref{buch:orthfkt:eqn:wellen:ode} wird 
auch noch eine Neumann-Randbedingung der Form
\[
\frac{\partial u}{\partial t}(0,x) = g(x)
\]
nötig, die zu einer Anfangsbedingung für die Ableitungen
\[
\dot{a}_n(0) = g_n
\]
für die gewöhnliche Differentialgleichung \eqref{buch:orthfkt:eqn:wellen:ode}
wird.

Die Bestimmung der Entwicklungskoeffizienten erfolgt jeweils mit Hilfe
der Skalarprodukte
\[
b_n = \langle u_n, \varrho\rangle,
\qquad
f_n = \langle u_n, f\rangle,
\qquad
g_n = \langle u_n, g\rangle
\]
gewonnen ist.

%
% Fourier-Theorie
%
\subsection{Fourier-Theorie
\label{buch:orthofkt:subsection:fourier-theorie}}
Die Funktionen der Fourier-Theorie entstehen als Eigenfunktionen des
Differentialoperators
\begin{equation}
D^2 = \frac{d^2}{dx^2}
\end{equation}
der zweiten Ableitung auf dem Intervall $[0,l]$.
Das Skalarprodukt ist das übliche $L^2$-Skalarprodukt.

\subsubsection{$D^2$ ist selbstadjungiert}
Seien $f$ und $g$ zwei zweimal stetig differenzierbare Funktionen,
wir berechnen das Skalarprodukt $\langle D^2f,g\rangle$ und
$\langle f,D^2g\rangle$:
\begin{align*}
\langle D^2f,g\rangle
&=
\int_0^{l}
f''(x) g(x)\,dx
=
\left[ f'(x) g(x) \right]_0^{l}
-
\int_0^{l} f'(x)g'(x)\,dx
\\
\langle f,D^2g\rangle
&=
\int_0^{l} f(x)g''(x)\,dx
=
\left[ f(x) g'(x) \right]_0^{l}
-
\int_0^{l} f'(x)g'(x)\,dx
\end{align*}
Die Integrale auf der rechten Seite stimmen überein.
Der Operator $D^2$ ist also genau dann selbstadjungiert, wenn die anderen
Terme auch übereinstimmen, wenn also
\[
\left[ f'(x) g(x) \right]_0^{l}
=
\left[ f(x) g'(x) \right]_0^{l}.
\]
Ausgeschrieben ist dies
\begin{align}
f'(l)g(l) - f'(0)g(0)
&=
f(l)g'(l) - f(0)g'(0)
\notag
\\
\Rightarrow\qquad
f'(l)g(l) - f(l)g'(l)
&=
f'(0)g(0)-f(0)g'(0)
\notag
\intertext{oder mit Determinanten geschrieben}
\biggl|\begin{matrix}
f'(l) & g'(l) \\
f(l)  & g(l)
\end{matrix}\biggr|
&=
\biggl|\begin{matrix}
f'(0) & g'(0) \\
f(0)  & g(0)
\end{matrix}\biggr|
\label{buch:orthofkt:pde:eqn:detfg}
\end{align}
Bei der Diskussion des Sturm-Liouville-Problems haben wir die beiden
Intervallenden unabhängig voneinander betrachtet und zwei unabhängige
Randbedingungen gefunden, welche erzwingen, dass 
\eqref{buch:orthofkt:pde:eqn:detfg} erfüllt ist.
Es gibt aber noch eine weitere Lösung, nämlich zu verlangen, dass
die Funktionswerte und Ableitung für $x=0$ und $x=l$ übereinstimmen.
Eine Funktion mit $f(0)=f(l)$ und $f'(0)=f'(l)$ kann periodisch
auf ganz $\mathbb{R}$ ausgedehnt werden.
Wegen $f(0)=f(l)$ ist die ausgedehnte Funktion stetig auf ganz $\mathbb{R}$
und wegen $f'(0)=f'(l)$ ist sogar die Ableitung stetig.

%
% Eigenfunktionen
%
\subsubsection{Eigenfunktionen}
Eigenfunktionen des Differentialoperators $D^2$ sind Lösungen der
Differentialgleichung
\[
y'' = \lambda y.
\]
Die übliche Lösungsmethode liefert Lösungen
\[
y(x) = e^{\pm\!\sqrt{\lambda} x} \quad
\text{ für $\lambda > 0$ und }
\qquad
y(x)
=
\begin{cases}
\cos\sqrt{-\lambda} x \\
\sin\sqrt{-\lambda} x
\end{cases}
\quad\text{für $\lambda < 0$.}
\]
Natürlich sind auch Linearkombinationen wieder Eigenfunktionen.

Die Funktionen erfüllen aber die Randbedingungen nicht.
Zum Beispiel ist es unmöglich, die periodischen Randbedingungen
mit den Exponentialfunktionen zu erfüllen, denn dazu müsste
$1=e^{0}=e^{\pm\!\sqrt{\lambda}l}$ sein, was nur für $\lambda=0$
gilt.

%
% Randbedingungen für \lambda < 0
%
\subsubsection{Randbedingungen für $\lambda<0$}
Je nach gewählten Randbedingungen sind die Eigenfunktionen des Operators
$D^2$, die auch die Randbedingungen erfüllen, verschieden.
Die allgemeine Theorie garantiert aber dass die Eigenfunktionen die
Form einer Linearkombination
\begin{equation}
y(x)
=
A\cos \sqrt{-\lambda}x
+
B\sin \sqrt{-\lambda}x
\label{buch:orthofkt:pde:periodisch}
\end{equation}
haben und untereinander orthogonal sind.
Wir untersuchen die Bedingungen an $A$ und $B$, die sich aus verschiedenen
Randbedingungen ergeben.

Homogene Dirichlet-Randbedingungen werden erfüllt, wenn $y(0)=y(l)=0$ ist.
In Folgenden bestimmen wir $A$ und $B$ für $l=\pi$.
Wegen
\[
y(0)
=
A\cos \sqrt{-\lambda}0
+
B\sin \sqrt{-\lambda}0
=
A
=
0
\]
folgt, dass eine Eigenfunktion eine Sinus-Funktion sein muss.
Die Randbedingung bei $x=\pi$ wird nur erfüllt, wenn
\[
y(\pi)
=
B\sin\sqrt{-\lambda}\pi
=
0
\qquad
\pi\sqrt{-\lambda}
=
k\pi,\; k\in\mathbb{Z},
\]
weil die Nullstellen der Sinusfunktion Vielfache von $\pi$ sind.
Es folgt, dass $\lambda = -k^2$ und dass die Funktionen $\sin kx$
die Eigenfunktionen dazu sind.

Homogene Neumann-Randbedingungen werden erfüllt, wenn $y'(0)=y'(l)=0$,
wir untersuchen wieder den Fall $l=\pi$.
Aus
\[
y'(x)
=
-A\sqrt{-\lambda}\sin\sqrt{-\lambda} x 
+
B\sqrt{-\lambda}\cos\sqrt{-\lambda} x 
\]
folgt für $x=0$, dass
\[
y'(0)
=
B\sqrt{-\lambda}
\qquad\Rightarrow\qquad B=0.
\]
Am rechten Rand $x=\pi$ folgt dann
\[
y'(\pi)
=
-A\sqrt{-\lambda}\sin\sqrt{-\lambda}\pi
=
0
\qquad\Rightarrow\qquad
\sqrt{-\lambda}\pi = k\pi,\; k\in\mathbb{Z}
\qquad\Rightarrow\qquad
\lambda=-k^2,\;k\in\mathbb{Z}.
\]
In diesem Fall sind also die Kosinusfunktionen die Eigenfunktionen.

Als Beispiel für eine gemischte Randbedingung betrachten wir den
Fall $y(0)=0$ und $y'(\pi)=0$.
Zunächst folgt aus der Diskussion der Dirichlet-Randbedingung, dass
$y(x)=A\sin\sqrt{-\lambda}x$ sein muss.
Am rechten Rand folgt dann
\[
y'(\pi)
=
A\sqrt{-\lambda}\cos\sqrt{-\lambda}\pi
=
0
\quad\Rightarrow\quad
\sqrt{-\lambda}\pi = (k+{\textstyle\frac12})\pi,\;k\in\mathbb{Z}
\]
und $\lambda = -(k+{\textstyle\frac12})^2$
Die Eigenfunktionen sind wieder Sinusfunktionen, aber mit anderen
Frequenzen.

Die Beispiele zeigen, dass die Eigenfunktionen entweder Sinus-
oder Kosinusfunktionen sind.
Mit diesen Funktionen lassen sich beliebige Funktionen approximieren,
die die Randbedingungen erfüllen.

%
% Periodische Randbedingungen für \lambda < 0
%
\subsubsection{Periodische Randbedingungen für $\lambda < 0$}
Beliebige periodische Funktionen auf dem Intervall $[0,2\pi]$ lassen sich
mit den im vorangegangenen Abschnitt gefundenen Funktionen nicht
approximieren, dazu müssen periodische Randbedingungen verlangt werden.
Die Funktion $y(x)$ von
\eqref{buch:orthofkt:pde:periodisch}
erfüllt periodische Randbedingungen für die Werte und Ableitungen, wenn
\begin{align*}
y(0) &= y(2\pi) 
&&\Rightarrow&
A &= A\cos\sqrt{-\lambda}2\pi + B\sin\sqrt{-\lambda}2\pi
\\
y'(0)&=y'(2\pi)
&&\Rightarrow&
\sqrt{-\lambda}B
&=
-
A\sqrt{-\lambda}\sin\sqrt{-\lambda}2\pi
+
B\sqrt{-\lambda}\cos\sqrt{-\lambda}2\pi.
\end{align*}
Die Koeffizienten $A$ und $B$ sind Lösungen des linearen Gleichungssystems
\[
\renewcommand{\arraycolsep}{2pt}
\begin{array}{rcrcr}
\cos2\pi\sqrt{-\lambda} \cdot A
&+&
\sin2\pi\sqrt{-\lambda} \cdot B
&=& A
\\
-\sin2\pi\sqrt{-\lambda} \cdot A
&+&
\cos2\pi\sqrt{-\lambda} \cdot B
&=& B
\end{array}
\Leftrightarrow
\begin{array}{rcrcl}
(\cos2\pi\sqrt{-\lambda}-1) \cdot A
&+&
\sin2\pi\sqrt{-\lambda} \cdot B
&=&
0
\\
-\sin2\pi\sqrt{-\lambda} \cdot A
&+&
(\cos2\pi\sqrt{-\lambda}-1) \cdot B
&=& 0.
\end{array}
\]
Als homogenes lineares Gleichungssystems kann es nur dann eine nichttriviale
Lösung geben, wenn die Determinante der Koeffizientenmatrix verschwindet,
wenn also
\begin{align*}
0
&=
\biggl|\begin{matrix}
\cos 2\pi\sqrt{-\lambda}-1 & \sin 2\pi\sqrt{-\lambda}   \\
-\sin 2\pi\sqrt{-\lambda}  & \cos 2\pi\sqrt{-\lambda}-1
\end{matrix}\biggr|
\\
&=
\cos^2 2\pi\sqrt{-\lambda} -2\cos2\pi\sqrt{-\lambda}+1
+
\sin^22\pi\sqrt{-\lambda}
\\
&=
2(1-\cos2\pi\sqrt{-\lambda})
\qquad\Rightarrow\qquad
2\pi\sqrt{-\lambda} = 2k\pi,\;k\in\mathbb{Z},
\end{align*}
da die Maxima der Kosinusfunktion bei den Vielfachen von $2\pi$ liegen.
Somit sind die möglichen Eigenwerte $\lambda=-k^2$.

Die Matrix des Gleichungssystems zur Bestimmung der Koeffizienten $A$ und $B$
wird für $\lambda=-k^2$ zur Nullmatrix, somit sind beliebige Koeffizienten
$A$ und $B$ möglich.
Jede Linearkombination von $\cos kx$ und $\sin kx$ ist Eigenfunktion
des Operators $D^2$ mit Eigenwerte $-k^2$ mit periodischen Randbedingungen.
Dies sind die Basisfunktionen, die für die Fourier-Theorie verwendet
werden.

%
% Randbedingungen für \lambda > 0
%
\subsubsection{Randbedingungen für $\lambda > 0$}
Im Fall $\lambda >0$ muss eine Eigenfunktion die Form
\[
y(x)
=
Ae^{\sqrt{\lambda}x}
+
Be^{-\sqrt{\lambda}x}
\quad\text{mit}\quad
y'(x)
=
A\sqrt{\lambda}e^{\sqrt{\lambda}x}
-
B\sqrt{\lambda}e^{-\sqrt{\lambda}x}
\]
haben.
Die Diskussion der Eigenfunktionen wird etwas einfacher auf einem
symmetrischen Intervall $[-l,l]$.

Homogene Dirichlet-Randbedingungen werden erfüllt, wenn
\begin{align*}
0&=y(\phantom{-}l) = Ae^{l\sqrt{\lambda}} + Be^{-l\sqrt{\lambda}}
\\
0&=y(         - l) = Ae^{-l\sqrt{\lambda}} + Be^{l\sqrt{\lambda}}
\end{align*}
Dies ist ein homogenes lineares Gleichungssystem, es hat nur dann eine
nichttriviale Lösung, wenn die Determinante nicht verschwindet, wenn also
\[
0
=
\biggl|
\begin{matrix}
e^{l\sqrt{\lambda}}  & e^{-l\sqrt{\lambda}} \\
e^{-l\sqrt{\lambda}} & e^{l\sqrt{\lambda}}
\end{matrix}
\biggr|
=
e^{2l\sqrt{\lambda}} - e^{-2l\sqrt{\lambda}}
0
e^{2l\sqrt{\lambda}} ( 1 - e^{-4l\sqrt{\lambda}}).
\]
Die Exponentialfunktion hat keine reellen Nullstellen, daher
muss der Klammerausdruck verschwinden.
Dies ist aber nur für $\lambda=0$ möglich.
Homogene Randbedingungen lassen sich also für $\lambda>0$ nicht erfüllen.

Für homogene Neumann-Randbedingungen sind die zu erfüllenden Gleichungen
\begin{align*}
0&= y'(\phantom{-}l)
=
A\sqrt{\lambda} e^{l\sqrt{\lambda}} - B\sqrt{\lambda}e^{-l\sqrt{\lambda}}
\\
0&= y'(         - l) =
A\sqrt{\lambda} e^{-l\sqrt{\lambda}} - B\sqrt{\lambda}e^{l\sqrt{\lambda}}.
\end{align*}
Die Determinante dieses homogenen linearen Gleichungssystems ist
\[
\biggl|\begin{matrix}
\sqrt{\lambda} e^{l\sqrt{\lambda}}  & -\sqrt{\lambda}e^{-l\sqrt{\lambda}}
\\
\sqrt{\lambda} e^{-l\sqrt{\lambda}} & -\sqrt{\lambda}e^{l\sqrt{\lambda}}
\end{matrix}\biggr|
=
-\lambda
\biggl|\begin{matrix}
e^{l\sqrt{\lambda}}  & e^{-l\sqrt{\lambda}}
\\
e^{-l\sqrt{\lambda}} & e^{l\sqrt{\lambda}}
\end{matrix}\biggr|
=
-\lambda
(
e^{2l\sqrt{\lambda}} - e^{-2l\lambda}
).
\]
Wie für Dirichlet-Randbedingungen folgt, dass auch diese Randbedingungen
nicht erfüllt werden können.

Wir versuchen auch noch periodische Randbedingungen zu erfüllen, also
die Gleichungen
\[
\begin{aligned}
y(-l) &= y(l)\\
y'(-l)&=y'(l)
\end{aligned}
\quad\Rightarrow\quad
\renewcommand{\arraycolsep}{2pt}
\begin{array}{rcrcrcr}
e^{-l\sqrt{\lambda}}A
&+&
e^{l\sqrt{\lambda}}B
&=&
e^{l\sqrt{\lambda}}A
&+&
e^{-l\sqrt{\lambda}}B
\phantom{.}
\\
\sqrt{\lambda}e^{-l\sqrt{\lambda}}A
&-&
\sqrt{\lambda}e^{l\sqrt{\lambda}}B
&=&
\sqrt{\lambda}e^{l\sqrt{\lambda}}A
&-&
\sqrt{\lambda}e^{-l\sqrt{\lambda}}B.
\end{array}
\]
Bringt man alles auf eine Seite, entsteht ein Gleichungssystem mit der
Koeffizientenmatrix
\[
\begin{pmatrix}
-e^{l\sqrt{\lambda}} + e^{-l\sqrt{\lambda}}
	& e^{l\sqrt{\lambda}} - e^{-l\sqrt{\lambda}}
\\
\sqrt{\lambda}(
-
e^{l\sqrt{\lambda}}
+
e^{-l\sqrt{\lambda}}
)
	& \sqrt{\lambda}(-e^{l\sqrt{\lambda}} + e^{-l\sqrt{\lambda}})
\end{pmatrix}
=
\begin{pmatrix*}[r]
-              2\sinh(l\sqrt{\lambda})&               2\sinh(l\sqrt{\lambda})\\
-\sqrt{\lambda}2\sinh(l\sqrt{\lambda})&-\sqrt{\lambda}2\sinh(l\sqrt{\lambda})
\end{pmatrix*}
\]
mit der Determinante
\[
4\sqrt{\lambda}
\bigl(
\sinh^2(l\!\sqrt{\lambda})
+
\sinh^2(l\!\sqrt{\lambda})
\bigr)
=
8\sqrt{\lambda} \sinh^2(l\sqrt{\lambda}),
\]
die nur verschwindet, wenn $l=0$ ist.
Somit lassen sich auch periodische Randbedingungen für $\lambda > 0$
nicht erfüllen.

%
% Funktionen auf einem Rechteck
%
\subsection{Funktionen auf einem Rechteck}
In diesem Abschnitt betrachten wir den Operator 
\[
\Delta
=
\frac{\partial^2}{\partial x^2}
+
\frac{\partial^2}{\partial y^2}
\]
auf einem Rechteck $\Omega = (0,l_x)\times (0,l_y)$.
Das Skalarprodukt ist wieder das übliche $L^2$-Skalarprodukt.


%
% $\Delta$ ist selbstadjungiert
%
\subsubsection{$\Delta$ ist selbstadjungiert}
Seien $f$ und $g$ zweimal stetig differenzierbare Funktionen und berechnen
das Skalarprodukt
\begin{align}
\langle \Delta f, g\rangle
&=
\int_0^{l_x}\int_0^{l_y} \Delta f(x,y) g(x,y) \,dx\,dy
\notag
\\
&=
\int_0^{l_x}\int_0^{l_y} \frac{\partial^2 f}{\partial x^2}(x,y)\,g(x,y)\,dx\,dy
+
\int_0^{l_x}\int_0^{l_y} \frac{\partial^2 f}{\partial y^2}(x,y)\,g(x,y)\,dx\,dy
\notag
\\
&=
\int_0^{l_x}\int_0^{l_y} \frac{\partial^2 f}{\partial x^2}(x,y)\,g(x,y)\,dx\,dy
+
\int_0^{l_y}\int_0^{l_x} \frac{\partial^2 f}{\partial y^2}(x,y)\,g(x,y)\,dy\,dx
\notag
\\
&=\phantom{+}
\int_0^{l_y}
\left[ \frac{\partial f}{\partial x}(x,y)\, g(x,y)\right]_0^{l_x}
\,dy
-
\int_0^{l_y}\int_0^{l_x}
\frac{\partial f}{\partial x}(x,y)\, \frac{\partial g}{\partial x}(x,y)
\,dx\,dy
\notag
\\
&\phantom{=}+
\int_0^{l_x}
\left[ \frac{\partial f}{\partial y}(x,y)\, g(x,y)\right]_0^{l_x}
\,dx
-
\int_0^{l_x}\int_0^{l_y}
\frac{\partial f}{\partial y}(x,y)\, \frac{\partial g}{\partial y}(x,y)
\,dy\,dx
\notag
\\
&=\phantom{+}
\int_0^{l_x} \frac{\partial f}{\partial y}(x,l_y)\, g(x,l_y) \,dx
-
\int_0^{l_x} \frac{\partial f}{\partial y}(x,0)\, g(x,0) \,dx
\label{buch:orthofkt:pdf:eqn:rechteck:r1}
\\
&\phantom{=}+
\int_0^{l_y} \frac{\partial f}{\partial x}(l_x,y)\, g(l_x,y) \,dy
-
\int_0^{l_y} \frac{\partial f}{\partial x}(0,y)\, g(0,y) \,dy
\label{buch:orthofkt:pdf:eqn:rechteck:r2}
\\
&\phantom{=}
-
\int_0^{l_x}
\int_0^{l_y}
\frac{\partial f}{\partial x} \frac{\partial g}{\partial x}
+
\frac{\partial f}{\partial y} \frac{\partial g}{\partial y}
\,dy\,dx.
\label{buch:orthofkt:pdf:eqn:rechteck:total}
\end{align}
Die ersten vier Integrale auf der rechten Seite hängen nur ab
von den Werten der Funktionen und der Ableitungen auf dem Rand.
Erfüllen die Funktionen homogene Dirichlet-Randbedingungen, dann
ist der Integrand immer 0 und auf der rechten Seite bleibt nur
der Term \eqref{buch:orthofkt:pdf:eqn:rechteck:total}.
%\[
%-\int_0^{l_x} \int_0^{l_y}
%\frac{\partial f}{\partial x} \frac{\partial g}{\partial x}
%+
%\frac{\partial f}{\partial y} \frac{\partial g}{\partial y}
%\,dy\,dx.
%\]
Der gleiche Term bleibt auch bei der Berechnung von
$\langle f,\Delta g\rangle$
stehen, so dass für Funktionen, die homogene Dirichlet-Randbedingungen
erfüllen, der Operator $\Delta$ selbstadjungiert ist.

Die Integrale
\eqref{buch:orthofkt:pdf:eqn:rechteck:r1}
über $x$ sind Integrale entlang der zur $x$-Achse parallelen Teile des
Randes von $\Omega$.
Die Ableitung $\partial f/\partial y$ ist die Ableitung in Richtung
senkrecht auf den Rand des Gebietes, die Normalableitung.
In den Integralen
\eqref{buch:orthofkt:pdf:eqn:rechteck:r2}
über $y$ sind Integrale entlang der zur $y$-Achse parallelen Teile
tritt die Ableitung $\partial f/\partial x$ auf, die die Normalableitung
ist.
Erfüllen die Funktionen homogene Neumann-Randbedingungen, dann sind
diese Normalableitungen $=0$ und die Integrale
\eqref{buch:orthofkt:pdf:eqn:rechteck:r1}
und
\eqref{buch:orthofkt:pdf:eqn:rechteck:r2}
verschwinden.
Auch in diesem Fall ist $\Delta$ auf den Funktionen, die homogene
Neumann-Randbedingungen erfüllen, selbstadjungiert.

%
% Eigenfunktionen
%
\subsubsection{Eigenfunktionen}
Eigenfunktionen des Laplace-Operators auf einem Rechteck sind
Lösungen der Differentialgleichung $\Delta f=\lambda f$.
Nimmt man an, dass die Funktion $f$ als Produkte zweier Funktionen
$X(x)$ und $Y(y)$ geschrieben werden kann, kann man den Ansatz
$f(x,y) = X(x)Y(y)$ in die Differentialgleichung einsetzen und erhält
\[
X''(x)Y(y) + X(x)Y''(y) = \lambda X(x)Y(y).
\]
Ausserhalb der Nullstellen von $X(x)$ und $Y(y)$ kann man durch das
Produkt teilen und bekommt
\[
\frac{X''(x)}{X(x)}
+
\frac{Y''(y)}{Y(y)}
=
\lambda
\qquad\Rightarrow\qquad
\frac{X''(x)}{X(x)}
=
\lambda
-
\frac{Y''(y)}{Y(y)}.
\]
Die linke Seite der letzten Gleichung hängt nur von $x$ ab,
die rechte Seite hängt nur von $y$ ab.
Dies bedeutet, dass beide Seiten konstant sind.
Es gibt also eine Konstante $\mu$ derart, dass
\begin{align*}
\frac{X''(x)}{X(x)}&= \mu
&&\Rightarrow
&
X''(x)&=\mu X(x)
&&\text{für $x\in (0,l_x)$}
\\
\frac{Y''(y)}{Y(y)}&=\lambda-\mu
&&\Rightarrow
&
Y''(y)&=(\lambda-\mu)Y(y)
&&\text{für $y\in (0,l_y)$}
\end{align*}
Damit ist das Problem auf das einimensionale Problem von
Abschnitt~ \ref{buch:orthofkt:subsection:fourier-theorie}
reduziert.
Für Dirichlet-Randbedingungen sind die Funktionen
\[
X(x) = \sin\frac{\pi k_x}{l_x} x
\quad\text{und}\quad
Y(y) = \sin\frac{\pi k_y}{l_y} y
\]
zu verwenden, die zugehörigen Eigenwerte sind
\[
\lambda
=
-\pi^2\biggl(
\frac{k_{x\mathstrut}^2\mathstrut}{l_x^2}
+
\frac{k_{y\mathstrut}^2\mathstrut}{l_y^2}
\biggr),
\quad
k_x,k_y\in\mathbb{Z}.
\]
Auf die gleiche Art und Weise können
auch für homogene Neumann-Randbedingungen Lösungen mit 
den Kosinusfunktionen gefunden werden.

%
% Periodische Randbedingungen
%
\subsubsection{Perdiodische Randbedingungen}
Die Integrale 
\eqref{buch:orthofkt:pdf:eqn:rechteck:r1}
und
\eqref{buch:orthofkt:pdf:eqn:rechteck:r2}
können aber auch zum verschwinden gebracht werden, wenn man sicherstellt,
wenn die Funktionswerte und Normalableitungen auf den Rändern des Gebietes
übereinstimmen.
Die Bedingungen
\begin{align*}
f(x,l_y) &= f(x,0)
&&\text{und}&
\frac{\partial f}{\partial x}(x,l_y) &= \frac{\partial f}{\partial x}(x,0)
\\
f(l_x,y) &= f(0,y)
&&\text{und}&
\frac{\partial f}{\partial y}(l_x,y) &= \frac{\partial f}{\partial y}(0,y).
\intertext{Aus den  ersten beiden Gleichung folgt ausserdem, dass}
\frac{\partial f}{\partial x}(x,0) &=
\frac{\partial f}{\partial x}(x,l_y)
&&\text{und}&
\frac{\partial f}{\partial y}(0,y) &=
\frac{\partial f}{\partial y}(l_x,y).
\end{align*}
Eine solche Funktion kann also $l_x$-periodisch in $x$ und $l_y$-periodisch
in $y$ auf ganz $\mathbb{R}$ erweitert werden, so dass die erweiterte Funktion
und ihre Ableitungen immer noch stetig sind.

%
% Funktionen in einem Kreisgebiet
%
\subsection{Funktionen in einem Kreisgebiet}
Ein Kreisgebiet $\Omega = \{(x,y)\mid x^2+y^2< r^2\}$ ist am einfachsten
mit Polarkoordinaten zu beschreiben.
Der Laplace-Operator in Polarkoordinaten ist
\[
\Delta 
=
\frac1r \frac{\partial}{\partial r} r \frac{\partial}{\partial r}
+
\frac{1}{r^2} \frac{\partial}{\partial\varphi}.
\]
Funktionen auf dem Kreisgebiet sind Funktion $f(r,\varphi)$ müssen
\[
f(0,\varphi) = f(0,0)
\qquad\text{und}\qquad
f(r,\varphi) = f(r,\varphi+2\pi)
\]
erfüllt sein, da die Funktion sonst nicht stetig sein kann.
Damit die Funktion auch im Nullpunkt differenzierbar ist, müssen die
Ableitungen für $r=0$ zusammenpassen.
Die Richtungsableitungen in $x$- und $y$-Richtung sind
\[
\frac{\partial f}{\partial r}(0,0)
\qquad\text{und}\qquad
\frac{\partial f}{\partial r}(0,{\textstyle\frac{\pi}2}).
\]
In jeder anderen Richtung ist die Richtungsableitung
\begin{equation}
\frac{\partial f}{\partial r}(0,\varphi)
=
\begin{pmatrix}
\frac{\partial f}{\partial r}(0,0)\\
\frac{\partial f}{\partial r}(0,{\textstyle\frac{\pi}2})\\
\end{pmatrix}
\cdot
\begin{pmatrix}
\cos\varphi\\
\sin\varphi
\end{pmatrix}
=
\frac{\partial f}{\partial r}(0,0)\cos\varphi
+
\frac{\partial f}{\partial r}(0,{\textstyle\frac{\pi}2})\sin\varphi.
\label{buch:orthofkt:eqn:phibed}
\end{equation}
Die Gleichung
\eqref{buch:orthofkt:eqn:phibed}
ist eine Bedingung, die Funktion $f$ erfüllen muss, damit $f$
differenzierbar ist.

%
% \Delta ist selbstadjungiert
%
\subsubsection{$\Delta$ ist selbstadjungiert}
Seien $f$ und $g$ auf dem Kreisgebiet definiert, dann können wir damit
das Skalarprodukt
\begin{align}
\langle \Delta f, g\rangle
&=
\int_0^R \int_0^{2\pi}  \Delta f(r,\varphi) g(r,\varphi)\,r\,d\varphi\,dr
\notag
\\
&=
\int_0^R \int_0^{2\pi}
\biggl(
\frac1r \frac{\partial}{\partial r}r\frac{\partial f}{\partial r}(r,\varphi)
+
\frac1{r^2}\frac{\partial^2 f}{\partial\varphi^2}(r,\varphi)
\biggr)
g(r,\varphi)
\, r\,d\varphi\,dr
\notag
\\
&=
\int_0^{2\pi}
\int_0^R
\frac{\partial}{\partial r}r\frac{\partial f}{\partial r}(r,\varphi)
g(r,\varphi)
\,dr
\,
d\varphi
+
\int_0^R
\frac1{r}
\int_0^{2\pi}
\frac{\partial^2 f}{\partial\varphi^2}(r,\varphi)
g(r,\varphi)
\,d\varphi
\,dr
\notag
\\
&=
\phantom{+}
\int_0^{2\pi}
\biggl[
\frac{\partial f}{\partial r}(r,\varphi)\, g(r,\varphi)
\biggr]_0^R
\,r
\,d\varphi
-
\int_0^R
\int_0^{2\pi}
\frac{\partial f}{\partial r}
\frac{\partial g}{\partial r}
\,d\varphi
\,dr
\label{buch:orthofkt:pde:eqn:kreisdr}
\\
&\phantom{=}
+
\int_{0}^R
\frac1r
\biggl[
\frac{\partial f}{\partial \varphi}(r,\varphi)
g(r,\varphi)
\biggr]_0^{2\pi}
\,dr
-
\int_{0}^R
\frac1r
\int_0^{2\pi}
\frac{\partial f}{\partial\varphi}(r,\varphi)
\frac{\partial g}{\partial\varphi}(r,\varphi)
\,d\varphi
\,dr
\label{buch:orthofkt:pde:eqn:kreisdphi}
\end{align}
berechnen.
Das erste Integral in 
\eqref{buch:orthofkt:pde:eqn:kreisdphi}
verschwindet, weil $f$ und $g$ $2\pi$-periodisch sind in $\varphi$.
Das erste Integral in 
\eqref{buch:orthofkt:pde:eqn:kreisdr}
hängt nur von den Werten von $g$ und der Ableitung von $f$ in $r$-Richtung
auf dem Rand des Gebietes ab.
Die doppelten Integrale sind symmetrisch in $f$ und $g$.
Daraus kann man wieder folgern, dass $\Delta$ auf Funktionen mit
einer homogenen Dirichlet- oder einer homogenen Neumann-Randbedingung
selbstadjungiert ist.

%
% Eigenfunktionen
%
\subsubsection{Eigenfunktionen}
Mit dem Ansatz $f(r,\varphi) = R(r) \Phi(\varphi)$ kann man versuchen,
Lösungen der partiellen Differentialgleichung $\Delta f=\lambda f$
zu finden.
Einsetzen in die Differentialgleichung führt auf
\[
\frac1r \frac{d}{dr} rR'(r) \Phi(\varphi)
+
\frac1{r^2} \Phi''(\varphi)
=
\lambda R(r)\Phi(\varphi).
\]
Auch hier können die Funktionen $R(r)$ und $\Phi(\varphi)$ mittels
Division durch $f(r,\varphi)$ getrennt werden.
Aus
\[
\frac1r \frac{R'(r) + rR''(r)}{R(r)}
+
\frac{1}{r^2}\frac{\Phi''(\varphi)}{\Phi(\varphi)}
=
\lambda
\]
werden die beiden Gleichungen
\[
\frac{r^2R''(r) + rR'(r)}{R(r)}
-
\lambda r^2
=
-\frac{\Phi''(\varphi)}{\Phi(\varphi)}
\quad\Rightarrow\quad
\left\{
\begin{aligned}
r^2R''(r) + rR'(r) +(m^2 - \lambda r^2)R(r) &= 0
\\
\Phi''(\varphi)&=-m^2 \Phi(\varphi)
\end{aligned}
\right.
\]
Die zweite Gleichung hat die trigonometrischen Funktionen
als Lösungen.
Die erste Gleichung ist etwas schwieriger zu lösen.
Sie kann zum Beispiel mit der Potenzreihenmethode gelöst werden, die
Lösungsfunktionen sind die Bessel-Funktionen.

%
% Funktionen auf einer Kugeloberfläche
%
\subsection{Funktionen auf einer Kugeloberfläche}
Die Wellenausbreitung auf der Erdoberfläche kann mit Funktionen
auf einer Kugeloberfläche modelliert werden.
Die Wellengleichung verwendet den Laplace-Operator, der in Kugelkoordinaten
die Form
\begin{equation}
\Delta
=
\frac{1}{r^2}
\frac{\partial}{\partial r}
r^2
\frac{\partial}{\partial r}
+
\frac{1}{r^2\sin\vartheta}
\frac{\partial}{\partial\vartheta}
\biggl(\sin\vartheta\frac{\partial}{\partial\vartheta}\biggr)
+
\frac{1}{r^2\sin^2\vartheta}
\frac{\partial^2}{\partial\varphi^2}
\label{buch:orthofkt:pde:laplacekugel}
\end{equation}
hat.
Die Abhängigkeit von $r$ wird auf der Kugeloberfläche nicht benötigt,
so dass sich der Operator vereinfacht auf
\[
\Delta
=
\frac{1}{\sin\vartheta}
\frac{\partial}{\partial\vartheta}
\biggl(\sin\vartheta\frac{\partial}{\partial\vartheta}\biggr)
+
\frac{1}{\sin^2\vartheta}
\frac{\partial^2}{\partial\varphi^2}.
\]
Als Skalarprodukt wird das gewöhnliche $L^2$-Skalarprodukt
\[
\langle f,g\rangle
=
\int_{0}^{2\pi}
\int_0^\pi
f(\vartheta,\varphi)
g(\vartheta,\varphi)
\sin \vartheta
\,d\vartheta
\,d\varphi
\]
verwendet.

%
% \Delta ist selbstadjungiert
%
\subsubsection{$\Delta$ ist selbstadjungiert}
Der Laplace-Operator ist auch für Funktionen auf der Kugeloberfläche
selbstadjungiert.
Da die Kugeloberfläche keinen Rand hat, sind die einzige Bedingungen an
die Funktionen, dass sie auf der Kugeloberfläche differenzierbar sind.
Dazu müssen sie in $\varphi$ $2\pi$-periodisch sein und 
$f(0,\varphi)$ und $f(\pi,\varphi)$ müssen konstant sein.
Damit die Ableitungen an den Polen der Kugel existieren, müssen die
Bedingungen
\[
\frac{\partial f}{\partial\vartheta}(\vartheta,\varphi)
=
\frac{\partial f}{\partial\vartheta}(\vartheta,0)
\cos\varphi
+
\frac{\partial f}{\partial\vartheta}(\vartheta,{\textstyle\frac{\pi}2})
\sin\varphi
\]
erfüllen.
\begin{align}
\langle \Delta f, g\rangle
&=
\int_{0}^{2\pi}
\int_0^\pi
\Delta
f(\vartheta,\varphi)
g(\vartheta,\varphi)
\sin \vartheta
\,d\vartheta
\,d\varphi
\notag
\\
&=
\int_{0}^{2\pi}
\int_0^\pi
\biggl(
\frac{1}{\sin\vartheta}\frac{\partial}{\partial\vartheta}
\biggl(
\sin\vartheta\frac{\partial f}{\partial\vartheta}
\biggr)
+
\frac{1}{\sin^2\vartheta}
\frac{\partial^2 f}{\partial\varphi^2}
\biggr)
g(\vartheta,\varphi)
\sin \vartheta
\,d\vartheta
\,d\varphi
\notag
\\
&=
\int_{0}^{2\pi}
\int_0^\pi
\frac{\partial}{\partial\vartheta}
\biggl(
\sin\vartheta\frac{\partial f}{\partial\vartheta}
\biggr)
\,d\vartheta
\,d\varphi
+
\int_{0}^{2\pi}
\int_0^\pi
\frac{1}{\sin^2\vartheta}
\frac{\partial^2 f}{\partial\varphi^2}
g(\vartheta,\varphi)
\sin \vartheta
\,d\vartheta
\,d\varphi
\notag
\\
&=\phantom{+}
\int_0^{2\pi}
\biggl[
\sin\vartheta\frac{\partial f}{\partial\vartheta}(\vartheta,\varphi)
g(\vartheta,\varphi)
\biggr]_0^\pi
\,d\varphi
-
\int_0^{2\pi}
\int_0^\pi
\frac{\partial f}{\partial\vartheta}(\vartheta,\varphi)
\frac{\partial g}{\partial\vartheta}(\vartheta,\varphi)
\sin\vartheta
\,d\vartheta
\,d\varphi
\label{buch:orthofkt:pde:deltakugel1}
\\
&\phantom{=}+
\int_0^\pi \frac{1}{\sin\vartheta}
\biggl[
\frac{\partial f}{\partial\varphi}(\vartheta,\varphi)\, g(\vartheta,\varphi)
\biggr]_0^{2\pi}
\,d\vartheta
-
\int_0^\pi
\int_{0}^{2\pi}
\frac{\partial f}{\partial\varphi}
\frac{\partial g}{\partial\varphi}
\sin\vartheta
\,d\varphi
\,d\vartheta
\label{buch:orthofkt:pde:deltakugel2}
\end{align}
Das erste Integral in
\eqref{buch:orthofkt:pde:deltakugel2} verschwindet, weil die
Funktion und die Ableitungen $2\pi$-periodisch in $\varphi$
sind.
Das erste Integral in
\eqref{buch:orthofkt:pde:deltakugel1}
kann man durch die Beobachtung vereinfachen, dass für
$\vartheta=0$ und $\vartheta=\pi$.
die Funktionen konstant sein müssen.
Das erste Integral in \eqref{buch:orthofkt:pde:deltakugel1}
verschwindet, weil $\sin0=\sin\vartheta=0$.
Die verbleibenden doppelten Integrale in \eqref{buch:orthofkt:pde:deltakugel1}
und \eqref{buch:orthofkt:pde:deltakugel2} sind symmetrisch in $f$ und $g$.
Somit ist der Laplace-Operator auf beliebigen auf der Kugeloberfläche
differenzierbaren Funktionen selbstadjungiert.
Aus der allgemeinen Theorie folgt somit, dass die Eigenfunktionen orthogonal
sind.

%
% Lösungen der Legendre-Differentialgleichung
%
\subsubsection{Lösungen der Legendre-Differentialgleichung}
Die Eigenfunktionen des Laplace-Operators auf der Kugeloberfläche
kann wieder mit einem Ansatz
$f(\vartheta,\varphi)=\Theta(\vartheta)\Phi(\varphi)$
gefunden werden.
Einsetzen in die Differentialgleichung gibt
\begin{align*}
\frac{1}{\sin\vartheta}
\frac{\partial}{\partial\vartheta}\bigl(
\sin\vartheta \Theta'(\vartheta)\Phi(\varphi)
\bigr)
+
\frac{1}{\sin^2\vartheta} \Phi''(\varphi)\Theta(\vartheta)
&=
\lambda
\Theta(\vartheta)\Phi(\varphi)
\\
\frac{1}{\sin\vartheta}
\bigl(
\cos\vartheta \Theta'(\vartheta)
+
\sin\vartheta\Theta''(\vartheta)
\bigr)
\Phi(\varphi)
+
\frac{1}{\sin^2\vartheta} \Phi''(\varphi)\Theta(\vartheta)
&=
\lambda
\Theta(\vartheta)\Phi(\varphi).
\intertext{Division durch $f(\vartheta,\varphi)$ und Multiplikation
mit $\sin^2\vartheta$ ergibt}
\sin\vartheta
\frac{
\cos\vartheta \Theta'(\vartheta) +\sin\vartheta\Theta''(\vartheta)
}{
\Theta(\vartheta)
}
+
\frac{\Phi''(\varphi)}{\Phi(\varphi)}
&=
\lambda\sin^2\vartheta,
\intertext{in der man $\vartheta$ auf die linke Seite und $\varphi$ auf
die rechte Seite verschieben kann}
\sin\vartheta
\frac{
\cos\vartheta\Theta'(\vartheta)
+
\sin\vartheta\Theta''(\vartheta)
-\lambda\sin\vartheta \Theta(\vartheta)
}{
\Theta(\vartheta)
}
&=
-\frac{\Phi''(\varphi)}{\Phi(\varphi)}
=
m^2
\end{align*}
Für die Funktionen $\Phi(\varphi)$ sind wieder beliebige Linearkombinationen
der trigonometrischen Funktionen $\cos m\varphi$ und $\sin m\varphi$
die Lösungen.
Alternativ können sie auch als komplexe Funktionen
$\Phi_m(\varphi) = e^{im\varphi}$ geschrieben werden.

Die Funktion $\Theta(\vartheta)$ erfüllt die Differentialgleichung
\[
\sin^2\vartheta \Theta''(\vartheta)
+
\sin\vartheta\cos\vartheta
\Theta'(\vartheta)
-
\lambda\sin^2\vartheta
\Theta(\vartheta)
=
m^2 \Theta(\vartheta).
\]
Division durch $\sin^2\vartheta$ kann die Gleichung etwas vereinfachen:
\begin{equation}
\Theta''(\vartheta)
+
\frac{\cos\vartheta}{\sin\vartheta} \Theta'(\vartheta)
+\biggl(
-
\lambda
-
\frac{m^2}{\sin^2\vartheta} 
\biggr)
\Theta(\vartheta)
=
0
\label{buch:orthfkt:pde:dglthetaphi}
\end{equation}
Diese Gleichung ist bekannt als die {\em assoziierte
Legendre-Differentialgleichung}.
Man kann zeigen, dass sie $\lambda=-l(l+1)$ mit $l\in\mathbb{N}$
als Eigenwerte hat.

\begin{satz}
\label{buch:orthofkt:pde:satz:kugel}
Der Laplace-Operator auf der Kugeloberfläche hat die Form
\eqref{buch:orthofkt:pde:laplacekugel}.
Seine Eigenfunktionen sind von der Form
$Y_l^m(\vartheta,\varphi)=e^{im\varphi}\Theta_l^m(\vartheta)$,
mit $l\in \mathbb{N}$ und ganzzahligen $m$ mit $-l\le m\le l$.
Die Funktionen $\Theta_l^m(\vartheta)$ sind Lösungen der
Differentialgleichung \eqref{buch:orthfkt:pde:dglthetaphi}, wobei
der Eigenwert $\lambda = -l(l+1)$ ist.
\end{satz}

Die Berechnung der Funktionen $\Theta_l^m(\vartheta)$ wird viel
einfacher in Abhängigkeit von der Koordinaten $z=\cos\vartheta$
anstelle von $\vartheta$.

%
% Koordinatentransformation z = cos theta
%
\subsubsection{Koordinatentransformation $z=\cos\vartheta$}
Die Parametrisierung mit $\vartheta$
in \eqref{buch:orthfkt:pde:dglthetaphi}
ist etwas unhandlich, daher soll der Parameter
$\vartheta$ durch $z=\cos\vartheta$ und
$\sin\vartheta = \sqrt{1-z^2}$
ersetzt werden.
Die Funktion $\Theta(\vartheta)$ ist dann eine Funktion $P(z)$ mit
$P(\cos\vartheta)=\Theta(\vartheta)$.
Die Ableitungen von $\Theta$ können durch Ableitungen von $P$
ausgedrückt werden wie folgt.
Zunächst sind die Ableitungen nach $\vartheta$:
\begin{align*}
\Theta'(\vartheta)
&=
\frac{d}{d\vartheta} P(\cos\vartheta)
=
-P'(\cos\vartheta) \sin\vartheta
\\
&=
-
\sqrt{1-z^2}
P'(z)
\\
\Theta''(\vartheta)
&=
\frac{d}{d\vartheta} 
\Theta'(\vartheta)
=
-\frac{d}{d\vartheta}\bigl(P'(\cos\vartheta) \sin\vartheta\bigr)
=
P''(\cos\vartheta)\sin^2\vartheta
-
P'(\cos\vartheta)\cos\vartheta
\\
&=
(1-z^2)
P''(z)
-
z
P'(z)
\end{align*}
Damit kann man die Differentialgleichung
\eqref{buch:orthfkt:pde:dglthetaphi}
in die Form
\[
(1-z^2)P''(z)
-
zP'(z)
-
\frac{z}{\sqrt{1-z^2}}\sqrt{1-z^2}P'(z)
-
\lambda P(z)
-
\frac{m^2}{1-z^2}P(z)
=
0
\]
für die Funktion $P(z)$ finden.
\begin{equation}
(1-z^2)P''(z) - 2zP'(z)
+
\biggl(
-
\lambda
-
\frac{m^2}{1-z^2}
\biggr)
P(z)
=
0
\label{buch:orthofkt:pde:eqn:legendreassocz}
\end{equation}
Diese Differentialgleichung kann auch in der Form des
Sturm-Liouville-Operators
\[
L
=
\frac{d}{dz}
(1-z^2)
\frac{d}{dz}
+
\biggl(-\lambda-\frac{m^2}{1-z^2}\biggr)
\]
geschrieben werden.
Solche Operatoren haben wir früher studiert.

%
% Eigenfunktionen
%
\subsubsection{Eigenfunktionen}
Die Lösungen der assozierten Legendre-Differentialgleichung
in der Form 
\[
(1-z^2)P''(z)
-2z P'(z) +
\biggl(
-l(l+1)-\frac{m^2}{1-z^2}
\biggr)P(z)
=
0
\]
können mit Polynomen gelöst werden.
Sie heissen die {\em assoziierten Legendre-Polynome} $P_l^m(z)$.
Eine ausführliche Diskussion dieser Lösungen kann in
\cite{buch:mathsem-spezfunk}
gefunden werden.

Mit Hilfe der assoziierten Legendre-Polynomen können die Eigenfunktion
von $\Delta$ als
\[
e^{im\varphi}
P_l^m(z)
\qquad\text{mit $-l\le m\le l$}
\]
geschrieben werden.



