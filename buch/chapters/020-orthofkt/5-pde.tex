%
% 5-pde.tex -- PDE
%
% (c) 2022 Prof Dr Andreas Müller, OST Ostschweizer Fachhochschule
%
\section{Partielle Differentialoperatoren
\label{buch:orthofkt:section:pde}}
\kopfrechts{Partielle Differentialoperatoren}
Selbstadjungierte partielle Differentialoperatoren auf einem Gebiet 
$\Omega$ führen auf natürlich Art und Weise zu einer verallgemeinerten
Theorie der harmonischen Analysis, die auch zur Lösung der
partiellen Differentialgleichung verwendet werden kann.
Die Vorgehensweise wird zunächst in
Abschnitt~\ref{buch:orthfkt:pde:subsection:pdeharm}
allgemein skizziert und anschliessend für ein paar interessante
Gebiete und Differentialoperatoren durchgeführt.

%
% Partielle Differentialgleichungen udn harmonische Analysis
%
\subsection{Partielle Differentialgleichung und harmonische Analysis
\label{buch:orthfkt:pde:subsection:pdeharm}}
In diesem Abschnitt soll illustriert werden, wie sich aus Aufgabenstellungen
aus der Theorie der partiellen Differentialgleichungen 

%
% Partielle Differentialoperatoren
%
\subsubsection{Lineare partielle Differentialoperatoren}
Ein linearer partieller Differentialoperator ist eine lineare
Funktion der partiellen Ableitungen einer Funktion, die auf einem
Gebiet definiert ist.

\begin{definition}
Ein Gebiet $\Omega$ ist eine offene Teilmenge von $\mathbb{R}^n$.
Ein partieller Differentialoperator der Ordnung $r$ ist eine Linearkombination
\[
L
=
\sum_{k_1+\dots+k_n\le r}
a_{k_1\dots k_n}(x)
\frac{\partial^{k_1+\dots+k_n}}{\partial x_1^{k_1}\dots\partial x_n^{k_n}}.
\]
\end{definition}

\begin{beispiel}
Der Laplace-Operator auf einem beliebigen Gebiet $\Omega\subset\mathbb{R}^n$
ist definiert als
\[
\Delta
=
\frac{\partial^2}{\partial x_1^2}
+
\dots
+
\frac{\partial^2}{\partial x_n^2}
\]
ist ein partieller Differentialoperator zweiter Ordnung.
\end{beispiel}

%
% Skalarprodukt
%
\subsubsection{Skalarprodukt}
Da $\Omega$ eine offene Teilmenge von $\mathbb{R}^n$ ist, lässt sich
sofort ein Skalarprodukt von Funktionen auf $\Omega$ definieren.

\begin{definition}
Ist $\Omega\subset \mathbb{R}^n$ ein Gebiet in $\mathbb{R}^n$, dann ist
\begin{equation}
\langle u,v\rangle_\Omega
=
\int_{\Omega}  \overline{u(x)} v(x)\,dx
\label{buch:orthofkt:pde:eqn:skalarprodukt}
\end{equation}
ein Skalarprodukt für Funktionen auf $\Omega$.
\end{definition}

Mit einer Gewichtsfunktion $w\colon \Omega \to \mathbb{R}^+$ lässt sich
das Skalarprodukt noch etwas verallgemeinern, es wird zu
\[
\langle u,v\rangle_{\Omega,w}
=
\int_{\Omega} \overline{u(x)}v(x)\,w(x)\,dx.
\]
Diese Erweiterung ist auch deshalb nötig, weil bei einem Wechsel
der Koordinaten nach $x_i=x_i(x_1',\dots,x_n')$, das Integral
\eqref{buch:orthofkt:pde:eqn:skalarprodukt}
den zusätzlichen Faktor der Funktionaldeterminante 
\[
\langle u,v\rangle_{\Omega}
=
\int_{\Omega} \overline{u(x)} v(x)\,dx 
=
\int_{\Omega'} \overline{u(x')} v(x')
\det\frac{\partial(x_1,\dots,x_n)}{\partial(x_1',\dots,x_n')}
\,dx'
\]
erhält, wobei $\Omega'$ das Gebiet in $\mathbb{R}^n$ ist, welches
durch die Koordinatenfunktionen $x_i(x_1',\dots,x_n')$ bijektiv auf
$\Omega$ abgebildet wird.


%
% Randbedingungen
%
\subsubsection{Randbedingungen}
Wie beim Sturm-Liouville-Problem entsteht ein interessantes 
erst dadurch, dass zusätzlich Randbedingungen formuliert werden.
Der Operator $L=d^2/dx^2$ auf dem Intervall $(a,b)\subset\mathbb{R}$
ist zum Beispiel nur selbstadjungiert, wenn man ihn auf Funktionen
einschränkt, die gewisse Randbedingungen erfüllen.
$L$ ist der Sturm-Liouville-Operator von Abschnitt
\ref{buch:orthofkt:sturmliouville:subsection:operator}
mit $p(x)=1$ und $q(x)=0$.
In Abschnitt~\ref{buch:orthofkt:sturmliouville:subsection:operator}
wurde gezeigt, dass der Operator $L$ nur selbstadjungiert bezüglich
des $L^2$-Skalarproduktes wird, wenn Randbedinungen der Form
\begin{equation}
h_a u(a) + k_a u'(a) =  0
\qquad\text{und}\qquad
h_b u(b) + k_a u'(b) =  0
\end{equation}
gefordert werden.
Die Wahl $h_a=h_b=1$ und $k_a=k_b=0$ hat zum Beispiel zur Folge,
dass die Funktionen die Randwerte $u(a)=u(b)=0$ haben müssen.
Die Wahl $h_a=h_b=0$ und $k_a=k_b=1$ hat dagegen zur Folge,
dass die Ableitungen der Funktionen in den Endpunkten $=0$ sein
müssen.
Für partielle Differentialoperatoren und die 

\begin{definition}[Dirichlet-Randbedingungen]
Sei $\Omega\subset\mathbb{R}^n$ ein Gebiet mit Rand $\partial\Omega$
und $f\colon \partial\Omega\mathbb R$ eine Funktion auf dem Rand.
Man sagt, eine Funktion $u\colon\overline{\Omega}\to\mathbb{R}$ erfüllt 
{\em Dirichlet-Randbedingungen}, wenn
\index{Randbedingung!Dirichlet-}%
\index{Dirichlet-Randbedingung}%
\(
u(x) = f(x)
\)
für $x\in \partial\Omega$ ist.
\end{definition}

Dirichlet-Randbedinungen entsprechen also der Vorgabe von Werten
oder dem Fall $h_a=h_b=0$ und $k_a=k_b=0$.

\begin{definition}
Sei $\Omega\in\mathbb{R}^n$ ein Gebiet mit glattem Rand und
$n(x)$ der Einheitsnormalenvektor im Punkt $x\in\partial\Omega$
auf dem Rand.
Die {\em Normalableitung} einer Funktion
\index{Normalableitung}%
$u\colon\overline{\Omega}\to \mathbb{R}$ auf dem Rand ist die
Richtungsableitung
\[
\frac{\partial u}{\partial n}
=
\frac{d}{dt} u(x+tn(x))\bigg|_{t=0}.
\]
Man sagt, die Funktion $u$ erfüllt eine Neumann-Randbedingung, wenn
\index{Neumann-Randbedingung}%
\index{Randbedingung!Neumann-}%
\[
\frac{\partial u}{\partial n}(x) = g(x)
\]
erfüllt ist.
\end{definition}

Neumann-Randbedingungen entsprechen also der Vorgabe der Ableitungen
oder dem Fall $h_a=h_b=0$ und $k_a=k_b=0$.
Die Dirichlet- und Neumann-Randbedingungen können auch in sogenannte
gemischte Randbedingungen zusammengefasst werden, indem 
\index{Randbedingung!gemischt}%
\index{gemischte Randbedingung}%
\begin{equation}
h(x) u(x) + k(x) \frac{\partial u}{\partial n}(x) = g(x)
\label{buch:orthofunkt:pde:eqn:gemischterandbedingung}
\end{equation}
für  $x \in\partial \Omega$ für auf dem Rand definierte Funktion
$h(x)$ und $k(x)$ gefordert wird.

%
% Homogene und inhomogene Differentialgleichungen und Randbedingungen
%
\subsubsection{Homogene und inhomogene Differentialgleichungen und
Randbedingungen}
Eine partielle Differentialgleichung ist eine Differentialgleichung
der Form
\begin{equation}
L u = v,
\label{buch:orthofunkt:pde:eqn:dgl}
\end{equation}
wobei $L$ ein Differentialoperator auf dem Gebiet $\Omega$ ist und
$h$ eine Funktion auf $\Omega$.
Eine Lösung der partiellen Differentialgleichung ist eine Funktion
$u\colon\overline{\Omega}\to \mathbb{R}$ derart, dass die Gleichung
\eqref{buch:orthofunkt:pde:eqn:dgl} in $\Omega$ gilt.
Ohne zusätzliche Randbedingungen wird die Lösung der Gleichung
im Allgemeinen nicht eindeutig sein.
Sind $u_1$ und $u_2$ Lösungen der Differentialgleichung
\eqref{buch:orthofunkt:pde:eqn:dgl}, dann ist
\[
L(u_1-u_2)
=
Lu_1-Lu_2
=
v-v
=0,
\]
die Differenz $u_1-u_2$ ist also eine Lösung der zugehörigen
{\em homogenen} Differentialgleichung.
\index{homogen}%
Eine Lösung $u_p$ von 
\eqref{buch:orthofunkt:pde:eqn:dgl} heisst auch eine
{\em partikuläre Lösung}.

Die beiden Lösungen können sich durch die Werte der Funktionen
oder der Normalableitungen auf dem Rand unterscheiden.
Es ist also im allgemeinen nötig, eine Randbedingung vorzugeben.
Eine partikuläre Lösung $u_p$ erfüllt die Randbedingung möglicherweise
nicht.
Eine Lösung $u$, die auch die Randbedingung erfüllt, unterschiedet sich
von $u_p$ um 
$\tilde{u} = u-u_p$ und erfüllt die Differentialgleichung
\[
L\tilde{u}
=
Lu - Lu_p
=
f - f
=
0
\]
und die Randbedingung
\begin{align}
h(x)\tilde{u}(x) + k(x)\frac{\partial \tilde{u}}{\partial x} (x)
&=
h(x)u(x) + k(x)\frac{\partial u}{\partial n}(x)
-
h(x)u_p(x) - k(x)\frac{\partial u_p}{\partial n}(x)
\notag
\\
&=
g(x)
-
h(x)u_p(x) - k(x)\frac{\partial u_p}{\partial n}(x).
\label{buch:orthofunkt:pde:eqn:utilderand}
\end{align}
Um die Lösung $\tilde{u}$ zu finden, reicht es daher, eine Lösung der
homogenen Differentialgleichung $L\tilde{u}=0$ zu finden, die die
Randbedingungen \eqref{buch:orthofunkt:pde:eqn:utilderand}
erfüllt.
Das Problem ist damit reduziert auf eine homogene Differentialgleichung
mit möglicherweise inhomogenen Randbedingungen.

Durch Vorgabe einer Randbedingung werden die möglichen Lösungen
eingeschränkt, es ist aber immer noch möglich, dass die Lösung
nicht eindeutig ist.
Seien also $u_1$ und $u_2$ Lösungen der Differentialgleichung mit der 
Randbedingung~\eqref{buch:orthofunkt:pde:eqn:gemischterandbedingung},
dann ist die Differenz $u=u_1-u_2$ eine Lösung der homogenen
Differentialgleichung $Lu=0$ und die Randbedingungen sind
\[
h(x) u(x) + k(x) \frac{\partial u}{\partial n}(x)
=
h(x) u_1(x) + k(x) \frac{\partial u_1}{\partial n}(x)
-
h(x) u_2(x) + k(x) \frac{\partial u_2}{\partial n}(x)
=
g(x)-g(x)=0.
\]
Die Differenz erfüllt also die homogene Differentialgleichung
{\em und} zusätzlich homogene Randbedingungen.
Die Frage nach der Eindeutigkeit der Lösung der partiellen
Differentialgleichung mit den gegebenen Randbedingungen wird also
beantwortet von der Lösungsmenge der homogenen Differentialgleichung
mit homogenen Randbedingungen.

%
% Harmonische Analysis
%
\subsubsection{Harmonische Analysis}
Die im Fall des Eigenwertproblems für den Sturm-Liouville-Operator
beschriebene Situation tritt sehr häufig auch bei partiellen
Differentialgleichungen auf.
Wir erwarten daher, dass zu einem partiellen Differentialoperator $L$
auf einem Gebiet $\Omega$ jeweils eine orthogonale Funktionenfamilie
von Eigenfunktionen  $u_n(x)$ des Operators mit Eigenwerten $\lambda_n$
gehört, die zur Synthese beliebiger Funktionen auf $\Omega$ verwendet
werden kann.

Für sogenannte elliptische partielle Differentialoperatoren zweiter
Ordnung lassen sich dann die Lösungsfunktionen der Gleichungen
\begin{align*}
Lu &= \varrho                            &&\text{Poisson-Gleichung}      \\
\intertext{oder für den Fall, dass $u(t,x)$ zusätzlich von der Zeit abhängt}
\frac{\partial u}{\partial t}     &= Lu  &&\text{Wärmeleitungsgleichung} \\
\frac{\partial^2 u}{\partial t^2} &= LU  &&\text{Wellengleichung}
\end{align*}
als Reihe 
geeigneten Randbedingungen lassen sich die Lösungsfunktionen dann in 
der Form
\begin{equation}
u(x)
=
\sum_{n=0}^\infty a_n u_n(x)
\label{buch:orthofkt:eqn:unreihe}
\end{equation}
entwickeln, wobei die Koeffizienten $a_n$ 
im Falle der Wärmeleitungslgeichung und der Wellengleichung
Funktionen der Zeit sind.
Eingesetzt in die Differentialgleichung ergeben sich dann für die
Koeffizienten $a_n$ die Gleichungenn
\begin{align}
\lambda_n a_n  &=  b_n      &&\text{Poisson-Gleichung} 
\notag
\\
\dot{a}_n &= \lambda_n a_n  &&\text{Wärmeleitungsgleichung}
\label{buch:orthfkt:eqn:waermeleitung:ode}
\\
\ddot{a}_n &= \lambda_n a_n &&\text{Wellengleichung,}
\label{buch:orthfkt:eqn:wellen:ode}
\end{align}
wobei die $b_n$ die Koeffizienten der Entwicklung von $\varrho$ nach
den Funktionen $u_n(x)$ ist.
Im Falle der Wärmeleitungsgleichung wird das Problem also reduziert
auf die Lösung einer gewöhnlichen Differentialgleichung.

Zusätzlich zu den Randbedingungen auf dem Rand des Gebietes $\Omega$,
auf dem der Differentialoperator $L$ definiert ist, braucht die
Wärmeleitungsgleichung auch noch eine Randbedingung der Form
\[
u(0,x) = f(x)
\]
hinzu, die durch Anwendung der Entwicklung~\eqref{buch:orthofkt:eqn:unreihe}
zur Anfangsbedinung 
\[
a_n(0) = f_n
\]
für \eqref{buch:orthfkt:eqn:waermeleitung:ode} wird, wobei $f_n$ die
Koeffizienten der Entwicklung der Funktion $f(x)$ nach den Eigenfunktionen
$u_n$ sind.
Für die Wellengleichung \eqref{buch:orthfkt:eqn:wellen:ode} wird 
auch noch eine Neumann-Randbedingung der Form
\[
\frac{\partial u}{\partial t}(0,x) = g(x)
\]
nötig, die zu einer Anfangsbedingung für die Ableitungen
\[
\dot{a}_n(0) = g_n
\]
für die gewöhnliche Differentialgleichung \eqref{buch:orthfkt:eqn:wellen:ode}
wird.

Die Bestimmung der Entwicklungskoeffizienten erfolgt jeweils mit Hilfe
der Skalarprodukte
\[
b_n = \langle u_n, \varrho\rangle,
\qquad
f_n = \langle u_n, f\rangle,
\qquad
g_n = \langle u_n, g\rangle
\]
gewonnen ist.

%
% Fourier-Theorie
%
\subsection{Fourier-Theorie}

%
% Funktionen auf einem Rechteck
%
\subsection{Funktionen auf einem Rechteck}

%
%
%
\subsection{Funktionen auf einer Kugeloberfläche}

