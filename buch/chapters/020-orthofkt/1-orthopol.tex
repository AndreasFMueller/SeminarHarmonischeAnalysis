%
% 1-orthopol.tex -- trigonometrische Funktionen als Beispiel
%
% (c) 2023 Prof Dr Andreas Müller, OST Ostschweizer Fachschule
%
\section{Orthogonale Polynome
\label{buch:orthofkt:section:orthopol}}
Das Orthogonalisierungsverfahren von Gram-Schmidt kann dazu verwendet
werden, Familien von orthogonalen Polynomen zu konstruieren.

%
% Das Orthogonalisierungsverfahren von Gram-Schmidt
%
\subsection{Das Orthogonalisierungsverfahren von Gram-Schmidt}
Das Orthogonalisierungsverfahren von Gram-Schmidt geht aus von einer
Basis $a_1,a_2,\dots,a_n\subset V$ eines Vektorraums mit einem
Skalarprodukt, welches wie in Kapitel~\ref{buch:chapter:skalarprodukt}
mit $\langle\;\,,\;\rangle$ bezeichnet wird.
Aus den Vektoren $a_i$ können jetzt die orthogonalen Vektoren $b_i$
konstruiert werden, die die zusätzliche Eigenschaft haben, dass der
von $b_i$ immer im von den Vektoren $a_1,\dots,a_i$ aufgespannten
Unterraum von $V$ liegt.
Diese Bedingungen legen nur die Richtung von $b_i$ fest. 
Auch die Forderung, dass die Norm $\|b_i\|=1$ sein soll, lässt noch
zwei Möglichkeiten $\pm b_i$ offen.
Durch die Forderung, dass in der Linearkombination
\[
b_i = \alpha_i a_i + \alpha_{i-1} a_{i-1} + \dots + \alpha_1 a_1
\]
der Koeffiziente $\alpha_i >0$ sein muss.

Der Vektor $b_1$ muss also ein positives Vielfaches von $a_1$ sein:
\begin{align}
b_1
&=
\frac{a_1}{\|a_1\|}.
\notag
\intertext{Die folgenden Vektoren müssen um Vielfache der bereits
gefundenen Vektoren $b_k$ mit $k<i$ korrigiert werden, damit das
Skalarprodukt verschwindet.
Ausserdem müssen Sie normiert werden, was mit der Formel}
b_i
&=
\frac{
a_i - \langle b_{i-1},a_i\rangle b_{i-1} - \langle b_{i-2},a_i\rangle b_{i-2}
- \dots - \langle b_1,a_i\rangle b_1
}{
\|
a_i - \langle b_{i-1},a_i\rangle b_{i-1} - \langle b_{i-2},a_i\rangle b_{i-2}
- \dots - \langle b_1,a_i\rangle b_1
\|
}.
\label{buch:orthofkt:gs}
\end{align}
Der Koeffizient von $a_i$ in \eqref{buch:orthofkt:gs} ist
\begin{equation}
\frac{1}{
\|
a_i - \langle b_{i-1},a_i\rangle b_{i-1} - \langle b_{i-2},a_i\rangle b_{i-2}
- \dots - \langle b_1,a_i\rangle b_1
\|
}
> 0,
\label{buch:orthfkt:gsnenner}
\end{equation}
wie gefordert.

Der Nenner von \eqref{buch:orthfkt:gsnenner} kann nicht verschwinden.
Wäre der Nenner $=0$, dann könnte
\[
a_i
=
\langle b_{i-1},a_i\rangle b_{i-1} + \langle b_{i-2},a_i\rangle b_{i-2}
+ \dots + \langle b_1,a_i\rangle b_1
\]
als Linearkombination der Vektoren $b_1,\dots,b_{i-1}$ geschrieben
werden.
Diese liegen aber alle im von $a_1,\dots,a_{i-1}$ aufgespannten Raum,
es müsste also auch $a_i$ in diesem Raum liegen.
Dies widerspricht der Voraussetzung, dass die Vektoren eine Basis bilden.

%
% Hilbertbasis eines Prähilbertraumes
%
\subsubsection{Hilbertbasis eines Prähilbertraumes}
Das Orthogonalisierungsverfahren von Gram-Schmidt kann dazu verwendet
werden, eine Hilbertbasis in einem Hilbertraum $H$ zu konstruieren.
Ausgangspunkt ist wieder eine Folge von Vektoren $a_i\in H$.
Damit der Nenner von \eqref{buch:orthfkt:gsnenner} nicht verschwindet,
muss verlangt werden, das $a_i$ nicht im Raum aufgespannt von den
Vektoren $a_1,\dots,a_{i-1}$ liegen darf.
Dies garantiert, dass \eqref{buch:orthofkt:gs} eine Folge $b_i$ hervorbringt,
mit der Eigenschaft, dass jeder Vektor $a_i$ sich eindeutig durch die
$b_i$ ausdrücken lässt.
Wenn sich jeder Vektor in $H$ beliebig gut durch eine Folge von
Linearkombinationen der $a_i$ approximieren lässt, dann gilt dies auch
für die $b_i$.
Somit produziert das Orthogonalisierungsverfahren von Gram-Schmidt
automatisch eine Hilbertbasis.

%
% Legendre Polynome
%
\subsection{Legendre-Polynome}
Wir betrachten stetige Funktionen in $C([-1,1])$ auf dem Intervall
$[-1,1]$ mit reellen Werten und dem Skalarprodukt
\[
\langle f,g\rangle
=
\int_{-1}^1 f(x)g(x)\,dx.
\]
$C([-1,1])$ ist ein Prähilbertraum.

Um mit dem Verfahren von Gram-Schmidt eine Familie von orthogonalen
Funktionen in $C([-1,1])$ zu produzieren, wird zunächst eine Familie
von nicht notwendigerweise orthogonalen Funktionen benötigt, mit der
sich jede Funktion in $C([-1,1])$ beliebig genau approximieren
lässt.
Dies ist gemäss dem folgenden Satz mit Polnomen möglich.

\begin{satz}[Weierstrass]
Jeder stetige Funktion auf dem Intervall $[a,b]$ lässt sich beliebig
genau durch Polynome approximieren: ist $f\colon [a,b]\to\mathbb{R}$
eine stetige Funktion, dann gibt es Polynome $p_n(x)$, die gleichmässig
gegen $f(x)$ konvergieren, d.~h.~\(
\|p_n-f\| \to 0
\)
für $n\to\infty$.
\end{satz}

Da die Polynome gleichmässig gegen $f(x)$ konvergieren, folgt für die
$L^2$-Norm von $L^2([a,b])$
\[
\| f-p_n \|_2^2
=
\int_a^b |f(x)-p_n(x)|^2\,dx
\le 
\int_a^b \varepsilon^2 \,dx
=
\varepsilon |b-a|.
\]
Die Folge der approximierenden Polynome $p_n(x)$ konvergiert also
auch in der $L^2$-Norm.

Dies zeigt, dass als Ausgangsfunktionen für den Gram-Schmidt-Prozess
die Monome
\begin{equation}
a_0(x)=1=x^0,\;
a_1(x)=x,\;
a_2(x)=x^2,\;
a_3(x)=x^3,\;
\ldots\;
a_n(x)=x^n,\;
a_{n+1}(x)=x^{n+1},\;
\label{buch:orthfkt:monome}
\end{equation}
verwendet werden können, die alle linear unabhängig sind.
Der von den Funktionen $a_0,\dots,a_n$ erzeugte Unterraum von
$C([a,b])$ besteht aus den Polynomen vom Grad $\le n$.
Die Vektoren $b_n$ sind daher ebenfalls Vektoren vom Grad $n$.

Wir führen jetzt den Gram-Schmidt-Prozess für die Monome
\eqref{buch:orthfkt:monome}
und das $L^2$-Skalarprodukt auf dem Intervall $[-1,1]$ durch.
Wir bezeichnen das Polynome vom Grad $n$, welches erzeugt wird mit $p_n(x)$.

Die Funktion $p_0$ hat die Norm
\[
\|a_0\|^2
=
\int_{-1}^1 a_0(x)^2\,dx
=
\int_{-1}^1 \,dx
=
2
\qquad\Rightarrow\qquad
\|a_0\| = \sqrt{2}.
\]
Somit ist
\[
p_0(x) = \frac{a_0(x)}{\|a_0\|} = \frac{1}{\sqrt{2}}.
\]

Der Vektor $b_1$ wird berechnet durch
\[
b_1
=
\frac{
a_1-\langle b_0,a_1\rangle b_0
}{
\|
a_1-\langle b_0,a_1\rangle b_0
\|
},
\]
wir müssen daher zunächst das Skalarprodukt
\[
\langle b_0,a_1\rangle
=
\int_{-1}^1 b_0(x)a_1(x)\,dx
=
\int_{-1}^1 \frac{1}{\sqrt{2}} x\,dx
=
\frac{1}{\sqrt{2}} \biggl[\frac12x^2\biggr]_{-1}^1=0
\]
berechnen.
Es folgt dass $a_1$ bereits orthogonal zu $b_0$ ist, er muss also
nur noch normiert werden.
Dazu müssen wir die Norm
\[
\| a_1\|^2
=
\int_{-1}^1 a_1(x)^2\,dx
=
\int_{-1}^1 x^2\,dx
=
\biggl[\frac13x^3\biggr]_{-1}^1
=
\frac{2}{3}
\qquad\Rightarrow\qquad
\|a_1\|
=
\sqrt{\frac{2}{3}}
\]
verwenden und bekommen
\[
p_1(x) = \sqrt{\frac{3}{2}} x.
\]

Die dritte Funktion $p_2(x)$ entsteht durch Orthonormierung des
Monoms $x^2$, dazu sind zunächst die Skalarprodukte
\begin{align*}
\langle b_0,a_2\rangle
&=
\int_{-1}^1 \frac{1}{\sqrt{2}}x^2\,dx
=
\frac{1}{\sqrt{2}} [\frac13x^3]_{-1}^1
=
\frac{1}{\sqrt{2}} \frac{2}{3}
=
\frac{\sqrt{2}}{3},
\\
\langle b_1,a_2\rangle
&=
\int_{-1}^1 \sqrt{\frac{3}{2}} x \cdot x^2\,dx
= 
0.
\end{align*}
Die Funktion $p_2(x)$ wird daher erhalten durch Normierung des Polynoms
\[
x^2 - 
\frac{\sqrt{2}}{3} b_0(x)
=
x^2-
\frac{\sqrt{2}}{3}\cdot\frac{1}{\sqrt{2}}
=
x^2-\frac13.
\]
Dazu muss die Norm von $x^2-\frac13$ bestimmt werden, sie ist
\begin{align*}
\|x^2-\frac13\|^2
&=
\int_{-1}^1 \biggl(x^2-\frac13\biggr)^2\,dx
=
\int_{-1}^1 x^4-\frac23x^2+\frac{4}{9}\,dx
\\
&=
\biggl[\frac{x^5}{5} -\frac{2x^3}{9}+\frac{4x}{9}\biggr]_{-1}^1
=
\frac{2}{5} - \frac{4}{9} + \frac{8}{9}
=
\frac{18-20+40}{45}
=
\frac{38}{45}
\\
\Rightarrow\qquad
\|x^2-\frac13\|
&=
\sqrt{\frac{45}{38}}
\end{align*}
und somit
\[
p_2(x)
= 
\sqrt{\frac{38}{45}} \biggl(x^2-\frac13\biggr).
\]

Dieser Prozess lässt sich beliebig weiterführen, die Rechnung wird
allerdings immer aufwendiger.
Ein praktischeres Verfahren ist dringend nötig und wird sich
aus der Sturm-Liouville-Theorie von
Abschnitt~\ref{buch:orthofkt:section:sturmliouville} ergeben.
Etwas vereinfacht werden kann er durch die Beobachtung, dass die
Polynome $p_{2n}(x)$ immer gerade sind und die Polynome $p_{2n+1}(x)$
immer ungerade, so dass die Skalarprodukte 
$\langle p_{2k},p_{2l+1}\rangle=0$ sind, weil das Produkt
$p_{2k}(x)p_{2l+1}(x)$ eine ungerade Funktion ist und daher auf
dem symmetrischen Intervall $[-1,1]$ verschwindendes Integral hat.

Weiter sind in den Polynomen $p_n(x)$ die Normierungsfaktoren mit
ihren komplizierten Wurzeln eher unangenehm.
Da zum Beispiel in der Gauss-Quadratur als Anwendung der Polynome
nur die Nullstellen von Interesse sind, werden sie oft in einer
anderen Normierung gemäss der folgenden Definition tabelliert.

\begin{definition}
Die {\em Legendre-Polynome} $P_n(x)$ sind bezüglich des $L^2$-Skalarproduktes 
auf $[-1,1]$ orthogonale Polynome mit positivem Leitkoeffizienten und
$P_n(1)=1$.
\end{definition}

Aus obiger Konstruktion ist klar, dass $P_n(x) = p_n(x)/p_n(1)$.
Für die oben ausgerechneten ersten drei Polynome finden wir daher die
Legendre-Polynome
\begin{align*}
P_0(x) &= 1 \\
P_1(x) &= x \\
P_2(x) &= \frac12(3x^2-1).
\end{align*}


%
% Andere Skalarprodukte
%
\subsection{Andere Skalarprodukte}
Das gewöhnliche $L^2$-Skalarprodukt ist nur eines von vielen möglichen
Skalarprodukten für Funktionen auf dem Intervall $(-1,1)$.
Jeder Wahl eines Skalarproduktes führt auf einen andere Funktionenfamilie.

\subsubsection{Tschebyscheff-Polynome}
Das Skalarprodukt
\begin{equation}
\langle f,g\rangle_{\text{T}}
=
\int_{-1}^1
f(x)g(x)\frac{1}{\sqrt{1-x^2}}\, dx
\label{buch:orthfkt:tschebyscheff}
\end{equation}
führt auf eine Familie $t_n(x)$ von orthonormierten Polynome.
Da ähnlich wie bei den Legendre-Polynomen oft nur die Nullstellen
wichtig sind, wird auch hier die Normierung $T_n(x) = t_n(x)/t_n(1)$
verwendet, was die
sogenannten Tschebyscheff-Polynome
\(
T_n(x)
\)
ergibt,
die durch Gleichung
\[
T_n(\cos \vartheta) = \cos n\vartheta
\]
gegeben sind.

Tatsächlich kann man direkt nachrechnen, dass die Polynome $T_n$ und $T_m$
mit $n\ne m$ orthogonal sind.
Dazu berechnen wir
\begin{align*}
\langle T_n,T_m\rangle
&=
\int_{-1}^1 T_n(x) T_m(x)\,dx
\intertext{und verwenden die Substitution $x=\cos\vartheta$ mit
$dx/d\vartheta=-\sin\vartheta$.}
&=
\int_{\pi}^0
T_n(\cos\vartheta) T_m(\cos\vartheta)
\sqrt{1-\cos^2\vartheta}
\, \frac{-d\vartheta}{-\sin\vartheta}
\\
&=
\int_0^{\pi}
\cos n\vartheta \cos m\vartheta \,d\vartheta
\\
&=
\int_0^\pi
\frac12(\cos(n-m)\vartheta + \cos(n+m)\vartheta)
\,d\vartheta
=
\frac12
\biggl[
\frac{\sin(n-m)\vartheta}{n-m}+\frac{\sin(n+m)\vartheta}{n+m}
\biggr]_0^{\pi}
=
0,
\end{align*}
da Sinus für ganzzahlige Vielfache von $\pi$ verschwindet.

%
% Gewichtsfunktionen
%
\subsubsection{Gewichtsfunktionen}
Sei $w\colon(-1,1)\to \mathbb{R}$ eine stetige Funktion mit
nichtnegativen Werten, dann kann das zugehörige Skalarprodukt
\[
\langle f,g\rangle_w
=
\int_{-1}^1 f(x)g(x)\,w(x)\,dx
\]
definiert werden.
Die Funktion $w(x)$ heisst die {\em Gewichtsfunktion} des
Skalarproduktes.
\index{Gewichtsfunktion}%

Das Skalarprodukt 
\eqref{buch:orthfkt:tschebyscheff},
welches auf die Tschebyscheff-Polynome geführt hat, gehört zur 
Gewichtsfunktion
\[
w_T(x)
= 
\frac{1}{\sqrt{1-x^2}}.
\]

\begin{definition}
Die {\em Jacobi-Gewichtsfunktion} $w^{(\alpha,\beta)}(x)$ ist 
\[
w^{(\alpha,\beta)}
\colon
(-1,1) \to \mathbb{R}
:
x\mapsto (1-x)^\alpha(1+x)^\beta.
\]
Das zugehörige Skalarprodukt wird auch
\[
\langle f,g\rangle_{w^{(\alpha,\beta)}}
=
\langle f,g\rangle_{(\alpha,\beta)}
=
\int_{-1}^1 f(x)g(x) w^{(\alpha,\beta)}\,dx
=
\int_{-1}^1 f(x)g(x) (1-x)^\alpha(1+x)^\beta\,dx
\]
geschrieben.
\end{definition}

Die Familie orthogonaler Polynome, die sich zu diesem Skalarprodukt
ergeben, heissen {\em Jacobi-Polynome}.
Die Gewichtsfunktion der Tschebyscheff-Polynome ist ein Spezialfall,
nämlich
\[
w_T(x)
=
\frac{1}{\sqrt{1-x^2}}
=
\frac{1}{\sqrt{(1-x)(1-x)}}
=
(1-x)^{-\frac12}(1+x)^{-\frac12}
=
w^{(-\frac12,-\frac12)}(x).
\]
Die $(-\frac12,-\frac12)$-Jacobi-Polynome sind also die
Tschebyscheff-Polynome.

%
% Hermite-Polynome
%
\subsubsection{Hermite-Polynome}
Auch für unendliche Intervalle lassen sich Gewichtsfunktionen und
Familien von orthogonalen Funktionen angeben.
Da Polynome für $x\to\infty$ beliebig schnell anwachsen können,
muss die Gewichtsfunktion exponentiell schnell gegen $0$ gehen, damit
das Integral beschränkt bleibt.

Für $\mathbb{R}$ als Definitionsbereich der Polynome kann man die
{\em Hermite-Gewichtsfunktion}
\[
w(x) = e^{-\frac{x^2}2}
\]
verwenden, sie führt auf die {\em Hermite-Polynome}.

%
% Laguerre-Polynome
%
\subsubsection{Laguerre-Polynome}
Für den Definitionsbereich $\mathbb{R}_{>0}$ kann die
Laguerre-Gewichtsfunktion
\[
w_{\text{Laguerre}}(x)
=
e^{-x}
\]
verwendet werden, die auf die sogenannten {\em Laguerre-Polynome}
führt.
Die Theorie der Laguerre- und Hermite-Polynome wird ausführlich
in \cite{buch:mathsem-spezfunk} behandelt.

