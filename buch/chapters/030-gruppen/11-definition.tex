%
% 1-gruppe.tex -- Konzept einer Gruppe
%
% (c) 2022 Prof Dr Andreas Müller, OST Ostschweizer Fachhochschule
%

%
% Definition
%
\subsection{Definition
\label{buch:gruppen:subsection:definition}}
Der Begriff einer Gruppe soll alle Arten von invertierbaren Rechenoperationen
erfassen, also Addition/Subtraktion, Multiplikation/Division oder
die Matrixmultiplikation mit der inversen Matrix.
Die minimal nötigen Eigenschaften fasst die folgende Definition zusammen.

\begin{definition}
\label{buch:gruppen:definition:gruppe}
Eine {\em Gruppe} $G$ ist eine Menge mit einer Verknüpfung
$\cdot \colon G\times G\to G : (x,y) \mapsto xy $, welche folgende
Eigenschaften hat:
\begin{enumerate}
\item
Die Verknüpfung ist assoziativ, d.~h.~$(xy)z=x(yz)$ für alle
$x,y,z\in G$.
\item
Es gibt ein {\em neutrales Element} $e\in G$, für welches $ex=x$ für alle
$x\in G$ gilt.
\index{neutrales Element}%
\item 
Zu jedem $x\in G$ gibt es ein {\em inverses Element} $x^{-1}\in G$ mit der
Eigenschaft $x^{-1}x=e$.
\index{inverses Element}%
\end{enumerate}
\end{definition}

Man beachte, dass die Defiinition nicht verlangt, dass die Faktoren
vertauscht werden können. 

\begin{beispiel}
Die Menge $\operatorname{GL}_n(\mathbb{R})$ der invertierbaren
$n\times n$-Matrizien mit reellen Einträgen und der Matrixmultiplikation
heisst die {\em allgemeine lineare Gruppe}.
\index{allgemeine lineare Gruppe}%
\index{Gruppe!allgemeine lineare}%
Das neutrale Element von $\operatorname{GL}_n(\mathbb{R})$ ist die
Einheitsmatrix, das inverse Element einer Matrix
$A\in \operatorname{GL}_n(\mathbb{R})$
ist die inverse Matrix $A^{-1}$.
\end{beispiel}

%
% Abelsche Gruppen
%
\subsubsection{Abelsche Gruppen}
Die Definition einer Gruppe verlangt nicht, dass die Verknüpfung der
Gruppenelemente kommutativ sein müsste.
Tatsächlich haben die meisten Matrizengruppen, wie weiter unten als 
Beispiele besprochen werden, eine nicht kommutative Verknüpfung.
Es stellt sich aber heraus, dass Gruppen mit einer kommutativen
Verknüpfung zu einer besonders reichhaltigen harmonischen Analysis
führen.

\begin{definition}
\label{buch:gruppen:definition:abelsch}
Eine Gruppe $G$ heisst {\em abelsch}, wenn 
die Verknüpfung in der Gruppe kommutativ ist, wenn also
$xy=yx$ für alle $x,y\in G$ gilt.
\end{definition}

Die Verknüpfung einer abelschen Gruppe ist also {\em kommutativ}.
\index{kommutativ}%
Abelsche Gruppen werden oft additiv geschrieben, d.~h.~mit einem
Pluszeichen als $x+y$ für $x,y\in G$.
Das neutrale Element heisst dann auch das {\em Nullelement} und wird $0$
geschrieben: $x+0=x$ für alle $x\in G$.
Das inverse Element von $x$ heisst dann auch das
{\em entgegengesetzte Element} und wird $-x$ geschrieben. 
Es gilt $x+(-x)=0$ für alle $x\in G$.

\begin{beispiel}
Die Gruppe $(\mathbb{R},+)$ ist eine abelsche Gruppe.
Die Gruppe der von $0$ verschiedenen Zahlen mit der Multiplikation
$(\mathbb{R}^*,\cdot)$ ist eine abelsche Gruppe.
\end{beispiel}

\begin{beispiel}
Die {\em Gruppe der Drehwinkel} in der Ebene ist die Menge
\(
\mathbb{R}/2\pi\mathbb{Z}
=
(-\pi,\pi].
\)
Als Gruppenoperation dient die Addition von Winkeln, die Summe
$\alpha+\beta$ zweier Winkel $\alpha$ und $\beta$ muss dazu durch
Subtraktion oder Addition von Vielfachen von $2\pi$ ins Intervall
$(-\pi,\pi]$ zurückgebracht werden.
Neutrales Element ist $0$, das inverse Elemente ist $-\alpha$ mit
dem Spezialfall $-\pi=\pi$.
\end{beispiel}

\begin{beispiel}
Die von $0$ verschiedenen Element $\mathbb{C}^*$ mit der Multiplikation
ist eine Gruppe.
Das neutrale Element ist die Zahl $1\in\mathbb{C}^*$ und das inverse
Element zu $z\in\mathbb{C}^*$ ist $z^{-1}$.
\end{beispiel}

%
% Untergruppen
%
\subsubsection{Untergruppen}
Da man Drehungen auch mit Hilfe von komplexen Zahlen beschreiben kann,
kann man die Gruppe der Drehwinkel auch als Menge von komplexen Zahlen
schreiben, nämlich als die Menge
\[
S^1
=
\{z\in\mathbb{C}\mid |z|=1\}
\]
der komplexen Zahlen vom Betrag $1$.
Die Gruppenoperation in $S^1$ ist die gleiche wie die Operation
in $\mathbb{C}^*$, von der $S^1$ eine Teilmenge ist.

\begin{definition}
\label{buch:gruppen:definition:def:untergruppe}
Sei $G$ eine Gruppe und $H\subset G$ eine Teilmenge derart,
dass mit jedem $x,y\in H$ auch $xy$ und $x^{-1}$ in $H$ sind.
Dann heisst $H$ eine {\em Untergruppe} von $G$.
\index{Untergruppe}%
\end{definition}

Jede Gruppe enthält als kleinste Untergruppe immer die Gruppe $G$,
die {\em triviale Gruppe}, die nur aus dem neutralen Element
$\{e\}\subset G$ besteht.

%
% Homomorphismen
%
\subsubsection{Homomorphismen}
Die Exponentialabbildung
\[
\exp
\colon
\mathbb{R}/2\pi\mathbb{Z} \to S^1
:
\alpha \mapsto e^{i\alpha}
\]
bildet Drehwinkel auf komplexe Zahlen vom Betrag $1$ ab.
Die Gruppenoperation bleibt dabei erhalten, es gilt
\[
\exp(\alpha + \beta)
=
e^{i(\alpha+\beta)}
=
e^{i\alpha}
e^{i\beta}
=
\exp(\alpha)
\exp(\beta).
\]
Das neutrale Element $0\in\mathbb{R}/2\pi\mathbb{Z}$ wird auf
das neutrale Element $1\in S^1$ abgebildet und das inverse
Element von $\exp(\alpha)$ ist
$ \exp(\alpha)^{-1} = \exp(-\alpha) $.
Ausserdem ist die Abbildung bijektiv.
Die Exponentialabbildung zeigt also, dass es zwischen den beiden
Gruppen $\mathbb{R}/2\pi\mathbb{Z}$ und $S^1$ nicht wirklich einen
Unterschied gibt.

Die Gruppe der Drehwinkel kann man auch als eine Matrizengruppe
verstehen, wie im folgenden Beispiel gezeigt wird.

\begin{beispiel}
Die Menge
\[
\operatorname{SO}(2)
=
\biggl\{
\begin{pmatrix}
\cos\alpha & -\sin\alpha \\
\sin\alpha &  \cos\alpha
\end{pmatrix}
\;
\bigg|
\;
\alpha\in\mathbb{R}
\biggr\}
\]
ist eine Gruppe mit der Matrixmultiplikation als Gruppenoperation,
der Einheitsmatrix als neutralem Element und der inversen Matrix
als inversem Element.
\end{beispiel}

Die Abbildung
\[
\varphi
\colon
\mathbb{R}/2\pi\mathbb{Z}
\to
\operatorname{SO}(2)
:
\alpha
\mapsto
D_\alpha
=
\begin{pmatrix}
\cos\alpha & -\sin\alpha \\
\sin\alpha &  \cos\alpha
\end{pmatrix}
\]
transportiert die Gruppenoperation von $\mathbb{R}/2\pi\mathbb{Z}$
nach $\operatorname{SO}(2)$ denn es gilt
\[
\varphi(\alpha)\varphi(\beta)
=
\begin{pmatrix}
\cos\alpha & -\sin\alpha \\
\sin\alpha &  \cos\alpha
\end{pmatrix}
\begin{pmatrix}
\cos\beta & -\sin\beta \\
\sin\beta &  \cos\beta
\end{pmatrix}
=
\begin{pmatrix}
\cos(\alpha+\beta) & -\sin(\alpha+\beta) \\
\sin(\alpha+\beta) &  \cos(\alpha+\beta)
\end{pmatrix}
=
\varphi(\alpha+\beta).
\]
Auch ist $\varphi(0)$ die Einheitsmatrix und
$\varphi(-\alpha)=\varphi(\alpha)^{-1}$.

\begin{definition}
\label{buch:gruppen:definition:def:homomorphismus}
Eine Abbildung $\varphi\colon G\to H$ zwischen zwei Gruppen $G$ und $H$
heisst ein {\em Homomorphismus}, wenn
$\varphi(gh)=\varphi(g)\varphi(h)$ gilt für alle $g,h\in G$
\end{definition}

Die Bildmenge $\varphi(G)$ eines Homomorphismus ist automatisch eine
Untergruppe $\varphi(G)\subset H$.
Sind $\varphi(x)$ und $\varphi(y)$ Elemente in $\varphi(G)$,
dann ist auch $\varphi(x)\varphi(y)=\varphi(xy)\in\varphi(G)$.

%
% Der Kern eines Homomorphismus
%
\subsubsection{Der Kern eines Homomorphismus}
Ist $\varphi\colon G\to H$ ein Homomorphismus von Gruppen und
$U\subset H$ eine Untergruppe von $H$, dann bilden die Elemente
\[
\varphi^{-1}(U)
=
\{g\in G\mid \varphi(g)\in U\}
\]
eine Untergruppe.
Sind nämlich $g_1,g_2\in\varphi^{-1}(U)$, dann ist
\[
\varphi(g_1g_2)
=
\varphi(g_1)\varphi(g_2)
\in U,
\]
da $U$ eine Untergruppe ist.
Dann ist aber auch $g_1g_2\in\varphi^{-1}(U)$, was zeigt, dass
$\varphi^{-1}(U)$ eine Untergruppe von $G$ ist.
Sie heisst die {\em Urbildgruppe} von $U$ unter dem Homomorphismus
$\varphi$.

Besonders wichtig ist die Urbildgruppe der trivialen Gruppe.

\begin{definition}
\label{buch:gruppen:definition:def:kern}
Der Kern eines Homomorphismus $\varphi \colon G\to H$ ist die
Untergruppe
\[
\ker \varphi = \varphi^{-1}(\{e\}).
\]
\end{definition}

Der Kern eines Homomorphismus kann dazu verwendet werden zu beurteilen,
ob der Homomorphismus injektiv ist.
Wenn nämlich $\varphi(x)=\varphi(y)$ ist, dann ist auch
\[
e
=
\varphi(x)\varphi(y)^{-1}
=
\varphi(xy^{-1})
\quad\Rightarrow\quad
xy^{-1} \in\ker\varphi.
\]
Es folgt also genau dann $x=y$, wenn der Kern $\ker\varphi$ nur
das neutrale Element enthält.

