%
% 1-gruppe.tex -- Konzept einer Gruppe
%
% (c) 2022 Prof Dr Andreas Müller, OST Ostschweizer Fachhochschule
%

%
% Definition
%
\subsection{Definition
\label{buch:gruppen:subsection:definition}}
Der Begriff einer Gruppe soll alle Arten von invertierbaren Rechenoperationen
erfassen, also Addition/Subtraktion, Multiplikation/Division oder
die Matrixmultiplikation mit der inversen Matrix.
Die minimal nötigen Eigenschaften fasst die folgende Definition zusammen.

\begin{definition}
\label{buch:gruppen:definition:gruppe}
Eine {\em Gruppe} $G$ ist eine Menge mit einer Verknüpfung
\index{Gruppe}%
$\cdot \colon G\times G\to G : (x,y) \mapsto xy $, welche folgende
Eigenschaften hat:
\begin{enumerate}
\item
Die Verknüpfung ist assoziativ, d.~h.~$(xy)z=x(yz)$ für alle
$x,y,z\in G$.
\index{assoziativ}%
\item
Es gibt ein {\em neutrales Element} $e\in G$, für welches $ex=x$ für alle
\index{neutrales}%
$x\in G$ gilt.
\index{neutrales Element}%
\item 
Zu jedem $x\in G$ gibt es ein {\em inverses Element} $x^{-1}\in G$ mit der
Eigenschaft $x^{-1}x=e$.
\index{inverses Element}%
\end{enumerate}
\end{definition}

Man beachte, dass die Definition nicht verlangt, dass die Faktoren
vertauscht werden können. 

\begin{beispiel}
Die Menge $\operatorname{GL}_n(\mathbb{R})$ der invertierbaren
$n\times n$-Matrizen mit reellen Einträgen und der Matrixmultiplikation
ist eine Gruppe, sie heisst die {\em allgemeine lineare Gruppe}.
\index{allgemeine lineare Gruppe}%
\index{Gruppe!allgemeine lineare}%
Das neutrale Element von $\operatorname{GL}_n(\mathbb{R})$ ist die
Einheitsmatrix, das inverse Element einer Matrix
$A\in \operatorname{GL}_n(\mathbb{R})$
ist die inverse Matrix $A^{-1}$.
\end{beispiel}

%
% Abelsche Gruppen
%
\subsubsection{Abelsche Gruppen}
Die Definition einer Gruppe verlangt nicht, dass die Verknüpfung der
Gruppenelemente kommutativ sein müsste.
Tatsächlich haben die meisten Matrizengruppen, wie weiter unten als 
Beispiele besprochen werden, eine nicht kommutative Verknüpfung.
Es stellt sich aber heraus, dass Gruppen mit einer kommutativen
Verknüpfung zu einer besonders reichhaltigen harmonischen Analysis
führen.

\begin{definition}
\label{buch:gruppen:definition:abelsch}
Eine Gruppe $G$ heisst {\em abelsch}, wenn 
\index{abelsch}%
die Verknüpfung in der Gruppe kommutativ ist, wenn also
$xy=yx$ für alle $x,y\in G$ gilt.
\end{definition}

Die Verknüpfung einer abelschen Gruppe ist somit {\em kommutativ}.
\index{kommutativ}%
Abelsche Gruppen werden oft additiv geschrieben, d.~h.~mit einem
Pluszeichen als $x+y$ für die Verknüpfung zweier Elemente $x,y\in G$.
Das neutrale Element heisst dann auch das {\em Nullelement} und wird $0$
\index{Nullelement}%
geschrieben: $x+0=x$ für alle $x\in G$.
Das inverse Element von $x$ heisst dann auch das
{\em entgegengesetzte Element} und wird $-x$ geschrieben. 
\index{entgegengesetztes Element}%
Es gilt $x+(-x)=0$ für alle $x\in G$.

\begin{beispiel}
Die Gruppe $(\mathbb{R},+)$ ist eine abelsche Gruppe.
Die Gruppe der von $0$ verschiedenen Zahlen mit der Multiplikation
$(\mathbb{R}^*,\cdot)$ ist eine abelsche Gruppe.
\end{beispiel}

\begin{beispiel}
Die {\em Gruppe der Drehwinkel} in der Ebene ist die Menge
$
%\mathbb{R}/2\pi\mathbb{Z}
%=
(-\pi,\pi]
$.
Als Gruppenoperation dient die Addition von Winkeln, die Summe
$\alpha+\beta$ zweier Winkel $\alpha$ und $\beta$ muss dazu durch
Subtraktion oder Addition von Vielfachen von $2\pi$ ins Intervall
$(-\pi,\pi]$ zurückgebracht werden.
Neutrales Element ist $0$, das inverse Elemente ist $-\alpha$ mit
dem Spezialfall $-\pi=\pi$.

Eigentlich kann jede Zahl als Drehwinkel verwendet werden, doch
Drehwinkel, die sich um ein ganzzahliges Vielfaches von $2\pi$ unterscheiden,
beschreiben die gleiche Drehung.
Die Darstellung der Drehung mit Winkeln zwischen $-\pi$ und $\pi$ ist nur
eine von vielen Möglichkeiten.
Die Zahlen im Intervall $[0,2\pi)$ sind genauso gut geeignet.

Schliesslich könnte man in der Menge der reellen Zahlen $\mathbb{R}$
alle Winkel zusammenfassen, die die gleiche Drehung beschreiben.
Die Menge aller Winkel, die mit der Drehung um den Winkel $\alpha$
gleichbedeutend sind, ist $\{\alpha + 2k\pi\mid k\in\mathbb{Z}\}$.
Die Menge dieser Klassen wird auch mit $\mathbb{R}/2\pi\mathbb{Z}$
bezeichnet.
Jede Klasse von $\mathbb{R}/2\pi\mathbb{Z}$ kann durch genau einen
Winkel in $(-\pi,\pi]$ oder in $[0,2\pi)$ wiedergegeben werden,
die drei Mengen beschreiben alle die gleiche Gruppe.
\end{beispiel}

\begin{beispiel}
Die von $0$ verschiedenen komplexen Zahlen $\mathbb{C}^*$ bilden mit
der Multiplikation als Verknüpfung eine Gruppe.
Das neutrale Element ist die Zahl $1\in\mathbb{C}^*$ und das inverse
Element zu $z\in\mathbb{C}^*$ ist $z^{-1}$.
\end{beispiel}

%
% Untergruppen
%
\subsubsection{Untergruppen}
Da man Drehungen auch mit Hilfe von komplexen Zahlen beschreiben kann,
kann man die Gruppe der Drehwinkel auch als Menge von komplexen Zahlen
schreiben, nämlich als die Menge
\[
S^1
=
\{z\in\mathbb{C}\mid |z|=1\}
\]
der komplexen Zahlen vom Betrag $1$.
Die Gruppenoperation in $S^1$ ist die gleiche wie die Operation
in $\mathbb{C}^*$, von der $S^1$ eine Teilmenge ist.

\begin{definition}
\label{buch:gruppen:definition:def:untergruppe}
Sei $G$ eine Gruppe und $H\subset G$ eine Teilmenge derart,
dass mit jedem $x,y\in H$ auch $xy$ und $x^{-1}$ in $H$ sind.
Dann heisst $H$ eine {\em Untergruppe} von $G$.
\index{Untergruppe}%
\end{definition}

Jede Gruppe enthält als kleinste Untergruppe immer die Gruppe $\{e\}$,
die {\em triviale Gruppe}, die nur aus dem neutralen Element
$\{e\}\subset G$ besteht.

\begin{beispiel}
Die positiven Zahlen $\mathbb{R}^+ = \{x\in \mathbb{R}\mid x>0\}$ 
bilden ein Untergruppe der nichtverschwindenden reellen Zahlen
$\mathbb{R}^*$ mit der Multiplikation als Verknüpfung:
$\mathbb{R}^+\subset\mathbb{R}^*$.
\end{beispiel}

\begin{beispiel}
\label{buch:gruppen:definition:bsp:so2}
Die Menge
\[
\operatorname{SO}(2)
=
\biggl\{
\begin{pmatrix}
\cos\alpha & -\sin\alpha \\
\sin\alpha &  \cos\alpha
\end{pmatrix}
\;
\bigg|
\;
\alpha\in\mathbb{R}
\biggr\}
\]
ist eine Gruppe mit der Matrixmultiplikation als Gruppenoperation,
der Einheitsmatrix als neutralem Element und der inversen Matrix
als inversem Element.
Sie ist eine Untergruppe der Gruppe $\operatorname{GL}_2(\mathbb{R})$
der invertierbaren $2\times 2$-Matrizen.
\end{beispiel}

%
% Homomorphismen
%
\subsubsection{Homomorphismen}
Die Exponentialabbildung
\[
\exp
\colon
\mathbb{R}/2\pi\mathbb{Z} \to S^1
:
\alpha \mapsto e^{i\alpha}
\]
bildet Drehwinkel auf komplexe Zahlen vom Betrag $1$ ab.
Die Gruppenoperation bleibt dabei erhalten, es gilt
\[
\exp(\alpha + \beta)
=
e^{i(\alpha+\beta)}
=
e^{i\alpha}
e^{i\beta}
=
\exp(\alpha)
\exp(\beta).
\]
Das neutrale Element $0\in\mathbb{R}/2\pi\mathbb{Z}$ wird auf
das neutrale Element $1\in S^1$ abgebildet und das inverse
Element von $\exp(\alpha)$ ist
$ \exp(\alpha)^{-1} = \exp(-\alpha) $.
Ausserdem ist die Abbildung bijektiv.
Die Exponentialabbildung zeigt also, dass es zwischen den beiden
Gruppen $\mathbb{R}/2\pi\mathbb{Z}$ und $S^1$ nicht wirklich einen
Unterschied gibt, wie im folgenden Beispiel gezeigt wird.

Die Gruppe der Drehwinkel kann man auch als eine Matrizengruppe
verstehen.
Dazu betrachten wir die Abbildung
\[
\varphi
\colon
\mathbb{R}/2\pi\mathbb{Z}
\to
\operatorname{SO}(2)
:
\alpha
\mapsto
D_\alpha
=
\begin{pmatrix}
\cos\alpha & -\sin\alpha \\
\sin\alpha &  \cos\alpha
\end{pmatrix}
\]
von $\mathbb{R}/2\pi\mathbb{Z}$ in die Gruppe der Drehmatrizen der
Ebene aus Beispiel~\ref{buch:gruppen:definition:bsp:so2}.
Die Abbildung transportiert die Gruppenoperation von
$\mathbb{R}/2\pi\mathbb{Z}$ nach $\operatorname{SO}(2)$ denn es gilt
\[
\varphi(\alpha)\varphi(\beta)
=
\begin{pmatrix}
\cos\alpha & -\sin\alpha \\
\sin\alpha &  \cos\alpha
\end{pmatrix}
\begin{pmatrix}
\cos\beta & -\sin\beta \\
\sin\beta &  \cos\beta
\end{pmatrix}
=
\begin{pmatrix}
\cos(\alpha+\beta) & -\sin(\alpha+\beta) \\
\sin(\alpha+\beta) &  \cos(\alpha+\beta)
\end{pmatrix}
=
\varphi(\alpha+\beta).
\]
Auch ist $\varphi(0)$ die Einheitsmatrix und
$\varphi(-\alpha)=\varphi(\alpha)^{-1}$.

\begin{definition}
\label{buch:gruppen:definition:def:homomorphismus}
Eine Abbildung $\varphi\colon G\to H$ zwischen zwei Gruppen $G$ und $H$
heisst ein {\em Homomorphismus}, wenn
$\varphi(gh)=\varphi(g)\varphi(h)$ gilt für alle $g,h\in G$
\end{definition}

Die Bildmenge $\varphi(G)$ eines Homomorphismus ist automatisch eine
Untergruppe $\varphi(G)\subset H$.
Sind $\varphi(x)$ und $\varphi(y)$ Elemente in $\varphi(G)$,
dann ist auch $\varphi(x)\varphi(y)=\varphi(xy)\in\varphi(G)$.

%
% Der Kern eines Homomorphismus
%
\subsubsection{Der Kern eines Homomorphismus}
Ist $\varphi\colon G\to H$ ein Homomorphismus von Gruppen und
$U\subset H$ eine Untergruppe von $H$, dann bilden die Elemente
\[
\varphi^{-1}(U)
=
\{g\in G\mid \varphi(g)\in U\}
\]
eine Untergruppe.
Sind nämlich $g_1,g_2\in\varphi^{-1}(U)$, dann ist
\[
\varphi(g_1g_2)
=
\varphi(g_1)\varphi(g_2)
\in U,
\]
da $U$ eine Untergruppe ist.
Dann ist aber auch $g_1g_2\in\varphi^{-1}(U)$, was zeigt, dass
$\varphi^{-1}(U)$ eine Untergruppe von $G$ ist.
Sie heisst die {\em Urbildgruppe} von $U$ unter dem Homomorphismus
$\varphi$.

Besonders wichtig ist die Urbildgruppe der trivialen Gruppe,
sie ist der Gegenstand der folgenden Definition.

\begin{definition}
\label{buch:gruppen:definition:def:kern}
Der Kern eines Homomorphismus $\varphi \colon G\to H$ ist die
Untergruppe
\[
\ker \varphi = \varphi^{-1}(\{e\}).
\]
\end{definition}

Der Kern eines Homomorphismus kann dazu verwendet werden, zu beurteilen,
ob der Homomorphismus injektiv ist.
Wenn nämlich $\varphi(x)=\varphi(y)$ ist, dann ist auch
\[
e
=
\varphi(x)\varphi(y)^{-1}
=
\varphi(xy^{-1})
\quad\Rightarrow\quad
xy^{-1} \in\ker\varphi.
\]
Es folgt also genau dann $x=y$, wenn der Kern $\ker\varphi$ nur
das neutrale Element enthält.
Wir fassen das Resultat im folgenden Satz zusammen.

\begin{satz}
Ein Homomorphismus $\varphi\colon G\to H$ ist genau dann injektiv,
wenn $\ker\varphi=\{e\}$ ist.
\end{satz}

%
% Die Menge aller Homomorphismen
%
\subsubsection{Die Menge aller Homomorphismen}
Homomorphismen dienen dazu, Gruppen miteinander zu vergleichen.
Die Menge der Homomorphismen zwischen zwei Gruppen kann daher
Licht auf die Struktur der einzelnen Gruppen werfen.

\begin{definition}
Sind $G$ und $H$ Gruppen, dann wird die Menge aller Homomorphismen 
von $G$ nach $H$ mit $\operatorname{Hom}(G,H)$ bezeichnet.
\end{definition}

Für beliebige Gruppen hat die Menge $\operatorname{Hom}(G,H)$
keine offensichtliche algebraische Struktur.
Im Spezialfalls $H=\mathbb{C}^*$ ist dies anders, wie das folgende
Beispiel zeigt.

\begin{beispiel}
Von besonderer Bedeutung werden später die Homomorphismen
$\operatorname{Hom}(G,\mathbb{C}^*)$ 
von einer Gruppe $G$ in die multiplikative Gruppe $\mathbb{C}^*$
sein.
Diese Menge ist sogar eine multiplikative Gruppe, denn für  zwei
Homomorphismen 
$f_k\colon G\to\mathbb{C}^*$ 
ist die Abbildung
\[
f_1f_2
\colon
G \to \mathbb{C}^*
:
g
\mapsto
f_1(g)f_2(g)
\]
sogar ein Homomorphismus.
Dies folgt aus der Rechnung
\[
(f_1f_2)(gh)
=
f_1(gh)f_2(gh)
=
f_1(g)f_1(h)f_2(g)f_1(h)
=
f_1(g) f_2(g) f_1(h) f_1(h)
=
(f_1f_2)(g)(f_1f_2)(gh).
\]
Das neutrale Element ist der konstante Homomorphismus
$1\colon G\to\mathbb{C}^*: g\mapsto 1$.
Das inverse Element eines Homomorphismus $f\colon G \to\mathbb{C}^*$
ist $1/f$ mit den Werten $(1/f)(g)=1/f(g)$.
Diese Argumentation benötigt nur, dass die Bildgruppe abelsch ist.
\end{beispiel}

Aus einem Homomorphismus $\psi \colon H_1\to H_2$ kann die Abbildung
$\psi_*\colon \operatorname{Hom}(G,H_1)\to\operatorname{Hom}(G,H_2)$ konstruiert
werden, die den Homomorphismus $f\colon G\to H_1$ auf
$\psi\circ f\colon G\to H_2$ abbildet.
Diese Abbildung wird auch mit $\psi_*=\operatorname{Hom}(G,\psi)$ 
bezeichnet.
Ebenso lässt sich zu einem Homomorphismus $\varphi\colon G_1\to G_2$ ein
Abbildung
$
\varphi^*
=
\operatorname{Hom}(\varphi,H)
\colon
\operatorname{Hom}(G_2,H)
\to
\operatorname{Hom}(G_1,H)
$
konstruieren, die den Homomorphismus $f\colon G_2\to H$ auf
den Homomorphismus $f\circ \varphi\colon G_1\to H$ abbildet.
Man beachte, dass die Abbildung $\varphi^*$ in umgekehrter Richtung
erfolgt.

\begin{definition}
Ein Homomorphismus von einer Gruppe $G$ nach $G$ heisst {\em Endomorphismus}
der Gruppe $G$.
In invertierbarer Endomorphismus heisst {\em Automorphismus}.
Die Menge aller Automorphismen wird 
$\operatorname{Aut}(G)$ geschrieben.
\end{definition}

\begin{beispiel}
Die Abbildung
\[
\varphi_a\colon \mathbb{R}\to\mathbb{R}
: x\mapsto a x
\]
mit $a\in\mathbb{R}^*$ ist ein Homomorphismus, denn
\begin{align*}
\varphi_a(x+y)
&=
a(x+y) = a x + a y = \varphi_a(x) + \varphi_a(y)
\\
\varphi_a(\lambda x)
&=
a(\lambda x)
=
\lambda (ax)
=
\lambda
\varphi_a(x).
\end{align*}
Dies sind natürlich das Distributivgesetz in $\mathbb{R}$ und das
Assoziativgesetz für die Multiplikation in $\mathbb{R}$.
$\varphi_a$ ist ein Endomorphismus von $\mathbb{R}$.

Der Homomorphismus $\varphi_a$ hat $\varphi_{a^{-1}}$ als Inverse,
er ist also auch umkehrbar.
Somit ist $\varphi_a$ auch ein Automorphismus.
\end{beispiel}

\begin{beispiel}
Für $a\in \mathbb{Z}\setminus\{0\}$ ist die Abbildung
\[
\varphi_a\colon \mathbb{Z}\to\mathbb{Z} : z\mapsto az
\]
ein Endomorphismus der additiven Gruppe $\mathbb{Z}$.
Für $k\ne \pm 1$ ist $\varphi_a$ aber nicht umkehrbar und damit
kein Automorphismus von $\mathbb{Z}$.
\end{beispiel}

\begin{beispiel}
Für $a>0$ ist die Abbildung
\[
\psi_a
\colon
\mathbb{R}^+ \to \mathbb{R}^+
:
x \mapsto x^a
\]
ein Automorphismus der multiplikativen Gruppe der positiven
reellen Zahlen.
\end{beispiel}

\begin{satz}
Die Menge $\operatorname{Aut}(G)$ ist eine Gruppe mit der Zusammensetzung
von Automorphismen als Gruppenoperation und der Umkehrabbildung als
inverses Element.
\end{satz}

%
% Kategorientheorie
%
\subsubsection{Kategorientheorie}
Die Konstruktion der Automorphismengruppe veranschaulicht ein allgemeineres
Prinzip, wie Gruppen entstehen.
Die Kategorientheorie behandelt die allgemeinen Eigenschaften von
mathematischen Strukturen und Abbildungen dazwischen, die mit diesen
Strukturen verträglich sind.
Die Kategorie der Gruppen zum Beispiel umfasst alle Gruppen und die
Menge der Homomorphismen von Gruppen.
Die konkrete Art der Struktur ist für die Überlegungen der Kategorientheorie
unwichtig.
Für eine interessante Kategorie werden nur die folgenden Folgerungen
gestellt.

\begin{definition}
Eine {\em Kategorie} $\mathscr{C}$ ist eine Klasse von Objekten
$\operatorname{obj}(\mathscr{C})$ zusammen mit einer
Menge von sogenannten {\em Morphismen} $\operatorname{Hom}(A,B)$ für
jedes Paar von Objekten $A$ und $B$, mit einer Verkknüpfung
$
\operatorname{Hom}(B,C)\times \operatorname{Hom}(A,B)
\to
\operatorname{Hom}(A,C)
$, die auch als Zusammensetzung $f\circ g$ für $f\in \operatorname{Hom}(B,C)$
und $g\in \operatorname{Hom}(A,B)$ geschrieben wird.
Ausserdem gibt es in $\operatorname{Hom}(A,A)$ das Element
$1_A\in\operatorname{Hom}(A,A)$ mit den Eigenschaften
\[
f\circ 1_A = f\quad\forall f\in \operatorname{Hom}(A,B)
\qquad\text{und}\qquad
1_B\circ g = g\quad\forall g\in\operatorname{Hom}(A,B).
\]
Die Menge der Morphismen von $A$ nach $A$ wird auch mit
$\operatorname{End}(A)=\operatorname{Hom}(A,A)$ bezeichnet.
Falls Objekte $A$ und $B$ in mehreren Kategorien sind, kann bei der
Spezifikation der Morphismenmenge die Kategorie als Index
angebeben werden: $\operatorname{Hom}_{\mathscr{C}}(A,B)$.
\end{definition}

\begin{beispiel}
Die reellen Vektorräume als Objekte mit den Mengen der linearen
Abbildungen zwischen den Vektorräumen als Morphismenmengen bilden
eine Kategorie $\mathscr{V}$.
Für einen Vektorraum $V$ ist die identische Abbildung das Element
$1_V\in\operatorname{Hom}(V,V)$.
\end{beispiel}

\begin{beispiel}
Die endlichen Mengen als Objekte mit den Abbildungen zwischen
endlichen Mengen als Morphismen bilden die Kategorie $\mathscr{E}$
der endlichen Mengen.
Für eine endliche Menge $A$ ist $1_A=\operatorname{id}_A$ die
identische Abbildung von $A$ auf sich.
\end{beispiel}

Die Kategorientheorie sagt nicht, dass $\operatorname{Hom}(A,B)$
eine Menge von Abbildungen ist, wenngleich dies ein sehr häufiger
Fall ist.

\begin{beispiel}
Aus einem Graph mit Knoten $V$ und Kantenmenge $E$ kann eine Kategorie
$\mathscr{G}$ wie folgt konstruiert werden.
Die Objekte von $\mathscr{G}$ sind die Knoten des Graphen.
Die Menge $\operatorname{Hom}(a,b)$ besteht aus allen Pfaden
von $a$ nach $b$ im Graphen.
Für $a=b$ enthält $\operatorname{Hom}(a,a)$ auch noch den Pfad der
Länge $0$, der mit $1_a$ bezeichnet wird.
Die Verknüpfung von Morphismen ist die Verkettung von Pfaden.

Die Morphismenmengen in der Kategorie $\mathscr{G}$ sind ganz offensichtlich
nicht Mengen von Abbildungen.
Da durch die Verkettung von Pfaden diese immer länger werden, ist es
nicht möglich, dass zwei Pfade, deren Länge nicht $0$ ist, nach
Verkettung zu $1_a$ werden.
\end{beispiel}

Die Kategorientheorie hat Werkzeuge entwickelt, mit denen Isomorphismen
zwischen Objekten definiert werden können, ohne dass man die Details der
involvierten Strukturen verstehen muss.
Sie kann daher auch erklären, was es heissen soll, dass jeder 
Morphismus in $\operatorname{Hom}(A,A)$ auf eindeutige Art invertiert
werden kann.
Dies macht die Menge der invertierbaren Morphismen mit der Verknüpfung
von Morphismen zu einer Gruppe mit dem neutralen Elemente $1_A$.
Die Automorphismengruppe einer Gruppe entsteht durch Anwendung dieser
Überlegungen auf die Gruppenkategorie.
Für die Kategorie der endlichdimensionalen Vektorräume
führt diese Konstruktion auf die Gruppe $\operatorname{GL}(V)$ der
invertierbaren linearen Abbildungen $V\to V$.

