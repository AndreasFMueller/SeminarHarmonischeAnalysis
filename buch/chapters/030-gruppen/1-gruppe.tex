%
% 1-gruppe.tex -- Konzept einer Gruppe
%
% (c) 2022 Prof Dr Andreas Müller, OST Ostschweizer Fachhochschule
%
\section{Gruppen
\label{buch:gruppen:section:gruppe}}
\kopfrechts{Gruppe}
Die Translationsinvarianz des Definitionsbereiches $\mathbb{R}$
entsteht aus der Tatsache, dass die Addition einer Zahl
nicht aus $\mathbb{R}$ herausführt und sich auch umkehren lässt.
Sie basiert also darauf, dass es in $\mathbb{R}$ eine umkehrbare
Verknüpfung gibt.
Diese Idee wird von der algebraischen Struktur einer Gruppe eingefangen.

%
% Definition
%
\subsection{Definition
\label{buch:gruppen:subsection:definition}}
Der Begriff einer Gruppe soll alle Arten von invertierbaren Rechenoperationen
erfassen, also Addition/Subtraktion, Multiplikation/Division oder
die Matrixmultiplikation mit der inversen Matrix.
Die minimal nötigen Eigenschaften fasst die folgende Definition zusammen.

\begin{definition}
\label{buch:gruppen:definition:gruppe}
Eine {\em Gruppe} $G$ ist eine Menge mit einer Verknüpfung
$\cdot \colon G\times G\to G : (x,y) \mapsto xy $, welche folgende
Eigenschaften hat:
\begin{enumerate}
\item
Die Verknüpfung ist assoziativ, d.~h.~$(xy)z=x(yz)$ für alle
$x,y,z\in G$.
\item
Es gibt ein {\em neutrales Element} $e\in G$, für welches $ex=x$ für alle
$x\in G$ gilt.
\index{neutrales Element}%
\item 
Zu jedem $x\in G$ gibt es ein {\em inverses Element} $x^{-1}\in G$ mit der
Eigenschaft $x^{-1}x=e$.
\index{inverses Element}%
\end{enumerate}
\end{definition}

Man beachte, dass die Defiinition nicht verlangt, dass die Faktoren
vertauscht werden können. 

\begin{beispiel}
Die Menge $\operatorname{GL}_n(\mathbb{R})$ der invertierbaren
$n\times n$-Matrizien mit reellen Einträgen und der Matrixmultiplikation
heisst die {\em allgemeine lineare Gruppe}.
\index{allgemeine lineare Gruppe}%
\index{Gruppe!allgemeine lineare}%
Das neutrale Element von $\operatorname{GL}_n(\mathbb{R})$ ist die
Einheitsmatrix, das inverse Element einer Matrix
$A\in \operatorname{GL}_n(\mathbb{R})$
ist die inverse Matrix $A^{-1}$.
\end{beispiel}

%
% Abelsche Gruppen
%
\subsubsection{Abelsche Gruppen}
\begin{definition}
\label{buch:gruppen:definition:abelsch}
Eine Gruppe $G$ heisst {\em abelsch}, wenn 
$xy=yx$ für alle $x,y\in G$ gilt.
\end{definition}

Die Verknüpfung einer abelschen Gruppe ist also {\em kommutativ}.
\index{kommutativ}%
Abelsche Gruppen werden oft additiv geschrieben, d.~h.~mit einem
Pluszeichen als $x+y$ für $x,y\in G$.
Das neutrale Element heisst dann auch das {\em Nullelement} und wird $0$
geschrieben: $x+0=x$ für alle $x\in G$.
Das inverse Element von $x$ heisst dann auch das
{\em entgegengesetzte Element} und wird $-x$ geschrieben. 
Es gilt $x+(-x)=0$ für alle $x\in G$.

\begin{beispiel}
Die Gruppe $(\mathbb{R},+)$ ist eine abelsche Gruppe.
Die Gruppe der von $0$ verschiedenen Zahlen mit der Multiplikation
$(\mathbb{R}^*,\cdot)$ ist eine abelsche Gruppe.
\end{beispiel}

\begin{beispiel}
Die {\em Gruppe der Drehwinkel} in der Ebene ist die Menge
\(
\mathbb{R}/2\pi\mathbb{Z}
=
(-\pi,\pi].
\)
Als Gruppenoperation dient die Addition von Winkeln, die Summe
$\alpha+\beta$ zweier Winkel $\alpha$ und $\beta$ muss dazu durch
Subtraktion oder Addition von Vielfachen von $2\pi$ ins Intervall
$(-\pi,\pi]$ zurückgebracht werden.
Neutrales Element ist $0$, das inverse Elemente ist $-\alpha$ mit
dem Spezialfall $-\pi=\pi$.
\end{beispiel}

\begin{beispiel}
Die von $0$ verschiedenen Element $\mathbb{C}^*$ mit der Multiplikation
ist eine Gruppe.
Das neutrale Element ist die Zahl $1\in\mathbb{C}^*$ und das inverse
Element zu $z\in\mathbb{C}^*$ ist $z^{-1}$.
\end{beispiel}

%
% Untergruppen
%
\subsubsection{Untergruppen}
Da man Drehungen auch mit Hilfe von komplexen Zahlen beschreiben kann,
kann man die Gruppe der Drehwinkel auch als Menge von komplexen Zahlen
schreiben, nämlich als die Menge
\[
S^1
=
\{z\in\mathbb{C}\mid |z|=1\}
\]
der komplexen Zahlen vom Betrag $1$.
Die Gruppenoperation in $S^1$ ist die gleiche wie die Operation
in $\mathbb{C}^*$, von der $S^1$ eine Teilmenge ist.

\begin{definition}
\label{buch:gruppen:definition:def:untergruppe}
Sei $G$ eine Gruppe und $H\subset G$ eine Teilmenge derart,
dass mit jedem $x,y\in H$ auch $xy$ und $x^{-1}$ in $H$ sind.
Dann heisst $H$ eine {\em Untergruppe} von $G$.
\index{Untergruppe}%
\end{definition}

Jede Gruppe enthält als kleinste Untergruppe immer die Gruppe $G$,
die {\em triviale Gruppe}, die nur aus dem neutralen Element
$\{e\}\subset G$ besteht.

%
% Homomorphismen
%
\subsubsection{Homomorphismen}
Die Exponentialabbildung
\[
\exp
\colon
\mathbb{R}/2\pi\mathbb{Z} \to S^1
:
\alpha \mapsto e^{i\alpha}
\]
bildet Drehwinkel auf komplexe Zahlen vom Betrag $1$ ab.
Die Gruppenoperation bleibt dabei erhalten, es gilt
\[
\exp(\alpha + \beta)
=
e^{i(\alpha+\beta)}
=
e^{i\alpha}
e^{i\beta}
=
\exp(\alpha)
\exp(\beta).
\]
Das neutrale Element $0\in\mathbb{R}/2\pi\mathbb{Z}$ wird auf
das neutrale Element $1\in S^1$ abgebildet und das inverse
Element von $\exp(\alpha)$ ist
$ \exp(\alpha)^{-1} = \exp(-\alpha) $.
Ausserdem ist die Abbildung bijektiv.
Die Exponentialabbildung zeigt also, dass es zwischen den beiden
Gruppen $\mathbb{R}/2\pi\mathbb{Z}$ und $S^1$ nicht wirklich einen
Unterschied gibt.

Die Gruppe der Drehwinkel kann man auch als eine Matrizengruppe
verstehen, wie im folgenden Beispiel gezeigt wird.

\begin{beispiel}
Die Menge
\[
\operatorname{SO}(2)
=
\biggl\{
\begin{pmatrix}
\cos\alpha & -\sin\alpha \\
\sin\alpha &  \cos\alpha
\end{pmatrix}
\;
\bigg|
\alpha\in\mathbb{R}
\biggr\}
\]
ist eine Gruppe mit der Matrixmultiplikation als Gruppenoperation,
der Einheitsmatrix als neutralem Element und der inversen Matrix
als inversem Element.
\end{beispiel}

Die Abbildung
\[
\varphi
\colon
\mathbb{R}/2\pi\mathbb{Z}
\to
\operatorname{SO}(2)
:
\alpha
\mapsto
D_\alpha
=
\begin{pmatrix}
\cos\alpha & -\sin\alpha \\
\sin\alpha &  \cos\alpha
\end{pmatrix}
\]
transportiert die Gruppenoperation von $\mathbb{R}/2\pi\mathbb{Z}$
nach $\operatorname{SO}(2)$ denn es gilt
\[
\varphi(\alpha)\varphi(\beta)
=
\begin{pmatrix}
\cos\alpha & -\sin\alpha \\
\sin\alpha &  \cos\alpha
\end{pmatrix}
\begin{pmatrix}
\cos\beta & -\sin\beta \\
\sin\beta &  \cos\beta
\end{pmatrix}
=
\begin{pmatrix}
\cos(\alpha+\beta) & -\sin(\alpha+\beta) \\
\sin(\alpha+\beta) &  \cos(\alpha+\beta)
\end{pmatrix}
=
\varphi(\alpha+\beta).
\]
Auch ist $\varphi(0)$ die Einheitsmatrix und
$\varphi(-\alpha)=\varphi(\alpha)^{-1}$.

\begin{definition}
\label{buch:gruppen:definition:def:homomorphismus}
Eine Abbildung $\varphi\colon G\to H$ zwischen zwei Gruppen $G$ und $H$
heisst ein {\em Homomorphismus}, wenn
$\varphi(gh)=\varphi(g)\varphi(h)$ gilt für alle $g,h\in G$
\end{definition}

Die Bildmenge $\varphi(G)$ eines Homomorphismus ist automatisch eine
Untergruppe $\varphi(G)\subset H$.
Sind $\varphi(x)$ und $\varphi(y)$ Elemente in $\varphi(G)$,
dann ist auch $\varphi(x)\varphi(y)=\varphi(xy)\in\varphi(G)$.

%
% Der Kern eines Homomorphismus
%
\subsubsection{Der Kern eines Homomorphismus}
Ist $\varphi\colon G\to H$ ein Homomorphismus von Gruppen und
$U\subset H$ eine Untergruppe von $H$, dann bilden die Elemente
\[
\varphi^{-1}(U)
=
\{g\in G\mid \varphi(g)\in U\}
\]
eine Untergruppe.
Sind nämlich $g_1,g_2\in\varphi^{-1}(U)$, dann ist
\[
\varphi(g_1g_2)
=
\varphi(g_1)\varphi(g_2)
\in U,
\]
da $U$ eine Untergruppe ist.
Dann ist aber auch $g_1g_2\in\varphi^{-1}(U)$, was zeigt, dass
$\varphi^{-1}(U)$ eine Untergruppe von $G$ ist.
Sie heisst die {\em Urbildgruppe} von $U$ unter dem Homomorphismus
$\varphi$.

Besonders wichtig ist die Urbildgruppe der trivialen Gruppe.

\begin{definition}
\label{buch:gruppen:definition:def:kern}
Der Kern eines Homomorphismus $\varphi \colon G\to H$ ist die
Untergruppe
\[
\ker \varphi = \varphi^{-1}(\{e\}).
\]
\end{definition}

Der Kern eines Homomorphismus kann dazu verwendet werden zu beurteilen,
ob der Homomorphismus injektiv ist.
Wenn nämlich $\varphi(x)=\varphi(y)$ ist, dann ist auch
\[
e
=
\varphi(x)\varphi(y)^{-1}
=
\varphi(xy^{-1})
\quad\Rightarrow\quad
xy^{-1} \in\ker\varphi.
\]
Es folgt also genau dann $x=y$, wenn der Kern $\ker\varphi$ nur
das neutrale Element enthält.

%
% Endliche Gruppen
%
\subsection{Endliche Gruppen
\label{buch:gruppen:subsection:endliche-gruppen}}
Während für theoretische Überlegungen die kontinuierliche
Fourier-Transformation auf Gruppen wie $\mathbb{R}$ oder
$\mathbb{R}/2\pi\mathbb{Z}$, ist f
In Ingenieuranwendungen bevorzugt man, mit kontinuierlichen

% XXX Die Gruppen \mathbb{Z}/n\mathbb{Z}
\subsubsection{Die zyklischen Gruppen $\mathbb{Z}/n\mathbb{Z}$}
Für die diskrete Fourier-Analysis besonders wichtig sind die zyklischen
Gruppen.

\begin{definition}
\label{buch:gruppen:endliche-gruppen:def:zyklisch}
Die Gruppe
\[
\mathbb{Z}/n\mathbb{Z}
=
\{0,1,2,\dots,n-1\}
\]
der Reste modulo $n$ mit der Addition von Resten ist eine abelsche
Gruppe.
\end{definition}

Die zyklischen Gruppen können auch als multiplikativ geschriebene
Untergruppen der von $0$ verschiedenen komplexen Zahlen geschrieben
werden.
Dazu verwendet man die Exponentialfunktion:
\[
C_n
=
\{ e^{2\pi ik/n}\mid k=0,1,\dots,n-1\}.
\]
Die Exponentialabbildung
\[
\exp
\colon
\mathbb{Z}/n\mathbb{Z}
\to
C_n
:
k\mapsto e^{2\pi ik/n}
\]
ist ein Homomorphismus, denn
\[
\exp(k+l)
=
e^{2\pi i(k+l)/n}
=
e^{2\pi ik/n}
e^{2\pi il/n}
=
\exp(k)\exp(l).
\]
Die Reste werden auf verschiedene Ecken eines regelmässigen
$n$-Ecks in der komplexen Ebene abgebildet, die Abbildung $\exp$
ist daher auch eine Bijektion.
Die additiv geschriebene Gruppe $\mathbb{Z}/n\mathbb{Z}$ und
die multiplikativ geschriebene Gruppe $C_n$ sind also isomorph.

%
% Die zyklischen Gruppen als Kern
%
\subsubsection{Die zyklischen Gruppen als Kern}
Die Gruppe $C_n$ wurde früher schon als Bild der Gruppe
$\mathbb{R}/2\pi\mathbb{Z}$ unter der Exponentialabbildung 
in $\mathbb{C}^*$ erkannt worden.
Man kann sie aber auch als Kern eines geeignet gewählten Homomorphismus
verstehen.

Die Abbildung
\[
\varphi
\colon
S^1\to S^1
:
z\mapsto z^n
\]
ist ein Homomorphismus, denn es ist ja $\varphi(z_1z_2)=(z_1z_2)^n
= z_1^nz_2^n=\varphi(z_1)\varphi(z_2)$.
Der Kern von $\varphi$ besteht aus den komplexen Zahlen mit der 
Eigenschaft $z^k=1$, das sind genau die Elemente von $C_n$.

%
% Permutationsgruppen
%
\subsubsection{Permutationsgruppen}
Die Menge $[n]=\{1,2,\dots,n\}$ hat $n$ Elemente.
Wir betrachten die Menge aller invertierbaren Abbildungen
$\varphi\colon [n] \to [n]$.
Zwei solche Abbildungen $\varphi$ und $\psi$ können zusammengesetzt
werden, oder sie können invertiert werden $\varphi^{-1}$.
Tatsächlich ist die Menge 
\[
S_n = \{\varphi\colon [n] \to [n]\mid \text{$\varphi$ ist invertierbar} \}
\]
eine Gruppe, sie heisst die {\em Permutationsgruppe von $n$ Elementen}
oder die {\em symmetrische Gruppe}.

Permutationen können besonders effizient als Matrizen mit zwei Zeilen
geschrieben werden.
Eine Abbildung $\varphi\colon [n]\to[n]$ bildet $i\in [n]$ auf $\varphi(u)$
ab, was man als die Matrix
\[
\varphi
=
\begin{pmatrix}
1&2&3&\dots&n\\
\varphi(1)&\varphi(2)&\varphi(3)&\dots&\varphi(n)
\end{pmatrix}
\]
schreiben kann.
Um die Komposition von zwei Abbildungen $\varphi$ und $\psi$ zu bestimmen,
kann man die beiden Matrizen übereinander schreiben und die Spalten der
unteren Matrix so sortieren, dass sie mit den Elementen in der unteren
Zeile der oberen übereinstimmen:
\[
\begin{aligned}
\varphi
&=
\begin{pmatrix}1&2&3&4\\2&3&1&4\end{pmatrix}
\\
\psi
&=
\begin{pmatrix}1&2&3&4\\3&2&4&1\end{pmatrix}
\end{aligned}
\quad\Rightarrow\quad
\psi\circ \varphi
=
\left\{
\begin{array}{c}
\displaystyle\begin{pmatrix}1&2&3&4\\2&3&1&4\end{pmatrix}\\
\displaystyle\begin{pmatrix}1&2&3&4\\3&2&4&1\end{pmatrix}
\end{array}
\right\}
=
\left\{
\begin{array}{c}
\displaystyle\begin{pmatrix}1&2&3&4\\2&3&1&4\end{pmatrix}\\
\displaystyle\begin{pmatrix}2&3&1&4\\2&4&3&1\end{pmatrix}
\end{array}
\right\}
=
\begin{pmatrix}
1&2&3&4\\
2&4&3&1
\end{pmatrix}.
\]

Die inverse Abbildung findet man, indem man die beiden Zeilen vertauscht
und die Spalten so sortiert, dass die Elemente in der ersten Zeile
wieder aufsteigend sind.
Zum Beispiel
\[
\varphi
=
\begin{pmatrix}
1&2&3&4&5\\
1&3&5&2&4
\end{pmatrix}
\quad\Rightarrow\quad
\varphi^{-1}
=
\begin{pmatrix}
1&3&5&2&4\\
1&2&3&4&5
\end{pmatrix}
=
\begin{pmatrix}
1&2&3&4&5\\
1&4&2&5&3
\end{pmatrix}.
\]
Die Zusammensetzung von $\varphi$ und $\varphi^{-1}$ ist
\[
\varphi\circ\varphi^{-1}
=
\left\{
\begin{array}{c}
\displaystyle
\begin{pmatrix}
1&2&3&4&5\\
1&4&2&5&3
\end{pmatrix}
\\
\displaystyle
\begin{pmatrix}
1&2&3&4&5\\
1&3&5&2&4
\end{pmatrix}
\end{array}
\right\}
=
\left\{
\begin{array}{c}
\displaystyle
\begin{pmatrix}
1&2&3&4&5\\
1&4&2&5&3
\end{pmatrix}
\\
\displaystyle
\begin{pmatrix}
1&4&2&5&3\\
1&2&3&4&5
\end{pmatrix}
\end{array}
\right\}
=
\begin{pmatrix}
1&2&3&4&5\\
1&2&3&4&5
\end{pmatrix}
=
e.
\]

% XXX Die Gruppen S_n

%
% Lie-Gruppen
%
\subsection{Lie-Gruppen
\label{buch:gruppen:subsection:lie-gruppen}}
Die endlichen Gruppen unterscheiden sich grundlegend von der Gruppe
der Drehwinkel.
In $\mathbb{R}$, $\mathbb{R}/2\pi\mathbb{Z}$, $S^1$ und $\mathbb{C}^*$
steht nicht nur die aus der Gruppenoperation abgeleitete algebraische
Struktur zur Verfügung.

%
% Topologische Gruppen
%
\subsubsection{Topologische Gruppen}
Vielmehr kann man auch von konvergenten Folgen von Gruppenelementen
sprechen und davon, ob eine Abbildung zwischen diesen Gruppen
stetig ist.
Man nennt eine solche Gruppe eine {\em topologische Gruppe}.
\index{topologische Gruppe}%

\begin{beispiel}
Die Menge
\(
\mathbb{Q}^*
=
\mathbb{Q} \setminus\{0\}
\)
ist eine Gruppe mit der Multiplikation als Gruppenoperation.
\end{beispiel}

Die Gruppe $\mathbb{Q}^*$ ist eine topologische Gruppe.
Als Teilmenge von $\mathbb{Q}$ ist klar, was eine Cauchy-Folge in
$\mathbb{Q}^*$ ist.
Es gibt aber auch Cauchy-Folgen in $\mathbb{Q}^*$, die nicht konvergieren.
Die Folge
\[
x_0=1,\;
x_{n+1} = \frac12\biggl(x_n+\frac{2}{x_n}\biggr),\; n\in\mathbb{N},
\]
ist eine Folge von rationalen Zahlen, die gegen einen Fixpunkt der
Funktion
\[
x\mapsto f(x)=\frac12\biggl(x+\frac{2}{x}\biggr)
\]
konvergiert.
Durch Multiplikation mit $x$ findet man
\[
x=f(x)
\quad\Rightarrow\quad
x^2=\frac12 x^2 + 1
\quad\Rightarrow\quad
\frac12x^2=1
\quad\Rightarrow\quad
x^2=2
\quad\Rightarrow\quad
x=\sqrt{2},
\]
insbesondere ist der Grenzwert nicht in $\mathbb{Q}^*$.

\begin{definition}
Eine topologische Gruppe $G$ ist eine Gruppe $G$, deren Verknüpfungsabbildung
$(x,y)\mapsto xy$ und das inverse Element $x\mapsto x^{-1}$ stetig sind.
Für konvergente Folgen $x_n\to x$ und $y_n\to y$ in $G$ gilt dann
\begin{align*}
\lim_{n\to \infty} x_ny_n &= \lim_{n\to\infty} x_n \lim_{n\to\infty} y_n = xy
\\
\lim_{n\to \infty} x_n^{-1} &= (\lim_{n\to\infty} x_n)^{-1} = x^{-1}
\end{align*}
\end{definition}

%
% 
%
\subsubsection{Ableitung}
Für Funktionen auf der Gruppe $\mathbb{R}$ ist sogar die Ableitung
definiert.
Eine solche lässt sich auch für Funktionen auf den Gruppen 
$S^1$ und $\operatorname{SO}(2)$ definieren, indem man diese Gruppen
mit einer geeigneten Parametrisierung beschreibt.



\begin{definition}
Eine $n$-dimensionale {\em Karte} $\alpha$ für eine offene Menge
$U_\alpha\subset M$ ist eine bijektive Abbildung
$\varphi_\alpha\colon U\to \mathbb{R}^n$.
\end{definition}

Man kann sich die Kartenabbildungen als lokale Koordinatensystem auf der
Menge $M$ vorstellen.
Der Wert $\varphi_\alpha(p)$ der Kartenabbildung $\varphi_\alpha$
eines Punktes $p\in M$ hat als Komponenten die Koordinaten $x_1,\dots,x_n$
dieses Punktes im gewählten Koordinatensystem.

Die Karten sollen der Menge $M$ ein Koordinatensystem geben, mit dem
man Ableitungen von Funktionen definieren kann.
Dazu ist notwendig, dass verschiedene Karten auf die gleiche
Ableitung führen.
Wegen der Kettenregel der Differentialrechnung bedeutet dies, dass
die Koordinatenumrechnung zwischen zwei Karten eine differenzierbare
Abbildung ist.

Seien also
$\varphi_\alpha\colon U_\alpha \to \mathbb{R}^n$
und
$\varphi_\beta\colon U_\beta \to \mathbb{R}^n$
zwei Karten, deren Definitionsbereiche $U_\alpha$ und $U_\beta$ sich
schneiden.
Sie statten also beide die Menge $U_{\alpha\beta}=U_\alpha\cap U_\beta$
mit einem Koordinatensystem aus.
Die Koordinatenwechselabbildung
\[
\varphi_{\beta\alpha}
=
\varphi_\beta
\circ
\varphi_\alpha^{-1}
\colon
\varphi_\alpha(U_\alpha\cap U_\beta)
\to
\varphi_\beta(U_\alpha\cap U_\beta)
\]
ist eine Abbildung zwischen offenen Teilmengen von $\mathbb{R}^n$.
Man sagt, der Kartenwechsel ist differenzierbar, wenn $\varphi_{\beta\alpha}$
differenzierbar ist.
Der Kartenwechsel in der umgekehrten Richtung ist $\varphi_{\alpha\beta}$.

\begin{definition}
Ein {\em differenzierbarer Atlas} von $M$ ist eine Menge von Karten derart,
dass alle Kartenwechselabbildungen differenzierbar sind.
\end{definition}

\begin{definition}
Eine {\em differenzierbare Mannigfaltigkeit} ist eine Menge $M$ mit einem
differenzierbaren Atlas derart, dass jeder Punkt von $M$ im
Definitionsgebiet mindestens einer Karte liegt.
\end{definition}

Eine differenzierbare Mannigfaltigkeit ist also eine Menge, die in
einer Umgebung jedes Punktes mit mindestens einem Koordinatensystem
ausgestattet werden kann auf eine Art, dass die Umrechnung zwischen
verschiedenen Koordinatensystemen immer differenzierbar ist.

\begin{beispiel}
Die reelle Achse $\mathbb{R}$ ist eine differenzierbare Mannigfaltigkeit,
sie lässt sich mit einer einzigen Karte parametrisieren.
\end{beispiel}

\begin{beispiel}
Die Kreislinie $S^1$ in der komplexen Ebene ist eine differenzierbare
Mannigfaltigkeit, Karten können wie folgt konstruiert werden.
Die Abbildung $\mathbb{R}\to S^1: x\mapsto e^{ix}$ bildet die ganze
reelle Achse auf die Kreislinie ab.
Die Abbildung ist allerdings nicht umkehrbar, weil $x$-Werte, die sich
um Vielfache von $2\pi$ unterscheiden, auf den gleichen Punkt in $S^1$
abgebildet werden.
Zu jedem Punkt $x\in\mathbb{R}$ gibt es aber ein Intervall
$U_x=(x-1,x+1)$, welches bijektiv auf eine Teilmenge von $S^1$
abgebildet wird.
Die Exponentialabbildung von $U_x\to S^1$ wie auch die Umkehrabbildung
von der Bildmenge zurück in $U_x$ sind stetig.
Die Koordinaten, die verschiedene solche Karten einem Punkt der Kreislinie
zuordnen können, unterscheiden sich immer um Vielfache von $2\pi$.
Die Koordinatenwechsel-Abbildung zwischen zwei Karten $U_x$ und $U_y$
sind also Abbildungen der Form $x\mapsto x+2\pi k$ mit $k\in\mathbb{Z}$,
also sicher differenzierbar.
Damit ist ein differenzierbarer Atlas für $S^1$ konstruiert, $S^1$
ist eine differenzierbare Mannigfaltigkeit.
\end{beispiel}

XXX differenzierbare Abbildungen

XXX Mannigfaltigkeit $S^2$

%
% Lie-Gruppen
%
\subsubsection{Lie-Gruppen}
Die Gruppen $S^1$ war als differenzierbare Mannigfaltigkeit erkannt
worden.
Damit die Struktur der Gruppe und die differenzierbare Struktur sinnvoll
miteinander verwendet werden können ist notwendig, dass die
Verknüpfungsabbildung $(x,y)\mapsto xy$ und die Umkehrabbildung
$x\mapsto x^{-1}$ nicht nur stetig, sondern sogar differenzierbar sind.

\begin{definition}
Eine Lie-Gruppe ist eine Gruppe, die gleichzeitig eine differenzierbare
Mannigfaltigkeit ist derart, dass die Gruppenoperation
$G\times G\to G:(x,y)\mapsto xy$
und die Invertierung $G\to G: x\mapsto x^{-1}$ differenzierbare Abbildungen
sind.
\end{definition}

In den für die Gruppe $S^1$ konstruierten Karten ist die Verknüpfung die
Addition von Koordinaten und die Invertierung ist der Vorzeichenwechsel.
Beide sind differenzierbar, daher ist $S^1$ eine Lie-Gruppe.

%
% Funktionen auf einer Gruppe
%
\subsection{Funktionen auf einer Gruppe
\label{buch:gruppen:subsection:funktionen}}
In diesem Abschnitt ist $G$ eine Gruppe, die wir multiplikativ
schreiben.
Die harmonische Analysis handelt von der Analyse von Funktionen.
Im Falle einer Lie-Gruppe kann man zusätzlich sinnvoll von Ableitungen
der Funktionen sprechen.
Wir definieren daher

\begin{definition}
Die Menge der stetigen reell- und komplexwertigen Funktionen wird mit
$C_{\mathbb{R}}(G)$ bzw.~$C_{\mathbb{C}}(G)$ bezeichnet.
Ist $G$ eine Lie-Gruppe, dann ist
$C_{\mathbb{R}}^\infty(G)$ die Menge der unendlich oft differenzierbaren
reellwertigen Funktionen auf $G$,
$C_{\mathbb{C}}^\infty(G)$ ist die Menge der unendlich oft differenzierbaren
komplexwertigen Funktionen.
\end{definition}

Die Gruppenstruktur ermöglich, lineare Operatoren auf $C_{\mathbb{R}}(G)$
und $C_{\mathbb{C}}(G)$ zu definieren.

\begin{definition}
\label{buch:gruppen:gruppe:def:translation}
Für $s\in G$ ist $T_s$ die Abbildung
\[
T_s
\colon
C_{\mathbb{R}}(G) \to C_{\mathbb{R}}(G)
:
f \mapsto T_sf
\quad
\text{mit}
\quad
(T_sf)(x) = f(s^{-1}x).
\]
Sie heisst die {\em Translation} um $s\in G$.
\end{definition}

Die Translation ist natürlich linear, denn
\begin{align*}
(T_s(f+g))(x)
&=
(f+g)(s^{-1}x)
\\
&=
f(s^{-1}x) + g(s^{-1}x)
=
(T_sf)(x) + (T_sg)(x)
&&\Rightarrow&
T_s(f+g)&=T_sf+T_sg
\\
(T_s(\lambda f))(x)
&=
\lambda f(s^{-1}x)
=
\lambda (T_sf)(x)
&&\Rightarrow&
T_s\lambda f
&=
\lambda T_sf
\end{align*}

%
% Eigenvektoren von T_s
%
\subsubsection{Eigenfunktionen des Translationsoperators}
Tatsächlich wurden in früheren Kapiteln Funktionen verwendet, die
bezüglich der Translation besondere Eigenschaften hatten.
Zum Beispiel sind die Funktionen $f(x)=e_k(x)=e^{ikx}$ auf $G=\mathbb{R}$
Eigenfunktionen des Translationsoperators, denn
\[
(T_se_k)(x)
=
e^{ik(x-s)}
=
e^{iks}e^{ikx}
=
e^{-iks} e_k(x).
\]
Insbesondere ist $e_k$ eine Eigenfunktion von $T_s$ mit Eigenwert
$\lambda=e^{-iks}$, also $T_se_k = \lambda e_k$.

%
% Gruppenstruktur der Translationen
%
\subsubsection{Gruppenstruktur der Translationen}
Wir berechnen die Zusammensetzung zweier Translationen ist $T_s$ und $T_t$.
Um $T_sT_t$ zu berechnen, muss zunächst die Funktion $T_tf$ bestimmt werden.
Es ist $(T_tf)(x) = f(t^{-1}x)$.
Die Translation $T_sg$ einer beliebigen Funktion auf dem Element $y\in G$
ist $(T_sg)(y)=g(s^{-1}y)$.
Setzt man $g=T_tf$ ein, ergibt sich
\[
(T_sT_tf)(x)
=
(T_tf)(s^{-1}x)
=
f(t^{-1}s^{-1}x)
=
f((st)^{-1}x)
=
(T_{st}f)(x),
\]
also $T_sT_t=T_{st}$.

%
% Rechtsoperation der Gruppe auf 
%
\subsubsection{Rechtsoperation von $G$ auf $C(G)$}
Die Operation $T_s$ ist genauer die Links-Translation, die Gruppenoperation
wirkt auf das Argument von links.
Für eine abelsche Gruppe spielt die Reihenfolge der Operanden keine
Rolle, für eine nichtabelsche Gruppe ergibt sich jedoch ein Unterschied.

\begin{definition}
Der Operator $R_s\colon C(G)\to C(G)$ der Rechts-Translation ist definiert
durch
\[
R_s
\colon
C_{\mathbb{R}}(G)\to C_{\mathbb{R}}(G)
:
f \mapsto R_sf
\quad\text{mit}\quad
(R_sf)(x) = f(xs).
\]
\end{definition}

Die Zusammensetzung von $R_s$ und $R_t$ kann ganz ähnlich wie für
$T_s$ und $T_t$ berechnet werden.
Zunächst ist $R_sg(y) = g(ys)$.
Wendet man dies auf $g=R_tf$ mit $g(x)=(R_tf)(x)=f(xt)$ an, bekommt man
\[
(R_sR_tf)(x)
=
(R_sg)(x)
=
g(xs)
=
(R_tf)(xs)
=
f(xst)
=
(R_{st}f)(x)
\]
oder kurz $R_sR_t=R_{st}$.






