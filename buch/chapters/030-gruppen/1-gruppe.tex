%
% 1-gruppe.tex -- Konzept einer Gruppe
%
% (c) 2022 Prof Dr Andreas Müller, OST Ostschweizer Fachhochschule
%
\section{Gruppen
\label{buch:gruppen:section:gruppe}}
\kopfrechts{Gruppe}
Die Translationsinvarianz des Definitionsbereiches $\mathbb{R}$
entsteht aus der Tatsache, dass die Addition einer Zahl
nicht aus $\mathbb{R}$ herausführt und sich auch umkehren lässt.
Sie basiert also darauf, dass es in $\mathbb{R}$ eine umkehrbare
Verknüpfung gibt.
Diese Idee wird von der algebraischen Struktur einer Gruppe eingefangen.

%
% Definition
%
\subsection{Definition
\label{buch:gruppen:subsection:definition}}
Der Begriff einer Gruppe soll alle Arten von invertierbaren Rechenoperationen
erfassen, also Addition/Subtraktion, Multiplikation/Division oder
die Matrixmultiplikation mit der inversen Matrix.
Die minimal nötigen Eigenschaften fasst die folgende Definition zusammen.

\begin{definition}
\label{buch:gruppen:definition:gruppe}
Eine {\em Gruppe} $G$ ist eine Menge mit einer Verknüpfung
$\cdot \colon G\times G\to G : (x,y) \mapsto xy $, welche folgende
Eigenschaften hat:
\begin{enumerate}
\item
Die Verknüpfung ist assoziativ, d.~h.~$(xy)z=x(yz)$ für alle
$x,y,z\in G$.
\item
Es gibt ein {\em neutrales Element} $e\in G$, für welches $ex=x$ für alle
$x\in G$ gilt.
\index{neutrales Element}%
\item 
Zu jedem $x\in G$ gibt es ein {\em inverses Element} $x^{-1}\in G$ mit der
Eigenschaft $x^{-1}x=e$.
\index{inverses Element}%
\end{enumerate}
\end{definition}

Man beachte, dass die Defiinition nicht verlangt, dass die Faktoren
vertauscht werden können. 

\begin{beispiel}
Die Menge $\operatorname{GL}_n(\mathbb{R})$ der invertierbaren
$n\times n$-Matrizien mit reellen Einträgen und der Matrixmultiplikation
heisst die {\em allgemeine lineare Gruppe}.
\index{allgemeine lineare Gruppe}%
\index{Gruppe!allgemeine lineare}%
Das neutrale Element von $\operatorname{GL}_n(\mathbb{R})$ ist die
Einheitsmatrix, das inverse Element einer Matrix
$A\in \operatorname{GL}_n(\mathbb{R})$
ist die inverse Matrix $A^{-1}$.
\end{beispiel}

\begin{definition}
\label{buch:gruppen:definition:abelsch}
Eine Gruppe $G$ heisst {\em abelsch}, wenn 
$xy=yx$ für alle $x,y\in G$ gilt.
\end{definition}

Die Verknüpfung einer abelschen Gruppe ist also {\em kommutativ}.
Abelsche Gruppen werden oft additiv geschrieben, d.~h.~mit einem
Pluszeichen als $x+y$ für $x,y\in G$.
Das neutrale Element heisst dann auch das {\em Nullelement} und wird $0$
geschrieben: $x+0=x$ für alle $x\in G$.
Das inverse Element von $x$ heisst dann auch das
{\em entgegengesetzte Element} und wird $-x$ geschrieben. 
Es gilt $x+(-x)=0$ für alle $x\in G$.

\begin{beispiel}
Die Gruppe $(\mathbb{R},+)$ ist eine abelsche Gruppe.
Die Gruppe der von $0$ verschiedenen Zahlen mit der Multiplikation
$(\mathbb{R}^*,\cdot)$ ist eine abelsche Gruppe.
\end{beispiel}

%
% Endliche Gruppen
%
\subsection{Endliche Gruppen
\label{buch:gruppen:subsection:endliche-gruppen}}

% XXX Die Gruppen \mathbb{Z}/n\mathbb{Z}
% XXX Die Gruppen S_n

%
% Lie-Gruppen
%
\subsection{Lie-Gruppen
\label{buch:gruppen:subsection:lie-gruppen}}
%
% XXX R und R/Z
% XXX SO(2)
% XXX SO(3) und andere Matrizengruppen

%
% Funktionen auf einer Gruppe
%
\subsection{Funktionen auf einer Gruppe
\label{buch:gruppen:subsection:funktionen}}
In diesem Abschnitt ist $G$ eine Gruppe, die wir multiplikativ
schreiben.
Die harmonische Analysis handelt von der Analyse von Funktionen.
Im Falle einer Lie-Gruppe kann man zusätzlich sinnvoll von Ableitungen
der Funktionen sprechen.
Wir definieren daher

\begin{definition}
Die Menge der stetigen reell- und komplexwertigen Funktionen wird mit
$C_{\mathbb{R}}(G)$ bzw.~$C_{\mathbb{C}}(G)$ bezeichnet.
Ist $G$ eine Lie-Gruppe, dann ist
$C_{\mathbb{R}}^\infty(G)$ die Menge der unendlich oft differenzierbaren
reellwertigen Funktionen auf $G$,
$C_{\mathbb{C}}^\infty(G)$ ist die Menge der unendlich oft differenzierbaren
komplexwertigen Funktionen.
\end{definition}

Die Gruppenstruktur ermöglich, lineare Operatoren auf $C_{\mathbb{R}}(G)$
und $C_{\mathbb{C}}(G)$ zu definieren.

\begin{definition}
Für $s\in G$ ist $T_s$ die Abbildung
\[
T_s
\colon
C_{\mathbb{R}}(G) \to C_{\mathbb{R}}(G)
:
f \mapsto T_sf
\quad
\text{mit}
\quad
(T_sf)(x) = f(s^{-1}x).
\]
Sie heisst die {\em Translation} um $s\in G$.
\end{definition}

Die Translation ist natürlich linear, denn
\begin{align*}
(T_s(f+g))(x)
&=
(f+g)(s^{-1}x)
\\
&=
f(s^{-1}x) + g(s^{-1}x)
=
(T_sf)(x) + (T_sg)(x)
&&\Rightarrow&
T_s(f+g)&=T_sf+T_sg
\\
(T_s(\lambda f))(x)
&=
\lambda f(s^{-1}x)
=
\lambda (T_sf)(x)
&&\Rightarrow&
T_s\lambda f
&=
\lambda T_sf
\end{align*}

%
% Eigenvektoren von T_s
%
\subsubsection{Eigenfunktionen des Translationsoperators}
Tatsächlich wurden in früheren Kapiteln Funktionen verwendet, die
bezüglich der Translation besondere Eigenschaften hatten.
Zum Beispiel sind die Funktionen $f(x)=e_k(x)=e^{ikx}$ auf $G=\mathbb{R}$
Eigenfunktionen des Translationsoperators, denn
\[
(T_se_k)(x)
=
e^{ik(x-s)}
=
e^{iks}e^{ikx}
=
e^{-iks} e_k(x).
\]
Insbesondere ist $e_k$ eine Eigenfunktion von $T_s$ mit Eigenwert
$\lambda=e^{-iks}$, also $T_se_k = \lambda e_k$.

%
% Gruppenstruktur der Translationen
%
\subsubsection{Gruppenstruktur der Translationen}
Wir berechnen die Zusammensetzung zweier Translationen ist $T_s$ und $T_t$.
Um $T_sT_t$ zu berechnen, muss zunächst die Funktion $T_tf$ bestimmt werden.
Es ist $(T_tf)(x) = f(t^{-1}x)$.
Die Translation $T_sg$ einer beliebigen Funktion auf dem Element $y\in G$
ist $(T_sg)(y)=g(s^{-1}y)$.
Setzt man $g=T_tf$ ein, ergibt sich
\[
(T_sT_tf)(x)
=
(T_tf)(s^{-1}x)
=
f(t^{-1}s^{-1}x)
=
f((st)^{-1}x)
=
(T_{st}f)(x),
\]
also $T_sT_t=T_{st}$.

%
% Rechtsoperation der Gruppe auf 
%
\subsubsection{Rechtsoperation von $G$ auf $C(G)$}
Die Operation $T_s$ ist genauer die Links-Translation, die Gruppenoperation
wirkt auf das Argument von links.
Für eine abelsche Gruppe spielt die Reihenfolge der Operanden keine
Rolle, für eine nichtabelsche Gruppe ergibt sich jedoch ein Unterschied.

\begin{definition}
Der Operator $R_s\colon C(G)\to C(G)$ der Rechts-Translation ist definiert
durch
\[
R_s
\colon
C_{\mathbb{R}}(G)\to C_{\mathbb{R}}(G)
:
f \mapsto R_sf
\quad\text{mit}\quad
(R_sf)(x) = f(xs).
\]
\end{definition}

Die Zusammensetzung von $R_s$ und $R_t$ kann ganz ähnlich wie für
$T_s$ und $T_t$ berechnet werden.
Zunächst ist $R_sg(y) = g(ys)$.
Wendet man dies auf $g=R_tf$ mit $g(x)=(R_tf)(x)=f(xt)$ an, bekommt man
\[
(R_sR_tf)(x)
=
(R_sg)(x)
=
g(xs)
=
(R_tf)(xs)
=
f(xst)
=
(R_{st}f)(x)
\]
oder kurz $R_sR_t=R_{st}$.






