%
% 2-haar.tex -- Haar-Maß
%
% (c) 2022 Prof Dr Andreas Müller, OST Ostschweizer Fachhochschule
%
\section{Haar-Maß
\label{buch:gruppen:section:haar}}
\kopfrechts{Haar-Maß}
Die Integration von Funktionen einer reellen Variablen oder auch
das in Abschnitt
\ref{buch:skalarprodukt:funktionenraeume:subsection:grenzen-riemann}
erwähnte Lebesque-Integral hat die Eigenschaft, dass das
Integral einer Funktion sich nicht ändert, wenn man die Funktion
entlang der reellen Achse verschiebt.
Für eine integrierbare Funktion $f\colon\mathbb{R}\to\mathbb{R}$
gilt
\[
\int_{-\infty}^\infty f(x)\,dx
=
\int_{-\infty}^\infty f(x-s)\,dx
=
\int_{-\infty}^\infty T_sf(x)\,dx,
\]
wobei die $T_s$ die in Definition~\ref{buch:gruppen:gruppe:def:translation}
eingeführt Translationsoperation ist.
Die harmonische Analysis mit den Exponentialfunktion $e^{ikx}$ 
verwendet diese Translationsinvarianz der Integration auf wesentliche Art,
das Skalarprodukt von Funktionen ändert sich ebenfalls nicht, wenn sie
entlang der Zeitachse verschoben werden.

Wenn die harmonische Analysis auf andere Gruppen als auf die reellen
Zahlen ausgedehnt werden soll, dann wird eine Integration auf diesen
Gruppen benötigt, welche die gleiche Invarianzeigenschaft unter
Translationen hat.

%
% Mittelung auf einer endlichen Gruppe
%
\subsection{Mittelung auf einer endlichen Gruppen
\label{buch:haar:subsection:endlich}}

\begin{definition}
\label{buch:gruppen:haar:def:mittelung}
Sei $G$ ein endliche Gruppe und $f$ eine Funktion $f\colon G\to \mathbb{R}$.
Die {\em Mittelungsoperation} über die Gruppe ist die Operation
\[
Mf = \frac{1}{|G|}\sum_{g\in G} f(g).
\]
\index{Mittelungsoperation}
\end{definition}

Die Mittelungsoperation ist natürlich linear, denn
\[
M(\lambda f + \mu g)
=
\frac{1}{|G|}
\sum_{h\in G}
(\lambda f(h)+\mu g(h))
=
\lambda \frac{1}{|G|}\sum_{h\in G}f(h)
+
\mu \frac{1}{|G|}\sum_{h\in G}g(h)
=
\lambda Mf + \mu Mg.
\]
Die Operation ist aber auch translationsinvariant.
Dazu muss
\begin{equation}
MT_sf
=
\frac{1}{|G|}
\sum_{g\in G} f(s^{-1}g)
\label{buch:gruppen:haar:eqn:mitteltranslation}
\end{equation}
mit $Mf$ verglichen werden.
Die Abbildung $g\mapsto s^{-1}g$ ist aber nur eine Permutation der endlich
vielen Elemente der Gruppe $G$.
Die Summe auf der rechten Seite von
\eqref{buch:gruppen:haar:eqn:mitteltranslation}
ist daher nur die Summe über alle Funktionswerte $f(g)$ in einer anderen
Reihenfolge oder
\[
MT_sf
=
\frac{1}{|G|} \sum_{g\in G}f(g)
=Mf.
\]
Damit ist gezeigt, dass die Mittelungsoperation translationsinvariant
ist.

Die Mittelungsoperation von Definition~\ref{buch:gruppen:haar:def:mittelung}
ermöglicht jetzt ein, das translationsinvariantes Skalarprodukt
\[
\langle f,g\rangle_G
=
M(\overline{f}g)
=
\frac{1}{|G|}
\sum_{h\in G}
\overline{f(h)}g(h)
\]
zu definieren.

Im Nachweis, dass die Mittelungsoperation invariant ist bezüglich
der Translation wurde ausschliesslich verwendet, dass die Abbildung
$g\mapsto s^{-1}g$ ein Permutation der Gruppenelemente ist.
Die Rechtstranslation $R_s\colon g\mapsto gs$ ist ebenfalls eine
Permutation, es folgt also, dass die Mittelungsoperation auch
bezüglich der Rechtstranslation invariant ist.

%
% Integration auf Lie-Gruppen
%
\subsection{Integration auf Lie-Gruppen
\label{buch:haar:subsection:lie-gruppen}}
Die Gruppe der reellen Zahlen ist auch ein Lie-Gruppe, da sie eine
eindimensionale Mannigfaltigkeit ist und die Translationsoperation
eine differenzierbare Abbildung ist.
Die Translationsinvarianz bedeutet dann, dass man einen Massstab für
die Längenmessung durch Translation vom Nullpunkt an jeden beliebigen
anderen Punkt von $\mathbb{R}$ transportieren kann.
Der Transport mit $T_s$ ist die Variablentransformation $x\mapsto x-t$,
die $dx$ in $dx$ überführt, das Integral also unverändert lässt.

Diese Idee lässt sich auf beliebige Lie-Gruppen übertragen.
Um ein Integral auf der Lie-Gruppe zu definieren, muss man in jedem
Punkt ein Koordinatensystem haben und ausserdem wissen, wie die
Integration bezüglich dieses Koordinatensystems durchzuführen ist,
man braucht das ``Volumenelement'' ausgedrückt in diesen Koordinaten.

Wir betrachten ein Koordinatensystem mit Koordinaten $x_1,\dots,x_n$
in einer Umgebung $U$ des neutralen Elementes $e$ einer Lie-Gruppe $G$.
Das Integral
\[
\int_{U} f(x_1,\dots,x_n) \,dx_1\,\dots\,dx_n
\]
der Funktion $f$ in der Umgebung $U$ ist aber nicht unbedingt
translationsinvariant.
Für ein Gruppenelement $s$ nahe beim neutralen Element der Gruppe
ist die Translation mit $T_s$ eine Koordinatentransformation
$(x_1,\dots,x_n)\mapsto (x_1',\dots,x_n')$.
Das Integral kann auch in den gestrichenen Koordinaten als
\[
\int_{U} f(x_1,\dots,x_n) \,dx_1\,\dots\,dx_n
=
\int_{U'} f(x_1',\dots,x_n')
\frac{\partial (x_1,\dots,x_n)}{\partial(x_1',\dots,x_n')}
\,dx_1'\,\dots\,dx_n'
\]
berechnet werden.
Die Funktionaldeterminante hängt natürlich von $s$ ab.
Die Integrale unterscheiden sich nur durch eine skalare Gewichtsfunktion.
Durch geeignete Wahl einer Funktion $w(x_1,\dots,x_n)$ kann man 
das Integral
\[
\int_{U} f(x_1,\dots,x_n) w(x_1,\dots,x_n)\,dx_1\dots\,dx_n
\]
mindestens in einer Umgebung des neutralen Elementes
translationsinvariant machen.

Möglicherweise gibt es kein globales Koordinatensystem für die 
die Gruppe $G$, aus diesem Grund wurde ja das Konzept der Karte
und des Atlas eingeführt.
Das Koordinatensystem in einer Umgebung des neutralen Elements und
die zugehörige translationsinvariante Integration kann aber mit
dem Operator $T_s$ in jeden beliebigen anderen Punkt $s$ transportiert
werden.
In einer Umgebung von $s$ gibt es daher ebenfalls eine
translationsinvariante Integration.
Da es nur genau ein einziges Element in $G$ gibt, welches die Translation
von $e$ nach $s$ ausführen kann, ist diese translationsinvariante 
Integration daher auf der ganzen Gruppe wohldefiniert.

Wir schliessen daher, dass es auf einer Lie-Gruppe ganz analog 
zu den reellen Zahlen ein translationsinvariante Integration
gibt.

\begin{beispiel}
Die Lie-Gruppe $S^1 = \operatorname{SO}(2)$ der Drehungen in der Ebene
kann mit dem Drehwinkel parametrisiert werden:
\[
g(x)
=
\begin{pmatrix*}[r]
\cos x&-\sin x\\
\sin x& \cos x
\end{pmatrix*}
\in
\operatorname{SO}(2)
\]
Die Koordinaten $x$ ist also ein mögliches Koordinatensystem in einer
Umgebung des neutralen Elementes.
Als Koordinatensystem für die ganze Gruppe ist problematisch daran,
dass das gleiche Gruppenelement für viele verschiedene $x$-Werte
erreicht wird, nämlich für alle, die sich um ein ganzzahliges
Vielfaches von $2\pi$ unterscheiden.
Die $x$-Werte im Intervall $[0,2\pi)$ beschreiben also alle Elemente
der Gruppe $\operatorname{SO}(2)$. 

Die Koordinatentransformation, die mit der Translation $T_s$ einhergeht,
ist die Abbildung $x\mapsto x-s$, die die Funktionaldeterminante $1$
hat.
Die Integration
\[
\int_{\operatorname{SO}(2)} f(g) \,dg
=
\int_0^{2\pi} f(g(x))\,dx
\]
ist daher eine translationsinvariante Integration.
Jedes skalare Vielfache davon ist aber ebenfalls eine translationsinvariante
Integration.
Die Integration
\[
\frac{1}{2\pi} f(g(x))\,dx
\]
hat die Eigenschaft, dass die konstante Funktion $1$ das Integral $1$
hat.
\end{beispiel}

%
% Das Haar-Maß
%
\subsection{Das Haar-Maß
\label{buch:haar:subsection:haar}}
Man kann zeigen, dass die Konstruktion, die im vorangegangenen Abschnitt
für Lie-Gruppen mit ihrer differenzierbaren Struktur auch wesentlich
abstrakter für beliebige topologische Gruppen durchgeführt werden kann.
Sie geht auf Alfred Haar zurück.

\begin{satz}[Haar]
\label{buch:gruppen:haar:satz:haar}
Ist $G$ eine lokal kompakte, topologische Gruppe, dann gibt es ein
translationsinvariantes Mass $\mu$ auf $G$, also ein Mass derart,
dass 
\[
\int_G f(g)\,d\mu(g)
=
\int_G T_sf(g)\,d\mu(g)
=
\int_G f(s^{-1}g)\,d\mu(g)
\]
gilt für alle $s\in G$.
Das Mass $\mu$ ist bis auf einen skalaren Faktor eindeutig bestimmt.
\end{satz}

Das im Satz~\ref{buch:gruppen:haar:satz:haar} definierte Mass
heisst das {\em linksinvariante haarsche Mass}.
Die Voraussetzung {\em lokal kompakt} ist eine Endlichkeitsbedingung,
die unendlichdimensionale Gruppen ausschliesst.
Sie verlangt, dass jeder Punkt eine kompakte Umgebung hat, dass also
in der Nähe eines Punktes die Gruppe wie ein endlichdimensioinaler 
Raum aussieht.

Die Konstruktion, die auf Satz~\ref{buch:gruppen:haar:satz:haar}
führt, kann auch für die Rechtstranslation mit $R_s$ durchgeführt
werden.
Es gibt also auf einer topologischen Gruppe auch ein bis auf einen
skalaren Faktor eindeutig bestimmtes rechtsinvariantes Haarsches Mass,
der Satz sagt aber nichts darüber aus, ob das linksinvariante und das
rechtsinvariante Mass übereinstimmen.

%
% Unimodulare Gruppen
%
\subsection{Unimodulare Gruppen
\label{buch:haar:subsection:unimodular}}
Für eine abelsche Gruppe wie die Gruppe der reellen Zahlen
oder $\mathbb{R}^n$ gibt es keinen Unterschied zwischen links- und
rechtstranslation, daher sind auch die links- und rechtsinvarianten
haarschen Masse identisch.
Die Mittelungsoperation einer endlichen Gruppe ist sowohl rechts-
als auch linksinvariant, selbst dann, wenn die Gruppe nicht abelsch
ist.
Für eine nicht kommutative Gruppe wie die Gruppe $\operatorname{SO}(3)$
der Drehungen des Raumes dagegen ist es nicht mehr klar, dass 
die rechts und linksinvarianten Masse gleich sein müssen.

Sei $G$ eine Gruppe mit dem linksinvarianten Mass $\mu$.
Dann ist $R_s\mu$ ebenfalls ein linksinvariantes Mass definiert
durch das Integral
\[
\int_G f(g) \, dR_r\mu(g)
=
\int_G R_rf(g) \,d\mu(g)
=
\int_G f(gr)\,d\mu(g).
\]
Setzen wir darin $T_sf$ ein, zeigt sich, dass wegen
\[
\int_G T_sf(gr)\,d\mu(g)
=
\int_G f(s^{-1}gr)\,d\mu(g)
=
\int_G f(gr)\,d\mu(g)
\]
$R_r\mu$ ein linksinvariantes Mass ist.
Nach dem Satz~\ref{buch:gruppen:haar:satz:haar} unterscheiden sich
die beiden Masse um einen skalaren Faktor.

\begin{definition}
Es gibt eine Funktion $\delta\colon G\to \mathbb{R}$ derart, dass
\[
R_r \mu = \delta(r) \mu
\]
für $r\in G$
ist.
Die Gruppe $G$ heisst {\em unimodular}, wenn $\delta(r)=1$ ist für
alle $r\in G$.
\end{definition}

Für eine unimodulare Gruppe unterscheiden sich die linksinvarianten
und die rechtsinvarianten Masse also nicht.
Abelsche Gruppen sind unimodular.
Man kann auch zeigen, dass $\delta(r)$ ein Homomorphismus ist,
was sich als ziemlich starke Einschränkung herausstellt.

\begin{satz}
Ist die lokalkompakte topologische Gruppe $G$ kompakt, dann ist
sie unimodular.
\end{satz}

\begin{proof}[Beweis]
Die Funktion $\delta\colon G\to\mathbb{R}$ ist ein stetiger Homomorphismus.
Als stetige Funktion hat sie auf dem kompakten Definitionsbereich $G$
mindestens ein Maximum und ein Minimum.
Wir bezeichen die Stellen des Maximums mit $g$.
Wenn $\delta(g)>1$ wäre, dann wäre auch
\[
\delta(gg)
=
\delta(g)
\delta(g)
>
\delta(g),
\]
der Wert an  der Stelle $gg$ wäre also grösser als das Maximum.
Der Widersprich zeigt, dass das Maximum $\delta(g)=1$ sein muss.
Analog sieht man ein, dass auch das Minimum von $\delta(g)=1$
sein muss, dass also $\delta(r)=1$ ist.
\end{proof}

\begin{beispiel}
Die Gruppe der Streckungen und Verschiebungen 
\[
G
=
\mathbb{R^+}\ltimes \mathbb{R}
\]
besteht aus Paaren $(s,t)$ mit der Verknüpfung
\[
(s_1,t_1)\cdot(s_2,t_2)
=
(s_1s_2,t_1 + s_1t_2).
\]
Die zweite Komponente im Produkt ist offensichtlich nicht symmetrisch
in den beiden Indizes $1$ und $2$, die Gruppe $G$ ist nicht abelsch.
Das inverse Element von $(s,t)$ ist $(s^{-1},-s^{-1}t)$, denn
\begin{align*}
(s,t)\cdot(s^{-1},-s^{-1}t)
&=
(ss^{-1}, t-ss^{-1}t)
=
(1,0)
\\
(s^{-1},-s^{-1}t)\cdot(s,t)
&=
(s^{-1}s,-s^{-1}t+s^{-1}t)
=
(1,0).
\end{align*}
Das linksinvariante haarsche Mass muss von der Form
\[
\int_G f(g) w(g)\,dg
=
\int_0^\infty \int_{-\infty}^\infty
f(\sigma,\tau)
w(\sigma,\tau)
\,d\tau\,d\sigma
\]
sein.
Die Linkstranslation mit $(s,t)$ ist 
\begin{align*}
\int_G f(h^{-1}g) w(g)\,dg
&=
\int_0^\infty
\int_{-\infty}^\infty
f((s^{-1},-s^{-1}t)(\sigma,\tau)\,
w(\sigma,\tau)
\,d\tau
\,d\sigma
\\
&=
\int_0^\infty
\int_{-\infty}^\infty
f(s^{-1}\sigma,-s^{-1}t+s^{-1}\tau)\,
w(\sigma,\tau)
\,d\tau
\,d\sigma.
\intertext{Wir schreiben $\sigma'=s^{-1}\sigma$ und
$\tau'=-s^{-1}t+s^{-1}\tau$ oder gleichbedeutend
$\sigma=s\sigma'$ und $\tau=s\tau'+t$ und bekommen
}
&=
\int_{0}^\infty
\int_{-\infty}^\infty
f(\sigma',\tau')\,
w(s\sigma',s\tau'+t)
\,s\,d\tau'
\,s\,d\sigma',
\intertext{
dabei wurde $d\tau = sd\tau'$ und $d\sigma=sd\sigma'$ verwendet.
Linksinvarianz würde bedeuten, dass dieses Integral für
alle Funktionen $f$ übereinstimmt mit}
&=
\int_{0}^\infty
\int_{-\infty}^\infty
f(\sigma',\tau')\,
w(\sigma',\tau')\, s^2
\,d\tau'
\,d\sigma'.
\end{align*}
Es folgt, dass die Funktion $w$ die Funktionalgleichung
\[
w(s\sigma',s\tau'+t) = w(\sigma',\tau') s^2
\]
für alle $s$ und $t$ erfüllen muss.
Daraus folgt zunächst, dass $w$ nicht vom zweiten Argument abhängt,
und weiter, dass $w(s,t)=as^2$.

Für die Rechtsinvarianz suchen wir eine Funktion $\tilde{w}(s,t)$ derart,
dass
\begin{align*}
\int_0^\infty
\int_{-\infty}^\infty
f(\sigma,\tau)
\,\tilde{w}(\sigma,\tau)
\,d\sigma\,d\tau
&=
\int_0^\infty
\int_{-\infty}^\infty
f((\sigma,\tau)\cdot(s,t))
\,\tilde{w}(\sigma,\tau)
\,d\sigma\,d\tau
\\
&=
\int_0^\infty
\int_{-\infty}^\infty
f(\sigma s,\tau +\sigma t)
\,\tilde{w}(\sigma,\tau)
\,d\sigma\,d\tau.
\intertext{Wie vorhin verwenden wir die Substitution $\sigma'=s\sigma$ und
$\tau' = \tau+\sigma t$ oder umgekehrt $\sigma=s^{-1}\sigma'$  und
$\tau = \tau' - \sigma t = \tau' - s^{-1}t\sigma'$ und erhalten}
&=
\int_0^\infty
\int_{-\infty}^\infty
f(\sigma',\tau')
\,
\tilde{w}(s^{-1}\sigma', \tau'-\sigma t)
\,d\sigma\,d\tau.
\\
&=
\int_0^\infty
\int_{-\infty}^\infty
f(\sigma',\tau')
\,
\tilde{w}(s^{-1}\sigma', \tau'-\sigma t)
\,s^{-1}\,d\sigma'
\,d\tau'.
\end{align*}
Dies muss für beliebige $(s,t)$ und beliebige Funktionen gelten,
woraus man schliessen kann, dass $\tilde{w}$ die Funktionalgleichung
\[
\tilde{w}(\sigma',\tau')
=
\tilde{w}(s^{-1}\sigma', \tau'-s^{-1}\sigma't)\,s^{-1}
\]
Wieder folgt, dass $\tilde{w}$ nicht vom zweiten Argument abhängt, und
die Abhängigkeit vom ersten Argument ist $\tilde{w}(s,t)=s$.

Da die beiden Funktionen $w(s,t)$ und $\tilde{w}(s,t)$ verschieden sind,
sind das rechtsinvariante und das linksinvariante Haarmass verschieden
und die Gruppe $G$ ist nicht unimodular.
\end{beispiel}

Die Tatsache dass $\mathbb{R}^+\ltimes\mathbb{R}$ nicht kompakt
und nicht abelsch ist, ist ist nicht ein ausreichender Grund zu
schliessen, dass $G$ nicht unimodular sein könne.
Wir werden später die Gruppe der Drehungen und Verschiebungen
$\operatorname{SO}(2)\ltimes \mathbb{R}^2$ kennenlernen,
die ebenfalls nicht abelsch und nicht kompakt aber trotzdem
unimodular ist.
