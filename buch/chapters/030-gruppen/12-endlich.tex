%
% 12-endlich.tex -- Konzept einer Gruppe
%
% (c) 2022 Prof Dr Andreas Müller, OST Ostschweizer Fachhochschule
%

%
% Endliche Gruppen
%
\subsection{Endliche Gruppen
\label{buch:gruppen:subsection:endliche-gruppen}}
Für theoretische Überlegungen sind die kontinuierliche
Fourier-Transformation auf Gruppen wie $\mathbb{R}$ oder
$\mathbb{R}/2\pi\mathbb{Z}$ besonders leistungsfähig, weil hier die
ganze Macht der Analysis zur Konstruktion der orthonormierten Basis
zur Verfügung steht.
In Ingenieuranwendungen reduziert man die unendlichdimensionalen
Vektorräume dagegen gerne auf endliche Menge, so dass man mit
Vektoren und Matrizen rechnen kann.
Dafür müssen endliche Gruppen als Definitionsbereiche für die Funktionen
gefunden werden.

%
% Die zyklischen Gruppen \mathbb{Z}/n\mathbb{Z}
%
\subsubsection{Die zyklischen Gruppen $\mathbb{Z}/n\mathbb{Z}$}
Für die diskrete Fourier-Analysis besonders wichtig sind die zyklischen
Gruppen.

\begin{definition}
\label{buch:gruppen:endliche-gruppen:def:zyklisch}
Die Gruppe
\[
\mathbb{Z}/n\mathbb{Z}
=
\{0,1,2,\dots,n-1\}
\]
der Reste modulo $n$ mit der Addition von Resten ist eine abelsche
Gruppe.
\end{definition}

\begin{beispiel}
Die Gruppe $\mathbb{Z}/7\mathbb{Z}$ besteht aus den Resten
$\{0,1,2,\dots,6\}$.
Die Verknüfung zweier Reste ist deren Summe, die aber auf einen
Rest in $\mathbb{Z}/7\mathbb{Z}$ reduziert werden muss.
Das inverse Element des Restes $r\in\mathbb{Z}/7\mathbb{Z}$
ist der Rest $7-r$ für $r\ne 0$ und $-0=0$.
\end{beispiel}

Die zyklischen Gruppen können auch als multiplikativ geschriebene
Untergruppen der von $0$ verschiedenen komplexen Zahlen $\mathbb{C}^*$
geschrieben werden.
Dazu verwendet man die Exponentialfunktion:
\[
C_n
=
\{ e^{2\pi ik/n}\mid k=0,1,\dots,n-1\}.
\]
Die Exponentialabbildung
\[
\exp
\colon
\mathbb{Z}/n\mathbb{Z}
\to
C_n
:
k\mapsto e^{2\pi ik/n}
\]
ist ein Homomorphismus, denn
\[
\exp(k+l)
=
e^{2\pi i(k+l)/n}
=
e^{2\pi ik/n}
e^{2\pi il/n}
=
\exp(k)\exp(l).
\]
Die Reste werden auf verschiedene Ecken eines regelmässigen
$n$-Ecks in der komplexen Ebene abgebildet.
Die Abbildung $\exp$ ist daher auch eine Bijektion.
Der Homomorphismus $\exp$ zeigt somit, dass die Gruppen
$\mathbb{Z}/n\mathbb{Z}$ und $C_n$ die gleiche Struktur haben.

Die additiv geschriebene Gruppe $\mathbb{Z}/n\mathbb{Z}$ und
die multiplikativ geschriebene Gruppe $C_n$ sind also isomorph.

%
% Die zyklischen Gruppen als Kern
%
\subsubsection{Die zyklischen Gruppen als Kern}
Die Gruppe $C_n$ wurde oben als Bild der Gruppe
$\mathbb{Z}/n\mathbb{Z}$ unter der Exponentialabbildung 
in $\mathbb{C}^*$ erkannt.
Man kann sie aber auch als Kern eines geeignet gewählten Homomorphismus
verstehen.

Die Abbildung
\[
\varphi
\colon
S^1\to S^1
:
z\mapsto z^n
\]
ist ein Homomorphismus, denn es ist ja $\varphi(z_1z_2)=(z_1z_2)^n
= z_1^nz_2^n=\varphi(z_1)\varphi(z_2)$.
Der Kern von $\varphi$ besteht aus den komplexen Zahlen mit der 
Eigenschaft $z^k=1$, das sind genau die Elemente von $C_n$.

%
% Permutationsgruppen
%
\subsubsection{Permutationsgruppen}
Die Menge $[n]=\{1,2,\dots,n\}$ hat $n$ Elemente.
Wir betrachten die Menge aller invertierbaren Abbildungen
$\varphi\colon [n] \to [n]$.
Zwei solche Abbildungen $\varphi$ und $\psi$ können zusammengesetzt
werden, oder sie können invertiert werden.
Tatsächlich ist die Menge 
\[
S_n = \{\varphi\colon [n] \to [n]\mid \text{$\varphi$ ist invertierbar} \}
\]
eine Gruppe, sie heisst die {\em Permutationsgruppe von $n$ Elementen}
oder die {\em symmetrische Gruppe}.

Permutationen können besonders effizient als Matrizen mit zwei Zeilen
geschrieben werden.
Eine Abbildung $\varphi\colon [n]\to[n]$ bildet $i\in [n]$ auf $\varphi(u)$
ab, was man als die Matrix
\[
\varphi
=
\begin{pmatrix}
1&2&3&\dots&n\\
\varphi(1)&\varphi(2)&\varphi(3)&\dots&\varphi(n)
\end{pmatrix}
\]
schreiben kann.
Um die Komposition von zwei Abbildungen $\varphi$ und $\psi$ zu bestimmen,
kann man die beiden Matrizen übereinander schreiben und die Spalten der
unteren Matrix so sortieren, dass sie mit den Elementen in der unteren
Zeile der oberen MatrixMatrix  übereinstimmen:
\[
\begin{aligned}
\varphi
&=
\begin{pmatrix}1&2&3&4\\2&3&1&4\end{pmatrix}
\\
\psi
&=
\begin{pmatrix}1&2&3&4\\3&2&4&1\end{pmatrix}
\end{aligned}
\quad\Rightarrow\quad
\psi\circ \varphi
=
\left\{
\begin{array}{c}
\displaystyle\begin{pmatrix}1&2&3&4\\2&3&1&4\end{pmatrix}\\
\displaystyle\begin{pmatrix}1&2&3&4\\3&2&4&1\end{pmatrix}
\end{array}
\right\}
=
\left\{
\begin{array}{c}
\displaystyle\begin{pmatrix}1&2&3&4\\2&3&1&4\end{pmatrix}\\
\displaystyle\begin{pmatrix}2&3&1&4\\2&4&3&1\end{pmatrix}
\end{array}
\right\}
=
\begin{pmatrix}
1&2&3&4\\
2&4&3&1
\end{pmatrix}.
\]

Die inverse Abbildung findet man, indem man die beiden Zeilen vertauscht
und die Spalten so sortiert, dass die Elemente in der ersten Zeile
wieder aufsteigend sind.
Zum Beispiel
\[
\varphi
=
\begin{pmatrix}
1&2&3&4&5\\
1&3&5&2&4
\end{pmatrix}
\quad\Rightarrow\quad
\varphi^{-1}
=
\begin{pmatrix}
1&3&5&2&4\\
1&2&3&4&5
\end{pmatrix}
=
\begin{pmatrix}
1&2&3&4&5\\
1&4&2&5&3
\end{pmatrix}.
\]
Die Zusammensetzung von $\varphi$ und $\varphi^{-1}$ ist
\[
\varphi\circ\varphi^{-1}
=
\left\{
\begin{array}{c}
\displaystyle
\begin{pmatrix}
1&2&3&4&5\\
1&4&2&5&3
\end{pmatrix}
\\
\displaystyle
\begin{pmatrix}
1&2&3&4&5\\
1&3&5&2&4
\end{pmatrix}
\end{array}
\right\}
=
\left\{
\begin{array}{c}
\displaystyle
\begin{pmatrix}
1&2&3&4&5\\
1&4&2&5&3
\end{pmatrix}
\\
\displaystyle
\begin{pmatrix}
1&4&2&5&3\\
1&2&3&4&5
\end{pmatrix}
\end{array}
\right\}
=
\begin{pmatrix}
1&2&3&4&5\\
1&2&3&4&5
\end{pmatrix}
=
e.
\]

%
% Transpositionen
%
\subsubsection{Transpositionen}
Permutationen können besser verstanden werden, wenn man sie in einfachere 
Bestandteile zerlegen kann.
Besonders nützlich dabei sind die sogenannten Transpositionen.

\begin{definition}
Eine {\em Transposition} $\tau\in S_n$ ist eine Permutation, die genau
zwei Elemente vertauscht und alle anderen Elemente fest bleiben.
\index{Transposition}%
\end{definition}

Die Permutationsgruppe $S_n$ ist für $n>2$ nicht abelsch, denn die beiden
Transpositionen
\[
\tau_{12}
=
\begin{pmatrix}
1&2&3&\dots&n\\
2&1&3&\dots&n
\end{pmatrix}
,
\qquad
\tau_{23}
=
\begin{pmatrix}
1&2&3&\dots&n\\
1&3&2&\dots&n
\end{pmatrix}
\]
haben die Produkte
\begin{align*}
\tau_{12}
\circ
\tau_{23}
&=
\begin{pmatrix}
1&2&3&\dots&n\\
2&3&1&\dots&n
\end{pmatrix}
&&\text{und}&
\tau_{23}
\circ
\tau_{12}
&=
\begin{pmatrix}
1&2&3&\dots&n\\
3&1&2&\dots&n
\end{pmatrix},
\end{align*}
die verschieden sind.

%
% Endliche Gruppen als Untergruppen von $S_n$
%
\subsubsection{Endliche Gruppen als Untergruppen von $S_n$}
Die Permutationsgruppen können als die ``ultimativen'' endlichen Gruppen
im folgenden Sinne betrachtet werden.
Sei $G=\{e=g_1,g_2,\dots,g_n\}$ irgend eine endliche Gruppe, und $g\in G$
ein Gruppenelement.
Für jedes Element $g_i$ ist $gg_i=g_k$ wieder ein Gruppenelement.
Damit wird eine Permutation $\pi\in S_n$ mit $\pi(i)=k$ definiert.
Die Gruppe $G$ kann damit als eine Untergruppe der Gruppe $S_n$
von Permutationen betrachtet werden.
Jede endliche Gruppe ist somit eine Untergruppe von $S_n$.
Für die Zwecke der harmonischen Analysis ist diese Betrachtungsweise
allerdings nicht allzu ergibig.

%
% Permutationsmatrizen
%
\subsubsection{Permutationsmatrizen}
Die Permutationsgruppe hat auch eine wichtige Rolle bei der Definition
der Determinante mit Hilfe des Entwicklungssatzes, wie zum Beispiel
in \cite[Kapitel~4]{buch:linalg} ausgeführt wird.
Darin spielen die Transpositionen eine besondere Rolle, da die
Vertauschungen zweier Spalten einer Matrix das Vorzeichen ihrer
Determinanten umkehrt.
Tatsächlich können die Permutationen von $S_n$ auch als
$n\times n$-Matrizen geschrieben werden.
Der Permutation $\pi\in S_n$ wird die lineare Abbildung
$\mathbb{R}^n\to\mathbb{R}^n$ zugeordnet, die den Standardbasisvektor
$e_k$ auf den Standardbasisvektor $e_{\sigma(k)}$ abbildet.
Die zugehörige Matrix $P_\sigma$ hat die Matrixelemente
\[
(P_\sigma)_{ik} 
=
\begin{cases}
1&\qquad i = \pi(k)\\
0&\qquad\text{sonst}.
\end{cases}
\]
Die Verknüpfung von Permutationen wird zur Matrixmultiplikation,
die Abbildung $\sigma \mapsto P_\sigma$ ist ein Homomorphismus
von der Gruppe $S_n$ in die Gruppe der Permutationsmatrizen.
Man sagt auch, die Permutationen werden durch Matrizen dargestellt oder
die Abbildung, die einer Permutation $\sigma$ die zugehörige
Permutationsmatrix $P_\sigma$ zuordnet, ist eine Darstellung der
Gruppe $S_n$ im Vektorraum $\mathbb{R}^n$.

Die Spalten einer Permutationsmatrix sind die permutierten
Standardbasisvektoren.
Man kann die Permutation rückgängig machen, indem man die Spalten
paarweise geeignet vertauscht.
Die Determinante der Matrix ändert dabei jedesmal das Vorzeichen.
Am Ende des Prozesses steht die Einheitsmatrix mit Determinante $1$.
Insbesondere ist die Determinanten einer Permutationsmatrix $\pm 1$.

\begin{definition}
\index{Vorzeichen einer Permutation}%
\index{Signum einer Permutation}%
Das {\em Vorzeichen} oder {\em Signum} einer Permutation
$\operatorname{sgn}(\sigma)$
ist die Determinante der zugehörigen Permutationsmatrix:
$\operatorname{sgn}(\sigma)=\det P_\sigma$.
Eine Permutation heisst {\em gerade}, wenn $\operatorname{sgn}(\sigma)=1$ ist,
sei heisst {\em ungerade}, wenn $\operatorname{sgn}(\sigma)=-1$ ist.
\index{gerade Permutation}%
\index{ungerade Permutation}%
\end{definition}

Die Permutationsmatrix $P_\tau$ einer Transposition $\tau$ hat nur
zwei vertauschte Spalten, also ist $\operatorname{sgn}(\tau)=\det P_\tau=-1$.
Der oben beschrieben Prozess, mit dem erkannt wurde, dass die Determinante
einer Permutationsmatrix $\pm 1$ sein muss, zeigt noch mehr.

\begin{satz}
Jede Permutation $\sigma$ kann als Produkt von Transpositionen
geschrieben werden.
Das Signum einer Permutation ist genau dann $+1$, wenn $\sigma$
als Zusammensetzung einer geraden Anzahl von Transpositionen
geschrieben werden kann.
Es ist $-1$, wenn $\sigma$ Zusammensetzung einer ungeraden Anzahl
von Transpositionen geschrieben werden kann.
\end{satz}

%
% Die alternierende truppe
%
\subsubsection{Die alternierende Gruppe}
Die Produktregel für Determinanten besagt, dass $\det(AB)=\det(A)\det(B)$,
die Determinante macht also aus der Matrixmultiplikation das Produkt von
reellen Zahlen.
Dies besagt auch, dass $\det$ ein Homomorphismus
$\operatorname{GL}_n(\mathbb{R}) \to \mathbb{C}^*$ ist.
Das Vorzeichen $\operatorname{sgn}$ ist die Zusammensetzung zweier
Homomorphismen, nämlich der Abbildung $\sigma\to P_\sigma$
und der Determinante.
Für zwei Permutationen $\sigma,\varrho\in S_n$ gilt daher für die Vorzeichen
\[
\operatorname{sgn}(\sigma\varrho)
=
\det P_{\sigma\varrho}
=
\det(P_\sigma P_\varrho)
=
\det(P_\sigma)\det( P_\varrho)
=
\operatorname{sgn}(\sigma)
\operatorname{sgn}(\varrho).
\]
Die Zusammensetzung von Homomorphismen ist auch wieder ein Homomorphismus.

Die Menge
\[
A_n
=
\{ \sigma\in S_n \mid \operatorname{sgn}(\sigma) = 1 \}
=
\ker
\operatorname{sgn}
\]
ist als Kern eines Homomorphismus $S_n\to\mathbb{R}$
eine Untergruppe von $S_n$, denn für $\sigma_1,\sigma_2\in S_n$ ist auch
\(
\operatorname{sgn}(\sigma_1\sigma_2)
=
\operatorname{sgn}(\sigma_1)
\operatorname{sgn}(\sigma_2)
=
1\cdot 1
\),
und daher auch $\sigma_1\sigma_2\in S_n$.
Ausserdem ist 
\(
\operatorname{sgn}(\sigma^{-1})=\det P_{\sigma^{-1}}
=
\det P_\sigma^{-1}
=
(\det P_\sigma)^{-1}
=
1^{-1}=1,
\)
somit ist mit $\sigma\in A_n$ auch $\sigma^{-1}\in A_n$.
Die Menge $A_n$ heisst die {\em alternierende Gruppe}.

