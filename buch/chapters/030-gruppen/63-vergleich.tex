%
% 63-vergleich.tex -- Vergleich von Darstellungen 
%
% (c) 2022 Prof Dr Andreas Müller, OST Ostschweizer Fachhochschule
%

%
% Vergleich von Darstellungen
%
\subsection{Vergleich von Darstellungen}
Die Einfachheit der regulären Darstellung hängt davon ab, dass man die
Basis sehr speziell wählen kann.
Eine beliebige Darstellung auf den ersten Blick sehr viel weniger gut
durchschaubar, weil eine beliebig gewählte Basis zu Matrizen führt,
die die Natur Gruppe verschleiern können.
Umgekehrt kann auch die reguläre Darstellung verbergen, dass dahinter
eigentlich einfachere Darstellungen stecken, die man besser erkennen
könnte, wenn man eine zweckmässigere Basis verwenden würde.
Das folgende Beispiel illustriert dies.

\begin{beispiel}
\label{buch:gruppen:darstellung:bsp:c3}
Wir betrachten die reguläre Darstellung der zyklischen Gruppe mit
drei Elementen $C_3=\{0,1,2\}$.
Die Gruppenelemente $1,2\in C_1$ werden in der regulären Darstellung
durch die Matrizen
\[
1\mapsto
P_1=
\begin{pmatrix}
0&0&1\\
1&0&0\\
0&1&0
\end{pmatrix}
\qquad\text{und}\qquad
2\mapsto
P_2
=
\begin{pmatrix}
0&1&0\\
0&0&1\\
1&0&0
\end{pmatrix}
=
P_1^2
\]
dargestellt.

Wir verwenden jetzt im Vektorraum $\mathbb{R}^3$, auf dem diese Matrizen
wirken, eine alternative Basis wie folgt:
\[
b_1
=
\frac{1}{\!\sqrt{3}}
\begin{pmatrix}1\\1\\1\end{pmatrix}
,
\qquad
b_2
=
\frac{1}{\!\sqrt{2}}
\begin{pmatrix*}[r]1 \\ -1 \\ 0\end{pmatrix*}
\qquad\text{und}\qquad
b_3
=
\frac{1}{\!\sqrt{6}}
\begin{pmatrix*}[r] 1\\1\\-2\end{pmatrix*}.
\]
Die Basis ist orthonormiert.
Die Wirkung der Matrix $P_1$ auf diesen Basisvektoren ist
\begin{align*}
P_1b_1
&=
b_1\\
P_1b_2
&=
\frac{1}{\!\sqrt{2}}
\begin{pmatrix*}[r]
0\\1\\-1
\end{pmatrix*}
=
(P_1b_2\cdot b_2) b_2
+
(P_1b_2\cdot b_3) b_3
=
-\frac{1}{2}
b_2
+
\frac{\!\sqrt{3}}{2}
b_3
\\
P_1b_3
&=
\frac{1}{\!\sqrt{6}}
\begin{pmatrix*}[r]
-2\\ 1\\1
\end{pmatrix*}
=
(P_1b_3\cdot b_2) b_2
+
(P_1b_3\cdot b_3) b_3
=
-\frac{\!\sqrt{3}}{2}
b_2
-
\frac{1}{2}
b_3
\end{align*}
Die Koeffizienten in der Darstellung der Bildvektoren in der
orthonormierten Basis $\{b_1,b_2,b_3\}$ auf der rechten Seite
können mit Hilfe des Skalarproduktes ermittel werden.
In der Basis $b_i$ bekommt die Matrix $P_1$ also die Matrix
\begin{equation}
\renewcommand{\arraystretch}{1.2}
\tilde{P}_1
=
\begin{pmatrix}
1&0&0\\
0&-\frac12 & \frac{\!\sqrt{3}}2\\
0&-\frac{\!\sqrt{3}}2&-\frac12
\end{pmatrix}
=
\begin{pmatrix}
1&0&0\\
0&\cos\bigl(-\frac{2\pi}{3}\bigr)& -\sin\bigl(-\frac{2\pi}{3}\bigr) \\
0&\sin\bigl(-\frac{2\pi}{3}\bigr)& \phantom{-}\cos\bigl(-\frac{2\pi}{3}\bigr)
\end{pmatrix}.
\label{buch:gruppen:darstellung:bsp:c3:blockform}
\end{equation}
In der neuen Basis zerfällt die Matrix in zwei Blöcke.
Der linke obere Block ist die eindimensionale triviale Darstellung.
der $2\times 2$-Block in der rechten unteren Ecke ist eine Drehmatrix mit
dem Drehwinkel $120^\circ$.

Geometrisch bedeutet die Blockform
\eqref{buch:gruppen:darstellung:bsp:c3:blockform}, dass man den
dreidimensionalen Raum in eine Gerade mit Richtung $b_1$ und
eine dazu senkrechte Ebene mit Basis $\{b_2,b_3\}$ aufteilen
kann.
Auf der geraden operiert die Gruppe $C_3$ trivial, der zweidimensionale
Unterraum ist eine zweidimensionale Darstellung der Gruppe $C_3$.
Es ist also gelungen, die reguläre Darstellung von $C_3$ zu zerlegen
in zwei einfachere Darstellungen.

Die reguläre Darstellung ist die direkte Summe der eindimensionalen
trivialen Darstellung und der zweidimensionalen Darstellung durch
Drehmatrizen.
\end{beispiel}.

Durch Wahl einer geeigneten Basis kann also eine Darstellung umgeformt
werden in eine andere.
Die Basistransformation mit der $n\times n$-Transformationsmatrix
$T:\mathbb{R}^n\to\mathbb{R}^n$ führt die 
Darstellung $\varrho\colon G\to\operatorname{GL}_n(\mathbb{R}$ über
in die Darstellung
\[
\tilde{\varrho}
\colon
G\to\operatorname{GL}_n(\mathbb{R})
:
g\mapsto T\varrho(g)T^{-1}.
\]
Tatsächlich ist $\tilde{\varrho}$ ein Homomorphismus, denn wir können
nachrechnen, dass
\begin{align*}
\tilde{\varrho}(e)
&=
T\varrho(e)T^{-1}
=
TIT^{-1}
=
I
\\
\tilde{\varrho}(gh)
&=
T\varrho(gh)T^{-1}
=
T\varrho(g)\varrho(h)T^{-1}
=
T\varrho(g)T^{-1}T\varrho(h)T^{-1}
=
\tilde{\varrho}(g)
\tilde{\varrho}(h).
\end{align*}

\begin{definition}
Zwei $n$-dimensionale Darstellungen $\varrho_1$ und $\varrho_2$
heissen isomorph, wenn es eine reguläre Matrix $T$ gibt derart, dass
$\varrho_1(g)=T\varrho_2(g)T^{-1}$ für alle $g\in G$.
\end{definition}

