%
% 69-orthouni.tex -- Orthogonale und unitäre Darstellungen
%
% (c) 2022 Prof Dr Andreas Müller, OST Ostschweizer Fachhochschule
%

%
% Orthogonale und unitäre Darstellungen
%
\subsection{Orthogonale und unitäre Darstellungen}
Im Beweis der Orthogonalität der Charktere war ein wesentlicher
Schritt die Erkenntnis, dass $\chi(g^{-1})=\overline{\chi(g)}$.
So eine Aussage fehlt uns bis jetzt und verunmöglicht damit, aus
den Sätzen
nicht nur Orthogonalitätsaussagen über die Spur, sondern über 
beliebige Paare von Matrixelementen zu gewinnen.
Eine solche Aussage ist auch nicht zu erwarten, denn die Gruppe
$G=\mathbb{R}$ hat die Darstellungen
\begin{equation}
\begin{array}{rclclclcll}
\varrho_k
&\colon&
\mathbb{R} &\to& \operatorname{GL}_2(\mathbb{R})
&:&
t &\mapsto& \begin{pmatrix} \cos kt&-\sin kt\\\sin kt&\cos kt\end{pmatrix}
&\qquad k\in\mathbb{R}
\\
\psi
&\colon&
\mathbb{R} &\to& \operatorname{GL}_2(\mathbb{R})
&:&
t &\mapsto& \begin{pmatrix} t&0\\0&t^{-1}\end{pmatrix}.
&
\end{array}
\end{equation}
Die Matrizen $\varrho_k$ sind als Drehmatrizen alle orthogonal, die
Matrixelemente sind beschränkt und aus der klassischen Fourier-Theorie
sind Orthogonalitätsrelationen dafür bekannt.
Die Matrixelemente der Darstellung $\psi$ dagegen sind unbeschränkt und
man kann sich keine sinnvollen Orthogonalitätsrelationen vorstellen.

Die Orthogonalitätsrelationen der Matrixelemente von $\varrho_k$ 
im obigen Beispiel scheinen damit zusammenzuhängen, dass die Matrizen
$\varrho_k$ orthogonal sind.

\begin{definition}
Eine Darstellgung $\varrho\colon G\to\operatorname{GL}(V)$ einer
Gruppe in einem reellen Vektorraum $V$ mit einem Skalarprodukt
$\langle\;\,,\;\rangle$ heisst {\em orthogonal}, wenn
$\langle\varrho(g)u\varrho(g)v\rangle=\langle u,v\rangle$
für alle $u,v\in V$.
Eine Darstellgung $\varrho\colon G\to\operatorname{GL}(V)$ einer
Gruppe in einem komplexen Vektorraum $V$ mit einem komplexen
Skalarprodukt $\langle\;\,,\;\rangle$ heisst {\em unitär}, wenn
$\langle\varrho(g)u\varrho(g)v\rangle=\langle u,v\rangle$
für alle $u,v\in V$.
\end{definition}

\begin{beispiel}
Die Menge der reellwertigen Funktionen $\mathbb{R}[G]$ auf einer endlichen
Gruppe ist ein Vektorraum mit dem reellen Skalarprodukt
\[
\langle u,v\rangle
=
\frac{1}{|G|}
\sum_{g\in G} u(g)v(g),
\]
auf der die Gruppe durch die Translation $\varrho(h)=T_h$ wirkt.
Da die Multiplikation $g\mapsto hg$ nur eine Permutation der Gruppenelement
ist, bleibt das Skalarprodukt unter der Translation erhalten.
Somit ist die reguläre Darstellung orthogonal.

Die Menge der komplexwertigen Funktionen $\mathbb{C}[G]$ mit dem
komplexen Skalarprodukt 
\[
\langle u,v\rangle
=
\frac{1}{|G|}
\sum_{g\in G} \overline{u(g)} v(g)
\]
ist eine unitäre Darstellung von $G$ unter der Translation.
\end{beispiel}

\begin{satz}
Sei $\varrho\colon G\to\operatorname{GL}(V)$ eine reelle Darstellung
der endlichen Gruppe $G$ im $n$-dimensionalen Vektorraum $V$, dann gibt
es ein reelles Skalarprodukt $\langle \;\,,\;\rangle$ so, dass
$\varrho$ eine orthogonale Darstellung ist.
\end{satz}

\begin{proof}[Beweis]
Sei $(\;,\;)$ ein beliebiges reelles Skalarprodukt auf dem Vektorraum $V$,
es muss nicht invariant sein, es muss also nicht
\[
(\varrho(g)u,\varrho(g)v)
=
(u,v)
\]
gelten.
Das gemittelte Skalarprodukt
\[
\langle u,v\rangle
=
\frac{1}{|G|}
\sum_{g\in G}
(\varrho(g)u,\varrho(g)v)
\]
hat die Eigenschaft
\[
\langle \varrho(g)u,\varrho(g)v\rangle
=
\langle u,v\rangle,
\]
hat aber auch alle Eigenschaften eines reellen Skalarproduktes:
\begin{enumerate}
\item $\langle\;\,,\;\rangle$ ist symmetrisch:
\[
\langle u,v\rangle
=
\frac{1}{|G|} \sum_{g\in G} (\varrho(g)u,\varrho(g)v)
=
\frac{1}{|G|} \sum_{g\in G} (\varrho(g)v,\varrho(g)u)
=
\langle v,u\rangle
\]
für alle $u,v\in V$.
\item $\langle\;\,,\;\rangle$ ist bilinear:
\begin{align*}
\langle u_1+u_2,v\rangle
&=
\frac{1}{|G|} \sum_{g\in G} (u_1+u_2,v)
=
\frac{1}{|G|} \sum_{g\in G} (u_1,v)
+
\frac{1}{|G|} \sum_{g\in G} (_2,v)
=
\langle u_1,v\rangle + \langle u_2,v\rangle
\\
\langle \lambda u,v\rangle
&=
\frac{1}{|G|} \sum_{g\in G} (\lambda u,v)
=
\lambda
\frac{1}{|G|} \sum_{g\in G} (u,v)
=
\lambda \langle u,v\rangle.
\end{align*}
\item $\langle \;\,;\;\rangle$ ist positiv definit: für $v\in V\setminus\{0\}$
\[
\langle v,v\rangle
=
\frac{1}{|G|} \sum_{g\in G}
\underbrace{(\varrho(g)v,\varrho(g)v)}_{\displaystyle > 0}
>
0.
\]
\end{enumerate}
Somit ist $\langle\;\,,\;\rangle$ ist ein invariantes Skalarprodukt und
die linearen Abbildung $\varrho(g)$ sind bezüglich dieses Skalarproduktes
Orthogonal.
$\varrho(g)$ wird in einer bezüglich $\langle\;\,,\;\rangle$
orthonormierten Basis von $V$ durch eine orthogonale Matrix dargestellt.
\end{proof}

Da also zu einer Darstellung immer ein Skalarprodukt gewählt werden
kann, so dass die Darstellung orthogonal wird, kann man auch eine
orthonormierte Basis wählen und so erreichen, dass die Matrizen
$\varrho(g)$ in dieser Basis orthogonale Matrizen sind.
Eine orthogonale Matrix $O$ hat die Eigenschaft $O^{-1} = O^t$,
für die Matrixelemente $(\varrho(g))_{ik}$ gilt daher
\begin{equation}
\varrho(g^{-1})
=
\varrho(g)^{-1}
\qquad\Rightarrow\qquad
\varrho(g^{-1})_{ik}
=
\varrho(g)_{ki}.
\label{buch:gruppen:darstellungen:eqn:invorth}
\end{equation}
Ausserdem gelten für eine irreduzible Darstellung $\varrho$
weiterhin die Relationen
\eqref{buch:gruppen:darstellungen:eqn:matrixnichtiso}
und
\eqref{buch:gruppen:darstellungen:eqn:matrixiso}.
Setzt man \eqref{buch:gruppen:darstellungen:eqn:invorth} ein, erhält man
\begin{align*}
\frac{1}{n}
\delta_{il}\delta_{jk}
&=
\frac{1}{|G|}
\sum_{g\in G}
\bigl(\varrho(g^{-1})\bigr)_{ij}
\bigl(\varrho(g)\bigr)_{kl}
=
\frac{1}{|G|}
\sum_{g\in G}
\bigl(\varrho(g)\bigr)_{ji}
\bigl(\varrho(g)\bigr)_{kl}
=
\langle (\varrho)_{ji},\varrho_{kl}\rangle.
\end{align*}
Die Matrixelemente einer orthogonalen, irreduziblen Darstellung
sind also orthogonal.
Wir fassen das Resultat im folgenden Satz zusammen.

\begin{satz}
Ist $\varrho\colon G\to\operatorname{GL}(V)$ eine irreduzible,
$n$-dimensionale Darstellung im reellen Vektorraum $V$, dann gibt
es eine Basis von $V$ derart, dass $\varrho(g)$ durch orthogonale
Matrizen beschrieben werden.
Die Matrixelemente $\bigl(\varrho(g)\bigr)_{ik}$ sind orthogonale
Funktionen auf $G$.
\end{satz}

Mit der gleichen Methode kann ein entsprechendes Resultat für komplexe
Darstellungen der Gruppe $G$ gewonnen werden.

\begin{satz}
Sei $\varrho\colon G\to\operatorname{GL}(V)$ eine Darstellung der endlichen
Gruppe $G$ in einem komplexen endlichdimensionalen Vektorraum, dann gibt
es ein komplexes Skalarprodukt auf $V$ so, dass $\varrho$ eine unitäre
Darstellung ist.
\end{satz}

\begin{satz}
Sei $\varrho\colon G\to\operatorname{GL}(V)$ eine unitäre Darstellung
der endlichen Gruppe $G$ im komplexen Vektorraum $V$.
Die Matrixelemente $(\varrho(g))_{ik}$ in einer orthonormierten Basis
von $V$ sind orthogonale Funktionen auf $G$.
\end{satz}

