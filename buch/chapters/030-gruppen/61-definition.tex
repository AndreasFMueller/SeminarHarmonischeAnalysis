%
% 61-definition.tex -- Darstellungen von Gruppen
%
% (c) 2022 Prof Dr Andreas Müller, OST Ostschweizer Fachhochschule
%

%
% Definition
%
\subsection{Definition}
Viele der klassischen Gruppen sind als Mengen von Matrizen definiert.
Man lernt sie meist bereits in der linearen Algebra kennen.
Die Verknüpfung der Gruppenelemente ist die Matrixmultiplikation,
das neutrale Element ist die Einheitsmatrix $I$.
Da sie auch invertierbar sein müssen, liegen sie in der 
allgemeinen linearen Gruppe.
Eine solche Gruppe ist daher eine Teilmenge von $\operatorname{GL}_n(\Bbbk)$
für einen geeignten Körper $\Bbbk$.

Die Permutationsgruppe $S_n$ von $n$ Elementen operiert durch 
Vertauschungen.
Zu jeder Permutation $\sigma\in S_n$ lässt sich die Matrix $P_\sigma$
mit den Matrixelementen
\[
(P_\sigma)_{ik}
=
\begin{cases}
1&\qquad\text{für $i=\sigma(k)$}\\
0&\qquad\text{sonst}.
\end{cases}
\]
konstruieren.
Man kann sofort nachrechnen, dass $P_{\sigma\pi} = P_\sigma P_\pi$,
dass $P_{e}=I$ und $P_{\sigma^{-1}}=P_\sigma^{-1}$.
Die Menge der Matrizen
$P_n=\{P_\sigma\mid\sigma\in S_n\}\subset\operatorname{GL}_n{\mathbb{R}}$
ist eine Gruppe von Matrizen.

Man kann die Relation, dass $S_n$ in Form der Permutationsmatrizen $P_n$
in $\operatorname{GL}_n(\mathbb{R})$ enthalten ist, auch als Homomorphismus
\[
\varrho
\colon 
S_n
\to
\operatorname{GL}(\mathbb{R})
:
\sigma
\mapsto
P_\sigma
\]
verstehen.
Noch etwas allgemeiner definieren wir eine Darstellung einer Gruppe
wie folgt.

\begin{definition}
Sei $G$ eine Gruppe und $V$ ein $\Bbbk$-Vektorraum.
Ein Homomorphismus $\varrho\colon G\to\operatorname{GL}(V)$
heisst {\em Darstellung} der Gruppe $G$ im Vektorraum $V$.
Sie heisst {\em $n$-dimensional}, wenn $V$ ein $n$-dimensionaler
Vektorraum ist.
Hat $V$ eine Basis aus $n$ Vektoren, können die linearen
Abbildungen $\varrho(x)$ als Matrizen in $M_{n\times n}(\Bbbk)$
geschrieben werden.
\end{definition}

%
% Die reguläre Darstellung
%
\subsubsection{Die reguläre Darstellung}
Sei $G$ eine endliche Gruppe und $h\in G$.
Die Verknüpfung mit $g$ definiert eine Permutation der Gruppenelemente
durch
\[
\pi_h \colon G \to G : g \mapsto hg.
\]
Zum Gruppenelement $h$ gehört daher die Permutationsmatrix
$P_{\pi_h}\in\operatorname{GL}_n(\mathbb{R})$.
Jede endliche Gruppe hat daher mindestens die Darstellung durch die
Matrizen $P_{\pi_h}$.
Sie heisst die {\em reguläre Darstellung}.

Indem man die Elemente der Gruppe als $e=g_0,g_1,g_2,\dots,g_{|G|-1}$
nummeriert, kann man auch Vektoren definieren, auf denen die
Gruppe durch die Matrizen $P_{\pi_h}$ operiert.
Sei $v_{g_i}$ der Spaltenvektor, der in der $i$-ten Zeile eine Eins
hat und lauter Nullen in allen anderen Zeilen.
Dann ist $\pi_{h}v_{g_i}=P_{\pi_h}v_{g_i}$, die Vektoren
$v_{g_0},\dots,g_{|G|-1}$ sind die Standardbasis eines $|G|$-dimensionalen
Vektorraumes, auf dem die Gruppe $G$ durch Permutationsmatrizen wirkt.

Die Vektoren $v_h$ kann man auch als die Funktionen
\[
v_h
\colon
G\to\mathbb{R}
:
g\mapsto
\begin{cases}
1&\qquad\text{falls $g=h$}\\
0&\qquad\text{sonst}
\end{cases}
\]
betrachten.
Die Menge der Funktionen $G\to\Bbbk$ bezeichnen wir mit $\Bbbk[G]$.
Die Gruppe $G$ operiert auf den Funktion durch die Translation.
Die Funktion $v_{kh}$ ist genau dann $1$, wenn $g=kh$ ist, also wenn $k^{-1}g=h$
ist.
Dasselbe gilt für die Funktion $g\mapsto v_h(k^{-1}h)$, daher definieren wir
die Translation für beliebige Funktionen $f\colon G\mapsto \mathbb{R}$
durch
\[
T_kf\colon G\to\mathbb{R}:g\mapsto f(k^{-1}g).
\]
Diese Definition deckt sich natürlich mit der
Definition~\ref{buch:gruppen:gruppe:def:translation}.
Die Translation ist also eine Darstellung der Gruppe $G$ auf dem
Vektorraum der Funktionen auf der Gruppe.

Für eine topologische Gruppe $G$ ist die Translation $T_g$ eine 
stetige lineare Abbildung auf dem Vektorraum der stetigen Funktionen
$C_{\mathbb{R}}(G)$, ausserdem ist die Abbildung $g\mapsto T_g$ eine
stetige Abbildung von $G$ in den Vektorraum der linearen Operatoren
auf $C_{\mathbb{R}}(G)$.
Für eine Lie-Gruppe ist $g\mapsto T_g$ sogar eine differenzierbare
Abbildung.
Auch diese Darstellungen von $G$ auf entsprechenden Funktionenräumen
heissen die {\em reguläre Darstellung} von $G$.

%
% Die Standarddarstellung einer Matrizengruppe
%
\subsubsection{Die Standarddarstellung einer Matrizengruppe}
Die reguläre Darstellung ist also eine Art ``universelle Darstellung'',
die man immer konstruieren kann, sie ist allerdings auch etwas kompliziert,
da die Dimension des Vektorraums $V$, auf dem die Gruppenelemente in der
Darstellung wirken, so gross ist wie die Kardinalität der Gruppe.
Von besonderem Nutzen sind Darstellungen, die eine kleine Dimension haben.
Die Elemente einer Gruppe $G\subset\operatorname{GL}_n(\mathbb{R})$
von $n\times n$-Matrizen operiert auf offensichtliche Weise auf einem
$n$-dimensionalen Vektorraum $\mathbb{R}^n$.
Die Untergruppen von $\operatorname{GL}_n(\mathbb{R})$, betrachtet
als abstrakte Gruppen, haben also immer eine $n$-dimensionale Darstellung,
die durch die Einbettung $G\to \operatorname{GL}_n(\mathbb{R})$ entsteht.
Sie heisst die {\em Standarddarstellung}.

%
% Die triviale Darstellung
%
\subsubsection{Die triviale Darstellung}
Für eine beliebige Gruppe $G$ ist die konstante Abbildung
\[
\varrho
\colon
G\to \operatorname{GL}_n(\mathbb{R})
:
g\mapsto I
\]
ein Homomorphismus und damit eine $n$-dimensionale Darstellung von $G$.
Sie heisst die {\em triviale $n$-dimensionale Darstellung}.

%
% Reguläre Darstellung und Faltung
%
\subsubsection{Reguläre Darstellung und Faltung}
Eine Gruppe $G$ operiert auf den Funktionen auf $G$ durch die 
lineare Operation der Translation $T_g$ für $g\in G$.
Sei $G$ eine endliche Gruppe und $f\colon G\mapsto \mathbb{R}$ eine
Funktion auf $G$.
Da die Translation $T_g$ eine lineare Abbildung ist, können wir die 
sie mit den Koeffizienten $T_g$ linear kombinieren.
Damit lässt sich eine Operation der Funktion $f$ auf den Funktionen
$h\colon G\to\mathbb{R}$ durch
\begin{equation}
T_fh
=
\sum_{g\in G} f(g)\cdot T_gh
\colon
G\to\mathbb{R}
:
x\mapsto
\biggl( \sum_{g\in G} f(g)\cdot T_gh\biggr) (x)
=
\sum_{g\in G} f(g)h(g^{-1}x)
=
(f*h)(x)
\end{equation}
definieren.
Die Faltung ist also nichts anderes als die Ausdehnung der regulären
Darstellung von einzelnen Gruppenelementen auf den Vektorraum
der Funktionen auf $G$.

Für eine topologische Gruppe $G$ und eine integrierbare Funktion
$f\colon G\to\mathbb{R}$
lässt sich die Konstruktion ebenfalls durchführen.
Statt einer Summe über die Gruppenelemente muss man allerdings das
Haar-Integral mit den Funktionswerten $f(g)$ als Gewichten verwenden.
Die Funktion $f$ operiert somit auf einer Funktion $h\colon G\to\mathbb{R}$
durch 
\[
T_fh\colon
x\mapsto
\biggl(\int_G f(g) T_gh\,dg\biggr)(x)
=
\int_G f(g) h(g^{-1}x)\,dg
=
(f*g)(x).
\]
Dies zeigt, dass das Studium der reguläre Darstellung nichts anderes
ist als die Untersuchung der Gruppenoperation auf der Gruppe mit Hilfe
der Algebrastruktur, die durch die Faltung erzeugt wird.

