%
% 65-charakter.tex -- Charakter einer Darstellung
%
% (c) 2022 Prof Dr Andreas Müller, OST Ostschweizer Fachhochschule
%

%
% Charakter einer Darstellung
%
\subsection{Charakter einer Darstellung}
Die reguläre Darstellung lässt die Gruppe auf Funktionen auf der
Gruppe operieren.
Der Begriff der irreduziblen Darstellung und das Lemma von Schur
lassen vermuten, dass nur sehr spezielle Funktionen für irreduzible
Darstellungen in Frage kommen.
Es ist daher eine Verbindung zwischen irreduziblen Darstellungen
und Funktionen auf der Gruppe herzustellen.
Diese Funktionen müssen unabhängig sein von der gewählten Basis.

\begin{definition}
\label{buch:gruppen:darstellungen:def:charakter}
Ist $\varrho_V\colon G\to\operatorname{GL}(V)$ eine Darstellung der
Gruppe $G$ im Vektorraum $V$, dann heisst die Funktion
\[
\chi_\varrho
\colon
G
\to
\mathbb{C}
:
g
\mapsto
\tr \varrho_V(g)
\]
der {\em Charakter} der Darstellung $\varrho$.
\end{definition}

Der Charakter einer Darstellung hängt nicht von der Wahl der Basis ab.
Zwei verschiedene Basen im Darstellungsraum $V$ führen auf verschiedene
Matrizen $\varrho_1(g),\varrho_2(g)\in\operatorname{GL}_n(\mathbb{C})$,
die aber durch eine Transformationsmatrix $T$ über die Formel
$\varrho_1(g)=T\varrho_2(g)T^{-1}$ miteinander verbunden sind.
Die Spur dieser Matrizen ist
\[
\tr \varrho_1(g)
=
\tr(T\varrho_2(g)T^{-1})
=
\tr(\varrho_2(g)T^{-1}T)
=
\tr \varrho_2(g),
\]
die Spuren sind also gleich.

Nicht jede Funktion kann ein Charakter sein, wie der folgende Satz
zeigt.

\begin{satz}
\label{buch:gruppen:darstellungen:satz:chareigenschaften}
Sei $\varrho$ eine $n$-dimensionale Darstellung von $G$ dann gilt
\begin{enumerate}
\item $\chi_\varrho(e) = n$
\item $\chi_\varrho(hgh^{-1}) = \chi_\varrho(g)$
\end{enumerate}
\end{satz}

\begin{proof}[Beweis]
Da $\varrho(e)=I_n$ die $n$-dimensionale Einheitsmatrix ist, ist
$\chi_\varrho(e) = \tr \varrho(e) = \tr I_n = n$, was 1.~beweist.
Aussage~3.~folgt aus
$\chi_\varrho(hgh^{-1})
=
\tr (\varrho(h)\varrho(g)\varrho(h)^{-1})
=
\tr \varrho(g)
=
\chi_\varrho(g)
$.
\end{proof}

Der Charakter einer Darstellung ist nur für eindimensionale Darstellungen
ein Homomorphismus, es ist also im Allgemeinen nicht möglich $\chi(g^{-1})$,
aus $\chi(g)$ zu berechnen.
Für endliche Gruppen gibt es aber einen Weg.

\begin{satz}
\label{buch:gruppen:darstellung:satz:charg-1}
Sei $\varrho$ eine $n$-dimensionale Darstellung einer endlichen Gruppe,
dann gilt
$\chi_\varrho(g^{-1}) = \overline{\chi_\varrho(g)}$.
\end{satz}

\begin{proof}[Beweis]
Wir verwenden eine Basis, in der die Matrix $\varrho(g)$ \JN hat.
Die \JN ist eine obere Dreiecksmatrix mit den Eigenwerten
$\lambda_i$ auf der Diagonalen.
Da spätestens die $n$-te Potenz von $g$ das neutrale Element ist und damit
die $n$-te Potenz der Matrix $\varrho(g)$ die Einheitsmatrix ist, müssen
alle Eigenwerte Betrag $1$ haben.
Die inverse Matrix $\varrho(g)^{-1}$ ist ebenfalls eine Dreiecksmatrix
mit den reziproken Eigenwerten $\lambda_i^{-1}$ auf der Diagonalen.
Da die $\lambda_i$ Betrag $1$ haben, ist $\lambda_i^{-1}=\overline{\lambda}_i$
und damit
\[
\chi_\varrho(g^{-1})
=
\tr \varrho(g^{-1})
=
\sum_{i=1}^n \lambda_i^{-1}
=
\sum_{i=1}^n \overline{\lambda}_i
=
\overline{\sum_{i=1}^n \lambda_i}
=
\overline{\tr \varrho(g)}
=
\overline{\chi_\varrho(g)}.
\]
\end{proof}

Der Beweis hängt davon ab, dass die Eigenwerte alle den Betrag $1$ haben,
eine Eigenschaft, die für eine endliche Gruppe dank der \JN leicht zu
erschliessen war.
Wir werden später sehen, dass man in vielen Fällen auch zeigen kann,
dass die Matrizen $\varrho(g)$ alle orthogonal oder unitär sind, was
ebenfalls zur Folge hat, dass die Eigenwerte Betrag $1$ haben.
Schliesslich kann auch die Einschränkung auf beschränkte Funktionen
ebenfalls Eigenwerte mit Betrag verschieden von $1$ ausschliessen.

%
%  Charaktere und Rechenoperationen mit Darstellungen
%
\subsubsection{Charaktere und Rechenoperationen mit Darstellungen}
In Abschnitt~\label{buch:gruppen:darstellungen:subsection:rechnen-mit-darstellungen}
wurden Rechenoperationen mit Darstellungen definiert.
Die direkte Summe $\varrho_1\oplus\varrho_2$ einer $n_1$-dimensionalen
Darstellung $\varrho_1$ und einer $n_2$-dimensionalen Darstellung
$\varrho_2$ ist eine $n_1+n_2$-dimensionale Darstellung.
Die Spur der Matrix~\eqref{buch:gruppen:darstellungen:eqn:dirsummatrix}
ist die Summe der Spuren der Teilmatrizen und daher
\[
\chi_{\varrho_1\oplus\varrho_2}
=
\chi_{\varrho_1}
+
\chi_{\varrho_2}.
\]

Wenn sich eine Darstellung in eine Summe von irreduziblen Darstellungen
zerlegen lässt, dann ist der Charakter der Darstellung die Summe der
Charaktere der irreduziblen Darstellungen.
Dies klingt wie eine Art harmonischer Analysis für Darstellungen: der
Charakter einer beliebigen Darstellung lässt sich zerlegen in eine
Linearkombination von Charakteren irreduzibler Darstellungen.

Um aus den Charakteren tatsächlich eine harmonische Analysis zu konstruieren,
müssen die Charaktere der irreduziblen Darstellungen bezüglich eines
geeigneten Skalarprodukts als Orthogonal erkannt werden.
Damit würde es möglich, beliebige Darstellungen einfach dadurch in
irreduzible Darstellungen zu zerlegen, dass man Skalarprodukte des
Charakters mit den Charakteren der irreduziblen Darstellungen bildet.

