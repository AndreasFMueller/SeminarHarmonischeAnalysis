%
% 66-mittelung.tex -- Mittelung und mittelbare Gruppen
%
% (c) 2022 Prof Dr Andreas Müller, OST Ostschweizer Fachhochschule
%

%
% Mittelung und mittelbare Gruppen
%
\subsection{Mittelung und mittelbare Gruppen}
Harmonische Analysis vergleicht Funktionen mit Hilfe eines Skalarproduktes.
Für lokalkompakte topologische Gruppen, zu denen auch die endlichen Gruppen
zählen, haben wir das Haar-Mass, mit dem sich ein translationsinvariantes
Skalarprodukt definieren lässt.
Das Haar-Mass kann aber auch verwendet werden, um Mittelwerte zu bilden
und neue Darstellungen zu konstruieren.

Sei $f$ eine Funktion auf der Gruppe $G$ mit Werten in einem Vektorraum.
Die Gruppe operiert auf den Funktionen mit der Translation.
Eine Gruppe $G$ heisst {\em mittelbar}, wenn diese Operation gemittelt
werden kann, wenn es also zu $f$ einen Mittelwert $Mf$ gibt, der 
translationsinvariant ist.
Falls $f$ bereits eine translationsinvariante Funktion ist, dann 
ist $Mf=f$.
Für eine endliche Gruppe kann man als Mittelwert einer Funktion
$f\colon G\to V$ das arithmetische Mittel
\[
Mf
=
\frac{1}{|G|}
\sum_{g\in G} T_gf
\]
verwenden.
Für eine lokal kompakte topologische Gruppe kann man die Mittelung
mit Hilfe des haarschen Masses als
\[
Mf
=
\int_G T_gf\,dg
\]
definieren, wobei dafür noch ein paar analytische Feinheiten der
Konvergenz geklärt werden müssten.
Um diesen Schwierigkeiten aus dem Weg zu gehen, werden wir im folgenden
die Aussagen nur für endliche Gruppen beweisen, auch wenn sie für
lokalkompakte topologische Gruppen mit kleinen Anpassungen ebenfalls
gültig sind.

Als erstes Beispiel für diese Idee zeigen wir, wie sich Darstellungen 
zerlegen lassen.
Wir verwenden dazu, dass Teilräume durch Projektionen beschrieben
werden können.

\begin{definition}
\label{buch:gruppen:darstellungen:def:projektion}
Eine {\em Projektion} in einem Vektorraum $V$ ist eine lineare Abbildung
$P\colon V\to V$ mit der Eigenschaft $P^2=P$.
\end{definition}

\begin{satz}
\label{buch:gruppen:darstellungen:satz:projektion}
Sei $\varrho\colon G\to\operatorname{GL}(V)$  eine endlichdimensionale
Darstellung der Gruppe $G$ und $W$ ein invarianter Unterraum von $V$,
also ein Unterraum, für den $\varrho(g)W\subset W$ für alle $g\in G$ gilt.
Dann gibt es einen invarianten Unterraum $W'$ mit $V=W\oplus W'$.
\end{satz}

\begin{proof}[Beweis]
Sei $W_0$ irgend ein komplementärer Unterraum mit $V=W\oplus W_0$, er muss
nicht invariant sein.
Zu dieser Zerlegung gibt es eine Projektionsabbildung $P\colon V\to V$ mit
$\operatorname{ker}P=W_0$ und $\operatorname{im}P=W$ ist.
Auch die Abbildung $P$ muss nicht mit der Darstellung vertauschen.
Der Mittelwert 
\[
P'
=
\frac{1}{|G|}
\sum_{g\in G}
\varrho(g) P \varrho(g)^{-1}
\]
ist wieder eine lineare Abbildung.
Das Bild eines Vektors $v$ unter $P\varrho(g)^{-1}$ liegt in $W$ und wird
von $\varrho(g)$ wieder in $W$ abgebildet.
Da $W$ ein Unterraum von $V$ ist, ist somit $P'v\in W$.

Für Vektoren $w\in W$ hat $P$ die Eigenschaft $Pw=w$.
Daraus berechnet man
\[
P'w
=
\frac{1}{|G|}
\sum_{g\in G}
\varrho(g) P\varrho(g)^{-1}w
=
\frac{1}{|G|}
\sum_{g\in G}
\varrho(g) \varrho(g)^{-1}w
=
\frac{1}{|G|}
\sum_{g\in G}
w
=
w.
\]
Somit ist $P'$ eine Projektion $P^{\prime 2}=P'$ auf $W$.

Die Projektion $P'$ vertauscht aber zusätzlich mit der Darstellung, wie
wir jetzt nachrechnen wollen.
Es ist zu zeigen, dass $P'\varrho(h) = \varrho(h)P'$ für alle
Gruppenelemente $h\in G$.
Da die Multiplikation $g\mapsto hg$ eine Permutation der Gruppenelemente
ist, gilt auch
\begin{align*}
\varrho(h)P'\varrho(h)^{-1}
&=
\varrho(h)
\biggl(\frac{1}{|G|}\sum_{g\in G} \varrho(g)P\varrho(g)^{-1}\biggr)
\varrho(h)^{-1}
\\
&=
\frac{1}{|G|}
\sum_{g\in G} \varrho(hg)P\varrho(hg)^{-1}
\\
&=
\frac{1}{|G|}
\sum_{g\in G} \varrho(g)P\varrho(g)^{-1}
=
P'.
\end{align*}
Durch Multiplikation mit $\varrho(h)$ von rechts ergibt sich die Behauptung.

Somit ist $P'$ eine Projektion von $V$ auf $W$, die mit der Darstellung
vertauscht.
Der Kern von $P'$ ist $W'=\ker P'$ ist ebenfalls invariant, denn
\[
v\in\ker P'
\quad\Rightarrow\quad
P'
\varrho(g)v
=
\varrho(g)P'v
=
\varrho(g)0
=
0
\quad\rightarrow\quad
\varrho(g)v
\in \ker P'.
\]
Es folgt, dass $V=W\oplus W'$ eine komplementäre Zerlegung von $V$ in
invariante Unterräume ist.
\end{proof}

Das Argument kann analog auch für eine lokalkompakte topologische Gruppe
durchgeführt werden.
Die Eigenschaft der Mittelbarkeit zeigt also, dass zu einer Unterdarstellung
immer auch eine komplementäre Darstellung existiert.
Da die reguläre Darstellung immer die triviale Darstellung auf dem
eindimensionalen Unterraum der konstanten Funktionen als invarianten
Unterraum enthält, muss es einen komplementären invarianten Unterraum $W'$
geben derart, dass $\mathbb{C}[G]=\mathbb{C}\oplus W'$ ist.

\begin{satz}
\label{buch:gruppen:darstellungen:satz:abbmittel}
Seien $\varrho_i\colon G\to \operatorname{GL}(V_i)$ zwei irreduzible
Darstellungen von $G$ und $f\colon V_1\to V_2$ eine lineare Abbildung.
Setze
\[
f'
=
\frac{1}{|G|}
\sum_{g\in G} \varrho_2(g)^{-1}\circ f \circ \varrho_1(g).
\]
Dann ist $f'$ eine lineare Abbildung, die mit den Darstellungen vertauscht
und für die gilt:
\begin{enumerate}
\item Wenn die beiden Darstellungen nicht isomorph sind, dann ist $f'=0$.
\item Wenn $V_1=V_2$ und $\varrho_1=\varrho_2$, dann ist
$f'=\frac1n\tr(f)\cdot I_n$
mit $n=\dim V_1$.
\end{enumerate}
\end{satz}

\begin{proof}[Beweis]
Wir rechnen zunächst nach, dass $f'$ mit den Darstellungen vertauscht.
Dazu berechnen wir
\begin{align*}
\varrho_2(h)^{-1}f'\varrho_1(h)
&=
\varrho_2(h)^{-1}
\biggl(
\frac{1}{|G|}
\sum_{g\in G} \varrho_2(g)^{-1}f\varrho_1(g)
\biggr)
\varrho_1(h)
\\
&=
\frac{1}{|G|}
\sum_{g\in G}
\varrho_2(gh)^{-1} f \varrho_1(gh)
\\
&=
\frac{1}{|G|}
\sum_{g\in G}
\varrho_2(g)^{-1} f \varrho_1(g)
=
f',
\end{align*}
wobei im zweitletzten Schritt verwendet wurde, dass die Multiplikation
$g\mapsto gh$ eine Permutation in der Gruppe ist, die den Mittelwert
nicht ändert.
Durch Multiplikation von links mit $\varrho_2(h)$ folgt jetzt
$
f'\varrho_1(h) = \varrho_1(h)f'
$.

Auf die lineare Abbildung $f'$ können wir jetzt das Lemma von Schur
anwenden.
Es besagt, dass $f'$ entweder ein Isomorphismus ist oder eine
Nullabbildung, daraus folgt bereits 1.

Für die Spur von $f'$ gilt
\[
\tr f'
=
\frac{1}{|G|}
\sum_{g\in G} \tr\bigl(\varrho_1(g)^{-1} f\varrho_1(g)\bigr)
=
\frac{1}{|G|}
\sum_{g\in G} \tr f
=
f.
\]
Andererseits sagt das Lemma von Schur, dass $f'=\lambda I_n$ sein muss.
Die Spur davon ist $\tr f = \tr f' = \lambda \tr I_n = n\lambda$ oder
\[
\lambda = \frac1n \tr f
\qquad\Rightarrow\qquad
f' = \frac{1}{n} \tr(f)\cdot I_n
\]
Damit ist 2.~gezeigt.
\end{proof}

