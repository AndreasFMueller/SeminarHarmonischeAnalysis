%
% 1r-funktionenaufg.tex -- Funktionen auf G
%
% (c) 2022 Prof Dr Andreas Müller, OST Ostschweizer Fachhochschule
%

%
% Funktionen auf einer Gruppe
%
\subsection{Funktionen auf einer Gruppe
\label{buch:gruppen:subsection:funktionen}}
In diesem Abschnitt ist $G$ eine Gruppe, die wir multiplikativ
schreiben.
Die harmonische Analysis handelt von der Analyse von Funktionen.
Im Falle einer Lie-Gruppe kann man zusätzlich sinnvoll von Ableitungen
der Funktionen sprechen.
Wir definieren daher

\begin{definition}[Funktionen auf einer Gruppe]
\label{buch:gruppen:gruppe:def:funktionenaufgruppe}
Die Menge der stetigen reell- oder komplexwertigen Funktionen wird mit
$C_{\mathbb{R}}(G)$ bzw.~$C_{\mathbb{C}}(G)$ bezeichnet.
Ist $G$ eine Lie-Gruppe, dann ist
$C_{\mathbb{R}}^\infty(G)$ die Menge der unendlich oft differenzierbaren
reellwertigen Funktionen auf $G$,
$C_{\mathbb{C}}^\infty(G)$ ist die Menge der unendlich oft differenzierbaren
komplexwertigen Funktionen.
\end{definition}

Die Gruppenstruktur ermöglicht, lineare Operatoren auf $C_{\mathbb{R}}(G)$
und $C_{\mathbb{C}}(G)$ zu definieren.

\begin{definition}[Translation]
\label{buch:gruppen:gruppe:def:translation}
Für $s\in G$ ist $T_s$ die Abbildung
\[
T_s
\colon
C_{\mathbb{R}}(G) \to C_{\mathbb{R}}(G)
:
f \mapsto T_sf
\quad
\text{mit}
\quad
(T_sf)(x) = f(s^{-1}x).
\]
Sie heisst die {\em Translation} um $s\in G$.
\end{definition}

Die Translation ist natürlich ein linearer Operator auf den Funktionen, denn
es gilt
\begin{align*}
(T_s(f+g))(x)
&=
(f+g)(s^{-1}x)
\\
&=
f(s^{-1}x) + g(s^{-1}x)
=
(T_sf)(x) + (T_sg)(x)
&&\Rightarrow&
T_s(f+g)&=T_sf+T_sg
\\
(T_s(\lambda f))(x)
&=
\lambda f(s^{-1}x)
=
\lambda (T_sf)(x)
&&\Rightarrow&
T_s\lambda f
&=
\lambda T_sf.
\end{align*}

%
% Eigenvektoren von T_s
%
\subsubsection{Eigenfunktionen des Translationsoperators}
Tatsächlich wurden in früheren Kapiteln Funktionen verwendet, die
bezüglich der Translation besondere Eigenschaften hatten.
Zum Beispiel sind die Funktionen $f(x)=e_k(x)=e^{ikx}$ auf $G=\mathbb{R}$
Eigenfunktionen des Translationsoperators, denn
\[
(T_se_k)(x)
=
e^{ik(x-s)}
=
e^{iks}e^{ikx}
=
e^{-iks} e_k(x).
\]
Insbesondere ist $e_k$ eine Eigenfunktion von $T_s$ mit Eigenwert
$\lambda=e^{-iks}$, also $T_se_k = \lambda e_k$.

%
% Gruppenstruktur der Translationen
%
\subsubsection{Gruppenstruktur der Translationen}
Wir berechnen die Zusammensetzung zweier Translationen ist $T_s$ und $T_t$.
Um $T_sT_t$ zu berechnen, muss zunächst die Funktion $T_tf$ bestimmt werden.
Es ist $(T_tf)(x) = f(t^{-1}x)$.
Die Translation $T_sg$ einer beliebigen Funktion auf dem Element $y\in G$
ist $(T_sg)(y)=g(s^{-1}y)$.
Setzt man $g=T_tf$ ein, ergibt sich
\[
(T_sT_tf)(x)
=
(T_tf)(s^{-1}x)
=
f(t^{-1}s^{-1}x)
=
f((st)^{-1}x)
=
(T_{st}f)(x),
\]
also $T_sT_t=T_{st}$.
Die Zusammensetzung zweier Translationen ist wieder eine Translation.
Auf die gleiche Art kann man auch $T_{s^{-1}}=T_s^{-1}$ zeigen.
Die Abbildung $s\to T_s$ ist daher sogar ein Homomorphismus von $G$
in die Gruppe der umkehrbaren linearen Selbstabbildungen
des Vektorraumes $C(G)$ der stetigen Funktionen, oder
$T_s\in\operatorname{GL}(C(G))$.

%
% Rechtsoperation der Gruppe auf 
%
\subsubsection{Rechtsoperation von $G$ auf $C(G)$}
Die Operation $T_s$ wird etwas genauer als die {\em Linkstranslation}
\index{Linksoperation}
bezeichnet, da die Gruppenoperation auf das Argument von links wirkt.
Für eine abelsche Gruppe spielt die Reihenfolge der Operanden keine
Rolle, für eine nichtabelsche Gruppe ergibt sich jedoch ein Unterschied.

\begin{definition}[Rechtsoperation]
\index{Rechtsoperation}
Der Operator $R_s\colon C(G)\to C(G)$ der Rechtstranslation ist definiert
durch
\[
R_s
\colon
C_{\mathbb{R}}(G)\to C_{\mathbb{R}}(G)
:
f \mapsto R_sf
\quad\text{mit}\quad
(R_sf)(x) = f(xs).
\]
\end{definition}

Die Zusammensetzung von $R_s$ und $R_t$ kann ganz ähnlich wie für
$T_s$ und $T_t$ berechnet werden.
Zunächst ist $R_sg(y) = g(ys)$.
Wendet man dies auf $g=R_tf$ mit $g(x)=(R_tf)(x)=f(xt)$ an, bekommt man
\[
(R_sR_tf)(x)
=
(R_sg)(x)
=
g(xs)
=
(R_tf)(xs)
=
f(xst)
=
(R_{st}f)(x)
\]
oder kurz $R_sR_t=R_{st}$.
Auch die Rechtsoperation ist also ein Homomorphismus
$G\to\operatorname{GL}(C(G))$.



