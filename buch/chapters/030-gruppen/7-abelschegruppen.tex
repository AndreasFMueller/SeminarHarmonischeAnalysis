%
% 7-abelschegruppen.tex
%
% (c) 2022 Prof Dr Andreas Müller, OST Ostschweizer Fachhochschule
%
\section{Abelsche Gruppen
\label{buch:gruppen:section:abelschegruppen}}
\kopfrechts{Abelsche Gruppen}
Sei $G$ jetzt eine abelsche Gruppe und $\varrho\colon G\to\operatorname{GL}(V)$
eine irreduzible Darstellung von $G$.
Die Darstellung $\varrho(g)$ ist eine lineare Abbildung $V\to V$ mit
der Eigenschaft
\[
\varrho(g)\circ\varrho(h) = \varrho(h)\circ\varrho(g)
\]
für alle Gruppenelemente $h\in G$.
Nach Satz~\ref{buch:gruppen:darstellung:satz:lemmavonschur}
folgt, dass $\varrho(g)=\lambda I$ ist, dass also alle Vektoren
von $V$ Eigenvektoren zum Eigenwert $\lambda$ sind.
Um irreduzible Darstellungen in $V$ zu finden, kann man also nach
Eigenvektoren suchen.

Wenden wir diese Idee auf die reguläre Darstellung von $G$ an.
In diesem Fall ist $\varrho(g)$ die Translation $T_g$ von Funktionen.
Die Schlussfolgerung des vorangegangenen Absatzes ist dann, dass die
irreduziblen Darstellungen von $G$ gefunden werden können, indem man
translationsinvariante Funktionen auf $G$ sucht.
Tatsächlich sind die in früher durchgerechneten Beispielen gefundenen
Basisfunktionen der irreduziblen Darstellungen jeweils translationsinvariant
gewesen.

Für eine abelsche Lie-Gruppe lässt sich die Idee noch etwas weiter
entwickeln.
Zu jedem Tangentialvektor $X$ im neutralen Element lässt sich eine
Kurve von Gruppenelementen $g(t)=\exp tX$ konstruieren.
Ist $\varrho\colon G\to\operatorname{GL}(V)$ eine irreduzible Darstellung
von $G$, dann ist $\varrho(g(t))=\lambda(g(t)) I$ für jeden Wert von $t$.
Dann ist aber auch die Ableitung ein Vielfaches der identischen Abbildung
und alle Vektoren von $V$ sind Eigenvektoren der Ableitung.
Angewendet auf die reguläre Darstellung folgt, dass die Basisfunktionen
der irreduziblen Darstellungen Eigenfunktionen des Ableitungsoperators
in Richtung $X$ sind.

\begin{beispiel}
Sei $G=\mathbb{R}/2\pi\mathbb{Z}$ die kompakte Lie-Gruppe der Winkel.
Um irreduzible Darstellungen in der regulären Darstellung finden, such
wir nach Funktionen, die invariant sind unter der Ableitung.
Gesucht sind also $2\pi$-periodische Funktionen
$f\colon\mathbb{R}\to\mathbb{C}$, die die Differentialgleichung
\[
f'(t) =\lambda f(t)
\]
erfüllt.
Die Lösungen können aber unmittelbar angegeben werden, sie sind
\[
f(t) = Ce^{\lambda t}.
\]
Die einzigen solchen Funktionen, die auch $2\pi$-periodisch sind, 
gehören zu $\lambda = ik$ mit $k\in \mathbb{Z}$.
Damit haben wir wieder die Basisfunktionen der klassischen Fourier-Theorie
gefunden.
\end{beispiel}


