%
% 64-schur.tex -- Irreduzible Darstellungen und das Lemma von Schur
%
% (c) 2022 Prof Dr Andreas Müller, OST Ostschweizer Fachhochschule
%

%
% Irreduzible Darstellungen und das Lemma von Schur
%
\subsection{Irreduzible Darstellungen und das Lemma von Schur}
Das Beispiel~\ref{buch:gruppen:darstellung:bsp:c3} hat gezeigt, dass
es möglich ist, die reguläre Darstellung der Gruppe $C_3$ bis auf
Isomorphie in eine direkte Summe zweier einfacherer Darstellungen
zu zerlegen.
Ganz ähnlich wie uns die harmonische Analysis in die Lage versetzt,
Funktionen in Summen von einfacheren Funktionen zu zerlegen, sollte
es auch möglich sein, beliebige Darstellungen in eine Summe von
Bausteinen zu zerlegen.
Dazu muss aber zunächst der Idee der Zerlegbarkeit einer Darstellung
eine eindeutige Definition gegeben werden.

Die Zerlegung der Darstellung von $C_3$ im 
Beispiel~\ref{buch:gruppen:darstellung:bsp:c3} war möglich, weil
der Vektorraum $\mathbb{R}^3$ in zwei Summanden zerlegt werden 
konnte, die unter der Wirkung Darstellung unverändert sind.
Die Ebene im Beispiel kann aber nicht mehr weiter in eindimensionale
Unterräume aufgespalten werden.

\begin{definition}
\label{buch:gruppen:darstellung:def:irreduzibel}
Eine Darstellung $\varrho\colon G\to \operatorname{GL}_n(\mathbb{R})$ 
heisst {\em irreduzibel}, wenn die einzigen Unterräume, die von allen
Matrizen $\varrho(g)$ in sich abgebildet werden, der Nullraum $\{0\}$
und der ganze Raum $\mathbb{R}^n$ sind.
\end{definition}

Nach dieser Definition ist die reguläre Darstellung von $C_3$ nicht
irreduzibel, da die beiden Unterräume
\[
\mathbb{R}b_1%\begin{pmatrix}1\\1\\1\end{pmatrix}
\qquad\text{und}\qquad
\{
xb_2 + yb_3
\mid
x,y\in\mathbb{R}
\}
\]
beide invariant sind.

\begin{satz}[Lemma von Schur]
Seien $G$ eine Gruppe sein $\varrho_V\colon G\to \operatorname{GL}(V)$
und $\varrho_W\colon G\to\operatorname{W}$ irreduzible Darstellungen
von $G$ in den Vektorräumen $V$ und $W$.
Sei ausserdem $f\colon V\to W$ eine lineare Abbildung, die mit den
Darstellungen vertauscht, also
\[
f\circ \varrho_V(g) = \varrho_W(g)\circ f
\]
Dann gilt
\begin{enumerate}
\item $f$ ist entweder die Nullabbildung oder ein Isomorphismus.
\item Falls $V=W$ ist, hat $f$ die Form $f(v)=\lambda v$ mit
$\lambda\in \mathbb{C}$.
\end{enumerate}
\end{satz}

\begin{proof}[Beweis]
Der Kern von $f$ ist der Unterraum
$\operatorname{ker}f = \{v\in V\mid f(v)=0\}$.
Für $v\in\operatorname{ker}f$ ist
$f(\varrho_V(g)v) = \varrho_W(g) f(v) = \varrho_W(g) 0=0$, also ist
$\varrho_V(g)\in\operatorname{ker}f$.
Der Kern von $f$ ist damit ein invarianter Unterraum der Darstellung
$\varrho_V$ von $G$.
Da $\varrho_V$ irreduzibel ist, ist $\operatorname{ker}f$ entweder
der ganze Raum, in diesem Fall ist $f=0$ oder der Kern ist $0$, in diesem
Fall ist $f$ injektiv.

Das Bild von $f$ ist der Unterraum
$\operatorname{im}f = \{f(v)\mid v\in V\}\subset W$.
Dann gilt
$\varrho_W(g)f(v) = f(\varrho_V(g)v)\in\operatorname{im}f$, das Bild von
$f$ ist damit auch ein invarianter Unterraum von $W$.
Da die Darstellung $\varrho_W$  irreduzibel ist, ist $\operatorname{im}f$
entweder der Nullraum, in diesem Fall ist $f=0$, oder das Bild ist der
ganze Raum $W$, in diesem Fall ist $f$ surjektiv.

Die beiden Aussagen zusammen ergeben, dass $f$ entweder die Nullabbildung
ist oder sowohl injektiv wie auch surjektiv ist.
Damit ist 1.~gezeigt.

Bleibt noch 2.~zu zeigen.
Sei $\lambda$ ein Eigenwert der Abbildung $f$ und $v$ ein Eigenvektor.
Dann ist $\lambda I$ eine
Abbildung $V\to V$, die mit der Darstellung vertauscht, denn
\[
\varrho_V(g)\lambda I =
\lambda \varrho_V(g) = \lambda I\varrho_V(g).
\]
Somit ist auch $f-\lambda I$ eine Abbildung $V\to V$, die mit $\varrho_V$
vertauscht.
Nach 1.~muss dies die Nullabbildung oder ein Isomorphismus sein.
Da aber der Eigenvektor $v$ auf $(f-\lambda I)v = \lambda v - \lambda v = 0$
abgebildet wird, kann $f-\lambda I$ nicht ein Isomorphismus sein, somit
ist $f-\lambda I=0$ und damit $f=\lambda I$.
\end{proof}

Die erste Aussage des Lemmas von Schur und sein Beweis zeigen, dass
irreduzible Darstellungen auch nicht teilweise aufeinander abgebildet
werden können.
Wenn eine Abbildung nicht die Nullabbildung ist, dann muss die
ganz Darstellung $V$ ohne Verlust auf die ganze Darstellung $W$
abgebildet werden.
Dies ist nur eine andere Formulierung für die Idee der Irreduzibilität.

Die zweite Aussage des Lemmas von Schur besagt im Wesentlichen, dass
es bis auf Faktor $\lambda$ nur eine Möglichkeit gibt, eine irreduzible
Darstellung mit sich selbst zu vergleichen.

