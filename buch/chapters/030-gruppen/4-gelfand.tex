%
% 4-gelfand.tex
%
% (c) 2022 Prof Dr Andreas Müller
%
\section{Charaktere und Gelfand-Transformation
\label{buch:gruppen:section:gelfand}}
\kopfrechts{Charaktere und Gelfand-Transformation}
In diesem Abschnitt suchen wir nach weiteren Möglichkeiten, einen
Zusammenhang zwischen der Algebra der Funktionen auf $G$ mit der Faltung
als Multiplikationsoperation und der Gruppe zu rekonstruieren.

%
% Charaktere der Faltungsalgebra
%
\subsection{Charaktere der Faltungsalgebra}
Die Faltung ist eine ziemlich komplizierte Operation.
Der Vektorraum der Funktionen auf $G$ ist ein unendlichdimensionaler
Vektorraum.
Die Unübersichtlichkeit der Faltung und die Dimension kann etwas
vorborgen werden, indem man komplexwertige Abbildungen untersucht,
die möglichst viel der Struktur erhalten.

\begin{definition}
Ein Vektorraum $A$ heisst {\em Algebra}, wenn aus auf $A$ eine bilineare
und assoziative Multiplikation $A\times A\to A$.
Wenn auf dem Vektorraunm $A$ ausserdem eine Norm definiert ist, dann
heisst $A$ eine normierte Algebra, wenn die Multiplikation die
Ungleichung
\[
\| xy \| \le \|x\|\cdot \|y\|\quad\text{für alle $x,y\in A$}.
\]
\end{definition}

\begin{definition}
Sei $A$ eine normierte Algebra.
Ein Charakter $\chi$ ist eine lineare Abbildung $\chi\colon A\to\mathbb{C}$,
die auch ein stetiger Algebrahomomorphismus ist, d.~h.~es gilt
$\chi(xy)=\chi(x)\chi(y)$.
\end{definition}

Als Beispiel betrachten wir den Vektorraum $C([a,b])$ der stetigen Funktionen
auf dem Intervall $[a,b]$ mit der Supremum Norm.
$C([a,b])$ ist aber auch eine Algebra, da Funktionen punktweise
multipliziert werden können.
Wenn $f,g\in C([a,b])$ zwei stetige Funktionen sind, dann ist
$fg\colon [a,b]\to\mathbb{C}:x\mapsto f(x)g(x)$ ein bilineares und
stetigs Produkt.
Für jeden Punkt $x\in[a,b]$ ist
\[
\chi_x
\colon
C([a,b] \to \mathbb{C}
:
f\mapsto f(x)
\]
eine stetige lineare Abbildung.
Sie ist aber auch ein Algebrahomomorphismus.
Dazu muss man nachrechnen, dass $\chi_x(fg)=\chi_x(f)\chi_y(f)$ ist.
Einsetzen der Definition von $\chi_x$ ergibt
\[
\chi_x(fg)
=
(fg)(x)
=
f(x)g(x)
=
\chi_x(f)\chi_x(g).
\]
Es gibt also zu jedem Punkt des Definitionsbereichs einen Charakter
der Algebra $C([a,b])$.

Man kann zeigen, dass die Homomorphismen von der Form $\chi_x$ 
die einzigen Homomorphismen $A\to\mathbb{C}$ sind.
Dies bedeutet, dass sich der Definitionsbereich der Algebra der
stetigen Funktionen vollständig aus der Algebra rekonstruiert werden 
kann.
Der Unterschied ist aber, dass die $[a,b]$ einfach nur eine Menge ist,
während $C([a,b])$ eine reichhaltige Algebrastruktur hat.

\begin{definition}
\label{buch:gruppen:gelfand:def:spektrum}
Ist $A$ eine Algebra, dann heisst die Menge
\[
\mathrm{X}(A)
=
\operatorname{Hom}(A,\mathbb{C})
\]
der Algebrahomomorphismen von $A$ nach $\mathbb{C}$ heisst das
{\em Spektrum} der Algebra $A$.
\end{definition}

Das Spektrum der Algebra der stetigen Funktionen auf einem Intervall
enthält die Charaktere $\chi_x$ mit $x\in [a,b]$, das Intervall
$[a,b]$ ist eine Teilmenge von $\mathbb{X}(C([a,b]))$.

%
% Homomorphismen G \to C^*
%
\subsection{Homomorphismen $G\to \mathbb{C}^*$}
Sei jetzt $G$ eine Gruppe mit einem Haarschen Mass und der Faltung 
von Funktionen mit kompaktem Träger.
Da die Funktionen alle in $L^2(G)$ drin sind, muss eine lineare
Abbildung nach dem Darstellungssatz von Riesz als ein Skalarprodukt
geschrieben werden können.
Für einen Charakter $\chi\colon L^2(G)\to\mathbb{C}$ muss es also eine
Funktion $\omega\colon G\to\mathbb{C}$ geben derart, dass
\begin{equation}
\langle \omega, f*g\rangle
=
\langle \omega, f\rangle
\langle \omega, g\rangle
\label{buch:gruppen:gelfand:eqn:omegahomo}
\end{equation}
gilt.
Aus dieser Bedingung lässt sich ableiten, dass $\omega$ sehr spezielle
Eigenschaften haben muss.

Seien $s,t\in G$ zwei Elemente der Gruppe.
Sei ausserdem $f$ eine Funktion, die nur in einer kleinen Umgebung 
des neutralen Elementes von $0$ verschieden ist.
Im Skalarprodukt
\[
\langle \omega, T_{s}f\rangle
=
\int_{G} \overline{\omega(x)} f(s^{-1}x) \,dx
\]
ist der Integrand nur für $x$ in unmittelbarer Nähe von $s$ 
$0$ verschieden.
Indem man die Funktion so verändert, dass der Träger kleiner wird,
dann konvergiert das Skalarprodukt gegen den Wert $\omega(s)$.
Aus \eqref{buch:gruppen:gelfand:eqn:omegahomo} folgt dann
\[
\omega(st) = \omega(s)\omega(t).
\]
Die Charaktere der Faltungsalgebra sind also genau die Homomorphismen
$G\mapsto\mathbb{C}^*$ von der Gruppe $G$ in die multiplikative
Gruppe von $\mathbb{C}$.

\begin{beispiel}
Sei $G= \mathbb{R}/2\pi\mathbb{Z}$ die Gruppe der Winkel mit der 
Addition von Winkeln als Gruppenoperation.
Um die Charaktere der Faltungsalgebra der $2\pi$-periodischen Funktionen
zu bestimmen, müssen Homomorphismen $G\to\mathbb{C}$ gefunden werden.
Ist $\omega$ ein solcher Homomorphismus, dann folgt aus
$h=\omega(\frac{2\pi}{n}$, $\omega(2\pi/n)^n = \omega(2\pi)=\omega(0)=1$.
Daher muss $h$ eine komplexe Zahl vom Betrag $1$ sein, die man als
$h=e^{it}$ schreiben kann.
Somit ist $\omega(2\pi k/n) = e^{ikt}$ eine Exponentialfunktion.
Da wir auch gefordert haben, das $\omega$ stetig sein muss folgt,
dass alle Homomorphismen von der Form $x\mapsto e^{ikx}$.
Da $\omega(2\pi)=1$ sein muss, folgt ausserdem, dass $k\in \mathbb{Z}$
ist.
Die Menge der Homomorphismen $G\to\mathbb{R}$ ist daher $\mathbb{Z}$ und
die Menge der Charaktere der Faltungsalgebra ist ebenfalls $\mathbb{Z}$.
\end{beispiel}


%
% Die Gelfand-Transformation
%
\subsection{Die Gelfand-Transformation}
Das Spektrum der Algebra $C([a,b])$ ist das Intervall $[a,b]$,
Es ist also möglich, aus der Algebra mit einer für alle Algebren
durchführbaren Prozedur eine Funktionenalgebra zu machen.
Den Funktionswert $f(x)$  kann man auch als Wert eines 
Algebrahomomorphismus $\chi_x(f) = f(x)$ bekommen.
Dies ist die Motivation für die folgende Definition.

\begin{definition}
Ist $A$ eine Algebra, dann ist die Gelfand-Transformation
die Abbildung
\[
\mathscr{G}
\colon
A \to C(\mathbb{X}(A))
:
a
\mapsto \mathscr{G}a = \hat{a}
\colon \chi \mapsto \hat{a}(\chi) = \chi(a).
\]
\end{definition}

\begin{satz}
Die Gelfand-Transformation ist ein Homomorphismus von Algebren.
\end{satz}

\begin{proof}
Es ist zu zeigen, dass $\mathscr{G}$ linear ist, ein Algebrahomomorphismus
und ausserdem stetig ist.
\begin{itemize}
\item
Linearität:
\begin{align*}
\mathscr{G}(\lambda a+\mu b) (\chi)
&=
\chi( \lambda a + \mu b)
=
\lambda\chi(a) + \mu\chi(b)
=
\lambda(\mathscr{G}a)(\chi)
+
\mu(\mathscr{G}b)(\chi)
\\
\Rightarrow\qquad
\mathscr{G}(\lambda a+ \mu b)
&=
\lambda\mathscr{G}a + \mu\mathscr{G}b.
\end{align*}
Somit ist $\mathscr{G}$ linear.
\item
Algebrahomomorphismus:
\begin{align*}
\mathscr{G}(ab)(\chi)
&=
\chi(ab)
=
\chi(a)\chi(b)
=
(\mathscr{G}a)(\chi)
(\mathscr{G}b)(\chi)
\\
\Rightarrow\qquad
\mathscr{G}(ab)
&=
\mathscr{G}a\cdot \mathscr{G}b.
\end{align*}
\item
\end{itemize}
\end{proof}

Sei jetzt $G$ eine Gruppe und $\mathscr{K}(G)$ die Algebra der
stetige Funktionen mit kompaktem Träger mit der Faltung.
Die Gelfand-Transformation macht aus einer Funktion auf $G$ 
eine Funktion auf dem Spektrum von $\mathscr{K}(G)$.
Ist $\chi\in\mathbb{X}(\mathscr{K}(G))$, dann hat die Gelfandtransformation
von $f\in\mathscr{K}(G)$ den Wert
\[
(\mathscr{G}f)(\chi) = \chi(f).
\]
Da die Gelfand-Transformatioin ein Algebrahomomorophismus ist,
muss  für jeden Homomorphismus $\chi$
\[
\mathscr{G}(f*g)(\chi)
=
(\mathscr{G}f\cdot \mathscr{G}g) (\chi)
=
\mathscr{G}f(\chi)
\mathscr{G}g(\chi)
\]
gelten.
Als Funktion auf $\mathbb{X}(\mathscr{K}(G))$ gilt für die
Gelfand-Transformtion daher
\[
\mathscr{G}(f*g)
=
\mathscr{G}f
\cdot
\mathscr{G}g,
\]
die Gelfand-Transformation macht also aus der Faltung die
gewöhnliche Multiplikation von Funktionen.

Für eine nichtabelsche Gruppe $G$ ist die Faltungsalgebra nicht
kommutativ.
Für zwei Funktionen $f$ und $g$ auf $G$ gilt jetzt aber
\[
\mathscr{G}(f*g)
=
\mathscr{G}f\cdot\mathscr{G}g
=
\mathscr{G}g\cdot\mathscr{G}f
=
\mathscr{G}(g*f).
\]
Die Gelfand-Transformation ignoriert also die Tatsache, dass im allgemeinen
$f * g\ne g*f$ ist.
Eine andere Möglichkeit, dies auszudrücken ist, dass der Kommutator
$f*g-g*f$ im Kern der Gelfand-Transformation liegt, also
\[
\mathscr{G}(f*g-g*f)=0.
\]

Wir wissen auch bereits, dass ein Algebrahomomorphismus
$\chi\colon \mathscr{K}(G)\to\mathbb{C}$ als Skalarprodukt mit
einem Gruppenhomomorphismus 
$\omega\colon G\to\mathbb{C}^*$ geschrieben werden kann.
Die Gelfand-Transformierte hat daher den Wert
\[
(\mathscr{G}f)(\chi)
=
\chi(f)
=
\langle \omega ,f\rangle
=
\int_G \overline{\omega(x)}\,f(x)\,dx.
\]

\begin{beispiel}
Für die Gruppe $\mathbb{R}/2\pi\mathbb{Z}$ der Winkel haben wir die
Menge der Homomorphismen bereits bestimmt, sie ist die Menge der
ganzen Zahlen $k\in\mathbb{Z}$ und der zugehörige Gruppenhomomorphismus
ist $\omega_k(x) = e^{ikx}$.
Die Gelfand-Transformation einer Funktion $f$ ist daher eine Funktion
auf den Zahlen $k\in \mathbb{Z}$ definiert durch
\[
(\mathscr{G}f)(k)
=
\hat{f}(k)
=
\langle \omega_k f\rangle
=
\int_{G/2\pi \mathbb{Z}} \overline{\omega_k(x)}\, f(x) \,dx
=
\int_0^{2\pi} \overline{e^{ikx}} f(x)\,dx
=
\int_0^{2\pi} e^{-ikx} f(x)\,dx.
\]
Die Werte der Fourier-Transformation auf $k\in\mathbb{Z}$ sind bis auf einen
Normierungsfaktor die Fourier-Koeffizienten der Funktion $f$.
\end{beispiel}

Die Gelfand-Transformation ist also eine Verallgemeinerung der
Fourier-Transformation, die automatisch auf eine Art von harmonischer
Analyse von Funktionen auf einer Gruppe führt.
Im Beispiel sind die Funktione $\omega_k$ bezüglich des
Skalarproduktes orthogonal, aber im Allgemeinen wissen wir dies 
noch nicht.
Wir werden in Abschnitt~\ref{buch:gruppen:section:darstellung}
auf diese Frage zurückkommen.



%
% Die duale Gruppe
%
\subsection{Die duale Gruppe
\label{buch:gruppen:gelfand:subsection:dual}}


