%
% 4-gelfand.tex
%
% (c) 2022 Prof Dr Andreas Müller
%
\section{Charaktere und Gelfand-Transformation
\label{buch:gruppen:section:gelfand}}
\kopfrechts{Charaktere und Gelfand-Transformation}
In diesem Abschnitt suchen wir nach weiteren Möglichkeiten, einen
Zusammenhang zwischen der Algebra der Funktionen auf $G$ mit der Faltung
als Multiplikationsoperation und der Gruppe zu rekonstruieren.

%
% Charaktere der Faltungsalgebra
%
\subsection{Charaktere der Faltungsalgebra}
Die Faltung ist eine ziemlich komplizierte Operation.
Der Vektorraum der Funktionen auf $G$ ist ein unendlichdimensionaler
Vektorraum.
Die Unübersichtlichkeit der Faltung und die Dimension kann etwas
vorborgen werden, indem man komplexwertige Abbildungen untersucht,
die möglichst viel der Struktur erhalten.

\begin{definition}
Ein Vektorraum $A$ heisst {\em Algebra}, wenn aus auf $A$ eine bilineare
und assoziative Multiplikation $A\times A\to A$.
Wenn auf dem Vektorraunm $A$ ausserdem eine Norm definiert ist, dann
heisst $A$ eine normierte Algebra, wenn die Multiplikation die
Ungleichung
\[
\| xy \| \le \|x\|\cdot \|y\|\quad\text{für alle $x,y\in A$}.
\]
\end{definition}

\begin{definition}
Sei $A$ eine normierte Algebra.
Ein Charakter $\chi$ ist eine lineare Abbildung $\chi\colon A\to\mathbb{C}$,
die auch ein stetiger Algebrahomomorphismus ist, d.~h.~es gilt
$\chi(xy)=\chi(x)\chi(y)$.
\end{definition}

Als Beispiel betrachten wir den Vektorraum $C([a,b])$ der stetigen Funktionen
auf dem Intervall $[a,b]$ mit der Supremum Norm.
$C([a,b])$ ist aber auch eine Algebra, da Funktionen punktweise
multipliziert werden können.
Wenn $f,g\in C([a,b])$ zwei stetige Funktionen sind, dann ist
$fg\colon [a,b]\to\mathbb{C}:x\mapsto f(x)g(x)$ ein bilineares und
stetigs Produkt.
Für jeden Punkt $x\in[a,b]$ ist
\[
\chi_x
\colon
C([a,b] \to \mathbb{C}
:
f\mapsto f(x)
\]
eine stetige lineare Abbildung.
Sie ist aber auch ein Algebrahomomorphismus.
Dazu muss man nachrechnen, dass $\chi_x(fg)=\chi_x(f)\chi_y(f)$ ist.
Einsetzen der Definition von $\chi_x$ ergibt
\[
\chi_x(fg)
=
(fg)(x)
=
f(x)g(x)
=
\chi_x(f)\chi_x(g).
\]
Es gibt also zu jedem Punkt des Definitionsbereichs einen Charakter
der Algebra $C([a,b])$.

Man kann zeigen, dass die Homomorphismen von der Form $\chi_x$ 
die einzigen Homomorphismen $A\to\mathbb{C}$ sind.
Dies bedeutet, dass sich der Definitionsbereich der Algebra der
stetigen Funktionen vollständig aus der Algebra rekonstruiert werden 
kann.
Der Unterschied ist aber, dass die $[a,b]$ einfach nur eine Menge ist,
während $C([a,b])$ eine reichhaltige Algebrastruktur hat.

%
% Homomorphismen G \to C^*
%
\subsection{Homomorphismen $G\to \mathbb{C}^*$}
Sei jetzt $G$ eine Gruppe mit einem Haarschen Mass und der Faltung 
von Funktionen mit kompaktem Träger.
Da die Funktionen alle in $L^2(G)$ drin sind, muss eine lineare
Abbildung nach dem Darstellungssatz von Riesz als ein Skalarprodukt
geschrieben werden können.
Für einen Charakter $\chi\colon L^2(G)\to\mathbb{C}$ muss es also eine
Funktion $\omega\colon G\to\mathbb{C}$ geben derart, dass
\begin{equation}
\langle \omega, f*g\rangle
=
\langle \omega, f\rangle
\langle \omega, g\rangle
\label{buch:gruppen:gelfand:eqn:omegahomo}
\end{equation}
gilt.
Aus dieser Bedingung lässt sich ableiten, dass $\omega$ sehr spezielle
Eigenschaften haben muss.

Seien $s,t\in G$ zwei Elemente der Gruppe.
Sei ausserdem $f$ eine Funktion, die nur in einer kleinen Umgebung 
des neutralen Elementes von $0$ verschieden ist.
Im Skalarprodukt
\[
\langle \omega, T_{s}f\rangle
=
\int_{G} \overline{\omega(x)} f(s^{-1}x) \,dx
\]
ist der Integrand nur für $x$ in unmittelbarer Nähe von $s$ 
$0$ verschieden.
Indem man die Funktion so verändert, dass der Träger kleiner wird,
dann konvergiert das Skalarprodukt gegen den Wert $\omega(s)$.
Aus \eqref{buch:gruppen:gelfand:eqn:omegahomo} folgt dann
\[
\omega(st) = \omega(s)\omega(t).
\]
Die Charaktere der Faltungsalgebra sind also genau die Homomorphismen
$G\mapsto\mathbb{C}^*$ von der Gruppe $G$ in die multiplikative
Gruppe von $\mathbb{C}$.

\begin{beispiel}
Sei $G= \mathbb{R}/2\pi\mathbb{Z}$ die Gruppe der Winkel mit der 
Addition von Winkeln als Gruppenoperation.
Um die Charaktere der Faltungsalgebra der $2\pi$-periodischen Funktionen
zu bestimmen, müssen Homomorphismen $G\to\mathbb{C}$ gefunden werden.
Ist $\omega$ ein solcher Homomorphismus, dann folgt aus
$h=\omega(\frac{2\pi}{n}$, $\omega(2\pi/n)^n = \omega(2\pi)=\omega(0)=1$.
Daher muss $h$ eine komplexe Zahl vom Betrag $1$ sein, die man als
$h=e^{it}$ schreiben kann.
Somit ist $\omega(2\pi k/n) = e^{ikt}$ eine Exponentialfunktion.
Da wir auch gefordert haben, das $\omega$ stetig sein muss folgt,
dass alle Homomorphismen von der Form $x\mapsto e^{ikx}$.
Da $\omega(2\pi)=1$ sein muss, folgt ausserdem, dass $k\in \mathbb{Z}$
ist.
Die Menge der Homomorphismen $G\to\mathbb{R}$ ist daher $\mathbb{Z}$ und
die Menge der Charaktere der Faltungsalgebra ist ebenfalls $\mathbb{Z}$.
\end{beispiel}

%
% Die duale Gruppe
%
\subsection{Die duale Gruppe}

%
% Die Gelfand-Transformation
%
\subsection{Die Gelfand-Transformation}

