Die Gruppe $S_3$ der Permutationen von $3$ Objekten enthält sechs
Permutationen.
Man kann sie alle schreiben als Zusammensetzungen von zwei Permutationen:
\[
\tau = \begin{pmatrix}1&2&3\\2&1&3\end{pmatrix}
\qquad\text{und}\qquad
\sigma = \begin{pmatrix}1&2&3\\2&3&1\end{pmatrix}.
\]
Für $\sigma$ und $\tau$ gelten die Rechenregeln
\[
\tau^2=e\;\text{oder}\;\tau^{-1}=\tau,
\qquad
\sigma^3=e\;\text{oder}\;\sigma^{-1}=\sigma^2,
\qquad
\sigma\tau
=
\begin{pmatrix}1&2&3\\3&2&1\end{pmatrix}
=
\tau\sigma^2.
\]
Damit kann man nachrechnen, dass die Gruppenelement als
\begin{align*}
\begin{pmatrix} 1&2&3\\1&2&3 \end{pmatrix} &= e &
\begin{pmatrix} 1&2&3\\2&3&1 \end{pmatrix} &= \sigma &
\begin{pmatrix} 1&2&3\\3&1&2 \end{pmatrix} &= \sigma^2 \\
\begin{pmatrix} 1&2&3\\2&1&3 \end{pmatrix} &= \tau &
\begin{pmatrix} 1&2&3\\1&3&2 \end{pmatrix} &= \tau\sigma &
\begin{pmatrix} 1&2&3\\3&2&1 \end{pmatrix} &= \tau\sigma^2
\end{align*}
Dargestellt werden können.
Die Rechenregeln in der Gruppe können auch aus der Multiplikationstabelle
\[
\begin{tabular}{|>{$}c<{$}|>{$}c<{$}>{$}c<{$}>{$}c<{$}>{$}c<{$}>{$}c<{$}>{$}c<{$}|}
\hline
\cdot        &e            &\sigma      &\sigma^2    &\tau        &\tau\sigma  &\tau\sigma^2\\
\hline
e            &e            &\sigma      &\sigma^2    &\tau        &\tau\sigma  &\tau\sigma^2\\
\sigma       &\sigma       &\sigma^2    &e           &\tau\sigma^2&\tau        &\tau\sigma  \\
\sigma^2     &\sigma^2     &e           &\sigma      &\tau\sigma  &\tau\sigma^2&\tau        \\
\tau         &\tau         &\tau\sigma  &\tau\sigma^2&e           &\sigma      &\sigma^2    \\
\tau\sigma   &\tau\sigma   &\tau\sigma^2&\tau        &\sigma^2    &e           &\sigma       \\
\tau\sigma^2 &\tau\sigma^2 &\tau        &\tau\sigma  &\sigma      &\sigma^2    &e            \\
\hline
\end{tabular}
\qquad
\begin{tabular}{|>{$}l<{$}|>{$}c<{$}>{$}c<{$}>{$}c<{$}>{$}c<{$}>{$}c<{$}>{$}c<{$}|}
\hline
x     & e & \sigma   & \sigma^2 & \tau & \tau\sigma & \tau\sigma^2 \\
\hline
x^{-1}& e & \sigma^2 & \sigma   & \tau & \tau\sigma & \tau\sigma^2 \\
\hline
\end{tabular}
\]
abgelesen werden.
Als Skalarprodukt wird 
\[
\langle f,g\rangle
=
\sum_{x\in S_3} \overline{f(x)}g(x)
\]
verwendet.
\begin{teilaufgaben}
\item
Rechnen Sie nach, dass die beiden Funktionen $e_\sigma\colon S_3\to\mathbb{R}$
und $e_\tau\colon S_3\to\mathbb{R}$ mit den Funktionswerten
\[
e_\sigma(x)
=
\begin{cases}
1&\quad \text{für $x=\sigma$}\\
0&\quad \text{sonst}
\end{cases}
\qquad\text{und}\qquad
e_\tau(x)
=
\begin{cases}
1&\quad \text{für $x=\tau$}\\
0&\quad \text{sonst}
\end{cases}
\]
orthonormiert sind.
\item
Für eine endliche Gruppe ist die Faltung gegeben durch
\begin{equation}
(f*g)(x) = \sum_{y\in G} f(y)g(y^{-1}x).
\label{buch:gruppen:301:faltung}
\end{equation}
Berechnen Sie die beiden Faltungen $e_\sigma * e_\tau$ und
$e_\tau* e_\sigma$.
\item
Es gibt zwei Homomorphismen $S_3\to\mathbb{R}^*$, nämlich
\[
h_0\colon S_3 \to \mathbb{R}^* : x \mapsto 1
\]
und die Funktion $h_1$ mit den Funktionswerten
\[
\begin{tabular}{|>{$}c<{$}|>{$}c<{$}>{$}c<{$}>{$}c<{$}>{$}c<{$}>{$}c<{$}>{$}c<{$}|}
\hline
    x &e &\sigma &\sigma^2 &\tau &\tau\sigma &\tau\sigma^2 \\
\hline
h_1(x)&1 &1      &1        &-1   &-1         &-1 \\
\hline
\end{tabular}
\]
Berechnen Sie $(\mathscr{G}f)(h_0)=\langle h_0,f\rangle$
und $(\mathscr{G}f)(h_1)=\langle h_1,f\rangle$
für die Funktionen $e_\sigma$, $e_\tau$, $e_\sigma*e_\tau$
und $e_\tau*e_\sigma$.
\item
Kann die Gelfand-Transformation die beiden Funktionen
$e_{\tau\sigma}$ und $e_{\tau\sigma^2}$ unterscheiden?
\item
Prüfen Sie nach, dass
\[
\mathscr{G}(e_\sigma*e_\tau)
=
\mathscr{G}e_\sigma
\cdot
\mathscr{G}e_\tau
=
\mathscr{G}(e_\tau*e_\sigma)
\]
\end{teilaufgaben}

\begin{loesung}
\begin{teilaufgaben}
\item
Die Skalarprodukte sind
\begin{align*}
\|e_\sigma\|^2
&=
\sum_{x\in S_3} |e_\sigma(x)|^2
=
0+6+0+0+0+0 =1\\
\|e_\tau\|^2
&=
\sum_{x\in S_3} |e_\tau(x)|^2
=
0+0+0+6+0+0 =1\\
\langle e_\sigma,e_\tau\rangle
&=
\sum_{x\in S_3} e_\sigma(x)e_\tau(x)
=
0+\!\sqrt{6}\cdot 0+0\cdot\!\sqrt{6}+0+0 = 0.
\end{align*}
Die Funktionen sind also orthonormiert.
\item
Wir verwenden die Formel \eqref{buch:gruppen:301:faltung}.
Das Produkt in der Summe kann nur dann von $0$ verschieden sein,
wenn sowohl $f(y)$ als auch $g(y^{-1}x)$ von $0$ verschieden sind.
Für die Funktionen $f=e_\sigma$ und $g=e_\tau$ bedeutet das, dass
nur für $y=\sigma$ und $y^{-1}x=\tau$ die Faltung von $0$ verschieden 
ist, also für $x=y\tau=\sigma\tau=\tau\sigma^2$.
Die Faltung ist daher
\[
(e_\sigma*e_\tau)(x)
=
e_{\tau\sigma^2}(x)
=
\begin{cases}
1&\quad\text{für $x=\tau\sigma^2$}\\
0&\quad\text{sonst.}
\end{cases}
\]
Auf die gleiche Art kann auch die Faltung $e_\tau*e_\sigma$ berechnet
werden, sie ist nur für $y=\tau$ und $y^{-1}x=\sigma$ und damit
für $x=y\sigma=\tau\sigma$ von $0$ verschieden, d.~h.
\[
(e_\tau * e_\sigma)
=
e_{\tau\sigma}(x)
=
\begin{cases}
1&\quad\text{für $x=\tau\sigma$}\\
0&\quad\text{sonst.}
\end{cases}
\]
Insbesondere sind die beiden Faltungen verschieden.
Das Faltungsprodukt ist nicht kommutativ, weil auch die Gruppe
$S_3$ nicht kommutativ ist.
\item
Für die Funktion $h_0(x)$ ist klar, dass
$1h_0(xy)=h_0(x)h_0(y)=1\cdot1$ ist.
Die Funktion $h_1(x)$ hat genau dann den Wert $-1$, wenn das
Element $x$ ein $\tau$ enthält.
Die Anzahl der $\tau$ in einem Produkt ist die Summe
der Anzahlen der $\tau$ in den Faktoren modulo 2.
Daraus folgt sofort, dass auch $h_1$ ein Homomorphismus ist.
\item
Die beiden Funktionen $e_{\tau\sigma}$ und $e_{\tau\sigma^2}$ 
haben gleich viele $\tau$, daher haben ihre Gelfand-Transfor\-mierten
die gleichen Werte auf $h_1$.
Sie lassen sich daher nicht unterscheiden.
\item
Die Gelfand-Transformation $\mathscr{G}f$ hat auf $h_0$
hat immer den Mittelwert der Funktionswerte als Wert.
Es folgt
\[
\left.
\begin{aligned}
\mathscr{G}e_\sigma(h_0)         &= 1 & \mathscr{G}e_\sigma(h_1)         &=  1 \\
\mathscr{G}e_\tau(h_0)           &= 1 & \mathscr{G}e_\tau(h_1)           &= -1 \\
\mathscr{G}e_{\tau\sigma}(h_0)   &= 1 & \mathscr{G}e_{\tau\sigma}(h_1)   &= -1 \\
\mathscr{G}e_{\tau\sigma^2}(h_0) &= 1 & \mathscr{G}e_{\tau\sigma^2}(h_1) &= -1 \\
\end{aligned}
\;
\right\}
\quad\Rightarrow\quad
\left\{
\begin{aligned}
\mathscr{G}(e_\sigma*e_\tau) &= \mathscr{G}e_\sigma\cdot \mathscr{G}e_\tau \\
\mathscr{G}(e_\tau*e_\sigma) &= \mathscr{G}e_\tau\cdot \mathscr{G}e_\sigma 
\end{aligned}
\right.
\]
die ``Fourier''-Faltungsformel.
\end{teilaufgaben}
\end{loesung}
