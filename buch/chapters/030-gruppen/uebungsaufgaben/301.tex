Die Gruppe $\mathbb{Z}/6\mathbb{Z} = \{0,1,2,3,4,5\}$  
ist kommutativ.
Zu jedem Gruppenelement gibt es eine Funktion
\[
e_k(x)
=
\begin{cases}
1&\quad\text{für $x=k$}\\
0&\quad\text{sonst.}
\end{cases}
\]
Wir verwenden das Skalarprodukt
\[
\langle f,g\rangle
=
\sum_{k=0}^5 \overline{f(x)} g(x)
\]

\begin{teilaufgaben}
\item
Rechnen Sie nach, dass die Funktionen $e_k$ orthonormiert sind.
\item
Berechnen Sie die Faltung $e_k*e_l$.
\item
Es gibt sechs Homomorphismen von $\mathbb{Z}/6\mathbb{Z}$ nach $\mathbb{C}^*$,
nämlich
\[
h_k
\colon
\mathbb{Z}/6\mathbb{Z} \to \mathbb{C}^*
:
x\mapsto e^{ikx}
\]
für $k=0,1,\dots,5$.
Rechnen Sie die Homomorphismuseigenschaft nach.
\item
Berechnen Sie die Werte der Gelfand-Transformation
$(\mathscr{G}e_l)(h_k)$.
\item
Überprüfen Sie die Fourier-Faltungsformel
\begin{equation}
\mathscr{G}(e_x*e_y)
=
\mathscr{G}(e_x)\cdot\mathscr{G}(e_y)
\label{buch:gruppen:301:gelfand-produkt}
\end{equation}
\end{teilaufgaben}

\begin{loesung}
\begin{teilaufgaben}
\item
Das Skalarprodukt von $e_k$ und $e_l$ ist
\[
\langle e_k,e_l\rangle
=
\sum_{x=0}^5 e_k(x)e_l(x)
=
\begin{cases}
1&\quad\text{für $k=l$}\\
0&\quad\text{sonst,}
\end{cases}
\]
d.~h.~die Funktionen sind orthonormiert.
\item
Die Faltung ist die Funktion
\[
(e_k*e_l)(x)
=
\sum_{y\in G} e_k(y)e_l(x-y).
\]
Sie kann nur dann von $0$ verschieden sein, wenn sowohl 
$e_k(y)$ als auch $e_l(x-y)$ von $0$ verschieden sind, also
für $k=y$ und $x-y=l$ oder $x=k+l$.
Somit ist $e_k*e_l=e_{k+l}$.
Insbesondere ist das Faltungsprodukt kommutativ, da auch die
Gruppe kommutativ ist.
\item
Es ist
\[
h_k(x+y)=e^{ik(x+y)} = e^{ikx}e^{iky} = h_k(x)h_k(y),
\]
also ist $h_k$ ein Homomorphismus.
\item
Der Wert der Gelfand-Transformation ist
\[
(\mathscr{G}e_l)(h_k)
=
\langle h_k,e_l\rangle
=
\sum_{x=0}^5 \overline{h_k(x)} e_l(x)
=
\sum_{x=0}^5 e^{-ikx}\delta_{lx}
=
e^{-ikl}.
\]
\item
Wir müssen nachprüfen, dass die Gleichung
\eqref{buch:gruppen:301:gelfand-produkt}
auf jedem $h_k$ gilt.
Aus $(\mathscr{G}e_x)(h_k)=e^{-ikx}$ und
$e_x*e_y=e_{x+y}$
folgt
\begin{align*}
(\mathscr{G}(e_x*e_y))(h_k)
=
(\mathscr{G}e_{x+y})(h_k)
=e^{-ik(x+y)}
\\
(\mathscr{G}e_x)(h_k)
\cdot
(\mathscr{G}e_y)(h_k)
=
e^{-ikx}\cdot e^{-iky}
=
e^{-ik(x+y)},
\end{align*}
die beiden Seiten der Gleichung
\eqref{buch:gruppen:301:gelfand-produkt}
stimmen also tatsächlich überein.
\end{teilaufgaben}
\end{loesung}
