%
% 67-relationen.tex -- Algegraische Relationen für die Matrixelemente
%
% (c) 2022 Prof Dr Andreas Müller, OST Ostschweizer Fachhochschule
%

%
% Algebraische Relationen für die Matrixelemente
%
\subsection{Algebraische Relationen für die Matrixelemente}
Das Lemma von Schur hat noch mehr zu bieten.
Man kann damit nämlich sogar über die einzelnen Matrixelemente
einer Darstellung eine Aussage machen.
Ist $\varrho\colon G\to \operatorname{GL}_n(\mathbb{R})$ eine 
$n$-dimensionale Darstellung der Gruppe $G$, dann ist $\varrho(g)$
eine $n\times n$-Matrix bestehend aus den Matrixelementen
$(\varrho(g))_{ik}$ für $i,k=1,\dots,n$.

\begin{satz}
\label{buch:gruppen:darstellungen:satz:matrixnichtiso}
Wenn die Darstellungen $\varrho_i$ nicht isomorph sind, dann gilt
\begin{equation}
\frac{1}{|G|}
\sum_{g} \bigl(\varrho_1(g)^{-1}\bigr)_{i\!j} \bigl(\varrho_2(g)\bigr)_{kl}
=
0
\label{buch:gruppen:darstellungen:eqn:matrixnichtiso}
\end{equation}
für alle $i,j,k,l$.
\end{satz}

\begin{proof}[Beweis]
Wir schreiben die Resultate von
Satz~\ref{buch:gruppen:darstellungen:satz:abbmittel}
in Matrixform.
Die Abbildung $f$ hat die Matrixelemente $(f)_{i\!j}$.
Die Bedingung, dass $f'=0$ ist, bedeutet, dass alle
Matrixelemente $(f')_{il}=0$ sind.
Ausgeschrieben
\begin{align*}
(f')_{il}
&=
\frac{1}{|G|}
\sum_{g\in G}
\sum_{s,t}
\bigl(\varrho_2(g)^{-1}\bigr)_{is} (f)_{st} \bigl(\varrho_1(g)\bigr)_{tl}
\end{align*}
Nach 
Satz~\ref{buch:gruppen:darstellungen:satz:abbmittel}
ist dies immer $=0$, ganz unabhängig davon, was für Werte man für
die Matrix $f$ einsetzt.
Wählt man 
\begin{equation}
(f)_{st}
=
\begin{cases}
1&\qquad\text{$s=j$ und $t=k$}\\
0&\qquad\text{sonst},
\end{cases}
\qquad\text{oder}\qquad
(f)_{st}
=
\delta_{s\!j}\delta_{tk}
\label{buch:gruppen:darstellungen:eqn:fst}
\end{equation}
dann folgt
\[
0
=
\frac{1}{|G|}
\sum_{g\in G} 
\bigl(\varrho_2(g)^{-1}\bigr)_{is}
\delta_{s\!j}
\delta_{tk}
\bigl(\varrho_1(g)\bigr)_{kl}
=
\frac{1}{|G|}
\sum_{g\in G}
\bigl(\varrho_2(g)^{-1}\bigr)_{i\!j}
\bigl(\varrho_1(g)\bigr)_{kl}.
\]
Damit ist die Aussage bewiesen.
\end{proof}

\begin{satz}
\label{buch:gruppen:darstellungen:satz:matrixiso}
Sei $\varrho\colon G\to\operatorname{GL}(V)$ eine $n$-dimensionale
irreduzible Darstellung von $G$.
Dann gilt
\begin{equation}
\frac{1}{|G|}
\sum_{g\in G}
\bigl(\varrho(g^{-1})\bigr)_{i\!j} 
\bigl(\varrho(g))_{kl}
=
\frac1n
\delta_{il}\delta_{jk}
\label{buch:gruppen:darstellungen:eqn:matrixiso}
\end{equation}
\end{satz}

\begin{proof}[Beweis]
Mit der gleichen Notation wie im Beweis von
Satz~\ref{buch:gruppen:darstellungen:satz:matrixnichtiso}
schreiben wir das Resultat 2.~von
Satz~\ref{buch:gruppen:darstellungen:satz:abbmittel}
in Matrixform.
Es ist $f'=\frac1n \tr f$ und somit
\begin{equation}
(f')_{il}
=
\frac1n\tr(f)\delta_{il}
=
\frac{1}{|G|}
\sum_{g\in G}
\sum_{s,t}
\bigl(\varrho(g^{-1})\bigr)_{is}
f_{st}
\bigl(\varrho(g)\bigr)_{tl}.
\label{buch:gruppen:darstellungen:eqn:ff}
\end{equation}
Auch hier kann wieder $f$ beliebig gewählt werden.
Mit der Wahl
\eqref{buch:gruppen:darstellungen:eqn:fst}
$\tr f=\delta_{jk}$
wird

\eqref{buch:gruppen:darstellungen:eqn:ff} zu
\[
\frac1n \delta_{jk} \delta_{il}
=
\frac{1}{|G|}
\sum_{g\in G}
\sum_{s,t}
\bigl(\varrho(g^{-1})\bigr)_{is}
\delta_{s\!j}
\delta_{tk}
\bigl(\varrho(g)\bigr)_{tl}
=
\frac{1}{|G|}
\sum_{g\in G}
\bigl(\varrho(g^{-1})\bigr)_{i\!j}
\bigl(\varrho(g)\bigr)_{kl},
\]
wie behauptet.
\end{proof}

Beide Sätze besagen, dass ein ähnliches Konstrukt wie das Skalarprodukt
angewendet auf zwei verschiedene Matrixelemente immer entweder $0$
ergibt oder den Wert $1/n$.
Das Konstrukt, von dem hier die Rede ist, ist die Bildung
\[
(u,v)
=
\frac{1}{|G|}
\sum_{g\in G}
u(g^{-1}) v(g).
\]
Dies ist jedoch nicht das Skalarprodukt von Funktionen auf der
Gruppe.
Dazu müsste $u(g^{-1})=\overline{u(g)}$ sein, was im Allgemeinen
nicht zutrifft.
Wenn wir aber später zeigen können, dass es im Vektorraum ein Skalarprodukt
gibt derart, dass die Matrizen der Darstellung unitär sind, dann ist
$\bigl(\varrho(g^{1})\bigr)_{ik} = \overline{\varrho(g)_{ki}}$ und
$(u,v)$ stimmt mit dem Skalarprodukt der Matrixelemente überein.


