%
% 5-darstellung.tex -- Darstellungen von Gruppen
%
% (c) 2022 Prof Dr Andreas Müller, OST Ostschweizer Fachhochschule
%
\section{Darstellungen
\label{buch:gruppen:section:darstellung}}
\kopfrechts{Darstellungen}
Für die Gelfand-Transformation wurden beschränkte Homomorphismen
$G\to\mathbb{C}^*$ benötigt.
Da $\mathbb{C}^*$ eine abelsch Gruppe ist, wird ein Homomorphismus
auf $xy$ und $xy$ den gleichen Wert
\[
h(xy) = h(x)h(y) = h(y)h(x) = h(yx)
\]
annehmen.
Wenn die Gruppe $G$ nicht abelsch und $xy\ne yx$ ist, wird der
Unterschied in den Werten von $h$ nicht mehr sichtbar.
Auch die Gelfand-Transformation kann daher nur einen ``kommutative''
Sicht auf die Gruppe $G$ vermitteln.

Diese Schwierigkeit kann überwunden werden, indem als Funktionswerte
von $h$ nicht nur Zahlen zugelassen werden, sondern beliebige
Matrizen.
Man spricht von einer {\em Darstellung} der Gruppe $G$ durch
Matrizen.
Da die Matrizenmultiplikation im Allgemeinen nicht kommutativ
ist, besteht die Möglichkeit, mindestens einen Teil der
Nichtkommutativität der Gruppe $G$ in den Matrizen $h(x)$
wiederzufinden.

%
% Definition
%
\subsection{Definition}

\begin{definition}
Sei $G$ eine Gruppe und $V$ ein $\Bbbk$-Vektorraum.
Ein Homomorphismus $\varrho\colon G\to\operatorname{GL}(V)$
heisst {\em Darstellung} der Gruppe $G$ im Vektorraum $V$.
Sie heisst {\em $n$-dimensional}, wenn $V$ ein $n$-dimensionaler
Vektorraum ist.
Hat $V$ eine Basis aus $n$ Vektoren, können die linearen
Abbildungen $\varrho(x)$ als Matrizen in $M_{n\times n}(\Bbbk)$
geschrieben werden.
\end{definition}

%
% Vergleich von Darstellungen
%
\subsection{Vergleich von Darstellungen}

%
% Irreduzible Darstellungen und das Lemma von Schur
%
\subsection{Irreduzible Darstellungen und das Lemma von Schur}

%
% Mittelung und mittelbare Gruppen
%
\subsection{Mittelung und mittelbare Gruppen}

%
% Orthogonale und unitäre Darstellungen
%
\subsection{Orthogonale und unitäre Darstellungen}

%
% Orthogonalität der Matrixelemente
%
\subsection{Orthogonalität der Matrixelemente}

%
% Charakter einer Darstellung
%
\subsection{Charakter einer Darstellung}

%
% Zerlegung einer Darstellung in irreduzible Darstellungen
%
\subsubsection{Zerlegung einer Darstellung in irreduzible Darstellungen}
