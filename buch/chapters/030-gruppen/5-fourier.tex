%
% 5-fourier.tex
%
% (c) 2023 Prof Dr Andreas Müller
%
\section{Fourier-Transformation
\label{buch:gruppen:section:fourier}}
In diesem Abschnitt soll die Gruppe $\mathbb{R}$ etwas genauer untersucht
werden.
Das Skalarprodukt für Funktionen auf dieser Gruppe ist eine Integral über
einen unendlichen Definitionsbereich, so entsteht das Problem, dass
selbst beschränkte Funktionen ein unendliches Skalarprodukt haben können.
Wir müssen uns daher auf Funktionen beschränken, die nur auf einer
kompakten Teilmenge von $0$ verschieden sind und den Funktionenraum
daraus durch Vervollständigung aufbauen.

Im Gegensatz zur Gruppe der Winkel ist die Gruppe $\mathbb{R}$ nicht kompakt
und die Menge der Homomorphismen ist nicht diskret.
Wir können also nicht mehr unendliche Summen verwenden, um eine Funktion
aus ihren Frequenzkomponenten zu synthetisieren.
Stattdessen müssen wir zu einem Integral übergehen.

%
% Die Gruppe $G=\mathbb{R}$
%
\subsection{Die Gruppe $G=\mathbb{R}$
\label{buch:gruppen:fourier:subsection:gruppeR}}
Die duale Gruppe von $G=\mathbb{R}$ besteht aus den stetigen Homomorphismen
$\chi:\mathbb{R}\to\mathbb{C}^*$.
Solche Funktionen erfüllen die Funktionalgleichung $\chi(a+b)=\chi(a)\chi(b)$.

Ableitung der Funktionalgleichung nach $b$ an der Stelle $b=0$ führt auf
die Differentialgleichung
\[
\chi'(a) = \chi(a)\chi'(0),
\]
die Homomorphismen erfüllen also die Differentialgleichung
\[
\operatorname{const}
=
\frac{\chi'(x)}{\chi(x)} 
=
\frac{d}{dx}\log\chi(x)
\qquad\Rightarrow\qquad
\log\chi(x) = \operatorname{const}\cdot x
\qquad\Rightarrow\qquad
\chi(x) = e^{\alpha x}.
\]
Nur diejenigen Homomorphismen sind beschränkt und können dazu verwendet,
für die die Konstante $\alpha=ik$ imaginär ist, mit $k\in\mathbb{R}$.
Die duale Gruppe ist daher
\[
\hat{G}
=
\{
\chi\colon\mathbb{R}\to\mathbb{C}^*
\mid
\chi(x) = e^{ikx},\quad k\in \mathbb{R}
\}.
\]
Zu jedem Homomorphismus $\chi$ gehört also eine eindeutig bestimmte
Zahl $k\in\mathbb{R}$, die sogenannte {\em Wellenzahl},
\index{Wellenzahl}%
und umgekehrt.

Die Gruppe $G=\mathbb{R}$ ist abelsch, daher erwarten wir, dass 
$\hat{G}$ ebenfalls eine Gruppe ist.
Die Gruppenoperation entsteht nach
Abschnitt~\ref{buch:gruppen:gelfand:subsection:dual} aus der 
Multiplikation der Charaktere.
Seien also $\chi_1$ und $\chi_2$ zwei Homomorphismen
$\mathbb{R}\to\mathbb{C}^*$ und $k_1$ bzw.~$k_2$ die zugehörigen
Wellenzahlen in $\mathbb{R}$, es ist also
\[
\chi_1(x) = e^{ik_1x}
\qquad\text{und}\qquad
\chi_2(x) = e^{ik_2x}.
\]
Das Produkt
\[
\chi_1(x)\chi_2(x)
=
e^{ik_1x}e^{ik_2x}
=
e^{i(k_1+k_2)x}
\]
der beiden Charaktere ist wieder ein Homomorphismus
$\mathbb{R}\to\mathbb{C}^*$, der zur Wellenzahl $k_1+k_2$ gehört.
$\hat{\mathbb{R}}$ ist daher nicht nur als Menge das gleiche wie
die Menge der Wellenzahlen $k\in \mathbb{R}$, sondern auch als
Gruppe mit der Addition also Gruppenoperation.

Da für $\mathbb{R}$ das Lebesque-Mass translationsinvariant ist,
ist auch das Haar-Mass auf der dualen Gruppe $\hat{\mathbb{R}}=\mathbb{R}$
bis auf einen Normierungsfaktor bekannt.
Wir können ihn so wählen, dass die Transformations- und
Rücktransformationsformeln möglichst einfach werden.

%
% Fourier-Transformation
%
\subsection{Fourier-Transformation
\label{buch:gruppen:fourier:subsection:transformation}}
Im Abschnitt~\ref{buch:gruppen:fourier:subsection:dualR} wurde die
duale Gruppe von $G=\mathbb{R}$ bestimmt.
Nach der allgemeinen Theorie der Gelfand-Transformation sind die
Werte der Gelfand-Transformation auf Funktionen mit kompaktem
Träger die Skalarprodukt
\begin{equation}
(\mathscr{G}f)(\chi)
=
\langle \chi,f\rangle
=
\int_{\mathbb{R}} \overline{\chi(x)} f(x)\,dx
=
\int_{\mathbb{R}} e^{-ikx} f(x)\,dx,
\label{buch:gruppen:fourier:ft:integral}
\end{equation}
wenn $k$ die Wellenzahl des Charakteres $\chi(x)=e^{ikx}$ ist.
Das Integral~\eqref{buch:gruppen:fourier:ft:integral} ist nicht nur
für Funktionen mit kompaktem Träger definiert, sondern wegen $|\chi(x)|=1$
für beliebige integrierbare Funktionen.

\begin{definition}[Fourier-Transformation auf $\mathbb{R}$]
Die Fourier-Transformation einer integrierbaren Funktion $f\in L^1(\mathbb{R})$
\index{Fourier-Transformation!auf $\mathbb{R}$}%
ist die Funktion
\[
\mathscr{F}f
\colon
k\mapsto
(\mathscr{F}f)(k)
=
\hat{f}(k)
=
\int_{\mathbb{R}} e^{-ikx}f(x)\,dx.
\]
\end{definition}

%
% Unschärfe-Relation
%
\subsection{Unschärferelation
\label{buch:gruppen:fourier:subsection:unschaerfe}}
Die Unschärferelation 

%
% Dirac-delta-Funktion
%
\subsubsection{Dirac-\textdelta-Funktion}
In der elementaren Theorie der Fourier-Transformation wird oft 
die ``Dirac-\textdelta-Funktion'' $\delta(x)$ eingeführt, die die
Eigenschaft haben soll, dass
\begin{equation}
\int_{-\infty}^\infty \delta(x) f(x)\,dx = f(0)
\label{buch:gruppen:fourier:eqn:deltadef}
\end{equation}
haben soll.
Eine solche Funktion im üblichen Sinne kann es natürlich nicht geben.
Da das Integral nicht von den Werten $f(x)$ mit $x\ne 0$ abhängen darf,
muss $\delta(x)=0$ sind für $x\ne 0$.
Damit kann das Integral nur vom Funktionswert $\delta(0)$ abhängen,
einzelne Werte eines Integrals können aber ein Integral nicht beeinflussen.

Es kann also eine Delta-Funktion nicht geben, aber die Folge von
glatten Funktionen
\[
d_n(x) = \frac{1}{\sqrt{2\pi n}} e^{-x^2/n}
\]
ist eine gute Approximation dafür.
Die Funktionen $d_n$ sind ausserdem in $L^2(\mathbb{R})$ und
$L^2(\mathbb{R})$.
Tatsächlich konvergiert $\langle d_n,f\rangle \to f(0)$.
Dies ist nicht die einzige Möglichkeit, die Idee der 
Dirac-\textdelta-Funktion auf eine solide mathematische Basis zu
stellen.
Welche Vorgehensweise man auch immer wählt, sie produziert ein
mathematisches Objekt $\delta(x)$, welches man wie eine Funktion
mit der ``unmöglichen'' Eigenschaft
\eqref{buch:gruppen:fourier:eqn:deltadef}
behandeln kann.

Die Fourier-Transformierte der Dirac-\textdelta-Funktion ist
\[
(\mathscr{F}\delta)(k)
=
\int_{-\infty}\infty e^{-ikx}\delta(x)\,dx
=
e^{ik\cdot 0}
=
1.
\]
Für eine um $y$ verschobene Dirac-\textdelta-Funktion ist die
Fourier-Transformierte
\[
(\mathscr{F}T_y\delta)(k)
=
\int_{-\infty}^\infty e^{-ikx}\delta(x-y)\,dx
=
\int_{-\infty}^\infty e^{-ik(x'+y)}\delta(x')\,dx'
=
e^{-iky}.
\]
Die Fourier-Transformierte der Dirac-\textdelta-Funktion hat also 
konstant den Betrag $1$, sie ist nicht einmal integrierbar.

%
% Lokalisierung
%
\subsubsection{Lokalisierung}
Die Dirac-\textdelta-Funktion ist ein extremes Beispiel für eine
Funktion, die sehr präzise lokalisiert ist, ihr Träger besteht nur
aus einem einzigen Punkt.
Die Fourier-Transformation von $\delta(x)$ ist dagegen überhaupt
nicht lokalisiert, ganz im Gegenteil, $\mathscr{F}\delta$ hat für alle
Wellenzahlen den gleichen Wert.

Diese Eigenschaften kann man auch bei anderen Funktionen beobachten.
Die Funktion
\[
f_a(x)
=
\begin{cases}
\frac1{\sqrt{2a}}&\qquad -a\le x \le a\\
0&\qquad\text{sonst}.
\end{cases}
\]
ist integrierbar und quadratintegrierbar, es gilt
\[
\|f_a\|_2^2
=
\int_{-\infty}^\infty
|f_a(x)|^2
\,dx
=
\int_{-a}^a \frac{1}{2a} \,dx
=
1.
\]
Die Fouriertransformierte von $f_a$ ist
\begin{align*}
\mathscr{F}f_a
\colon k\mapsto
(\mathscr{F}f_a)(k)
&=
\int_{-\infty}^\infty e^{-ikx} f_a(x)\,dx
=
\frac{1}{\sqrt{2a}}
\int_{-a}^a e^{-ikx}\,dx
\\
&=
\frac{1}{\sqrt{2a}}
\biggl[
\frac{1}{-ik} e^{-ikx}
\biggr]_{-a}^a
=
\frac{1}{\sqrt{2a}}\biggl(
\frac{
e^{-ika}-e^{ika}
}{
-ik
}
\biggr)
\\
&=
\frac{2}{\sqrt{2a}k}\frac{e^{ika}-e^{-ika}}{2i}
=
\sqrt{\frac{2}{a}} \frac{\sin ka}{k}
=
\sqrt{2a} \frac{\sin ka}{ka}
=
\sqrt{2a}
\operatorname{si}(ka).
\end{align*}
Die Funktion
\[
\operatorname{si}(x)
=
\begin{cases}
1&\qquad \text{für $x = 0$}\\
\displaystyle\frac{\sin x}{x}&\qquad\text{sonst}
\end{cases}
\]
ist der nicht normierte Kardinalsinus.

XXX Abbildung

%
% Die Heisenberg-Pauli-Weyl-Ungleichung
%
\subsubsection{Die Heisenberg-Pauli-Weyl-Ungleichung}
Um das in vorangegangenen Beispielen diskutierte Phänomen allgemein
zu fassen, dass eine gut lokalisierte Funktion eine schlecht lokalisierte
Fourier-Transformierte hat, benötigen wir ein Mass für die Lokalisierung.
Die Dirac-\textdelta-Funktion hat perfekte Lokalisierung, sie ist 
konzentriert auf ein Intervall der Länge $0$.
Ebenso sind die Funktionen $f_a$ auf einem Intervall der Länge $2a$
lokalisiert.
Die Länge des Trägerintervalls ist aber ein schlechtes Lokalisierungsmass,
denn der Träger aller Fourier-Transformierten $\mathscr{F}f_a$ ist jeweils
ganz $\mathbb{R}$.

Die Wahrscheinlichkeitsrechnung bietet ein besseres Mass für die
Lokalisierung an.
Mit geeigneter Normierung kann man man $|f(x)|^2$ als eine
Wahrscheinlichkeitsverteilung betrachten.
Die Varianz der Zufallsgrösse mit der Wahrscheinlichkeitsverteilung
$|f(x)|^2$ bietet sich als Lokalisierungsmass an.
Die Voraussetzung der Normierung kann immer erreicht werden, indem man
$f$ durch $f/\|f\|$ ersetzt.


\begin{definition}
Ist $f\in L^2(\mathbb{R}$ und $x|f(x)|^2$ integrierbar, dann sind
die Grössen
\begin{align*}
(\Delta x)^2
&=
\frac{1}{\|f\|^2}
\int_{-\infty}^\infty
(x-\langle x\rangle)^2
|f|^2(x)
\,dx
\\
(\Delta k)^2
&=
\frac{1}{\|\mathscr{F}f\|_2^2}
\int_{-\infty}^\infty
(k-\langle k\rangle)^2
|(\mathscr{F}f)(k)|^2
\,dk
\end{align*}
die Varianz einer Zufallsvariable mit Wahrscheinlichkeitsverteilung
$f^2/\|f\|_2^2$ bzw.~$\mathscr{F}f^2/\|\mathscr{F}f\|_2^2$.
\end{definition}

In der Wahrscheinlichkeitstheorie wird gezeigt, dass der Erwartungswert
$\langle x\rangle$ derjenige Wert der Konstanten $\mu$ in
\begin{equation*}
\frac{1}{\|f\|^2}
\int_{-\infty}^\infty
(x-\mu)^2 
|f(x)|^2
\,dx
\end{equation*}
ist, für den das Integral minimal wird.
Die Grösse $(\Delta x)^2$ ist als das Minimum der Grösse
\begin{equation}
\frac{1}{\|f\|_2^2}
\int_{-\infty}^\infty
(x-\mu)^2 
|f(x)|^2
\,dx
=
\frac{1}{\|f\|_2^2}
\int_{-\infty}^\infty
x^2\, |(T_{-\mu}f)(x)|^2
\,dx.
\label{buch:gruppen:fourier:eqn:translation}
\end{equation}
Da $(\Delta x)^2$ das Minimum des
Integrals~\eqref{buch:gruppen:fourier:eqn:translation} ist, wird
eine Abschätzung dieser Grösse immer auch für $(\Delta x)^2$ gelten.

Durch eine Translation wie $T_{-\mu}$ in
\eqref{buch:gruppen:fourier:eqn:translation}
ändert sich $\mathscr{F}f$ nur um einen Phasenfaktor,
der Erwartungswert und die Varianz von $k$ ändern sich nicht.
Und auch umgekehrt ändert eine Verschiebung der Fourier-Transformierten
$\mathscr{F}f$ entlang der $k$-Achse die Funktion $f$ nur um einen
Phasenfaktor, der sich auch $|f(x)|^2$ nicht auswirkt. 
Für die Beweis der folgenden Aussagen können wir daher wenn nötig annehmen,
dass $\langle x\rangle=0$ und $\langle k\rangle = 0$.

\begin{satz}[Heisenberg-Pauli-Weyl-Unschärferelation]
\label{buch:gruppen:fourier:satz:heisenberg-pauli-weyl}
Sind die Funktion $f(x)$, $xf(x)$ und $k(\mathscr{F}f)(k)$ alle in
$L^2(\mathbb{R})$ und $\lim_{x\to\pm\infty} \sqrt{x}|f(x)|=0$, dann
gilt
\[
(\Delta x)^2
(\Delta k)^2
\ge 
\frac14.
\]
Gleichheit gilt genau für die Gauss-Funktionen $f(x)=Ce^{-ax^2}$, $a>0$.
\end{satz}

\begin{proof}[Beweis]
Wir schreben $F(k) = (\mathscr{F}f)(k)$ und möchten $(\Delta x)^2(\Delta k)^2$
nach unten abschätzen.
Nach der Bemerkung vor dem Satz dürfen wir davon ausgehen, dass die
Erwartungswerte $0$ sind.
Es genügt als  wie folgt zu rechnen:
\begin{align}
\|f\|_2^2
\|F\|_2^2
(\Delta x)^2
(\Delta k)^2
&=
\int_{-\infty}^\infty |x\,f(x)|^2\,dx
\int_{-\infty}^\infty |k\,F(k)|^2\,dk.
\notag
\intertext{Aus der Ableitungseigenschaft $ikF(k) = (\mathscr{F}f')(k)$
der Fourier-Transformation folgt}
&=
\int_{-\infty}^\infty |x\,f(x)|^2\,dx
\int_{-\infty}^\infty |(\mathscr{F}f')(k)|^2\,dk.
\notag
\intertext{Die Fourier-Transformation ändert die $L^2$-Norm nicht, es
folgt daher}
&=
\int_{-\infty}^\infty |x\,f(x)|^2\,dx
\int_{-\infty}^\infty |f'(x)|^2\,dx
=
\|xf(x)\|_2^2\cdot \|f'\|_2^2
.
\notag
\intertext{Nach der Cauchy-Schwarz-Ungleichung kann dieses Produkt
nach unten durch das Skalarprodukt}
&\ge
|\langle xf(x),f'(x)\rangle|
=
\biggl|
\int_{-\infty}^{\infty} xf(x)f'(x)\,dx
\biggr|^2
\label{buch:gruppen:fourier:eqn:cauchyschwarz}
\intertext{abgeschätzt werden.
Allerdings können beide Faktoren noch mit einem beliebigen Phasenfaktor
multipliziert werden oder die Faktoren können komplex konjugiert werden,
ohne dass sich der Wert ändert.
Wir wählen die Kombination}
&=
\biggl|
\int_{-\infty}^{\infty} xf(x)\overline{f'(x)}\,dx.
\biggr|
\notag
\intertext{Verwendet man nur den Realteil des Integranden, wird das Integral
noch kleiner, nämlich}
&\ge
\biggl|
\int_{-\infty}^{\infty} \Re(xf(x)\overline{f'(x)})\,dx
\biggr|^2
\label{buch:gruppen:fourier:eqn:realteil}
\\
&=
\biggl|
\int_{-\infty}^{\infty} \frac12\bigl(
xf(x)\overline{f'(x)}
+
\overline{ xf(x)\overline{f'(x)}}
\bigr)\,dx
\biggr|^2
\notag
\\
&=
\biggl|
\int_{-\infty}^{\infty}
\frac12x
\bigl(
f(x)\overline{f'(x)} + \overline{f(x)}f'(x)
\bigr)
\,dx
\biggr|^2
=
\biggl|
\int_{-\infty}^{\infty}
\frac12x
\frac{d}{dx}\bigl(f(x)\overline{f(x)}\bigr)
\,dx
\biggr|^2
\notag
\\
&=
\frac14 \biggl|
\int_{-\infty}^{\infty}
x\frac{d}{dx}(|f(x)|^2)
\biggr|^2
\notag
\intertext{Partielle Integration ergibt}
&=
\frac14
\biggl|
\biggl[
x|f(x)|^2
\biggr]_{-\infty}^\infty
-
\int_{-\infty}^{\infty}
|f(x)|^2
\,dx
\biggr|^2.
\notag
\intertext{Nach Voraussetzung verschwindet der erste Term.
Das Integral im zweiten Term ist das Quaddrat der $L^2$-Norm, also}
&=
\frac14\|f\|_2^4.
\notag
\intertext{Verfolgen wird die gesamte Ungleichungskette, finden wir}
\|f\|_2^2
\|F\|_2^2
(\Delta x)^2
(\Delta k)^2
&\ge 
\frac14
\|f\|_2^4.
\notag
\intertext{Verwendet man erneut, dass die Fourier-Transformation die $L^2$-Norm
nicht ändert, kann man die Normen wegkürzen und erhält schliesslich die
Ungleichung}
(\Delta x)^2
(\Delta k)^2
&\ge 
\frac14.
\notag
\end{align}
Damit ist die Ungleichung bewiesen, es bleibt noch zu zeigen, dass Gleichheit
nur für Gauss-Funktionen auftreten kann.
Gleichheit tritt auf, wenn in den Ungleichungen
\eqref{buch:gruppen:fourier:eqn:cauchyschwarz}
und
\eqref{buch:gruppen:fourier:eqn:realteil}
Gleichheit vorliegt.
Die zweite ist einfach: hier wird zum Realteil übergegangen, wenn sich
dabei nichts ändert, liegt Gleichheit vor, also genau für reelle Funktionen.
Die erste ist die Cauchy-Schwarz-Ungleichung, in der Gleichheit genau
dann vorliegt, wenn die Faktoren linear abhängig sind, wenn also
die Gleichung
\[
f'(x)
=
c
xf(x)
\]
gilt.
Die Differentialgleichung  lässt sich separieren:
\[
\frac{f'(x)}{f(x)} = \frac{d}{dx}\log f(x) = cx
\quad\Rightarrow\quad
\log f(x) = \frac12cx^2 + d
\quad\Rightarrow\quad
f(x) = Ce^{-ax^2},
\]
mit $-a = \frac12c$.
Die Konstante $a$ muss positiv sein, weil sonst $f(x)$ nicht
integrierbar ist.
Gleichheit tritt also genau auf für Gauss-Funktionen.
\end{proof}

%
% Unschärferelation inder Quantemechanik
%
\subsubsection{Unschärferelation in der Quantenmechanik}







