%
% 3-faltung.tex
%
% (c) 2022 Prof Dr Andreas Müller, OST Ostschweizer Fachhochschule
%
\section{Faltung
\label{buch:gruppen:section:faltung}}
\kopfrechts{Faltung}
Für einen Definitionsbereich $X$, der nur eine Menge ist, können
von Funktionen $f,g\in C(G)$ nur die Werte im gleichen Punkt
miteinander verglichen werden. 
Daher sind die einzigen algebraischen Operationen, die wir auf
$C(G)$ definieren können, die punktweise Addition $f+g$ und 
die punktweise Multiplikation $f\cdot g$ mit
\begin{align*}
(f+g)(x)      &= f(x)+g(x) \\
(f\cdot g)(x) &= f(x) g(x)
\end{align*}
Die Gruppenstruktur ermöglicht, verschiedene Punkte im Definitionsbereich
mit Hilfe einer Translation $T_x$ zur Deckung zu bringen und somit
Funktionswerte auf verschiedenen Gruppenelemente miteinander verrechnen.

%
% Hall
%
\subsection{Hall
\label{buch:gruppen:faltung:subsection:hall}}

\begin{definition}
Die Faltung zweier Funktion $f,g\in C(G)$ ist die Funktion
\begin{equation}
(f*g)(x)
=
\int_G f(y)g(y^{-1}x)\,dy.
\label{buch:gruppen:faltung:eqn:deffaltung}
\end{equation}
Im Fall einer additiv geschriebenen, abelschen Gruppe wird die Faltung zu
\begin{equation}
(f*g)(x)
=
\int_G f(y)g(x-y)\,dy
\label{buch:gruppen:faltung:eqn:deffaltungadditiv}
\end{equation}
\end{definition}

%
% Faltung als Produkt
%
\subsection{Faltung als Produkt
\label{buch:gruppen:faltung:subsection:produkt}}
Die Faltung ist ein Produkt, d.~h.~man kann mit Faltungsprodukten 
genau so rechnen, wie man es sich von anderen (nicht kommutativen)
Produkten wie zum Beispiel dem Matrizenprodukt gewöhnt ist.
Dazu müssen zwei Bedingungen erfüllt sein: es müssen das
Assoziativ- und das Distributivgesetzt gelten.

%
% Assoziativität
%
\subsubsection{Assoziativität}
Das Assoziativgesetz besagt, dass es in einem Produkt mit mehr als
zwei Faktoren nicht auf die Reihenfalge ankommt, in der man die
Produkte ausführt.
Dies ermöglicht, Faltungsprodukte von drei Funktionen $f$, $g$ und $h$
auch einfach als $f*g*h$ zu schreiben.

\begin{satz}
Die Faltung ist assoziativ, also $(f*g)*h=f*(g*h)$ für Funktionen
$f,g,h\in C(G)$, für die alle Faltungen definiert sind.
\end{satz}

\begin{proof}[Beweis]
Ausgehend vond er Definition der Faltung in
\eqref{buch:gruppen:faltung:eqn:deffaltung}
kann man die Faltung der zweifachen Faltung als
\begin{align*}
((f*g)*h)(x)
&=
\int_G (f*g)(y) h(y^{-1}x)\,dy
=
\int_G \int_G f(z)g(z^{-1}y) \,dz\, h(y^{-1}x)\,dy
\intertext{berechnen.
Unter der postulierten Voraussetzung, dass alle Integrale existieren,
gilt der Satz von Fubini, der die Reihenfolge der Integrationen zu
vertauschen gestattet.
So wird die Faltung zu
}
&=
\int_G \int_G f(z)g(z^{-1}y) h(y^{-1}x) \,dy \,dz
\\
&=
\int_G f(z) \int_G g(z^{-1}y) h(y^{-1}x) \,dy \,dz
\intertext{Das Integral über $y$ ist invariant unter der Translation
mit $z$, wir schreiben daher $s=z^{-1}y$ und integrieren über $s$
statt $y=zs$:}
&=
\int_G f(z) \int_G g(s) h(s^{-1}z^{-1}x) \,ds \,dz
=
\int_G f(z) (g*h)(z^{-1}x) \,dz
=
(f*(g*h))(x)
\end{align*}
Damit ist die Assoziativität gezeigt.
\end{proof}

%
% Distributivität
%
\subsubsection{Distributivität}
Für ein Produkt wird zusätzlich erwartet, dass man damit rechnen kann
wie mit jedem anderen Produkt.
Dies bedeutet, dass für die Faltung und die Addition von Funktionen
das Distributivgesetz gilt, dass man also Faltungsprodukte ausmultiplizieren
und faktorisieren kann wie gewöhnliche punktweise Produkte von Funktionen.

\begin{satz}
Für die Faltung und die Addition von Funktionen gilt das Distributivgesetz
\begin{equation}
\begin{aligned}
f*(g+h) &= f*g + f*h
&&\text{und}&
(f+g)*h &= f*h + g*h
\\
(\lambda f) * g &= \lambda (f*g) 
&&&
f*(\lambda f) &= \lambda (f*g).
\end{aligned}
\label{buch:gruppen:faltung:eqn:distributiv}
\end{equation}
\end{satz}

\begin{proof}[Beweis]
Das Distributivgesetz folgt sofort aus der Distributivität der Multiplikation
und der Linearität des Integrals:
\begin{align*}
(f*(g+h))(x)
&=
\int_G f(y) (g+h)(y^{-1}x)\,dy
\\
&=
\int_G f(y) g(y^{-1}x)\,dy
+
\int_G f(y) h(y^{-1}x)\,dy
=
(f*g)(x)
+
(f*h)(x)
\end{align*}
Die anderen Gleichungen \eqref{buch:gruppen:faltung:eqn:distributiv}
folgen auf die gleiche Art.
\end{proof}

%
% Kommutivität
%
\subsubsection{Kommutativität}
Die Faltung von Funktionen auf $\mathbb{R}$ mit der Addition ist
\begin{align*}
(f*g)(x)
&=
\int_{\mathbb{R}} f(y)g(x-y)\,dy
\intertext{mit der Substitution $s=x-y$ wird daraus}
=
\int_{-\infty}^\infty f(x-s) g(s)\,ds
=
(g*f)(x).
\end{align*}
Das Faltungsprodukt von Funktionen auf $\mathbb{R}$ ist also kommutativ.
Dies gilt jedoch im allgemeinen nicht.

\begin{beispiel}
Die symmetrische Gruppe ist die kleinste nichtabelsche Gruppe.
Es gibt zwei Permutationen $\sigma$ und $\tau$, für die
$\sigma\tau\ne \tau\sigma$ gilt.
Wir berechnen die Faltung der beiden Funktionen 
\[
\delta_\sigma(x) 
=
\begin{cases}
1&\qquad \text{falls $x=\sigma$}\\
0&\qquad \text{sonst}
\end{cases}
\qquad\text{und}\qquad
\delta_\tau(x) 
=
\begin{cases}
1&\qquad \text{falls $x=\tau$}\\
0&\qquad \text{sonst}
\end{cases}
\]
\end{beispiel}


