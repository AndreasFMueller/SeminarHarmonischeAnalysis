%
% 68-orthochar.tex -- Orthogonalität der Charktere
%
% (c) 2022 Prof Dr Andreas Müller, OST Ostschweizer Fachhochschule
%

%
% Orthogonalität der Charaktere
%
\subsection{Orthogonalität der Charaktere}
Am Ende des vorangegangenen Abschnitts haben wir angedeutet, dass
die algebraischen Relationen zwischen den Matrixelementen einer
Darstellung fast wie Orthogonalitätsrelationen aussehen, dass die
Summe aber nicht das Skalarprodukt von Funktionen auf der Gruppe ist.
Charaktere von irreduziblen Darstellungen sind aber speziell und damit
ist es möglich, für Charaktere die folgenden Orthogonalitätsrelationen
zu beweisen.
Wir schreiben
\[
\langle \chi_1,\chi_2\rangle
=
\frac{1}{|G|} \sum_{g\in G} \overline{\chi_1(g)}\chi_2(g)
\]
für zwei Charaktere $\chi_1$ und $\chi_2$.

\begin{satz}
Ist $\chi$ der Charakter einer $n$-dimensionalen irreduziblen Darstellung,
dann ist $\langle \chi,\chi\rangle = 1$.
Sind $\chi_1$ und $\chi_2$ Charaktere von nichtisomorphen
irreduziblen $n_1$- bzw.~$n_2$-dimensionalen Darstellungen, dann 
ist $\langle \chi_1,\chi_2\rangle = 0$.
\end{satz}

\begin{proof}[Beweis]
Für den Charakter wissen wir aus
Satz~\ref{buch:gruppen:darstellung:satz:charg-1}, dass
$\chi(g^{-1}) = \overline{\chi(g)}$.
Dies bedeutet, dass wir in der Formeln
\eqref{buch:gruppen:darstellungen:eqn:matrixnichtiso}
und
\eqref{buch:gruppen:darstellungen:eqn:matrixiso}
die Inverse durch die komplexe Konjugation ersetzen können,
sofern wir die Formenl nur brauchen, um die Spur zu berechnen.

Das Skalarprodukt ist
\begin{align*}
\langle \chi_1,\chi_2\rangle
&=
\frac{1}{|G|}
\sum_{g\in G}
\overline{\chi_1(g)}
\chi_2(g)
=
\frac{1}{|G|}
\sum_{g\in G}
\chi_1(g^{-1})
\chi_2(g)
\\
&=
\frac{1}{|G|}
\sum_{g\in G}
\tr \varrho_1(g^{-1})
\cdot
\tr \varrho_2(g)
=
\frac{1}{|G|}
\sum_{g\in G}
\biggl(
\sum_{i}
\bigl(\varrho_1(g^{-1})\bigr)_{ii}
\biggl)
\cdot
\biggl(
\sum_{k}
\bigl(\varrho_2(g)\bigr)_{kk}
\biggl)
\\
&=
\sum_{i,k}
\frac{1}{|G|}
\sum_{g\in G}
\bigl(\varrho_1(g^{-1})\bigr)_{ii}
\bigl(\varrho_2(g)\bigr)_{kk}.
\intertext{Für nicht isomorphe Darstellungen verschwinden alle Summen
über die Gruppe und somit ist in diesem Fall $\langle \chi_1,\chi_2\rangle=0$.
Falls $\varrho_1=\varrho_2$ ist, also im Fall $\chi_1=\chi_2$ kann 
Formel~\ref{buch:gruppen:darstellungen:eqn:matrixiso}
mit $j=i$ und $l=k$ verwendet werden:}
&=
\sum_{i,k}
\frac1n\delta_{ik}\delta_{ik}
=
1.
\end{align*}
Damit ist die Orthonormierung der Charaktere bewiesen.
\end{proof}

