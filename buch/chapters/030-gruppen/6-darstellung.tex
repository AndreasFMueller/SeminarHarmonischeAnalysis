%
% 5-darstellung.tex -- Darstellungen von Gruppen
%
% (c) 2022 Prof Dr Andreas Müller, OST Ostschweizer Fachhochschule
%
\section{Darstellungen
\label{buch:gruppen:section:darstellung}}
\kopfrechts{Darstellungen}
Für die Gelfand-Transformation wurden beschränkte Homomorphismen
$G\to\mathbb{C}^*$ benötigt.
Da $\mathbb{C}^*$ eine abelsch Gruppe ist, wird ein Homomorphismus
auf $xy$ und $xy$ den gleichen Wert
\[
h(xy) = h(x)h(y) = h(y)h(x) = h(yx)
\]
annehmen.
Wenn die Gruppe $G$ nicht abelsch und $xy\ne yx$ ist, wird der
Unterschied in den Werten von $h$ nicht mehr sichtbar.
Auch die Gelfand-Transformation kann daher nur einen ``kommutative''
Sicht auf die Gruppe $G$ vermitteln.

Diese Schwierigkeit kann überwunden werden, indem als Funktionswerte
von $h$ nicht nur Zahlen zugelassen werden, sondern beliebige
Matrizen.
Man spricht von einer {\em Darstellung} der Gruppe $G$ durch
Matrizen.
Da die Matrizenmultiplikation im Allgemeinen nicht kommutativ
ist, besteht die Möglichkeit, mindestens einen Teil der
Nichtkommutativität der Gruppe $G$ in den Matrizen $h(x)$
wiederzufinden.

%
% Definition
%
\subsection{Definition}
Viele der klassischen Gruppen lernt man in der linearen Algebra
kennen und sind bereits Mengen von Matrizen definiert.
Die Verknüpfung der Gruppenelemente ist die Matrixmultiplikation,
das neutrale Element ist die Einheitsmatrix $I$.
Da sie auch invertierbar sein müssen, liegen sie in der 
allgemeinen linearen Gruppe.
Eine solche Gruppe ist daher eine Teilmenge von $\operatorname{GL}_n(\Bbbk)$
für einen geeignten Körper $\Bbbk$.

Die Permutationsgruppe $S_n$ von $n$ Elementen operiert durch 
Vertauschungen.
Zu jeder Permutation $\sigma\in S_n$ lässt sich die Matrix $P_\sigma$
mit den Matrixelementen
\[
(P_\sigma)_{ik}
=
\begin{cases}
1&\qquad\text{für $i=\sigma(k)$}\\
0&\qquad\text{sonst}.
\end{cases}
\]
konstruieren.
Man kann sofort nachrechnen, dass $P_{\sigma\pi} = P_\sigma P_\pi$,
dass $P_{e}=I$ und $P_{\sigma^{-1}}=P_\sigma^{-1}$.
Die Menge der Matrizen
$P_n=\{P_\sigma\mid\sigma\in S_n\}\subset\operatorname{GL}_n{\mathbb{R}}$
ist eine Gruppe von Matrizen.

Man kann die Relation, dass $S_n$ in Form der Permutationsmatrizen $P_n$
in $\operatorname{GL}_n(\mathbb{R})$ enthalten ist, auch als Homomorphismus
\[
\varrho
\colon 
S_n
\to
\operatorname{GL}(\mathbb{R})
:
\sigma
\mapsto
P_\sigma
\]
verstehen.
Noch etwas allgemeiner definieren wir eine Darstellung einer Gruppe
wie folgt.

\begin{definition}
Sei $G$ eine Gruppe und $V$ ein $\Bbbk$-Vektorraum.
Ein Homomorphismus $\varrho\colon G\to\operatorname{GL}(V)$
heisst {\em Darstellung} der Gruppe $G$ im Vektorraum $V$.
Sie heisst {\em $n$-dimensional}, wenn $V$ ein $n$-dimensionaler
Vektorraum ist.
Hat $V$ eine Basis aus $n$ Vektoren, können die linearen
Abbildungen $\varrho(x)$ als Matrizen in $M_{n\times n}(\Bbbk)$
geschrieben werden.
\end{definition}

\subsubsection{Die reguläre Darstellung}
Sei $G$ eine endliche Gruppe und $h\in G$.
Die Verknüpfung mit $g$ definiert eine Permutation der Gruppenelemente
durch
\[
\pi_h \colon G \to G : g \mapsto hg.
\]
Zum Gruppenelement $h$ gehört daher die Permutationsmatrix
$P_{\pi_h}\in\operatorname{GL}_n(\mathbb{R})$.
Zu jeder endlichen Gruppe gehört daher die Darstellung durch die
Matrizen $P_{\pi_h}$.
Sie heisst die {\em reguläre Darstellung}.

Indem man die Elemente der Gruppe als $e=g_0,g_1,g_2,\dots,g_{|G|-1}$
nummeriert, kann man auch Vektoren definieren, auf denen die
Gruppe durch die Matrizen $P_{\pi_h}$ operiert.
Sei $v_{g_i}$ der Spaltenvektor, der in der $i$-ten Zeile eine Eins
hat und lauter Nullen in allen anderen Zeilen.
Dann ist $\pi_{h}v_{g_i} P_{\pi_h}v_{g_i}$, die Vektoren
$v_{g_0},\dots,g_{|G|-1}$ sind die Standardbasis eines $|G|$-dimensionalen
Vektorraumes, auf dem die Gruppe $G$ durch Permutationsmatrizen wirkt.

Die Vektoren $v_h$ kann man auch als die Funktionen
\[
v_h
\colon
G\to\mathbb{R}
:
g\mapsto
\begin{cases}
1&\qquad\text{falls $g=h$}\\
0&\qquad\text{sonst}
\end{cases}
\]
betrachten.
Die Menge der Funktionen $G\to\Bbbk$ bezeichnen wir mit $\Bbbk[G]$.
Die Gruppe $G$ operiert auf den Funktion durch die Translation.
Funktion $v_{kh}$ ist genau dann $1$, wenn $g=kh$ ist, also wenn $k^{-1}g=h$
ist.
Das selbe gilt die Funktion $g\mapsto v_h(k^{-1}h)$, daher definieren wir
die Translation für beliebige Funktionen $f\colon G\mapsto \mathbb{R}$
durch
\[
T_kf\colon G\to\mathbb{R}:g\mapsto f(k^{-1}g).
\]
Diese Definition deckt sich natürlich mit der
Definition~\ref{buch:gruppen:gruppe:def:translation}.
Die Translation ist also eine Darstellung der Gruppe $G$ auf dem
Vektorraum der Funktionen auf der Gruppe.

Für eine topologische Gruppe $G$ ist die Translation $T_g$ eine 
stetige lineare Abbildung auf dem Vektorraum der stetigen Funktionen
$C(G,\mathbb{R})$, ausserdem ist die Abbildung $g\mapsto T_g$ eine
stetige Abbildung von $G$ in den Vektorraum der linearen Operatoren
auf $C(G,\mathbb{R})$.
Für eine Lie-Gruppe ist $g\mapsto T_g$ sogar eine differenzierbare
Abbildung.
Auch diese Darstellungen von $G$ auf entsprechenden Funktionenräumen
heissen die {\em reguläre Darstellung} von $G$.

%
% Die Standarddarstellung einer Matrizengruppe
%
\subsubsection{Die Standarddarstellung einer Matrizengruppe}
Die reguläre Darstellung ist also eine Art ``universelle Darstellung'',
die man immer konstruieren kann, sie ist allerdings auch etwas kompliziert,
da die Dimension des Vektorraums $V$, auf dem die Gruppenelemente in der
Darstellung wirken, so gross ist wie die Kardinalität der Gruppe.
Von besonderem Nutzen sind Darstellungen, die eine kleine Dimension haben.
Die Elemente einer Gruppe $G\subset\operatorname{GL}_n(\mathbb{R})$
von $n\times n$-Matrizen operiert auf offensichtliche Weise auf einem
$n$-dimensionalen Vektorraum $\mathbb{R}^n$.
Die Untergruppen von $\operatorname{GL}_n(\mathbb{R})$, betrachtet
als abstrakte Gruppen, haben also immer eine $n$-dimensionale Darstellung,
die durch die Einbettung $G\to \operatorname{GL}_n(\mathbb{R})$ entsteht.
Sie heisst die {\em Standarddarstellung}.

%
% Die triviale Darstellung
%
\subsubsection{Die triviale Darstellung}
Für eine beliebige Gruppe $G$ ist die konstante Abbildung
\[
\varrho
\colon
G\to \operatorname{GL}_n(\mathbb{R})
:
g\mapsto I
\]
ein Homomorphismus und damit eine $n$-dimensionale Darstellung von $G$.
Sie heisst die {\em triviale $n$-dimensionale Darstellung}.

%
% Reguläre Darstellung und Faltung
%
\subsubsection{Reguläre Darstellung und Faltung}
Eine Gruppe $G$ operiert auf den Funktionen auf $G$ durch die 
lineare Operation der Translation $T_g$ für $g\in G$.
Sei $G$ eine endliche Gruppe und $f\colon G\mapsto \mathbb{R}$ eine
Funktion auf $G$.
Da die Translation $T_g$ eine lineare Abbildung ist, können wir die 
sie mit den Koeffizienten $T_g$ linear kombinieren.
Damit lässt sich die eine Operation der Funktion $f$ auf den Funktionen
$h\colon G\to\mathbb{R}$ durch
\begin{equation}
T_fh
=
\sum_{g\in G} f(g)\cdot T_gh
\colon
G\to\mathbb{R}
:
x\mapsto
\biggl( \sum_{g\in G} f(g)\cdot T_gh\biggr) (x)
=
\sum_{g\in G} f(g)h(g^{-1}x)
=
(f*h)(x)
\end{equation}
definieren.
Die Faltung ist also nichts anderes als die Ausdehnung der regulären
Darstellung von einzelnen Gruppenelementen auf den Vektorraum
der Funktionen auf $G$.

Für eine topologische Gruppe $G$ und eine integrierbare Funktion
$f\colon G\to\mathbb{R}$
lässt sich die Konstruktion ebenfalls durchführen.
Statt einer Summe über die Gruppenelemente muss man allerdings das
Haar-Integral mit den Funktionswerten $f(g)$ als Gewichten verwenden.
Die Funktion $f$ operiert somit auf einer Funktion $h\colon G\to\mathbb{R}$
durch 
\[
T_fh\colon
x\mapsto
\biggl(\int_G f(g) T_gh\,dg\biggr)(x)
=
\int_G f(g) h(g^{-1}x)\,dg
=
(f*g)(x).
\]
Dies zeigt, dass das Studium der reguläre Darstellung nichts anderes
ist als die Untersuchung der Gruppenoperation auf der Gruppe mit Hilfe
der Algebrastruktur, die durch die Faltung erzeugt wird.

%
% Vergleich von Darstellungen
%
\subsection{Vergleich von Darstellungen}
Die Einfachheit der regulären Darstellung hängt davon ab, dass man die
Basis sehr speziell wählen kann.
Eine beliebige Darstellung auf den ersten Blick sehr viel weniger gut
durchschaubar, weil eine beliebig gewählte Basis zu Matrizen führt,
die die Natur Gruppe verschleiern können.
Umgekehrt kann auch die reguläre Darstellung verbergen, dass dahinter
eigentlich einfachere Darstellungen stecken, die man besser erkennen
könnte, wenn man eine zweckmässigere Basis verwenden würde.
Das folgende Beispiel illustriert dies.

\begin{beispiel}
Wir betrachten die reguläre Darstellung der zyklischen Gruppe mit
drei Elementen $C_3=\{0,1,2\}$.
Die Gruppenelemente $1,2\in C_1$ werden in der regulären Darstellung
durch die Matrizen
\[
1\mapsto
P_1=
\begin{pmatrix}
0&0&1\\
1&0&0\\
0&1&0
\end{pmatrix}
\qquad\text{und}\qquad
2\mapsto
P_2
=
\begin{pmatrix}
0&1&0\\
0&0&1\\
1&0&0
\end{pmatrix}
=
P_1^2
\]
dargestellt.

Wir verwenden jetzt im Vektorraum $\mathbb{R}^3$, auf dem diese Matrizen
wirken, eine alternative Basis wie folgt:
\[
b_1
=
\frac{1}{\!\sqrt{3}}
\begin{pmatrix}1\\1\\1\end{pmatrix}
,
\qquad
b_2
=
\frac{1}{\!\sqrt{2}}
\begin{pmatrix*}[r]1 \\ -1 \\ 0\end{pmatrix*}
\qquad\text{und}\qquad
b_3
=
\frac{1}{\!\sqrt{6}}
\begin{pmatrix*}[r] 1\\1\\-2\end{pmatrix*}.
\]
Die Basis ist orthonormiert.
Die Wirkung der Matrix $P_1$ auf diesen Basisvektoren ist
\begin{align*}
P_1b_1
&=
b_1\\
P_1b_2
&=
\frac{1}{\!\sqrt{2}}
\begin{pmatrix*}[r]
0\\1\\-1
\end{pmatrix*}
=
(P_1b_2\cdot b_2) b_2
+
(P_1b_2\cdot b_3) b_3
=
-\frac{1}{2}
b_2
+
\frac{\sqrt{3}}{2}
b_3
\\
P_1b_3
&=
\frac{1}{\!\sqrt{6}}
\begin{pmatrix*}[r]
-2\\ 1\\1
\end{pmatrix*}
=
(P_1b_3\cdot b_2) b_2
+
(P_1b_3\cdot b_3) b_3
=
-\frac{\!\sqrt{3}}{2}
b_2
-
\frac{1}{2}
b_3
\end{align*}
Die Koeffizienten in der Darstellung der Bildvektoren in der
orthonormierten Basis $\{b_1,b_2,b_3\}$ auf der rechten Seite
können mit Hilfe des Skalarproduktes ermittel werden.
In der Basis $b_i$ bekommt die Matrix $P_1$ also die Matrix
\begin{equation}
\tilde{P}_1
=
\begin{pmatrix}
1&0&0\\
0&-\frac12 & \frac{\!\sqrt{3}}2\\
0&-\frac{\!\sqrt{3}}2&-\frac12
\end{pmatrix}
=
\begin{pmatrix}
1&0&0\\
0&\cos(-120^\circ)& -\sin(-120^\circ) \\
0&\sin(-120^\circ)& \cos(-120^\circ)
\end{pmatrix}.
\label{buch:gruppen:darstellung:bsp:c3:blockform}
\end{equation}
In der neuen Basis zerfällt die Matrix in zwei Blöcke.
Der linke obere Block ist die eindimensionale triviale Darstellung.
der $2\times 2$-Block in der rechten unteren Ecke ist eine Drehmatrix mit
dem Drehwinkel $120^\circ$.

Geometrisch bedeutet die Blockform
\eqref{buch:gruppen:darstellung:bsp:c3:blockform}, dass man den
dreidimensionalen Raum in eine Gerade mit Richtung $b_1$ und
eine dazu senkrechte Ebene mit Basis $\{b_2,b_3\}$ aufteilen
kann.
Auf der geraden operiert die Gruppe $C_3$ trivial, der zweidimensionale
Unterraum ist eine zweidimensionale Darstellung der Gruppe $C_3$.
Es ist also gelungen, die reguläre Darstellung von $C_3$ zu zerlegen
in zwei einfachere Darstellungen.
\end{beispiel}


%
% Irreduzible Darstellungen und das Lemma von Schur
%
\subsection{Irreduzible Darstellungen und das Lemma von Schur}

%
% Mittelung und mittelbare Gruppen
%
\subsection{Mittelung und mittelbare Gruppen}

%
% Orthogonale und unitäre Darstellungen
%
\subsection{Orthogonale und unitäre Darstellungen}

%
% Orthogonalität der Matrixelemente
%
\subsection{Orthogonalität der Matrixelemente}

%
% Charakter einer Darstellung
%
\subsection{Charakter einer Darstellung}

%
% Zerlegung einer Darstellung in irreduzible Darstellungen
%
\subsubsection{Zerlegung einer Darstellung in irreduzible Darstellungen}
