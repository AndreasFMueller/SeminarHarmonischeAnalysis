%
% 6a-lie.tex -- Darstellungen von kompakten Lie-Gruppen
%
% (c) 2022 Prof Dr Andreas Müller, OST Ostschweizer Fachhochschule
%

%
% Darstellungen kompakter Lie-Gruppen
%
\subsection{Darstellungen kompakter Lie-Gruppen}
Die Resultate des letzten Abschnitts wurden jeweils nur für endliche Gruppen
beweisen.
In diesem Abschnitte sollen ein paar Hinweise zusammengetragen werden,
die illustrieren, wie die Resultate auch für die wichtige Klasse
der kompakten Lie-Gruppen, zu denen zum Beispiel die Drehgruppen
$\operatorname{SO}(n)$ und die unitären Gruppen $\operatorname{U}(n)$
gehören, erweitert werden können.
Dafür ist entscheidend, dass kompakte Lie-Gruppen ein rechts- und
linksinvariantes Haar-Mass haben.
Da die Gruppe kompakt ist, darf man sogar annehmen, dass das
Volumen
\[
\operatorname{vol}(G)
=
\int_G 1\, dg
\]
der Gruppe den Wert $1$ hat.
Für eine beliebige Funktion $f\colon G\to \mathbb{R}$ ist
\[
Mf
=
\int_G f(g)\,dg
\]
der Mittelwert der Funktion, die die für endliche Gruppen verwendete
normierte Summe $1/|G| \sum_{g\in G}f(g)$ als Mittelung ersetzen kann.

%
% Unitäre Darstellungen
%
\subsubsection{Unitäre Darstellungen}
Wie für eine endliche Gruppe kann man auch für eine endlichdimensionale
komplexe Darstellung $\varrho$ einer kompakten Lie-Gruppe auf einem
Vektorraum $V$ aus einem beliebigen Skalarprodukt $(\;,\;)$ auf $V$
ein invariantes Skalarprodukt mit der Mittelung
\begin{equation}
\langle u,v\rangle
=
\int_G (\varrho(g)u,\varrho(g)v)\,dg
\label{buch:gruppen:darstellungen:eqn:invskalarprodukt}
\end{equation}
konstruieren.
Die Konvergenz des Integrals ist für eine kompakte Lie-Gruppe immer 
gegeben.


%
% Eigenwerte
%
\subsubsection{Eigenwerte}
In Satz~\ref{buch:gruppen:darstellung:satz:charg-1} wurde gezeigt,
dass für den Charakter einer Darstellung einer endlichen Gruppe
$\chi(g^{-1})=\overline{\chi(g)}$ gilt.
Der Beweis hat gebraucht, dass die Eigenwerte komplexe Zahlen vom
Betrag $1$ sein mussten.

Das invariante Skalarprodukt
\eqref{buch:gruppen:darstellungen:eqn:invskalarprodukt}
zeigt, dass eine Darstellung einer kompakten Lie-Gruppe in einer
geeigneten Basis immer durch unitäre Matrizen möglich ist.
Die Wahl einer Basis hat aber keinen Einfluss auf die Eigenwerte.
Da unitäre Matrizen als Eigenwerte nur komplexe Zahlen vom Betrag $1$
haben können, folgt wie für endliche Gruppen, dass
$\chi(g^{-1})=\overline{\chi(g)}$ ist.

%
% Projektion
%
\subsubsection{Projektion}
In Satz~\ref{buch:gruppen:darstellungen:satz:projektion} wurde gezeigt,
dass zu einem invarianten Unterraum $W\subset V$ einer endlichdimensionalen
Darstellung $\varrho\colon G\to \operatorname{GL}(V)$ immer ein komplementärer
invarianter Unterraum $W'$ gefunden werden kann.
Der Beweis basierte auf der Idee, dass die Projektion $P\colon V\to W$,
die nicht unbedingt mit der Darstellung vertauschen muss, durch Mittelung
zu einer Projektion gemacht werden kann, die mit der Darstellung vertauscht.
Dies ist auch auf einer kompakten Lie-Gruppe möglich, man verwendet
\[
P'
=
\int_G \varrho(g)P\varrho(g)^{-1}\,dg,
\]
die Konvergenz des Integrals ist wieder durch die Kompaktheit der Gruppe
garantiert.
Damit folgt wieder, dass sich zu einem invarianten Unterraum $W\subset V$
eine direkte Zerlegung $V=W\oplus W'$ von Darstellungen finden lässt.

%
% Abbildungen zwischen Darstellungen
%
\subsubsection{Abbildungen zwischen Darstellungen}
In Satz~\ref{buch:gruppen:darstellungen:satz:abbmittel} wurde aus einer
linearen Abbildung $f\colon V_1\to V_2$ zwischen den Vekträumen zweier
Darstellungen $\varrho_i\colon G\to\operatorname{GL}(V_i)$ eine
gemittelte Abbildung 
\[
f
=
\int_G \varrho_w(g) \circ f \circ \varrho_1(g)\,dg
\]
konstruiert, die mit den Darstellungen vertauscht.
Die Schlussfolgerungen mit dem Lemma von Schur sind damit genau gleich
anwendbar zeigen, dass die Aussage von
Satz~\ref{buch:gruppen:darstellungen:satz:abbmittel} auch für kompakte
Lie-Gruppen gilt.

%
% Orthogonalitätseigenschaften
%
\subsubsection{Orthogonalitätseigenschaften}
Die Matrixform des Satzes~\ref{buch:gruppen:darstellungen:satz:abbmittel} 
hat auf Formeln der Sätze 
\ref{buch:gruppen:darstellungen:satz:matrixnichtiso}
und
\ref{buch:gruppen:darstellungen:satz:matrixiso}
ergeben, diese gelten jetzt auch für kompakte Lie-Gruppen.
Damit sind alle Voraussetzungen gegeben um zu schliessen, dass auch
die Orthogonalitätseigenschaften für die Charaktere von irreduziblen
Darstellungen und der Matrix-Elemente gelten.

