%
% 2-casimir.tex
%
% (c) 2022 Prof Dr Andreas Müller, OST Ostschweizer Fachhochschule
%
\section{Casimir-Operator
\label{buch:operatoren:section:casimir}}
\kopfrechts{Casimir-Operator}
Warum spielt der Laplace-Operator so eine wichtige Rolle in den
Anwendungen?
Und gibt es für andere Definitionsgebiete einen Operator, der eine
ähnliche Rolle spielt?

%
% Invariante Differentialoperatoren
%
\subsection{Invariante Differentialoperatoren}
In diesem Abschnitt betrachten wir Differentialoperatoren auf
glatten Funktionen $f\in C^\infty(\mathbb{R}^n)$ auf $\mathbb{R}^n$.
Ein Differentialoperator $P$ der Ordnung $m$ kann geschrieben werden als
eine Summe
\[
Pf
=
\sum_{|\mathbf{l}|\le m}
a_{\bm{l}}(x)
D^{\bm{l}}
f.
\]
Der Operator kann natürlich auch als Pseudodifferentialoperator geschrieben
werden, er bekommt dann die Form
\begin{align*}
(Pf)(x)
&=
\int_{\mathbb{R}^n}
\frac{1}{(2\pi)^n}
\int_{\mathbb{R}^n}
e^{ik\cdot (x-y)}
\sum_{|\bm{l}|\le m}
a_{\bm{l}}(x)(ik)^{\bm{l}}
\,dk
\,dy
\end{align*}
mit dem Symbol
\[
a(x,k)
=
\sum_{|\bm{l}|\le m} a_{\bm{l}}(x) i^{|\bm{l}|} k^{\bm{l}}.
\]
Das Symbol eines Differentialoperators ist also ein Polynom in den
den Variablen $k_1,\dots,k_n$.

Der Raum $\mathbb{R}^n$ ist invariant unter sogenannten Bewegungen, also
beliebigen Verschiebungen und Drehungen.
Wir untersuchen jetzt, ob ein Differentialoperator wie $P$ unter solchen
Operationen ebenfalls invariant sein kann.
Damit ist folgendes gemeint.
Ist $\varphi\colon\mathbb{R}^n\to\mathbb{R}^n$ eine dieser Transformationen,
dann erzeugt die Zusammensetzung mit $\varphi$ eine lineare Abbildung
\[
T_\varphi
\colon
C^{\infty}(\mathbb{R}^n) \to C^\infty(\mathbb{R}^n)
:
f \mapsto f\circ \varphi.
\]
Die Umkehrabbildung ist $T_\varphi^{-1}=T_{\varphi^{-1}}$.
Der Operator $P$ heisst invariant, wenn 
\[
T_\varphi^{-1} P T_\varphi = P
\]
gilt.

%
% Die Transformation $T_\varphi$ und das Symbol
%
\subsubsection{Die Transformation $T_\varphi$ und das Symbol}
Bei einer Translation des Raumes $\mathbb{R}^n$ ändert die Differenz
$x-y$ in der Exponentialfunktion der Definition eines
Pseudodifferentialoperators nicht, es kommt also nur auf das Verhalten
des Symbols $a(x,k)$ in Abhängigkeit von $x$ an.
Der Differentialoperator $P$ ist invariant unter Translationen,
wenn das Symbol $a(x,k)$ nicht von $x$ abhängt.

Eine Drehung des Raumes um einen Punkt $x$ kann durch eine Translation
des Punktes $x$ in den Punkt $0$ als Drehung um den Nullpunkt
mit einer Drehmatrix $R$ geschrieben werden.
Das Symbol muss dazu durch $\tilde{a}(y,k) = a(y+x,k)$ ersetzt werden.
Es genügt also, Drehungen mit einer orthogonalen Matrix $R$ um den
Nullpunkt zu betrachten.

Wir suchen also Kriterien dafür, dass der Differentialoperator $P$ unter
einer Drehung um den Nullpunkt invariant bleibt.
Da $P$ ein Differentialoperator ist, ist sein Träger im Nullpunkt
konzentriert, die Werte $a(x,k)$ des Symbols für $x\ne 0$ spielen also
keine Rolle, nur die Abhängigkeit $k\mapsto a(0,k)$ ist relevant.

Die Fourier-Transformierte einer Funktion $f\in C^{\infty}(\mathbb{R}^n)$
ändert sich unter der Drehung um den Nullpunkt gemäss
\[
\widehat(T_Rf)(k)
=
\frac{1}{(\!\sqrt{2\pi})^n}
\int_{\mathbb{R}^n}
e^{-ik\cdot Rx}
f(Rx)\,dx
=
\frac{1}{(\!\sqrt{2\pi})^n}
\int_{\mathbb{R}^n}
e^{-i(R^tk)\cdot x}
f(Rx)\,dx
=
\hat{f}(R^tk)
=
(T_{R^t}f)(k).
\]
Die Drehung mit der Matrix im Ortsraum läuft also auf eine entgegengesetzte
Drehung im Frequenzraum hinaus.
Für den Pseudodifferentialperator $P$ bedeutet dies, dass $P$ genau dann
invariant ist, wenn das Symbol $a(x,k)$ sich nicht ändert, wenn man $k$ durch
$R^tk$ ersetzt.
Die rotationsinvarianten Differentialoperatoren sind also genau jene, 
für die $k\mapsto a(x,k)$ rotationsinvariant ist.

%
% Differentialoperatoren erster Ordnung
%
\subsubsection{Differentialoperatoren erster Ordnung}
Die Abbildung $k\mapsto a(x,k)$ ist ein Polynom in den Variablen
$k_1,\dots,k_n$.
Bei einem Differentialoperator erster Ordnung ist das Symbol daher
eine lineare Funktion
\begin{equation}
a(x,k)
=
a_0(x,k)
+
a_1(x)k_1
+
\dots
+
a_n(x)k_n.
\label{buch:operatoren:casimir:eqn:oplin}
\end{equation}
Die Permutationen der Variablen $k_1,\dots,k_n$ sind orthogonale Abbildungen.
Im Operator \eqref{buch:operatoren:casimir:eqn:oplin} ändert dabei nur
die Reihenfolge der Koeffizienten der linearen Terme, er kann also nur dann
invariant sein, wenn alle $a_1(x),\dots,a_n(x)$ gleich sind.
Die Spiegelung $x_i\mapsto -x_i$ ist aber auch eine orthogonale Abbildung,
der Operator \eqref{buch:operatoren:casimir:eqn:oplin} kann nur invariant
sein, wenn sich bei einem Vorzeichenwechsel von $k_i$ nichts ändert,
d.~h.~wenn alle Koeffizienten $a_1(x)=\dots = a_n(x)=0$ verschwinden.
Es gibt also keinen invarianten Differentialoperator erster Ordnung.

%
% Invariante Differentialoperatoren höhrer Ordnung
%
\subsubsection{Invariante Differentialoperatoren höherer Ordnung}
Die Überlegungen des vorangegangenen Abschnittes zeigen auch,
dass das Symbol eines invarianten Differentialoperators nur gerade
Potenzen der Variablen $k_1,\dots,k_n$ enthalten kann, und
dass alle Koeffizienten $a_{\bm{l}}(x)$, die durch Permutation der
Indizes im Multiindex $\bm{l}$ auseinander hervorgehen, gleich
sein müssen.

Die genannten Bedingungen sind allerdings noch nicht ausreichend.
Für $n=2$ ist das Polynom $k_1^2k_2^2$ vierten Grades zwar Invariant
unter der Vertauschung der Indizes $1$ und $2$, aber für eine
beliebige Drehung um den Winkel $\alpha$ werden $k_1$ und $k_2$
zu
$k_1' = k_1\cos\alpha -k_2\sin\alpha$
bzw.~
$k_2' = k_1\sin\alpha +k_2\cos\alpha$.
Eingesetzt in das Polynom entsteht
\begin{align*}
k_1^{\prime 2}
k_2^{\prime 2}
&=
(k_1\cos\alpha -k_2\sin\alpha)^2
(k_1\sin\alpha +k_2\cos\alpha)^2
\\
&=
(k_1^2\cos^2\alpha -2k_1k_2\cos\alpha\sin\alpha +k_2^2\sin^2\alpha)
(k_1^2\sin^2\alpha +2k_1k_2\cos\alpha\sin\alpha +k_2^2\cos^2\alpha)
\\
&=
(k_1^4-4k_1^2k_2^2+k_2^4)\cos^2\alpha\sin^2\alpha
+
k_1^2k_2^2(\cos^4\alpha+\sin^4\alpha)
\\
&\qquad
+
2k_1^3k_2\cos^3\alpha\sin\alpha
-
2k_1^3k_2\cos\alpha\sin^3\alpha
-
2k_1k_2^3\cos^3\alpha\sin\alpha
+
2k_1k_2^3\cos\alpha\sin^3\alpha
\\
&=
(k_1^4-2k_1^2k_2^2+k_2^4)\cos^2\alpha\sin^2\alpha
+
k_1^2k_2^2(\cos^4\alpha-2\cos^2\alpha\sin^2\alpha+\sin^4\alpha)
\\
&\qquad
+
2k_1k_2\cos\alpha\sin\alpha(
k_1^2\cos^2\alpha
-
k_1^2\sin^2\alpha
-
k_2^2\cos^2\alpha
+
k_2^2\sin^2\alpha
)
\\
&=
(k_1^2-k_2^2)^2
\cos^2\alpha\sin^2\alpha
+
k_1^2k_2^2
(\cos^2\alpha-\sin^2\alpha)^2
\\
&\qquad
+
2k_1k_2
\cos\alpha\sin\alpha
(k_1^2-k_2^2)\cos2\alpha
\\
&=
(k_1^2-k_2^2)^2
\frac14\sin^22\alpha
+
k_1^2k_2^2
\cos^22\alpha
\\
&\qquad
+
k_1k_2
\cos2\alpha
(k_1^2-k_2^2)
\sin2\alpha
\\
&=
\bigl(k_1k_2\cos2\alpha + {\textstyle\frac12}(k_1^2-k_2^2)\sin2\alpha\bigr)^2.
\end{align*}
Für $\alpha=0$ stimmen die beiden Seiten überein.
Invarianz bedeutet, dass dieser Ausdruck für alle Paare $(k_1,k_2)$
nicht von $\alpha$ abhängt, das ist aber für $k_1\ne0$ und $k_2=0$ nicht,
der Fall, denn in diesem Fall wird die Gleichung zu
\[
0
=
-
\frac12 k_2^2\sin2\alpha,
\]
was nur für $\alpha=\pi s$, $s\in\mathbb{Z}$ erfüllt ist.

%
% Der Laplace-Operator
%
\subsubsection{Der Laplace-Operator}
Nach den Erkenntnissen des vorangegangenen Abschnitts ist der einzige
unter Drehungen invariante Differentialoperator zweiter Ordnung der 
Operator mit dem Symbol
\[
a(x,k) 
=
a(x)(k_1^2+\dots+k_n^2).
\]
Zurückübersetzt in einen gewöhnlichen Differentialoperator entspricht
dies dem Operator
\[
-a(x)
\biggl(
\frac{\partial^2}{\partial x_1^2}
+
\dots
+
\frac{\partial^2}{\partial x_n^2}
\biggr)
,
\]
also einem Vielfachen des Laplace-Operators
\[
\Delta
=
\frac{\partial^2}{\partial x_1^2}
+
\dots
+
\frac{\partial^2}{\partial x_n^2}.
\]
Das verbreitete Auftreten des Laplace-Operators in Anwendungen ist
also ein Ausdruck der Isotropie des Raumes: in einem Punkt ändern sich
die Naturgesetze bei einer Drehung des Raumes um den Punkt nicht.

%
% Invariante Differentialoperatoren höherer Ordnung
%
\subsubsection{Invariante Differentialoperatoren höhrer Ordnung}
Für das Symbol $k_1^2k_2^2$ wurde bereits gezeigt, dass es nicht das Symbol
eines invarianten Differentialoperators vierter Ordnung sein kann.
Ein Differentialoperator vierter Ordnung auf $C^\infty(\mathbb{R}^2)$
muss also aus weiteren Monomen vierten Grades zusammensetzen.
Die Koeffizienten der Terme $k_1^4$ und $k_2^4$ müssen dabei gleich
sein, das Symbol muss also die Form
\[
a(x,k)
=
ak_1^4 + 2bk_1^2k_2^2 + ak_2^4
\]
haben.
Dies lässt sich aber immer schreiben als
\[
a(x,k)
=
a(k_1^2+k_2^2) + (b-a)k_1^2k_2^2.
\]
Der erste Term ist als Quadrat der Länge invariant unter Drehungen,
vom zweiten Term wissen wir bereits, dass er nicht invariant ist.
Somit ist das einzige möglich invariante Symbol vierten Grades
der Falle $b=a$ und damit  $a(x,k) = (k_1^2+k_2^2)^2$.
Der zugehörige Differentialoperator ist dann 
\[
\Delta \Delta
=
\biggl(
\frac{\partial^2}{\partial x_1^2}
+
\frac{\partial^2}{\partial x_2^2}
\biggr)^2.
\]

Für einen Differentialoperator vierter Ordnung auf $\mathbb{R}^n$ kann
man den Unterraum der Funktionen betrachten, die nur von zwei der
Variablen abhängen.
Auf dem Unterraum der nur von $x_k$ und $x_l$ abhängigen Funktionen
besteht der Operator nur aus den Ableitungen
nach e
\[
\biggl(
\frac{\partial^2}{\partial x_k^2}
+
\frac{\partial^2}{\partial x_l^2}
\biggr)^2.
\]
Daraus leitet man ab, dass der einzige unter Drehungen invariante
Differentialoperator vierter Ordnung ein Vielfaches von
\[
\Delta\Delta 
=
\biggl(
\frac{\partial^2}{\partial x_1^2}
+
\dots
+
\frac{\partial^2}{\partial x_n^2}
\biggr)^2
\]
ist.
Dieser Operator heisst auch der {\em biharmonische Operator}, er
kommt zum Beispiel in der Plattengleichung vor.

%
% Differentialoperatoren auf einer Lie-Gruppe
%
\subsection{Invariante Differentialoperatoren auf einer Lie-Gruppe}
Der Laplace-Operator hat sich als der Operator herausgestellt, der
der Isotropie des Raumes in jedem Punkt Rechnung trägt.
Der Operator ist insbesondere um die Drehungen um jeden Punkt
invariant.
Die Invarianzeigenschaft ist natürlich eine Eigenschaft der 
Zur Verfügung stehenden Gruppe, die im vorliegenden Fall die
orthogonale Gruppe $\operatorname{O}(n)$ ist.







