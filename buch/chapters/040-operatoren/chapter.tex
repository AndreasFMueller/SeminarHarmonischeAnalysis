%
% chapter.tex -- Skalarprodukt
%
% (c) 2021 Prof Dr Andreas Müller, Hochschule Rapperswil
%
% !TeX spellcheck = de_CH
\chapter{Operatoren
\label{buch:chapter:operatoren}}
\kopflinks{Operatoren}
Differentialoperatoren haben sich als wichtige Quelle von orthogonalen
Funktionensystem herausgestellt.
Der Laplace-Operator tritt in vielen Anwendungen immer wieder auf
und seine Eigenfunktionen bilden oft die Grundlage von Lösungsansätzen.
Die Kugelfunktionen sind zum Beispiel Eigenfunktionen des Laplace-Operators
auf der Kugeloberfläche und werden auch in der Praxis für die Lösung der
Strömungsdifferentialgleichungen im Rahmen der numerischen Wetterprognose 
genutzt, wie Kapitel~\ref{chapter:spektral} zeigt.
Die Verbindung zwischen Differentialoperatoren und anderen linearen
Operatoren auf Funktionen ist jedoch noch enger und involviert wieder
die harmonische Analysis.

In diesem Kapitel wird gezeigt, wie man die Fourier-Theorie verwenden
kann, um eine grössere Klasse von Operatoren zu definieren, die sowohl
die Differentialoperatoren wie auch gewisse Umkehroperatoren umfasst.
Die besondere Rolle des Laplace-Operators wird verständlich, wenn man
seine Herkunft von der Theorie der Lie-Gruppen in Betracht zieht.
Zu kompakten Lie-Gruppen lässt sich immer ein invarianter
Differentialoperator zweiter Ordnung konstruieren, der sogenannte
Casimir-Operator, der in den konkreten Fällen von $\mathbb{R}^n$
und der Kugeloberfläche zum bekannten Laplace-Operator wird.
Seine Konstruktion wird in Abschnitt \ref{buch:operatoren:section:casimir}
skizziert.

%
% 1-definition.tex
%
% (c) 2023 Prof Dr Andreas Müller, OST Ostschweizer Fachhochschule
%
\section{Pseudodifferentialoperatoren
\label{buch:opertoren:section:pseudodifferentialoperatoren}}
\kopfrechts{Definition}
In früheren Untersuchungen wurden verschiedene Operatoren
auf dem Vektorraum der glatten Funktion $C^\infty(\mathbb{R})$ 
untersucht.
Die Multiplikation mit einer Funktion und die Ableitungen sind 
lineare Operatoren, aber auch die Faltung mit einer anderen Funktion.
Alle diese Operationen werden durch völlig verschiedene Formeln
gegeben, was schwierig macht, eine gemeinsame Theorie zu formulieren.
Pseudodifferentialoperatoren bilden eine Klasse von Operatoren, die
gross genug ist, dass sie alle diese Beispiel umfasst, aber trotzdem
immer noch genügend interessante Eigenschaften hat, dass man damit
rechnen kann.

%
% Motivation: Matrizen
%
\subsection{Motivation: Matrizen
\label{buch:operatoren:pseudodifferential:subsection:motivation}}
Lineare Operatoren auf einem endlichdimensionalen Vektorraum können
durch Wahl einer Basis immer in der Form einer Matrix geschrieben
werden.
Sei $A$ eine lineare Abbildung $U\to V$ zwischen Vektorräumen
$U$ und $V$ mit Basisvektoren $u_k$ bzw.~$v_i$.
Ein Vektor $u\in U$ kann also Linearkombination der $u_k$ geschrieben
werden, und der Bildvektor $Av$ kann als Linearekombination der $v_i$
geschrieben werden.
Wir nennen die zugehörigen Koeffizieten $x_k$ bzw.~$y_i$ und erhalten
\[
u = \sum_{k=1}^n x_ku_k
\qquad\text{und}\qquad
v = \sum_{j=1}^m y_jv_j.
\]
Der Zusammenhang zwischen den Koeffizienten ist
\[
y_j = \sum_{k=1}^n a_{jk}x_k,\qquad j=1,\dots,m.
\]
Der Operator $A$ ist also in der Basis beschrieben durch die Matrixelemente
$a_{jk}$.

Der Funktionenraum der glatten Funktionen ist unendlichdimensional, die
Idee, mit einer Basis eine Matrix zu konstruieren, ist also nicht
durchführbar.
Die Fourier-Theorie zeigt aber einen möglichen Weg auf, wie man etwas
Analoges konstruieren könnte.
Für Funktionen auf $\mathbb{R}/2\pi\mathbb{Z}$ oder für $2\pi$-periodische
Funktionen $f$ sind die Funktionen $e_k(x)=e^{ikx}$ mit $k\in\mathbb{Z}$ 
eine Hilbertbasis.
Die Fourier-Koeffizienten
\[
\hat{f}_k
=
c_k
=
\langle e_k,f\rangle
=
\frac{1}{\!\sqrt{2\pi}}
\int_{-\pi}^{\pi} e^{-ikx}f(x)\,dx
\]
entsprechen den Koordinaten.
Man könnte daher als verallgemeinerten linearen Operator Koeffizienten
$a_{jk}$ wählen, damit die Fourier-Koeffizienten wie mit einer Matrix
transformieren und anschliessend die Funktion durch Rücktransformation
zurückgewinnen.
Die Wirkung dieses Operators ist also gegeben durch
\begin{equation}
Af(y)
=
\sum_{j\in\mathbb{Z}}
e^{ijy}
\sum_{k\in\mathbb{Z}} a_{jk}
c_k
=
\frac{1}{2\pi}
\sum_{j\in\mathbb{Z}}
e^{ijy}
\sum_{k\in\mathbb{Z}} a_{jk}
\int_{-\pi}^\pi e^{-ikx} f(x)\,dx.
\label{buch:operatoren:pseudodifferential:eqn:matrixop}
\end{equation}
Mit dieser Art von Operator lassen sich alle obengannten Operationen
durchführen.

\subsubsection{Ableitung}
Die Fourier-Transformierte der Ableitung $f'(x)$
\begin{align*}
\langle e_k,f'\rangle
=
\frac{1}{\!\sqrt{2\pi}}
\int_{-\pi}^\pi e^{-ikx}f'(x)\,dx
=
\frac{1}{\!\sqrt{2\pi}}
\biggl[
e^{-ikx}
f(x)
\biggr]_{-\pi}^\pi
+ik
\frac{1}{\!\sqrt{2\pi}}
\int_{-\pi}^\pi e^{-ikx} f(x)\,dx
=
ikc_k.
\end{align*}
Der Ableitungsoperator kann also mit Koeffizienten
\[
a_{jk}
=
\begin{cases}
-ik&\qquad\text{falls $j=k$}\\
0  &\qquad\text{sonst}
\end{cases}
\]
in der Form~\eqref{buch:operatoren:pseudodifferential:eqn:matrixop}
geschrieben werden.

\subsubsection{Faltungsoperator}
Für die Faltung mit einer Funktion $g(x)$ gilt der Faltungssatz
\[
\widehat{g*f}_k
=
\hat{g}_k\cdot \hat{f}_k
\]
der Fourier-Transformation.
Setzt man
\[
A_{jk}
=
\begin{cases}
\hat{g}_k&\qquad\text{falls $j=k$}\\
0        &\qquad\text{sonst,}
\end{cases}
\]
kann man auch den Faltungsoperator $f\mapsto g*f$ in der 
Form~\eqref{buch:operatoren:pseudodifferential:eqn:matrixop}
schreiben.

\subsubsection{Multiplikation}
Der Multiplikationsoperator
\[
M_g\colon
C^{\infty}(\mathbb{R}/2\pi\mathbb{Z})
\to
C^{\infty}(\mathbb{R}/2\pi\mathbb{Z})
:
f\mapsto M_gf : x \mapsto g(x)f(x)
\]
wird durch die Fourier-Transformation in eine Faltung mit den
Fourier-Koeffizienten von $g$ umgewandelt:
\[
\widehat{M_gf}_j
=
\sum_{l\in\mathbb{Z}}
\hat{g}_{j-k}
\hat{f}_k.
\]
Setzt man
\[
A_{jk}
=
\hat{g}_{j-k},
\]
bekommt auch der Multiplikationsoperator mit der Funktion $g$ die 
Form~\eqref{buch:operatoren:pseudodifferential:eqn:matrixop}.

Die Beispiele zeigen, dass die Fourier-Transformation ermöglich,
alle drei eingangs erwähnten Arten von Operatoren auf eine einheitliche
Art zu beschreiben.

%
% Integraloperatoren
%
\subsection{Integraloperatoren}
Die Idee der Transformation periodischer Funktionen auf
Fourier-Koeffizienten mit anschliessender Abbildung der
Koeffizienten mit einer unendlich ausgedehnten Matrix funktioniert
nicht mehr für glatte Funktionen $f(x)$ auf $\mathbb{R}$.
Die Fourier-Transformierte $\hat{f}(k)$ einer solchen Funktion ist wieder
eine Funktion auf $\mathbb{R}$.
Als Indexmenge für eine Summation kommt $\mathbb{R}$ nicht in Frage.
Man kann aber die Summation durch eine Integration ersetzen.

\begin{definition}
Seien $X$ und $Y$ Teilmengen von $\mathbb{R}$
und $K\colon X\times Y\to\mathbb{R}$ eine stetige Funktion.
Dann definiert $K$ einen linearen Operator
\[
K
\colon
C(Y) \to C(X)
:
f \mapsto K\cdot f: x\mapsto \int_Y K(x,y) f(y)\,dy,
\]
er heisst der {\em Integraloperator mit Kernfunktion $K(x,y)$}.
\end{definition}

\begin{beispiel}
Sei $g$ eine stetige Funktion mit kompaktem Träger auf $\mathbb{R}$.
Dann ist der Faltungsoperator
\[
g*
\colon 
C_0(\mathbb{R})
\to
C_0(\mathbb{R})
:
f
\mapsto g*f
\]
für stetige Funktionen mit kompaktem Träger der Integraloperator
\[
(g*f)(x)
=
\int_{\mathbb R} g(x-y) f(y)\,dy
\]
mit der Kernfunktion $K(x,y) = g(x-y)$.
\end{beispiel}

Solche Integraloperatoren sind sehr praktisch bei der Lösung von
partiellen Differentialgleichungen.
In einem gewissen Sinne lassen sich Ableitungsoperatoren mit
Integraloperatoren invertieren, wie das folgende Beispiel illustriert.

\begin{beispiel}
Sei $f(x)$ eine stetige Funktion auf dem Intervall $[0,1]$.
Gesucht ist eine Funktion $u(x)$, die Lösung der Differentialgleichung 
\begin{equation}
\frac{d^2}{dx^2}u(x) = f(x)
\label{buch:operatoren:pseudodifferential:eqn:dgl1}
\end{equation}
ist mit Randbedingungen $u(0)=u(1)=0$.
Wir zeigen, dass der Integraloperator mit der Kernfunktion
\[
G(x,y)
=
\begin{cases}
y(x-1)&\qquad\text{für $x > y$}\\
x(y-1)&\qquad\text{für $x\le y$}
\end{cases}
\]
ein linearer Operator $C([0,1]) \to C^2([0,1])$ ist, der eine Lösung
$u=G\cdot f$ der Differentialgleichung mit homogenen Randbedingungen 
liefert.

Der Operator $G$ liefert die Funktion
\begin{align*}
u(x)
&=
\int_0^1 G(x,y)f(y)\,dy
=
\int_0^x y(x-1)f(y)\,dy
+
\int_x^1 x(y-1)f(y)\,dy.
\intertext{Wir berechnen die Ableitung nach $x$:}
u'(x)
&=
y(x-1)f(y)\bigg|_{x=y} +  \int_0^x y f(y)\,dy
-
x(y-1)f(y)\bigg|_{x=y} + \int_x^1 (y-1)f(y)\,dy
\\
&=
x(x-1)f(x) + \int_0^x yf(y)\,dy
-x(x-1)f(x) + \int_x^1 (y-1)f(y)\,dy
\\
&= \int_0^x yf(y)\,dy + \int_x^1 (y-1)f(y)\,dy.
\intertext{Die zweite Ableitung ist}
u''(x)
&=
xf(x) - (x-1)f(x)
=
f(x).
\end{align*}
Somit ist gezeigt, dass der Operator $G$ eine Inverse der zweiten
Ableitung ist.
Die Funktion $G(x,y)$ heisst auch die Greensche Funktion der
Differentialgleichung~\eqref{buch:operatoren:pseudodifferential:eqn:dgl1}.
\end{beispiel}


%
% Differentialoperatoren
%
\subsection{Differentialoperatoren}
Der Integraloperator mit Kernfunktion $K(x,y)$ erzeugt aus einer Funktion
$f(y)$ die Funktion $K\cdot f$.
Wir betrachten den Punkt $x_0\in X$.
Der Funktionswert
\[
(K\cdot f)(x_0)
=
\int_Y K(x_0,y)f(y)\,dy
\]
hängt von allen Werten der Funktion $f(y)$ für $y\in Y$ ab, für die
$K(x_0,y)\ne 0$.
Für jeden inneren Punkt $y_0\in \operatorname{supp} K \cap \{x_0\}\times Y$
gibt es eine Funktion $h\colon Y\to\mathbb{R}$ mit Träger in 
$\operatorname{supp} K \cap \{x_0\}\times Y$ derart, dass
$(K\cdot h)(x_0)\ne 0$ und damit $K\cdot(f+h)\ne K\cdot f$.

Ein Differentialoperator ist dagegen perfekt lokalisiert im
folgenden Sinn.
Sei $f$ eine differenzierbare Funktion Funktion auf $\mathbb{R}$
und $x_0\in \mathbb{R}$.
Für einen Punkt $x\ne x_0$ sei $g$ eine Funktion, die sich nur 
im Intervall $(x-\varepsilon,x+\varepsilon)$ mit $\varepsilon=\frac12|x-x_0|$
von der Funktion $f$ unterscheidet.
Dann sind alle Ableitungen von $f$ und $g$ im Punkt $x_0$ gleich:
\[
f^{(n)}(x) = g^{(n)}(x) \quad \forall\; n\in\mathbb{N}.
\]
Die Funktion $f$ kann also beliebig nahe beim Punkt $x_0$ verändert
werden, ohne dass dies die Ableitungen im Punkt $x_0$ ändert.
Man sagt auch, die Ableitung ist eine Distribution mit Träger im
Punkt $x_0$.

Dieses Beispiel zeigt, dass es unrealistisch ist zu erwarten, dass
die Ableitung als Integraloperator ausgedrückt werden kann.
In Abschnitt~\ref{buch:operatoren:pseudodifferential:subsection:motivation}
wurden Ableitungsoperatoren durch Fourier-Transformation als 
Summen ausgedrückt.
Die Indexmenge für die Summe war die duale Gruppe der
Definitionsbereichsgruppe $\mathbb{R}/2\pi\mathbb{Z}$.
Für Funktionen auf $\mathbb{R}$ ist die duale Gruppe $\mathbb{R}$, so
die analoge Vorgehensweise dass statt auf eine Summe auf ein
Integral führt.
Wir versuchen daher, einen Differentialoperator im Frequenzbereich als
Integraloperator zu schreiben.

Sei also $f$ eine differenzierbare Funktion auf $\mathbb{R}$ mit kompaktem
Träger und 
\[
\hat{f}(k)
=
\frac{1}{\!\sqrt{2\pi}} \int_{\mathbb{R}} e^{-ikx} f(x)\,dx
\]
die Fourier-Transformierte.
Die Ableitung von $f$ hat die Fourier-Transformierte
\begin{align*}
\widehat{f'}(k)
&=
\frac{1}{\!\sqrt{2\pi}}
\int_{\mathbb{R}} e^{-ikx} f'(x)\,dx
\\
&=
\frac{1}{\!\sqrt{2\pi}}
\biggl[
e^{-ikx}f(x)
\biggr]_{-\infty}^\infty
+
ik
\frac{1}{\!\sqrt{2\pi}}
\int_{-\infty}^\infty e^{-ikx}f(x)\,dx
\\
&=
ik \hat{f}(k).
\end{align*}
Durch Fourier-Inversion lässt sich daraus die Ableitung bestimmen:
\begin{align}
f'(y)
&=
\frac{1}{\!\sqrt{2\pi}}
\int_{-\infty}^\infty
e^{iky} 
ik \hat{f}(k)
\,dk
\notag
\\
&=
\frac{1}{2\pi}
\int_{-\infty}^\infty
e^{iky} ik
\int_{-\infty}^\infty e^{-ikx} f(x) \,dx
\,dk.
\intertext{Die formale Vertauschung der Integrationen liefert die Form}
&=
\int_{-\infty}^\infty
\underbrace{
\frac{1}{2\pi}
\int_{-\infty}^\infty
ik e^{ik(y-x)}
\,dk
}_{\displaystyle =K(y,x)}
f(x)
\,dx.
\label{buch:operatoren:pseudodifferential:eqn:ablkern}
\end{align}
Wenigstens formal ist das innere Integral eine ``Kernfunktion'', mit der
der Ableitungsoperator als Integraloperator geschrieben werden kann.

Das Problem mit der ``Kernfunktion''
\eqref{buch:operatoren:pseudodifferential:eqn:ablkern}
ist jedoch, dass es gar keine Funktion ist.
Zum Beispiel führt die Formel für $x=y$ auf
\[
K(x,x)
=
\frac{1}{2\pi}
\int_{-\infty}^\infty 
ik\,dk,
\]
was nicht konvergiert.
Das Problem ist nicht spezifisch für die Ableitung, auch die identische
Abbildung, die die Funktion $f(x)$ via
\begin{align}
f(y)
&=
\frac{1}{\!\sqrt{2\pi}}
\int_{-\infty}^\infty
e^{iky}
\frac{1}{\!\sqrt{2\pi}}
\int_{-\infty}^\infty
e^{-ikx}
f(x)
\,dx
\,dk
=
\int_{-\infty}^\infty
\underbrace{
\frac{1}{2\pi}
\int_{-\infty}^\infty
e^{ik(y-x)}
\,dk
}_{\displaystyle = K(y,x)}
f(x)
\,dx
\label{buch:operatoren:pseudodifferential:eqn:delta}
\end{align}
mit der gleichen formalen Vertauschung der Integrationen reproduzieren
soll.
Auch in diesem Fall ist das innere Integral nicht konvergent.
Andererseits ist bekannt, dass die ``Dirac-\textdelta-Funktion'' 
mit der Eigenschaft
\[
f(y) = \int_{-\infty}^\infty \delta(y-x) f(x)\,dx
\]
die Funktion $f$ reproduzieren kann, man muss daher dem inneren
Integral in \eqref{buch:operatoren:pseudodifferential:eqn:delta} den
Wert
\[
\frac{1}{2\pi}
\int_{-\infty}^\infty e^{ik(y-x)} \,dk
=
\delta (y-x)
\]
zuweisen.
Für
\eqref{buch:operatoren:pseudodifferential:eqn:ablkern}
muss man
\[
\frac{1}{2\pi}
\int_{-\infty}^\infty ike^{ik(y-x)} \,dk
=
\delta'(y-x)
\]
setzen.

Offenbar funktioniert diese Idee nicht, wenn man sich auf den 
klassischen Begriff der Funktion beschränkt.
Mindestens für die Dirac-\textdelta-Funktion und ihre Ableitung
scheint es aber plausibel zu sein, dass es einen erweiterten
Funktionsbegriff geben könnte, mit dem diesen ``Funktionen'' $K(y,x)$
ein wohldefinierter Sinn gegeben werden kann.
Dies ist der Begriff der Distribution, auf den hier aber nicht
im Detail eingegangen werden soll.
Die Fourier-Transformation kann auf Distributionen ausgedehnt und
damit kann eine konsistene Theorie von Operatoren der folgenden Form

\begin{definition}
Ein {\em Pseudodifferentialoperator} für Funktionen $f$ auf $\mathbb{R}$ ist
ein Operator der Form
\[
(K\cdot f)(x)
=
\int_{-\infty}^\infty
\frac{1}{2\pi}
\int_{-\infty}^\infty
e^{ik(x-y)}
a(x,k)
\,dk
f(y)\,dy
\]
wobei $a$ eine Funktion $\mathbb{R}^2\to\mathbb{C}$ ist.
Die Funktion $a$ heisst das {\em Symbol} des Operators.
\end{definition}

Die Klasse der Pseudodifferentialoperatoren ist gross genug, dass sie
viele Operatoren enthalten kann.
Die folgenden Beispiele zeigen, dass Differntialoperatoren,
Faltungsoperatoren und Multiplikationsoperatoren alle von dieser
Form sind.

\begin{beispiel}
Der Laplace-Operator ist der Pseudodifferentialoperator mit dem
dem Symbol
\(
a(x,k)=
-k^2
\).
Im Falle eines reinen Differentialoperators ist also das Symbol eine 
Funktion, die nicht von $x$ und nur von $k$ abhängt.
\end{beispiel}

\begin{beispiel}
Der Faltungsoperator mit der Funktion $g$ wird nach Fourier-Transformation
zu einem Multiplikationsoperator mit der Fourier-Transformierten $\hat{g}$:
\begin{align*}
(g * f)(x)
&=
\frac{1}{\!\sqrt{2\pi}}
\int_{-\infty}^\infty e^{ikx} \hat{g}(k) \hat{f}(k) \,dk
=
\frac{1}{\!\sqrt{2\pi}}
\int_{-\infty}^\infty
e^{ikx}
\hat{g}(k) 
\frac{1}{\!\sqrt{2\pi}}
\int_{-\infty}^\infty
e^{-iky}
f(x)
\,dy
\,dk
\\
&=
\int_{-\infty}^\infty
\frac{1}{2\pi}
\int_{-\infty}^\infty
e^{ik(x-y)}
\hat{g}(k)
\,dk
\,dy
=
\int_{-\infty}^\infty
g(x-y)
\,dy,
\end{align*}
das Symbol des Faltungsoperators ist somit $a(x,k)=\hat{g}(k)$.
\end{beispiel}

\begin{beispiel}
Die identische Abbildung kann als der Pseudodifferentialoperator
\[
f(x)
=
\int_{-\infty}^\infty 
\frac{1}{2\pi}
\int_{-\infty}^\infty 
e^{ik(x-y)}
\,dx
f(y)
\,dy
\]
mit dem Symbol $a(x,k)=1$ geschrieben werden.
Der Multiplikationsoperator $M_g$ mit der Funktion $g$ ergibt an
der Stelle $x$
\[
(M_gf)(x)
=
g(x) f(x)
=
\int_{-\infty}^\infty
\frac{1}{2\pi}
\int_{-\infty}^\infty
g(x) e^{ik(x-y)}
\,dk
\,dy.
\]
Daraus liest man ab, dass der Multiplikationsoperator $M_g$ ein
Pseudodifferentialoperator mit dem Symbol $a(x,k)=g(x)$ ist.
\end{beispiel}





%
% 2-casimir.tex
%
% (c) 2022 Prof Dr Andreas Müller, OST Ostschweizer Fachhochschule
%
\section{Casimir-Operator
\label{buch:operatoren:section:casimir}}
\kopfrechts{Casimir-Operator}



