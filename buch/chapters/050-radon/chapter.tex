%
% chapter.tex -- Radon-Transformation
%
% (c) 2021 Prof Dr Andreas Müller, Hochschule Rapperswil
%
% !TeX spellcheck = de_CH
\chapter{Radon-Transformation
\label{buch:chapter:radon}}
\kopflinks{Radon-Transformation}
Die Fourier-Transformation für Funktionen auf $\mathbb{R}^n$ findet
Wellenkomponenten, die in verschiedene Richtungen laufen. 
Es ist daher naheliegend, das Fourier-Integral in
ein Integral parallel zu den Wellenfronten und ein Integral in 
Ausbreitungsrichtung zu zerlegen.
Die Radon-Transformation erreicht genau dies und ermöglicht so
ein geometrisches Verständnis für ein Wellenfeld.
Ausserdem macht sie Anwendungen wie den CAT-Scanner oder 
MRI möglich.

%
% 1-definition.tex
%
% (c) 2023 Prof Dr Andreas Müller, OST Ostschweizer Fachhochschule
%
\section{Definition
\label{buch:skalarprodukte:section:definition}}
\kopfrechts{Definition}
Ein Skalarprodukt ist vor allem deshalb besonders einfach anzuwenden,
weil es bilinear ist.
Dies bedeutet, dass man Skalarprodukte ausmultiplizieren kann, die
Intuition für Produkte, die man aus der elementaren Algebra mitbringt,
führt zum Erfolg.
Allerdings braucht es für ein erfolgreiches Skalarprodukt noch
etwas mehr.

%
% Symmetrische Bilinearformen
%
\subsection{Symmetrische Bilinearformen}
Das aus der Vektorgeometrie bekannte Skalarprodukt
\index{Vektorgeometrie}%
\index{Skalarprodukt}%
\[
\vec{u}\cdot \vec{v}
=
\sum_{i=1}^n u_iv_i
\]
auf $\mathbb{R}^n$ ist deshalb besonders nützlich, weil sich damit
so rechnen lässt, wie man es sich von einem Produkt in der Algebra
gewohnt ist.
Dazu gehört, dass man Produkte ausmultiplizieren kann:
\begin{equation}
\begin{aligned}
(\lambda\vec{u}+\mu\vec{w})\cdot\vec{v}
&=
\lambda\vec{u}\cdot\vec{v}+\mu\vec{w}\cdot\vec{v}
\\
\vec{u}\cdot(\lambda\vec{v}+\mu\vec{w})
&=
\lambda\vec{u}\cdot\vec{v}+\mu\vec{u}\cdot\vec{w}.
\end{aligned}
\label{buch:skalarprodukt:eqn:ausmultiplizieren}
\end{equation}
Umgekehrt kann man gemeinsame Faktoren auch ausklammern.
Die Rechenregeln \eqref{buch:skalarprodukt:eqn:ausmultiplizieren}
besagen, dass die Funktion
\[
\cdot
\;
\colon
\mathbb{R}^n \times \mathbb{R}^n
\to
\mathbb{R}
:
(\vec{u},\vec{v}) \mapsto \vec{u}\cdot\vec{v}
\]
in jedem Faktor linear ist.
Die folgende Definition verallgemeinert die Idee auf einen
beliebigen reellen Vektorraum $V$.

\begin{definition}
Eine Funktion
\[
b\colon
V\times V \to \mathbb{R}
:
(u,v) \mapsto b(u,v)
\]
heisst {\em bilinear} oder {\em Bilinearform},
wenn sie linear ist in jedem Argument, wenn also
\index{bilinear}%
\index{Bilinearform}%
\[
\begin{aligned}
b(\lambda u+\mu w,v) &= \lambda b(u,v) + \mu(w,v)
\\
\text{und}\qquad
b(u,\lambda v+\mu w) &= \lambda b(u,v) + \mu(u,w)
\end{aligned}
\]
gilt für beliebige Vektoren $u,v,w\in V$ und Skalare
$\lambda,\mu\in\mathbb{R}$.
\end{definition}

Das Skalarprodukt der Vektorgeometrie hat aber noch eine weitere
wichtige Eigenschaft.
Es ist kommutativ, es kommt nicht auf die Reihenfolge der Faktoren an.
Dies ist zum Beispiel wichtig, um den Kosinus-Satz der ebenen Trigonometrie
zu erhalte.
Der für den Vektor
\index{Kosinus-Satz}%
\index{Trigonometrie}%
$\vec{c} = \vec{b}-\vec{a}$ 
kann man ihn durch Berechnung des Skalarproduktes von $\vec{c}$ mit
sich selbst als
\begin{align*}
|\vec{c}|^2
&=
\vec{c}\cdot\vec{c}
=
(\vec{b}-\vec{a})\cdot(\vec{b}-\vec{a})
=
\vec{b}\cdot\vec{b}
-
\vec{b}\cdot\vec{a}
-
\vec{a}\cdot\vec{b}
+
\vec{a}\cdot\vec{a}
\\
&=
|\vec{a}|^2 + |\vec{b}|^2 - 2 \vec{a}\cdot\vec{b}
=
|\vec{a}|^2 + |\vec{b}|^2 - 2 |\vec{a}|\;|\vec{b}|\cos\alpha
\end{align*}
erhalten.
Dabei wurde verwendet, dass $\vec{a}\cdot\vec{b}=\vec{b}\cdot\vec{a}$ ist.
Für ein Skalarprodukt heisst die Kommutativität allerdings anders.

\begin{definition}
Eine Funktion $b\colon V\times V \to\mathbb{R}$ heisst {\em symmetrisch},
wenn $b(u,v)=b(v,u)$ für alle $u,v\in V$.
\index{symmetrisch}%
\end{definition}

Eine symmetrische Bilinearform erfüllt die binomische Formel
\begin{align*}
b(u+v,u+v)
&=
b(u,u+v) + b(v,u+v)
=
b(u,u)+b(u,v)+b(v,u)+b(v,v)
\\
&=
b(u,u) + 2b(u,v) + b(v,v),
\end{align*}
ein weiteres Indiz dafür, dass das Skalarprodukt bei der algebraischen
Rechnung wie ein ``gewöhnliches'' Produkt behandelt werden kann.

Eine Bilinearform auf einem endlichdimesionalen Vektorraum $V$
kann mit Hilfe einer Basis der Berechnung leichter zugänglich
gemacht werden.
Seien $b_1,\dots,b_n\in V$ die Vektoren einer Basis von $V$.
Dann können Vektoren $u,v\in V$ mit Hilfe der Koordinaten
$u_i\in\mathbb{R}$ und $v_i\in\mathbb{R}$ als Linearkombinationen
\[
u = \sum_{i=1}^n u_ib_i,
\qquad \text{und}\qquad
v = \sum_{i=1}^n v_ib_i
\]
aus Basisvektoren geschrieben werden.
Für das Skalarprodukt folgt dann
\begin{align*}
\langle u,v\rangle
&=
\biggl\langle \sum_{i=1}^n u_ib_i, \sum_{k=1}^n v_kb_k\biggr\rangle
=
\sum_{i=1}^n
\sum_{k=1}^n
u_i \underbrace{\langle b_i,b_k\rangle}_{\displaystyle = g_{ik}} v_k.
\end{align*}
Schreibt man die $u_i$ und $v_k$ in einen $n$-dimensionalen Spaltenvektor
und die $g_{ik}$ in eine $n\times n$-Matrix $G$, kann das Skalarprodukt in
Matrixschreibweise als
\[
\langle u,v\rangle
=
\transpose{%
\begin{pmatrix}
u_1\\u_2\\\vdots\\u_n
\end{pmatrix}}
G
\begin{pmatrix}
v_1\\v_2\\\vdots\\v_n
\end{pmatrix}
\]
geschrieben werden.
Die Matrix $G$ heisst die Matrix des Skalarproduktes in der Basis
oder auch die {\em Gram-Matrix}.
\index{Gram-Matrix}%
Die Eigenschaften des Skalarproduktes schlagen sich in Eigenschaften
dieser Matrix nieder.
Die Symmetrie des Skalarproduktes hat zum Beispiel zur Folge, dass
\[
g_{ik} = \langle b_i,b_k\rangle = \langle b_k,b_i\rangle = g_{ki},
\]
die Matrix $G$ ist also symmetrisch.

%
% Norm
%
\subsection{Norm}
In der Vektorgeometrie wird das Skalarprodukt auch dazu verwendet,
mit $|\vec{v}|^2 = \vec{v}\cdot\vec{v}$ 
die Länge eines Vektors zu berechnen.
Jeder Vektor $\ne 0$ hat eine positive Länge.
Für eine beliebige Bilinearform ist jedoch nicht automatisch
sichergestellt, dass $b(u,u)\ne 0$ ist für $u\ne 0$.
Ausserdem kann ein Längenbegriff nur dann aus $b$ abgeleitet werden,
wenn zusätzlich $b(u,u)>0$ ist für $u\ne 0$, da sich andernfalls
die Wurzel nicht ziehen lässt.

\begin{beispiel}
\label{buch:skalarprodukt:definition:bsp:hyperbolisch}
Die symmetrische Bilinearform
\[
b(x,y)
=
x_1y_1-x_2y_2
\]
auf $\mathbb{R}^2$ ist nicht dazu geeignet, eine Länge zu definieren,
dann der zweite Standardbasisvektor $e_2$ hat das Produkt
$b(e_2,e_2) = -1$.
Auch gibt es einen Vektor, der ``Länge'' 0 hat,
nämlich $v=e_1+e_2$ mit
\[
b(v,v)
=
b(e_1+e_2,e_1+e_2)
=
b(e_1,e_1) + 2\underbrace{b(e_1,e_2)}_{\displaystyle =0} + b(e_2,e_2)
=
1-1
=
0.
\qedhere
\]
\end{beispiel}

\begin{definition}
Eine symmetrische Bilinearform $\langle\;\,,\;\rangle$
heisst {\em positiv definit}, wenn $\langle u,u\rangle > 0$ 
für alle von 0 verschiedenen Vektoren $u\ne 0$ gilt.
\end{definition}

\begin{definition}
\label{buch:skalarprodukt:definition:def:skalarprodukt}
Eine {\em Skalarprodukt} ist eine positiv definite, symmetrische Bilinearform.
\index{Skalarprodukt}%
\end{definition}

%
% kreise.tex -- Kreise von skalarprodukten
%
% (c) 2021 Prof Dr Andreas Müller, OST Ostschweizer Fachhochschule
%
\documentclass[tikz]{standalone}
\usepackage{amsmath}
\usepackage{times}
\usepackage{txfonts}
\usepackage{pgfplots}
\usepackage{csvsimple}
\usetikzlibrary{arrows,intersections,math}
\begin{document}
\def\skala{1}
\def\l{2.1}
\def\axes{
\draw[->] (-\l,0) -- (\l,0) coordinate[label={$x_1$}];
\draw[->] (0,-\l) -- (0,\l) coordinate[label={$x_1$}];
}
\begin{tikzpicture}[>=latex,thick,scale=\skala]

\begin{scope}[xshift=-4.5cm]
\axes
\draw[color=red,line width=1.4pt] (0,0) circle[radius=1.5];
\draw (1.5,-0.05) -- (1.5,0.05);
\draw (-1.5,-0.05) -- (-1.5,0.05);
\draw (-0.05,1.5) -- (0.05,1.5);
\draw (-0.05,-1.5) -- (0.05,-1.5);
\node at (1.5,0) [below right] {$1$};
\node at (-1.5,0) [below left] {$-1$};
\node at (0,1.5) [above left] {$1$};
\node at (0,-1.5) [below left] {$-1$};
\node at (0,{-\l-0.3}) {$x_1^2+x_2^2=1\mathstrut$};
\end{scope}

\begin{scope}
\axes
\def\s{1.4142}
\begin{scope}[rotate=45]
\draw[color=red,line width=1.4pt] (0,0) ellipse (1cm and 2cm);
\end{scope}
\draw (\s,-0.05) -- (\s,0.05);
\draw (-\s,-0.05) -- (-\s,0.05);
\draw (-0.05,\s) -- (0.05,\s);
\draw (-0.05,-\s) -- (0.05,-\s);
\node at (\s,0) [below right] {$1$};
\node at (-\s,0) [below left] {$-1$};
\node at (0,\s) [above right] {$1$};
\node at (0,-\s) [below left] {$-1$};
\draw[line width=0.3pt] (\l,-\l) -- (-\l,\l);
\draw[line width=0.3pt] (-\l,-\l) -- (\l,\l);
\node at (0,{-\l-0.3}) {$5x_1^2+6x_1x_2+5x_2^2=4\mathstrut$};
\fill[color=red] ({0.5*\s},{0.5*\s}) circle[radius=0.06];
\fill[color=red] ({-0.5*\s},{-0.5*\s}) circle[radius=0.06];
\fill[color=red] ({\s},{-\s}) circle[radius=0.06];
\fill[color=red] ({-\s},{\s}) circle[radius=0.06];
\end{scope}

\begin{scope}[xshift=4.5cm]
\axes
\draw[color=red,line width=1.4pt]
	plot[domain=-1.4:1.4,samples=20]
		({cosh(\x)},{sinh(\x)});
\draw[color=red,line width=1.4pt]
	plot[domain=-1.4:1.4,samples=20]
		({-cosh(\x)},{sinh(\x)});
\draw (1,-0.05) -- (1,0.05);
\draw (-1,-0.05) -- (-1,0.05);
\draw (-0.05,1) -- (0.05,1);
\draw (-0.05,-1) -- (0.05,-1);
\node at (0,1) [above left] {$1$};
\node at (0,-1) [below left] {$-1$};
\node at (1,0) [above left] {$1$};
\node at (-1,0) [above left] {$-1$};
\node at (0,{-\l-0.3}) {$x_1^2-x_2^2=1\mathstrut$};
\end{scope}

\end{tikzpicture}
\end{document}

%

Verschiedene Skalarprodukte in zwei Dimensionen kann man durch ihre
zugehörten ``Kreise'' visualisieren
(Abbildung~\ref{buch:skalarprodukt:definition:fig:kreise}).
Damit meinen wir die Menge der Vektoren, die Norm $1$ haben.
Abbildung~\ref{buch:skalarprodukt:definition:fig:kreise} zeigt
die Kreise für drei verschiedene Skalarprodukte.
Links ist der Kreis für das Standard\-skalarprodukt $x_1y_1+x_2y_2$
dargestellt, der tatsächlich die Form eines Kreises hat.
Das Skalarprodukt von
Beispiel~\ref{buch:skalarprodukt:definition:bsp:hyperbolisch}
führt auf die Kurven mit der Gleichung
\[
x_1^2-x_2^2=1,
\]
dies sind Hyperbeln.
Es heisst daher auch das hyperbolische Skalarprodukt, obwohl es
\index{Skalarprodukt!hyperbolisch}
im strengen Sinne der
Definition~\ref{buch:skalarprodukt:definition:def:skalarprodukt}
kein Skalarprodukt ist.
In der Mitte schliesslich ist der Kreis für das Skalarprodukt 
\[
\langle x,y\rangle
=
2
\cdot
\frac{x_1+x_2}{\!\sqrt{2}}
\cdot
\frac{y_1+y_2}{\!\sqrt{2}}
+
\frac12
\cdot
\frac{x_1-x_2}{\!\sqrt{2}}
\cdot
\frac{y_1-y_2}{\!\sqrt{2}}
=
\frac{5x_1y_1 + 3x_1y_2 + 3x_2y_1 + 5x_2y_2}{2}
\]
dargestellt.
Die zugehörige Norm ist
\[
\|x\|^2
=
\frac14(
5x_1^2 + 5x_2^2 + 6x_1x_2).
\]
Der Kreis bestehen aus den Punkten einer Ellipse mit den Halbachsen
$1$ und $\frac12$, die um $45^\circ$ gegenüber den Koordinatenachsen
verdreht sind.
Diese Ellipse verläuft durch die Punkte $(\pm\frac12,\pm\frac12)$ und
$(\pm1,\mp1)$.

Für die Gram-Matrix $G$ eines Skalarproduktes bedeutet die Forderung,
dass das Skalarprodukt bilinear ist, dass die Matrix $G$ positiv
definit sein muss.
Aus der linearen Algebra ist bekannt, dass $G$ genau dann positiv
definit ist, wenn $G$ eine Cholesky-Zerlegung $G=L\transpose{L}$ hat,
deren Diagonalelemente alle positiv sind.

Ein Skalarprodukt hat jetzt alle Eigenschaften, die erlauben, einen 
Abstandsbegriff zu definieren.

\begin{definition}
\label{buch:skalarprodukt:definition:def:norm}
Die zu einem Skalarprodukt $\langle\;\,,\;\rangle$ gehörige Norm ist
definiert als
\[
\| v\|
=
\!\sqrt{\langle v,v\rangle}
\]
für $v\in V$.
\end{definition}

Die Norm erfüllt $\|\lambda v\| = |\lambda|\,\|v\|$ für jeden Vektor
$v\in V$ und $\lambda\in\mathbb{R}$.
In Worten bedeutet dies, dass bei der Skalierung eines Vektors die Norm
auf die gleiche Art skaliert.

%
% Sesquilineare Funktionen
%
\subsection{Sesquilineare Funktionen}
Sei jetzt $V$ ein komplexer Vektorraum.
Aus einer bilinearen Funktion
\[
b\colon V\times V \to \mathbb{C} : (u,v) \mapsto b(u,v)
\]
auf $V$ kann jedoch keine brauchbare Norm abgeleitet werden.
Eine solche müsste $\| v\|=b(v,v)\ge 0$ erfüllen.
Selbst wenn $b(u,u)> 0$ ist für einen speziellen Vektor $u\in V$,
ist das Skalarprodukt von $iu$ mit sich selbst
\[
b(iu,iu)
=
i^2 b(u,u)
=
-b(u,u)
<
0.
\]
Da $|i|=1$ ist, würde man eher erwarten, dass $iu$ die gleiche 
Länge hat wie $u$, dass also $b(iu,iu)=b(u,u)$.
Bilinearität funktioniert also nicht als Bedingung, um ein Skalarprodukt
zu konstruieren, für welches auch die geometrische Intuition des Abstands
anwendbar bleibt.

\begin{definition}
Eine Funktion $f\colon V\to U$ zwischen komplexen Vektorräumen 
heisst {\em konjugiert linear}, wenn 
\[
f(\lambda u + \mu v)
=
\overline{\lambda} f(u) + \overline{\mu} (v)
\]
ist für alle $u,v\in V$ und $\lambda,\mu\in \mathbb{C}$.
\end{definition}

Im obengenannten Beispiel wird $b(iu,iu)>0$, wenn $b$ im ersten Faktor
konjugiert linear ist.
Dann ist nämlich $b(iu,iu) = -ib(u,iu) = -i^2 b(u,u) = b(u,u)>0$.

\begin{definition}
Eine Funktion
\[
\langle\;\,,\;\rangle
\colon
V\times V \to \mathbb{C}
:
(u,v) \mapsto \langle u,v\rangle
\]
heisst {\em sesquilinear} oder {\em Sesquilinearform}
wenn sie linear ist im zweiten Argument
\end{definition}

Das lateinische Wort {\em sesqui} bedeudet eineinhalb, eine
sesquilineare Funktion ist linear in einem Faktor, aber nur
halb linear im anderen.

\begin{beispiel}
Die Form
\[
\langle u,v\rangle = \sum_{i=1}^n \overline{u}_i v_i
\]
mit $u,v\in \mathbb{C}^n$ ist sesquilinear.
Tatsächlich gilt
\begin{align*}
\langle u,\lambda v+\mu w\rangle
&=
\sum_{i=1}^n \overline{u}_i (\lambda v_i+\mu w_i)
=
\lambda
\sum_{i=1}^n \overline{u}_i v_i
+
\mu
\sum_{i=1}^n \overline{u}_i w_i
=
\lambda\langle u,v\rangle
+
\mu\langle u,w\rangle
\\
\langle \lambda u+\mu w, v\rangle
&=
\sum_{i=1}^n \overline{(\lambda u_i+\mu w_i)}v_i
=
\overline{\lambda}
\sum_{i=1}^n \overline{u}_i v_i
+
\overline{\mu}
\sum_{i=1}^n \overline{w}_iv_i
=
\overline{\lambda}\langle u,v\rangle
+
\overline{\mu}\langle w,v\rangle.
\qedhere
\end{align*}
\end{beispiel}

Eine Sesquilinearform auf einem endlichdimensionalen Vektorraum $V$
kann wie im reellen Fall mit Hilfe einer Matrix beschrieben werden.
Eine Darstellung der Vektoren $u$ und $v$ mit Koordinaten $u_i$ und
$v_i$ in der Basis $b_1,\dots,b_n\in V$ führt auf das Skalarprodukt
\[
f(u,v)
=
f\biggl( \sum_{i=1}^n u_ib_i, \sum_{k=1}^n v_kb_k \biggr)
=
\sum_{i=1}^n\sum_{k=1}^n
\overline{u}_i
\underbrace{f(b_i, b_k)}_{\displaystyle=h_{ik}}
v_k.
\]
In Matrixform mit der Matrix $H$ mit Einträgen $h_{ik}$ ist dies
gleichbedeutend mit
\[
f(u,v)
=
\transpose{
\begin{pmatrix}
\overline{u}_1\\
\overline{u}_2\\
\vdots\\
\overline{u}_n
\end{pmatrix}
}
H
\begin{pmatrix}
v_1\\v_2\\\vdots\\v_n
\end{pmatrix}.
\]
Der transponierte Vektor $\transpose{\overline{u}}$ mit komplex
konjugierten Einträgen heisst auch konjugiert transponiert zu $u$.

\begin{definition}
Ist $A\in M_{m\times n}(\mathbb{C})$ eine komplexe $m\times n$-Matrix.
Die Matrix $\overline{A}\in M_{m\times n}(\mathbb{C})$ mit 
den komplex konjugierten Matrixelementen von $A$ heisst die
(komplex) konjugierte Matrix zu $A$.
\index{komplex konjugierte Matrix}%
\index{konjugierte Matrix}%
Die Matrix 
\[
A^*
=
\transpose{\overline{A}}
=
\transpose{
\begin{pmatrix}
\overline{a_{11}}&\overline{a_{12}}&\dots &\overline{a_{1n}}\\
\overline{a_{21}}&\overline{a_{22}}&\dots &\overline{a_{2n}}\\
\vdots           &\vdots           &\ddots&\vdots           \\
\overline{a_{m1}}&\overline{a_{m2}}&\dots &\overline{a_{mn}}
\end{pmatrix}
}
=
\begin{pmatrix}
\overline{a_{11}}&\overline{a_{21}}&\dots &\overline{a_{n1}}\\
\overline{a_{12}}&\overline{a_{22}}&\dots &\overline{a_{n2}}\\
\vdots           &\vdots           &\ddots&\vdots           \\
\overline{a_{1m}}&\overline{a_{2m}}&\dots &\overline{a_{nm}}
\end{pmatrix}
\in
M_{n\times m}(\mathbb{C})
\]
heisst die {\em transponiert konjugierte} oder {\em hermitesch konjugierte}
\index{hermitesch konjugiert}%
Matrix.
\end{definition}

Eine Sesquilinearform kann immer geschrieben werden als
\(
f(u,v) = u^*Hv
\)
mit einer hermiteschen Matrix $H$.

%
% Hermitesche Formen
%
\subsection{Hermitesche Formen}
Damit aus einer sesquilinearen Funktion eine Norm abgeleitet werden
kann, muss das Produkt $\langle u,u\rangle$ für jeden Vektor $u\in V$
eine reelle Zahl sein.
Selbst für die sesquilineare Funktion
\[
\langle\;\,,\;\rangle
\colon
\mathbb{C}\times\mathbb{C}
\to
\mathbb{C}
:
(u,v) \mapsto i\overline{u}v
\]
ist dies jedoch nicht der Fall, da $\langle 1,1\rangle = i\not\in\mathbb{R}$
ist.
Die folgende Eigenschaft kann aber garantieren, dass
$\langle u,u\rangle\in\mathbb{R}$.

\begin{definition}
Eine sesquilinear Funktion 
\[
\langle \;\,,\;\rangle
\colon
V\times V
\to
\mathbb{C}
\]
heisst in {\em konjugiert symmetrisch} oder {\em hermitesch}, wenn
\index{konjugiert symmetrisch}%
\index{hermitesch!Sesquilinearform}%
\[
\langle u,v\rangle = \overline{\langle v,u\rangle}
\]
für alle $u,v\in V$ gilt.
\end{definition}

\begin{beispiel}
Die Standard-Sesquilinearform
\[
\langle u,v\rangle
=
\sum_{i=1}^n \overline{u}_i v_i
\]
auf $V=\mathbb{C}^n$ ist konjugiert symmetrisch, denn
\begin{align*}
\langle u,v\rangle
&=
\sum_{i=1}^n \overline{u}_i v_i
=
\overline{
\sum_{i=1}^n u_i \overline{v}_i
}
=
\overline{\langle v,u\rangle}.
\qedhere
\end{align*}
\end{beispiel}

Die Gram-Matrix einer hermitesche Sesquilinearform hat die Matrix-Elemente
\[
h_{ik}
=
\langle b_i,b_k\rangle
=
\overline{\langle b_k,b_i\rangle}
=
\overline{h}_{ki}
\qquad\Rightarrow\qquad
H^* = H.
\]
Man sagt auch, die Matrix $H$ ist {\em hermitesch}.
\index{hermitesch!Matrix}%

%
% Komplexe Skalarprodukte
%
\subsection{Komplexe Skalarprodukte}
Wie bei einem reellen Skalarprodukt reichen auch im Fall eines
komplexen Vektorraums die Eigenschaften der Sesquilinearität
und der hermiteschen Symmetrie nicht aus, ein sinnvolles
Skalarprodukt zu definieren.

\begin{definition}
Eine hermitesche Sesquilinearform $\langle\;\,,\;\rangle$
auf dem komplexen Vektorraum $V$ heisst {\em positiv definit}, wenn
\[
\langle u,u\rangle > 0
\]
gilt für alle $u\ne 0$ in $V$.
Ein {\em komplexes Skalarprodukt} ist eine positiv definite hermitesche
Sesquilinearform.
Die zugehörige {\em Norm} eines Vektors ist
$\|v\| = \!\sqrt{\langle u, u\rangle}$.
\end{definition}

Auch für ein komplexes Skalarprodukt gilt die Skalierungseigenschaft
\[
\|\lambda v\|^2
=
\langle \lambda v,\lambda v\rangle
=
\overline{\lambda}\lambda\langle v,v\rangle
=
|\lambda|^2\,\|v\|^2
\qquad\Rightarrow\qquad
\|\lambda v\|
=
|\lambda|\, \|v\|,
\]
ganz analog zur entsprechenden Skalierungseigenschaft für ein
reelles Skalarprodukt.






%
% 2-ableitungen.tex
%
% (c) 2023 Prof Dr Andreas Müller
%
\section{Radon-Transformation und Ableitungen
\label{buch:radon:section:ableitungen}}
Die Fourier-Transformation verwandelt Ableitungen in eine Multiplikation.
Die Radon-Transformation führt einen Teil der Fourier-Transformation
durch, es ist also zu erwarten, dass die \RT ebenfalls
sehr spezielle Ableitungseigenschaften hat.
Es wird sich zeigen, dass die \RT $\mathscr{R}u(s,\omega)$
einer Lösung der Wellengleichung ihrerseits Lösung einer
eindimensionalen Wellengleichung entlang der Richtung $\omega$ ist,
dies ist ein mathematischer Ausdruck für das Huygenssche Prinzip.

%
% Partielle Ableitungen
%
\subsection{Partielle Ableitungen
\label{buch:radon:ableitungen:subsection:partiell}}
Sei $u\colon \mathbb{R}^n\to\mathbb{C}$ eine Funktion mit kompaktem
Träger und sei $\omega = e_1 = (1,0,\dots,0)$ der erste Standarbasisvektor
Da die zugehörige Hyperebene konstante erste Koordinaten hat, ist
die Radon-Transformierte von $u$ gegeben durch
\[
\mathscr{R}u(s,\omega)
\int_{\mathbb{R}^{n-1}}
u(s,x_2,\dots,x_n)\,dx_2\dots\,dx_n.
\]
Die Radon-Transformierte der Ableitung nach $x_1$ ist dann
\begin{align}
\biggl(\mathscr{R}\frac{\partial u}{\partial x_1}\biggr)(s,\omega)
&=
\int_{\mathbb{R}^{n-1}}
\frac{\partial u}{\partial s}(s,x_2,\dots,x_n)
\,dx_2\dots\,dx_n
\notag
\\
&=
\frac{\partial}{\partial s}
\int_{\mathbb{R}^{n-1}}
u(s,x_2,\dots,x_n)
\,dx_2\dots\,dx_n
=
\frac{\partial}{\partial s}\mathscr{R}u(s,\omega).
\notag
\intertext{Durch Iteration findet man auch die Radon-Transformierte
der $m$-ten Ableitung}
\biggl(
\mathscr{R}\frac{\partial^mu}{\partial x_1^m}
\biggr)(s,\omega)
&=
\frac{\partial^m}{\partial s^m}
\mathscr{R}u(s,\omega).
\notag
\intertext{Die Radon-Transformierten einer Ableitung nach einer
anderen Koordinate $x_k$ mit $k>1$
ist dagegen}
\biggl(
\mathscr{R}
\frac{\partial u}{\partial x_k}
\biggr)(s,\omega)
&=
\int_{\mathbb{R}^{n-1}}
\frac{\partial u}{\partial x_k}(s,x_2,\dots,x_k)
\,dx_2\dots\,dx_n
\notag
\\
&=
\int_{\mathbb{R}^{n-2}}
\biggl(
\int_{\mathbb{R}}
\frac{\partial u}{\partial x_k}(s,x_2,\dots,x_n)
\,dx_k
\biggr)
\,dx_2\dots\widehat{dx_k}\dots dx_n
\notag
\\
&=
\int_{\mathbb{R}^{n-2}}
\underbrace{
u(s,x_2,\dots,\infty,\dots,x_n)
-
u(s,x_2,\dots,-\infty,\dots,x_n)
}_{\displaystyle=0}
\,dx_2\dots\widehat{dx_k}\dots dx_n
\\
&=0.
\label{buch:gruppen:radon:ableitungen:eqn:querrichtung}
\end{align}
Im zweitletzten Schritt wurde verwendet, dass $u$ kompakten Träger
hat, also ausserhalb einer beschränkten Menge verschwindet.

%
% Richtungsableitung
%
\subsection{Richtungsableitung
\label{buch:radon:ableitungen:subsection:richtungsableitung}}
Sei jetzt $\omega\in S^{n-1}$ ein beliebiger Einheitsvektor.
Es gibt immer eine Drehmatrix $R$, mit der man den ersten
Standardbasisvektor auf die Richtung $\omega$ drehen kann.
Die Ableitung der zusammengesetzten Funktion $u\circ R$ nach
der ersten Koordinate ist die Richtungsableitung der Funktion
$u$ in Richtung $\omega$.
Die \RT der Richtungsableitung der Funktion $u$ in
Richtung $\omega$ ist die Ableitung nach dem Parameter $s$ der
\RT{}n $\mathscr{R}u(s,\omega)$.
Umkgekehrt verschwindet
nach~\ref{buch:gruppen:radon:ableitungen:eqn:querrichtung}
die \RT der Richtungsableitung
von $u$ in eine Richtung senkrecht auf $\omega$ im Punkt $\omega$.
Damit haben wir folgenden Satz gefunden:

\begin{satz}
Sei $u\colon\mathbb{R}^n\to\mathbb{C}$ eine integrierbare Funktion
mit kompaktem Träger:
Sei weiter $b\perp \omega$ ein zu $\omega$ orthogonaler Einheitsvektor.
Dann sind die Radon-Transformierten der Richtungsableitungen
\begin{align*}
(\mathscr{R}D_\omega u)(s,\omega)
&=
\frac{\partial}{\partial s}\mathscr{R}u(s,\omega)
\\
(\mathscr{R}D_b u)(s,\omega)
&=
0
\end{align*}
\end{satz}

Da man die Richtungsableitung in Richtung des Vektors $v$ mit den
Komponenten $v_i$ auch als
\[
D_v 
=
\sum_{i=1}^n v_i \frac{\partial}{\partial x_i}
\]
schreiben kann, finden wir die Komponentenformeln
\begin{align}
\biggl(
\mathscr{R}
\sum_{i=1}^n
\omega_i\frac{\partial u}{\partial x_i}
\biggr)(s,\omega)
&=
\sum_{i=1}^n
\omega_i
\biggl(\mathscr{R}\frac{\partial u}{\partial x_i}\biggr)
(s,\omega)
=
\frac{\partial}{\partial s}\mathscr{R}u(s,\omega)
\notag
\\
\text{und}\qquad
\sum_{i=1}^n
b_i
\biggl(
\mathscr{R}\frac{\partial u}{\partial x_i}
\biggr)(s,\omega)
&=0.
\label{buch:radon:ableitungen:eqn:babl}
\end{align}
Die Bedingung \eqref{buch:radon:ableitungen:eqn:babl}
bedeutet, dass der Vektor $v$ mit den Komponenten
\[
v_i
=
\biggl(\mathscr{R}\frac{\partial u}{\partial x_i}\biggr)(s,\omega)
\]
orthogonal ist zu allen Vektoren $b\perp \omega$, er hat daher
die Richtung $\omega$.
Da $|\omega|=1$ ist, ist die Länge $|v|=v\cdot \omega$ die Projektion
von $v$ auf die Richtung $\omega$ oder
\[
|v|
=
v\cdot\omega
=
\sum_{i=1}^n \omega_i
\biggl(
\mathscr{R}\frac{\partial u}{\partial x_i}
\biggr)(s,\omega)
=
\frac{\partial}{\partial s}\mathscr{R}u(s,\omega).
\]
Die einzelnen Komponenten sind daher
\[
v_i
=
|v|\omega_i
=
\omega_i
\frac{\partial}{\partial s}\mathscr{R}(s,\omega).
\]

\begin{satz}
Ist $u\colon\mathbb{R}^n\to\mathbb{C}$ eine ingegrierbare Funktion
mit kompaktem Träger, dann ist die Ra\-don-Trans\-for\-ma\-tion der
Ableitungen von $u$
\[
\biggl(
\mathscr{R}
\frac{
\partial^{|\bm{\alpha}|}
}{
\partial x_1^{\alpha_1}\cdots\partial x_n^{\alpha_n}
}
u
\biggr)(s,\omega)
=
(\mathscr{R}
D^{\bm{\alpha}} u
)(s,\omega)
=
\omega_1^{\alpha_1} \cdots \omega_n^{\alpha_n}
\frac{
\partial^{|\bm{\alpha}|}
}{
\partial s^{|\bm{\alpha}|}
}
\mathscr{R}u(s,\omega)
\]
\end{satz}

%
% Laplace-Operator
%
\subsection{Laplace-Operator
\label{buch:radon:ableitungen:subsection:laplace}}
Die Radon-Transformation $\mathscr{R}u$ zerlegt die Information
in der Funktion $u$ in Information, die von der Richung $\omega$
abhängt und solche, die vom Radius $s$ abhängt.
Ableitungen hängen dann nur noch von $s$ ab, was sich auch
auf den Laplace-Operator auswirkt.

\begin{satz}
\label{buch:radon:ableitungen:satz:laplace}
Für eine zweimal stetig differenzierbare Funktion
$u\colon\mathbb{R}^n\to\mathbb{C}$ mit kompaktem Träger gilt
\[
(\mathscr{R}\Delta u)(s,\omega)
=
\frac{\partial^2}{\partial s^2} \mathscr{R}u(s,\omega).
\]
\end{satz}

\begin{proof}[Beweis]
Der Laplace-Operator ist
\[
\Delta
=
\sum_{i=1}^n
\frac{\partial^2}{\partial x_i^2}.
\]
Zusammen mit der Ableitungsformel folgt
\begin{align*}
(\mathscr{R}\Delta u)(s,\omega)
&=
\sum_{i=1}^n
\biggl(\mathscr{R}\frac{\partial^2}{\partial x_i^2}u\biggr)(s,\omega)
\\
&=
\sum_{i=1}^n \omega_i^2 \frac{\partial^2}{\partial s^2}\mathscr{R}u(s,\omega)
\\
&=
|\omega|^2 \frac{\partial^2}{\partial s^2} \mathscr{R}u(s,\omega)
=
\frac{\partial^2}{\partial s^2} \mathscr{R}u(s,\omega).
\qedhere
\end{align*}
\end{proof}

Die Radon-Transformation transformiert den Laplace-Operator
auf einen zweiten Ableitungsoperator nach nur einer einzigen
Variablen.

%
% Wellengleichung
%
\subsection{Wellengleichung
\label{buch:radon:ableitungen:subsection:wellengleichung}}
Der Satz~\ref{buch:radon:ableitungen:satz:laplace} über die
Radon-Transformierte des Laplace-Operators ermöglich, auch die
Lösung der Wellengleichung mit Hilfe der Radon-Transformation zu
vereinfachen.
Die Wellengleichung ist die Gleichung
\[
\frac{\partial^2}{\partial t^2}u = \Delta u
\]
auf dem Gebiet $\Omega=\mathbb{R}^n$.
Nach Anwendung der Radon-Transformation wird aus den beiden Seiten
\begin{align*}
\biggl(\mathscr{R}\frac{\partial^2}{\partial t^2}u\biggr)(s,\omega)
&=
\frac{\partial^2}{\partial t^2}(\mathscr{R}u)(s,\omega)
\\
(\mathscr{R}\Delta u)(s,\omega)
&=
\frac{\partial^2}{\partial s^2}(\mathscr{R}u)(s,\omega)
\end{align*}
Schreibt man $U(t,s\omega)=\mathscr{R}u(s,\omega)$, dann folgt
\[
\frac{\partial^2 U}{\partial t^2}
=
\frac{\partial^2 U}{\partial s^2},
\]
also eine eindimensionale Wellengleichung.
Durch Mittelung über Hyperebenen orthogonal zu $\omega$ macht
die Radon-Transformation aus der $n$-dimensionalen Wellengleichung
eine eindimensionale Wellengleichung für die Ausbreitung des
Mittelwertes $U(t,s,\omega)$ in Richtung $\omega$.
Dieses Prinzip wird manchmal auch als das Huygens-Prinzip 
bezeichnet, da es verwandt ist mit der anschaulichen Vorstellung
von Huygens, wie sich Wellenfronten aus Elementarwellen zusammensetzen.
Letztere wird in Kapitel~\ref{chapter:opt} genauer studiert.



%
% 2-rueckprojektion.tex
%
% (c) 2022 Prof Dr Andraes Müller, OST Ostschweizer Fachhochschule
%
\section{Rückprojektion
\label{buch:radon:section:rueckprojektion}}
\kopfrechts{Rückprojektion}
Sei $u\colon \mathbb{R}^n\to\mathbb{C}$ eine Funktion und
$\mathscr{R}u\colon \mathbb{R}\times S^{n-1}\to\mathbb{C}$
die Radon-Transformierte.
Ändert man die Funktion $u$ in einer kleinen Umgebung eines Punkts $y$,
dann ändert die Radon-Transformiert nur für diejenigen Hyperbenen,
die in der Nähe von $y$ vorbeigehen.
Daraus kann man schliessen, dass genau die Werte
$\mathscr{R}(\omega\cdot y,\omega)$ Information über den Wert der
Funktion $u$ im Punkt $y$ enthalten.
Sie enthalten aber auch noch Information über alle anderen Punkte.

\begin{definition}
Die {\em Rückprojektion} $\mathscr{R}^*f$ einer Funktion
$f\colon \mathbb{R}\times S^{n-1}\to\mathbb{C}$ ist die Funktion
\[
(\mathscr{R}^*f)(y)
=
\int_{S_+^{n-1}} f(\omega\cdot y,\omega)\,d\omega.
\]
$\mathscr{R}^*$ heisst auch die {\em duale Transformation}.
\end{definition}

Die Rückprojektion ist also der Mittelwert der Werte von $f$ über
alle Richtungen, aber in der Entfernung von $y$ vom Nullpunkt.
Wenn $f$ die Radon-Transformierte $f=\mathscr{R}u$ einer Funktion $u$
ist, dann ist die Rückprojektion $(\mathscr{R}^*\mathscr{R}u)(y)$
der Mittelwert aller Werte von $\mathscr{R}u$, die durch Integration
entlang einer Hyperbene durch $y$ entstanden sind.
In der Rückprojektion sind also alle Werte von $u$ gemittelt, aber
der Wert im Punkt $y$ hat besonders grosses Gewicht.
Die duale Transformation ist also eine erste Approximation für $u$.
In den weiteren Entwicklungen hoffen wir, daraus eine exakte Rekonstruktion
von $u$ zu konstruieren.

%
% Rückprojektion und Ableitungen
%
\subsection{Rückprojektion und Ableitungen}
In diesem Abschnitt ist $v\colon \mathbb{R}\times S^{n-1}\to\mathbb{C}$
eine differenzierbare Funktion mit kompaktem Träger, so dass alle
im Folgenden betrachteten Integrale und Ableitungen existierten.
Aus der Definition
\begin{align*}
\mathscr{R}^*v(x)
&=
\int_{S_+^{n-1}} v(\omega\cdot x,\omega)\,d\omega
\intertext{kann man jetzt auch die Ableitung nach $x_i$ berechnen und
bekommt}
\frac{\partial}{\partial x_i}\mathscr{R}^*v(x)
&=
\int_{S_+^{n-1}}
\frac{\partial}{\partial x_i} v(\omega\cdot x,\omega)\,d\omega
\\
&=
\int_{S_+^{n-1}}
\omega_i
\frac{\partial v}{\partial s}(\omega\cdot x,\omega)
\,d\omega
\end{align*}
Für den Laplace-Operator findet man dann
\begin{align}
\Delta\mathscr{R}^*v
&=
\int_{S_+^{n-1}}
\underbrace{
\biggl(\sum_{i=1}^n \omega_i^2\biggr)
}_{\displaystyle=1}
\frac{\partial^2 v}{\partial s^2}(\omega\cdot x,\omega)
\,d\omega
\notag
\\
&=
\mathscr{R}^*
\frac{\partial^2}{\partial s^2} v
\qquad\Rightarrow\qquad
\Delta\mathscr{R}^* = \mathscr{R}^*\frac{\partial^2}{\partial s^2}
\label{buch:radon:rueckprojektion:eqn:laplacedual}
\end{align}
In Satz
\ref{buch:radon:ableitungen:satz:laplace}
wurde gezeigt, wie der Laplace-Operator mit der Radon-Transformation
vertauscht.
Damit kann jetzt aus der
Formel~\ref{buch:radon:rueckprojektion:eqn:laplacedual}
die Identität
\begin{equation}
\Delta\mathscr{R}^*\mathscr{R}u
=
\mathscr{R}^*\frac{\partial^2}{\partial s^2}\mathscr{R}u
=
\mathscr{R}^*\mathscr{R}\Delta u
\end{equation}
gewinnen.

%
% Rückprojektion und die Umkehrformel
%
\subsection{Rückprojektion und die Umkehrformel}
In Abschnitt~\ref{XXX}
haben wir gefunden, dass die Fourier-Transformierte der Funktion $u$
in die Radon-Transformation und eine eindimensionale Fourier-Transformation
zerlegt werden kann.
Dies wird durch die Formel
\[
\mathscr{F}u(k)
=
\frac{1}{(2\pi)^{n/2}}
\int_{\mathbb{R}} 
e^{-i|k|s}
\mathscr{R}u(s,k^0)\,ds
\]
Die Fourier-Umkehrformel ermöglicht, die Funktion aus $\mathscr{F}u$ 
wieder zu berechnen, sie ist
\begin{align*}
u(x)
&=
\frac{1}{(2\pi)^{n/2}}
\int_{\mathbb{R}^n} e^{ik\cdot x}
\mathscr{F}u(k)\,dk.
\intertext{Das Integral über $\mathbb{R}$ kann wieder in den
``Radon-Koordinaten''
$(r,\omega)$ berechnet werden mit $r=|k|$ und $\omega = k^0$.
}
&=
\frac{1}{(2\pi)^{n/2}}
\int_0^\infty
\int_{S^{n-1}} e^{i(r\omega)\cdot x}
\mathscr{F}u(r\omega)
\,d\omega
r^{n-1}
\,dr.
\intertext{Der Faktor $r^{n-1}$ kommt von der Koordinatentransformation.
Das Integral über die Kugeloberfläche $S^{n-1}$ kann in zwei Integrale
über die Halbkugeln
$S_+^{n-1}=\{x\in S^{n-1}\mid x_1\ge 0\}$
und
$S_-^{n-1}=\{-\omega\mid \omega\in S_+^{n-1}\}$
zerlegt werden, die durch die Spiegelung
$S^{n-1}\to S^{n-1}:\omega\mapsto-\omega$ ineinander übergeführt werden.
So erhält man}
&=
\frac{1}{(2\pi)^{n/2}} \int_0^\infty \int_{S_+^{n-1}}
e^{i(r\omega)\cdot x} \mathscr{F}u(r\omega)\,d\omega r^{n-1}\,dr
\\
&\quad+
\frac{1}{(2\pi)^{n/2}} \int_0^\infty \int_{S_+^{n-1}}
e^{i(-r\omega)\cdot x} \mathscr{F}u(-r\omega)\,d(-\omega) r^{n-1}\,dr
\intertext{Wir möchten das zweite $r$-Integral in ein Integral über
$(-\infty,0)$ verwandeln.
Dazu müssen wir $-r$ an allen Stellen haben.
Wir können gleichzeitig noch das Minuszeichen in $d(-\omega)$ entfernen,
denn die Spiegelung $\omega\mapsto -\omega$ hat die Determinante
$(-1)^{n-1}$. 
Damit wird
}
u(x)
&=
\frac{1}{(2\pi)^{n/2}} \int_0^\infty \int_{S_+^{n-1}}
e^{i(r\omega)\cdot x} \mathscr{F}u(r\omega)\,d\omega r^{n-1}\,dr
\\
&\quad+
\frac{1}{(2\pi)^{n/2}} \int_0^\infty \int_{S_+^{n-1}}
e^{i(-r\omega)\cdot x} \mathscr{F}u(-r\omega)\,(-1)^{n-1}d\omega
(-r)^{n-1}(-1)^{n-1}\,d(-r)
\\
&=
\frac{1}{(2\pi)^{n/2}} \int_0^\infty \int_{S_+^{n-1}}
e^{i(r\omega)\cdot x} \mathscr{F}u(r\omega)\,d\omega r^{n-1}\,dr
\\
&\quad+
(-1)^{n-1}
\frac{1}{(2\pi)^{n/2}} \int_{-\infty}^0 \int_{S_+^{n-1}}
e^{ir\omega\cdot x} \mathscr{F}u(r\omega)
d\omega
|r|^{n-1}\,dr
\intertext{Jetzt kann man die beiden Integrale in einziges
$r$-Integral über $\mathbb{R}$ zusammenfassen 
}
u(x)
&=
\frac{1}{(2\pi)^{n/2}}
\int_{\mathbb{R}} int_{S_+^{n-1}}
e^{ir\omega\cdot x}
\mathscr{F}u(r\omega)
\,d\omega
|r|^{n-1}\,dr
\intertext{Durch Vertauschung der Integrationsreihenfolge
entsteht als inneres Integral die Fourier-Inversionsformel
für eine eindimensionale Fourier-Transformierte:}
u(x)
&=
\frac{1}{(2\pi)^{n/2}}
\int_{S_+^{n-1}}
\int_{\mathbb{R}}
e^{ir\omega\cdot x}
\mathscr{F}(r\omega)
|r|^{n-1}\,dr
\,
d\omega
\end{align*}
Das äussere Integral ist die Rückprojektion $\mathscr{R}^*$.
Wir wissen bereits, dass die Fourier-Transformation $\mathscr{F}$
zerlegt werden kann in die Radon-Transformation und die
eindimensionale Fourier-Transformation, die wir früher mit
$\mathscr{F}_r$ bezeichnet haben.
Damit ergibt sich der folgende Satz:

\begin{satz}[Gefilterte Rückprojektion]
Für eine integrierbare Funktion $u\colon\mathbb{R}^n\to\mathbb{C}$ mit kompaktem Träger gilt
\[
u
=
\frac{1}{(2\pi)^{n-1}}
\mathscr{R}^*
\mathscr{F}_r^{-1}
|r|^{n-1}
\mathscr{F}_r
\mathscr{R}u,
\]
Wobei der Term $|r|^{n-1}$ der Multiplikationsoperator mit der
Funktion $r\mapsto |r|^{n-1}$ ist.
\end{satz}

Dieser Satz zeigt eine neue Möglichkeit, die Radon-Transformation
zu invertieren.
Dazu muss die $s$-Abhängigkeit der Radon-Transformierten
$\mathscr{R}u(r,\omega)$ 
Zunächst mit $\mathscr{F}_r$ in den Frequenzbereich transformiert werden.
Im Frequenzbereich wird mit $|r|^{n-1}$ multipliziert, dadurch
werden die hohen Frequenzen verstärkt.
Schliesslich wird mit $\mathscr{F}_r^{-1}$ in den $s$-Bereich
zurücktransformiert.
Die nachfolgende duale Transformation mit $\mathscr{R}^*$ entsteht
das ursprüngliche Bild.
Die mittleren drei Schritte $\mathscr{F}_r^{-1}|r|^{n-1}\mathscr{F}_r$
entsprechen einer frequenzabhängigen Filterung.
Der Satz besagt also, dass die Radon-Transformierte zunächst
gefiltert werden muss um anschliessend mit der Rückprojektion
$\mathscr{R}^*$ die ursprüngliche Funktion zurückgibt.
Die Zusammensetzung heisst aus diesem Grund die {\em gefilterte
Rückprojektion}.

Der Satz macht auch verständlich, warum die Rekunstruktion nicht
unbedingt stabil ist. 
Der Faktor $|r|^{n-1}$ verstärkt Rauschen umso mehr, je höher
die Frequenz ist.
Da weisses Rauschen über das ganze Spektrum gleiche Leistungsdichte
hat (zur Thematik Rauschen und harmonische Analysis siehe auch 
Kapitel~\ref{chapter:brown}),
wird das Resultat vom Rauschen dominiert.
Rauschen ist aber in praktischen Anwendungen unvermeidlich, es
entsteht durch Messfehler aber auch auch durch die unausweichliche
Diskretisation.

Die gefilterte Rückprojektion zeigt aber auch einen Ausweg und 
damit eine praktisch realisierbare Möglichkeit, die Radon-Transformation
zu invertieren.
Dazu wird der Filter $r\mapsto |r|^{n-1}$ für grosse $r$ abgeschnitten,
die Funktion wird also durch eine Funktion ersetzt, die sehr
hohe Frequenzen nicht weiter verstärkt.
Damit verliert man zwar Bildschärfe, bekommt aber das Rauschen unter
Kontrolle.

In Kapitel~\ref{chapter:ct} werden Abbildungen der gefilterten
Rückprojektion gezeigt.



%\section*{Übungsaufgaben}
%\rhead{Übungsaufgaben}
%\aufgabetoplevel{chapters/010-potenzen/uebungsaufgaben}
%\begin{uebungsaufgaben}
%\uebungsaufgabe{101}
%\uebungsaufgabe{102}
%\uebungsaufgabe{103}
%\uebungsaufgabe{104}
%\end{uebungsaufgaben}

