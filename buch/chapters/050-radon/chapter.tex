%
% chapter.tex -- Radon-Transformation
%
% (c) 2021 Prof Dr Andreas Müller, Hochschule Rapperswil
%
% !TeX spellcheck = de_CH
\chapter{Radon-Transformation
\label{buch:chapter:radon}}
\kopflinks{Radon-Transformation}
Die Fourier-Transformation für Funktionen auf $\mathbb{R}^n$ findet
Wellenkomponenten, die in verschiedene Richtungen laufen. 
Ist ist daher naheliegend, eine Zerlegung des Fourier-Integrals in
ein Integral parallel zu den Wellenfronten und ein Integral in 
Ausbreitungsrichtung zu zerlegen.
Die Radon-Transformation erreicht genau dies und ermöglicht so
ein geometrisches Verständnis für eine ein Wellenfeld.
Ausserdem macht Sie Anwendungen wie den CAT-Scanner oder 
MRI möglich.

%
% 1-definition.tex
%
% (c) 2023 Prof Dr Andreas Müller, OST Ostschweizer Fachhochschule
%
\section{Definition
\label{buch:opertoren:section:definition}}
\kopfrechts{Definition}


%
% 2-ableitungen.tex
%
% (c) 2023 Prof Dr Andreas Müller
%
\section{Radon-Transformation und Ableitungen
\label{buch:radon:section:ableitungen}}
Die Fourier-Transformation verwandelt Ableitungen in eine Multiplikation.
Die Radon-Trans\-for\-ma\-tion führt einen Teil der Fourier-Transformation
durch, es ist also zu erwarten, dass die \RT{} ebenfalls
sehr spezielle Ableitungseigenschaften hat.
Es wird sich zeigen, dass die \RT{} $\mathscr{R}u(s,\omega)$
einer Lösung der Wellengleichung ihrerseits Lösung einer
eindimensionalen Wellengleichung entlang der Richtung $\omega$ ist,
dies ist ein mathematischer Ausdruck für das huygenssche Prinzip.

%
% Partielle Ableitungen
%
\subsection{Partielle Ableitungen
\label{buch:radon:ableitungen:subsection:partiell}}
Sei $u\colon \mathbb{R}^n\to\mathbb{C}$ eine Funktion mit kompaktem
Träger und sei $\omega = e_1 = (1,0,\dots,0)$ der erste Standarbasisvektor.
Da die zugehörige Hyperebene konstante erste Koordinaten hat, ist
die Radon-Transformierte von $u$ gegeben durch
\[
\mathscr{R}u(s,\omega)
\int_{\mathbb{R}^{n-1}}
u(s,x_2,\dots,x_n)\,dx_2\dots\,dx_n.
\]
Die Radon-Transformierte der Ableitung nach $x_1$ ist dann
\begin{align}
\biggl(\mathscr{R}\frac{\partial u}{\partial x_1}\biggr)(s,\omega)
&=
\int_{\mathbb{R}^{n-1}}
\frac{\partial u}{\partial s}(s,x_2,\dots,x_n)
\,dx_2\dots\,dx_n
\notag
\\
&=
\frac{\partial}{\partial s}
\int_{\mathbb{R}^{n-1}}
u(s,x_2,\dots,x_n)
\,dx_2\dots\,dx_n
=
\frac{\partial}{\partial s}\mathscr{R}u(s,\omega).
\notag
\intertext{Durch Iteration findet man auch die Radon-Transformierte
der $m$-ten Ableitung}
\biggl(
\mathscr{R}\frac{\partial^mu}{\partial x_1^m}
\biggr)(s,\omega)
&=
\frac{\partial^m}{\partial s^m}
\mathscr{R}u(s,\omega).
\notag
\intertext{Die Radon-Transformierten einer Ableitung nach einer
anderen Koordinate $x_k$ mit $k>1$
ist dagegen}
\biggl(
\mathscr{R}
\frac{\partial u}{\partial x_k}
\biggr)(s,\omega)
&=
\int_{\mathbb{R}^{n-1}}
\frac{\partial u}{\partial x_k}(s,x_2,\dots,x_k)
\,dx_2\dots\,dx_n
\notag
\\
&=
\int_{\mathbb{R}^{n-2}}
\biggl(
\int_{\mathbb{R}}
\frac{\partial u}{\partial x_k}(s,x_2,\dots,x_n)
\,dx_k
\biggr)
\,dx_2\dots\widehat{dx_k}\dots dx_n
\notag
\\
&=
\int_{\mathbb{R}^{n-2}}
\underbrace{
u(s,x_2,\dots,\infty,\dots,x_n)
-
u(s,x_2,\dots,-\infty,\dots,x_n)
}_{\displaystyle=0}
\,dx_2\dots\widehat{dx_k}\dots dx_n
\\
&=0.
\label{buch:gruppen:radon:ableitungen:eqn:querrichtung}
\end{align}
Im zweitletzten Schritt wurde verwendet, dass $u$ kompakten Träger
hat, also ausserhalb einer beschränkten Menge verschwindet.

%
% Richtungsableitung
%
\subsection{Richtungsableitung
\label{buch:radon:ableitungen:subsection:richtungsableitung}}
Sei jetzt $\omega\in S^{n-1}$ ein beliebiger Einheitsvektor.
Es gibt immer eine Drehmatrix $R$, mit der man den ersten
Standardbasisvektor auf die Richtung $\omega$ drehen kann.
Die Ableitung der zusammengesetzten Funktion $u\circ R$ nach
der ersten Koordinate ist die Richtungsableitung der Funktion
$u$ in Richtung $\omega$.
Die \RT der Richtungsableitung der Funktion $u$ in
Richtung $\omega$ ist die Ableitung nach dem Parameter $s$ der
\RT{}n $\mathscr{R}u(s,\omega)$.
Umkgekehrt verschwindet
nach~\ref{buch:gruppen:radon:ableitungen:eqn:querrichtung}
die \RT der Richtungsableitung
von $u$ in eine Richtung senkrecht auf $\omega$ im Punkt $\omega$.
Damit haben wir folgenden Satz gefunden:

\begin{satz}[Radon-Transformation und Richtungsableitung]
Sei $u\colon\mathbb{R}^n\to\mathbb{C}$ eine integrierbare Funktion
mit kompaktem Träger:
Sei weiter $b\perp \omega$ ein zu $\omega$ orthogonaler Einheitsvektor.
Dann sind die Radon-Transformierten der Richtungsableitungen
\begin{align*}
(\mathscr{R}D_\omega u)(s,\omega)
&=
\frac{\partial}{\partial s}\mathscr{R}u(s,\omega)
\\
\text{und}\qquad
(\mathscr{R}D_b u)(s,\omega)
&=
0.
\end{align*}
\end{satz}

Da man die Richtungsableitung in Richtung des Vektors $v$ mit den
Komponenten $v_i$ auch als
\[
D_v 
=
\sum_{i=1}^n v_i \frac{\partial}{\partial x_i}
\]
schreiben kann, finden wir die Komponentenformeln
\begin{align}
\biggl(
\mathscr{R}
\sum_{i=1}^n
\omega_i\frac{\partial u}{\partial x_i}
\biggr)(s,\omega)
&=
\sum_{i=1}^n
\omega_i
\biggl(\mathscr{R}\frac{\partial u}{\partial x_i}\biggr)
(s,\omega)
=
\frac{\partial}{\partial s}\mathscr{R}u(s,\omega)
\notag
\\
\text{und}\qquad
\sum_{i=1}^n
b_i
\biggl(
\mathscr{R}\frac{\partial u}{\partial x_i}
\biggr)(s,\omega)
&=0.
\label{buch:radon:ableitungen:eqn:babl}
\end{align}
Die Bedingung \eqref{buch:radon:ableitungen:eqn:babl}
bedeutet, dass der Vektor $v$ mit den Komponenten
\[
v_i
=
\biggl(\mathscr{R}\frac{\partial u}{\partial x_i}\biggr)(s,\omega)
\]
orthogonal ist zu allen Vektoren $b\perp \omega$, er hat daher
die Richtung $\omega$.
Da $|\omega|=1$ ist, ist die Länge $|v|=v\cdot \omega$ die Projektion
von $v$ auf die Richtung $\omega$ oder
\[
|v|
=
v\cdot\omega
=
\sum_{i=1}^n \omega_i
\biggl(
\mathscr{R}\frac{\partial u}{\partial x_i}
\biggr)(s,\omega)
=
\frac{\partial}{\partial s}\mathscr{R}u(s,\omega).
\]
Die einzelnen Komponenten sind daher
\[
v_i
=
|v|\omega_i
=
\omega_i
\frac{\partial}{\partial s}\mathscr{R}(s,\omega).
\]

\begin{satz}[Radon-Transformation und Ableitung]
Ist $u\colon\mathbb{R}^n\to\mathbb{C}$ eine ingegrierbare Funktion
mit kompaktem Träger, dann ist die Ra\-don-Trans\-for\-ma\-tion der
Ableitungen von $u$
\[
\biggl(
\mathscr{R}
\frac{
\partial^{|\bm{\alpha}|}
}{
\partial x_1^{\alpha_1}\cdots\partial x_n^{\alpha_n}
}
u
\biggr)(s,\omega)
=
(\mathscr{R}
D^{\bm{\alpha}} u
)(s,\omega)
=
\omega_1^{\alpha_1} \cdots \omega_n^{\alpha_n}
\frac{
\partial^{|\bm{\alpha}|}
}{
\partial s^{|\bm{\alpha}|}
}
\mathscr{R}u(s,\omega).
\]
\end{satz}

%
% Laplace-Operator
%
\subsection{Laplace-Operator
\label{buch:radon:ableitungen:subsection:laplace}}
Die Radon-Transformation $\mathscr{R}u$ zerlegt die Information
in der Funktion $u$ in Information, die von der Richung $\omega$
abhängt und solche, die vom Radius $s$ abhängt.
Ableitungen hängen dann nur noch von $s$ ab, was sich auch
auf den Laplace-Operator auswirkt.

\begin{satz}[Radon-Transformation und Laplace-Operator]
\label{buch:radon:ableitungen:satz:laplace}
Für eine zweimal stetig differenzierbare Funktion
$u\colon\mathbb{R}^n\to\mathbb{C}$ mit kompaktem Träger gilt
\[
(\mathscr{R}\Delta u)(s,\omega)
=
\frac{\partial^2}{\partial s^2} \mathscr{R}u(s,\omega).
\]
\end{satz}

\begin{proof}[Beweis]
Der Laplace-Operator ist
\[
\Delta
=
\sum_{i=1}^n
\frac{\partial^2}{\partial x_i^2}.
\]
Zusammen mit der Ableitungsformel folgt
\begin{align*}
(\mathscr{R}\Delta u)(s,\omega)
&=
\sum_{i=1}^n
\biggl(\mathscr{R}\frac{\partial^2}{\partial x_i^2}u\biggr)(s,\omega)
\\
&=
\sum_{i=1}^n \omega_i^2 \frac{\partial^2}{\partial s^2}\mathscr{R}u(s,\omega)
\\
&=
|\omega|^2 \frac{\partial^2}{\partial s^2} \mathscr{R}u(s,\omega)
=
\frac{\partial^2}{\partial s^2} \mathscr{R}u(s,\omega).
\qedhere
\end{align*}
\end{proof}

Die Radon-Transformation transformiert den Laplace-Operator
auf einen zweiten Ableitungsoperator nach nur einer einzigen
Variablen.

%
% Wellengleichung
%
\subsection{Wellengleichung
\label{buch:radon:ableitungen:subsection:wellengleichung}}
Der Satz~\ref{buch:radon:ableitungen:satz:laplace} über die
Radon-Transformierte des Laplace-Operators ermöglicht, auch die
Lösung der Wellengleichung mit Hilfe der Radon-Transformation zu
vereinfachen.
Die Wellengleichung ist die Gleichung
\[
\frac{\partial^2}{\partial t^2}u = \Delta u
\]
auf dem Gebiet $\Omega=\mathbb{R}^n$.
Nach Anwendung der Radon-Transformation wird aus den beiden Seiten
\begin{align*}
\biggl(\mathscr{R}\frac{\partial^2}{\partial t^2}u\biggr)(s,\omega)
&=
\frac{\partial^2}{\partial t^2}(\mathscr{R}u)(s,\omega)
\\
(\mathscr{R}\Delta u)(s,\omega)
&=
\frac{\partial^2}{\partial s^2}(\mathscr{R}u)(s,\omega)
\end{align*}
Schreibt man $U(t,s\omega)=\mathscr{R}u(s,\omega)$, dann folgt
\[
\frac{\partial^2 U}{\partial t^2}
=
\frac{\partial^2 U}{\partial s^2},
\]
also eine eindimensionale Wellengleichung.
Durch Mittelung über Hyperebenen orthogonal zu $\omega$ macht
die Radon-Transformation aus der $n$-dimensionalen Wellengleichung
eine eindimensionale Wellengleichung für die Ausbreitung des
Mittelwertes $U(t,s,\omega)$ in Richtung $\omega$.
Dieses Prinzip wird manchmal auch als das Huygens-Prinzip 
bezeichnet, da es verwandt ist mit der anschaulichen Vorstellung
von Huygens, wie sich Wellenfronten aus Elementarwellen zusammensetzen.
Letztere wird in Kapitel~\ref{chapter:opt} genauer studiert.



%
% 2-rueckprojektion.tex
%
% (c) 2022 Prof Dr Andraes Müller, OST Ostschweizer Fachhochschule
%
\section{Rückprojektion
\label{buch:radon:section:rueckprojektion}}
\kopfrechts{Rückprojektion}
Sei $u\colon \mathbb{R}^n\to\mathbb{C}$ eine Funktion und
$\mathscr{R}u\colon \mathbb{R}\times S^{n-1}\to\mathbb{C}$
die Radon-Transformierte.
Ändert man die Funktion $u$ in einer kleinen Umgebung eines Punkts $y$,
dann ändert die Radon-Transformiert nur für diejenigen Hyperbenen,
die in der Nähe von $y$ vorbeigehen.
Daraus kann man schliessen, dass genau die Werte
$\mathscr{R}(\omega\cdot y,\omega)$ Information über den Wert der
Funktion $u$ im Punkt $y$ enthalten.
Sie enthalten aber auch noch Information über alle anderen Punkte.

\begin{definition}
Die {\em Rückprojektion} $\mathscr{R}^*f$ einer Funktion
$f\colon \mathbb{R}\times S^{n-1}\to\mathbb{C}$ ist die Funktion
\[
(\mathscr{R}^*f)(y)
=
\int_{S_+^{n-1}} f(\omega\cdot y,\omega)\,d\omega.
\]
$\mathscr{R}^*$ heisst auch die {\em duale Transformation}.
\end{definition}

Die Rückprojektion ist also der Mittelwert der Werte von $f$ über
alle Richtungen, aber in der Entfernung von $y$ vom Nullpunkt.
Wenn $f$ die Radon-Transformierte $f=\mathscr{R}u$ einer Funktion $u$
ist, dann ist die Rückprojektion $(\mathscr{R}^*\mathscr{R}u)(y)$
der Mittelwert aller Werte von $\mathscr{R}u$, die durch Integration
entlang einer Hyperbene durch $y$ entstanden sind.
In der Rückprojektion sind also alle Werte von $u$ gemittelt, aber
der Wert im Punkt $y$ hat besonders grosses Gewicht.
Die duale Transformation ist also eine erste Approximation für $u$.
In den weiteren Entwicklungen hoffen wir, daraus eine exakte Rekonstruktion
von $u$ zu konstruieren.

%
% Rückprojektion und Ableitungen
%
\subsection{Rückprojektion und Ableitungen}
In diesem Abschnitt ist $v\colon \mathbb{R}\times S^{n-1}\to\mathbb{C}$
eine differenzierbare Funktion mit kompaktem Träger, so dass alle
im Folgenden betrachteten Integrale und Ableitungen existierten.
Aus der Definition
\begin{align*}
\mathscr{R}^*v(x)
&=
\int_{S_+^{n-1}} v(\omega\cdot x,\omega)\,d\omega
\intertext{kann man jetzt auch die Ableitung nach $x_i$ berechnen und
bekommt}
\frac{\partial}{\partial x_i}\mathscr{R}^*v(x)
&=
\int_{S_+^{n-1}}
\frac{\partial}{\partial x_i} v(\omega\cdot x,\omega)\,d\omega
\\
&=
\int_{S_+^{n-1}}
\omega_i
\frac{\partial v}{\partial s}(\omega\cdot x,\omega)
\,d\omega
\end{align*}
Für den Laplace-Operator findet man dann
\begin{align}
\Delta\mathscr{R}^*v
&=
\int_{S_+^{n-1}}
\underbrace{
\biggl(\sum_{i=1}^n \omega_i^2\biggr)
}_{\displaystyle=1}
\frac{\partial^2 v}{\partial s^2}(\omega\cdot x,\omega)
\,d\omega
\notag
\\
&=
\mathscr{R}^*
\frac{\partial^2}{\partial s^2} v
\qquad\Rightarrow\qquad
\Delta\mathscr{R}^* = \mathscr{R}^*\frac{\partial^2}{\partial s^2}
\label{buch:radon:rueckprojektion:eqn:laplacedual}
\end{align}
In Satz
\ref{buch:radon:ableitungen:satz:laplace}
wurde gezeigt, wie der Laplace-Operator mit der Radon-Transformation
vertauscht.
Damit kann jetzt aus der
Formel~\ref{buch:radon:rueckprojektion:eqn:laplacedual}
die Identität
\begin{equation}
\Delta\mathscr{R}^*\mathscr{R}u
=
\mathscr{R}^*\frac{\partial^2}{\partial s^2}\mathscr{R}u
=
\mathscr{R}^*\mathscr{R}\Delta u
\end{equation}
gewinnen.

%
% Rückprojektion und die Umkehrformel
%
\subsection{Rückprojektion und die Umkehrformel}
In Abschnitt~\ref{XXX}
haben wir gefunden, dass die Fourier-Transformierte der Funktion $u$
in die Radon-Transformation und eine eindimensionale Fourier-Transformation
zerlegt werden kann.
Dies wird durch die Formel
\[
\mathscr{F}u(k)
=
\frac{1}{(2\pi)^{n/2}}
\int_{\mathbb{R}} 
e^{-i|k|s}
\mathscr{R}u(s,k^0)\,ds
\]
Die Fourier-Umkehrformel ermöglicht, die Funktion aus $\mathscr{F}u$ 
wieder zu berechnen, sie ist
\begin{align*}
u(x)
&=
\frac{1}{(2\pi)^{n/2}}
\int_{\mathbb{R}^n} e^{ik\cdot x}
\mathscr{F}u(k)\,dk.
\intertext{Das Integral über $\mathbb{R}$ kann wieder in den
``Radon-Koordinaten''
$(r,\omega)$ berechnet werden mit $r=|k|$ und $\omega = k^0$.
}
&=
\frac{1}{(2\pi)^{n/2}}
\int_0^\infty
\int_{S^{n-1}} e^{i(r\omega)\cdot x}
\mathscr{F}u(r\omega)
\,d\omega
r^{n-1}
\,dr.
\intertext{Der Faktor $r^{n-1}$ kommt von der Koordinatentransformation.
Das Integral über die Kugeloberfläche $S^{n-1}$ kann in zwei Integrale
über die Halbkugeln
$S_+^{n-1}=\{x\in S^{n-1}\mid x_1\ge 0\}$
und
$S_-^{n-1}=\{-\omega\mid \omega\in S_+^{n-1}\}$
zerlegt werden, die durch die Spiegelung
$S^{n-1}\to S^{n-1}:\omega\mapsto-\omega$ ineinander übergeführt werden.
So erhält man}
&=
\frac{1}{(2\pi)^{n/2}} \int_0^\infty \int_{S_+^{n-1}}
e^{i(r\omega)\cdot x} \mathscr{F}u(r\omega)\,d\omega r^{n-1}\,dr
\\
&\quad+
\frac{1}{(2\pi)^{n/2}} \int_0^\infty \int_{S_+^{n-1}}
e^{i(-r\omega)\cdot x} \mathscr{F}u(-r\omega)\,d(-\omega) r^{n-1}\,dr
\intertext{Wir möchten das zweite $r$-Integral in ein Integral über
$(-\infty,0)$ verwandeln.
Dazu müssen wir $-r$ an allen Stellen haben.
Wir können gleichzeitig noch das Minuszeichen in $d(-\omega)$ entfernen,
denn die Spiegelung $\omega\mapsto -\omega$ hat die Determinante
$(-1)^{n-1}$. 
Damit wird
}
u(x)
&=
\frac{1}{(2\pi)^{n/2}} \int_0^\infty \int_{S_+^{n-1}}
e^{i(r\omega)\cdot x} \mathscr{F}u(r\omega)\,d\omega r^{n-1}\,dr
\\
&\quad+
\frac{1}{(2\pi)^{n/2}} \int_0^\infty \int_{S_+^{n-1}}
e^{i(-r\omega)\cdot x} \mathscr{F}u(-r\omega)\,(-1)^{n-1}d\omega
(-r)^{n-1}(-1)^{n-1}\,d(-r)
\\
&=
\frac{1}{(2\pi)^{n/2}} \int_0^\infty \int_{S_+^{n-1}}
e^{i(r\omega)\cdot x} \mathscr{F}u(r\omega)\,d\omega r^{n-1}\,dr
\\
&\quad+
(-1)^{n-1}
\frac{1}{(2\pi)^{n/2}} \int_{-\infty}^0 \int_{S_+^{n-1}}
e^{ir\omega\cdot x} \mathscr{F}u(r\omega)
d\omega
|r|^{n-1}\,dr
\intertext{Jetzt kann man die beiden Integrale in einziges
$r$-Integral über $\mathbb{R}$ zusammenfassen 
}
u(x)
&=
\frac{1}{(2\pi)^{n/2}}
\int_{\mathbb{R}} int_{S_+^{n-1}}
e^{ir\omega\cdot x}
\mathscr{F}u(r\omega)
\,d\omega
|r|^{n-1}\,dr
\intertext{Durch Vertauschung der Integrationsreihenfolge
entsteht als inneres Integral die Fourier-Inversionsformel
für eine eindimensionale Fourier-Transformierte:}
u(x)
&=
\frac{1}{(2\pi)^{n/2}}
\int_{S_+^{n-1}}
\int_{\mathbb{R}}
e^{ir\omega\cdot x}
\mathscr{F}(r\omega)
|r|^{n-1}\,dr
\,
d\omega
\end{align*}
Das äussere Integral ist die Rückprojektion $\mathscr{R}^*$.
Wir wissen bereits, dass die Fourier-Transformation $\mathscr{F}$
zerlegt werden kann in die Radon-Transformation und die
eindimensionale Fourier-Transformation, die wir früher mit
$\mathscr{F}_r$ bezeichnet haben.
Damit ergibt sich der folgende Satz:

\begin{satz}[Gefilterte Rückprojektion]
Für eine integrierbare Funktion $u\colon\mathbb{R}^n\to\mathbb{C}$ mit kompaktem Träger gilt
\[
u
=
\frac{1}{(2\pi)^{n-1}}
\mathscr{R}^*
\mathscr{F}_r^{-1}
|r|^{n-1}
\mathscr{F}_r
\mathscr{R}u,
\]
Wobei der Term $|r|^{n-1}$ der Multiplikationsoperator mit der
Funktion $r\mapsto |r|^{n-1}$ ist.
\end{satz}

Dieser Satz zeigt eine neue Möglichkeit, die Radon-Transformation
zu invertieren.
Dazu muss die $s$-Abhängigkeit der Radon-Transformierten
$\mathscr{R}u(r,\omega)$ 
Zunächst mit $\mathscr{F}_r$ in den Frequenzbereich transformiert werden.
Im Frequenzbereich wird mit $|r|^{n-1}$ multipliziert, dadurch
werden die hohen Frequenzen verstärkt.
Schliesslich wird mit $\mathscr{F}_r^{-1}$ in den $s$-Bereich
zurücktransformiert.
Die nachfolgende duale Transformation mit $\mathscr{R}^*$ entsteht
das ursprüngliche Bild.
Die mittleren drei Schritte $\mathscr{F}_r^{-1}|r|^{n-1}\mathscr{F}_r$
entsprechen einer frequenzabhängigen Filterung.
Der Satz besagt also, dass die Radon-Transformierte zunächst
gefiltert werden muss um anschliessend mit der Rückprojektion
$\mathscr{R}^*$ die ursprüngliche Funktion zurückgibt.
Die Zusammensetzung heisst aus diesem Grund die {\em gefilterte
Rückprojektion}.

Der Satz macht auch verständlich, warum die Rekunstruktion nicht
unbedingt stabil ist. 
Der Faktor $|r|^{n-1}$ verstärkt Rauschen umso mehr, je höher
die Frequenz ist.
Da weisses Rauschen über das ganze Spektrum gleiche Leistungsdichte
hat (zur Thematik Rauschen und harmonische Analysis siehe auch 
Kapitel~\ref{chapter:brown}),
wird das Resultat vom Rauschen dominiert.
Rauschen ist aber in praktischen Anwendungen unvermeidlich, es
entsteht durch Messfehler aber auch auch durch die unausweichliche
Diskretisation.

Die gefilterte Rückprojektion zeigt aber auch einen Ausweg und 
damit eine praktisch realisierbare Möglichkeit, die Radon-Transformation
zu invertieren.
Dazu wird der Filter $r\mapsto |r|^{n-1}$ für grosse $r$ abgeschnitten,
die Funktion wird also durch eine Funktion ersetzt, die sehr
hohe Frequenzen nicht weiter verstärkt.
Damit verliert man zwar Bildschärfe, bekommt aber das Rauschen unter
Kontrolle.

In Kapitel~\ref{chapter:ct} werden Abbildungen der gefilterten
Rückprojektion gezeigt.

%
% Gefilterte Rückprojektion
%
\subsection{Gefilterte Rückprojektion und Hilbert-Transformation}
Das Stabilitätsproblem bei der Invertierung der Radon-Transformation
kann noch etwas deutlicher gemacht werden, in dem man die gefilterte
Rücktransformation durch die Hilbert-Transformation ausdrückt.
Dieses wird mit einem normalerweise divergenten Integral definiert,
das als der sogenannte Hauptwert interpretiert werden muss.

\begin{definition}
Ist $f\colon[a,b]\setminus\{c\}\to\mathbb{R}$ eine stetige Funktion,
dann heisst
\[
\operatorname{PV}
\int_a^b f(x)\,dx
=
\lim_{\varepsilon\to 0+}
\biggl(
\int_a^{b-\varepsilon} f(x)\,dx
+
\int_{b+\varepsilon}^c f(x)\,dx
\biggr)
\]
der {\em Hauptwert} oder {\em principal value} des Integrals.
\end{definition}

Der Hauptwert gibt gewissen uneigentlichen Integralen einen Sinn, 
die mit der konventionellen Definition keinen vernünftigen Wert haben.

\begin{beispiel}
Die Funktion $f\colon x\mapsto 1/x$ ist an der Stelle $x=0$ nicht definiert und
das Integral über das Interval $[-1,1]$ divergiert, wie die folgende Rechnung
zeigt.
Das Integral muss an der Stelle $x=0$ aufgeteilt werden:
\begin{equation}
\int_{-1}^1 f(x)\,dx
=
\lim_{\varepsilon\to 0-}
\int_{-1}^{\varepsilon} \frac{dx}{x}
+
\lim_{\epsilon\to 0+}
\int_{\epsilon}^{1} \frac{dx}{x}.
\label{buch:radon:rueckprojektion:bsp:1x:eqn1}
\end{equation}
Das erste Integral kann mit der Substituion $t=-s$ und $dt=-ds$ vereinfacht
werden zu
\begin{align*}
\int_{-1}^{\varepsilon} \frac{dx}{x}
&=
\int_{1}^{-\varepsilon} \frac{ds}{s}
=
-
\int_{-\varepsilon}^1 \frac{ds}{s}
=
-
\biggl[
\log s
\biggr]_{-\varepsilon}^1
=
-
(
\log 1 - \log(-\varepsilon)
)
=
\log(-\varepsilon).
\end{align*}
Eingesetzt in 
\eqref{buch:radon:rueckprojektion:bsp:1x:eqn1}
erhält man
\[
\int_{-1}^1 f(x)\,dx
=
\lim_{\varepsilon\to 0-}
\log(-\varepsilon)
+
\lim_{\epsilon\to 0+}
(-\log(\epsilon))
=
\underbrace{
\lim_{\varepsilon\to 0+}
\log\varepsilon
}_{\displaystyle\to \infty}
-
\underbrace{
\lim_{\epsilon\to 0+}
\log(\epsilon)
}_{\displaystyle\to \infty}
=
\infty - \infty.
\]
Da nach der konventionellen Definition eines uneigentlichen
Integrals die beiden Grenzwerte unabhängig voneinander genommen werden,
kann dem Integral kein sinnvoller Wert zugewiesen werden.
Dies ändert sich jedoch mit dem Hauptwert, wo die beiden Grenzen
$\varepsilon=\epsilon$ gekoppelt werden.
Dann ergibt sich
\[
\operatorname{PV}
\int_{-1}^1 \frac{dx}{x}
=
\lim_{\varepsilon\to 0+}
\biggl(
\int_{-1}^{-\varepsilon}\frac{dx}{x}
+
\int_{\varepsilon}^{1} \frac{dx}{x}
\biggr)
=
\lim_{\varepsilon\to 0+}
\biggl(
\log\varepsilon - \log\varepsilon
\biggr)
=
0.
\]
Der Hauptwert des Integrals ist also wohldefiniert.
\end{beispiel}

\begin{definition}
Ist $f\colon \mathbb{R}\to\mathbb{C}$ eine stetige und integrierbare
Funktion, dann ist die {\em Hilbert-Transformierte} definiert
\[
(\mathscr{H}f)(t)
=
\frac{1}{\pi} \operatorname{PV} \int_{-\infty}^\infty \frac{f(t)}{t-x}\,dx,
\]
wobe der {\em Hauptwert} des Integral gleichbedeutend ist mit
\[
(\mathscr{H}f)(t)
=
\lim_{\varepsilon\to 0+}
\biggl(
\int_{-\infty}^{t-\varepsilon}+\int_{t+\varepsilon}^\infty
\frac{f(x)}{t-x}\,dx
\biggr).
\]
\end{definition}

Die Definition der Hilbert-Transformation sieht aus wie eine 
Faltung mit der Funktion $x\mapsto 1/x$, da diese aber nicht
integrierbar ist, ist die konventionelle Definition der Faltung
nicht anwendbar.

\begin{satz}
Die Hilbert-Transformierte $\mathscr{H}f$ einer stetigen Funktion $f$
hat die Fourier-Transformierte
\[
\mathscr{F}\mathscr{H} f
=
-i\sign(k) \mathscr{F}f,
\]
wobei $\operatorname{sign}(r)$ die Multiplikation mit der Vorzeichenfunktion
\[
\sign(k)
=
\begin{cases}
\phantom{+}1&\qquad x > 0\\
\phantom{+}0&\qquad x = 0\\
         - 1&\qquad x < 0
\end{cases}
\]
ist.
\end{satz}

Die Hilbert-Transformation ist im Frequenzbereich die Multiplikation
mit $-i\sign(k)$, während die Ableitung im Frequenzbereich zur Multiplikation
mit $-ik$ wird.
Der Filter der gefilterten Rückprojektion benötigt im Frequenzbereich 
den Faktor $|k|^{n-1}$, 
Diesen kann man wegen $|k|=\sign(k)\cdot k$ aus dem Vorzeichen und der
dem Wert zusammensetzen, d.~h.~.
Für den einfachsten Fall $n=2$ muss man die Filterung mit
\[
|r|
=
\sign(r)
\cdot r 
=
(-i\sign(r))
\cdot
(-ir)
\]
durchführen.
Die gefilterte Rückprojektionsformel ist
\begin{align*}
u
&=
\frac{1}{2\pi}
\mathscr{R}^*
\mathscr{F}_r^{-1}
|r|
\mathscr{F}_r
\mathscr{R}u
\\
&=
\frac{1}{2\pi}
\mathscr{R}^*
\mathscr{F}_r^{-1}
(-i\sign(r))
\cdot
(-ir)
\mathscr{F}_r
\mathscr{R}u
\\
&=
\frac{1}{2\pi}
\mathscr{R}^*
\mathscr{F}_r^{-1}
(-i\sign(r))
\mathscr{F}_r
\mathscr{F}_r^{-1}
(-ir)
\mathscr{F}_r
\mathscr{R}u
\\
&=
\frac{1}{2\pi}
\mathscr{R}^*
\mathscr{H}
\frac{\partial}{\partial s}
\mathscr{R}u.
\end{align*}
Auch diese Form der gefilterten Rücktransformation illustriert, dass
die Bestimmung der Funktion aus der Radon-Transformation nicht sonderlich
stabil sein kann.
Ist die Radon-Transformierte $\mathscr{R}u$ mit weissem Rauschen verrauscht,
wie das in der Praxis wegen Mess- und Quantisierungsfehlern immer der Fall
ist, dann verstärkt der Operator $\partial/\partial s$ dieses Rauschen
zusätzlich in einem Mass, dass die Rücktransformation sogar die
Hilbert-Transformation, spätestens aber die Fourier-Rücktransformation
nicht mehr definiert ist.



%\section*{Übungsaufgaben}
%\rhead{Übungsaufgaben}
%\aufgabetoplevel{chapters/010-potenzen/uebungsaufgaben}
%\begin{uebungsaufgaben}
%\uebungsaufgabe{101}
%\uebungsaufgabe{102}
%\uebungsaufgabe{103}
%\uebungsaufgabe{104}
%\end{uebungsaufgaben}

