%
% 2-ableitungen.tex
%
% (c) 2023 Prof Dr Andreas Müller
%
\section{Radon-Transformation und Ableitungen
\label{buch:radon:section:ableitungen}}
Die Fourier-Transformation verwandelt Ableitungen in eine Multiplikation.
Die Radon-Transformation führt einen Teil der Fourier-Transformation
durch, es ist also zu erwarten, dass die \RT ebenfalls
sehr spezielle Ableitungseigenschaften hat.
Es wird sich zeigen, dass die \RT $\mathscr{R}u(s,\omega)$
einer Lösung der Wellengleichung ihrerseits Lösung einer
eindimensionalen Wellengleichung entlang der Richtung $\omega$ ist,
dies ist ein mathematischer Ausdruck für das Huygenssche Prinzip.

%
% Partielle Ableitungen
%
\subsection{Partielle Ableitungen
\label{buch:radon:ableitungen:subsection:partiell}}
Sei $u\colon \mathbb{R}^n\to\mathbb{C}$ eine Funktion mit kompaktem
Träger und sei $\omega = e_1 = (1,0,\dots,0)$ der erste Standarbasisvektor
Da die zugehörige Hyperebene konstante erste Koordinaten hat, ist
die Radon-Transformierte von $u$ gegeben durch
\[
\mathscr{R}u(s,\omega)
\int_{\mathbb{R}^{n-1}}
u(s,x_2,\dots,x_n)\,dx_2\dots\,dx_n.
\]
Die Radon-Transformierte der Ableitung nach $x_1$ ist dann
\begin{align}
\biggl(\mathscr{R}\frac{\partial u}{\partial x_1}\biggr)(s,\omega)
&=
\int_{\mathbb{R}^{n-1}}
\frac{\partial u}{\partial s}(s,x_2,\dots,x_n)
\,dx_2\dots\,dx_n
\notag
\\
&=
\frac{\partial}{\partial s}
\int_{\mathbb{R}^{n-1}}
u(s,x_2,\dots,x_n)
\,dx_2\dots\,dx_n
=
\frac{\partial}{\partial s}\mathscr{R}u(s,\omega).
\notag
\intertext{Durch Iteration findet man auch die Radon-Transformierte
der $m$-ten Ableitung}
\biggl(
\mathscr{R}\frac{\partial^mu}{\partial x_1^m}
\biggr)(s,\omega)
&=
\frac{\partial^m}{\partial s^m}
\mathscr{R}u(s,\omega).
\notag
\intertext{Die Radon-Transformierten einer Ableitung nach einer
anderen Koordinate $x_k$ mit $k>1$
ist dagegen}
\biggl(
\mathscr{R}
\frac{\partial u}{\partial x_k}
\biggr)(s,\omega)
&=
\int_{\mathbb{R}^{n-1}}
\frac{\partial u}{\partial x_k}(s,x_2,\dots,x_k)
\,dx_2\dots\,dx_n
\notag
\\
&=
\int_{\mathbb{R}^{n-2}}
\biggl(
\int_{\mathbb{R}}
\frac{\partial u}{\partial x_k}(s,x_2,\dots,x_n)
\,dx_k
\biggr)
\,dx_2\dots\widehat{dx_k}\dots dx_n
\notag
\\
&=
\int_{\mathbb{R}^{n-2}}
\underbrace{
u(s,x_2,\dots,\infty,\dots,x_n)
-
u(s,x_2,\dots,-\infty,\dots,x_n)
}_{\displaystyle=0}
\,dx_2\dots\widehat{dx_k}\dots dx_n
\\
&=0.
\label{buch:gruppen:radon:ableitungen:eqn:querrichtung}
\end{align}
Im zweitletzten Schritt wurde verwendet, dass $u$ kompakten Träger
hat, also ausserhalb einer beschränkten Menge verschwindet.

%
% Richtungsableitung
%
\subsection{Richtungsableitung
\label{buch:radon:ableitungen:subsection:richtungsableitung}}
Sei jetzt $\omega\in S^{n-1}$ ein beliebiger Einheitsvektor.
Es gibt immer eine Drehmatrix $R$, mit der man den ersten
Standardbasisvektor auf die Richtung $\omega$ drehen kann.
Die Ableitung der zusammengesetzten Funktion $u\circ R$ nach
der ersten Koordinate ist die Richtungsableitung der Funktion
$u$ in Richtung $\omega$.
Die \RT der Richtungsableitung der Funktion $u$ in
Richtung $\omega$ ist die Ableitung nach dem Parameter $s$ der
\RT{}n $\mathscr{R}u(s,\omega)$.
Umkgekehrt verschwindet
nach~\ref{buch:gruppen:radon:ableitungen:eqn:querrichtung}
die \RT der Richtungsableitung
von $u$ in eine Richtung senkrecht auf $\omega$ im Punkt $\omega$.
Damit haben wir folgenden Satz gefunden:

\begin{satz}
Sei $u\colon\mathbb{R}^n\to\mathbb{C}$ eine integrierbare Funktion
mit kompaktem Träger:
Sei weiter $b\perp \omega$ ein zu $\omega$ orthogonaler Einheitsvektor.
Dann sind die Radon-Transformierten der Richtungsableitungen
\begin{align*}
(\mathscr{R}D_\omega u)(s,\omega)
&=
\frac{\partial}{\partial s}\mathscr{R}u(s,\omega)
\\
(\mathscr{R}D_b u)(s,\omega)
&=
0
\end{align*}
\end{satz}

Da man die Richtungsableitung in Richtung des Vektors $v$ mit den
Komponenten $v_i$ auch als
\[
D_v 
=
\sum_{i=1}^n v_i \frac{\partial}{\partial x_i}
\]
schreiben kann, finden wir die Komponentenformeln
\begin{align}
\biggl(
\mathscr{R}
\sum_{i=1}^n
\omega_i\frac{\partial u}{\partial x_i}
\biggr)(s,\omega)
&=
\sum_{i=1}^n
\omega_i
\biggl(\mathscr{R}\frac{\partial u}{\partial x_i}\biggr)
(s,\omega)
=
\frac{\partial}{\partial s}\mathscr{R}u(s,\omega)
\notag
\\
\text{und}\qquad
\sum_{i=1}^n
b_i
\biggl(
\mathscr{R}\frac{\partial u}{\partial x_i}
\biggr)(s,\omega)
&=0.
\label{buch:radon:ableitungen:eqn:babl}
\end{align}
Die Bedingung \eqref{buch:radon:ableitungen:eqn:babl}
bedeutet, dass der Vektor $v$ mit den Komponenten
\[
v_i
=
\biggl(\mathscr{R}\frac{\partial u}{\partial x_i}\biggr)(s,\omega)
\]
orthogonal ist zu allen Vektoren $b\perp \omega$, er hat daher
die Richtung $\omega$.
Da $|\omega|=1$ ist, ist die Länge $|v|=v\cdot \omega$ die Projektion
von $v$ auf die Richtung $\omega$ oder
\[
|v|
=
v\cdot\omega
=
\sum_{i=1}^n \omega_i
\biggl(
\mathscr{R}\frac{\partial u}{\partial x_i}
\biggr)(s,\omega)
=
\frac{\partial}{\partial s}\mathscr{R}u(s,\omega).
\]
Die einzelnen Komponenten sind daher
\[
v_i
=
|v|\omega_i
=
\omega_i
\frac{\partial}{\partial s}\mathscr{R}(s,\omega).
\]

\begin{satz}
Ist $u\colon\mathbb{R}^n\to\mathbb{C}$ eine ingegrierbare Funktion
mit kompaktem Träger, dann ist die Ra\-don-Trans\-for\-ma\-tion der
Ableitungen von $u$
\[
\biggl(
\mathscr{R}
\frac{
\partial^{|\bm{\alpha}|}
}{
\partial x_1^{\alpha_1}\cdots\partial x_n^{\alpha_n}
}
u
\biggr)(s,\omega)
=
(\mathscr{R}
D^{\bm{\alpha}} u
)(s,\omega)
=
\omega_1^{\alpha_1} \cdots \omega_n^{\alpha_n}
\frac{
\partial^{|\bm{\alpha}|}
}{
\partial s^{|\bm{\alpha}|}
}
\mathscr{R}u(s,\omega)
\]
\end{satz}

%
% Laplace-Operator
%
\subsection{Laplace-Operator
\label{buch:radon:ableitungen:subsection:laplace}}
Die Radon-Transformation $\mathscr{R}u$ zerlegt die Information
in der Funktion $u$ in Information, die von der Richung $\omega$
abhängt und solche, die vom Radius $s$ abhängt.
Ableitungen hängen dann nur noch von $s$ ab, was sich auch
auf den Laplace-Operator auswirkt.

\begin{satz}
\label{buch:radon:ableitungen:satz:laplace}
Für eine zweimal stetig differenzierbare Funktion
$u\colon\mathbb{R}^n\to\mathbb{C}$ mit kompaktem Träger gilt
\[
(\mathscr{R}\Delta u)(s,\omega)
=
\frac{\partial^2}{\partial s^2} \mathscr{R}u(s,\omega).
\]
\end{satz}

\begin{proof}[Beweis]
Der Laplace-Operator ist
\[
\Delta
=
\sum_{i=1}^n
\frac{\partial^2}{\partial x_i^2}.
\]
Zusammen mit der Ableitungsformel folgt
\begin{align*}
(\mathscr{R}\Delta u)(s,\omega)
&=
\sum_{i=1}^n
\biggl(\mathscr{R}\frac{\partial^2}{\partial x_i^2}u\biggr)(s,\omega)
\\
&=
\sum_{i=1}^n \omega_i^2 \frac{\partial^2}{\partial s^2}\mathscr{R}u(s,\omega)
\\
&=
|\omega|^2 \frac{\partial^2}{\partial s^2} \mathscr{R}u(s,\omega)
=
\frac{\partial^2}{\partial s^2} \mathscr{R}u(s,\omega).
\qedhere
\end{align*}
\end{proof}

Die Radon-Transformation transformiert den Laplace-Operator
auf einen zweiten Ableitungsoperator nach nur einer einzigen
Variablen.

%
% Wellengleichung
%
\subsection{Wellengleichung
\label{buch:radon:ableitungen:subsection:wellengleichung}}
Der Satz~\ref{buch:radon:ableitungen:satz:laplace} über die
Radon-Transformierte des Laplace-Operators ermöglich, auch die
Lösung der Wellengleichung mit Hilfe der Radon-Transformation zu
vereinfachen.
Die Wellengleichung ist die Gleichung
\[
\frac{\partial^2}{\partial t^2}u = \Delta u
\]
auf dem Gebiet $\Omega=\mathbb{R}^n$.
Nach Anwendung der Radon-Transformation wird aus den beiden Seiten
\begin{align*}
\biggl(\mathscr{R}\frac{\partial^2}{\partial t^2}u\biggr)(s,\omega)
&=
\frac{\partial^2}{\partial t^2}(\mathscr{R}u)(s,\omega)
\\
(\mathscr{R}\Delta u)(s,\omega)
&=
\frac{\partial^2}{\partial s^2}(\mathscr{R}u)(s,\omega)
\end{align*}
Schreibt man $U(t,s\omega)=\mathscr{R}u(s,\omega)$, dann folgt
\[
\frac{\partial^2 U}{\partial t^2}
=
\frac{\partial^2 U}{\partial s^2},
\]
also eine eindimensionale Wellengleichung.
Durch Mittelung über Hyperebenen orthogonal zu $\omega$ macht
die Radon-Transformation aus der $n$-dimensionalen Wellengleichung
eine eindimensionale Wellengleichung für die Ausbreitung des
Mittelwertes $U(t,s,\omega)$ in Richtung $\omega$.
Dieses Prinzip wird manchmal auch als das Huygens-Prinzip 
bezeichnet, da es verwandt ist mit der anschaulichen Vorstellung
von Huygens, wie sich Wellenfronten aus Elementarwellen zusammensetzen.
Letztere wird in Kapitel~\ref{chapter:opt} genauer studiert.


