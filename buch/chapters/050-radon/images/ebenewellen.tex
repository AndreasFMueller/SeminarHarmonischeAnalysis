%
% ebenewellen.tex -- Ebene Wellen
%
% (c) 2021 Prof Dr Andreas Müller, OST Ostschweizer Fachhochschule
%
\documentclass[tikz]{standalone}
\usepackage{amsmath}
\usepackage{times}
\usepackage{txfonts}
\usepackage{pgfplots}
\usepackage{csvsimple}
\usetikzlibrary{arrows,intersections,math,calc}
\begin{document}
\def\skala{1}
\definecolor{darkgreen}{rgb}{0,0.7,0}
\begin{tikzpicture}[>=latex,thick,scale=\skala]

\def\kx{0.6}
\def\ky{0.8}
\def\l{3.0}
\def\w{0.025}
\def\streifen#1#2{
	\pgfmathparse{100*sin((#2)*120)*sin((#2)*120)}
	\xdef\c{\pgfmathresult}
	\fill[color=#1!\c]
		({\kx*(#2)},{\ky*(#2)})
		-- ++({8*\ky},{-8*\kx})
		-- ++({\w*\kx},{\w*\ky})
		-- ++({-16*\ky},{16*\kx})
		-- ++({-\w*\kx},{-\w*\ky})
		-- cycle;
}

\begin{scope}
	\clip (-6.4,-4.0) rectangle (6.4,4.0);
	\foreach \d in {-9,-6,...,9}{
		\foreach \h in {0,0.02,...,1.5}{
			\streifen{darkgreen}{(\h)+(\d)}
		}
		\foreach \h in {1.5,1.52,...,3.0}{
			\streifen{blue}{(\h)+(\d)}
		}
	}
	\draw[color=red]
		($({\l*\kx},{\l*\ky})+({-8*\ky},{8*\kx})$)
		--
		++({16*\ky},{-16*\kx});
\end{scope}

\node[color=red] at ({\l*\kx},{\l*\ky})
	[above,rotate={-atan(\kx/\ky)}]
	{$k\cdot x + l\cdot y = \operatorname{const}$};

\node[color=red] at ({0.5*\l*\kx},{0.5*\l*\ky})
	[above left] {$\begin{pmatrix}k\\l\end{pmatrix}$};

\node[color=white] at ($({-0.70*\kx},{-0.70*\ky})+({-3*\ky},{3*\kx})$)
	[rotate={-atan(\kx/\ky)}]
	{$\varphi_{kl}(x,y)=e^{i(kx+ly)}$};

\draw[->] (-6.5,0) -- (6.8,0) coordinate[label={$x$}];
\draw[->] (0,-4.1) -- (0,4.3) coordinate[label={right:$y$}];

\draw[->,color=red,line width=2.0pt] (0,0) -- ({\l*\kx},{\l*\ky});

\end{tikzpicture}
\end{document}

