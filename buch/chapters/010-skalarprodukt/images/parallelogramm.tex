%
% parallelogramm.tex -- Parallelogramm-Regel
%
% (c) 2021 Prof Dr Andreas Müller, OST Ostschweizer Fachhochschule
%
\documentclass[tikz]{standalone}
\usepackage{amsmath}
\usepackage{times}
\usepackage{txfonts}
\usepackage{pgfplots}
\usepackage{csvsimple}
\usetikzlibrary{arrows,intersections,math,calc}
\begin{document}
\def\skala{0.5}
\def\punkt#1{
\fill #1 circle[radius=0.12];
\fill[color=white] #1 circle[radius=0.08];
}
\begin{tikzpicture}[>=latex,thick,scale=\skala]

\coordinate (O) at (0,0);
\coordinate (A) at (7,1);
\coordinate (B) at (0.1,6);

\fill[color=gray!10] (O) -- (A) -- ($(A)+(B)$) -- (B) -- cycle;

\draw[->,line width=1pt,color=blue] (O) -- (A);
\draw[->,line width=1pt,color=blue] (O) -- (B);
\draw[->,line width=1pt,color=blue] (A) -- ($(A)+(B)$);
\draw[->,line width=1pt,color=blue] (B) -- ($(A)+(B)$);

\draw[->,line width=1pt,color=red] (A) -- (B);
\draw[->,line width=1pt,color=red] (O) -- ($(A)+(B)$);

\node[color=blue] at ($0.5*(A)$) [below] {$x$};
\node[color=blue] at ($0.5*(B)$) [left] {$y$};
\node[color=blue] at ($(A)+0.5*(B)$) [right] {$y$};
\node[color=blue] at ($(B)+0.5*(A)$) [above] {$x$};

\node[color=red] at ($0.75*(A)+0.75*(B)$) [left] {$x+y$};
\node[color=red] at ($0.35*(A)+0.70*(B)$) [below left] {$x-y$};

\punkt{(O)}
\punkt{(A)}
\punkt{(B)}
\punkt{($(A)+(B)$)}

\node at (O) [below left] {$A$};
\node at (A) [below right] {$B$};
\node at (B) [above left] {$D$};
\node at ($(A)+(B)$) [above right] {$C$};

\node at (9,3.5) [right] {$|AC|^2 + |BD|^2= |AB|^2 + |BC|^2 + |AD|^2 + |CD|^2$};

\end{tikzpicture}
\end{document}

