%
% kreise.tex -- Kreise von skalarprodukten
%
% (c) 2021 Prof Dr Andreas Müller, OST Ostschweizer Fachhochschule
%
\documentclass[tikz]{standalone}
\usepackage{amsmath}
\usepackage{times}
\usepackage{txfonts}
\usepackage{pgfplots}
\usepackage{csvsimple}
\usetikzlibrary{arrows,intersections,math}
\begin{document}
\def\skala{1}
\def\l{2.1}
\def\axes{
\draw[->] (-\l,0) -- (\l,0) coordinate[label={$x_1$}];
\draw[->] (0,-\l) -- (0,\l) coordinate[label={$x_1$}];
}
\begin{tikzpicture}[>=latex,thick,scale=\skala]

\begin{scope}[xshift=-4.5cm]
\axes
\draw[color=red,line width=1.4pt] (0,0) circle[radius=1.5];
\draw (1.5,-0.05) -- (1.5,0.05);
\draw (-1.5,-0.05) -- (-1.5,0.05);
\draw (-0.05,1.5) -- (0.05,1.5);
\draw (-0.05,-1.5) -- (0.05,-1.5);
\node at (1.5,0) [below right] {$1$};
\node at (-1.5,0) [below left] {$-1$};
\node at (0,1.5) [above left] {$1$};
\node at (0,-1.5) [below left] {$-1$};
\node at (0,{-\l-0.3}) {$x_1^2+x_2^2=1\mathstrut$};
\end{scope}

\begin{scope}
\axes
\def\s{1.4142}
\begin{scope}[rotate=45]
\draw[color=red,line width=1.4pt] (0,0) ellipse (1cm and 2cm);
\end{scope}
\draw (\s,-0.05) -- (\s,0.05);
\draw (-\s,-0.05) -- (-\s,0.05);
\draw (-0.05,\s) -- (0.05,\s);
\draw (-0.05,-\s) -- (0.05,-\s);
\node at (\s,0) [below right] {$1$};
\node at (-\s,0) [below left] {$-1$};
\node at (0,\s) [above right] {$1$};
\node at (0,-\s) [below left] {$-1$};
\draw[line width=0.3pt] (\l,-\l) -- (-\l,\l);
\draw[line width=0.3pt] (-\l,-\l) -- (\l,\l);
\node at (0,{-\l-0.3}) {$5x_1^2+6x_1x_2+5x_2^2=4\mathstrut$};
\fill[color=red] ({0.5*\s},{0.5*\s}) circle[radius=0.06];
\fill[color=red] ({-0.5*\s},{-0.5*\s}) circle[radius=0.06];
\fill[color=red] ({\s},{-\s}) circle[radius=0.06];
\fill[color=red] ({-\s},{\s}) circle[radius=0.06];
\end{scope}

\begin{scope}[xshift=4.5cm]
\axes
\draw[color=red,line width=1.4pt]
	plot[domain=-1.4:1.4,samples=20]
		({cosh(\x)},{sinh(\x)});
\draw[color=red,line width=1.4pt]
	plot[domain=-1.4:1.4,samples=20]
		({-cosh(\x)},{sinh(\x)});
\draw (1,-0.05) -- (1,0.05);
\draw (-1,-0.05) -- (-1,0.05);
\draw (-0.05,1) -- (0.05,1);
\draw (-0.05,-1) -- (0.05,-1);
\node at (0,1) [above left] {$1$};
\node at (0,-1) [below left] {$-1$};
\node at (1,0) [above left] {$1$};
\node at (-1,0) [above left] {$-1$};
\node at (0,{-\l-0.3}) {$x_1^2-x_2^2=1\mathstrut$};
\end{scope}

\end{tikzpicture}
\end{document}

