%
% 4-hilbertraum.tex
%
% (c) 2022 Prof Dr Andreas Müller, OST Ostschweizer Fachhochschule
%
\section{Hilbertraum
\label{buch:skalarprodukt:section:hilbertraum}}
\kopfrechts{Hilbertraum}
Ein Skalarprodukt stattet einen Vektorraum mit einer Norm aus.
Es ermöglicht auch, orthonormierte Vektoren zu finden.
In endlichdimensionalen Vektorräumen können so besonders nützliche
Basen konstruiert werden.
In den Funktionenräumen von
Abschnitt~\ref{buch:skalarprodukt:section:funktionenraeume},
die unendlichdimensional sind, kann der Orthonormalisierungsprozess
ohne Ende weitergeführt werden.
Im Gegensatz zu einem endlichdimensionalen Vektorraum bilden diese
orthonormierten Vektoren keine Basis, denn nicht jeder Vektor lässt
sich als Linearkombination schreiben.
Dies wird erst mit Hilfe von Reihenentwicklungen möglich, doch dazu
müssen Fragen der Konvergenz solcher Reihen geklärt werden.
Der in diesem Abschnitt eingeführte Begriff des Hilbertraumes tut dies.


\subsection{Prähilbertraum}

\subsection{Vollständigkeit}

\subsection{Endlichdimensionale Hilberträume}

\subsection{Hilbertbasis}

\subsection{Der Hilbertraum $l^2$}

\subsection{Der Hilbertraum $L^2$}

