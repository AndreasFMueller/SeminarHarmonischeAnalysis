%
% 4-hilbertraum.tex
%
% (c) 2022 Prof Dr Andreas Müller, OST Ostschweizer Fachhochschule
%
\section{Hilbert-Raum
\label{buch:skalarprodukt:section:hilbertraum}}
\kopfrechts{Hilbert-Raum}
Ein Skalarprodukt stattet einen Vektorraum mit einer Norm aus.
Es ermöglicht auch, orthonormierte Vektoren zu finden.
In endlichdimensionalen Vektorräumen können so besonders nützliche
Basen konstruiert werden.
In den Funktionenräumen von
Abschnitt~\ref{buch:skalarprodukt:section:funktionenraeume},
die unendlichdimensional sind, kann der Orthonormalisierungsprozess
ohne Ende weitergeführt werden.
Im Gegensatz zu einem endlichdimensionalen Vektorraum bilden diese
orthonormierten Vektoren keine Basis, denn nicht jeder Vektor lässt
sich als Linearkombination schreiben.
Dies wird erst mit Hilfe von Reihenentwicklungen möglich, doch dazu
müssen Fragen der Konvergenz solcher Reihen geklärt werden.
Der in diesem Abschnitt eingeführte Begriff des Hilbert-Raumes tut dies.

%
% Prähilbertraum
%
\subsection{Prähilbertraum}
Die Funktionenräume, in denen wir harmonische Analysis betreiben wollen,
zeichnen sich durch das Vorhandensein eines Skalarproduktes aus.
Wir fassen diese Eigenaschaften im Begriff des Prähilbertraumes
zusammen.

\begin{definition}
Ein reeller Prähilbertraum ist ein reller Vektorraum mit einem
(reellen) Skalarprodukt.
\index{Prähilbertraum}%
Eine komplexer Prähilbertraum ist ein komplexer Vektorraum mit einem
sesquilinearen Skalarprodukt.
\end{definition}

\begin{beispiel}
Der endlichdimensionale reelle Vektorraum $\mathbb{R}^n$ ist ein
reller Prähilbertraum mit dem Skalarprodukt
\[
\langle u,v\rangle
=
\sum_{i=1}^n u_iv_i
\]
für Vektoren $u,v\in\mathbb{R}^n$.
\end{beispiel}

\begin{beispiel}
Der endlichedimensionale komplexe Vektorrau $\mathbb{C}$ ist ein
komplexer Prähilbertraum mit dem Skalarprodukt
\[
\langle u,v\rangle
=
\sum_{i=1}^n \overline{u}_iv_i
\]
für Vektoren $u,v\in\mathbb{C}^n$.
\end{beispiel}

Die Skalarprodukte in den Beispielen sind nicht die einzig möglichen
Skalarprodukte.
Alternative Skalarprodukte auf einem reellen Prähilbertraum können 
durch eine beliebige positiv definite Matrix $A$ durch
\[
\langle u,v\rangle_A
=
\sum_{i,j=1}^n u_ia_{ij}v_j
\]
definiert werden.
Wir schreiben die aus $\langle\;,\;\rangle_A$ abgeleitete Norm mit
$\normfunc_A$.
Solange unser primäres Interesse der Approximation von Funktionen gilt,
kommt es vor allem darauf an, dass die Norm, die aus dem Skalarprodukt
abgeleitet wird, zu den gleichen konvergenten Folgen führen.
Die Funktion $u\mapsto \|u\|_A$ ist stetig, sie hat daher auf der
Einheitskugel des Prähilbertraumes ein Maximum und eine Minimum,
welches wir mit $M$ bzw.~$m$ bezeichen.
Dann folgt, dass
\[
m\|u\|\le \|u\|_A\le M\|u\|
\]
für beliebige Vektoren $u\in\mathbb{R}^n$.
Daraus kann man jetzt ableiten, dass die beiden Normen $\normfunc$
und $\normfunc_A$ auf die gleichen Cauchy-Folgen und die gleichen
konvergenten Folgen führen.
Wir zeigen dies für Cauchy-Folgen:
\begin{enumerate}
\item
Sei $u_k$ eine Cauchy-Folge bezüglich der Norm $\normfunc$,
und $\varepsilon>0$.
Wir müssen zeigen, dass $u_k$ auch eine Cauchy-Folge ist bezüglich
der Norm $\normfunc_A$.
Da $u_k$ eine Cauchy-Folge bezüglich der Norm $\normfunc$ ist,
gibt es ein $N>0$ derart, dass
$\|u_k-u_l\|<\varepsilon/M$ für $k,l>N$.
Dann folgt aber
\[
\|u_k-u_l\|_A
\le
M\|u_k-u_l\|
<
M\frac{\varepsilon}{M}
=
\varepsilon
\]
für $k,l>N$.
Somit ist $u_k$ eine Cauchy-Folge bezüglich der Norm $\normfunc_A$.
\item
Ist umgekehrt  $u_k$ eine Cauchy-Folge bezüglich der Norm $\|\,\cdot\,\|_A$,
dann gibt es ein $N>0$ derart, dass $\|u_k-u_l\|_A<m\varepsilon$ ist für
$k,l>N$.
Dann folgt
\[
m\|u_k-u_l\|\le \|u_k-u_l\|_A < m\varepsilon
\qquad\Rightarrow\qquad \|u_k-u_l\|<\varepsilon
\]
für $k,l>N$, also ist $u_k$ auch eine Cauchy-Folge bezüglich der Norm
$\|\,\cdot\,\|$.
\end{enumerate}
In einem endlichdimensionalen Prähilbertraum hat die Wahl des Skalarproduktes
keinen Einfluss darauf, ob eine Folge eine Cauchy-Folge ist oder nicht.
Orthonormierte Vektoren werden natürlich im Allgemeinen nicht mehr
orthonormiert, dies ist jedoch ein Aspekt, dem wir uns erst später
zuwenden werden.

%
% Orthonormierte Vektoren
%
\subsection{Orthonormierte Vektoren in einem Prähilbertraum}
Der Gram-Schmidt-Orthogonalisierungsprozess kann auf eine beliebige
\index{Gram-Schmidt}%
linear unabhängige Menge von Vektoren in einem Prähilbertraum angewendet
werden.
Aus den linear unabhängigen Vektoren $a_1,a_2,\dots$ werden die
orthonormierten Vektoren
\begin{align*}
b_1
&=
\frac{a_1}{\|a_1\|}
\\
b_2
&=
\frac{
a_2 - \langle b_1,a_2\rangle b_1
}{
\|a_2 - \langle b_1,a_2\rangle b_1\|
}
\\
&\phantom{i}\vdots
\\
b_n
&=
\frac{\displaystyle
a_n - \sum_{k=1}^{n-1} \langle b_k,a_n\rangle b_k
}{\displaystyle
\biggl\|a_n - \sum_{k=1}^{n-1} \langle b_k,a_n\rangle b_k\biggr\|
}.
\end{align*}

In einem endlichdimensionalen Vektorraum der Dimension $n$ bricht
der Prozess ab, sobald eine orthonormierte Basis $b_1,\dots,b_n$
aus $n$ Vektoren gefunden wurde.
Jeder andere Vektor $v$ lässt sich dann als Linearkombination
\begin{equation}
v
=
\langle b_1,v\rangle b_1 + \langle b_2,v\rangle b_2 + \dots
=
\sum_{k=1}^n \langle b_1,v\rangle b_1
\label{buch:skalarprodukt:hilbertraum:synthese}
\end{equation}
schreiben.
Da die Summe auf der rechten Seite endlich ist, entstehen keine
Bedenken bezüglich Konvergenz, wie das bei einer unendlichen
Reihe der Fall wäre.

%
% Vollständigkeit
%
\subsection{Vollständigkeit}
In einem unendlichdimensionalen Prähilbertraum bricht der
Orthogonalisierungsprozess nicht ab, es gibt immer noch einen
linear unabhängigen Vektor, der nicht in dem von den bereits
gefundenen Vektoren aufgespannten Raum liegt.
Die Summe~\ref{buch:skalarprodukt:hilbertraum:synthese} wird dann
eine unendliche Summe, die nur im Sinne eines Grenzwertes der
Partialsummenfolge
\begin{equation*}
s_n = \sum_{k=1}^n \langle b_k,v\rangle b_k
\end{equation*}
ausgewertet werden kann.
Man darf zwar aufgrund der Konstruktion aus $v$ davon ausgehen,
dass $s_n$ gegen $v$ konvergiert,
aber für eine beliebige Folge von Koeffizienten $c_k$ ist nicht
garantiert, dass die Summe
\[
\sum_{k=1}^\infty c_kb_k
=
\lim_{n\to\infty} \sum_{k=1}^n c_kb_k
\]
einen Grenzwert hat.
Ein nützliche Theorie kann nur entstehen, wenn gefordert wird,
dass jede Cauchy-Folge des Prähilbertraums tatschächlich konvergiert.

\begin{definition}
Ein Prähilbertraum heisst {\em Hilbert-Raum}, wenn er vollständig ist.
\end{definition}

Endlichdimensionale Vektorräume über sind automatisch vollständig,
da gibt es also gar keinen Unterschied zwischen Prähilbertraum und
Hilbert-Raum.
Das folgende Beispiel zeigt, dass dies für unendlichdimensionale
Hilberträume nicht mehr zutrifft.

\begin{beispiel}
\label{buch:skalarprodukt:hilbertraum:bsp:sinreihe}
Der Funktionenraum
\(
C_{\mathbb{R}}([-\pi,\pi])
\)
der stetigen Funktionen auf dem Intervall $[-\pi,\pi]$ wird mit
dem Skalarprodukt
\[
\langle f,g\rangle
=
\int_{-\pi}^\pi f(x)g(x)\,dx
\]
zu einem Prähilbert-Raum.
Die Summanden der Reihe~\eqref{buch:skalarprodukt:eqn:rechteckreihe} 
sind Sinus-Funktionen, von denen wir später zeigen werden, dass sie
orthogonal sind.
Seien $s_n(x)$ die Partialsummen der Reihe, also
\begin{equation}
s_n(x) = \frac{4}{\pi}\sum_{k=0}^n \frac{\sin (2k+1)x}{2k+1},
\label{buch:skalarprodukt:hilbertraum:eqn:sn}
\end{equation}
dann kann man auch die Norm $\|s_n-s_m\|$, es gilt nämlich
\begin{equation}
\|s_n-s_m\|
=
\biggl\|
\frac{4}{\pi}
\sum_{k=m}^n \frac{\sin (2k+1)x}{2k+1}
\biggr\|,
\label{buch:skalarprodukt:hilbertraum:eqn:snsm}
\end{equation}
wobei wir $n>m$ angenommen haben, was wir ohne Beschränkung der 
Allgemeinheit tun dürfen.
Die Norm eines einzeln Terms ist
\begin{align}
\|\sin rx\|^2
&=
\int_{-\pi}^\pi \sin^2 rx\,dx
=
\int_{-\pi}^\pi \frac12 - \frac{\cos rx}{2}\,dx
=
\int_{-\pi}^\pi \frac12\,dx - \int_{-\pi}^\pi \frac{\cos rx}{2}\,dx.
\notag
\intertext{Der zweite Term ist ein Integral über eine Periode des
Integranden und verschindet daher.
Der erste Term ergibt daher}
\|\sin rx\|^2
&= \pi.
\intertext{Für die Terme der Summe
\eqref{buch:skalarprodukt:hilbertraum:eqn:sn}
folgt daher}
\biggl\|
\frac{\sin{2k+1}x}{2k+1}
\biggr\|^2
&=
\frac{\pi}{(2k+1)^2}.
\notag
\intertext{Für die Differenz
\eqref{buch:skalarprodukt:hilbertraum:eqn:snsm} finden wir daher}
\|s_n-s_m\|^2
\notag
&=
\frac{16}{\pi^2}
\sum_{k=m}^m \frac{\pi}{(2k+1)^2}
=
\frac{16}{\pi}
\sum_{k=m}^m \frac{1}{(2k+1)^2}.
\label{buch:skalarprodukt:hilbertraum:eqn:bsprest}
\end{align}
Da aus dem Analysisunterricht bekannt ist, dass die Reihe $\sum_k\frac1{k^2}$
konvergiert, kann die rechte Seite von 
\eqref{buch:skalarprodukt:hilbertraum:eqn:bsprest}
beliebig klein gemacht werden, die 
Reihe~\eqref{buch:skalarprodukt:eqn:rechteckreihe} 
ist also eine Cauchy-Folge im Prähilbertraum $C_{\mathbb{R}}([-\pi,\pi])$.
Die Grenzfunktion ist die Rechteckfunktion von
Abbildung~\ref{buch:skalarprodukt:fig:fourierrechteck}, sie ist nicht
stetig.
Wir haben also eine Cauchy-Folge im Prähilbertraum 
$C_{\mathbb{R}}([-\pi,\pi])$ gefunden, die darin nicht konvergiert.
\end{beispiel}

%
% Hilbertbasis
%
\subsection{Hilbert-Basis}
Sei jetzt $H$ ein Hilbert-Raum.
Führt man wieder die Konstruktion einer orthonormierten Basis durch,
entsteht eine Menge $\mathcal{B}=\{b_1,b_2,\dots\}$ orthonormierter
Vektoren.
In einem unendlichdimensionalen Hilbert-Raum ist $\mathcal{B}$ eine
undendliche Menge.
Die Vollständigkeit des Hilbertraumes garantiert, dass jede
Cauchy-Folge konvergiert, insbesondere können wir zu jedem beliebigen
Vektor $v$ die Koeffizienten $c_k=\langle b_k,v \rangle$ bestimmen
und versuchen, mit der
Summe~\eqref{buch:skalarprodukt:hilbertraum:synthese}
den Vektor zurückzugewinnen.
Vollständigkeit garantiert zwar die Konvergenz gegen einen Grenzwert
\[
v_0 = \sum_{k=1}^\infty c_k b_k,
\]
aber es gibt keine Garantie, dass $v=v_0$ ist.

\begin{beispiel}
Die Funktionen
\[
b_k(x) = \sin (2k+1)x
\qquad\text{mit}\quad
k\in \mathbb{N}
\]
wurden in Beispiel~\ref{buch:skalarprodukt:hilbertraum:bsp:sinreihe}
die Rechteckfunktion synthetisiert.
Doch reichen diese Funktionen nicht aus, um alle Funktionen auf
dem Intervall $[-\pi,\pi]$ zu synthetisieren.

Für die konstante Funktion $v(x)$ sind die Koeffizienten
\[
\langle v,b_k\rangle
=
\int_{-\pi}^\pi v(x)b_k(x)\,dx
=
\int_{-\pi}^\pi \sin (2k+1)x\,dx
=
0,
\]
die Summe
\[
v_0
=
\sum_{k=0}^\infty
c_k
b_k(x)
=
0
\]
verschwindet also, $v\ne v_0$.
\end{beispiel}

Wir berechnen das Skalarprodukt 
\[
\langle v-v_0,b_k\rangle
=
\langle v,b_k\rangle
-
\biggl\langle \sum_{l=1}^\infty c_lb_l,b_k\biggr\rangle
=
c_k - \sum_{l=1}^\infty c_l\langle b_l,b_k\rangle
=
c_k - \sum_{l=1}^\infty c_l\delta_{lk}
=
c_k-c_k=0.
\]
Der Vektor $v-v_0$ muss also orthogonal sein zu allen Vektoren
$b_k$.
Die Menge der Vektoren $b_k$ kann also vergrössert werden um einen
weiteren Vektor, der zu allen vorhandenen Vektoren orthogonal ist.

\begin{satz}
Eine Menge von orthogonalen Vektoren in einem Prähilbertraum ist linear 
unabhängig.
\end{satz}

\begin{proof}[Beweis]
Angenommen, die orthogonalen Vektoren  $a_1,\dots,a_n$ sind linear
abhängig.
Dann gibt es Zahlen $\lambda_i$, die nicht alle $=0$ sind und derart,
dass
\[
\sum_{k=1}^n \lambda_ka_k
=
\lambda_1 a_1 + \ldots + \lambda_n a_n
=
0.
\]
Das Skalarprodukt mit $a_k$ ergibt
\[
0
=
\langle a_k,\lambda_1a_1 +\ldots + \lambda_na_n\rangle
=
\lambda_1
\langle a_k,a_1\rangle
+
\ldots
+
\lambda_n
\langle a_k,a_n\rangle
=
\lambda_k \|a_k\|^2
\]
für alle $k=1,\dots,n$.
Da die Vektoren $a_k\ne 0$ sind folgt, dass $\lambda_k=0$ sein muss
für alle $k=1,\dots,n$, im Widerspruch zur Voraussetzung, dass die
$\lambda_k$ nicht alle verschwinden.
Der Widerspruch zeigt, dass die Annahme der linearen Abhängigkeit
falsch ist.
\end{proof}

In einem endlichdimensionalen Prähilbertraum findet man immer eine
orthonormierte Basis, aus der sich jeder Vektor linear kombinieren
lässt.
In einem unendlichdimensionalen Raum muss dieser Begriff etwas 
erweitert werden.

\begin{definition}
Eine {\em Hilbert-Basis} des Hilbertraumes $H$ ist eine Menge von orthonormalen
\index{Hilbert-Basis}%
Vektoren $b_k\in H$, $k\in \mathbb{N}$ derart, für jeden Vektor $v\in H$
die Reihe 
\[
\sum_{k\in\mathbb{N}} \langle b_k,v\rangle b_k
\]
konvergiert und die Summe $v$ hat.
\end{definition}

Auf eine Hilbert-Basis ist die Inuition anwendbar, die man von 
einer orthonormierten Basis eines endlichdimensionalen Prählibertraumes
hat.

\begin{definition}
\index{separabel}%
Ein Hilbert-Raum, der eine Hilbert-Basis besitzt, heisst {\em separabel}.
\end{definition}

Nicht separable Hilberträume sind deutlich grösser.
In einem nicht separable Hilbertraum ist es nicht möglich, alle Vektoren
aus einer Folge von orthonormierten Vektoren linear zu kombinieren.
Dies verträgt sich nicht mit der Intuition, die in endlichdimensionalen
Vektorräumen entstanden ist.
Separabilität ist eine Art ``Endlichkeitseigenschaft''.
Man findet sie zum Beispiel bei den $L^2$-Räumen mit kompaktem
Definitionsgebiet.
Kompaktheit kann als eine topologische
Endlichkeitseigenschaft angesehen werden.
Es lassen sich aber auch nicht separable Hilberträume konstruieren,
zum Beispiel sogenannte fastperiodische Funktionen.
\index{fastperiodische Funktionen}%
Für unsere Anwendungen sind nicht separable Hilberträume jedoch nicht
notwendig und wir können im folgenden annehmen, dass alle Hilberträume
separabel sind und somit eine Hilbert-Basis haben.

%
% Der Hilbertraum $l^2$
%
\subsection{Der Hilbertraum $l^2$}
Die abstrakte Definition eines separablen Hilbert-Raumes suggeriert,
dass es sehr viele verschiedene Hilbert-Räume geben könnte.
Dem ist aber nicht so, ein Hilbert-Raum ist eine ziemlich starre Struktur.
In diesem Abschnitt konstrieren wir zunächst den besonders übersichtlichen
separablen Hilbert-Raum $l^2$ und zeigen anschliessend, dass sich jeder
separable HIlbertraum isometrisch auf $l^2$ abbilden lässt.

\begin{satz}
Der Vektorraum der unendlichen Folgen
\[
l^2_{\mathbb{R}}
=
\biggl\{
(x_0,x_1,\dots)
\in
\mathbb{R}^{\mathbb{N}}
\,\bigg|\,
\sum_{i=0}^\infty x_i^2<\infty\biggr\}
\}
\]
mit dem Skalarprodukt
\[
\langle x,y\rangle_2
=
\sum_{k=0}^\infty x_ky_k
\]
ist ein separabler Hilbert-Raum.
\end{satz}

\begin{proof}[Beweis]
Die Standardbasis des Vektorraums $l^2$ besteht aus den Vektoren $e_i$
mit $i\in\mathbb{N}$ mit 
\begin{align*}
e_0 &= (1,0,0,0,\dots)
\\
e_1 &= (0,1,0,0,\dots)
\\
e_2 &= (0,1,0,0,\dots)
\\  &\;\vdots
\end{align*}
Es ist klar, dass die Vektoren $e_i$ orthonormiert sind.
Wir müssen uns überlegen, dass jede Folge in $l^2$ durch Linearkombinationen
approximiert werden kann.
Wäre das nicht so, dann gäbe es einen von $0$ verschiedenen
Vektor $v=(v_0,v_1,v_2,\dots)\in l^2$, der auf allen Vektoren $e_i$
orthogonal ist.
Dann wäre aber $0=\langle v,e_i\rangle = v_i$, d.~h.~Vektor $v$ ist der
Nullvektor.
Dieser Widerspruch zeigt, dass die Vektoren $e_i$ eine Hilbert-Basis
bilden.
\end{proof}

Der Hilbert-Raum $l^2$ ist die einfachste, direkte Verallgemeinerung der
endlichdimensionalen Hilbert-Raume $\mathbb{R}^n$ mit dem Standardskalarprodukt.
Er ist aber auch der ``einzige'' separable Hilbert-Raum, alle Hilberträume
lassen sich isometrisch auf $l^2$ abbilden.

\begin{satz}
\label{buch:skalarprodukt:hilbertraum:satz:phil2}
Ist $H$ ein separabler Hilbert-Raum, dann gibt es eine isometrische
lineare Abbildung $\varphi\colon H\to l^2$.
\end{satz}

\begin{proof}[Beweis]
Da $H$ separabel ist, gibt es eine Hilbert-Basis $b_0,b_1,b_2,\ldots\in H$
und $v$ ein beliebiger Vektor mit der Darstellung
\[
v = \sum_{k=0}^\infty c_kb_k.
\]
Das Skalarprodukt mit $b_k$ ist
\[
\langle b_i,v\rangle
=
\biggl\langle b_i,\sum_{k=0}^\infty c_kb_k\biggr\rangle
=
\sum_{k=0}^\infty c_k \underbrace{\langle b_i,b_k\rangle}_{\delta_{ik}}
=
c_i.
\]
Die Abbildung
\[
\varphi\colon H \to l^2
:
v \mapsto (\langle b_0,v\rangle,\langle b_1,v\rangle,\langle b_2,v\rangle,\dots)
=
(c_0,c_1,c_2,\dots)=c\in l^2
\]
ist linear und umkehrbar durch die Abbildung
\[
\varphi^{-1} \colon l^2 \to H
:
(c_0,c_1,c_2,\dots) \mapsto \sum_{k=0}^\infty c_k b_k.
\]
Die Norm von $v$ ist
\[
\|v\|^2
=
\biggl\|
\sum_{k=0}^\infty c_k b_k
\biggr\|^2
=
\sum_{k=0}^\infty c_k^2
=
\|c\|_2^2.
\qedhere
\]
\end{proof}

Man verliert also nichts, wenn man statt im Hilbert-Raum $H$ im
Hilbert-Raum $l^2$ rechnet.
Die später zu besprechende Fourier-Transformation ist genau
so eine Abbildung $\varphi$ wie in
Satz~\ref{buch:skalarprodukt:hilbertraum:satz:phil2}, wenn 
trigonometrische Funktionen als Basisfunktionen des Raumes
$L^2([-\pi,\pi])$ verwendet werden.
Die Tatsache, dass die Abbildung eine Isometrie ist, hat einen
speziellen Namen.

\begin{definition}
\label{buch:skalarprodukt:hilbertraum:def:parseval}
Die Identität
\[
\|v\|^2 = \|\varphi(v)\|_2^2
\]
heisst {\em Parseval-Gleichung}.
Aus der Polaridentität folgt dann auch die
{\em Parseval-Plancherel-Formel}
\[
\langle u,v\rangle_H
=
\langle\varphi(u),\varphi(v)\rangle_2
=
\sum_{k=0}^\infty
\langle b_k,u\rangle
\langle b_k,v\rangle.
\]
\end{definition}

Die endlichdimensionalen Hilberträume können auf
natürliche Art in $l^2$ eingebettet werden, indem die Koordinaten
eines $n$-dimensionalen Vektors auf die ersten $n$-Komponenten der
Folge in $l^2$ abgebildet werden.
Aus dieser Idee lässt sich auch ein Prähilbertraum konstruieren,
in dem einfacher gerechnet werden kann.

\begin{definition}
\label{buch:skalarprodukt:hilbertraum:def:l0}
Sei 
\[
l_0
=
\{
x\in \mathbb{R}^{\mathbb{N}}
\mid
\text{$x_k\ne 0$ nur für endlich viele $k$}
\}
\]
der Vektorraum der Folgen, die nur endlich viele nicht verschwindende
Folgenglieder haben.
Die Standardbasisvektoren von $l^2$ sind alle in $l_0$.
\end{definition}

Da die Norm eines Elementes $x\in l_0$ eine endliche Summe ist,
ist $x\in l^2$ und damit $l_0\subset l^2$.
Jeder Vektor $v\in l^2$ kann aber beliebig genau durch einen Vektor
in $l_0$ approximiert werden.
Dazu konstruiert man die Folge
\bgroup
\def\el#1{\phantom{v_0}\llap{#1}}
\begin{align*}
x_0 &= (v_0,\el{0},\el{0},\el{0},\el{0},\dots) \in l_0\\
x_1 &= (v_0,v_1,\el{0},\el{0},\el{0},\dots) \in l_0\\
x_2 &= (v_0,v_1,v_2,\el{0},\el{0},\dots) \in l_0\\
x_3 &= (v_0,v_1,v_2,v_3,\el{0},\dots) \in l_0\\
    &\;\vdots
\end{align*}
\egroup
Die Differenz $x_n-v$ hat die Norm
\[
\|x_n-v\|^2
=
\sum_{k=n+1}^\infty v_k^2,
\]
die für $n\to\infty$ gegen $0$ geht.
Da es in $l^2$ Vektoren gibt, die nicht in $l_0$ drin sind, zeigt
dies, dass $l_0$ kein Hilbert-Raum ist, dass aber die Vervollständigung
von $l_0$ der Hilbert-Raum $l^2$ ist.

Der Prähilbert-Raum rechtfertigt, dass man in Anwendungen im Sinne
einer Approximation mit endlich vielen der Koeffizienten $c_k$
rechnen kann.
Im Kontext der Fourier-Theorie sind dies
die trigonometrischen Polynome, mit denen sich jede periodische
Funktion approximieren lässt.

%
% Der Hilbert-Raum $L^2$
%
\subsection{Der Hilbert-Raum $L^2$}
Der Vektorraum der stetigen Funktionen auf dem Intervall $[-\pi,\pi]$
mit dem bekannten Skalarprodukt ist ein Prähilbertraum.
Wir werden später zeigen, dass die Funktionen
\[
1, \cos kx \quad\text{und}\quad \sin kx
\qquad \text{mit $k\in\mathbb{N}$ und $ k\ge 1$}
\]
orthogonal sind bezüglich dieses Skalarproduktes.
Es wird sich zeigen, dass diese Funktionen nach geeigneter Normierung
eine Hilbert-Basis bilden von $L^2$ bilden.
Die in Satz~\ref{buch:skalarprodukt:hilbertraum:satz:phil2}
konstruierte Abbildung bezeichnen wir mit $\mathscr{F}:L^2([-\pi,\pi])\to l^2$.
Die Parseval-Plancherel-Formel besagt dann, dass man das Skalarprodukt
von Funktionen auch durch das Skalarprodukt der zugehörigen
Fourier-Koeffizienten berechnen kann:
\[
\langle f,g\rangle_{L^2{[-\pi,\pi]}}
=
\langle \mathscr{F}f,\mathscr{F}g\rangle_{l^2}.
\]
Die Fourier-Transformation formalisiert also die Erkenntnis, dass
man alle Fragen, die man mit Skalarprodukten beantworten kann, statt
mit Funktionen auch mit Skalarprodukten von Fourier-Koeffizienten
finden kann.


