%
% 3-funktionenraeume.tex
%
% (c) 2022 Prof Dr Andreas Müller, OST Ostschweizer Fachhochschule
%
\section{Funktionenräume
\label{buch:skalarprodukt:section:funktionenraeume}}
\kopfrechts{Funktionenräume}
Ziel der harmonischen Analysis ist die effiziente Approximation einer
grossen Klasse von Funktionen.
Als approximierende Funktionen kommen stetige Funktionen, Polynome,
trigonometrische Polynome oder eine ähnlich, einfach konstruierbare
Funktionenfamilie in Frage.
Es gilt zunächst herauszufinden, was ``Approximation'' genau heissen
soll und von welchen Funktionen man überhaupt erwarten kann, dass sie
approximiert werden können.

%
% Stetige Funktionen
%
\subsection{Stetige Funktionen
\label{buch:skalarprodukt:subsection:stetige-funktionen}}
Der frühe intuitive Funktionsbegriff ging oft von der Vorstellung einer
in einem Strich gezeichneten Kurve aus, wie man sie von den Graphen
der Polynome oder der trigonometrischen Funktionen her kennt.
In moderner Sprechweise sind dies die stetigen Funktionen.

\begin{definition}
Eine Funktion $f\colon I\to\mathbb{R}$ mit $I\subset \mathbb{R}$
heisst stetig in einem Punkt $x_0\in I$, wenn für jedes $\varepsilon>0$
ein $\delta>0$ existiert derart, dass $f(x)-f(x_0)|<\delta$ sobald
$|x-x_0|<\varepsilon$.
\end{definition}

Nur die Eigenschaft, eine Abstandsmessung zu besitzen, wird vom
Definitionsbereich $I\subset \mathbb{R}$ verlangt.
Der Stetigkeitsbegriff kann daher verallgemeinert werden auf den
Begriff des metrischen Raumes.

\begin{definition}
Eine {\em Metrik} auf einer Menge $X$ ist eine Funktion
\index{Metrik}%
$d\colon X\times X\to \mathbb{R}$
mit den folgenden Eigenschaften
\begin{enumerate}
\item
Positiv definit: $d(x,y)\ge 0$ und $d(x,y)$ genau dann, wenn $x=y$.
\item
Symmetrie: \(d(x,y)=d(y,x)\)
\item
Dreiecksungleichung: \( d(x,y) \le d(x,z) + d(z,y) \).
\end{enumerate}
Ein {\em metrischer Raum} ist ein Menge $X$ mit einer Metrik.
\index{matrischer Raum}%
\end{definition}

In einem metrischen Raum ist der Begriff des Grenzwertes übertragbar.
Mit dem Begriff des Grenzwertes lässt sich auch der Begriff der
Stetigkeit verallgemeinern.

\begin{definition}
Ist $x_n\in X$ eine Folge von Punkten in einem metrischen Raum $X$,
dann heisst $x$ der Grenzwert der Folge $x_n$, wenn es für jedes
$\varepsilon>0$ ein $N>0$ gibt derhart, dass
$d(x_n,x)\le \varepsilon$ für alle $n>N$.
Eine Funktion $f\colon X\to Y$ zwischen metrischen Räumen heisst
stetig im Punkt $x\in X$, wenn für jede Folge $x_n\in X$ mit
Grenzwert $x$ auch die Folge $y_n=f(x_n)\in Y$ konvergiert und
den Grenzwert $y=f(x)$ hat.
\end{definition}

Teilmengen von $\mathbb{R}$ oder $\mathbb{R}^n$ tragen natürlich
die Struktur eines metrischen Raumes mit der Abstandsmessung in 
$\mathbb{R}^n$ als Metrik
\[
d(x,y) = \sqrt{(x_1-y_1)^2 + \ldots + (x_n-y_n)^2} = \|x-y\|.
\]
Die Eigenschaften einer Metrik wurden bereits in Abschnitt
\ref{buch:skalarprodukte:section:cauchyschwarz} nachgewiesen.

Der Begriff des Grenzwertes klärt, was mit der Approximation von $x$
durch eine Folge $x_n$ gemeint ist.
Wenn man darauf aufbauend die Konvergenz einer Folge von Funktionen
gegen eine Grenzfunktion definieren will, braucht man einen Abstansbegriff
zwischen Funktionen.
Ein erster Versuch könnte sein, als Abstand zwischen zwei Funktionen
$f$ und $g$ die Funktion
\[
d(f,g) = |f(x_0) - g(x_0)|.
\]
Die Menge der Funktionen wird dadurch jedoch nicht zu einem metrischen
Raum.
Zwar gilt sicher die Symmetrie und Dreiecksungleichung, und auch 
$d(f,g)\ge 0$ für beliebige Funktionen.
Aber wenn $d(f,g)=0$ ist, heisst das nur, dass $f$ und $g$ im Punkt
$x_0$ den gleichen Wert haben.
Ausser in trivialen Fällen wird es Funktionen geben, die zwar im Punkt
$x_0$ übereinstimmen, sich aber in mindestens einem anderen Punkt
unterscheiden.

%
% Normierte Räume
%
\subsubsection{Normierte Räume}
Die stetigen Funktionen bilden aber keine strukturlose Menge, sie
bilden einen Vektorraum: die Summe von stetigen Funktionen ist ebenfalls
stetig, multiplizieren einer stetigen Funktion mit einem Skalar führt
nicht aus der Menge der stetigen Funktionen heraus.
Die für den Grenzwertbegriff von Funktionen verwendete Abstandsmessung 
sollte der Vektorraumstruktur ebenfalls Rechnung tragen.

\begin{definition}
\label{buch:skalaprodukt:funktionenraume:def:norm}
Sei $V$ ein Vektorraum über $\mathbb{R}$, dann heisst eine Funktion
\( \|\;\cdot\;\| \colon V \to \mathbb{R}\) eine {\em Norm}, wenn gilt
\index{Norm}
\begin{enumerate}
\item
Definit: $ \|x\| = 0 \Rightarrow x=0$
\item
Homogeneität: $ \| \lambda x \| = |\lambda| \cdot \|x\|$
\item
Dreiecksungleichung: $\|x+y\| \le \|x\| + \|y\|$
\end{enumerate}
Ein {\em normierter Raum} ist ein Vektorraum $V$ mit einer Norm.
\end{definition}

%
% Vollständigkeit
%
\subsubsection{Vollständigkeit}
In den rationalen Zahlen hat nicht jede Folge einen Grenzwert.
Die Zahl $\sqrt{2}$ lässt sich beliebig genau durch rationale Zahlen
approximieren, sie ist aber nicht in $\mathbb{Q}$.
Ähnlich lässt sich die Funktion $x\mapsto \sqrt{x}$ beliebig genau 
durch Polyome approximieren, sie ist aber selbst kein Poylnome

\begin{definition}
Ein Folge $x_n\in X$ in einem metrischen Raum heisst {\em Cauchy-Folge},
wenn es für jedes $\varepsilon>0$ ein $N>0$ gibt derart, dass 
$|x_n-x_m|<\varepsilon$ wenn $n,m>N$ ist.
\end{definition}

Cauchy-Folgen sind also Folgen, die sich für genügend grossen Index
kaum mehr ändern und für die man daher Konvergenz erwarten würde.

\begin{definition}
Ein normierter Raum heisst {\em vollständig} oder ein Banach-Raum,
wenn jede Cauchy-Folge einen Grenzwert hat.
\end{definition}

Die rationalen Zahlen $\mathbb{Q}$ bilden keinen vollständigen
metrischen Raum, aber die reellen Zahlen $\mathbb{R}$ enthalten
alle Grenzwerte von Cauchy-Folgen, $\mathbb{R}$ ist eine vollständiger
metrischer Raum.
Die Menge der Polynome, betrachtet als Teilmenge der Menge der
stetigen Funktionen $[0,1]\to\mathbb{R}$ ist nicht vollständig,
da es eine Folge $f_n(x)$ von Approximationsfunktionen der Funktion
$x\mapsto \sqrt{x}$ gibt.
Als Cauchy-Folge konvergiert sie zwar gegen eine stetige Funktion,
aber die Grenzfunktion ist nicht mehr im Raum der Polynome.

Das Ziel der folgenden Kapitel ist also, zu geeignet interessanten
Funktionenfamilien ``gute'' Normen zu finden derart, dass Cauchy-Folgen
konvergieren gegen Funktionen, die immer noch ausreichend viele
nützliche Eigenschaften haben.
Im besten Fall konvergieren stetige Funktionen gegen stetige Funktionen,
es wird sich aber zeigen, dass diese Anforderung zu streng ist.

%
% Norm fpr stetige Funktionen
%
\subsection{Norm für stetige Funktionen
\label{buch:skalarprodukt:subsection:normfuerstetigefunktionen}}
Damit man von Konvergenz von Folgen stetiger Funktionen sprechen kann,
brauchen wir jetzt also eine Norm für stetige Funktionen.

\begin{definition}
Sei $X$ ein metrischer Raum und
\[
C(X)
=
C_{\mathbb{R}}(X)
=
\{
f\colon X \to\mathbb{R}\mid
\text{$f$ ist stetig}
\}
\]
der Vektorraum der stetigen Funktion auf $X$.
Die Norm von $C(X)$ ist definiert als
\[
\|f\| = \sup_{x\in X} |f(x)|.
\]
Sie heisst die {\em Supremum-Norm}.
\end{definition}

Wir prüfen nach, dass die Supremum-Norm tatsächlich eine Norm ist.
Dazu sind die definierenden Eigenschaften nachzurechnen:
\begin{enumerate}
\item Definit: 
\[
0
=
\|f\|
=
\sup_{x\in X} |f(x)|
\quad\Rightarrow\quad
f(x)=0 \;\forall x\in X
\quad\Rightarrow\quad
f\in C(X).
\]
\item Homogeneität:
\[
\|\lambda f\|
=
\sup_{x\in X} |\lambda f(x)|
=
|lambda| \sup_{x\in X} |f(x)|
=
|lambda| \cdot \|f\|.
\]
\item
Dreiecksungleichung:
\[
\|f+g\|
=
\sup_{x\in X}|f(x)+g(x)|
\le
\sup_{x\in X}(|f(x)|+|g(x)|)
\le
\sup_{x\in X}|f(x)|+\sup_{x\in X}|g(x)|
=
\|f\| + \|g\|.
\]
\end{enumerate}

Eine Cauchy-Folge $f_n$ von Funktionen $X\to \mathbb{R}$ hat die
Eigenschaft, dass für jedes $\varepsilon >0$ ein $N>0$ existiert,
derart dass $\|f_n-f_m\|<\varepsilon$ ist.
Da die Norm der maximale Unterschied von Funktionswerten ist,
folgt dass für eine Cauchy-Folge in $C(X)$ die Folge $f_n(x)$ eine
Cauchy-Folge in $\mathbb{R}$ ist und damit einen Grenzwert in $\mathbb{R}$
hat.
Die Funktion $f(x) = \lim_{n\to\infty}f_n(x)$ ist die Grenzfunktion.
Die Konvergenz bezüglich der Norm besagt, dass für jedes $\varepsilon>0$
es ein $N>0$ gibt derart, dass
\[
\varepsilon 
>
\|f_n-f\|
\ge 
|f_n(x)-f(x)|
\]
ist für alle $n>N$ und unabhängig von $x\in X$.
Die Konvergenz bezüglich der $\|\;\cdot\;\|$-Norm ist also die wohlbekannte
gleichmässige Konvergenz.
Es kann gezeigt werden, dass die Grenzfunktion wieder stetig ist.

\begin{satz}
Der Raum der stetigen Funktion $C(X)$ mit der Supremumg-Norm ist
ein Banach-Raum.
\end{satz}

%
% Skalarprodukt
%
\subsection{Skalarprodukt
\label{buch:skalarprodukt:subsection:skalarprodukt}}
Die Supremum-Norm auf dem Raum der stetigen Funktionen hat den
Begriff der gleichmässig konvergenten Funktionenfolgen ergeben.
Cauchy-Folgen von stetigen Funktionen in der Supremum-Norm konvergieren
wieder gegen eine stetige Funktione.
Ist eine Funktion nicht stetig, lässt Sie sich im Sinne der Supremum-Norm
nicht durch stetige Funktionen approximieren.
Andererseits hat Fourier gezeigt, wie man technische wichtige Funktionen
wie die Rechteckfunktion durch trigonometrische Polynome approximieren
kann.
Diese sind alle stetig und kommen der Rechteckfunktion in jedem Punkt,
in dem die Funktion stetig ist, beliebig nahe.
An den Stellen $x = n\pi$ hat die Grenzfunktion eine Sprungstellt,
die approximierenden Funktionen haben dort immer Abstand $1$.
% XXX Bild der Approximation der Rechteckfunktion
Die Folge ist also keine Cauchy-Folge und sie konvergiert nicht im
Sinne der Supremum-Norm.
Für solche Anwendungen muss eine besser geeignete Norm gefunden werden,
in der die Folge konvergiert.




