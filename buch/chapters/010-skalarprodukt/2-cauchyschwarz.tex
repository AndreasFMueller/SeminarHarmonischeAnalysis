%
% 2-cauchyschwarz.tex
%
% (c) 2022 Prof Dr Andreas Müller, OST Ostschweizer Fachhochschule
%
\section{Cauchy-Schwarz Ungleichung
\label{buch:skalarprodukte:section:cauchyschwarz}}
\kopfrechts{Cauchy-Schwarz-Ungleichung}
In der Vektorgeometrie wird gelehrt, dass die Länge eines Vektors $u$
durch die Norm $\|u\|$ wiedergegeben wird und dass die geometrische
Intuition dazu passt.
Dazu gehört vor allem, dass die Dreiecksungleichung erfüllt ist,
dass also für drei Punkt $A$, $B$ und $C$
\[
\overline{AB} \le \overline{AC} + \overline{BC}
\]
gilt.
In Vektorform bedeutet dies, dass
\[
\| b-a\|
\le
\| c-a\| + \|b-c\|.
\]
Schreibt man $u=c-a$ und $v=b-c$, dann ist $u+v=b-a$ und somit
\begin{equation}
\| u+v\| \le \|u\| + \|v\|.
\label{skalarprodukt:cauchyschwarz:eqn:dreieck0}
\end{equation}
Dies ist die Dreiecksungleichung in
Vektorform~\eqref{skalarprodukt:cauchyschwarz:eqn:dreieck0}.
Ziel dieses Abschnitts ist zu zeigen, dass jedes reelle oder
komplexe Skalarprodukt diese und weitere Eigenschaften automatisch
mitbringt.

%
% Cauchy-Schwarz-Ungleichung
%
\subsection{Cauchy-Schwarz-Ungleichung}

%
% Dreiecksungleichung
%
\subsection{Dreiecksungleichung}

%
% Polaridentität
%
\subsection{Polaridentität}
