%
% chapter.tex -- Skalarprodukt
%
% (c) 2021 Prof Dr Andreas Müller, Hochschule Rapperswil
%
% !TeX spellcheck = de_CH
\chapter{Skalarprodukte
\label{buch:chapter:skalarprodukte}}
\kopflinks{Skalarprodukte}

Das Skalarprodukt ist die Basis jeder Form von harmonischer Analysis.
Es misst die Ähnlichkeit zwischen Funktionen, hängt aber auch vom
Definitionsbereich und von den Ableitungen ab, die mitverglichen werden
sollen.
Alle diese vielen verschiedenartigen Skalarprodukte müssen mindestens
soviel gemeinsam haben, dass sich für jeden Funktionenraum geeignete
Familien von orthogonalen Basisfunktionen finden lassen.
Grundlage für diese Konstruktion sind die gemeinsamen Eigenschaften
aller Skalarprodukte, die in den ersten beiden Abschnitten hergeleitet
werden.
Die nachfolgenden Abschnitte dieses Kapitels sind den Eigenschaften der
Funktionenräume gewidmet, in denen in späteren Kapiteln harmonische
Analysis betrieben werden soll.

%
% 1-definition.tex
%
% (c) 2023 Prof Dr Andreas Müller, OST Ostschweizer Fachhochschule
%
\section{Definition
\label{buch:skalarprodukte:section:definition}}
\kopfrechts{Definition}
Ein Skalarprodukt ist vor allem deshalb besonders einfach anzuwenden,
weil es bilinear ist.
Dies bedeutet, dass man Skalarprodukte ausmultiplizieren kann, die
Intuition für Produkte, die man aus der elementaren Algebra mitbringt,
führt zum Erfolg.
Allerdings braucht es für ein erfolgreiches Skalarprodukt noch
etwas mehr.

%
% Symmetrische Bilinearformen
%
\subsection{Symmetrische Bilinearformen}
Das aus der Vektorgeometrie bekannte Skalarprodukt
\index{Vektorgeometrie}%
\index{Skalarprodukt}%
\[
\vec{u}\cdot \vec{v}
=
\sum_{i=1}^n u_iv_i
\]
auf $\mathbb{R}^n$ ist deshalb besonders nützlich, weil sich damit
so rechnen lässt, wie man es sich von einem Produkt in der Algebra
gewohnt ist.
Dazu gehört, dass man Produkte ausmultiplizieren kann:
\begin{equation}
\begin{aligned}
(\lambda\vec{u}+\mu\vec{w})\cdot\vec{v}
&=
\lambda\vec{u}\cdot\vec{v}+\mu\vec{w}\cdot\vec{v}
\\
\vec{u}\cdot(\lambda\vec{v}+\mu\vec{w})
&=
\lambda\vec{u}\cdot\vec{v}+\mu\vec{u}\cdot\vec{w}.
\end{aligned}
\label{buch:skalarprodukt:eqn:ausmultiplizieren}
\end{equation}
Umgekehrt kann man gemeinsame Faktoren auch ausklammern.
Die Rechenregeln \eqref{buch:skalarprodukt:eqn:ausmultiplizieren}
besagen, dass die Funktion
\[
\cdot
\;
\colon
\mathbb{R}^n \times \mathbb{R}^n
\to
\mathbb{R}
:
(\vec{u},\vec{v}) \mapsto \vec{u}\cdot\vec{v}
\]
in jedem Faktor linear ist.
Die folgende Definition verallgemeinert die Idee auf einen
beliebigen reellen Vektorraum $V$.

\begin{definition}
Eine Funktion
\[
b\colon
V\times V \to \mathbb{R}
:
(u,v) \mapsto b(u,v)
\]
heisst {\em bilinear} oder {\em Bilinearform},
wenn sie linear ist in jedem Argument, wenn also
\index{bilinear}%
\index{Bilinearform}%
\[
\begin{aligned}
b(\lambda u+\mu w,v) &= \lambda b(u,v) + \mu(w,v)
\\
\text{und}\qquad
b(u,\lambda v+\mu w) &= \lambda b(u,v) + \mu(u,w)
\end{aligned}
\]
gilt für beliebige Vektoren $u,v,w\in V$ und Skalare
$\lambda,\mu\in\mathbb{R}$.
\end{definition}

Das Skalarprodukt der Vektorgeometrie hat aber noch eine weitere
wichtige Eigenschaft.
Es ist kommutativ, es kommt nicht auf die Reihenfolge der Faktoren an.
Dies ist zum Beispiel wichtig, um den Kosinus-Satz der ebenen Trigonometrie
zu erhalte.
Der für den Vektor
\index{Kosinus-Satz}%
\index{Trigonometrie}%
$\vec{c} = \vec{b}-\vec{a}$ 
kann man ihn durch Berechnung des Skalarproduktes von $\vec{c}$ mit
sich selbst als
\begin{align*}
|\vec{c}|^2
&=
\vec{c}\cdot\vec{c}
=
(\vec{b}-\vec{a})\cdot(\vec{b}-\vec{a})
=
\vec{b}\cdot\vec{b}
-
\vec{b}\cdot\vec{a}
-
\vec{a}\cdot\vec{b}
+
\vec{a}\cdot\vec{a}
\\
&=
|\vec{a}|^2 + |\vec{b}|^2 - 2 \vec{a}\cdot\vec{b}
=
|\vec{a}|^2 + |\vec{b}|^2 - 2 |\vec{a}|\;|\vec{b}|\cos\alpha
\end{align*}
erhalten.
Dabei wurde verwendet, dass $\vec{a}\cdot\vec{b}=\vec{b}\cdot\vec{a}$ ist.
Für ein Skalarprodukt heisst die Kommutativität allerdings anders.

\begin{definition}
Eine Funktion $b\colon V\times V \to\mathbb{R}$ heisst {\em symmetrisch},
wenn $b(u,v)=b(v,u)$ für alle $u,v\in V$.
\index{symmetrisch}%
\end{definition}

Eine symmetrische Bilinearform erfüllt die binomische Formel
\begin{align*}
b(u+v,u+v)
&=
b(u,u+v) + b(v,u+v)
=
b(u,u)+b(u,v)+b(v,u)+b(v,v)
\\
&=
b(u,u) + 2b(u,v) + b(v,v),
\end{align*}
ein weiteres Indiz dafür, dass das Skalarprodukt bei der algebraischen
Rechnung wie ein ``gewöhnliches'' Produkt behandelt werden kann.

Eine Bilinearform auf einem endlichdimesionalen Vektorraum $V$
kann mit Hilfe einer Basis der Berechnung leichter zugänglich
gemacht werden.
Seien $b_1,\dots,b_n\in V$ die Vektoren einer Basis von $V$.
Dann können Vektoren $u,v\in V$ mit Hilfe der Koordinaten
$u_i\in\mathbb{R}$ und $v_i\in\mathbb{R}$ als Linearkombinationen
\[
u = \sum_{i=1}^n u_ib_i,
\qquad \text{und}\qquad
v = \sum_{i=1}^n v_ib_i
\]
aus Basisvektoren geschrieben werden.
Für das Skalarprodukt folgt dann
\begin{align*}
\langle u,v\rangle
&=
\biggl\langle \sum_{i=1}^n u_ib_i, \sum_{k=1}^n v_kb_k\biggr\rangle
=
\sum_{i=1}^n
\sum_{k=1}^n
u_i \underbrace{\langle b_i,b_k\rangle}_{\displaystyle = g_{ik}} v_k.
\end{align*}
Schreibt man die $u_i$ und $v_k$ in einen $n$-dimensionalen Spaltenvektor
und die $g_{ik}$ in eine $n\times n$-Matrix $G$, kann das Skalarprodukt in
Matrixschreibweise als
\[
\langle u,v\rangle
=
\transpose{%
\begin{pmatrix}
u_1\\u_2\\\vdots\\u_n
\end{pmatrix}}
G
\begin{pmatrix}
v_1\\v_2\\\vdots\\v_n
\end{pmatrix}
\]
geschrieben werden.
Die Matrix $G$ heisst die Matrix des Skalarproduktes in der Basis
oder auch die {\em Gram-Matrix}.
\index{Gram-Matrix}%
Die Eigenschaften des Skalarproduktes schlagen sich in Eigenschaften
dieser Matrix nieder.
Die Symmetrie des Skalarproduktes hat zum Beispiel zur Folge, dass
\[
g_{ik} = \langle b_i,b_k\rangle = \langle b_k,b_i\rangle = g_{ki},
\]
die Matrix $G$ ist also symmetrisch.

%
% Norm
%
\subsection{Norm}
In der Vektorgeometrie wird das Skalarprodukt auch dazu verwendet,
mit $|\vec{v}|^2 = \vec{v}\cdot\vec{v}$ 
die Länge eines Vektors zu berechnen.
Jeder Vektor $\ne 0$ hat eine positive Länge.
Für eine beliebige Bilinearform ist jedoch nicht automatisch
sichergestellt, dass $b(u,u)\ne 0$ ist für $u\ne 0$.
Ausserdem kann ein Längenbegriff nur dann aus $b$ abgeleitet werden,
wenn zusätzlich $b(u,u)>0$ ist für $u\ne 0$, da sich andernfalls
die Wurzel nicht ziehen lässt.

\begin{beispiel}
\label{buch:skalarprodukt:definition:bsp:hyperbolisch}
Die symmetrische Bilinearform
\[
b(x,y)
=
x_1y_1-x_2y_2
\]
auf $\mathbb{R}^2$ ist nicht dazu geeignet, eine Länge zu definieren,
dann der zweite Standardbasisvektor $e_2$ hat das Produkt
$b(e_2,e_2) = -1$.
Auch gibt es einen Vektor, der ``Länge'' 0 hat,
nämlich $v=e_1+e_2$ mit
\[
b(v,v)
=
b(e_1+e_2,e_1+e_2)
=
b(e_1,e_1) + 2\underbrace{b(e_1,e_2)}_{\displaystyle =0} + b(e_2,e_2)
=
1-1
=
0.
\qedhere
\]
\end{beispiel}

\begin{definition}
Eine symmetrische Bilinearform $\langle\;\,,\;\rangle$
heisst {\em positiv definit}, wenn $\langle u,u\rangle > 0$ 
für alle von 0 verschiedenen Vektoren $u\ne 0$ gilt.
\end{definition}

\begin{definition}
\label{buch:skalarprodukt:definition:def:skalarprodukt}
Eine {\em Skalarprodukt} ist eine positiv definite, symmetrische Bilinearform.
\index{Skalarprodukt}%
\end{definition}

%
% kreise.tex -- Kreise von skalarprodukten
%
% (c) 2021 Prof Dr Andreas Müller, OST Ostschweizer Fachhochschule
%
\documentclass[tikz]{standalone}
\usepackage{amsmath}
\usepackage{times}
\usepackage{txfonts}
\usepackage{pgfplots}
\usepackage{csvsimple}
\usetikzlibrary{arrows,intersections,math}
\begin{document}
\def\skala{1}
\def\l{2.1}
\def\axes{
\draw[->] (-\l,0) -- (\l,0) coordinate[label={$x_1$}];
\draw[->] (0,-\l) -- (0,\l) coordinate[label={$x_1$}];
}
\begin{tikzpicture}[>=latex,thick,scale=\skala]

\begin{scope}[xshift=-4.5cm]
\axes
\draw[color=red,line width=1.4pt] (0,0) circle[radius=1.5];
\draw (1.5,-0.05) -- (1.5,0.05);
\draw (-1.5,-0.05) -- (-1.5,0.05);
\draw (-0.05,1.5) -- (0.05,1.5);
\draw (-0.05,-1.5) -- (0.05,-1.5);
\node at (1.5,0) [below right] {$1$};
\node at (-1.5,0) [below left] {$-1$};
\node at (0,1.5) [above left] {$1$};
\node at (0,-1.5) [below left] {$-1$};
\node at (0,{-\l-0.3}) {$x_1^2+x_2^2=1\mathstrut$};
\end{scope}

\begin{scope}
\axes
\def\s{1.4142}
\begin{scope}[rotate=45]
\draw[color=red,line width=1.4pt] (0,0) ellipse (1cm and 2cm);
\end{scope}
\draw (\s,-0.05) -- (\s,0.05);
\draw (-\s,-0.05) -- (-\s,0.05);
\draw (-0.05,\s) -- (0.05,\s);
\draw (-0.05,-\s) -- (0.05,-\s);
\node at (\s,0) [below right] {$1$};
\node at (-\s,0) [below left] {$-1$};
\node at (0,\s) [above right] {$1$};
\node at (0,-\s) [below left] {$-1$};
\draw[line width=0.3pt] (\l,-\l) -- (-\l,\l);
\draw[line width=0.3pt] (-\l,-\l) -- (\l,\l);
\node at (0,{-\l-0.3}) {$5x_1^2+6x_1x_2+5x_2^2=4\mathstrut$};
\fill[color=red] ({0.5*\s},{0.5*\s}) circle[radius=0.06];
\fill[color=red] ({-0.5*\s},{-0.5*\s}) circle[radius=0.06];
\fill[color=red] ({\s},{-\s}) circle[radius=0.06];
\fill[color=red] ({-\s},{\s}) circle[radius=0.06];
\end{scope}

\begin{scope}[xshift=4.5cm]
\axes
\draw[color=red,line width=1.4pt]
	plot[domain=-1.4:1.4,samples=20]
		({cosh(\x)},{sinh(\x)});
\draw[color=red,line width=1.4pt]
	plot[domain=-1.4:1.4,samples=20]
		({-cosh(\x)},{sinh(\x)});
\draw (1,-0.05) -- (1,0.05);
\draw (-1,-0.05) -- (-1,0.05);
\draw (-0.05,1) -- (0.05,1);
\draw (-0.05,-1) -- (0.05,-1);
\node at (0,1) [above left] {$1$};
\node at (0,-1) [below left] {$-1$};
\node at (1,0) [above left] {$1$};
\node at (-1,0) [above left] {$-1$};
\node at (0,{-\l-0.3}) {$x_1^2-x_2^2=1\mathstrut$};
\end{scope}

\end{tikzpicture}
\end{document}

%

Verschiedene Skalarprodukte in zwei Dimensionen kann man durch ihre
zugehörten ``Kreise'' visualisieren
(Abbildung~\ref{buch:skalarprodukt:definition:fig:kreise}).
Damit meinen wir die Menge der Vektoren, die Norm $1$ haben.
Abbildung~\ref{buch:skalarprodukt:definition:fig:kreise} zeigt
die Kreise für drei verschiedene Skalarprodukte.
Links ist der Kreis für das Standard\-skalarprodukt $x_1y_1+x_2y_2$
dargestellt, der tatsächlich die Form eines Kreises hat.
Das Skalarprodukt von
Beispiel~\ref{buch:skalarprodukt:definition:bsp:hyperbolisch}
führt auf die Kurven mit der Gleichung
\[
x_1^2-x_2^2=1,
\]
dies sind Hyperbeln.
Es heisst daher auch das hyperbolische Skalarprodukt, obwohl es
\index{Skalarprodukt!hyperbolisch}
im strengen Sinne der
Definition~\ref{buch:skalarprodukt:definition:def:skalarprodukt}
kein Skalarprodukt ist.
In der Mitte schliesslich ist der Kreis für das Skalarprodukt 
\[
\langle x,y\rangle
=
2
\cdot
\frac{x_1+x_2}{\!\sqrt{2}}
\cdot
\frac{y_1+y_2}{\!\sqrt{2}}
+
\frac12
\cdot
\frac{x_1-x_2}{\!\sqrt{2}}
\cdot
\frac{y_1-y_2}{\!\sqrt{2}}
=
\frac{5x_1y_1 + 3x_1y_2 + 3x_2y_1 + 5x_2y_2}{2}
\]
dargestellt.
Die zugehörige Norm ist
\[
\|x\|^2
=
\frac14(
5x_1^2 + 5x_2^2 + 6x_1x_2).
\]
Der Kreis bestehen aus den Punkten einer Ellipse mit den Halbachsen
$1$ und $\frac12$, die um $45^\circ$ gegenüber den Koordinatenachsen
verdreht sind.
Diese Ellipse verläuft durch die Punkte $(\pm\frac12,\pm\frac12)$ und
$(\pm1,\mp1)$.

Für die Gram-Matrix $G$ eines Skalarproduktes bedeutet die Forderung,
dass das Skalarprodukt bilinear ist, dass die Matrix $G$ positiv
definit sein muss.
Aus der linearen Algebra ist bekannt, dass $G$ genau dann positiv
definit ist, wenn $G$ eine Cholesky-Zerlegung $G=L\transpose{L}$ hat,
deren Diagonalelemente alle positiv sind.

Ein Skalarprodukt hat jetzt alle Eigenschaften, die erlauben, einen 
Abstandsbegriff zu definieren.

\begin{definition}
\label{buch:skalarprodukt:definition:def:norm}
Die zu einem Skalarprodukt $\langle\;\,,\;\rangle$ gehörige Norm ist
definiert als
\[
\| v\|
=
\!\sqrt{\langle v,v\rangle}
\]
für $v\in V$.
\end{definition}

Die Norm erfüllt $\|\lambda v\| = |\lambda|\,\|v\|$ für jeden Vektor
$v\in V$ und $\lambda\in\mathbb{R}$.
In Worten bedeutet dies, dass bei der Skalierung eines Vektors die Norm
auf die gleiche Art skaliert.

%
% Sesquilineare Funktionen
%
\subsection{Sesquilineare Funktionen}
Sei jetzt $V$ ein komplexer Vektorraum.
Aus einer bilinearen Funktion
\[
b\colon V\times V \to \mathbb{C} : (u,v) \mapsto b(u,v)
\]
auf $V$ kann jedoch keine brauchbare Norm abgeleitet werden.
Eine solche müsste $\| v\|=b(v,v)\ge 0$ erfüllen.
Selbst wenn $b(u,u)> 0$ ist für einen speziellen Vektor $u\in V$,
ist das Skalarprodukt von $iu$ mit sich selbst
\[
b(iu,iu)
=
i^2 b(u,u)
=
-b(u,u)
<
0.
\]
Da $|i|=1$ ist, würde man eher erwarten, dass $iu$ die gleiche 
Länge hat wie $u$, dass also $b(iu,iu)=b(u,u)$.
Bilinearität funktioniert also nicht als Bedingung, um ein Skalarprodukt
zu konstruieren, für welches auch die geometrische Intuition des Abstands
anwendbar bleibt.

\begin{definition}
Eine Funktion $f\colon V\to U$ zwischen komplexen Vektorräumen 
heisst {\em konjugiert linear}, wenn 
\[
f(\lambda u + \mu v)
=
\overline{\lambda} f(u) + \overline{\mu} (v)
\]
ist für alle $u,v\in V$ und $\lambda,\mu\in \mathbb{C}$.
\end{definition}

Im obengenannten Beispiel wird $b(iu,iu)>0$, wenn $b$ im ersten Faktor
konjugiert linear ist.
Dann ist nämlich $b(iu,iu) = -ib(u,iu) = -i^2 b(u,u) = b(u,u)>0$.

\begin{definition}
Eine Funktion
\[
\langle\;\,,\;\rangle
\colon
V\times V \to \mathbb{C}
:
(u,v) \mapsto \langle u,v\rangle
\]
heisst {\em sesquilinear} oder {\em Sesquilinearform}
wenn sie linear ist im zweiten Argument
\end{definition}

Das lateinische Wort {\em sesqui} bedeudet eineinhalb, eine
sesquilineare Funktion ist linear in einem Faktor, aber nur
halb linear im anderen.

\begin{beispiel}
Die Form
\[
\langle u,v\rangle = \sum_{i=1}^n \overline{u}_i v_i
\]
mit $u,v\in \mathbb{C}^n$ ist sesquilinear.
Tatsächlich gilt
\begin{align*}
\langle u,\lambda v+\mu w\rangle
&=
\sum_{i=1}^n \overline{u}_i (\lambda v_i+\mu w_i)
=
\lambda
\sum_{i=1}^n \overline{u}_i v_i
+
\mu
\sum_{i=1}^n \overline{u}_i w_i
=
\lambda\langle u,v\rangle
+
\mu\langle u,w\rangle
\\
\langle \lambda u+\mu w, v\rangle
&=
\sum_{i=1}^n \overline{(\lambda u_i+\mu w_i)}v_i
=
\overline{\lambda}
\sum_{i=1}^n \overline{u}_i v_i
+
\overline{\mu}
\sum_{i=1}^n \overline{w}_iv_i
=
\overline{\lambda}\langle u,v\rangle
+
\overline{\mu}\langle w,v\rangle.
\qedhere
\end{align*}
\end{beispiel}

Eine Sesquilinearform auf einem endlichdimensionalen Vektorraum $V$
kann wie im reellen Fall mit Hilfe einer Matrix beschrieben werden.
Eine Darstellung der Vektoren $u$ und $v$ mit Koordinaten $u_i$ und
$v_i$ in der Basis $b_1,\dots,b_n\in V$ führt auf das Skalarprodukt
\[
f(u,v)
=
f\biggl( \sum_{i=1}^n u_ib_i, \sum_{k=1}^n v_kb_k \biggr)
=
\sum_{i=1}^n\sum_{k=1}^n
\overline{u}_i
\underbrace{f(b_i, b_k)}_{\displaystyle=h_{ik}}
v_k.
\]
In Matrixform mit der Matrix $H$ mit Einträgen $h_{ik}$ ist dies
gleichbedeutend mit
\[
f(u,v)
=
\transpose{
\begin{pmatrix}
\overline{u}_1\\
\overline{u}_2\\
\vdots\\
\overline{u}_n
\end{pmatrix}
}
H
\begin{pmatrix}
v_1\\v_2\\\vdots\\v_n
\end{pmatrix}.
\]
Der transponierte Vektor $\transpose{\overline{u}}$ mit komplex
konjugierten Einträgen heisst auch konjugiert transponiert zu $u$.

\begin{definition}
Ist $A\in M_{m\times n}(\mathbb{C})$ eine komplexe $m\times n$-Matrix.
Die Matrix $\overline{A}\in M_{m\times n}(\mathbb{C})$ mit 
den komplex konjugierten Matrixelementen von $A$ heisst die
(komplex) konjugierte Matrix zu $A$.
\index{komplex konjugierte Matrix}%
\index{konjugierte Matrix}%
Die Matrix 
\[
A^*
=
\transpose{\overline{A}}
=
\transpose{
\begin{pmatrix}
\overline{a_{11}}&\overline{a_{12}}&\dots &\overline{a_{1n}}\\
\overline{a_{21}}&\overline{a_{22}}&\dots &\overline{a_{2n}}\\
\vdots           &\vdots           &\ddots&\vdots           \\
\overline{a_{m1}}&\overline{a_{m2}}&\dots &\overline{a_{mn}}
\end{pmatrix}
}
=
\begin{pmatrix}
\overline{a_{11}}&\overline{a_{21}}&\dots &\overline{a_{n1}}\\
\overline{a_{12}}&\overline{a_{22}}&\dots &\overline{a_{n2}}\\
\vdots           &\vdots           &\ddots&\vdots           \\
\overline{a_{1m}}&\overline{a_{2m}}&\dots &\overline{a_{nm}}
\end{pmatrix}
\in
M_{n\times m}(\mathbb{C})
\]
heisst die {\em transponiert konjugierte} oder {\em hermitesch konjugierte}
\index{hermitesch konjugiert}%
Matrix.
\end{definition}

Eine Sesquilinearform kann immer geschrieben werden als
\(
f(u,v) = u^*Hv
\)
mit einer hermiteschen Matrix $H$.

%
% Hermitesche Formen
%
\subsection{Hermitesche Formen}
Damit aus einer sesquilinearen Funktion eine Norm abgeleitet werden
kann, muss das Produkt $\langle u,u\rangle$ für jeden Vektor $u\in V$
eine reelle Zahl sein.
Selbst für die sesquilineare Funktion
\[
\langle\;\,,\;\rangle
\colon
\mathbb{C}\times\mathbb{C}
\to
\mathbb{C}
:
(u,v) \mapsto i\overline{u}v
\]
ist dies jedoch nicht der Fall, da $\langle 1,1\rangle = i\not\in\mathbb{R}$
ist.
Die folgende Eigenschaft kann aber garantieren, dass
$\langle u,u\rangle\in\mathbb{R}$.

\begin{definition}
Eine sesquilinear Funktion 
\[
\langle \;\,,\;\rangle
\colon
V\times V
\to
\mathbb{C}
\]
heisst in {\em konjugiert symmetrisch} oder {\em hermitesch}, wenn
\index{konjugiert symmetrisch}%
\index{hermitesch!Sesquilinearform}%
\[
\langle u,v\rangle = \overline{\langle v,u\rangle}
\]
für alle $u,v\in V$ gilt.
\end{definition}

\begin{beispiel}
Die Standard-Sesquilinearform
\[
\langle u,v\rangle
=
\sum_{i=1}^n \overline{u}_i v_i
\]
auf $V=\mathbb{C}^n$ ist konjugiert symmetrisch, denn
\begin{align*}
\langle u,v\rangle
&=
\sum_{i=1}^n \overline{u}_i v_i
=
\overline{
\sum_{i=1}^n u_i \overline{v}_i
}
=
\overline{\langle v,u\rangle}.
\qedhere
\end{align*}
\end{beispiel}

Die Gram-Matrix einer hermitesche Sesquilinearform hat die Matrix-Elemente
\[
h_{ik}
=
\langle b_i,b_k\rangle
=
\overline{\langle b_k,b_i\rangle}
=
\overline{h}_{ki}
\qquad\Rightarrow\qquad
H^* = H.
\]
Man sagt auch, die Matrix $H$ ist {\em hermitesch}.
\index{hermitesch!Matrix}%

%
% Komplexe Skalarprodukte
%
\subsection{Komplexe Skalarprodukte}
Wie bei einem reellen Skalarprodukt reichen auch im Fall eines
komplexen Vektorraums die Eigenschaften der Sesquilinearität
und der hermiteschen Symmetrie nicht aus, ein sinnvolles
Skalarprodukt zu definieren.

\begin{definition}
Eine hermitesche Sesquilinearform $\langle\;\,,\;\rangle$
auf dem komplexen Vektorraum $V$ heisst {\em positiv definit}, wenn
\[
\langle u,u\rangle > 0
\]
gilt für alle $u\ne 0$ in $V$.
Ein {\em komplexes Skalarprodukt} ist eine positiv definite hermitesche
Sesquilinearform.
Die zugehörige {\em Norm} eines Vektors ist
$\|v\| = \!\sqrt{\langle u, u\rangle}$.
\end{definition}

Auch für ein komplexes Skalarprodukt gilt die Skalierungseigenschaft
\[
\|\lambda v\|^2
=
\langle \lambda v,\lambda v\rangle
=
\overline{\lambda}\lambda\langle v,v\rangle
=
|\lambda|^2\,\|v\|^2
\qquad\Rightarrow\qquad
\|\lambda v\|
=
|\lambda|\, \|v\|,
\]
ganz analog zur entsprechenden Skalierungseigenschaft für ein
reelles Skalarprodukt.






%
% 2-cauchyschwarz.tex
%
% (c) 2022 Prof Dr Andreas Müller, OST Ostschweizer Fachhochschule
%
\section{Cauchy-Schwarz Ungleichung
\label{buch:skalarprodukte:section:cauchyschwarz}}
\kopfrechts{Cauchy-Schwarz-Ungleichung}

%
% 3-funktionenraeume.tex
%
% (c) 2022 Prof Dr Andreas Müller, OST Ostschweizer Fachhochschule
%
\section{Funktionenräume
\label{buch:skalarprodukt:section:funktionenraeume}}
\kopfrechts{Funktionenräume}
Ziel der harmonischen Analysis ist die effiziente Approximation einer
grossen Klasse von Funktionen durch einfach zu berechnende Funktionen
oder durch Funktionen mit besonderen Eigenschaften.
Als approximierende Funktionen kommen stetige Funktionen, Polynome,
trigonometrische Polynome oder eine ähnlich, einfach konstruierbare
Funktionenfamilie in Frage.
Es gilt zunächst herauszufinden, was ``Approximation'' genau heissen
soll und von welchen Funktionen man überhaupt erwarten kann, dass sie
approximiert werden können.

%
% Stetige Funktionen
%
\subsection{Stetige Funktionen
\label{buch:skalarprodukt:subsection:stetige-funktionen}}
Der frühe intuitive Funktionsbegriff ging oft von der Vorstellung einer
in einem Strich gezeichneten Kurve aus, wie man sie von den Graphen
der Polynome oder der trigonometrischen Funktionen her kennt.
In moderner Sprechweise sind dies die stetigen Funktionen.

\begin{definition}
Eine Funktion $f\colon I\to\mathbb{R}$ mit $I\subset \mathbb{R}$
heisst stetig in einem Punkt $x_0\in I$, wenn für jedes $\varepsilon>0$
ein $\delta>0$ existiert derart, dass $f(x)-f(x_0)|<\delta$ sobald
$|x-x_0|<\varepsilon$.
\end{definition}

Nur die Eigenschaft, eine Abstandsmessung zu besitzen, wird vom
Definitionsbereich $I\subset \mathbb{R}$ verlangt.
Der Stetigkeitsbegriff kann daher verallgemeinert werden auf den
Begriff des metrischen Raumes.

\begin{definition}
Eine {\em Metrik} auf einer Menge $X$ ist eine Funktion
\index{Metrik}%
$d\colon X\times X\to \mathbb{R}$
mit den folgenden Eigenschaften
\begin{enumerate}
\item
Positiv definit: $d(x,y)\ge 0$ und $d(x,y)$ genau dann, wenn $x=y$.
\item
Symmetrie: \(d(x,y)=d(y,x)\)
\item
Dreiecksungleichung: \( d(x,y) \le d(x,z) + d(z,y) \).
\end{enumerate}
Ein {\em metrischer Raum} ist ein Menge $X$ mit einer Metrik.
\index{metrischer Raum}%
\end{definition}

In einem metrischen Raum ist der Begriff des Grenzwertes übertragbar.
Da Stetigkeit von Funktionen mit Grenzwerten definiert werden kann,
lässt sich sich so auch der Begriff der Stetigkeit auf Abbildungen
zwischen beliebigen metrischen Räumen ausdehnen.

\begin{definition}
Ist $x_n\in X$ eine Folge von Punkten in einem metrischen Raum $X$,
dann heisst $x$ der Grenzwert der Folge $x_n$, wenn es für jedes
$\varepsilon>0$ ein $N>0$ gibt derart, dass
$d(x_n,x)\le \varepsilon$ für alle $n>N$.
Eine Funktion $f\colon X\to Y$ zwischen metrischen Räumen heisst
stetig im Punkt $x\in X$, wenn für jede Folge $x_n\in X$ mit
Grenzwert $x$ auch die Folge $y_n=f(x_n)\in Y$ konvergiert und
den Grenzwert $y=f(x)$ hat.
\end{definition}

Teilmengen von $\mathbb{R}$ oder $\mathbb{R}^n$ tragen natürlich
die Struktur eines metrischen Raumes mit der Abstandsmessung in 
$\mathbb{R}^n$ als Metrik
\[
d(x,y) = \!\sqrt{(x_1-y_1)^2 + \ldots + (x_n-y_n)^2} = \|x-y\|.
\]
Die Eigenschaften einer Metrik wurden bereits in Abschnitt
\ref{buch:skalarprodukte:section:cauchyschwarz} nachgewiesen.

Der Begriff des Grenzwertes klärt, was mit der Approximation von $x$
durch eine Folge $x_n$ gemeint ist.
Wenn man darauf aufbauend die Konvergenz einer Folge von Funktionen
gegen eine Grenzfunktion definieren will, braucht man einen Abstansbegriff
zwischen Funktionen.
Ein erster Versuch könnte sein, als Abstand zwischen zwei Funktionen
$f$ und $g$ die Funktion
\[
d(f,g) = |f(x_0) - g(x_0)|.
\]
Die Menge der Funktionen wird dadurch jedoch nicht zu einem metrischen
Raum.
Zwar gilt sicher die Symmetrie und Dreiecksungleichung, und auch 
$d(f,g)\ge 0$ für beliebige Funktionen.
Aber wenn $d(f,g)=0$ ist, heisst das nur, dass $f$ und $g$ im Punkt
$x_0$ den gleichen Wert haben.
Ausser in trivialen Fällen wird es Funktionen geben, die zwar im Punkt
$x_0$ übereinstimmen, sich aber in mindestens einem anderen Punkt
unterscheiden.
Es müssen also alle Funktionswerte in die Definition eines Abstands
berücksichtig werden.
Für Vektorräume, mit denen wir hier in erster Linie zu tun habe, 
müssen aber auch noch die algebraischen Eigenschaften einfliessen.

Der Begriff der Norm von
Definition~\ref{buch:skalarprodukt:cauchyschwarz:def:norm}
trägt diesen Aspekten Rechnung.
Dies reicht aber noch nicht, denn wir möchten auch sicherstellen,
dass Grenzwerte existieren.
Der im nächsten Abschnitt eingeführte Begriff der Vollständigkeit
erreicht dies.
Abschnitt~\ref{buch:skalarprodukt:subsection:normfuerstetigefunktionen}
zeigt, wie die Supremum-Norm für stetige Funktionen auf den Begriff
der gleichmässigen Konvergenz führt.

\begin{definition}
Ein reeller oder komplexer Vektorraum mit einer Norm heisst ein
{\em normierter Raum}.
\end{definition}

%Für stetige Funktionen bietet sich daher die Supremum-Norm
%\[
%\|f\|_\infty = \sup_{x\in X} |f(x)|
%\]
%als Norm an.
%Eine Folge $(f_n)$ von Funktionen, die in dieser Norm gegen die
%Grenzfunktion $f$ konvergiert, hat die Eigenschaft, dass für jedes
%$\varepsilon>0$ ein $N>0$ existiert, so dass
%\[
%\|f_n-f\|_\infty \varepsilon\quad\forall\; n>N
%\]
%gilt.
%Aus der Definition der Supremum-Norm folgt dann auch
%\[
%|f_n(x)-f(x)|<\varepsilon\quad\forall\; n>N
%\]
%ist.
%Dies ist die Definition der gleichmässigen Konvergenz einer Funktionenfolge.
%Konvergenz in der Supremum-Norm ist daher die gleichmässige Konvergenz
%von Funktionenfolgen.

%
% Vollständigkeit
%
\subsubsection{Vollständigkeit}
In den rationalen Zahlen hat nicht jede Folge einen Grenzwert.
Die Zahl $\!\sqrt{2}$ lässt sich beliebig genau durch rationale Zahlen
approximieren, sie ist aber nicht in $\mathbb{Q}$.
Ähnlich lässt sich die Funktion $x\mapsto \!\sqrt{x}$ beliebig genau 
durch Polyome approximieren, sie ist aber selbst kein Poylnome

\begin{definition}
Ein Folge $x_n\in X$ in einem metrischen Raum heisst {\em Cauchy-Folge},
\index{Cauchy-Folge}%
wenn es für jedes $\varepsilon>0$ ein $N>0$ gibt derart, dass 
$|x_n-x_m|<\varepsilon$ wenn $n,m>N$ ist.
\end{definition}

Cauchy-Folgen sind also Folgen, die sich für genügend grossen Index
kaum mehr ändern und für die man daher Konvergenz erwarten würde.

\begin{definition}
Ein normierter Raum heisst {\em vollständig} oder ein Banach-Raum,
wenn jede Cauchy-Folge einen Grenzwert hat.
\index{vollständig}%
\end{definition}

Die rationalen Zahlen $\mathbb{Q}$ bilden keinen vollständigen
metrischen Raum, aber die reellen Zahlen $\mathbb{R}$ enthalten
alle Grenzwerte von Cauchy-Folgen, $\mathbb{R}$ ist eine vollständiger
metrischer Raum.
Die Menge der Polynome, betrachtet als Teilmenge der Menge der
stetigen Funktionen $[0,1]\to\mathbb{R}$ ist nicht vollständig,
da es eine Folge $f_n(x)$ von Approximationsfunktionen der Funktion
$x\mapsto \!\sqrt{x}$ gibt.
Als Cauchy-Folge konvergiert sie zwar gegen eine stetige Funktion,
aber die Grenzfunktion ist nicht mehr im Raum der Polynome.

Das Ziel der folgenden Kapitel ist also, zu geeignet interessanten
Funktionenfamilien ``gute'' Normen zu finden derart, dass Cauchy-Folgen
konvergieren gegen Funktionen, die immer noch ausreichend viele
nützliche Eigenschaften haben.
Im besten Fall konvergieren stetige Funktionen gegen stetige Funktionen,
es wird sich aber zeigen, dass diese Anforderung zu streng ist.

%
% Norm für stetige Funktionen
%
\subsection{Norm für stetige Funktionen
\label{buch:skalarprodukt:subsection:normfuerstetigefunktionen}}
Damit man von Konvergenz von Folgen stetiger Funktionen sprechen kann,
brauchen wir jetzt also eine Norm für stetige Funktionen.

\begin{definition}
Sei $X$ ein metrischer Raum und
\[
C(X)
=
C_{\mathbb{R}}(X)
=
\{
f\colon X \to\mathbb{R}\mid
\text{$f$ ist stetig}
\}
\]
der Vektorraum der stetigen Funktion auf $X$.
Die Norm von $C(X)$ ist definiert als
\[
\|f\| = \sup_{x\in X} |f(x)|.
\]
Sie heisst die {\em Supremum-Norm}.
\end{definition}

Wir prüfen nach, dass die Supremum-Norm tatsächlich eine Norm ist.
Dazu sind die definierenden Eigenschaften nachzurechnen:
\begin{enumerate}
\item Definit: 
\[
0
=
\|f\|
=
\sup_{x\in X} |f(x)|
\quad\Rightarrow\quad
f(x)=0 \;\forall x\in X
\quad\Rightarrow\quad
f\in C(X).
\]
\item Homogeneität:
\[
\|\lambda f\|
=
\sup_{x\in X} |\lambda f(x)|
=
|\lambda| \sup_{x\in X} |f(x)|
=
|\lambda| \cdot \|f\|.
\]
\item
Dreiecksungleichung:
\[
\|f+g\|
=
\sup_{x\in X}|f(x)+g(x)|
\le
\sup_{x\in X}(|f(x)|+|g(x)|)
\le
\sup_{x\in X}|f(x)|+\sup_{x\in X}|g(x)|
=
\|f\| + \|g\|.
\]
\end{enumerate}

Eine Cauchy-Folge $f_n$ von Funktionen $X\to \mathbb{R}$ hat die
Eigenschaft, dass für jedes $\varepsilon >0$ ein $N>0$ existiert
derart, dass $\|f_n-f_m\|<\varepsilon$ ist.
Da die Norm der maximale Unterschied von Funktionswerten ist,
folgt dass für eine Cauchy-Folge in $C(X)$ die Folge $f_n(x)$ eine
Cauchy-Folge in $\mathbb{R}$ ist und damit einen Grenzwert in $\mathbb{R}$
hat.
Die Funktion $f(x) = \lim_{n\to\infty}f_n(x)$ ist die Grenzfunktion.
Die Konvergenz bezüglich der Norm besagt, dass für jedes $\varepsilon>0$
es ein $N>0$ gibt derart, dass
\[
\varepsilon 
>
\|f_n-f\|
\ge 
|f_n(x)-f(x)|
\]
ist für alle $n>N$ und unabhängig von $x\in X$.
Die Konvergenz bezüglich der $\normfunc$-Norm ist also die wohlbekannte
{\em gleichmässige Konvergenz}.
Es kann gezeigt werden, dass die Grenzfunktion wieder stetig ist.

\begin{satz}
Der Raum der stetigen Funktion $C(X)$ mit der Supremum-Norm ist
ein Banach-Raum.
\end{satz}

%
% Skalarprodukt
%
\subsection{Skalarprodukt
\label{buch:skalarprodukt:subsection:skalarprodukt}}
\begin{figure}
\centering
\includegraphics{chapters/010-skalarprodukt/images/fourierrechteck.pdf}
\caption{Approximation der Rechteckfunktion (rot) durch eine Folge
von Partialsummen der Fourier-Reihe.
\label{buch:skalarprodukt:fig:fourierrechteck}}
\end{figure}%
Die Supremum-Norm auf dem Raum der stetigen Funktionen hat den
Begriff der gleichmässig konvergenten Funktionenfolgen ergeben.
Cauchy-Folgen von stetigen Funktionen in der Supremum-Norm konvergieren
wieder gegen eine stetige Funktione.
Ist eine Funktion nicht stetig, lässt Sie sich im Sinne der Supremum-Norm
nicht durch stetige Funktionen approximieren.
Andererseits hat Fourier gezeigt, wie man technische wichtige Funktionen
wie die Rechteckfunktion durch trigonometrische Polynome
\begin{equation}
f_n(x)
=
\frac{4}{\pi} \sum_{k=0}^n \frac{\sin kx}{k}
=
\frac{4}{\pi} \biggl(
\sin x
+
\frac{\sin 3x}{3}
+
\frac{\sin 5x}{5}
+
\frac{\sin 7x}{7}
+
\ldots
\biggr)
\label{buch:skalarprodukt:eqn:rechteckreihe}
\end{equation}
approximieren kann.
Diese sind alle stetig und kommen der Rechteckfunktion in jedem Punkt,
in dem die Funktion stetig ist, beliebig nahe.
An den Stellen $x = n\pi$ hat die Grenzfunktion eine Sprungstelle,
die approximierenden Funktionen haben dort immer Abstand $1$
(siehe Abbildung~\ref{buch:skalarprodukt:fig:fourierrechteck}).
Die Folge ist also keine Cauchy-Folge und sie konvergiert nicht im
Sinne der Supremum-Norm.
Für solche Anwendungen muss eine besser geeignete Norm gefunden werden,
in der die Folge konvergiert.

%
% Skalarprodukt von Funktion
%
\subsubsection{Die $L^1$-Norm einer Funktion}
Die Supremum-Norm sieht nur den grössten Wert, die Konvergenz der Folge
\eqref{buch:skalarprodukt:eqn:rechteckreihe} ist aber nicht gleichmässig,
die maximale Abweichung ist immer $1$.
Gesucht ist eine Norm, die für die Folge
\eqref{buch:skalarprodukt:eqn:rechteckreihe} 
nur im Mittel eine Abweichung feststellt.
Für die Berechnung des Mittelwerts kann das Integral verwendet werden:

\begin{definition}
\label{buch:skalaprodukt:definition:l1norm}
Für eine stetige Funktion $X\to\mathbb{R}$, für die $x\mapsto |f(x)|$
integrierbar ist, heisst
\begin{equation}
\|f\|_1 = \int_X |f(x)|\,dx
\label{buch:skalarprodukt:eqn:l1norm}
\end{equation}
die {\em $L^1$-Norm} der Funktion $f$.
\end{definition}

Die $L^1$-Norm ist tatsächlich eine Norm, wir verifizieren die
definierenden Eigenschaften einer Norm.
\begin{enumerate}
\item
Positiv definit: Sei $f$ eine stetige Funktion mit $\|f\|_1=0$
Wäre $f\ne 0$, dann gäbe es einen Punkt $x_0\in X$ mit $f(x_0) \ne 0$.
Da $f$ stetig ist, ist $f|(x)| > \frac12|f(x_0)|$ für $x$ in einer
$\delta$-Umgebung von $x_0$.
Dann folgt für die $L^1$-Norm
\begin{align*}
\|f\|_1
=
\int_X |f(x)|\,dx
\ge
\frac12 |f(x_0)| \cdot \delta 
> 0.
\end{align*}
Dies widerspricht der Annahme, dass $\|f\|_1=0$ ist, also muss $f=0$ sein.
\item
Homogeneität folgt durch direkte Rechnung
\[
\|\lambda f\|_1
=
\int_X |\lambda f(x)|\,dx
=
|\lambda|
\int_X |f(x)|\,dx
=
|\lambda| \cdot \|f\|.
\]
\item
Die Dreiecksungleichung folgt aus
\begin{align*}
\|f+g\|_1
&=
\int_X |f(x) + g(x)|\,dx
\\
&\le
\int_X |f(x)| + |g(x)|\,dx
=
\int_X |f(x)|\,dx + \int_X |g(x)|\,dx
=
\|f\|_1 + \|g\|_1.
\end{align*}
\end{enumerate}
Die $L^1$-Norm ist etwas ``schwächer'' als die Supremum-Norm im
folgenden Sinne.
Eine in der Supremum-Norm konvergente Funktionenfolge auf einem
kompakten Definitionsbereich $X$ ist auch in der $L^1$-Norm konvergent.
Zur Unterscheidung der verschiedenen Normen werden wir in Zukunft die
Supremum-Norm manchmal auch als $\|f\|_{\infty} = \|f\|$ schreiben.

\begin{satz}
Ist $X$ eine kompakte Teilmenge von $\mathbb{R}$ und $f_n$ eine
in der Supremum-Norm konvergente Folge stetiger Funktionen $f_n$,
dann ist $f_n$ auch in der $L^1$-Norm konvergent.
\end{satz}

\begin{proof}
Konvergenz in der Supremum-Norm bedeutet, dass für jedes $\varepsilon>0$
ein $N>0$ existiert derart, dass $|f_n(x)-f(x)|<\varepsilon$ für alle
$x\in X$ und alle $n>N$.
Für die $L^1$-Norm gilt dann
\begin{align*}
\|f_n-f\|_1
&=
\int_X |f_n(x) - f(x)|\,dx
\le
\int_X \varepsilon \,dx
=
\varepsilon \int_X \,dx
=
\varepsilon \operatorname{vol}(X).
\end{align*}
Da für einen kompakten Definitionsbereich $\operatorname{vol}(X)<\infty$
gilt, bedeutet dies, dass die $\|f_n-f\|_1\to 0$, dass also $f_n$ in
der $L^1$-Norm konvergiert.
\end{proof}

%
% l1konvergenz.tex
%
% (c) 2023 Prof Dr Andreas Müller
%
\begin{figure}
\centering
\includegraphics{chapters/010-skalarprodukt/images/l1konvergenz.pdf}
\caption{Konvergenz der Folge $f_n$ von
\eqref{buch:skalarprodukt:eqn:rechteckreihe} in der $L^1$-Norm.
Die $L^1$-Norm von $f_n-f$ ist beschränkt durch die farbige Fläche,
sie kann beliebig klein gemacht werden, indem $\varepsilon$ und
$\delta$ klein gewählt werden.
\label{buch:skalarprodukt:funktionenraeume:fig:l1konvergenz}}
\end{figure}


\begin{beispiel}
Die Folge $f_n(x)$ von \eqref{buch:skalarprodukt:eqn:rechteckreihe}
konvergiert tatsächlich in der $L^1$-Norm auf dem Intervall $[0,2\pi]$.
Zwar ist $f_n$ nicht gleichmässig konvergent, aber fast.
Die Abbildung~\ref{buch:skalarprodukt:funktionenraeume:fig:l1konvergenz}
illustriert das nachfolgende Argument.

Man kann zeigen, dass für jedes $\delta>0$, die Funktionen
$f_n(x)$ in Punkten $x$, die weiter als $\delta$ von den
Punkten $k\pi$ mit $k\in\mathbb{Z}$, gleichmässig konvergieren.
Liegt $x$ innerhalb einer $\delta$-Umgebung der Vielfachen von $\pi$
ist die Differenz $f_n(x)-f(x)$ beschränkt.
Die genaue Schranke ist nicht wichtig, wir nennen sie $M$ und bekommen
\[
|f_n(x)-f(x)|
\le M
\quad\forall x\in X.
\]
Ausserhalb einer kleinen Umgebung konvergiert die Folge gleichmässig,
denn zu jedem $\varepsilon>0$ gibt es also ein $N>0$ derart, dass
\[
|f_n(x)-f(x)|<\varepsilon
\]
für $x$, die weiter als $\delta$ von $k\pi$ entfernt sind.
Für die $L^1$-Norm folgt dann
\begin{align*}
\|f_n-f\|_1
&=
\int_0^{2\pi} |f_n(x)-f(x)|\,dx
\\
&=
\int_0^\delta |f_n(x)-f(x)|\,dx
+
\int_\delta^{\pi-\delta} |f_n(x)-f(x)|\,dx
+
\int_{\pi-\delta}^{\pi+\delta} |f_n(x)-f(x)|\,dx
\\
&\qquad
+
\int_{\pi+\delta}^{2\pi-\delta} |f_n(x)-f(x)|\,dx
+
\int_{2\pi-\delta}^{2\pi} |f_n(x)-f(x)|\,dx
\\
&\le
\delta M
+
\varepsilon (\pi -2\delta)
+
2\delta M
+
\varepsilon (\pi -2\delta)
+
\delta M
\le
4\delta M + 2\pi\varepsilon
\end{align*}
für $n>N$.
Dadurch, dass man $\delta$ und $\varepsilon$ klein macht, kann man
also immer ein $N$ finden, so dass $\|f_n-f\|_1$ beliebig klein wird
für $n>N$.
Damit ist gezeigt, dass die Folge $f_n$ in der $L^1$-Norm konvergiert.
\end{beispiel}

Das Beispiel zeigt, dass die $L^1$-Norm eine schwächere Form der Konvergenz
ist, die eine erweiterte Klasse von Funktionen durch stetige Funktionen
zu approximieren erlaubt.

%
% Das $L^2$-Skalarprodukt
%
\subsubsection{Das $L^2$-Skalarprodukt}
Die $L^1$-Norm ist weniger strikt als die Supremum-Norm, aber sie ist
immer noch recht weit von der Intuition entfernt, die wir von der
Entfernungsmessung in der Geometrie haben, die von einem Skalarprodukt
herrühren.
Das Beispiel~\ref{buch:skalarprodukt:cauchyschwarz:beispiel:skalarprodukt}
weist den Weg, mit dem wir eine Norm für stetige Funktionen gewinnen
können, die von einem Skalarprodukt herkommt.

\begin{definition}
\label{buch:skalarprodukt:funktionraeume:definition:skalarprodukt}
Das {\em Skalarprodukt} stetiger Funktionen auf $X\subset \mathbb{R}$
ist definiert durch
\begin{equation}
\langle f,g\rangle
=
\int_X f(x)g(x)\,dx.
\label{buch:skalarprodukt:funktionraeume:eqn:skalarprodukt}
\end{equation}
\end{definition}

Es genügt nachzurechnen, dass $\langle f,g\rangle$ die Eigenschaften
eines Skalarproduktes hat, dann folgt die Dreiecksungleichung automatisch.
Zunächst ist klar,
dass~\eqref{buch:skalarprodukt:funktionraeume:eqn:skalarprodukt}
bilinear ist:
\begin{align*}
\langle \lambda f_1+\mu f_2,g\rangle
=
\int_X (\lambda f_1(x) + \mu f_2(x)) g(x)\,dx
&=
\lambda\int_Xf_1(x)g(x)\,dx + \mu\int_X f_2(x)g(x)\,dx
\\
&=
\lambda\langle f_1,g\rangle + \mu\langle f_2,g\rangle
\\
\langle f,\lambda g_1+\mu g_2\rangle
=
\int_X f(x)(\lambda g_1(x)+\mu g_2(x))\,dx
&=
\lambda\int_X f(x)g_1(x)\,dx + \mu\int_X f(x)g_2(x)\,dx
\\
&=
\lambda\langle f,g_1\rangle + \mu\langle f,g_2\rangle.
\end{align*}
Die Bilinearform ist aber auch positiv definit: Für eine stetige
Funktion $f(x)$ gilt
\[
\langle f,f\rangle
=
\int_X f(x)^2\,dx \ge 0,
\]
wie bereits in der Einleitung gezeigt wurde.
Da auch $f(x)^2$ eine stetige Funktion ist,
verschwindet das Integral genau dann, wenn $f(x)=0\;\forall x\in X$ ist.

Die zum Skalarprodukt gehörige Norm 
\[
\|f\|_2
=
\int_X |f(x)|^2\,dx
\]
heisst auch die {\em $L^2$-Norm}.

%
% Nicht kompakte Definitionsbereiche
%
\subsubsection{Nicht kompakter Definitionsbereich}
Für stetige Funktionen auf einem kompakten Definitionsbereich scheinen
die drei Normen $\normfunc_\infty$, $\normfunc_1$ und
$\normfunc_2$ zu den gleichen Konvergenzbegriffen zu führen.
In diesem Abschnitt soll gezeigt werden, dass dies für nicht kompakte
Definitionsbereiche nicht mehr gilt.
Nicht einmal die Menge der Funktionen, die eine endliche Norm haben,
ist gleich.
Die Funktionen der nachfolgenden Beispiele sind in
Abbildung~\ref{buch:skalarprodukt:funktionenraeume:fig:normbeispiele}
dargestellt.

%
% normbeispiele.tex -- template for standalon tikz images
%
% (c) 2021 Prof Dr Andreas Müller, OST Ostschweizer Fachhochschule
%
\documentclass[tikz]{standalone}
\usepackage{amsmath}
\usepackage{times}
\usepackage{txfonts}
\usepackage{pgfplots}
\usepackage{csvsimple}
\usetikzlibrary{arrows,intersections,math}
\begin{document}
\def\skala{1}
\begin{tikzpicture}[>=latex,thick,scale=\skala]

% add image content here

\begin{scope}
	\fill[color=red!10] (0,0) -- (0,-4)
		--
		plot[domain=-2:0,samples=50] ({2*exp(\x)},{2*\x})
		-- cycle;
	\draw[->] (-0.1,0) -- (6.5,0) coordinate[label={$x$}];
	\draw[->] (0,-4) -- (0,2.0) coordinate[label={right:$y$}];
	\begin{scope}
		\clip (0,-4) rectangle (6,2);
		\draw[color=red,line width=2pt]
			plot[domain=-2.1:1.1,samples=50] ({2*exp(\x)},{2*\x});
	\end{scope}
	\fill[color=gray!50,opacity=0.5] (2,-4) rectangle (6,2);
	\node[color=red] at (0.7,-0.5) {$\|\log\|_1$};
	\node[color=red] at (0.6,{2*ln(0.3)}) [below,rotate=74] {$y=\log(x)$};
\end{scope}

\begin{scope}[yshift=-4cm,xshift=7cm]
	\fill[color=gray!50,opacity=0.5] (0,0) rectangle (2,6);
	\draw[->] (-0.1,0) -- (6.5,0) coordinate[label={$x$}];
	\draw[->] (0,-0.1) -- (0,6.3) coordinate[label={right:$y$}];
	\begin{scope}
		\clip (0,0) rectangle (6,6);
		\fill[color=red!10]
			(6,0) -- (2,0)
			--
			plot[domain=1:3.2,samples=100] ({2*\x},{2/\x})
			-- cycle;
		\fill[color=blue!20,opacity=0.5]
			(6,0) -- (2,0)
			--
			plot[domain=1:3.2,samples=100] ({2*\x},{2/(\x*\x)})
			-- cycle;
		\draw[color=red,line width=2pt]
			plot[domain=0.333:3,samples=100]
				({2*\x},{2/\x});
		\draw[color=blue,line width=1.2pt]
			plot[domain=0.333:3,samples=100]
				({2*\x},{2/(\x*\x)});
		\node[color=blue] at (2.2,0.2) [above right] {$\|f\|_2$};
		\node[color=red] at (4.3,1.5) [above] {$\|f\|_1$};
		\draw[->,color=red,line width=0.7pt] (4.3,1.5) -- (4.3,0.65);
	\end{scope}
	\node[color=red] at (5.5,4.5) [below left]
		{$\displaystyle
			f(x) = \frac{1}{x}
		$};
	\def\x{0.61}
	\coordinate (A) at ({2*\x},{2/(\x*\x)});
	\coordinate (B) at ({1.9},{2/(\x*\x)+0.1});
	\draw[color=blue,line width=0.3pt] (A) -- (B);
	\node[color=blue] at (B) [right] {$\displaystyle f(x)^2=\frac{1}{x^2}$};
\end{scope}

\end{tikzpicture}
\end{document}



\begin{beispiel}
Auf dem Definitionsbereich $X=(0,1]$ hat die Funktion
$f(x)=\log x$ endliche $L^1$-Norm aber unendliche Supremum-Norm.

\medskip
\noindent
Wegen $\lim_{x\to 0+}\log x = -\infty$ folgt $\|\log\|_\infty=\infty$.
Für die $L^1$-Norm folgt mit der Substitution $y=\log x$ und
$dy = dx/x$ oder $dx = e^y\,dy$
\begin{align*}
\|\log\|_1
&=
\int_0^1|\log x|\,dx
=
-\int_0^1\log x\,dx
=
-\int_{-\infty}^0 e^y\,dy
=
-\biggl[ e^y \biggr]_0^{-\infty}
=
1.
\end{align*}
Insbesondere ist die $L^1$-Norm beschränkt.
\end{beispiel}

\begin{beispiel}
Auf dem Definitionsbereich $X=[1,\infty)$ hat die Funktion
$f(x)=1/x$ endliche $L^2$-Norm aber unendliche $L^1$-Norm.

\medskip
\noindent
Die Integrale für die Normen ergeben:
\begin{align*}
\|f\|_1
&=
\int_1^\infty \frac{1}{x}\,dx
&
\|f\|_2^2
&=
\int_1^\infty \frac{1}{x^2}\,dx
\\
&=\biggl[\log x\biggr]_1^\infty
&
&=\biggl[-\frac{2}{x}\biggr]_1^\infty
\\
&=\infty
&
&=2.
\end{align*}
Insbesondere ist die $L^1$-Norm unbeschränkt, die $L^2$-Norm dagegen
beschränkt.
\end{beispiel}

\begin{satz}
Eine stetige Funktion auf einem beschränkten Definitionsbereich $X$
mit endlicher $L^2$-Norm hat auch endliche $L^1$-Norm.
\end{satz}

\begin{proof}[Beweis]
Aus der Cauchy-Schwarz-Ungleichung folgt
\begin{align*}
\|f\|_1
=
\int_X |f(x)|\,dx
&=
\langle |f|, 1\rangle
\le
\|f\|_2\cdot \|1\|_2.
\end{align*}
Nach Voraussetzung an die Funktion $f$ ist der erste Faktor beschränkt,
der zweite Faktor ist $\operatorname{vol}(X)$ und nach Voraussetzung
auch beschränkt.
\end{proof}

Die Beispiele zeigen, dass die Existenz der Normen selbst für stetige
Funktionen für nicht kompakten Definitionsbereich nicht garantiert ist.
Die Erweiterung auf nicht stetige Funktionen muss daher beschränkt
werden auf eine Klasse von Funktionen, für die die entsprechende Norm
existiert.
Das kann bedeuten, dass nicht alle stetigen Funktionen in Betracht 
kommen und dass neue Funktionen, die nicht stetig sind, als
Grenzwerte auftreten können.

Eine Möglichkeit, die Konvergenz etwas unter Kontrolle zu bringen
besteht darin, von den Funktionen zusätzlich zu verlangen, dass sie
ausserhalb einer kompakten Teilmenge verschwinden.

\begin{definition}
Der {\em Träger} einer Funktion $f\colon X\to\mathbb{R}$ ist die Teilmenge
\[
\operatorname{supp}(f)
=
\{x\in X\mid f(x)\ne 0\}
\]
des Definitionsgebietes.
\end{definition}

\begin{definition}
Die Menge $C_0(X)$ besteht aus den stetigen Funktionen auf $X$,
so dass der Abschluss $\overline{\operatorname{supp}(f)}$ des Trägers
kompakt ist.
\end{definition}

Für $X=\mathbb{R}$ besteht $C_0(\mathbb{R})$ aus stetigen Funktionen,
ausserhalb eines beschränkten Intervalls verschwinden.

%
% Grenzen des Riemann-Integrals
%
\subsection{Grenzen des Riemann-Integrals
\label{buch:skalarprodukt:funktionenraeume:subsection:grenzen-riemann}}
In den vorangegangenen Rechnungen sind wir immer vom Riemann-Integral
ausgegangen, welches man im Analysisunterricht als erstes kennenlernt.
Man zeigt dort, dass es für stetige Funktionen existiert und für
gleichmässig konvergente Folgen von Funktionen der Grenzwert des
Integrals mit dem Integral des Grenzwertes übereinstimmt:
\[
\int_X \lim_{n\to\infty} f_n(x)\,dx
=
\lim_{n\to\infty}
\int_X f_n(x)\,dx
\]
Der vorangegangene Abschnitt hat gezeigt, dass wir die Klasse der
Funktionen ausdehnen müssen auf Funktionen, die nicht stetig sind,
für die aber immer noch die $L^1$- oder die $L^2$-Norm existiert.
Hier zeigen sich die Schwächen des Riemann-Integrals.
In diesem Abschnitt soll an Beispielen gezeigt werden, was schief
gehen kann, und wie das Problem gelöst werden kann.

%
% Abzählbar viele Stetigkeitsstellen
%
\subsubsection{Abzählbar viele Unstetigkeitsstellen}
Wir konstruieren eine Funktionenfolge von Riemann-integrierbaren 
Funktionen, die alle das Integral $0$ haben, deren Grenzfunktion
aber nicht mehr Riemann-integrierbar ist.

Die rationalen Zahlen im Intervall $[0,1]$ sind abzählbar, d.~h.~es
gibt eine Folge $n\mapsto q_n\in[0,1]\cap\mathbb{Q}$, in der jede
rationale Zahl im Intervall vorkommt.
Aus der Folge $q_n$ konstruieren wir jetzt die Folge von Funktionen
\[
f_n(x)
=
\begin{cases} 
1&\qquad\text{$x$ ist einer der Werte $q_1,q_2,\ldots,q_n$}\\
0&\qquad\text{sonst}.
\end{cases}
\]
Die Funktion $f_n(x)$ ist also an genau $n$ Stellen von $0$ erschieden
und hat dort den Wert $1$.
Das Riemann-Integral ``sieht'' endlich viele Sprungstellen nicht,
die Funktionen $f_n$ sind also alle Riemann-integrierbar und haben
das Integral
\[
\int_0^1 f_n(x)\,dx=0.
\]
Insbesondere ist auch
\[
\lim_{n\to\infty}\int_0^1 f_n(x)\,dx = 0.
\]
Andererseits ist die Grenzfunktion
\begin{equation}
f(x)
=
\begin{cases}
1&\qquad\text{$x\in[0,1]\cap\mathbb{Q}$ ist rational}\\
0&\qquad\text{sonst, $x$ ist irrational.}
\end{cases}
\label{buch:skalarprodukt:funktionenraeume:eqn:ratfunk}
\end{equation}
Das Riemann-Integral der Funktion $f(x)$ existiert nicht.
Dazu müsste man ja für eine Unterteilung $0=x_0<x_1<\dots x_n=1$
die Riemann-Summen
\[
\overline{I}
=
\sum_{k=0}^{n-1}
(x_{k+1}-x_k) \sup_{x_k\le \xi \le x_{k+1}} f(\xi)
\qquad\text{und}\qquad
\underline{I}
=
\sum_{k=0}^{n-1}
(x_{k+1}-x_k) \inf_{x_k\le \xi \le x_{k+1}} f(\xi)
\]
berechnen, und sie müssten bei Verfeinerung der Unterteilung
gegeneinander konvergieren.
Aufgrund der Konstruktion der Funktion $f(x)$ ist aber
\[
\sup_{x_k\le \xi \le x_{k+1}} f(\xi) = 1
\qquad\text{und}\qquad
\inf_{x_k\le \xi \le x_{k+1}} f(\xi) = 0,
\]
sodass
$\overline{I}=1$ und $\underline{I}=0$ ist, ganz unabhängig von
der Unterteilung.

Der Riemannsche Integralbegriff muss also für die Zwecke der Approxmation
mit der $L^1$ oder $L^2$-Norm erweitert werden, so dass er sinnvoll mit
abzählbar vielen Unstetigkeitsstellen umgehen kann.
Insbesondere sollte er als Integral der Funktion $f(x)$ 
von \eqref{buch:skalarprodukt:funktionenraeume:eqn:ratfunk}
den Wert $0$ liefern.

%
% Masse
%
\subsubsection{Masstheorie}
Gesucht wird also ein Integral, das für eine grössere Klasse von
Funktionen definiert ist und welches sich bezüglich Grenzwerten
besser verhält als das Riemann-Integral.
Das Integral ist nur dann nützlich, wenn es für viele Funktionen
die gleichen Werte ergibt.

Die einfachsten Funktionen, die wir integrieren wollen, sind die
{\em Indikatorfunktionen}, Funktionen, die durch eine Teilmenge
\index{Indikatorfunktion}
$A\subset X$ definiert sind durch
\[
1_A(x)
=
\begin{cases}
1&\qquad\text{für $x\in A$}\\
0&\qquad\text{sonst}.
\end{cases}
\]
Für ein Intervall der Länge $\lambda(A)$ ist
\[
\int_X 1_A(x)\,dx = \lambda(A).
\]
Für Mengen, die sich aus vielen Intervallen zusammensetzen, erwarten wir
die Summenformel
\[
A=\bigcup_{k=1}^\infty A_k,
\quad
A_k\cap A_j = \emptyset\;\forall k\ne j
\qquad\Rightarrow\qquad
\lambda(A) = \sum_{k=1}^\infty \lambda(A_k).
\]
Ausserdem sollte für eine Teilmenge $A\subset B$ der Inhalt der
Differenz $\lambda(A\setminus B)=\lambda(A)-\lambda(B)$ sein.

So entsteht eine Klasse von Mengen, denen sinnvoll ein Inhalt 
zugeordnet werden kann.
Solche Mengen heissen {\em messbar}.
Dazu gehören alle Intervalle, aber auch alle Differenzen und
abzählbaren Vereinigungen von Intervallen und messbaren Mengen
sind wieder messbar.
Die Klasse der messbaren Mengen ist also sehr gross.
Es braucht natürlich noch einiges an Arbeit, um zu zeigen, dass
eine widerspruchsfreie Definition der Funktion $\lambda(A)$
tatsächlich möglich ist, die jeder messbaren Menge einen
Inhalt zuordnet.
Eine solche Funktion heisst ein {\em Mass}, das aus der Intervalllänge
konstruierte Mass heisst auch das Lebesgue-Mass nach Henri Léon Lebesgue..
\index{Lebesgue-Mass}%
\index{Mass}%

Von besonderem Interesse sind Mengen, deren Inhalt $0$ ist.

\begin{definition}
\label{buch:skalarprodukt:funktionenraeume:definition:nullmenge}
Eine Nullmenge bezüglich des Masses $\lambda$ ist eine messbare
Menge $A$ mit Mass $\lambda(A)=0$.
\index{Nullmenge}
\end{definition}

Der Riemannsche Integralbegriff lässt bei der Bestimmung des Masses
nur endlich viele Intervalle zu. 
Die Menge $Q=\mathbb{Q}\cap [0,1]$ der rationalen Zahlen im Intervall
$[0,1]$ ist abzählbar unendlich.
In jeder beliebigen Umgebung einer reellen Zahl in $[0,1]$ findet man
rationale Zahlen in $Q$, eine Überdeckung der Menge der rationalen
Zahlen mit endlich vielen Intervallen enthält daher immer auch alle
reellen Zahlen, mit der möglichen Ausnahme von endlich vielen Zahlen.
Der Inhalt, den der Riemannsche Integralbegriff der Menge $Q$ zuordnen
muss, ist daher $1$.

Der neue Massbegriff erlaubt, die Menge mit abzählbar vielen messbaren
Mengen zu überdecken.
Sei $q_k$ eine Folge, die alle rationalen Zahlen in $Q$ durchläuft.
Zu jedem $k$ konstruieren wir das Intervall
\[
A_k = (q_k-\varepsilon2^{-k},q_k+\varepsilon2^{-k})
\]
mit Inhalt $\lambda(A_k) = 2\varepsilon2^{-k}$.
Es ist klar, dass die Intervalle $A_k$ die ganze Menge $Q$ überdecken,
also
\[
Q\subset \bigcup_{k=1}^\infty A_k.
\]
Der Inhalt der Menge $Q$ ist daher
\[
\lambda(Q)
\le
\sum_{k=1}^\infty \lambda(A_k)
=
\sum_{k=1}^\infty 2\varepsilon 2^{-k}
=
2\varepsilon
\sum_{k=1}^\infty 2^{-k}
=
2\varepsilon.
\]
Da $\varepsilon$ beliebig klein gewählt werden kann, folgt, dass
$\lambda(Q)=0$ sein muss.
Aus diesem Beispiel lässt sich erahnen, dass der Lebesguesche Massbegriff
mit Grenzwerten besser umgehen kann als der aus dem Riemannschen Integral
abgeleitete.

Beim Riemannschen Integral haben endliche Mengen und Mengen mit endlich
vielen Häufungspunkten Inhalt $0$.
Viele abzählbare Mengen haben dagegen positiven Inhalt.
Das Lebesguesche Mass gibt allen abzählbaren Mengen den Inhalt 0.

%
% Lebesgue-Integral
%
\subsubsection{Lebesgue-Integral}
Aus der Konstruktion eines Masses $\lambda$ kann jetzt die Konstruktion
eines Integrals an die Hand genommen werden.
Dazu werden Funktionen durch Stufenfunktionen approximiert, die
von der Form
\[
f(x) = \sum_{k=1}^\infty a_k 1_{A_k}(x)
\]
sind, wobei $A_k$ messbare Mengen sind.
Für solche Funktionen ist die naheliegende Definition des Integrals
\[
\int_X f(x)\,d\lambda(x)
=
\sum_{k=1}^\infty a_k \lambda(A_k).
\]
Der wesentliche Unterschied zur Riemannschen Konstruktion ist,
dass nicht nur Intervalle zulässig sind sondern beliebige messbare Mengen.
Die Berechnung des Inhalts einer messbaren Menge beinhaltet bereits
die Möglichkeit, Grenzwerte zu bilden.
Auch hier ist viel Arbeit notwendig um nachzuweisen, dass sich aus diesem
Ansatz ein widerspruchsfreier neuer Integralbegriff ergibt.
Das so konstruierte Integral heisst das {\em Lebesgue-Integral} und
\index{Lebesgue-Integral}%
wird zur Unterscheidung vom gewöhnlichen Riemannschen Integral und
wegen der Bedeutung des Masses $\lambda$, welches eine grosse Rolle
bei seiner Konstruktion spielt mit
\[
\int_X f(x) \,d\lambda(x)
\]
bezeichnet.

Unterscheiden sich zwei Funktionen $f$ und $g$ nur auf einer Nullmenge,
sagt man, sie seien {\em fast überall} gleich, geschrieben
\[
f(x) = g(x) \qquad \text{fast überall}.
\]
Zwei fast überall gleiche Funktionen haben das gleiche Integral, denn
\[
\int_X f(x)\,dx - \int_X g(x)\,dx
=
\int_X f(x)-g(x)\,dx
=
\int_X 0\,dx=0
\]
weil das Integral einer fast überall verschwindenden Funktion $0$ ist.

%
% Funktionsklassen
%
\subsubsection{Klassen von fast überall gleichen Funktionen}
Verwendet man die mit dem Lebesgque-Integral berechnete $L^1$- oder
$L^2$-Norm, dann können Funktionen nicht voneinander unterschieden werden,
die fast überall gleich sind.
Grenzwerte von Funktionenfolgen in der $L^1$- oder $L^2$-Norm sind
daher nur bis auf eine Nullmenge bestimmt.

\begin{definition}
Die Menge der Lebesgue-integrierbaren Funktionen auf dem Definitionsbereich
$X\subset\mathbb{R}$ wird mit
\[
\mathscr{L}^1(X)
=
\mathscr{L}^1_{\mathbb{R}}(X)
=
\left\{ f\colon X\to \mathbb{R}
\;\left|\;
\text{$f$ ist $\lambda$-integrierbar und $\int_X|f(x)|\,dx< \infty$}
\right.\right\}
\]
bezeichnet.
Entsprechend besteht $\mathscr{L}^2(X)$ aus den Funktionen $X\to \mathbb{R}$,
für die $|f(x)|^2$ integrierbar ist.
Sie heissen auch die {\em quadratintegrierbaren} Funktionen.
\end{definition}

Das Lebesgue-Integral kann Funktionen, die sich nur auf einer Nullmenge
verschieden sind, nicht unterscheiden. 
Daher ist es notwenig, solche Funktionen in Klassen zusammenzufassen:

\begin{definition}
Die Relation
\[
f\sim g
\qquad:\Leftrightarrow \qquad f(x) = g(x)\quad\text{fast überall}
\]
ist eine Äquivalenzrelation.
Die Menge der Äquivalenzklassen von Funktionen in $\mathscr{L}^1(X)$
bezüglich dieser Relation werden mit $L^1(X)$ bezeichnet, ebenso werden
die Äquivalenzklassen von $\mathscr{L}^2(X)$ bezüglich der Relation $\sim$
mit $L^2(X)$ bezeichnet.
\end{definition}

Mit den Funktionsklassen in $L^1(X)$ und $L^2(X)$ lässt sich genau
so rechnen, wie man es sicht gewohnt ist.
Für die Summe von Funktionen $f_1\sim f_2$ und $g_1\sim g_2$ gilt
zum Beispiel
\[
\left.
\begin{aligned}
f_1(x)&=f_2(x)&&\text{fast überall}\\
g_1(x)&=g_2(x)&&\text{fast überall}\\
\end{aligned}
\quad
\right\}
\qquad
\Rightarrow
\qquad
f_1(x)+g_1(x) = f_2(x)+g_2(x)\quad\text{fast überall},
\]
denn die Menge, auf der sich $f_1+f_2$ und $g_1+g_2$ unterscheiden
ist höchstens die Vereinigung der Mengen, auf denen sich $f_1$ und 
$f_2$ bzw.~$g_1$ und $g_2$ unterscheiden.
Die Vereinigung von Nullmengen ist aber wieder eine Nullmenge.

%
% Lebesgue-Integral
%
\subsubsection{Dominierte Konvergenz}
Die Entwicklung des Lebesgueschen Integrallbegriffs war motiviert
vom Wunsch, ein Integral zu erhalten, welches sich bezüglich
Konvergenz von Funktionenfolgen besser verhält.
Tatsächlich liefert die Theorie den folgenden zentralen Satz.

\begin{satz}[Dominierte Konvergenz]
\label{buch:skalarprodukt:satz:dominierte-konvergenz}
Sei $f_n$ eine auf dem Definitionsbereich $X$ punktweise konvergente
Folge Lebesgue-integrierbarer Funktionen mit Grenzfunktion 
\[
f(x) = \lim_{n\to \infty} f_n(x).
\]
Sei ausserdem $g$ eine Lebesgue-integrierbare Funktion mit
$|f_n(x)|<g(x)$ für alle $x\in X$.
Dann ist $f$ Lebesgue-integrierbar und es gilt
\[
\lim_{n\to\infty} \int_X f(x)\,d\lambda(x)
=
\int_X f(x)\,d\lambda(x)
\]
\end{satz}

Der Satz der dominierten Konvergenz von Lebesgue ersetzt also die
Bedingung der gleichmässigen Konvergenz, die beim Riemann-Integral
erfolgreich war, durch die viel schwächere Bedingung, dass alle
Funktionen unterhalb einer gemeinsamen integrierbaren Funktion bleiben.
Dadurch wird verhindert, dass die Funktionen $f_n$ nach $\infty$
``ausbrechen'' kann und gegen eine Funktion konvergieren, die nicht
mehr integrierbar ist.


%
% Berechnung von Lebesgue-Integralen
%
\subsubsection{Berechnung von Lebesgue-Integralen}
Das Lebesque-Integral löst also die technischen Probleme, die das
Riemann-Integral manchmal bei Funktionenfolgen hat, die gegen ein
Grenzfunktion konvergieren, der man ein sinnvolles Integral im
Lebesgueschen Sinnen zuordnen kann.
Doch wie berechnet man ein Lebesgue-Integral?

Stetige Funktionen lassen sich beliebig genau durch Treppenfunktionen
approximieren.
Die Konvergenz des Lebesgue-Integrals für solche Funktionenfolgen
garantiert daher, dass das Lebesgue-Integral für stetige
Funktionen mit dem Riemann-Integral übereinstimmt.
Insbesondere braucht es keinen neuen Formalismus für die 
Berechnung von Integralen.
Auch für Funktionen, die an höchstens endlich vielen Stellen unstetig
sind, stimmt das Riemann-Integral mit dem Lebesgue-Integral überein.

Man soll sich daher das Lebesgue-Integral vor allem als eine 
Erweiterung des wohlbekannten Integrals auf Funktionen vorstellen,
die als Grenzwerte von Folgen stetiger Funktionen im Sinne der $L^1$-
oder der $L^2$-Norm auftreten können.
Stetigkeit kann dabei verloren gehen, aber Konvergenzeigenschaften
wie die dominierte Konvergenz von
Satz~\ref{buch:skalarprodukt:satz:dominierte-konvergenz}
bleiben erhalten.
Am Kalkül zur Berechnung von Stammfunktionen ändert sich nichts.




%
% 4-hilbertraum.tex
%
% (c) 2022 Prof Dr Andreas Müller, OST Ostschweizer Fachhochschule
%
\section{Hilbert-Raum
\label{buch:skalarprodukt:section:hilbertraum}}
\kopfrechts{Hilbert-Raum}
Ein Skalarprodukt stattet einen Vektorraum mit einer Norm aus.
Es ermöglicht auch, orthonormierte Vektoren zu finden.
In endlichdimensionalen Vektorräumen können so besonders nützliche
Basen konstruiert werden.
In den Funktionenräumen von
Abschnitt~\ref{buch:skalarprodukt:section:funktionenraeume},
die unendlichdimensional sind, kann der Orthonormalisierungsprozess
ohne Ende weitergeführt werden.
Im Gegensatz zu einem endlichdimensionalen Vektorraum bilden diese
orthonormierten Vektoren keine Basis, denn nicht jeder Vektor lässt
sich als Linearkombination schreiben.
Dies wird erst mit Hilfe von Reihenentwicklungen möglich, doch dazu
müssen Fragen der Konvergenz solcher Reihen geklärt werden.
Der in diesem Abschnitt eingeführte Begriff des Hilbert-Raumes tut dies.

%
% Prähilbertraum
%
\subsection{Prähilbertraum}
Die Funktionenräume, in denen wir harmonische Analysis betreiben wollen,
zeichnen sich durch das Vorhandensein eines Skalarproduktes aus.
Wir fassen diese Eigenaschaften im Begriff des Prähilbertraumes
zusammen.

\begin{definition}[Prähilbertraum]
Ein reeller Prähilbertraum ist ein reeller Vektorraum mit einem
(reellen) Skalarprodukt.
\index{Prähilbertraum}%
Eine komplexer Prähilbertraum ist ein komplexer Vektorraum mit einem
sesquilinearen Skalarprodukt.
\end{definition}

\begin{beispiel}
Der endlichdimensionale reelle Vektorraum $\mathbb{R}^n$ ist ein
reller Prähilbertraum mit dem Skalarprodukt
\[
\langle u,v\rangle
=
\sum_{i=1}^n u_iv_i
\]
für Vektoren $u,v\in\mathbb{R}^n$.
\end{beispiel}

\begin{beispiel}
Der endlichdimensionale komplexe Vektorrau $\mathbb{C}$ ist ein
komplexer Prähilbertraum mit dem Skalarprodukt
\[
\langle u,v\rangle
=
\sum_{i=1}^n \overline{u}_iv_i
\]
für Vektoren $u,v\in\mathbb{C}^n$.
\end{beispiel}

Die Skalarprodukte in den endlichdimensionalen Beispielen sind nicht
die einzig möglichen Skalarprodukte.
Alternative Skalarprodukte auf einem reellen Prähilbertraum können 
durch eine beliebige positiv definite Matrix $A$ durch
\[
\langle u,v\rangle_A
=
\sum_{i,j=1}^n u_ia_{ij}v_j
\]
definiert werden.
Wir schreiben die aus $\langle\;,\;\rangle_A$ abgeleitete Norm mit
$\normfunc_A$.
Solange unser primäres Interesse der Approximation von Funktionen gilt,
kommt es vor allem darauf an, dass die Norm, die aus dem Skalarprodukt
abgeleitet wird, zu den gleichen konvergenten Folgen führen.
Die Funktion $u\mapsto \|u\|_A$ ist stetig, sie hat daher auf der
Einheitskugel des Prähilbertraumes ein Maximum und eine Minimum,
welches wir mit $M$ bzw.~$m$ bezeichen.
Dann folgt, dass
\[
m\|u\|\le \|u\|_A\le M\|u\|
\]
für beliebige Vektoren $u\in\mathbb{R}^n$.
Daraus kann man jetzt ableiten, dass die beiden Normen $\normfunc$
und $\normfunc_A$ auf die gleichen Cauchy-Folgen und die gleichen
konvergenten Folgen führen.
Wir zeigen dies für Cauchy-Folgen:
\begin{enumerate}
\item
Sei $u_k$ eine Cauchy-Folge bezüglich der Norm $\normfunc$,
und $\varepsilon>0$.
Wir müssen zeigen, dass $u_k$ auch eine Cauchy-Folge ist bezüglich
der Norm $\normfunc_A$.
Da $u_k$ eine Cauchy-Folge bezüglich der Norm $\normfunc$ ist,
gibt es ein $N>0$ derart, dass
$\|u_k-u_l\|<\varepsilon/M$ für $k,l>N$.
Dann folgt aber
\[
\|u_k-u_l\|_A
\le
M\|u_k-u_l\|
<
M\frac{\varepsilon}{M}
=
\varepsilon
\]
für $k,l>N$.
Somit ist $u_k$ eine Cauchy-Folge bezüglich der Norm $\normfunc_A$.
\item
Ist umgekehrt  $u_k$ eine Cauchy-Folge bezüglich der Norm $\normfunc_A$,
dann gibt es ein $N>0$ derart, dass $\|u_k-u_l\|_A<m\varepsilon$ ist für
$k,l>N$.
Dann folgt
\[
m\|u_k-u_l\|\le \|u_k-u_l\|_A < m\varepsilon
\qquad\Rightarrow\qquad \|u_k-u_l\|<\varepsilon
\]
für $k,l>N$, also ist $u_k$ auch eine Cauchy-Folge bezüglich der Norm
$\normfunc$.
\end{enumerate}
In einem endlichdimensionalen Prähilbertraum hat die Wahl des Skalarproduktes
keinen Einfluss darauf, ob eine Folge eine Cauchy-Folge ist oder nicht.
Orthonormierte Vektoren werden natürlich im Allgemeinen nicht mehr
orthonormiert, dies ist jedoch ein Aspekt, dem wir uns erst später
zuwenden werden.

%
% Orthonormierte Vektoren
%
\subsection{Orthonormierte Vektoren in einem Prähilbertraum}
Der Gram-Schmidt-Orthogonalisierungsprozess kann auf eine beliebige
\index{Gram-Schmidt}%
linear unabhängige Menge von Vektoren in einem Prähilbertraum angewendet
werden.
Aus den linear unabhängigen Vektoren $a_1,a_2,\dots$ werden die
orthonormierten Vektoren
\begin{align*}
b_1
&=
\frac{a_1}{\|a_1\|}
\\
b_2
&=
\frac{
a_2 - \langle b_1,a_2\rangle b_1
}{
\|a_2 - \langle b_1,a_2\rangle b_1\|
}
\\
&\phantom{i}\vdots
\\
b_n
&=
\frac{\displaystyle
a_n - \sum_{k=1}^{n-1} \langle b_k,a_n\rangle b_k
}{\displaystyle
\biggl\|a_n - \sum_{k=1}^{n-1} \langle b_k,a_n\rangle b_k\biggr\|
}.
\end{align*}

In einem endlichdimensionalen Vektorraum der Dimension $n$ bricht
der Prozess ab, sobald eine orthonormierte Basis $b_1,\dots,b_n$
aus $n$ Vektoren gefunden wurde.
Jeder andere Vektor $v$ lässt sich dann als Linearkombination
\begin{equation}
v
=
\langle b_1,v\rangle b_1 + \langle b_2,v\rangle b_2 + \dots
=
\sum_{k=1}^n \langle b_1,v\rangle b_1
\label{buch:skalarprodukt:hilbertraum:synthese}
\end{equation}
schreiben.
Da die Summe auf der rechten Seite endlich ist, entstehen keine
Bedenken bezüglich Konvergenz, wie das bei einer unendlichen
Reihe der Fall wäre.

%
% Vollständigkeit
%
\subsection{Vollständigkeit}
In einem unendlichdimensionalen Prähilbertraum bricht der
Orthogonalisierungsprozess nicht ab, es gibt immer noch einen
linear unabhängigen Vektor, der nicht in dem von den bereits
gefundenen Vektoren aufgespannten Raum liegt.
Die Summe~\ref{buch:skalarprodukt:hilbertraum:synthese} wird dann
eine unendliche Summe, die nur im Sinne eines Grenzwertes der
Partialsummenfolge
\begin{equation*}
s_n = \sum_{k=1}^n \langle b_k,v\rangle b_k
\end{equation*}
ausgewertet werden kann.
Man darf zwar aufgrund der Konstruktion aus $v$ davon ausgehen,
dass $s_n$ gegen $v$ konvergiert,
aber für eine beliebige Folge von Koeffizienten $c_k$ ist nicht
garantiert, dass die Summe
\[
\sum_{k=1}^\infty c_kb_k
=
\lim_{n\to\infty} \sum_{k=1}^n c_kb_k
\]
einen Grenzwert hat.
Ein nützliche Theorie kann nur entstehen, wenn gefordert wird,
dass jede Cauchy-Folge des Prähilbertraums tatschächlich konvergiert.

\begin{definition}[Hilbert-Raum]
Ein Prähilbertraum heisst {\em Hilbert-Raum}, wenn er vollständig ist.
\end{definition}

Endlichdimensionale Vektorräume über sind automatisch vollständig,
da gibt es also gar keinen Unterschied zwischen Prähilbertraum und
Hilbert-Raum.
Das folgende Beispiel zeigt, dass dies für unendlichdimensionale
Hilbert-Räume nicht mehr zutrifft.

\begin{beispiel}
\label{buch:skalarprodukt:hilbertraum:bsp:sinreihe}
Der Funktionenraum
\(
C_{\mathbb{R}}([-\pi,\pi])
\)
der stetigen Funktionen auf dem Intervall $[-\pi,\pi]$ wird mit
dem Skalarprodukt
\[
\langle f,g\rangle
=
\int_{-\pi}^\pi f(x)g(x)\,dx
\]
zu einem Prähilbert-Raum.
Die Summanden der Reihe~\eqref{buch:skalarprodukt:eqn:rechteckreihe} 
sind Sinus-Funktionen, von denen wir später zeigen werden, dass sie
orthogonal sind.
Seien $s_n(x)$ die Partialsummen der Reihe, also
\begin{equation}
s_n(x) = \frac{4}{\pi}\sum_{k=0}^n \frac{\sin (2k+1)x}{2k+1},
\label{buch:skalarprodukt:hilbertraum:eqn:sn}
\end{equation}
dann kann man auch die Norm $\|s_n-s_m\|$ bestimmen, es gilt nämlich
\begin{equation}
\|s_n-s_m\|
=
\biggl\|
\frac{4}{\pi}
\sum_{k=m}^n \frac{\sin (2k+1)x}{2k+1}
\biggr\|,
\label{buch:skalarprodukt:hilbertraum:eqn:snsm}
\end{equation}
wobei wir $n>m$ angenommen haben, was wir ohne Beschränkung der 
Allgemeinheit tun dürfen.
Die Norm eines einzelnen Terms ist
\begin{align}
\|\sin rx\|^2
&=
\int_{-\pi}^\pi \sin^2 rx\,dx
=
\int_{-\pi}^\pi \frac12 - \frac{\cos rx}{2}\,dx
=
\int_{-\pi}^\pi \frac12\,dx - \int_{-\pi}^\pi \frac{\cos rx}{2}\,dx.
\notag
\intertext{Der zweite Term ist ein Integral über eine Periode des
Integranden und verschindet daher.
Der erste Term ergibt daher}
\|\sin rx\|^2
&= \pi.
\intertext{Für die einzelnen Terme der Summe
\eqref{buch:skalarprodukt:hilbertraum:eqn:sn}
folgt daher}
\biggl\|
\frac{\sin{2k+1}x}{2k+1}
\biggr\|^2
&=
\frac{\pi}{(2k+1)^2}.
\notag
\intertext{Für die Differenz
\eqref{buch:skalarprodukt:hilbertraum:eqn:snsm} finden wir daher}
\|s_n-s_m\|^2
&=
\frac{16}{\pi^2}
\sum_{k=m}^m \frac{\pi}{(2k+1)^2}
=
\frac{16}{\pi}
\sum_{k=m}^m \frac{1}{(2k+1)^2}.
\label{buch:skalarprodukt:hilbertraum:eqn:bsprest}
\end{align}
Da aus dem Analysisunterricht bekannt ist, dass die Reihe $\sum_k\frac1{k^2}$
konvergiert, kann die rechte Seite von 
\eqref{buch:skalarprodukt:hilbertraum:eqn:bsprest}
beliebig klein gemacht werden, die 
Reihe~\eqref{buch:skalarprodukt:eqn:rechteckreihe} 
ist also eine Cauchy-Folge im Prähilbertraum $C_{\mathbb{R}}([-\pi,\pi])$.
Die Grenzfunktion ist die Rechteckfunktion von
Abbildung~\ref{buch:skalarprodukt:fig:fourierrechteck}, sie ist nicht
stetig.
Wir haben also eine Cauchy-Folge im Prähilbertraum 
$C_{\mathbb{R}}([-\pi,\pi])$ gefunden, die darin nicht konvergiert.
\end{beispiel}

%
% Hilbert-Basis
%
\subsection{Hilbert-Basis}
Sei jetzt $H$ ein Hilbert-Raum.
Führt man wieder die Konstruktion einer orthonormierten Basis durch,
entsteht eine Menge $\mathcal{B}=\{b_1,b_2,\dots\}$ orthonormierter
Vektoren.
In einem unendlichdimensionalen Hilbert-Raum ist $\mathcal{B}$ eine
unendliche Menge.
Die Vollständigkeit des Hilbert-Raumes garantiert, dass jede
Cauchy-Folge konvergiert, insbesondere können wir zu jedem beliebigen
Vektor $v$ die Koeffizienten $c_k=\langle b_k,v \rangle$ bestimmen
und versuchen, mit der
Summe~\eqref{buch:skalarprodukt:hilbertraum:synthese}
den Vektor zurückzugewinnen.
Vollständigkeit garantiert zwar die Konvergenz gegen einen Grenzwert
\[
v_0 = \sum_{k=1}^\infty c_k b_k,
\]
aber es gibt keine Garantie, dass $v=v_0$ ist.

\begin{beispiel}
Die Funktionen
\[
b_k(x) = \sin (2k+1)x
\qquad\text{mit}\quad
k\in \mathbb{N}
\]
wurden in Beispiel~\ref{buch:skalarprodukt:hilbertraum:bsp:sinreihe}
die Rechteckfunktion synthetisiert.
Doch reichen diese Funktionen nicht aus, um alle Funktionen auf
dem Intervall $[-\pi,\pi]$ zu synthetisieren.

Für die konstante Funktion $v(x)$ sind die Koeffizienten
\[
\langle v,b_k\rangle
=
\int_{-\pi}^\pi v(x)b_k(x)\,dx
=
\int_{-\pi}^\pi \sin (2k+1)x\,dx
=
0,
\]
die Summe
\[
v_0
=
\sum_{k=0}^\infty
c_k
b_k(x)
=
0
\]
verschwindet also, $v\ne v_0$.
\end{beispiel}

Wir berechnen das Skalarprodukt 
\[
\langle v-v_0,b_k\rangle
=
\langle v,b_k\rangle
-
\biggl\langle \sum_{l=1}^\infty c_lb_l,b_k\biggr\rangle
=
c_k - \sum_{l=1}^\infty c_l\langle b_l,b_k\rangle
=
c_k - \sum_{l=1}^\infty c_l\delta_{lk}
=
c_k-c_k=0.
\]
Der Vektor $v-v_0$ muss also orthogonal sein zu allen Vektoren
$b_k$.
Die Menge der Vektoren $b_k$ kann also um einen weiteren Vektor
vergrössert werden, der zu allen vorhandenen Vektoren orthogonal ist.

\begin{satz}
Eine Menge von orthogonalen Vektoren in einem Prähilbertraum ist linear 
unabhängig.
\end{satz}

\begin{proof}[Beweis]
Angenommen, die orthogonalen Vektoren  $a_1,\dots,a_n$ sind linear
abhängig.
Dann gibt es Zahlen $\lambda_i$, die nicht alle $=0$ sind und derart,
dass
\[
\sum_{k=1}^n \lambda_ka_k
=
\lambda_1 a_1 + \ldots + \lambda_n a_n
=
0.
\]
Das Skalarprodukt mit $a_k$ ergibt
\[
0
=
\langle a_k,\lambda_1a_1 +\ldots + \lambda_na_n\rangle
=
\lambda_1
\langle a_k,a_1\rangle
+
\ldots
+
\lambda_n
\langle a_k,a_n\rangle
=
\lambda_k \|a_k\|^2
\]
für alle $k=1,\dots,n$.
Da die Vektoren $a_k\ne 0$ sind folgt, dass $\lambda_k=0$ sein muss
für alle $k=1,\dots,n$, im Widerspruch zur Voraussetzung, dass die
$\lambda_k$ nicht alle verschwinden.
Der Widerspruch zeigt, dass die Annahme der linearen Abhängigkeit
falsch ist.
\end{proof}

In einem endlichdimensionalen Prähilbertraum findet man immer eine
orthonormierte Basis, aus der sich jeder Vektor linear kombinieren
lässt.
In einem unendlichdimensionalen Raum muss dieser Begriff etwas 
erweitert werden.

\begin{definition}[Hilbert-Basis]
Eine {\em Hilbert-Basis} des Hilbert-Raumes $H$ ist eine Menge von orthonormalen
\index{Hilbert-Basis}%
Vektoren $b_k\in H$, $k\in \mathbb{N}$ derart, für jeden Vektor $v\in H$
die Reihe 
\[
\sum_{k\in\mathbb{N}} \langle b_k,v\rangle b_k
\]
konvergiert und die Summe $v$ hat.
\end{definition}

Auf eine Hilbert-Basis ist die Inuition anwendbar, die man von 
einer orthonormierten Basis eines endlichdimensionalen Prählibertraumes
hat.

\begin{definition}[separabler Hilbert-Raum]
\index{separabel}%
Ein Hilbert-Raum, der eine Hilbert-Basis besitzt, heisst {\em separabel}.
\end{definition}

Nicht separable Hilbert-Räume sind deutlich grösser.
In einem nicht separable Hilbert-Raum ist es nicht möglich, alle Vektoren
aus einer Folge von orthonormierten Vektoren linear zu kombinieren.
Dies verträgt sich nicht mit der Intuition, die in endlichdimensionalen
Vektorräumen entstanden ist.
Separabilität ist eine Art ``Endlichkeitseigenschaft''.
Man findet sie zum Beispiel bei den $L^2$-Räumen mit kompaktem
Definitionsgebiet.
Kompaktheit kann als eine topologische
Endlichkeitseigenschaft angesehen werden.
Es lassen sich aber auch nicht separable Hilbert-Räume konstruieren,
zum Beispiel sogenannte fastperiodische Funktionen.
\index{fastperiodische Funktionen}%
Für unsere Anwendungen sind nicht separable Hilbert-Räume jedoch nicht
notwendig und wir können im Folgenden annehmen, dass alle Hilbert-Räume
separabel sind und somit eine Hilbert-Basis haben.

%
% Der Hilbert-Raum $l^2$
%
\subsection{Der Hilbert-Raum $l^2$}
Die abstrakte Definition eines separablen Hilbert-Raumes suggeriert,
dass es sehr viele verschiedene Hilbert-Räume geben könnte.
Dem ist aber nicht so, ein Hilbert-Raum ist eine ziemlich starre Struktur.
In diesem Abschnitt konstrieren wir zunächst den besonders übersichtlichen
separablen Hilbert-Raum $l^2$ und zeigen anschliessend, dass sich jeder
separable Hilbert-Raum isometrisch auf $l^2$ abbilden lässt.

\begin{satz}
Der Vektorraum der unendlichen Folgen
\[
l^2_{\mathbb{R}}
=
\biggl\{
(x_0,x_1,\dots)
\in
\mathbb{R}^{\mathbb{N}}
\,\bigg|\,
\sum_{i=0}^\infty x_i^2<\infty\biggr\}
\}
\]
mit dem Skalarprodukt
\[
\langle x,y\rangle_2
=
\sum_{k=0}^\infty x_ky_k
\]
ist ein separabler Hilbert-Raum.
\end{satz}

\begin{proof}[Beweis]
Die Standardbasis des Vektorraums $l^2$ besteht aus den Vektoren $e_i$
mit $i\in\mathbb{N}$ mit 
\begin{align*}
e_0 &= (1,0,0,0,\dots)
\\
e_1 &= (0,1,0,0,\dots)
\\
e_2 &= (0,1,0,0,\dots)
\\  &\;\vdots
\end{align*}
Es ist klar, dass die Vektoren $e_i$ orthonormiert sind.
Wir müssen uns überlegen, dass jede Folge in $l^2$ durch Linearkombinationen
approximiert werden kann.
Wäre das nicht so, dann gäbe es einen von $0$ verschiedenen
Vektor $v=(v_0,v_1,v_2,\dots)\in l^2$, der auf allen Vektoren $e_i$
orthogonal ist.
Dann wäre aber $0=\langle v,e_i\rangle = v_i$, d.~h.~Vektor $v$ ist der
Nullvektor.
Dieser Widerspruch zeigt, dass die Vektoren $e_i$ eine Hilbert-Basis
bilden.
\end{proof}

Der Hilbert-Raum $l^2$ ist die einfachste, direkte Verallgemeinerung der
endlichdimensionalen Hilbert-Raume $\mathbb{R}^n$ mit dem Standardskalarprodukt.
Er ist aber auch der ``einzige'' separable Hilbert-Raum, alle Hilbert-Räume
lassen sich isometrisch auf $l^2$ abbilden.

\begin{satz}
\label{buch:skalarprodukt:hilbertraum:satz:phil2}
Ist $H$ ein separabler Hilbert-Raum, dann gibt es eine isometrische
lineare Abbildung $\varphi\colon H\to l^2$.
\end{satz}

\begin{proof}[Beweis]
Da $H$ separabel ist, gibt es eine Hilbert-Basis $b_0,b_1,b_2,\ldots\in H$
und $v$ ein beliebiger Vektor mit der Darstellung
\[
v = \sum_{k=0}^\infty c_kb_k.
\]
Das Skalarprodukt mit $b_k$ ist
\[
\langle b_i,v\rangle
=
\biggl\langle b_i,\sum_{k=0}^\infty c_kb_k\biggr\rangle
=
\sum_{k=0}^\infty c_k \underbrace{\langle b_i,b_k\rangle}_{\delta_{ik}}
=
c_i.
\]
Die Abbildung
\[
\varphi\colon H \to l^2
:
v \mapsto (\langle b_0,v\rangle,\langle b_1,v\rangle,\langle b_2,v\rangle,\dots)
=
(c_0,c_1,c_2,\dots)=c\in l^2
\]
ist linear und umkehrbar durch die Abbildung
\[
\varphi^{-1} \colon l^2 \to H
:
(c_0,c_1,c_2,\dots) \mapsto \sum_{k=0}^\infty c_k b_k.
\]
Die Norm von $v$ ist
\[
\|v\|^2
=
\biggl\|
\sum_{k=0}^\infty c_k b_k
\biggr\|^2
=
\sum_{k=0}^\infty c_k^2
=
\|c\|_2^2.
\qedhere
\]
\end{proof}

Man verliert also nichts, wenn man statt im Hilbert-Raum $H$ im
Hilbert-Raum $l^2$ rechnet.
Die später zu besprechende Fourier-Transformation ist genau
so eine Abbildung $\varphi$ wie in
Satz~\ref{buch:skalarprodukt:hilbertraum:satz:phil2}, wenn 
trigonometrische Funktionen als Basisfunktionen des Raumes
$L^2([-\pi,\pi])$ verwendet werden.
Die Tatsache, dass die Abbildung eine Isometrie ist, hat einen
speziellen Namen.

\begin{definition}[Parseval]
\label{buch:skalarprodukt:hilbertraum:def:parseval}
Die Identität
\[
\|v\|^2 = \|\varphi(v)\|_2^2
\]
heisst {\em Parseval-Gleichung}.
\index{Parseval-Gleichung}%
Aus der Polaridentität folgt dann auch die
{\em Parseval-Plancherel-Formel}
\index{Parseval-Plancherel-Formel}%
\[
\langle u,v\rangle_H
=
\langle\varphi(u),\varphi(v)\rangle_2
=
\sum_{k=0}^\infty
\langle b_k,u\rangle
\langle b_k,v\rangle.
\]
\end{definition}

Die endlichdimensionalen Hilbert-Räume können auf
natürliche Art in $l^2$ eingebettet werden, indem die Koordinaten
eines $n$-dimensionalen Vektors auf die ersten $n$-Komponenten der
Folge in $l^2$ abgebildet werden.
Aus dieser Idee lässt sich auch ein Prähilbertraum konstruieren,
in dem einfacher gerechnet werden kann.

\begin{definition}[$l^0$]
\label{buch:skalarprodukt:hilbertraum:def:l0}
Sei 
\[
l_0
=
\{
x\in \mathbb{R}^{\mathbb{N}}
\mid
\text{$x_k\ne 0$ nur für endlich viele $k$}
\}
\]
der Vektorraum der Folgen, die nur endlich viele nicht verschwindende
Folgenglieder haben.
Die Standardbasisvektoren von $l^2$ sind alle in $l_0$.
\end{definition}

Da die Norm eines Elementes $x\in l_0$ eine endliche Summe ist,
ist $x\in l^2$ und damit $l_0\subset l^2$.
Jeder Vektor $v\in l^2$ kann aber beliebig genau durch einen Vektor
in $l_0$ approximiert werden.
Dazu konstruiert man die Folge
\bgroup
\def\el#1{\phantom{v_0}\llap{#1}}
\begin{align*}
x_0 &= (v_0,\el{0},\el{0},\el{0},\el{0},\dots) \in l_0\\
x_1 &= (v_0,v_1,\el{0},\el{0},\el{0},\dots) \in l_0\\
x_2 &= (v_0,v_1,v_2,\el{0},\el{0},\dots) \in l_0\\
x_3 &= (v_0,v_1,v_2,v_3,\el{0},\dots) \in l_0\\
    &\;\vdots
\end{align*}
\egroup
Die Differenz $x_n-v$ hat die Norm
\[
\|x_n-v\|^2
=
\sum_{k=n+1}^\infty v_k^2,
\]
die für $n\to\infty$ gegen $0$ geht.
Da es in $l^2$ Vektoren gibt, die nicht in $l_0$ drin sind, zeigt
dies, dass $l_0$ kein Hilbert-Raum ist, dass aber die Vervollständigung
von $l_0$ der Hilbert-Raum $l^2$ ist.

Der Prähilbert-Raum rechtfertigt, dass man in Anwendungen im Sinne
einer Approximation mit endlich vielen der Koeffizienten $c_k$
rechnen kann.
Im Kontext der Fourier-Theorie sind dies
die trigonometrischen Polynome, mit denen sich jede periodische
Funktion approximieren lässt.

%
% Der Hilbert-Raum $L^2$
%
\subsection{Der Hilbert-Raum $L^2$}
Der Vektorraum der stetigen Funktionen auf dem Intervall $[-\pi,\pi]$
mit dem bekannten Skalarprodukt ist ein Prähilbertraum.
Wir werden später zeigen, dass die Funktionen
\[
1, \cos kx \quad\text{und}\quad \sin kx
\qquad \text{mit $k\in\mathbb{N}$ und $ k\ge 1$}
\]
orthogonal sind bezüglich dieses Skalarproduktes.
Es wird sich zeigen, dass diese Funktionen nach geeigneter Normierung
eine Hilbert-Basis von $L^2$ bilden.
Die in Satz~\ref{buch:skalarprodukt:hilbertraum:satz:phil2}
konstruierte Abbildung bezeichnen wir mit $\mathscr{F}:L^2([-\pi,\pi])\to l^2$
und nennt sie die Fourier-Transformation.
Die Parseval-Plancherel-Formel besagt dann, dass man das Skalarprodukt
von Funktionen auch durch das Skalarprodukt der zugehörigen
Fourier-Koeffizienten berechnen kann:
\index{Fourier-Koeffizienten}%
\[
\langle f,g\rangle_{L^2{[-\pi,\pi]}}
=
\langle \mathscr{F}f,\mathscr{F}g\rangle_{l^2}.
\]
Die Fourier-Transformation formalisiert also die Erkenntnis, dass
man alle Fragen, die man mit Skalarprodukten beantworten kann, statt
mit Funktionen auch mit Skalarprodukten von Fourier-Koeffizienten
finden kann.

%
% Der Darstellungssatz von Riesz
%
\subsection{Der Darstellungssatz von Riesz}
Eine Linearform $l(x)$ auf dem Vektorraum $\mathbb{R}^n$ ist gegeben durch
die Werte $l(e_i)$ auf den Standardbasisvektoren.
Der Wert von auf $x=x_1e_1+\dots+x_ne_n$ ist dann
\[
l(x)
=
x_1l(e_1)+\dots+x_nl(e_n)
=
\begin{pmatrix}
l(e_1)\\\vdots\\l(e_n)
\end{pmatrix}
\cdot
\begin{pmatrix}
x_1\\\vdots\\x_n
\end{pmatrix}.
\]
Insbesondere lässt sich jede Linearform als Skalarprodukt mit einem 
geeigneten Vektor geschrieben werden.
Diese Eigenschaft gilt auch in einem Hilbert-Raum.

\begin{satz}[Riesz]
\label{buch:skalarprodukt:hilbertraum:satz:riesz}
Sei $H$ ein Hilbert-Raum und $l\colon H\to\mathbb{R}$ eine stetige
Linearform.
Dann gibt es eine $v\in H$ derart, dass $l(x) = \langle v,x\rangle$.
\end{satz}

Der Beweis dieser Eigenschaft kann man zum Beispiel in \cite{buch:wavelets}
finden.
Der Satz gilt natürlich auch in einem komplexen Hilbert-Raum.




%
% 5-sobolevraum.tex
%
% (c) 2022 Prof Dr Andreas Müller, OST Ostschweizer Fachhochschule
%
\section{Sobolev-Räume
\label{buch:skalarprodukt:section:sobolev}}
\kopfrechts{Sobolev-Räume}
Im Beispiel~\ref{buch:skalarprodukt:hilbertraum:bsp:sinreihe} haben 
wir eine Reihe kennengelernt, deren Partialsummen alle stetig sind,
da sie trigonometrische Polynome sind.
Doch die Grenzfunktion ist die Rechteckfunktion, die ganz offensichtlich
nicht stetig sind.
Die Norm des Hilbertraums $L^2$ ist offenbar nicht stark genug, die
Stetigkeit der Grenzfunktion sicherzustellen.

\begin{figure}
\centering
\includegraphics{chapters/010-skalarprodukt/images/fourierdreieck.pdf}
\caption{Die 
Reihe~\ref{buch:skalarprodukt:sobolevraum:eqn:fourierdreieck}
mit beliebig oft stetig differenzierbaren Partialsummen konvergiert
gleichmässig gegen die stetige Dreiecksfunktion, aber die Grenzfunktion
ist nicht mehr überall differenzierbar.
\label{buch:skalarprodukt:sobolevraum:fig:fourierdreieck}}
\end{figure}
Die Supremum-Norm ist zwar stärker, gleichmässig konvergente Folgen
von stetigen Funktionen konvergieren gegen einen stetigen Grenzwert.
Aber sie ist nicht stark genug Differenzierbarkeit zu erzwingen.
Die Reihe
\begin{equation}
f(x)
=
\frac{4}{\pi}\biggl(
\sin x - \frac{\sin 3x}{3^2} + \frac{\sin 5x}{5^2} -\frac{\sin 7x}{7^2} +\dots
\biggr)
\label{buch:skalarprodukt:sobolevraum:eqn:fourierdreieck}
\end{equation}
hat beliebig oft stetig differenzierbare Partialsummen und sie konvergiert
gleichmässig gegen die stetige, aber an den Stellen $k\frac{\pi}2$,
$k\in\mathbb{Z}$, nicht differenzier Dreicksfunktion (siehe
Abbildung~\ref{buch:skalarprodukt:sobolevraum:fig:fourierdreieck}).

Die Beispiele zeigen, dass Stetigkeits- und Differenzierbarkeitseigenschaften
beim Arbeiten in einem Hilbertraum verloren gehen könnten.
Andererseits möchte man einen Hilbertraum dazu verwenden,
Differentialgleichungen zu lösen, wo man aus der Theorie weiss, dass
die Lösungen beliebig oft stetig differenzierbar sind.
Es ist daher anzunehmen, dass wir diese als Grenzfunktionen bezüglich
einer stärkeren Norm erhalten können, welche die Differenzierbarkeit
garantiert.
In diesem Abschnitt sollen die wesentlichen Ideen der Konstruktion
dieser sogenannten Sobolev-Räume skizziert werden.

%
% Ableitungen
%
\subsection{Ableitungen}

%
% Skalaraprodukt mit Ableitungen
%
\subsection{Skalarprodukt mit Ableitungen}

%
% Konvergenz der Ableitungen
%
\subsection{Konvergenz der Ableitungen}







%\uebungsabschnitt
%\aufgabetoplevel{chapters/010-potenzen/uebungsaufgaben}
%\begin{uebungsaufgaben}
%\uebungsaufgabe{101}
%\uebungsaufgabe{102}
%\uebungsaufgabe{103}
%\uebungsaufgabe{104}
%\end{uebungsaufgaben}
%\endgroup

