%
% 1-definition.tex
%
% (c) 2023 Prof Dr Andreas Müller, OST Ostschweizer Fachhochschule
%
\section{Definition
\label{buch:skalarprodukte:section:definition}}
\kopfrechts{Definition}
Ein Skalarprodukt ist vor allem deshalb besonders einfach anzuwenden,
weil es bilinear ist.
Dies bedeutet, dass man Skalarprodukte ausmultiplizieren kann, die
Intuition, die man aus der elementaren Algebra mitbringt, führt zum
Erfolg.
Allerdings braucht es für ein erfolgreiches Skalarprodukt noch
etwas mehr.

%
% Symmetrische Bilinearformen
%
\subsection{Symmetrische Bilinearformen}
Das aus der Vektorgeometrie bekannte Skalarprodukt
\index{Vektorgeometrie}%
\index{Skalarprodukt}%
\[
\vec{u}\cdot \vec{v}
=
\sum_{i=1}^n u_iv_i
\]
auf $\mathbb{R}^n$ ist deshalb besonders nützlich, weil sich damit
so rechnen lässt, wie man es sich von einem Produkt in der Algebra
gewohnt ist.
Dazu gehört, dass man Produkte ausmultiplizieren kann:
\begin{equation}
\begin{aligned}
(\lambda\vec{u}+\mu\vec{w})\cdot\vec{v}
&=
\lambda\vec{u}\cdot\vec{v}+\mu\vec{w}\cdot\vec{v}
\\
\vec{u}\cdot(\lambda\vec{v}+\mu\vec{w})
&=
\lambda\vec{u}\cdot\vec{v}+\mu\vec{u}\cdot\vec{w}.
\end{aligned}
\label{buch:skalarprodukt:eqn:ausmultiplizieren}
\end{equation}
Umgekehrt kann man gemeinsame Faktoren auch ausklammern.
Die Rechenregeln \eqref{buch:skalarprodukt:eqn:ausmultiplizieren}
besagen, dass die Funktion
\[
\cdot
\;
\colon
\mathbb{R}^n \times \mathbb{R}^n
\to
\mathbb{R}
:
(\vec{u},\vec{v}) \mapsto \vec{u}\cdot\vec{v}
\]
linear ist in jedem Faktor.

\begin{definition}
Eine Funktion
\[
b\colon
V\times V \to \mathbb{R}
:
(u,v) \mapsto b(u,v)
\]
heisst {\em bilinear} oder {\em Bilinearform},
wenn sie linear ist in jedem Argument, also also
\index{bilinear}%
\index{Bilinearform}%
\[
\begin{aligned}
b(\lambda u+\mu w,v) &= \lambda b(u,v) + \mu(w,v)
\\
b(u,\lambda v+\mu w) &= \lambda b(u,v) + \mu(u,w)
\end{aligned}
\]
gilt.
\end{definition}

Das Skalarprodukt der Vektorgeometrie hat aber noch eine weitere
wichtige Eigenschaft.
Der Kosinus-Satz der ebenen Trigonometrie ist folgt für
\index{Kosinus-Satz}%
\index{Trigonometrie}%
$\vec{c} = \vec{b}-\vec{a}$ 
durch Berechnung der Länge
\[
|\vec{c}|^2
=
\vec{c}\cdot\vec{c}
=
(\vec{b}-\vec{a})\cdot(\vec{b}-\vec{a})
=
\vec{b}\cdot\vec{b}
-
\vec{b}\cdot\vec{a}
-
\vec{a}\cdot\vec{b}
+
\vec{a}\cdot\vec{a}
=
|\vec{a}|^2 + |\vec{b}|^2 - 2 \vec{a}\cdot\vec{b}
=
|\vec{a}|^2 + |\vec{b}|^2 - 2 |\vec{a}|\;|\vec{b}|\cos\alpha,
\]
wobei wir verwendet haben, dass $\vec{a}\cdot\vec{b}=\vec{b}\cdot\vec{a}$
ist.
Die zusätzliche Eigenschaft ist also, dass sich die Faktoren
des Skalaproduktes vertauschen lassen.

\begin{definition}
Eine Funktion $b\colon V\times V \to\mathbb{R}$ heisst {\em symmetrisch},
wenn $b(u,v)=b(v,u)$ für alle $u,v\in V$.
\index{symmetrisch}%
\end{definition}

Für eine symmetrische Bilinearform erfüllt die binomische Formel
\begin{align*}
b(u+v,u+v)
&=
b(u,u+v) + b(v,u+v)
=
b(u,u)+b(u,v)+b(v,u)+b(v,v)
\\
&=
b(u,u) + 2b(u,v) + b(v,v).
\end{align*}

%
% Norm
%
\subsection{Norm}
In der Vektorgeometrie wird das Skalarprodukt auch dazu verwendet,
mit $|\vec{v}|^2 = \vec{v}\cdot\vec{v}$ 
die Länge eines Vektors zu berechnen.
Jeder Vektor $\ne 0$ hat eine positive Länge.
Für eine beliebige Bilinearform ist jedoch nicht automatisch
sichergestellt, dass $b(u,u)\ne 0$ ist für $u\ne 0$.
Ausserdem kann ein Längenbegriff nur dann aus $b$ abgeleitet werden,
wenn zusätzlich $b(u,u)>0$ ist ür $u\ne 0$, da sich andernfalls
die Wurzel nicht ziehen lässt.

\begin{beispiel}
Die symmetrische Bilinearform
\[
b(x,y)
=
x_1y_1-x_2y_2
\]
auf $\mathbb{R}^2$ ist nicht dazu geeignet, eine Länge zu definieren,
dann der zweite Standardbasisvektor $e_2$ hat das Produkt
$b(e_2,e_2) = -1$.
Auch gibt es einen Vektor, der ``Länge'' 0 hat,
nämlich $v=e_1+e_2$ mit
\[
b(v,v)
=
b(e_1+e_2,e_1+e_2)
=
b(e_1,e_1) + 2\underbrace{b(e_1,e_2)}_{\displaystyle =0} + b(e_2,e_2)
=
1-1
=
0.
\qedhere
\]
\end{beispiel}

\begin{definition}
Eine symmetrische Bilinearform $\langle\;\,,\;\rangle$
heisst {\em positiv definit}, wenn $\langle u,u\rangle > 0$ 
für alle von 0 verschiedenen Vektoren $u\ne 0$ gilt.
\end{definition}

\begin{definition}
Eine {\em Skalarprodukt} ist eine positiv definite symmetrische Bilinearform.
\index{Skalarprodukt}%
\end{definition}

Ein Skalarprodukt hat jetzt alle Eigenschaften, die erlauben, einen 
Abstandsbegriff zu definieren.

\begin{definition}
Die zu einem Skalarprodukt $\langle\;\,,\;\rangle$ gehörige Norm ist
definiert als
\[
\| v\|
=
\sqrt{\langle v,v\rangle}
\]
für $v\in V$.
\end{definition}

%
% Sesquilineare Funktionen
%
\subsection{Sesquilineare Funktionen}
Sei jetzt $V$ ein komplexer Vektorraum.
Aus einer bilinearen Funktion
\[
b\colon V\times V \to \mathbb{C} : (u,v) \mapsto b(u,v)
\]
auf $V$ kann jedoch keine brauchbare Norm abgeleitet werden.
Eine solche müsste $\| v\|=b(v,v)\ge 0$ erfüllen.
Selbst wenn $b(u,u)> 0$ ist für einen speziellen Vektor $u\in V$,
ist das Skalarprodukt von $iu$ mit sich selbst
\[
b(iu,iu)
=
i^2 b(u,u)
=
-b(u,u)
<
0.
\]
Da $|i|=1$ ist, würde man eher erwarten, dass $iu$ die gleiche 
Länge hat wie $u$, dass also $b(iu,iu)=b(u,u)$.

\begin{definition}
Eine Funktion $f\colon V\to U$ zwischen komplexen Vektorräumen 
heisst {\em konjugiert linear}, wenn 
\[
f(\lambda u + \mu v)
=
\overline{\lambda} f(u) + \overline{\mu} (v)
\]
für alle $u,v\in V$ und $\lambda,\mu\in \mathbb{C}$.
\end{definition}

Im obengenannten Beispiel wird $b(iu,iu)>0$, wenn $b$ im ersten Faktor
konjugiert linear ist.
Dann ist nämlich $b(iu,iu) = -ib(u,iu) = -i^2 b(u,u) = b(u,u)>0$.

\begin{definition}
Eine Funktion
\[
\langle\;\,,\;\rangle
\colon
V\times V \to \mathbb{C}
:
(u,v) \mapsto \langle u,v\rangle
\]
heisst {\em sesquilinear} oder {\em Sesquilinearform}
wenn sie linear ist im zweiten Argument
\end{definition}

Das lateinische Wort {\em sesqui} bedeudet eineinhalb, eine
sesquilineare Funktion ist linear in einem Faktor, aber nur
halb linear im anderen.

\begin{beispiel}
Die Form
\[
\langle u,v\rangle = \sum_{i=1}^n \overline{u}_i v_i
\]
mit $u,v\in \mathbb{C}^n$ ist sesquilinear.
Tatsächlich gilt
\begin{align*}
\langle u,\lambda v+\mu w\rangle
&=
\sum_{i=1}^n \overline{u}_i (\lambda v_i+\mu w_i)
=
\lambda
\sum_{i=1}^n \overline{u}_i v_i
+
\mu
\sum_{i=1}^n \overline{u}_i w_i
=
\lambda\langle u,v\rangle
+
\mu\langle u,w\rangle
\\
\langle \lambda u+\mu w, v\rangle
&=
\sum_{i=1}^n \overline{(\lambda u_i+\mu w_i)}v_i
=
\overline{\lambda}
\sum_{i=1}^n \overline{u}_i v_i
+
\overline{\mu}
\sum_{i=1}^n \overline{w}_iv_i
=
\overline{\lambda}\langle u,v\rangle
+
\overline{\mu}\langle w,v\rangle.
\qedhere
\end{align*}
\end{beispiel}

%
% Hermitesche Formen
%
\subsection{Hermitesche Formen}
Damit aus einer sesquilinearen Funktion eine Norm abgeleitet werden
kann, muss das Produkt $\langle u,u\rangle$ für jeden Vektor $u\in V$
eine reelle Zahl sein.
Selbst für die sesquilineare Funktion
\[
\langle\;\,,\;\rangle
\colon
\mathbb{C}\times\mathbb{C}
\to
\mathbb{C}
:
(u,v) \mapsto i\overline{u}v
\]
ist dies jedoch nicht der Fall, da $\langle 1,1\rangle = i\not\in\mathbb{R}$
ist.
Die folgende Eigenschaft kann aber garantieren, dass
$\langle u,u\rangle\in\mathbb{R}$.

\begin{definition}
Eine sesquilinear Funktion 
\[
\langle \;\,,\;\rangle
\colon
V\times V
\to
\mathbb{C}
\]
heisst in {\em konjugiert symmetrisch} oder {\em hermitesch}, wenn
\[
\langle u,v\rangle = \overline{\langle v,u\rangle}
\]
für alle $u,v\in V$ gilt.
\end{definition}

\begin{beispiel}
Die Standard-Sesquilinearform
\[
\langle u,v\rangle
=
\sum_{i=1}^n \overline{u}_i v_i
\]
auf $V=\mathbb{C}^n$ ist konjugiert symmetrisch, denn
\begin{align*}
\langle u,v\rangle
&=
\sum_{i=1}^n \overline{u}_i v_i
=
\overline{
\sum_{i=1}^n u_i \overline{v}_i
}
=
\overline{\langle v,u\rangle}.
\qedhere
\end{align*}
\end{beispiel}

%
% Komplexe Skalarprodukte
%
\subsection{Komplexe Skalarprodukte}
Wie bei einem reellen Skalarprodukt reichen auch im Fall eines
komplexen Vektorraums die Eigenschaften der Sesquilinearität
und der hermiteschen Symmetrie nicht aus, ein sinnvolles
Skalarprodukt zu definieren.

\begin{definition}
Eine hermitesche Sesquilinearform $\langle\;\,,\;\rangle$
auf dem komplexen Vektorraum $V$ heisst {\em positiv definit}, wenn
\[
\langle u,u\rangle > 0
\]
für alle $u\ne 0$ in $V$.
Ein {\em komplexes Skalarprodukt} ist eine positiv definite hermitesche
Sesquilinearform.
Die zugehörige {\em Norm} eines Vektors ist
$\|v\| = \sqrt{\langle u, u\rangle}$.
\end{definition}


%
% Verallgemeinerte Skalarprodukte
%
\subsection{Verallgemeinerte Skalarprodukte}


