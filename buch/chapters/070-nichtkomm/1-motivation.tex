%
% 1-motivation.tex
%
% (c) 2023 Prof Dr Andreas Müller
%
\section{Das Registrierungsproblem
\label{buch:nichtkomm:section:motivation}}
Das Registrierungsproblem ist ein wichtiges Beispiel für die Bedeutung
der Gruppentheorie in der harmonischen Analysis.
In diesem Abschnitt stellen wir das Registrierungsproblem vor
und Lösen den wichtigen Fall des kommutativen Registrierungsproblems
mit Fourier-Theorie als Motivation für die spätere Entwicklung 
der nichtkommutativen harmonischen Analysis.

%
% Problemstellung
%
\subsection{Problemstellung
\label{buch:nichtkomm:subsection:problemmstellung}}
Bilder sind Funktionen auf der zweidimensionalen Ebene
$\mathbb{R}^2\to\mathbb{R}$.
Das Registrierungsproblem ist die folgende Aufgabe.

\begin{aufgabe}
Gegben zwei Funktionen $f,g\colon\mathbb{R}^2\to\mathbb{R}$, die
sich nur durch eine geometrische Transformation des Definitionsbereiches 
unterscheiden.
Finde die Transformation $\varphi\colon\mathbb{R}^2\to\mathbb{R}^2$ derart,
dass $f=g\circ \varphi$.
\end{aufgabe}

In der Aufgabenstellung bleibt offen, was mit einer ``geometrischen
Transformation'' gemeint ist.
Die Transformation sollte natürlich umkehrbar sein, sie sollte
nicht zwei Pixel im einen Bild auf das gleiche Pixel um anderen
Bild abbilden.
Auch sollte man solche Abbildungen zusammensetzen können, zum
Beispiel auch um die Problemlösung in Schritten zu ermöglichen.
Es ist daher naheliegend, eine Gruppe von Transformationen
der zweidimensionalen Ebene $\mathbb{R}^2$ festzulegen, in der
$\varphi$ gefunden werden muss.
Die Aufgabenstellung wird damit etwas genauer

\begin{aufgabe}
\label{buch:nichtkomm:aufgabe:regG}
Gegeben ist eine Gruppe $G$ von Transformationenen
$\varphi:\mathbb{R}^2 \to\mathbb{R}^2$ der Ebene und 
zwei Funktionen $f,g\colon\mathbb{R}^2\to\mathbb{R}$, die sich nur durch
eine Transformation $\varphi\in G$ unterscheiden.
Finde die Transformation $\varphi\in G$ derart
dass $f=g\circ \varphi$.
\end{aufgabe}

\begin{beispiel}
Die {\em Bewegungen} der Ebene sind affinen isometrischen
Transformationen der Ebene.
Dazu gehören Tranlationen, Drehungen, Spiegelungen und beliebige
Kombinationen davon.
In vielen praktischen Anwendungen weiss man, dass die Orientierung
der Bilder erhalten ist, damit fallen die Spiegelungen weg, es
müssen nur noch Translationen und Drehungen berücksichtig werden.
Translationen werden durch Vektoren $v\in\mathbb{r}$ beschrieben,
Drehungen durch Matrizen in $\operatorname{SO}(2)$.
\end{beispiel}

\begin{beispiel}
Bilder der Himmelskugel sind Funktionen $f\colon S^2\to\mathbb{R}$
definiert auf der Kugeloberfläche $S^2$.
Sie können durch dreidimensionale Drehungen
in $G=\operatorname{SO}(3)$ zur Deckung gebracht werden.
\end{beispiel}

Die Aufgabenstellung~\ref{buch:nichtkomm:aufgabe:regG}
erscheint zwar natürlich, ist aber in der Praxis
nicht realistisch.
In der Praxis enthalten Bilder immer auch Sensorrauschen, Restverzerrung
der Optik, atmosphärische Verzerrungen oder Quantisierungsrauschen, welches
durch die Digitalisierung und die Auflösung in Pixel verursacht wird.
Die Aufgabe muss daher revidiert werden.

\begin{aufgabe}
\label{buch:nichtkomm:motivation:aufgabe:regGaehnlich}
Gegeben ist eine Gruppe $G$ von Transformationenen
$\varphi:\mathbb{R}^2 \to\mathbb{R}^2$ der Ebene und 
zwei Funktionen $f,g\colon\mathbb{R}^2\to\mathbb{R}$, die sich nur durch
eine Transformation $\varphi\in G$ unterscheiden.
Finde die Transformation $\varphi\colon\mathbb{R}^2\to\mathbb{R}^2$ derart,
dass $f$ und $g\circ \varphi$ ``möglichst ähnlich'' sind.
\end{aufgabe}

Als Mass dafür, wie ähnlich sich die beiden Bilder $f$ und
$g\circ\varphi$ sind, kann nach den Erkenntnissen von
Kapitel~\ref{buch:chapter:skalarprodukte} das Skalarprodukt
verwendet werden.
Daraus leitet man den folgenden Lösungsansatz ab.
Unter allen Transformationen $\varphi\in G$ finde diejenige,
für die das Skalarprodukt
\[
\langle f,g\circ \varphi\rangle
\]
maximal wird.
Dieser intuitive Ansatz hat aber das Problem, dass 
das Skalarprodukt nicht nur gross werden kann, weil die Funktionen
$f$ und $g\circ\varphi$ sich ähnlicher werden, sondern auch,
weil die Norm von $g\circ\varphi$ anwächst.
Um diesem Problem aus dem Weg zu gehen, könnte man auch 
die Funktion
\[
G\to\mathbb{F}
\colon
\varphi\mapsto
\frac{\langle f,g\circ \varphi\rangle}{\|f\|\cdot\|g\circ\varphi\|}
\]
betrachten, die aber wesentlich unhandlicher ist.
Sie ist insbesondere nicht mehr linear in $f$ und $g$.
Man kann dieser Schwierigkeit aus dem Weg gehen, wenn man
verlangt, dass die Transformationen $\varphi\in G$ die Norm nicht
ändern, dass also
\[
\|g\|
=
\|g\circ \varphi\|
\]
ist für alle $\varphi\in G$.
Dies ist im Beispiel der Bewegungen der Ebene oder der Drehungen
der Kugeloberfläche sichergestellt, da diese Transformationen
die Flächeninhalte nicht verändern.

%
% Lösung des Registrierungsproblems für Translationen
%
\subsection{Registrierung mit Translationen
\label{buch:nichtkomm:motivation:subsection:translationen}}
Für Funktionen in der Ebene und die Gruppe der Translationen
ist das Registrierungsproblem mit harmonischer Analysis lösbar.
Dies schürt die Hoffnung, dass nichtkommutative harmonische
Analysis das Registrierungsproblem auch für nichtkommutative
Gruppen von Transformationen zu lösen hilft.

\begin{aufgabe}
\label{buch:nichtkomm:motivation:aufgabe:r2}
Gegen sind zwei Funktionen $f, g\colon\mathbb{R}\to\mathbb{R}$, die
sich nur durch eine Translation unterscheiden.
Finde einen Vektor $v\in\mathbb{R}^2$ derart, dass die Translation
\[
T_vg
\colon
\mathbb{R}\to \mathbb{R}
:
x\mapsto g(x-v)
\]
des Bildes $g$ dem Bild $f$ ``möglichst ähnlich'' ist.
\end{aufgabe}

%
% Skalarprodukt und Faltung
%
\subsubsection{Skalarprodukt und Faltung}
Wie im Lösungsansatz zur Aufgabe
\ref{buch:nichtkomm:motivation:aufgabe:regGaehnlich}
angedeutet, soll das Skalarprodukt als Mass für die ``Ähnlichkeit''
von Funktionen verwendet werden.
Wir betrachten daher die Funktion
\begin{equation}
\mathbb{R}^2 \to \mathbb{R}
:
v\mapsto \langle f,T_vg\rangle
=
\int_{\mathbb{R}^2} f(x)T_vg(x)\,dx
=
\int_{\mathbb{R}^2} f(x)g(x-v)\,dx
\label{buch:nichtkomm:motivation:eqn:kreuzkorr}
\end{equation}
und suchen nach demjenigen Vektor $v$, der das Skalarprodukt
maximiert.
Das Integral rechts in~\eqref{buch:nichtkomm:motivation:eqn:kreuzkorr}
heisst die {\em Kreuzkorrelation} der beiden Bilder $f$ und $g$ und
wird manchmal auch $f\star g(v)$ geschrieben.
Man beachte, dass hier ein fünfzackiger Stern $\star$ verwendet wird,
nicht der für die Faltung verwendete Asterisk $*$.

Die Kreuzkorrelation unterscheidet sich von der Faltung nur durch
das Vorzeichen des Argumentes von $g$, also
\begin{equation}
(f\star g)(v)
=
\int_{\mathbb{R}^2} f(x) g(-(x-v))\,dx
=
\int_{\mathbb{R}^2} f(x) \check{g}(v-x)\,dx
=
(f *\check{g}) (v).
\label{buch:nichtkomm:motivation:eqn:kkorrfaltung}
\end{equation}
Dabei haben wir wieder die Notation $\check{g}(x)=g(-x)$ verwendet.
Der Gewinn dabei ist vor allem, dass für die Faltung die harmonische
Analysis zusätzliche Rechenregeln bereitstellt, die für die
Kreuzkorrelation nicht unmittelbar zur Verfügung stehen.

%
% Effizente Berechnung
%
\subsubsection{Effiziente Berechnung}
Für die direkte Berechnung der
Kreuzkorrelation~\eqref{buch:nichtkomm:motivation:eqn:kreuzkorr}
wird für jeden von $O(n^2)$ Vektoren $v$ die Berechnung des Skalarproduktes
mit $O(n^2)$ Multiplikationen nötig.
Sie wird jedoch wird jedoch durch die Formulierung als Faltung bedeutend
einfacher.
Die Fourier-Transformation
von~\eqref{buch:nichtkomm:motivation:eqn:kkorrfaltung}
ist
\begin{align*}
\mathscr{F}{(f\star g)}
&=
\mathscr{F}(f*\check{g})
=
\mathscr{F}f\cdot\mathscr{F}\check{g}
\intertext{oder nach Fourier-Rücktransformation}
(f\star g)(v)
&=
\mathscr{F}^{-1} (\mathscr{F}f\cdot\mathscr{F}g)(v)
\end{align*}
Für digitale Bilder kann man die diskrete Fourier-Transformation
verwenden.
Dank der Existenz der schnellen Fourier-Transformation wird die
Berechnung der Faltung somit in $O(n^2\log(n)^2$) möglich.
Danach ist die Bestimmung des Vektors $v$, für den das Maximum
von $f*\check{g}$ erreicht wird, mit dem Aufwand $O(n^2)$ möglich.

%
% Nichtkommutative Registrierungsprobleme
%
\subsection{Nichtkommutative Registrierungsprobleme
\label{buch:nichtkomm:motivation:subsection:nichtkommutativ}}
Die Lösung des Registrierungsproblems konnte mit Hilfe der
Fourier-Transformation erfolgreich gelöst werden.
Die Einschränkung auf Translationen ist jedoch in vielen
Anwendungen nicht realistisch.
Dies ist nur möglich, wenn die Bilder zum Beispiel durch geeignete
mechanische Vorrichtungen gegen Verdrehungen zu schützen.
Dies ist zum Beispiel nicht möglich in der Astrophotographie.
Da werden viele Bilder mit wenigen Minuten Belichtungszeit 
kombiniert.
Unzulänglichkeiten in der Aufstellung des Teleskops verursachen
aber unweigerlich Verdrehungen.
Das Registrierungsproblem für Translationen sind also nicht
ausreichend.

Der erste Schritt, diesen Lösungsansatz auf grössere Gruppen
zu verallgemeinern, besteht darin, die Gruppen von Transformationen
besser kennen zu lernen, die für realistischere Registrierungsprobleme
von Interesse sein könnten.

%
% Gruppenoperationen und Funktionen
%
\subsection{Gruppenoperationen auf Funktionen
\label{buch:nichtkomm:motivation:subsection:gruppenoperationen}}
Das Registrierungsproblem versucht, zwei Bilder durch eine geometrische
Transformation des Definitionsbereichs der Funktionen zur Deckung
zu bringen.
In den bisherigen Beispielen war dies die Ebene $\mathbb{R}^2$ oder
die Kugeloberfläche $S^2$.
Wir vereinheitlichen die Notation und studieren jetzt Funktionen
auf dem Definitionsbereich $X$.
Die Gruppe $G$ ist jetzt eine Gruppe von Transformationen des
Definitionsbereichs $X$.
Die Operation eines Elements $g\in G$ auf $X$ schreiben wir
$g(x) = g\cdot x$, $x\in X$.
Daraus lässt sich jetzt auch eine Operation auf den Funktionen auf
$X$ konstruieren:

\begin{definition}
Ist $g\in G$ und $f\colon X\to\mathbb{R}$ eine Funktion auf $X$, dann
bezeichnen wir mit
\begin{equation}
(g\cdot f)(x)
=
T_gf(x)
=
f(g^{-1}x)
\label{buch:nichtkomm:motivation:eqn:opf}
\end{equation}
die um $g$ verschobene Funktion $f$.
\end{definition}

Der Leser mag sich fragen, warum in
\eqref{buch:nichtkomm:motivation:eqn:opf}
das inverse Element $g^{-1}$ von $g$ verwendet wird.
Damit wird erreicht, dass die Abbildung $g\mapsto T_g$
ein Homomorphismus wird, denn für ein weiteres Element $h\in G$
ist dann
\[
T_g(T_hf)(x)
=
(T_hf)(g^{-1}x)
=
f(h^{-1}g^{-1}x)
=
f((gh)^{-1}x)
=
(T_{gh}f)(x),
\]
also $T_g\circ T_h= T_{gh}$.
Man sagt auch, $T$ definiert eine Links-Operation der Gruppe $G$ 
auf der Menge der Funktionen.

Die Notation $T_gf$ für die um $g$ verschobene Funktion $f$ fällt
mit der Notation zusammen, die wir fürher für den Definitionsbereich
$\mathbb{R}^2$ und die Verschiebungen in der Ebene verwendet haben.
Tatsächlich ist dies der Fall des Definitionsbereichs $X=\mathbb{R}^2$,
auf dem die Gruppe $G=\mathbb{R}^2$ durch die gewähnliche
Translation wirkt, also $v\cdot x = x +v$ für $x\in X$ und $v\in G$.
Die Formel \eqref{buch:nichtkomm:motivation:eqn:opf} wird in diesem
Fall zu
\[
(T_vf)(x) = f(-v + x) = f(x-v)
\]
wie in Aufgabe~\ref{buch:nichtkomm:motivation:aufgabe:r2}.

%
% Die Gruppe SO(2)\ltimes R^2
%
\subsection{Die Gruppe $\operatorname{SO}(2)\ltimes \mathbb{R}^2$
\label{buch:nichtkomm:motivation:subsection:so2lr^2}}
Das ebene Registrierungsproblem verwendet Drehungen und Verschiebungen
in der Ebene, um Bilder zur Deckung zu bringen.
Die zugehörige Gruppe wird mit $\operatorname{SO}(2)\ltimes\mathbb{R}^2$
bezeichnet und soll in diesem Abschnitt konstruiert werden.
Das besondere an dieser Situation ist, dass die Drehungen auch auf
die Richtung der Translation wirken können.
Wir verallgemeinern diese Situation in den folgenden Definitionen.

\begin{definition}
Sei $A$ eine abelsche Gruppe, die im folgenden jeweils additiv geschrieben
werden soll.
Ein {\em Endorphismus} von $A$ ist eine Abbildung $h\colon A\to A$ mit
$h(a_1+a_2)=h(a_1)+h(a_2)$.
und
$h(0)=0$.
Die Menge der Endomorphismen wird mit
$\operatorname{End}(A)=\operatorname{Hom}(A,A)$ bezeichnet.
\end{definition}

Die Translationen der Ebene bilden eine abelsche Gruppe, wir haben
Sie früher mit den Vektoren $\mathbb{R}^2$ identifiziert.
Die inverterbaren Endomorphismen der Gruppe der Translationen der
Ebene ist die allgemeine lineare Gruppe $\operatorname{GL}_2(\mathbb{R})$,
ist also eine Gruppe.
Dies gilt aber auch ganz allgemein für die Menge der invertierbaren
Endomorphismen, die Menge
\[
\{ s \in \operatorname{End}(A)
\mid
\text{$s$ ist invertierbar}
\}
\]
ist eine Gruppe.

\begin{definition}
Sei $G$ eine Gruppe und $A$ eine abelsche Gruppe.
Man sagt, die Gruppe {\em $G$ wirkt auf $A$}, wenn
es einen Homomorphismus $\vartheta\colon G\to\operatorname{Hom}(A,A)$ gibt.
Dies bedeutet, dass $\vartheta(g)$ für jedes $g\in G$ ein Endomorphismus
der Gruppe $A$ ist, also
\begin{align*}
\vartheta(g)(a_1+a_2)&=\vartheta(g)(a_1) + \vartheta(g)(a_2).
\end{align*}
Es befeutet auch, dass die Zusammensetzung von zwei solchen
Endomorphismen assoziativ wirkt, d.~h.
\[
\vartheta(gh)=\vartheta(g)\vartheta(h)
\quad\Rightarrow\quad
\vartheta(gh)(a) = \vartheta(g)(\vartheta(h)a)
\]
Die Operation von $g$ auf $a$ wird oft auch $g\cdot a = \vartheta(g)a$
abgekürzt.
\end{definition}

Die Gruppe der Drehungen $G=\operatorname{SO}(2)$ operiert auf den
Translationen wie folgt.
Eine Translation wird durch den Verschiebungsvektor $v$ beschrieben.
Ist $g\in\operatorname{SO}(2)$ eine Drehmatrix, dann ist $gv$ ein
Vektor in $\mathbb{R}^2$, der wieder eine Verschiebung beschreibt.

\begin{satz}
Sei $G$ eine Gruppe, die auf einer abelschen Gruppe $A$ wirkt.
Die Menge $G\times A$ mit der Verknüpfung
\[
(g_1,v_1)\cdot(g_2,v_2)
=
g_1
\]
ist eine Gruppe mit dem neutralen Element $(e,0)$.
Sie wird mit $G\ltimes A$ bezeichnet und heisst das {\em semidirekte Produkt}
von $G$ und $A$.
\end{satz}

\begin{proof}[Beweis]
Wir müssen nachrechnen, dass die Operation assoziativ ist und dass
jedes Element von $G\ltimes A$ ein inverses Element hat.

Seien also $(g_1,a_1)$, $(g_2,a_2)$ und $(g_3,a_3)$ drei Elemente von
$G\times A$, wir berechnen die Produkt
\begin{align*}
(g_1,a_1)\cdot((g_2,a_2)\cdot(g_3,a_3))
&=
(g_1,a_1)\cdot(g_2g_3,a_2+a_2a_3)
\\
&=
(g_1g_2g_3,g_1(g_2a_3+a_2)+a_1)
=
(g_1g_2g_3,g_1g_2a_3+g_2a_2+a_1)
\\
((g_1,a_1)\cdot(g_2,a_2))\cdot(g_3,a_3)
&=
(g_1g_2,g_1a_2+a_1)\cdot(g_3,a_3)
\\
&=
(g_1g_2g_3, g_1g_2a_3+g_1a_2+a_1)
\end{align*}
Die beiden Produkte stimmen überein, somit ist die Verknüpfung assoziativ.

Wir kontrollieren, dass $(e,0)$ tatsächlich das neutrale Element
von $G\ltimes A$ ist.
Tatsächlich ist $(e,0)\cdot (g,a)=(eg,ea+0)=(eg,a)=(g,a)$ und
$(g,a)\cdot(e,0)=(ge,g0+a)=(g,a)$.

Es bleibt noch nachzuweisen, dass das Element $(g,a)$ ein inverses Element
hat.
Es muss also $(h,b)$ gefunden werden derart, dass $(g,a)\cdot (h,b)=(e,0)$.
Aus $(g,a)\cdot(h,b)=(gh,gb+a)=(e,0)$ folgt zunächst, dass $h=g^{-1}$ sein
muss.
Um $b$ zu bestimmen, löst man die zweite Komponente auf, also
\[
gb+a=0
\quad\Rightarrow\quad
b=g^{-1}(-a)=-g^{-1}a.
\]
Tatsächlich ist $(g,a)\cdot(g^{-1},-g^{-1}a)=(gg^{-1},g(-g^{-1}a)+a)=(e,0)$
und $(g^{-1},-g^{-1}a)\cdot (g,a)=(g^{-1}g, g^{-1}a-g^{-1}a)=(e,0)$,
somit ist $(g^{-1},-g^{-1}a)$ das zu $(g,a)$ inverse Elemente in $G\ltimes A$.
\end{proof}

Wir möchten jetzt illustrieren, wie die Drehungen und Verschiebungen
der Ebene genau die Struktur der Gruppe
$\operatorname{SO}(2)\ltimes\mathbb{R}^2$ ergeben.
Dazu lassen wir ein Paar $(g,v)\in\operatorname{SO}(2)\times \mathbb{R}^2$
durch die Formel
\[
(g,v)\cdot x
=
gx + v
\]
auf den Vektoren $x\in \mathbb{R}^2$ wirken.
Ein weiteres Element $(h,w)\in\operatorname{SO}(2)\times\mathbb{R}^2$
macht daraus
\[
(h,w)\cdot ((g,v)\cdot x)
=
(h,w)\cdot (gx+v)
=
h(gx+v)+w
=
(hg)x + hv + w.
\]
Die gleiche Wirkung hat das Element
$(hg, hv+w)\in\operatorname{SO}(2)\times\mathbb{R}^2$,
das ist aber nichts anderes als die Verknüpfung in der Gruppe
$\operatorname{SO}(2)\ltimes \mathbb{R}^2$.

Die Gruppe $\operatorname{SO}(2)\ltimes\mathbb{R}^2$ ist also genau
die Gruppe, die dem ebenen Registrierungsproblem zu Grunde liegt.
Wir halten diese Beobachtung in der folgenden Aufgabe fest.

\begin{aufgabe}[Ebenes Registrierungsproblem]
Zu gegebenen Funktionen $f,g\colon \mathbb{R}^2\to\mathbb{R}$, die sich
nur durch eine Drehung und Verschiebung unterscheiden, finde ein
Element $s\in\operatorname{SO}(2)\ltimes \mathbb{R}^2$ derart, das
$f$ und $T_sg$ sich ``möglichst ähnlich'' sind.
\end{aufgabe}

%
% Die Gruppe SO(3)
%
\subsection{Die Gruppe $\operatorname{SO}(3)$
\label{buch:nichtkomm:motivation:subsection:so3}}
Für das Registrierungsproblem auf der Kugeloberfläche ist der
Definitionsbereich der zu registrierenden Funktion die
zweidimensionale Kugeloberfläche $X=S^2$.
Darauf operatiert die volle Gruppe der dreidimensionalen
Drehmatrizen $\operatorname{SO}(3)$.

\begin{aufgabe}[Sphärisches Registrierungsproblem]
Gegeben zwei Funktionen $f,g\colon S^2\to\mathbb{R}$ auf der Kugeloberfläche,
die sich nur durch eine Drehung aus $\operatorname{SO}(3)$ unterschieden,
finde $s\in\operatorname{SO}(3)$ derart, dass $f$ und $T_sg$ so ähnlich
wie möglich sind.
\end{aufgabe}




