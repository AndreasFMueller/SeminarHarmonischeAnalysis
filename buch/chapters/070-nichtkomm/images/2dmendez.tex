%
% 2dreg.tex -- Mendez Transformation für R^2
%
% (c) 2021 Prof Dr Andreas Müller, OST Ostschweizer Fachhochschule
%
\documentclass[tikz]{standalone}
\usepackage{amsmath}
\usepackage{times}
\usepackage{txfonts}
\usepackage{pgfplots}
\usepackage{csvsimple}
\usetikzlibrary{arrows,intersections,math,calc}
\definecolor{darkgreen}{rgb}{0,0.6,0}
\begin{document}
\def\skala{1}
\begin{tikzpicture}[>=latex,thick,scale=\skala]

\draw[->,color=red] (2,2) -- +(6,0) coordinate[label={$r$}];
\fill[color=darkgreen] (2,2) circle[radius=0.05];

\begin{scope}
\clip (-1,-1) rectangle (7.8,4.8);
\foreach \r in {0.5,1,...,8}{
	\draw[color=darkgreen,line width=0.3pt] (2,2) circle[radius=\r];
}
\end{scope}
\draw[color=darkgreen,line width=1.0pt] (2,2) circle[radius=2.5];
\fill[color=darkgreen] (4,3.5) circle[radius=0.05];
\node at (4,3.5) [above right] {$y$};
\node at (2,2) [below left] {$x$}; 
\def\w{-20}
\node at ($(2,2)+(\w:2.5)$) [above left,rotate={90+\w}] {$S_xy$};

\draw[->] (-1,0) -- (8,0) coordinate[label={$x$}];
\draw[->] (0,-1) -- (0,5) coordinate[label={right:$y$}];

\end{tikzpicture}
\end{document}

