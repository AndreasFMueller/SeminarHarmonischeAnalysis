%
% 3-mittelung.tex
%
% (c) 2022 Prof Dr Andreas Müller, OST Ostschweizer Fachhochschule
%
\section{Mittelungsoperation
\label{buch:nichtkomm:section:mittelung}}
\kopfrechts{Mittelungsoperation}
Im letzten Abschnitt wurde gezeigt, dass sich Registrierungsprobleme
immer durch die Angabe einer Gruppe $G$ und einer abgeschlossenen
Untergruppe $K\subset G$ festlegen lassen.
Dies bedeutet aber, dass das Registrierungsproblem von $X$ zuerst
auf die Gruppe $G$ transportiert werden muss.

%
% Invariante Funktionen
%
\subsection{Invariante Funktionen}
Im ursprünglichen Registrierungsproblem ging es um Funktionen auf
dem homogenen Raum $X$, der jetzt durch $G/K$ ersetzt werden muss.

%
% Funktionen auf G/K
%
\subsubsection{Funktionen auf $G/K$}
Die Abbildung $\pi\colon G\to G/K$, die jedem Gruppenelement den Orbit $gK$
zuordnet, ist eine surjektive Abbildung.
Eine Funktion $f\colon G/K\to\mathbb{R}$ auf $G/K$ wird durch Zusammensetzung
mit $\pi$ zu einer Funktion auf $f\circ\pi\colon G\to\mathbb{R}$.
Sie hat aber zusätzlich die Eigenschaft, dass
\[
f\circ\pi(gk) = f(gkK) = f(gK) = f\circ \pi (g).
\]
Sie ändert sich also bei einer Rechtsoperation mit einem Element
von $K$ nicht.
Man kann also Funktionen auf $G/K$ auch gleichsetzen mit rechtsinvarianten
Funktionen auf $G$.

%
% Die Operation $g\mapsto \check{g}$
%
\subsubsection{Die Operation $f\mapsto\check{f}$}
Für die Lösung des ursprünglichen Registrierungsproblem wurde die
Kreuzkorrelation der Bilder mit Hilfe der Abbildung $f\mapsto \check{f}$
in eine Faltung umgewandelt.
Für eine $K$-rechtsinvariante Funktion $f$ ist aber $\check{f}$ nicht
mehr rechtsinvariant.
Vielmehr gilt
\[
\check{f}(kg)
=
f((kg)^{-1})
=
f(g^{-1}k^{-1})
=
f(g^{-1})
=
\check{f}(g),
\]
die Funktion ist jetzt also linksinvariant.
Umgekehrt wird eine linksinvariante Funktion $f$ zu einer
rechtsinvarianten Funktion $\check{f}$.

%
% Biinvariante Funktionen
%
\subsection{Biinvariante Funktionen}
Die Lösung des kommutativen Registrierungsproblems mit Hilfe harmonischer
Analysis basiert auf der Möglichkeit, die Kreuzkorrelation mit Hilfe
der Abbildung $f\mapsto\check{f}$ in eine Faltung umzuwandeln und dann
den Faltungssatz zu verwenden.

%
% Faltung
%
\subsubsection{Faltung}
Seien $f$ und $g$ integrierbare Funktionen mit kompaktem Träger
in $\mathscr{K}(G)$.
Dann ist die Faltung die Funktion
\[
f*g(x) = \int_G f(y)g(y^{-1}x)\,dy.
\]
Falls $g\in\mathscr{K}(G/K)$ ausserdem rechtsinvariant ist, folgt
\[
f*g(xt)
=
\int_G f(y)g(y^{-1}xt)\,dy
=
\int_G f(y)g(y^{-1}x)\,dy
\]
für jedes $t\in K$, die Funktion $f*g$ ist also ebenfalls rechtsinvariant.
Man sagt, die rechtsinvarianten Funktionen bilden innerhalb der Algebra
$\mathscr{G}$ der Funktionen auf $G$ ein Linksideal.

Ist die Funktion $f$ rechtsinvariant, dann ist $f(yt)=f(y)$.
Mit Hilfe der Linksinvarianz des Haar-Masses erhält man aus
der Definition der Faltung zunächst
\begin{align*}
f*g(tx)
&=
\int_G f(y)g(y^{-1}tx)\,dy
=
\int_G f(ty) g((ty)^{-1}tx)\,dy
\intertext{für jedes $t\in K$, und unter Verwendung der Linksinvarianz
der Funktion $f$ schliesslich}
&=
\int_G f(y) g(y^{-1}t^{-1}tx)\,dy
=
\int_G f(y) g(y^{-1}x)\,dy
=
f*g(x),
\end{align*}
die Funktion $f*g$ ist also linksinvariant.
Die linksinvarianten Funktionen bilden also ein Rechtsideal.

Im Allgemeinen hat aber die Faltung einer rechtsinvarianten Funktion
mit einer beliebigen Funktion keine besonderen Invarianzeigenschaften
mehr.
Noch schlimmer, sind $f$ und $g$ rechtsinvariante Funktionen, dann
ist $\check{f}*g$ sowohl rechts- als auch linksinvariant, während
$f*\check{g}$ keine besonderen Symmetrieeigenschaften hat.
Sie ist damit nicht mehr in einer Algebra von Funktionen, die viel
grösser ist als die Funktionen, die man mit Bildern identifizieren
kann.
Für die Lösung des Registrierungsproblems wird es daher zwischenzeitlich
nötig sein, sich auf Funktionen einzuschränken, die zusätzliche
Invarianzeigenschaften haben.

%
% Funktionen auf $K\backslash G/K$
%
\subsubsection{Funktionen auf $K\backslash G/K$}
Sind $f$ und $g$ Funktionen, die sowohl rechtsinvariant wie auch
linksinvariant sind.
Dann ist nach obigen Rechnungen auch die Faltung $f*g$ sowohl rechts-
als auch linksinvariant.
Ausserdem ist mit jeder Funktion $f\in \mathscr{K}(K\backslash G/K)$
auch die Funktion $\check{f}\in\mathscr{K}(K\backslash G/K)$.

\begin{beispiel}
Für die triviale abgeschlossene Untergruppe $K=\{e\}$ ist jede Funktion
auf $G$ sowohl rechts- als auch linksinvariant und damit auch die
Faltung.
Dieses Beispiel zeigt, dass es im allgemeinen nötig sein wird, die
Untergruppe $K$ gross genug zu wählen, damit das Registrierungsproblem
einer Lösung näher gebracht werden kann.
\end{beispiel}

\begin{definition}[biinvariant]
Funktionen, die bezüglich der Untergruppe $K$ sowohl rechts- als auch
linksinvariant sind, heissen {\em biinvariant}.
\index{biinvariant}%
Die Menge der bezüglich rechts- und linksinvarianten Funktionen auf $G$
mit kompaktem Träger $\mathscr{K}(K\backslash G/K)$ bezeichnet.
\index{KKGK@$\mathscr{K}(K\backslash G/K)$}%
\end{definition}

%
% Mittelungsoperation
%
\subsection{Mittelungsoperation}
Die Bilder auf $X$ sind rechtsinvariante Funktionen auf $G$.
Für die Arbeit in $\mathscr{K}(K\backslash G/K)$ muss man sich
aber auf biinvariante Funktionen einschränken, wird dazu eine
geeignete Operation benötigt.

Sie $f$ eine beliebige integrierbare Funktion mit kompaktem
Träger auf $G$.
Wir definieren die Funktionen
\begin{align*}
f^\natural(x)
&=
\int_K f(xk)\,dk,
\\
\mathstrut^\natural f(x)
&=
\int_K f(kx)\,dk.
\end{align*}
Für eine bereits rechtsinvariante Funktion $f$ ist $f^\natural(x)$ 
ein Integral über eine konstante Funktion und ist daher nur dann
definiert, wenn das Haar-Mass von $K$ beschränkt ist.
Wir nehmen daher im Folgenden immer an, dass das Integral der konstanten
Funktion $1$ über $K$ ebenfalls $1$ ist, dass also die Operationen
Mittelungsoperationen sind.

Die Funktion $f^\natural$ ist rechtsinvariant, dann ist
\begin{align*}
f^\natural(xt)
=
\int_K f((xt)k)\,dk
=
\int_K f(x(tk))\,dk
=
\int_K f(xk)\,dk
=
f^\natural(x)
\end{align*}
wegen der Linksinvarianz des Haar-Masses auf $K$.
Ähnlich ist die Funktion
\[
\mathstrut^\natural f(tx)
=
\int_K f(k(tx)) \,dk
=
\int_K f((kt)x) \,dk
=
\int_K f(kx) \,dk
=
\mathstrut^\natural f(x)
\]
wegen der Rechtsinvarianz des Haar-Masses auf $K$.
Schliesslich ist die Funktion
\[
\mathstrut^\natural f^\natural (x)
=
\int_K \int_K f(txs) \,ds\,dt
\]
eine biinvariante Funktion auf $G$.

\begin{definition}[Mittelung]
Die Operation $f\mapsto f^\natural$ heisst die {\em Rechtsmittelung},
$f\mapsto\mathstrut^\natural f$ heisst die {\em Linksmittelung} und
$f\mapsto \mathstrut^\natural f^\natural$ die {\em beidseitige
Mittelung} der Funktion $f$.
\index{Rechtsmittelung}%
\index{Linksmittelung}%
\index{Mittelung, beidseitig}%
\end{definition}
