%
% 2-homogen.tex
%
% (c) 2023 Prof Dr Andreas Müller
%
\section{Homogene Räume
\label{buch:nichtkomm:section:homogeneraeume}}
Im vorangegangenen Abschnitt haben wir eine gemeinsame mathematische
Formalisierung für verschiedene Varianten des Registrierungsproblems
gefunden.
In diesem Abschnitt wenden wir uns den Gemeinsamkeiten des 
abstrakten Problems zu und ermitteln die Struktur, die uns später
helfen wird, das Registrierungsproblem auf einheitliche Weise
in einfachere Teilprobleme aufzuteilen.

%
% Stabilisator
%
\subsection{Stabilisator
\label{buch:nichtkomm:homogen:subsection:stabilisator}}
In die grosse Menge der Transformationen des Definitionsbereiches
$X$ lässt sich etwas Ordnung bringen, indem man Fixpunkte dieser
Transformationen sucht.
\index{Fixpunkt}

\begin{definition}
Wenn die Gruppe $G$ auf dem Raum $X$ operiert und $x\in X$ ist, dann
heisst $x$ ein {\em Fixpunkt} einer Transformation $s\in G$, wenn 
\index{Fixpunkt}%
$s\cdot x = x$ ist.
Die Menge
\[
S_x = \{ s\in G \mid s\cdot x = x \}
\]
aller Transformationen, die $x$ als Fixpunkt haben,
heisst der {\em Stabilisator} von $x$.
\index{Stabilisator}%
\end{definition}

Der Stabilisator eines Punktes $x\in X$ ist immer eine Untergruppe.
Dazu muss man nur überprüfen, ob die Zusammensetzung und die Inverse
von Elementen aus $S_x$ wieder in $S_x$ liegt.
Dies ist aber klar, denn wenn zwei Elemente $g,h\in S_x$ den Punkt $x$
festhalten, dann ist auch $(gh)\cdot x= g\cdot(h\cdot x) = g\cdot x = x$
und analog für die Inverse.

\begin{beispiel}
Die Gruppe $\operatorname{SO}(2)\ltimes\mathbb{R}^2$ operiert auf der
Mengen $X=\mathbb{R}^2$.
Nicht jede Transformation hat einen Fixpunkt, zum Beispiel haben reine
Verschiebungen um einen Vektor $v\in\mathbb{R}^2$, $v\ne 0$, keinen
Fixpunkt.
Für eine Transformation $(s,a)\in\operatorname{SO}(2)\ltimes \mathbb{R}^2$
mit $s\ne e$, also eine Transformation mit einer nichttrivialen
Drehkomponente, lässt sich immer ein Fixpunkt durch Lösung der 
Gleichung
\[
sx+a=x
\quad\Rightarrow\quad
x = -(s-e)^{-1}a
\]
ermitteln.
Diese Lösung ist genau dann möglich, wenn $s-e$ invertierbar ist.
Dies kann mit der Determinante der Matrix $D_\alpha-I$ bestimmt werden,
die
\begin{align*}
\det(D_\alpha-I)
&=
(\cos\alpha-1)^2+\sin^2\alpha
\\
&=
\cos^2\alpha-2\cos\alpha +1+\sin^2\alpha
\\
&=
2-2\cos\alpha 
\\
&=
2(1-\cos\alpha)
\end{align*}
ist.
Diese verschwindget genau dann, wenn $\alpha$ ein Vielfaches von $2\pi$ ist.
Damit ist gezeigt, dass alle Transformationen mit einer nichttrivialen
Drehkomponente einen Fixpunkt haben.

Der Stabilisator eines Punktes $x\in\mathbb{R}$ besteht aus denjenigen
$(g,a)\in\operatorname{SO}(2)\ltimes \mathbb{R}^2$, für die
\[
(g,a)\cdot x = g\cdot x+a = x
\]
gilt.
Daraus leitet man die Bedingung 
\[
(g-e)\cdot x = -a
\]
ab.
Für jede Drehung $g\in\operatorname{SO}(2)$ gibt es also einen Vektor
$a=-(g-e)$ derart, dass $(g,a)\in\operatorname{SO}(2)\ltimes\mathbb{R}^2$
den Punkt $x$ als Fixpunkt hat.
Die Abbildung $g\mapsto(g,-gx+x)$ ist daher ein Isomorphismus von
$\operatorname{SO}(2)$ auf den Stabilisator
$S_x\subset \operatorname{SO}(2)\ltimes \mathbb{R}^2$.
\end{beispiel}

\begin{beispiel}
Die Gruppe $G=\operatorname{SO}(2)$ operiert auf der Kugeloberfläche $S^2$.
Der Stabilisator eines Punktes $x\in S^2$ auf der Kugeloberfläche besteht
aus allen Drehungen, die die Achse mit Richtung $x$ oder $-x$ fest
lassen.
Die Untergruppe der Drehungen um die Achse ist isomorph zur Gruppe der
zweidimensionalen Drehmatrizen.
Ein Isomorphismus kann wie folgt konstruieren.
Zunächst wählt man zwei Einheitsvektoren $b_1,b_2\in \mathbb{R}^3$ derart,
dass $b_1,b_2$ und $x$ eine orthonormierte Basis bilden.
In dieser Basis können Drehungen $s$ um die Achse $x$ durch Matrizen der Form
\[
s
=
\left(
\begin{array}{cc|c}
\cos\alpha & -\sin\alpha & 0 \\
\sin\alpha &  \cos\alpha & 0 \\
\hline
     0     &       0     & 1
\end{array}
\right)
=
\left(
\begin{array}{cc|c}
\multicolumn{2}{c|}{
\multirow{2}{*}{$D_\alpha$}
}&0\\
&&0\\
\hline
0&0&1
\end{array}
\right)
\quad\text{mit}\quad
D_\alpha\in\operatorname{SO}(2)
\]
dargestellt werden.
Die Abbildung $D_\alpha\mapsto s$ ist ein Isomorphismus von
$\operatorname{SO}(2)$ auf den Stabilsator $S_x\subset\operatorname{SO}(3)$.
\end{beispiel}

%
% Homogener Raum
%
\subsection{Homogener Raum
\label{buch:nichtkomm:homogen:subsection:homogen}}
Die Definitionsgebiete $X$, die wir im Registrierungsproblem untersuchen,
haben noch eine zusätzliche Eigenschaft.
Für jedes Paar von Punkten $x$ und $y$ in $X$ lässt sich immer mindestens
eine Transformation $g\in G$ finden derart, dass $g\cdot x = y$ ist.
Man sagt, die Gruppe $G$ operiert {\em transitiv} auf $X$.

Typischerweise gibt es nicht nur ein Gruppenelement $g$, welches $x$ in
$y$ transportiert.
Ist $s\in S_x$, dann ist auch $gsx=gx=y$ und für $t\in S_y$ ist
auch $tgx=ty=y$.

\begin{satz}
Falls $gx=hx=x$, dann gibt es eine Element $s\in S_x$ derart, dass
$gs=h$.
\end{satz}

\begin{proof}[Beweis]
Aus $gx=hx$ folgt $x=g^{-1}hx$, also $s=g^{-1}h\in S_x$.
Ausserdem ist $gs=gg^{-1}h=h$, wie verlangt.
\end{proof}

\begin{satz}
Wenn die Gruppe $G$ transitiv auf dem Raum $X$ operiert und $x,y\in X$
zwei Punkte in $X$ sind, dann sind die Stabilisatoren $S_x$ und $S_y$
isomorp.
Ist $gx=y$ dann ist die Abbildung
\[
c_g\colon
S_y\to S_x: h\mapsto ghg^{-1}
\]
ein Isomorphismus der Stabilisatoren.
\end{satz}

\begin{proof}[Beweis]
Zunächst können wir leicht nachrechnen, dass $c_g$ ein Homomorphismus ist,
denn
\[
c_g(hk)
=
ghkg^{-1}
=
gh(g^{-1}g)kg^{-1}
=
(ghg^{-1})gkg^{-1})
=
c_g(h)c_g(k).
\]
Wir müssen überprüfen, dass $c_g(s)\in S_y$ ist, wenn $s\in S_x$ ist.
Dazu rechnen wir nach, dass
\begin{equation}
c_g(s)y = gsg^{-1} y = gsx=gx=y
\quad\Rightarrow\quad
c_g(s)\in S_x.
\label{buch:nichtkomm:homogen:eqn:cgimage}
\end{equation}
Der Homomorphismus $c_g$ ist aber auch umkehrbar, denn es gilt
\[
c_gc_{g^{-1}}(h)
=
g(g^{-1}hg)g^{-1}h
=
h
\quad\Rightarrow\quad
c_gc_{g^{-1}}=\operatorname{id}.
\]
Aus \eqref{buch:nichtkomm:homogen:eqn:cgimage} folgt jetzt, dass
$c_{g^{-1}}\colon S_y\to S_x$.
Somit ist $c_g$ ein Isomorphismus.
\end{proof}

Der Satz besagt also, dass in jedem Punkt des Raumes $X$ der gleiche
Stabilisator gefunden wird.
Mit den  Mitteln der Gruppentheorie von $G$ lassen sich zwei Punkte 
in $X$ nicht unterscheiden.
Diese spezielle Situation hat einen Namen.

\begin{definition}
Ein Raum $X$, auf dem eine Gruppe $G$ transitiv wirkt, heisst
{\em homogen}, wenn der Stabilisator in jedem Punkt von $x$ gleich ist.
\index{homogener Raum}%
\end{definition}

Da die Gruppe $G$ transitiv auf $X$ wirkt, kann man die Punkte $y$ von 
$X$ auch durch die Transformationen bechreiben, die einen gewählten
Punkte $x$ in $y$ überführen.

\begin{beispiel}
Sei $x$ der Nordpol der zweidimensionalen Kugeloberfläche $S^2$.
Ist $y\in S^2$ ein weiterer Punkt, gibt es eine Drehung um die Achse
$x\times y$, die den Nordpol in den Punkt $y$ transportiert.
Alle anderen Drehungen, die den Nordpol in den Punkt $y$ überführen,
bestehen aus einer Drehung um die Nord-Süd-Achse gefolgt von der
genannten Drehung um $x\times y$.
\end{beispiel}

%
% Gruppenoperation auf $G/K$
%
\subsection{Gruppenoperation auf $G/K$
\label{buch:nichtkomm:homogen:subsection:opaufgk}}



