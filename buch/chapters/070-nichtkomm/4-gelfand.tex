%
% 4-gelfand.tex
%
% (c) 2022 Prof Dr Andreas Müller, OST Ostschweizer Fachhochschule
%
\section{Gelfand-Paare
\label{buch:nichtkomm:section:gelfand}}
\kopfrechts{Gelfand-Paare}
Die Theorie der Gelfand-Transformation von Abschnitt
\ref{buch:gruppen:section:gelfand}
kann nur Information über kommutative Algebren liefern.
Daher war sie für die Konstruktion der Fourier-Transformation
auf abelschen Gruppen auch erfolgreich.
Die Fourier-Umkehrformeln beweisen, dass durch die Gelfand-Transformation
keine Information über die Funktionen auf einer abelschen Gruppen
verloren gehen.

\begin{definition}
Sei $G$ eine unimodulare lokalkompakte topologische Gruppe und
$K$ eine kompakte Untergruppe von $G$.
Das Paar $(G,K)$ heisst ein {\em Gelfand-Paar} wenn, die Algebra
$\mathscr{K}(K\backslash G/K)$ der bezüglich $K$ biinvarianten Funktionen
kommutativ ist.
\end{definition}

Für abelsche Gruppen $G$ ist das Paar $(G,\{e\})$ mit der trivialen
Untergruppe $K=\{e\}$ ein Gelfand-Paar.
die zugehörige Theorie der harmonischen Analysis ist die Fourier-Theorie.

Die früher untersuchten Paare $(\operatorname{SO}(3),\operatorname{SO}(2))$
und $(\operatorname{SO}(2)\ltimes \mathbb{R}^2,\operatorname{SO}(2))$
sind Gelfand-Paar.
Ausserdem kann man zeigen, dass $(\operatorname{SO}(n+1),\operatorname{SO}(n))$
für alle $n\ge 2$ ein Gelfand-Paar ist.
Es verallgemeinert die Aufgabenstellung des Registrierungsproblems
auf einer Kugeloberfläche auf Sphären beliebiger Dimension, denn mit
der gleichen Argumentationsweise wie im Fall $n=2$ kann man zeigen,
dass $S^n = \operatorname{SO}(n+1)/\operatorname{SO}(n)$.



