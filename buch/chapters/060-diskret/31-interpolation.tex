%
% 21-interpolation.tex
%
% (c) 2022 Prof Dr Andreas Müller, OST Ostschweizer Fachhochschule
%

%
% Interpolation und die Vandermonde-Matrix
%
\subsection{Interpolation und die Vandermonde-Matrix
\label{buch:diskret:subsection:interpolation}}
Interpolation löst das folgende Problem.

\begin{aufgabe}
Gegeben sind die Zahlen $x_0,\dots,x_n\in \mathbb{R}$ und die
Werte $f_0,\dots,f_n\in\mathbb{R}$.
Man finde ein Polynom $p(x)\in \mathbb{R}[x]$ vom Grad $\le n$ 
mit reellen Koeffizienten derart, dass $p(x_i) = f_i$ für $i=0,\dots,n$.
\end{aufgabe}

Wir schreiben das gesuchte Polynom $p(x)$ Mit den Koeffizienten
$p_,\dots,p_n\in \mathbb{R}$ als
\[
p(x) = p_0 + p_1x + p_2x^2 + \dots p_{n-1}x^{n-1} p_nx^n.
\]
Zu bestimmen sind also die $n+1$ Koeffizienten $p_0,\dots,p_n$.
Dazu sind aus den Datenpunkten $(x_i,f_i)$ Gleichungen zu finden,
für die Koeffizienten zu bestimmen.
Einsetzen der Zahlen $x_i$ in $p(x_i)$ 
\begin{equation}
\def\v{\vdots\phantom{.}}
\begin{array}{rcrcrcccrcrcr}
p_0&+&x_0p_1&+&x_0^2p_2&+&\dots &+&x_0^{n-1}p_{n-1}&+&x_0^np_n&=&f_0\\
p_1&+&x_1p_1&+&x_1^2p_2&+&\dots &+&x_1^{n-1}p_{n-1}&+&x_1^np_n&=&f_1\\
p_2&+&x_2p_1&+&x_2^2p_2&+&\dots &+&x_2^{n-1}p_{n-1}&+&x_2^np_n&=&f_2\\
\v&&\v&& \v& &\ddots\phantom{.}& &          \v& &  \v& &\v\\
p_n&+&x_np_1&+&x_n^2p_2&+&\dots &+&x_n^{n-1}p_{n-1}&+&x_n^np_n&=&f_n
\end{array}
\label{buch:diskret:eqn:interpolation}
\end{equation}
Schreibt man die Koeffizienten $p_0,\dots,p_n$ und die Werte
$f_0,\dots,f_n$ als $n+1$-dimensionale Spaltenvektoren $p$ und $f$,
dann kann das Gleichungssystem~\eqref{buch:diskret:eqn:interpolation}
in Matrixform als $Xp=f$ mit der Matrix
\[
X
=
\renewcommand\arraystretch{1.15}
\begin{pmatrix}
1&x_0&x_0^2&\dots&x_0^{n-1}&x_0^n\\
1&x_1&x_1^2&\dots&x_1^{n-1}&x_1^n\\
1&x_2&x_2^2&\dots&x_2^{n-1}&x_2^n\\[-2pt]
\vdots&\vdots&\vdots&\ddots&\vdots&\vdots\\
1&x_n&x_n^2&\dots&x_n^{n-1}&x_n^n
\end{pmatrix}
\]
geschrieben werden.

\begin{definition}
Die $n\times n$-Matrix
\[
V(x_1,\dots,x_n)
=
\renewcommand\arraystretch{1.15}
\begin{pmatrix}
1&x_1&x_1^2&\dots&x_1^{n-2}&x_1^{n-1}\\
1&x_2&x_2^2&\dots&x_2^{n-2}&x_2^{n-1}\\
\vdots&\vdots&\vdots&\ddots&\vdots&\vdots\\
1&x_n&x_n^2&\dots&x_n^{n-2}&x_n^{n-1}
\end{pmatrix}
\in
M_{n\times n}(\mathbb R)
\]
heisst die {\em Vandermonde-Matrix}.
Die {\em Vandermonde-Determinante} ist die Determinante
$\det V(x_1,\dots,x_n)$ der Vandermonde-Matrix.
\end{definition}

Die Matrix $X$ ist also die Vandermonde-Matrix $X=V(x_0,\dots,x_n)$.
Das Interpolationsproblem ist genau dann lösbar, wenn $X$
regulär ist, was gleichbedeutend ist damit, dass die
Vandermonde-Determinante von $0$ verschieden ist.
Glücklicherweise ist dies unter den gegebenen Voraussetzungen immer
der Fall, denn man kann den folgenden Satz beweisen.

\begin{satz}
\label{buch:diskret:vandermonde:satz:vandermonde}
Der Wert der Vandermonde-Determinante ist
\[
\det V(x_1,\dots,x_n)
=
\prod_{1\le i < k\le n} (x_k-x_i).
\]
Insbesondere ist die Vandermonde-Matrix genau dann regulär, wenn
alle Zahlen $x_i$ verschieden sind.
\end{satz}

