%
% 6-unschaerfe.tex
%
% (c) 2023 Prof Dr Andreas Müller
%
\section{Unschärfe-Relation
\label{buch:diskret:section:unschaerfe}}
Für die Fourier-Transformation auf der Gruppe $G=\mathbb{R}$ wurde die
Heisenberg-Pauli-Weyl-Unschärferelation hergeleitet.
Sie besagt, dass eine Funktion und ihre Fourier-Transformierte nicht
gleichzeitig beliebig gut lokalisiert sein können.
Eine ähliche Eigenschaft gilt auch für Funktionen auf einer endlichen
Gruppe, sie ist der Inhalt des Satzes von Donoho-Stark
\cite{buch:donoho-stark}.
Das früher verwendete Lokalisierungsmass der Varianz lässt sich aber
nicht sinnvoll auf endliche Gruppen übertragen, daher beginnt die 
folgende Diskussion mit einer allgemeinen Diskussion von Lokalisierungsmassen.
Sie stützt sich auf das Paper \cite{buch:widgerson}.

%
% Lokalisierungsmasse
%
\subsection{Lokalisierungsmasse}
Für die Heisenberg-Pauli-Weyl-Unschärferelation verwendet die Varianz
als Mass für die Lokalisierung einer Funktion.
In diesem Abschnitt wird erst gezeigt, warum dieses Mass für Funktionen
auf endlichen Gruppen nicht funktioniert, danach wird ein alternatives
Lokalisierungsmass entwickelt.

\subsubsection{Varianz als Lokalisierungsmass in $\mathbb{R}$}
Zur Konstruktion eines Masses für die Lokalisierung einer Funktion
$f\colon \mathbb{R}\to\mathbb{C}$ wurde zunächst die nichtnegative
Funktion $\varphi(x) = \overline{f}(x)f(x) = |f(x)|^2$ gebildet,
die dann durch Normierung als Wahrscheinlichkeitsdichte interpretiert
wurde.
Die Varianz einer Zufallsvariable mit dieser Wahrscheinlichkeitsverteilung
$\varphi(x)$ ist dann ein mögliches Lokalisierungsmass.

Für die folgende Diskussion nehmen wir an, dass der Erwartungswert $0$
ist, dies lässt sich durch Translation der Funktion $f(x)$ immer erreichen.
Und auch für die Fourier-Transformierte $\hat{f}$ lässt sich dies
durch Multiplikation mit einem Phasenfaktor erreichen, der den
Erwartungswert von $f$ nicht ändert.

Die Normierung von $|f(x)|^2$ verwendet das Integral
\[
\int_{-\infty}^\infty \varphi(x) \,dx
=
\int_{-\infty}^\infty |f(x)|^2\,dx
=
\|f\|_2^2,
\]
also die $L^2$-Norm von $f(x)$.
Der Wert der Varianz ist damit
\[
H(f)
=
\int_{-\infty}^{\infty}
x^2
\frac{|f(x)|^2}{\|f\|_2^2} \,dx
=
\frac{1}{\|f\|_2^2}
\int_{-\infty}^{\infty} x^2\,|f(x)|^2\,dx.
\]
Dieses Unschärfemass lässt sich aber nicht auf periodische Funktionen
oder auf einen diskreten Definitionsbereich verallgemeinern.

Für periodische Funktionen bereitet bereits die Definition
des Erwartungswertes bereitet Schwierigkeiten.
Wenn $f(x)$ eine $2\pi$-periodische stetige Funktion ist, dann ist
\[
E(\delta)
=
\delta\mapsto 
\frac{1}{\|f\|_2^2}
\int_{-\pi+\delta}^{\pi+\delta}
x^2
|f(x)|^2
\,dx
\]
eine stetige Funktion mit der Eigenschaft $E(\delta+\pi)=E(\delta)+\pi$.
Der Wert $E(\delta)$ ist der Erwartungswert einer Zufallsvariablen
mit der Wahrscheinlichkeitsdichte $|f(x)|^2/\|f\|_2$ im Intervall
$[-\pi+\delta,\pi+\delta]$.
Es gibt also keinen einzelnen sinnvollen Wert für den Erwartungswert.
Die Translation ist damit nicht möglich.

Für einen diskreten Definitionsbereich wird das Problem noch grösser.
Selbst wenn man sich auf ein Integrationsintervall festlegen könnte,
gäbe es Funktionen mit nicht ganzzahligem Erwartungswert.
Es muss also ein anderer Ansatz gefunden werden.

%
% Lokalisierung und Normen
%
\subsubsection{Lokalisierung und Normen}
Das Lokalisierungsmass soll anzeigen, in welchem Teil des Definitionsbereiches
die grösste ``Masse'' der Funktion gefunden werden kann.
Eine Funktion, deren Funktionswerte alle den gleichen Betrag
haben, hat keinen Argumentwert, der sinnvoll als ``Schwerpunkt''
angesehen werden könnte.

Auf der anderen Seite ist eine Funktion, die nur einen von $0$
verschiedenen Wert hat, ganz offensichtlich perfekt lokalisiert.
Die beiden Fälle können durch Vergleich geeigneter Normen 
unterschieden werden.
Sei also $f$ eine Funktion $G\to\mathbb{C}$ mit nur den einen von $0$
verschiedenen Wert $f(x_0)$ hat.
Die Supremum-Norm
\[
\|f\|_\infty
=
\sup_{x\in G} |f(x)|
\]
von $f$ ist der maximale Wert $\|f\|_\infty = |f(x_0)|$.
Eine gar nicht lokalisierte Funktion $g\colon G\to\mathbb{C}$
mit der gleichen Supremum-Norm hätte nur Werte mit Betrag
\[
|g(x)|
=
\|f\|_\infty\quad \forall x\in G.
\]
Ein grosser Unterschied zwischen $f$ und $g$ zeigt sich in
der $L^1$-Norm:
\begin{align*}
\|f\|_1
&=
\sum_{x\in G} |f(x)|
=
|f(x_0)|
=
\|f\|_\infty
&&\Rightarrow& \frac{\|f\|_1}{\|f\|_\infty}&=1
\\
\|g\|_1
&=
\sum_{x\in G} |g(x)|
=
|G|\cdot \|f\|_\infty
&&\Rightarrow& \frac{\|f\|_1}{\|f\|_\infty}&=|G|.
\end{align*}
Die nicht lokalisierte Funktion $f$ hat den minimal 
möglichen Wert $1$ und die lokalisierte Funktion $g$
hat den maximal möglichen Wert $|G|$.

\begin{definition}
\label{buch:diskret:unschaerfe:def:lokalisierungsmass}
Für eine Funktion $f\colon G\to\mathbb{C}$ ist das Lokalisierungsmass
\[
H(f)
=
\frac{\|f\|_1}{\|f\|_\infty}
=
\frac{\displaystyle\sum_{x\in G} |f(x)|}{\displaystyle\sup_{x\in G} |f(x)|}.
\]
\end{definition}

%
% Das allgemeine Unschärfeprinzip
%
\subsection{Das allgemeine Unschärfeprinzip}
Die Unschärferelation für die Fourier-Transformation vergleicht
die Lokalisierungsmasse der Funktionen $f$ und $\mathscr{F}f$.
As allgemeine Unschärfeprinzip verallgemeinert dies auf eine
Aussage über beliebige lineare Abbildungen $A$.
Da das Lokalisierungsmass die $\|\,\cdot\,\|_1$ und $\|\,\cdot\,\|_\infty$
miteinander vergleicht, muss untersucht werden, wie die Matrix $A$
diese Normen beeinflusst.

\begin{definition}
\label{buch:diskret:unschaerfe:def:normUV}
Seien $U$ und $V$ zwei Vektorräume mit Normen $\|\,\cdot\,\|_U$ und
$\|\,\cdot\,\|_V$ und $A\colon U\to V$ eine lineare Abbildung.
Die Norm von $A$ ist
\begin{equation}
\|A\|_{U\to V}
=
\sup_{u\in U\setminus\{0\}} \frac{\| Au\|_V}{\|u\|_U}
\label{buch:diskret:unschaerfe:normUV}
\end{equation}
\end{definition}

Die Definition
\eqref{buch:diskret:unschaerfe:normUV}
der Norm bedeutet, dass
\[
\|Au\|_V \le \|A\|_{U\to V}\cdot \|u\|_U
\]
für alle $u\in U\setminus\{0\}$.
Da auf den Vektorräumen, auf denen wir das Lokalisierungsmass
berechnen wollen verschiedene Normen definiert sein können,
müssen wir die Definition~\ref{buch:diskret:unschaerfe:def:normUV}.
noch etwas verallgemeinern.

\begin{definition}
\label{buch:diskret:unschaerfe:def:norm1i}
Ist $A\colon U\to V$ eine lineare Abbildung und ist $\|\,\cdot\,\|_p$
die $p$-Norm auf den Vektorräumen $U$ oder $V$, dann setzen wir
\[
\|A\|_{p\to q}
=
\sup_{u\in U\setminus\{0\}}
\frac{\|Au\|_q}{\|u\|_p}.
\]
\end{definition}

Mit diesen Voraussetzungen lässt sich jetzt das primäre Unschärfeprinzip
formulieren \cite{buch:widgerson}.

\begin{satz}[Das primäre Unschärfeprinzip]
\label{buch:diskret:unschaerfe:satz:primaer}
Sind $U$ und $V$ komplexe Vektorräume mit zwei Normen $\|\,\cdot\,\|_\infty$
und $\|\,\cdot\,\|_1$ und $A\colon U\to V$ und $B\colon V\to U$ lineare
Abbildungen mit $\|A\|_{1\to\infty}\le 1$ und $\|G\|_{1\to\infty}\le 1$.
Weiter sei angenommen, dass es eine Konstante $k>0$ gibt mit
$\|BAu\|_\infty \ge k\|u\|_\infty$.
Dann gilt für
\[
\|u\|_1\cdot \|Au\|_1 \ge k\|u\|_\infty\cdot \|Au\|_\infty
\]
alle $u\in U$.
\end{satz}

Die Bedingung für das Produkt $BA$ bedeutet, dass $BA$ invertierbar ist
und dass die Norm der inversen Abbildung
\[
\|(BA)^{-1}\|_{\infty\to\infty}
\ge k
\]
ist.
Insbesondere ist $A$ auch injektiv.

\begin{proof}[Beweis]
Aus $\|A\|_{1\to\infty}\le 1$ und $\|B\|_{1\to\infty}\le 1$ folgen
zunächst die Ungleichungen
\begin{align*}
\|Au\|_\infty &\le \|u\|_1
&&\text{und}&
\|BAu\|_\infty &\le \|Au\|_1.
\end{align*}
Für das Produkt gilt dann
\[
\|Au\|_\infty
\cdot
\|BAu\|_\infty
\le
\|u\|_1
\cdot
\|Au\|_1.
\]
Aus der Bedinung an das Produkt $BA$ kann die linke Seite gegen unten
abgeschätzt werden durch
\[
k\|u\|_\infty
\cdot
\|Au\|_\infty
\le
\|Au\|_\infty
\cdot
\|BAu\|_\infty
\le
\|u\|_1
\cdot
\|Au\|_1,
\]
dies ist die behauptete Ungleichung.
\end{proof}

In Definition \ref{buch:diskret:unschaerfe:def:lokalisierungsmass}
wurde der Quotient von $L^1$-Norm und Supremum-Norm als
Lokalsierungsmass definiert.
Damit kann das Unschärfeprinzip wie folgt umformuliert werden.

\begin{satz}
Unter den Voraussetzungen des
Satzes~\ref{buch:diskret:unschaerfe:satz:primaer}
gilt
\[
H(u)\cdot H(Au)
=
\frac{\|u\|_1}{\|u\|_\infty}
\cdot
\frac{\|Au\|_1}{\|Au\|_\infty}
\ge k
\]
für alle $u\in U\setminus\{0\}$.
\end{satz}

\begin{proof}[Beweis]
Die Behauptung folgt sofort aus Satz~\ref{buch:diskret:unschaerfe:satz:primaer}
durch Division durch $\|u\|_\infty\cdot \|Au\|_\infty$.
Allerdings ist noch zu verifizieren, dass der Faktor $\|Au\|_\infty$
nicht verschwinden kann.
Da aber aus der Voraussetzung an $BA$ folgt, dass $A$ injektiv ist
folgt aus $u\ne 0$, dass auch $Au\ne 0$ und damit $\|Au\|_\infty> 0$.
\end{proof}

%
% $k$-Hadamard-Matrizen
%
\subsection{$k$-Hadamard-Matrizen}
Die Bedingungen an das Produkt $BA$ in
Satz~\ref{buch:diskret:unschaerfe:satz:primaer}
motivieren die folgende Definition.

\begin{definition}
\label{buch:diskret:unschaerfe:def:khadamard}
Eine $n\times n$-Matrix heisst $k$-Hadamard wenn alle Einträge $a_{ik}$ von
$A$ Betrag $|a_{ik}|\le 1$ haben und wenn
\begin{equation}
\|A^*Au\|_\infty \ge k\|u\|_\infty
\label{buch:diskret:unschaerfe:eqn:khadamard}
\end{equation}
gilt für alle $n$-dimensionalen Vektoren.
\end{definition}

Die Bedingung 
\eqref{buch:diskret:unschaerfe:eqn:khadamard}
besagt, dass $A^*$ die Rolle der Matrix $B$ im primären
Unschärfeprinzip von Satz~\ref{buch:diskret:unschaerfe:satz:primaer}
einnehmen kann.

\begin{satz}
Ist $A$ eine $k$-Hadamard-Matrix dann gilt
\[
\|u\|_1\cdot \|Au\|_1 \ge k\|u\|_\infty \cdot \|Au\|_\infty
\]
für alle $u\in U$ oder
\[
H(u)\cdot H(Au)\ge k
\]
für alle $u \in U\setminus\{0\}$.
\end{satz}

Die Fourier-Transformation $\mathscr{F}_n$ ist eine $n\times n$-Matrix,
deren Einträge alle exakt den Betrag $1$ haben.
Ausserdem ist $\mathscr{F}_n^* = \overline{\mathscr{F}}_n$ und aus den 
bekannten Eigenschaften der Fourier-Transformation
\[
\mathscr{F}_n^*\mathscr{F}_n
=
\overline{\mathscr{F}}_n\mathscr{F}_n
=
nI
\qquad\Rightarrow\qquad
\| \mathscr{F}_n^*\mathscr{F}_n f\|_\infty
=
n \|f\|_\infty
\]
für alle $u\in\mathbb{C}^n$.
Die Fourier-Transformation $\mathscr{F}_n$ ist also eine
$n$-Hadamard Matrix.
Damit folgt jetzt der primäre Unschärfesatz für die Fourier-Transformation.

\begin{satz}
Für beliebige $f\in \mathbb{C}^n\setminus\{u\}$ gilt
\begin{equation}
H(u)\cdot H(\mathscr{F}_nu) \ge n.
\end{equation}
\end{satz}

%
% Träger als Lokalisierungsmass
%
\subsection{Träger als Lokalisierungsmass}
Das bisher verwendet Verhältnis $H(f)$ von $L^1$- und Supremum-Norm
eines Vektors $f\in\mathbb{C}^n$ ist etwas aufwendig zu berechnen.
Ein einfacheres Mass ist die Anzahl der von $0$ verschiedenen
Funktionswerte.

\begin{definition}
\label{buch:diskret:unschaerfe:def:support}
Der Träger $\operatorname{supp}f$ einer Funktion $f\colon G\to\mathbb{C}$ ist
\index{Träger}%
die Menge
\[
\operatorname{supp} f
=
\{x\in G\mid f(x)\ne 0\}.
\]
\end{definition}

Die Kardinalität des Trägers einer Funktion kann ebenfalls als
Lokalisierungsmass verwendet werden.
Je kleiner $|\operatorname{supp}f|$ ist, desto besser ist die
Funktion $f$ lokalisiert.

\begin{satz}[Unschärferelation für Träger]
\label{buch:diskret:unschaerfe:satz:supportsize}
Ist $A\in M_{m\times n}$ eine $k$-Hadamard-Matrix, dann gilt
\begin{equation}
|\operatorname{supp}u|
\cdot
|\operatorname{supp}Au|
\ge
k
\end{equation}
für jeden nichtverschwindenden Vektor $u\in\mathbb{C}^m$.
\end{satz}

\begin{proof}[Beweis]
\end{proof}

\begin{satz}[Donoho-Stark]
\label{buch:diskret:unschaerfe:satz:donoho-stark}
Für $f\colon G\to\mathbb{C}\setminus\{0\}$ gilt
\[
|\operatorname{supp}f|
\cdot
|\operatorname{supp}\mathscr{F}_nf|
\ge n.
\]
\end{satz}

\begin{proof}[Beweis]
Der Satz von Donoho-Stark folgt unmittelbar aus
Satz~\ref{buch:diskret:unschaerfe:satz:supportsize} und der
Tatsache, dass die Fourier-Transformation $\mathscr{F}_n$
$n=|G|$-Hadamard ist.
\end{proof}

%
% $\varepsilon$-Träger
%
\subsection{$\varepsilon$-Träger}


	
