%
% 26-faktorisierung.tex
%
% (c) 2023 Prof Dr Andreas Müller
%

%
% Faktorisierung für n = pq
%
\subsection{Allgemeine Faktorisierung für $n=pq$}
Durch Nachrechnen kann man jetzt nachrechnen, dass
\[
\mathscr{F}_n = A(p,q,\omega) F(p,q,\omega)
\qquad\text{mit $\omega^{pq}=1$}.
\]
Mit den früher gefundenen Resultaten über die beiden Faktoren
$A(p,q,\omega)$ und $F(p,q,\omega)$ ist damit der folgende
Satz bewiesen.

\begin{satz}
\label{buch:diskret:vandermonde:satz:fourierfaktorisierung}
Ist $n=pq$ und $\omega=\omega_n=e^{-2\pi i/n}$,
dann kann die Fourier-Transformation $\mathscr{F}_n$
faktorisiert werden in
\[
\mathscr{F}_n
=
A(p,q,\omega)
(
\mathscr{F}_p
\otimes
I_q
).
\]
Ebenso gilt
\[
\overline{\mathscr{F}}_n
=
A(p,q,\overline{\omega})
(
\overline{\mathscr{F}}_p
\otimes
I_q
).
\]
Die inverse Fourier-Transformation wird faktorisiert in
\[
\mathscr{F}_n^{-1}
=
\frac{1}{n}\overline{\mathscr{F}}_n
=
\frac{1}{q} A(p,q,\overline{\omega})
\biggl(
\frac1p\overline{\mathscr{F}}_p
\otimes
I_q
\biggr)
=
\frac{1}{q} A(p,q,\overline{\omega})
(\mathscr{F}_p^{-1} \otimes I_q).
\]
\end{satz}

