%
% Fourier-Transformation und die Vandermonde-Matrix
%
\subsection{Fourier-Transformation und die Vandermonde-Matrix
\label{buch:diskret:subsection:fourier}}
Die Fourier-Transformation der Zahlen $f_0,\dots,f_{n-1}$ ist
gegeben durch die Summe
\begin{equation}
\hat{f}_k
=
(\mathscr{F}f)_k
=
\sum_{x=0}^{n-1} e^{-2\pi i kx/n} f_x
\label{buch:diskret:vandermonde:eqn:fourier}
\end{equation}
und die Umkehrtransformation
\begin{equation}
f_x
=
\frac{1}{n}
(\overline{\mathscr{F}}\hat{f})_x
=
\frac{1}{n}
\sum_{k=0}^{n-1} e^{2\pi i kx/n} \hat{f}_k.
\label{buch:diskret:vandermonde:eqn:ifourier}
\end{equation}
Mit $\omega= e^{-2\pi i/n}$ können
\eqref{buch:diskret:vandermonde:eqn:fourier}
und
\eqref{buch:diskret:vandermonde:eqn:ifourier}
kompakter als
\[
\hat{f}_k
=
\sum_{x=0}^{n-1} \omega^{kx}f_x
\qquad\text{und}\qquad
f_x
=
\sum_{k=0}^{n-1} \overline{\omega}^{kx}\hat{f}_k
\]
geschrieben werden.

Schreibt man $f_x$ und $\hat{f}_k$ als Spaltenvektoren, können die
Transformationen $\mathscr{F}$ und $\overline{\mathscr{F}}$ als
Matrizen geschrieben werden, sie sind
\bgroup
\def\v{\vdots}
\def\o{\overline{\omega}}
\renewcommand\arraystretch{1.15}
\begin{align*}
\mathscr{F}
&=
\begin{pmatrix}
 1&     1      &       1&       1&\dots &           1   &           1   \\
 1&\omega      &\omega^2&\omega^3&\dots &\omega^{n-2}   &\omega^{n-1}   \\
 1&\omega^2    &\omega^4&\omega^6&\dots &\omega^{2(n-2)}&\omega^{2(n-1)}\\
 1&\omega^3    &\omega^6&\omega^9&\dots &\omega^{3(n-2)}&\omega^{3(n-1)}\\[-2pt]
\v&\v&\v&\v&\ddots&\v&\v\\
 1&\omega^{n-1}&\omega^{2(n-1)}&\omega^{3(n-1)}&\dots &\omega^{(n-1)(n-2)}&\omega^{(n-1)^2}
\end{pmatrix}
\\
\overline{\mathscr{F}}
&=
\begin{pmatrix}
 1&     1  &       1   &   1       &\dots &           1   &       1   \\
 1&\o      &\o^2       &\o^3       &\dots &\o^{n-2}       &\o^{n-1}   \\
 1&\o^2    &\o^4       &\o^6       &\dots &\o^{2(n-2)}    &\o^{2(n-1)}\\
 1&\o^3    &\o^6       &\o^9       &\dots &\o^{3(n-2)}    &\o^{3(n-1)}\\[-2pt]
\v&\v&\v&\ddots&\v&\v\\
 1&\o^{n-1}&\o^{2(n-1)}&\o^{3(n-1)}&\dots &\o^{(n-1)(n-2)}&\o^{(n-1)^2}
\end{pmatrix}
\end{align*}
\egroup
Beide Matrizen sind Vandermonde-Matrizen
\begin{align}
\mathscr{F}
&=
V(1,\omega,\omega^2,\omega^3,\dots,\omega^{n-1})
\label{buch:diskret:vandermonde:eqn:vanderfourier}
\\
\overline{\mathscr{F}}
&=
V(1,\overline{\omega},\overline{\omega}^2,\overline{\omega}^3,\dots,\overline{\omega}^{n-1})
\label{buch:diskret:vandermonde:eqn:vanderifourier}
\end{align}
Da die Potenzen $1=\omega^0,\omega,\omega^2,\dots,\omega^{n-1}$ alle
verschieden sind, folgt nach
Satz~\ref{buch:diskret:vandermonde:satz:vandermonde},
dass $\mathscr{F}$ und $\overline{\mathscr{F}}$ invertierbar sind.
