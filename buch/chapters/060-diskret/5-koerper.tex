%
% 5-koerper.tex
%
% (c) 2022 Prof Dr Andreas Müller, OST Ostschweizer Fachhochschule
%
\section{Fourier-Transformation in endlichen Körpern
\label{buch:diskret:section:koerper}}
\kopfrechts{Endliche Körper}
Sowohl die Fourier-Transformation wie auch die Faktorisierung
der Fourier-Matrix beschränkt sich nicht auf die komplexen Zahlen
als Wertebereich für die Funktionen.
Damit wird es möglich, die Fourier-Transformation zur Codierung von
digitalen Signalen oder Bildern zu verwenden, die dann auch verlustlos
rekonstruiert werden können.
Sie eignen sich aber nicht für Approximationen wie sie zum Beispiel
bei der JPEG-Codierung verwendet werden sollen.
\index{JPEG}%
Diese beruhen darauf, dass hochfrequente Komponenten in der
Fourier-Transformierten weggelassen werden können, weil sie für die
meisten Bilder ``klein'' sind.
In einem endlichen Körper gibt es kein sinnvolles Konzept ``kleiner''
Werte.

%
% Endliche Körper
%
\subsection{Endliche Körper
\label{buch:diskret:koerper:subsection:koerper}}
Die diskrete Fourier-Transformation multipliziert Komponenten des
Input-Vektors mit den Potenzen von $\omega$ und addiert die erhaltenen
Produkte.
Ein alternativer Zahlenbereich, in dem die Fourier-Transformation
definiert werden kann, muss also mindestens eine Gruppe bezüglich 
der Addition sein und ausserdem über eine Multiplikation verfügen.
Eine solche Struktur ist bekannt als ein {\em Ring} $R$.
Genauer, ein Ring mit Eins.

%
% Nullteiler
%
\subsubsection{Nullteiler}
Das Rechnen mit Resten zeigt, dass die Menge $\mathbb{Z}/k\mathbb{Z}$
der Reste modulo $k$ ein Ring mit Eins ist.
Dieser Ring hat jedoch nicht immer für die vorliegenden Zwecke günstige
Eigenschaften.
Wenn zum Beispiel $k=p_1\cdot p_2$ ein Produkt von zwei Primzahlen
ist, dann sind die Reste $p_1$ und $p_2$ in $\mathbb{Z}/k\mathbb{Z}$
von $0$ verschieden, aber deren Produkt ist
$p_1\cdot p_2 = k \equiv 0 \mod k$.
Die Zahlen $p_1$ und $p_2$ heissen {\em Nullteiler}.
\index{Nullteiler}%
Die Existenz von Nullteilern hat zur Folge, dass Gleichungen der
Form $p_1 x=b$ sich in $\mathbb{Z}/k\mathbb{Z}$ nicht mehr
eindeutig lösbar sind.
Ist nämlich $x_1$ eine Lösung, erfüllt also $p_1x_1=b$, dann ist
$p_1(x_1+p_2)=p_1x_1+p_1p_2=p_1x_1=b$ und damit ist auch $x_2=x_1+p_2$ 
eine Lösung.

Die umgekehrte Schlussfolgerung ist auch gültig.
Wenn die Gleichung $ax=b$ mit $a\ne 0$ zwei verschiedene Lösungen
$x_1$ und $x_2$ hat, dann ist
\[
ax_1=b
\wedge
ax_2=b
\quad\Rightarrow\quad
a(\underbrace{x_1-x_2}_{\displaystyle\ne 0})=0,
\]
die Zahl $a$ ist daher ein Nullteiler.
Es folgt, dass die Multiplikation mit $a$ genau dann eindeutig umkehrbar
ist, wenn $a$ kein Nullteiler ist.

%
% Division durch die Vektorlänge
%
\subsubsection{Division durch die Vektorlänge}
Die Umkehrung der Fourier-Transformation erfordert aber zusätzlich,
dass durch die Länge des Vektors dividiert werden kann.
Es muss also möglich sein, die natürliche Zahl $n$, die die Länge
des Vektors angibt, in den Zahlenbereich abzubilden.
Da $n = 1+\dots + 1$ die Summe von $n$ Einsen in $\mathbb{N}$ ist,
kann man $n$ auf die Summe $1+\dots+1$ von $n$ Einsen im Ring abbilden.
Das Bild von $n$ in $R$ ist also wohldefiniert.

Durch $n \in R$ muss jetzt aber dividiert werden können.
Dazu ist zunächst notwendig, dass $n\ne 0$ ist.
In der Digitaltechnik ist es naheliegend, die Menge $\mathbb{Z}/2\mathbb{Z}$
der Reste modulo $2$ zu verwenden.
Für die bei der schnellen Fourier-Transformation besonders wichtige
Anzahl $n=2^t$, die eine Zweierpotenz ist, ist dann aber $2^t=0$ in
$\mathbb{Z}/2\mathbb{Z}$.
Die Reste modulo $2$ und damit binäre Bits kommen daher als alternatives
Zahlensystem für die Fourier-Transformation nicht in Frage.

Dass $n\ne 0$ ist, reicht aber noch nicht.
Wenn $n\in \mathbb{Z}/k\mathbb{Z}$ ein Nullteiler hat, dann hat die
Gleichung $nx=b$ keine eindeutige Lösung und es ist nicht sinnvoll
von $x = b/n$ zu sprechen.
Die Fourier-Transformation kann also nur invertiert werden, wenn
$n\in \mathbb{Z}/k\mathbb{Z}$ nicht Nullteiler ist.
Erst recht ist dies der Fall, wenn $\mathbb{Z}/k\mathbb{Z}$ keine
Nullteiler hat und $n$ nicht durch $k$ teilbar ist.
Am meisten freie Wahl von $n$ hat man also, wenn $k$ eine Primzahl
ist, dann muss nur noch verlangt werden, dass sie $n$ nicht teilt.

%
% Die endlichen Körper F_p
%
\subsubsection{Die endlichen Körper $\mathbb{F}_p$}
Ist $p$ eine Primzahl, dann wird die
die Menge $\mathbb{Z}/p\mathbb{Z}$ auch mit $\mathbb{F}_p$
bezeichnet.
Sie ist nicht nur ein Ring, da sie keine Nullteiler enthält, ist auch
die Division durch jeden von $0$ verschiedene Rest eindeutig möglich.
Die Menge $\mathbb{F}_p$ ist daher nicht nur eine Gruppe bezüglich
der Addition, sondern $\mathbb{F}_p^*=\mathbb{F}_p\setminus\{0\}$
ist auch eine Gruppe bezüglich der Multiplikation, jedes Element 
in $\mathbb{F}_p^*$ kann bezüglich der Multiplikation invertiert
werden.
Man nennt eine solche Menge einen {\em Körper}.
Der Körper $\mathbb{F}_p$ heisst auch der {\em Galois-Körper} 
der Primzahlordnung $p$.
\index{Galois-Körper}%

Die Bestimmung des multiplikativen Inversen eines Elements
$a\in\mathbb{F}_p$ ist gleichbedeutend damit, dass eine Zahl
$s$ gefunden werden muss derart, dass $as \equiv 1\mod p$.
Eine simplizistische Möglichkeit, dieses Problem zu lösen, ist
die Menge $\mathbb{F}_p$ nach so einem Element zu durchsuchen.
Eine bessere Methode ist zu verwenden, dass es gibt dann eine
Zahl $t$ geben muss derart, dass $as+pt=1$.
Da $a<p$ und $p$ eine Primzahl ist, sind $a$ und $p$ teilerfremd.
Der euklidische Algorithmus besagt dann, dass es die Zahlen $s$ und $t$
geben muss.
Der erweiterte euklidische Algorithmus ermöglicht auch, diese Zahlen 
effizient zu finden \cite[Abschnitt~3.C]{buch:linalg}.

%
% Hinzufügen einer Nullstelle
%
\subsubsection{Hinzufügen einer Nullstelle}
Die Körper $\mathbb{F}_p$ sind leider nicht gross genug, um als Wertebereich
für die Fourier-Transformation zu dienen.
Die Schwierigkeit rührt daher, die Fourier-Transformation die Grösse
$\omega\ne\pm1$ benötigt, deren $n$-te Potenz $1$ ist.
Für die Vektorlänge $n=2^3=8$ und den Körper $\mathbb{F}_7$ zum Beispiel
gibt es keine solchen Elemente, denn die achten Potenzen aller Elemente
von $\mathbb{F}_7$ sind
\begin{center}
\begin{tabular}{>{$}l<{$}|cccccc}
 x & 1& 2& 3& 4& 5& 6\\
\hline
x^8& 1& 4& 2& 2& 4& 1
\end{tabular}
\end{center}
Das Element $6\in\mathbb{F}_7$ ist $6=-1$, kommt also für $\omega$
nicht in Frage.

Die Schwierigkeit ist mit dem Problem verwandt, dass die Gleichung
$\omega^n=1$ auch im Körper $\mathbb{R}$ der reellen Zahlen nicht
genügend verschiedene Lösungen hat.
Das Problem konnte erst gelöst werden, indem die reellen Zahlen
zum Körper der komplexen Zahlen $\mathbb{C}$ erweitert wurden.
Noch etwas treffender ist der Vergleich mit dem Körper $\mathbb{Q}$
der rationalen Zahlen, dem kleinsten Körper, der die ganzen Zahlen
enthält.
Der Körper $\mathbb{Q}$ enthält zum Beispiel die Zahl $\!\sqrt{2}$ nicht,
kann aber zu einem Körper erweitert werden, der die Zahl $\!\sqrt{2}$
enthält.
Dazu verwendet man ein neues Symbol $\alpha$, mit dem wie mit
rationalen Zahlen gerechnet werden kann, welches aber die zusätzliche
Eigenschaft $\alpha^2-2=0$ erfüllt.
Die Menge
\[
\mathbb{Q}(\alpha)
=
\{
a+b\alpha\mid a,b\in\mathbb{Q}
\}
\]
ist dann ein Körper, der $\!\sqrt{2}=\alpha$ enthält.
Er wird auch mit $\mathbb{Q}(\!\sqrt{2})$ bezeichnet.

Die Konstruktion von $\mathbb{Q}(\!\sqrt{w})$ kann auch auf endliche
Körper übertragen werden.
Zum Beispiel gilt für alle von $0$ verschiedenen Elemente $x$ des Körpers
$\mathbb{F}_5$, dass $x^2\ne 2$ ist.
$\mathbb{F}_5$ enthält also die Quadratwurzel $2$ nicht.
Fügt man wieder das Symbol $\alpha$ mit der Eigenschaft $\alpha^2-2=0$
hinzu, wird
\[
\mathbb{F}_5(\alpha)
=
\{
a+b\alpha
\mid
a,b\in\mathbb{F}_5
\}
=
\mathbb{F}_5(\alpha)
\]
zu einem Körper, der ein Element enthält, welches als Quadratwurzel von
$2$ betrachtet werden kann.

Der Prozess kann verallgemeinert werden auf die Situtation,
in der dem Körper $\mathbb{F}_p$ eine Nullstelle $\alpha$ eines beliebigen
Polynoms hinzugefügt werden soll.
Damit die Menge
\[
L
=
\{
a_{n-1}\alpha^{n-1} + a_{n-2}\alpha^{n-2} + \dots + a_1\alpha + a_0
\mid
a_0,\dots,a_{n-1}\in\mathbb{F}_p
\}
\]
ein Ring wird, muss sichergestellt sein, dass sich $\alpha^n$ durch
die kleineren Potenzen ausdrücken lässt.
Es gibt also ein Polynom $m(X)\in\mathbb{F}_p$ derart, dass
\[
m(\alpha)
=
\alpha^n + m_{n-1}\alpha^{n-1} + \dots + m_2\alpha^2 + m_1\alpha + m_0
=
0
\]
ist.

Die Menge $L$ kann auch betrachtet werden als die Menge der Polynomereste
\[
L
=
\mathbb{F}_p[X] / m(X)\mathbb{F}_p[X]
\]
bei Polynomdivision durch $m(X)$.
Falls sich das Polynom $m(X)$ in zwei Faktoren $p(X),q(X)\in\mathbb{F}_p[X]$
zerlegen lässt, also $m(X)=p(X)q(X)$, dann sind $p,q\in L$ Nullteiler.
Wie bei der Konstruktion der Körper $\mathbb{F}_p$ muss also sichergestellt
werden, dass sich das Polynom $m(X)$ nicht faktorisieren lässt.

\begin{definition}[irreduzibeles Polynom]
Ein Polynom $p(X)\in K[X]$ mit Koeffizienten in einem Körper $K$ heisst
{\em ir\-re\-du\-zi\-bel}, wenn es keine Faktorisierung von $p(X)$ in Faktoren
\index{irreduzibel}%
$p(X)=a(X)b(X)$ mit $a(X),b(X)\in K[X]$ gib.t
\end{definition}

%
% Die Körper F_{p^l}
%
\subsubsection{Die endlichen Körper $\mathbb{F}_{p^l}$}
Zu einem beliebigen irreduziblen Polynom $m(X)\in K[X]$ kann jetzt
ein Körper konstruiert werden, der ein Element enhält, welches Nullstelle
des Polynoms $m(X)$ ist.

\begin{satz}[endlicher Körper]
Ist $m(X)\in K[X]$ ein irreduzibles Polynom mit Koeffizienten in einem
Körper $K$, dann ist die Menge
\[
K(\alpha)
=
K[X] / m(X) K[X]
\]
ein Körper, der den Körper $K$ enthält.
Die Restklasse des Polynoms $X$ wird mit $\alpha$ bezeichnet und hat
die Eigenschaft, dass $m(\alpha)=0$ ist.
\end{satz}

\begin{proof}[Beweis]
Aus der Forderung, dass $m(X)$ ein irreduzibles Polynom ist, folgt
bereits, dass $K(\alpha)$ ein nullteilerfreier Ring ist.
Es ist nur noch sicherzustellen, dass es zu jedem von $0$ verschiedenen
Polynom $a(X)\in K(\alpha)$ ein Polynom $s(X)$ gibt derart, dass
$a(X)s(X) \equiv 1 \mod m(X)$ ist.
Dies ist gleichbedeutend damit, dass es ein Polynom $t(X)$ mit
\[
a(X)s(X) + t(X) m(X) = 1
\]
ist.
Da $\deg a(X)<\deg m(X)$ ist und $m(X)$ als irreduzibles Polynom
keine Teiler hat, sind die Polynome $a(X)$ und $m(X)$ teilerfremd.
Nach dem euklidischen Algorithmus gibt es daher tatsächlich Polynome 
$s(X),t(X)\in \mathbb{K}[X]$ der verlangten Art.
Sie können mit dem verallgemeinerten euklidischen Algorithmus
gefunden werden \cite{buch:linalg}.
\end{proof}

Das Polynom $m(X)=X^2+1\in\mathbb{R}[X]$ ist irreduzibel, dann die
einzig mögliche Faktorisierung wäre von der Form $(X-a)(X-b)$, dann
wären $a$ und $b$ Nullstellen des Polynoms.
Wegen $x^2+1 >0$ für alle $x\in\mathbb{R}$ hat das Polynom aber keine
Nullstellen.
Der Erweiterungskörper $\mathbb{R}(\alpha)$ enthält das Element $\alpha$,
welches $\alpha^2+1=0$ erfüllt, also die definierende Gleichung der
imaginären Einheit.
Der Erweiterungskörper ist als der Körper der komplexen Zahlen.

Für den endlichen Körper $K=\mathbb{F}_p$ entstehen die Körper
$\mathbb{F}_{p^l}$.

\begin{satz}[Körper $\mathbb{F}_{p^l}$]
\label{buch:diskret:koerper:satz:fpl}
Ist $m(X)\in \mathbb{F}_p[X]$ ein irreduzibles Polynom vom Grad $l$ mit
Koeffizienten $\mathbb{F}_p$, dann ist $\mathbb{F}_{p}(\alpha) = \mathbb{F}_p/m(X)\mathbb{F}_p$ ein Körper mit $p^l$-Elementen.
\end{satz}

\begin{definition}[Körper $\mathbb{F}_{p^l}$]
\label{buch:diskret:koerper:def:fpl}
Der Körper von Satz~\ref{buch:diskret:koerper:satz:fpl} heisst der
endliche Körper oder Galois-Körper der Ordnung $p^l$.
\end{definition}

%
% Forier-Transformation
%
\subsection{Fourier-Transformation
\label{buch:diskret:koeper:subsection:fourier}}
Die Fourier-Transformation verwendet die duale Gruppe $\hat{G}$ einer
Gruppe $G$, also die Menge der Homomorphismen $G\to \mathbb{C}$.
Für die Fourier-Transformation über einen endlichen Körper
$\mathbb{F}_{p^l}$ sind Homomorphismen
\(
G \to \mathbb{F}_{p^l}
\)
gesucht.

Für die zyklischen Gruppen $G=\mathbb{Z}/n\mathbb{Z}$ sind diese wie
im komplexen Fall durch den Wert auf dem erzeugenden Element
$1\in\mathbb{Z}/n\mathbb{Z}$ gegeben.
Gesucht ist daher ein Element $\omega\in\mathbb{F}_{p^l}$ derart,
dass $\omega^n=1$.
Ausserdem müssen alle Potenzen $\omega^k$ mit $k=0,\dots,n-1$
verschieden sein.
Das Elemente $\omega\in\mathbb{F}_{p^l}$ muss also eine
zyklische Untergruppe der Ordnung $n$ in $\mathbb{F}_{p^l}$ 
erzeugen.
Da das Polynom $X^n-1$ immer als
\[
X^n-1
=
(X-1)(X^{n-1}+X^{n-2}+\dots+X^2+X+1)
\]
faktorisiert werden kann und $\omega$ keine Nullstelle des Faktors
$X-1$ sein kann, muss $\omega$ eine Nullstelle des zweiten Faktors sein,
es muss also
\begin{equation}
\omega^{n-1}+\omega^{n-2}+\dots+\omega^2 + \omega + 1 = 0
\label{buch:diskret:koerper:eqn:summeomega}
\end{equation}
sein.

Die Fourier-Matrix $\mathscr{F}_n$ über dem Körper $\mathbb{F}_{p^l}$
ist dann wie im komplexen Fall die Vandermonde-Matrix
\begin{align*}
\mathscr{F}_n
&=
V(1,\omega,\omega^2,\dots,\omega^{n-1})
\\
&=
\begin{pmatrix}
1&1&1&\dots&1\\
1&\omega&\omega^2&\dots&\omega^{n-1}\\
1&\omega^2&\omega^4&\dots&\omega^{2(n-1)}\\
\vdots&\vdots&\vdots&\ddots&\vdots\\
1&\omega^{n-1}&\omega^{2(n-1)}&\dots&\omega^{(n-1)^2}
\end{pmatrix}
\end{align*}
Für die ``konjugierte'' Matrix muss
\begin{align*}
\overline{\mathscr{F}}_n
&=
V(1,\omega^{-1},\omega^{-2},\dots,\omega^{-(n-1)})
\\
&=
\begin{pmatrix}
1&1&1&\dots&1\\
1&\omega^{-1}&\omega^{-2}&\dots&\omega^{-(n-1)}\\
1&\omega^{-2}&\omega^{-4}&\dots&\omega^{-2(n-1)}\\
\vdots&\vdots&\vdots&\ddots&\vdots\\
1&\omega^{-(n-1)}&\omega^{-2(n-1)}&\dots&\omega^{-(n-1)^2}
\end{pmatrix}
\end{align*}
Das Produkt $\overline{\mathscr{F}}_n\mathscr{F}_n$ kann direkt
berechnet werden.
Die Zeile $i$ der Matrix $\mathscr{F}_n$ enthält die Elemente 
\[
1,\omega^{i-1},\omega^{2(i-1)},\dots,\omega^{(n-1)(i-1)},
\]
die der Matrix $\overline{\mathscr{F}}_n$ die Elemente
\[
1,\omega^{-(i-1)},\omega^{-2(i-1)},\dots,\omega^{-(n-1)(i-1)}.
\]
Das Diagonalelement des Produktes ist daher
\[
(\overline{\mathscr{F}}_n
\mathscr{F}_n)_{ii}
=
\sum_{k=0}^{n-1}
\omega^{-(i-1)k}\omega^{(i-1)k}
=
\sum_{k=0}^{n-1} 1
=
n.
\]
Die Matrix $\overline{\mathscr{F}}_n$ kann also nur dann zur
inversen Matrix von $\mathscr{F}_n$ werden, wenn im Körper
$\mathbb{F}_{p^l}$ durch $n$ geteilt werden kann.
Das ist genau dann möglich, wenn $p$ kein Teiler von $n$ ist.

Für das ausserdiagonale Element in Zeile $i$ und Spalte $j$
findet man entsprechend
\begin{equation}
(\overline{\mathscr{F}}_n
\mathscr{F}_n)_{i\!j}
=
\sum_{k=0}^{n-1}
\omega^{-(i-1)k}\omega^{(j-1)k}
=
\sum_{k=0}^{n-1} \omega^{(j-i)k}.
\label{buch:diskret:koerper:eqn:ausserdiagonal}
\end{equation}
Da $p$ und $n$ teilerfremd sind, sind die Exponenten $(j-i)k$
modulo $n$ genau die Exponenten $0$ bis $n-1$.
Die Summe
\eqref{buch:diskret:koerper:eqn:ausserdiagonal}
ist daher wegen
\eqref{buch:diskret:koerper:eqn:summeomega}
immer $=0$.
Damit ist gezeigt, dass $\frac1n\overline{\mathscr{F}}_n$ die
inverse Matrix von $\mathscr{F}_n$ ist.

%
% Faktorisierung und schnelle Algorithmen
%
\subsection{Faktorisierung und schnelle Algorithmen
\label{buch:diskret:koerper:subsection:faktorisierung}}
In Abschnitt~\ref{buch:diskret:koeper:subsection:fourier} wurde
gezeigt, wie die Fourier-Transformation auf der Gruppen
$\mathbb{Z}/n\mathbb{Z}$ über jedem endlichen Körper $\mathbb{F}_{p^l}$
konstruiert werden kann, wenn $p$ kein Teiler von $n$ ist und
der endliche Körper so gewählt wird, dass eine primitive $n$-te
Einheitswurzel darin gefunden werden kann.

Die Faktorisierung der Fourier-Transformation in
Abschnitt~\ref{buch:diskret:section:vandermonde}
und die Konstruktion schneller Algorithmen in
Abschnitt~\ref{buch:diskret:section:schnell} können damit
genauso versucht werden.
Dabei treten die Fourier-Transformationen $\mathscr{q}$ für Primfaktoren
$q$ von $n$ auf.
Um diese zu konstruieren wird eine primitive $q$-te Einheitswurzel benötigt.
Da aber $\omega$ eine primitive $n$-te Einheitswurzel von ist, ist
$\omega^{n/q}$ eine primitive $q$-te Einheitswurzel.

Für die inverse Abbildung wird die inverse Fourier-Transformation
$\frac{1}{q}\overline{\mathscr{F}}_q$ benötigt.
Da $q$ ein Primfaktor von $n$ ist, sind $p$ und $q$ verschieden.
Damit ist auch $q$ in $\mathbb{F}_{p^l}$ invertierbar und es steht
der Konstruktion der inversen Fourier-Transformation nichts im Wege.

Die schnelle Fourier-Transformation für die Dimension $n$ funktioniert
also ganz genau gleich wie im Fall $\mathbb{C}$, es werden die gleichen
Matrizen verwendet, nur die arithmetischen Operationen in $\mathbb{C}$
müssen durch arithmetische Operationen in $\mathbb{F}_{p^l}$ ersetzt
werden und es muss die primitve $n$-te Einheitswurzel
$\omega\in\mathbb{F}_{p^l}$ verwendet werden.

