%
% 5-koerper.tex
%
% (c) 2022 Prof Dr Andreas Müller, OST Ostschweizer Fachhochschule
%
\section{Fourier-Transformation in endlichen Körpern
\label{buch:diskret:section:koerper}}
\kopfrechts{Endliche Körper}
Sowohl die Fourier-Transformation wie auch die Faktorisierung
der Fourier-Matrix beschränkt sich nicht auf die komplexen Zahlen
als Wertebereich für die Funktionen.
Damit wird es möglich, die Fourier-Transformation zur Codierung von
digitalen Signalen oder Bildern zu verwenden, die dann auch verlustlos
rekonstruiert werden können.
Sie eigenen sich aber nicht für Approximationen wie sie zum Beispiel
bei der JPEG-Codierung verwendet werden sollen.
Diese beruhen darauf, dass hochfrequente Komponenten in der
Fourier-Transformierten weggelassen werden können, weil sie für die
meisten Bilder ``klein'' sind.
In einem endlichen Körper gibt es kein sinnvolles Konzept ``kleiner''
Werte.

%
% Endliche Körper
%
\subsection{Endliche Körper
\label{buch:diskret:koerper:subsection:koerper}}

%
% Forier-Transformation
%
\subsection{Fourier-Transformation
\label{buch:diskret:koeper:subsection:fourier}}
Die Fourier-Transformation verwendet die duale Gruppe $\hat{G}$ einer
Gruppe $G$, also die Menge der Homomorphismen $G\to \mathbb{C}$.
Für die Fourier-Transformation über einen endlichen Körper
$\mathbb{F}_{p^l}$ sind Homomorphismen
\(
G \to \mathbb{F}_{p^l}
\)
gesucht.

Für die zyklischen Gruppen $G=\mathbb{Z}/n\mathbb{Z}$ sind diese wie
im komplexen Fall durch den Wert auf dem erzeugenden Element
$1\in\mathbb{Z}/n\mathbb{Z}$ gegeben.
Gesucht ist daher ein Element $\omega\in\mathbb{F}_{p^l}$ derart,
dass $\omega^n=1$.
Ausserdem müssen alle Potenzen $\omega^k$ mit $k=0,\dots,n-1$
verschieden sein.
Das Elemente $\omega\in\mathbb{F}_{p^l}$ muss also eine
zyklische Untergruppe der Ordnung $n$ in $\mathbb{F}_{p^l}$ 
erzeugen.
Da das Polynom $X^n-1$ immer als
\[
X^n-1
=
(X-1)(X^{n-1}+X^{n-2}+\dots+X^2+X+1)
\]
faktorisiert werden kann und $\omega$ keine Nullstelle des Faktors
$X-1$ sein kann, muss $\omega$ eine Nullstelle des zweiten Faktors sein,
es muss also
\begin{equation}
\omega^{n-1}+\omega^{n-2}+\dots+\omega^2 + \omega + 1 = 0
\label{buch:diskret:koerper:eqn:summeomega}
\end{equation}
sein.

Die Fourier-Matrix $\mathscr{F}_n$ über dem Körper $\mathbb{F}_{p^l}$
ist dann wie im komplexen Fall die Vandermonde-Matrix
\begin{align*}
\mathscr{F}_n
&=
V(1,\omega,\omega^2,\dots,\omega^{n-1})
\\
&=
\begin{pmatrix}
1&1&1&\dots&1\\
1&\omega&\omega^2&\dots&\omega^{n-1}\\
1&\omega^2&\omega^4&\dots&\omega^{2(n-1)}\\
\vdots&\vdots&\vdots&\ddots&\vdots\\
1&\omega^{n-1}&\omega^{2(n-1)}&\dots&\omega^{(n-1)^2}
\end{pmatrix}
\end{align*}
Für die ``konjugierte'' Matrix muss
\begin{align*}
\overline{\mathscr{F}}_n
&=
V(1,\omega^{-1},\omega^{-2},\dots,\omega^{-(n-1)})
\\
&=
\begin{pmatrix}
1&1&1&\dots&1\\
1&\omega^{-1}&\omega^{-2}&\dots&\omega^{-(n-1)}\\
1&\omega^{-2}&\omega^{-4}&\dots&\omega^{-2(n-1)}\\
\vdots&\vdots&\vdots&\ddots&\vdots\\
1&\omega^{-(n-1)}&\omega^{-2(n-1)}&\dots&\omega^{-(n-1)^2}
\end{pmatrix}
\end{align*}
Das Produkt $\overline{\mathscr{F}}_n\mathscr{F}_n$ kann direkt
berechnet werden.
Die Zeile $i$ der Matrix $\mathscr{F}_n$ enthält die Elemente 
\[
1,\omega^{i-1},\omega^{2(i-1)},\dots,\omega^{(n-1)(i-1)},
\]
die der Matrix $\overline{\mathscr{F}}_n$ die Elemente
\[
1,\omega^{-(i-1)},\omega^{-2(i-1)},\dots,\omega^{-(n-1)(i-1)}.
\]
Das Diagonalelement des Produktes ist daher
\[
(\overline{\mathscr{F}}_n
\mathscr{F}_n)_{ii}
=
\sum_{k=0}^{n-1}
\omega^{-(i-1)k}\omega^{(i-1)k}
=
\sum_{k=0}^{n-1} 1
=
n.
\]
Die Matrix $\overline{\mathscr{F}}_n$ kann also nur dann zur
inversen Matrix von $\mathscr{F}_n$ werden, wenn im Körper
$\mathbb{F}_{p^l}$ durch $n$ geteilt werden kann.
Das ist genau dann möglich, wenn $p$ kein Teiler von $p$ ist.

Für das ausserdiagonale Element in Zeile $i$ und Spalte $j$
findet man entsprechend
\begin{equation}
(\overline{\mathscr{F}}_n
\mathscr{F}_n)_{i\!j}
=
\sum_{k=0}^{n-1}
\omega^{-(i-1)k}\omega^{(j-1)k}
=
\sum_{k=0}^{n-1} \omega^{(j-i)k}.
\label{buch:diskret:koerper:eqn:ausserdiagonal}
\end{equation}
Da $p$ und $n$ teilerfremd sind, sind die Exponenten $(j-i)k$
modulo $n$ genau die Exponenten $0$ bis $n-1$.
Die Summe
\eqref{buch:diskret:koerper:eqn:ausserdiagonal}
ist daher wegen
\eqref{buch:diskret:koerper:eqn:summeomega}
immer $=0$.
Damit ist gezeigt, dass $\frac1n\overline{\mathscr{F}}_n$ die
inverse Matrix von $\mathscr{F}_n$ ist.

%
% Faktorisierung und schnelle Algorithmen
%
\subsection{Faktorisierung und schnelle Algorithmen
\label{buch:diskret:koerper:subsection:faktorisierung}}
In Abschnitt~\ref{buch:diskret:koeper:subsection:fourier} wurde
gezeigt, wie die Fourier-Transformation auf der Gruppen
$\mathbb{Z}/n\mathbb{Z}$ über jedem endlichen Körper $\mathbb{F}_{p^l}$
konstruiert werden kann, wenn $p$ kein Teiler von $n$ ist und
der endliche Körper so gewählt wird, dass eine primitive $n$-te
Einheitswurzel darin gefunden werden kann.

Die Faktorisierung der Fourier-Transformation in
Abschnitt~\ref{buch:diskret:section:vandermonde}
und die konstruktion schneller Algorithmen in
Abschnitt~ \ref{buch:diskret:section:schnell} können damit
genauso versucht werden.
Dabei treten die Fourier-Transformationen $\mathscr{q}$ für Primfaktoren
$q$ von $n$ auf.
Um diese zu konstruieren wird eine primitive $q$-te Einheitswurzel benötigt.
Da aber $\omega$ eine primitive $n$-te Einheitswurzel von ist, ist
$\omega^{n/q}$ eine primitive $q$-te Einheitswurzel.

Für die inverse Abbildung wird die inverse Fourier-Transformation
benötigt, die durch $\frac{1}{q}\overline{\mathscr{F}}_q$ ist.
Da $q$ ein Primfaktor von $n$ ist, sind $p$ und $q$ verschieden.
Damit ist auch $q$ in $\mathbb{F}_{p^l}$ invertierbar und es steht
der Konstruktion der inversen Fourier-Transformation nichts im Wege.

Die schnelle Fourier-Transformation für die Dimension $n$ funktioniert
also ganz genau gleich wie im Fall $\mathbb{C}$, es werden die gleichen
Matrizen verwendet, nur die arithmetischen Operationen in $\mathbb{C}$
müssen durch arithmetische Operationen in $\mathbb{F}_{p^l}$ ersetzt
werden und es muss die primitve $n$-te Einheitswurzel
$\omega\in\mathbb{F}_{p^l}$ verwendet werden.

