%
% etraeger.tex -- epsilon-Traeger als Lokalisierungsmass
%
% (c) 2021 Prof Dr Andreas Müller, OST Ostschweizer Fachhochschule
%
\documentclass[tikz]{standalone}
\usepackage{amsmath}
\usepackage{times}
\usepackage{txfonts}
\usepackage{pgfplots}
\usepackage{csvsimple}
\usetikzlibrary{arrows,intersections,math}
\begin{document}
\def\skala{1}
\definecolor{darkgreen}{rgb}{0,0.6,0}
\def\eps{0.4464}
\def\kurve{
	({0.0000*\dx},{0.6524*\dy})
	--({0.1000*\dx},{0.5836*\dy})
	--({0.2000*\dx},{0.0991*\dy})
	--({0.3000*\dx},{0.2236*\dy})
	--({0.4000*\dx},{0.0562*\dy})
	--({0.5000*\dx},{0.6498*\dy})
	--({0.6000*\dx},{0.8176*\dy})
	--({0.7000*\dx},{0.5386*\dy})
	--({0.8000*\dx},{0.4778*\dy})
	--({0.9000*\dx},{0.1475*\dy})
	--({1.0000*\dx},{0.7115*\dy})
	--({1.1000*\dx},{0.6894*\dy})
	--({1.2000*\dx},{0.3160*\dy})
	--({1.3000*\dx},{0.4238*\dy})
	--({1.4000*\dx},{0.8661*\dy})
	--({1.5000*\dx},{0.3537*\dy})
	--({1.6000*\dx},{0.8405*\dy})
	--({1.7000*\dx},{0.8517*\dy})
	--({1.8000*\dx},{0.3924*\dy})
	--({1.9000*\dx},{0.8533*\dy})
	--({2.0000*\dx},{0.8153*\dy})
	--({2.1000*\dx},{0.5055*\dy})
	--({2.2000*\dx},{0.0356*\dy})
	--({2.3000*\dx},{0.3083*\dy})
	--({2.4000*\dx},{0.3475*\dy})
	--({2.5000*\dx},{0.2491*\dy})
	--({2.6000*\dx},{0.3707*\dy})
	--({2.7000*\dx},{0.4769*\dy})
	--({2.8000*\dx},{0.2076*\dy})
	--({2.9000*\dx},{0.4047*\dy})
	--({3.0000*\dx},{0.2673*\dy})
	--({3.1000*\dx},{0.5988*\dy})
	--({3.2000*\dx},{0.6834*\dy})
	--({3.3000*\dx},{0.2689*\dy})
	--({3.4000*\dx},{0.1356*\dy})
	--({3.5000*\dx},{0.6214*\dy})
	--({3.6000*\dx},{0.2594*\dy})
	--({3.7000*\dx},{0.1701*\dy})
	--({3.8000*\dx},{0.6442*\dy})
	--({3.9000*\dx},{0.5127*\dy})
	--({4.0000*\dx},{0.2494*\dy})
	--({4.1000*\dx},{0.7401*\dy})
	--({4.2000*\dx},{0.5647*\dy})
	--({4.3000*\dx},{0.1825*\dy})
	--({4.4000*\dx},{0.0253*\dy})
	--({4.5000*\dx},{0.3223*\dy})
	--({4.6000*\dx},{0.3474*\dy})
	--({4.7000*\dx},{0.3482*\dy})
	--({4.8000*\dx},{0.0631*\dy})
	--({4.9000*\dx},{0.2508*\dy})
	--({5.0000*\dx},{0.6235*\dy})
	--({5.1000*\dx},{0.7459*\dy})
	--({5.2000*\dx},{0.8067*\dy})
	--({5.3000*\dx},{0.0316*\dy})
	--({5.4000*\dx},{0.3868*\dy})
	--({5.5000*\dx},{0.3655*\dy})
	--({5.6000*\dx},{0.2425*\dy})
	--({5.7000*\dx},{0.6495*\dy})
	--({5.8000*\dx},{0.5813*\dy})
	--({5.9000*\dx},{0.8036*\dy})
	--({6.0000*\dx},{0.7463*\dy})
	--({6.1000*\dx},{0.2223*\dy})
	--({6.2000*\dx},{1.0225*\dy})
	--({6.3000*\dx},{1.3405*\dy})
	--({6.4000*\dx},{2.2053*\dy})
	--({6.5000*\dx},{3.0764*\dy})
	--({6.6000*\dx},{3.8907*\dy})
	--({6.7000*\dx},{4.6786*\dy})
	--({6.8000*\dx},{4.9753*\dy})
	--({6.9000*\dx},{6.0000*\dy})
	--({7.0000*\dx},{5.7206*\dy})
	--({7.1000*\dx},{5.2258*\dy})
	--({7.2000*\dx},{4.8801*\dy})
	--({7.3000*\dx},{4.4441*\dy})
	--({7.4000*\dx},{3.5583*\dy})
	--({7.5000*\dx},{2.9746*\dy})
	--({7.6000*\dx},{2.5950*\dy})
	--({7.7000*\dx},{1.1729*\dy})
	--({7.8000*\dx},{1.3161*\dy})
	--({7.9000*\dx},{0.9417*\dy})
	--({8.0000*\dx},{0.6263*\dy})
	--({8.1000*\dx},{0.1131*\dy})
	--({8.2000*\dx},{0.5731*\dy})
	--({8.3000*\dx},{0.2706*\dy})
	--({8.4000*\dx},{0.2447*\dy})
	--({8.5000*\dx},{0.4680*\dy})
	--({8.6000*\dx},{0.6283*\dy})
	--({8.7000*\dx},{0.0903*\dy})
	--({8.8000*\dx},{0.7743*\dy})
	--({8.9000*\dx},{0.4777*\dy})
	--({9.0000*\dx},{0.5508*\dy})
	--({9.1000*\dx},{0.0191*\dy})
	--({9.2000*\dx},{0.6640*\dy})
	--({9.3000*\dx},{0.7770*\dy})
	--({9.4000*\dx},{0.7892*\dy})
	--({9.5000*\dx},{0.8672*\dy})
	--({9.6000*\dx},{0.5762*\dy})
	--({9.7000*\dx},{0.3338*\dy})
	--({9.8000*\dx},{0.2111*\dy})
	--({9.9000*\dx},{0.0703*\dy})
	--({10.0000*\dx},{0.6257*\dy})
	--({10.1000*\dx},{0.4016*\dy})
	--({10.2000*\dx},{0.4444*\dy})
	--({10.3000*\dx},{0.3527*\dy})
	--({10.4000*\dx},{0.6758*\dy})
	--({10.5000*\dx},{0.0952*\dy})
	--({10.6000*\dx},{0.7666*\dy})
	--({10.7000*\dx},{0.8723*\dy})
	--({10.8000*\dx},{0.5896*\dy})
	--({10.9000*\dx},{0.5149*\dy})
	--({11.0000*\dx},{0.7341*\dy})
	--({11.1000*\dx},{0.7989*\dy})
	--({11.2000*\dx},{0.4449*\dy})
	--({11.3000*\dx},{0.3409*\dy})
	--({11.4000*\dx},{0.6671*\dy})
	--({11.5000*\dx},{0.5363*\dy})
	--({11.6000*\dx},{0.2182*\dy})
	--({11.7000*\dx},{0.4024*\dy})
	--({11.8000*\dx},{0.8016*\dy})
	--({11.9000*\dx},{0.4318*\dy})
	--({12.0000*\dx},{0.8864*\dy})
}

\def\dx{1}
\def\dy{1}
\def\xmin{6}
\def\xmax{8}
\begin{tikzpicture}[>=latex,thick,scale=\skala]

\fill[color=blue!20] (0,0) rectangle (12,{\eps*\dy});
\draw[color=blue,line width=0.1pt] (0,{\eps*\dy}) -- (12,{\eps*\dy});
\fill[color=blue!20] (\xmin,0) rectangle (\xmax,6);

\draw[color=red] (0,6) --(12,6);
\node[color=red] at ({\xmin/2},6) [above] {$\|u\|_\infty$};

\draw[->] (-0.1,0) -- (13.5,0) coordinate[label={$x$}];
\draw[->] (0,-0.1) -- (0,6.5) coordinate[label={right:$y$}];

\node at ({0.5*(\xmin+\xmax)},0) [below] {$T=\operatorname{supp}_\varepsilon^1(u)$};

\draw[color=blue,line width=1.4pt]
	(\xmin,{\eps*\dy}) -- (\xmin,6) -- (\xmax,6) -- (\xmax,{\eps*\dy});

\begin{scope}
	\clip (0,0) rectangle (12,6);
	\fill[color=darkgreen!40,opacity=0.5] 
		(0,0) -- \kurve -- (12,0) -- cycle;

	\draw[color=darkgreen,line width=1.4pt]
		\kurve;

	\node[color=darkgreen] at ({0.5*(\xmin+\xmax)},2) {$\|u\|_1$};
\end{scope}

\node[color=blue] at (\xmin,3) [above,rotate=90] {$\|u\|_\infty\cdot\chi_T$};

\node[color=blue] at (\xmax+0.5,5) [right] {$\|u\|_\infty\cdot |T|$};
\draw[<-,color=blue] ({\xmax-0.3},5) -- ({\xmax+0.5},5);

\fill[color=blue] (12,{\eps*\dy}) circle[radius=0.03];
\draw[color=blue,line width=0.1pt] (12,{\eps*\dy}) -- (12.7,1);
\node[color=blue] at (12.7,1) [right] {$\varepsilon$};

\end{tikzpicture}
\end{document}

