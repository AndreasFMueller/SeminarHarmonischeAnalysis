%
% tikztemplate.tex -- template for standalon tikz images
%
% (c) 2021 Prof Dr Andreas Müller, OST Ostschweizer Fachhochschule
%
\documentclass[tikz]{standalone}
\usepackage{amsmath}
\usepackage{times}
\usepackage{txfonts}
\usepackage{pgfplots}
\usepackage{csvsimple}
\usepackage{mathrsfs}
\usetikzlibrary{arrows,intersections,math}
\begin{document}
\def\skala{1}
%
% f8common.tex -- common stuff for the 
%
% (c) 2021 Prof Dr Andreas Müller, OST Ostschweizer Fachhochschule
%
\def\w{4}
\def\h{0.8}
\def\t{0.3}
\def\s{0.3}
\def\punkt#1#2{({(#1)*\w},{-(#2)*\h})}
\def\nummer#1#2#3{
        \fill[color=white] ({(#3)*\w},{-(#1)*\h}) circle[radius=0.12];
        \draw ({(#3)*\w},{-(#1)*\h}) circle[radius=0.12];
        \node at ({(#3)*\w},{-(#1)*\h}) {$\scriptscriptstyle #2$};
}
\def\ffff#1#2#3#4#5{
	\fill[color=white]
		\punkt{#5+0.5-\s/\w}{0.25*(#1+#2+#3+#4)-\s/\h}
		rectangle
		\punkt{#5+0.5+\s/\w}{0.25*(#1+#2+#3+#4)+\s/\h};
	\draw
		\punkt{#5+0.5-\s/\w}{0.25*(#1+#2+#3+#4)-\s/\h}
		rectangle
		\punkt{#5+0.5+\s/\w}{0.25*(#1+#2+#3+#4)+\s/\h};
	\node at \punkt{#5+0.5}{0.25*(#1+#2+#3+#4)} {$\mathscr{F}_2$};
}
\def\spinne#1#2#3#4#5#6{
	\draw[<-,shorten <= #6]
		\punkt{#5}{#1} -- \punkt{#5+0.5}{0.25*(#1+#2+#3+#4)};
	\draw[<-,shorten <= #6]
		\punkt{#5}{#2} -- \punkt{#5+0.5}{0.25*(#1+#2+#3+#4)};
	\draw
		\punkt{#5+1}{#3} -- \punkt{#5+0.5}{0.25*(#1+#2+#3+#4)};
	\draw
		\punkt{#5+1}{#4} -- \punkt{#5+0.5}{0.25*(#1+#2+#3+#4)};
	\ffff{#1}{#2}{#3}{#4}{#5}
}
\def\knoten#1#2#3#4#5{
	\fill[color=gray!40]
		\punkt{\t*(#1)+(1-\t)*(#3)}{\t*(#2)+(1-\t)*(#4)}
		circle[radius=0.2];
	\draw
		\punkt{\t*(#1)+(1-\t)*(#3)}{\t*(#2)+(1-\t)*(#4)}
		circle[radius=0.2];
	\node[color=white] at \punkt{\t*(#1)+(1-\t)*(#3)}{\t*(#2)+(1-\t)*(#4)}
		{$\scriptstyle #5$};
}
\def\gitter#1#2{
	\foreach \x in {0,...,#1}{
		\draw \punkt{\x}{-0.2} -- \punkt{\x}{#2+0.2};
	}
	\node at \punkt{#1}{-0.2} [above] {$f$};
	\node at \punkt{0}{-0.2} [above] {$\hat{f}$};

	\foreach \y in {0,...,#2}{
		\foreach \x in {1,...,#1}{
			\fill \punkt{\x}{\y} circle[radius=0.05];
		}
		\foreach \x in {0,#1}{
			\nummer{\y}{\y}{\x}
			\fill[color=white] \punkt{\x}{\y} circle[radius=0.12];
			\draw \punkt{\x}{\y} circle[radius=0.12];
			\node at \punkt{\x}{\y} {$\scriptscriptstyle\y$};
		}
	}
}


\def\h{0.8}
\def\w{3}
\def\punkt#1#2{({(#1)*\w},{-(#2)*\h})}
\begin{tikzpicture}[>=latex,thick,scale=\skala]

\foreach \y in {0,...,7}{
	\spinne{2*\y}{2*\y+1}{2*\y}{2*\y+1}{3}{0.05cm}
}

\foreach \y in {0,...,3}{
	\spinne{4*\y}{4*\y+2}{4*\y}{4*\y+2}{2}{0.05cm}
	\spinne{4*\y+1}{4*\y+3}{4*\y+1}{4*\y+3}{2}{0.05cm}
}

\foreach \y in {0,1}{
	\foreach \u in {0,1,2,3}{
		\spinne{8*\y+\u}{8*\y+4+\u}{8*\y+\u}{8*\y+4+\u}{1}{0.05cm}
	}
}

\foreach \y in {0,...,7}{
	\spinne{\y}{\y+8}{\y}{\y+8}{0}{0.15cm}
}
\foreach \y in {0,...,7}{
	\ffff{\y}{\y+8}{\y}{\y+8}{0}
}

\gitter{4}{15}

\nummer{0}{0}{4}
\nummer{1}{8}{4}
\nummer{2}{4}{4}
\nummer{3}{12}{4}
\nummer{4}{2}{4}
\nummer{5}{10}{4}
\nummer{6}{6}{4}
\nummer{7}{14}{4}
\nummer{8}{1}{4}
\nummer{9}{9}{4}
\nummer{10}{5}{4}
\nummer{11}{13}{4}
\nummer{12}{3}{4}
\nummer{13}{11}{4}
\nummer{14}{7}{4}
\nummer{15}{15}{4}

\knoten{0.5}{9}{1}{13}{\omega}
\knoten{0.5}{10}{1}{14}{\omega^2}
\knoten{0.5}{11}{1}{15}{\omega^3}

\knoten{1.5}{4}{2}{6}{\omega^2}
\knoten{1.5}{5}{2}{7}{\omega^4}
\knoten{1.5}{12}{2}{14}{\omega^2}
\knoten{1.5}{13}{2}{15}{\omega^4}

\knoten{2.5}{2}{3}{3}{\omega^4}
\knoten{2.5}{6}{3}{7}{\omega^4}
\knoten{2.5}{10}{3}{11}{\omega^4}
\knoten{2.5}{14}{3}{15}{\omega^4}

\end{tikzpicture}
\end{document}

