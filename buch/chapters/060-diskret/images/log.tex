%
% log.tex -- template for standalon tikz images
%
% (c) 2021 Prof Dr Andreas Müller, OST Ostschweizer Fachhochschule
%
\documentclass[tikz]{standalone}
\usepackage{amsmath}
\usepackage{times}
\usepackage{txfonts}
\usepackage{pgfplots}
\usepackage{csvsimple}
\usetikzlibrary{arrows,intersections,math}
\begin{document}
\def\skala{1}
\def\a{0.35}
\def\l{6}
\pgfmathparse{atan(2*\a)}
\xdef\w{\pgfmathresult}
\pgfmathparse{atan(\a*(1+1/(ln(10)*\l)))}
\xdef\v{\pgfmathresult}
\pgfmathparse{atan(\a)}
\xdef\u{\pgfmathresult}
\begin{tikzpicture}[>=latex,thick,scale=\skala]


\draw[color=gray!40,line width=1.4pt]
	plot[domain=00:10,samples=10] ({\x},{\a*\x});
\node[color=gray!40] at (\l,{\a*\l}) [below,rotate=\u] {$O(n)$};
\draw[color=blue,line width=1.4pt]
	plot[domain=0:10,samples=10] ({\x},{\a*2*\x});
\node[color=blue] at (\l,{\a*2*\l}) [above,rotate=\w] {$O(n^2)$};
\node[color=red] at (\l,{\a*(\l+log10(\l))}) [above,rotate=\v] {$O(n\log n)$};
\begin{scope}
\clip (0,0) rectangle (10,7);
\draw[color=red,line width=1.4pt]
	plot[domain=0.1:10,samples=100] ({\x},{\a*(\x+log10(\x))});
\end{scope}
\draw[->] (0,-0.1) -- (0,7.5) coordinate[label={right:$\log(\text{Aufwand})$}];
\draw[->] (-0.1,0) -- (10.5,0) coordinate[label={$\log n$}];

\end{tikzpicture}
\end{document}

