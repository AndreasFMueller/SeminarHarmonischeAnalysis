%
% f23.tex -- Faktorisierung 6 = 2 * 3
%
% (c) 2021 Prof Dr Andreas Müller, OST Ostschweizer Fachhochschule
%
\documentclass[tikz]{standalone}
\usepackage{amsmath}
\usepackage{times}
\usepackage{txfonts}
\usepackage{pgfplots}
\usepackage{csvsimple}
\usepackage{mathrsfs}
\usetikzlibrary{arrows,intersections,math,calc}
\begin{document}
\def\skala{1}
\definecolor{darkgreen}{rgb}{0,0.6,0}
%
% fcommon.tex -- common stuff for transform decomposition
%
% (c) 2023 Prof Dr Andreas Müller
%
\def\h{1.2}
\def\w{4.8}

\def\punkt#1#2{ ({(#1)*\w},{-(#2)*\h}) }

\def\t{0.7}
\def\faktor#1#2#3{
	\fill[color=gray!50] \punkt{\t*1+(1-\t)*0.5}{\t*(#2)+(1-\t)*(#1)} circle[radius=0.27];
	\draw \punkt{\t*1+(1-\t)*0.5}{\t*(#2)+(1-\t)*(#1)} circle[radius=0.27];
	\node[color=white] at \punkt{\t*1+(1-\t)*0.5}{\t*(#2)+(1-\t)*(#1)} {$#3\mathstrut$};
}

\def\hintergrund{
	\fill[color=darkgreen!10] \punkt{1+0.2/\w}{-0.2/\h} rectangle \punkt{2-0.2/\w}{5+0.2/\h};
	\fill[color=blue!10] \punkt{0+0.2/\w}{-0.2/\h} rectangle \punkt{1-0.2/\w}{5+0.2/\h};
}
\def\gitter{
	\foreach \x in {0,1,2}{
		\draw \punkt{\x}{-0.2} -- \punkt{\x}{5.2};
	}
}

\def\knoten{
	\foreach \x in {0,1,2}{
		\foreach \y in {0,...,5}{
			\fill[color=white] \punkt{\x}{\y} circle[radius=0.12];
			\draw \punkt{\x}{\y} circle[radius=0.12];
			\node at \punkt{\x}{\y+0.01} {$\scriptscriptstyle\y\mathstrut$};
		}
	}
}



\begin{tikzpicture}[>=latex,thick,scale=\skala]

\hintergrund
\gitter

\node at \punkt{0.5}{0.15} {$A(3,2,\omega)\mathstrut$};
\node at \punkt{0.5}{4.95} {$(I_3\otimes\mathscr{F}_2)\cdot D(3,2,\omega)\mathstrut$};
\node at \punkt{1.5}{0.15} {$F(3,2,\omega)=\mathscr{F}_3\otimes I_2\mathstrut$};

\foreach \y in {0,1}{
	\draw[->,shorten >= 0.45cm] \punkt{2}{\y} -- \punkt{1.5}{\y+2};
	\draw[->,shorten >= 0.3cm] \punkt{2}{\y+2} -- \punkt{1.5}{\y+2};
	\draw[->,shorten >= 0.45cm] \punkt{2}{\y+4} -- \punkt{1.5}{\y+2};
	\draw[->,shorten >= 0.1cm] \punkt{1.5}{\y+2} -- \punkt{1}{\y};
	\draw[->,shorten >= 0.1cm] \punkt{1.5}{\y+2} -- \punkt{1}{\y+2};
	\draw[->,shorten >= 0.1cm] \punkt{1.5}{\y+2} -- \punkt{1}{\y+4};
	\fill[color=white] \punkt{1.5-0.35/\w}{\y+2-0.35/\h} rectangle \punkt{1.5+0.35/\w}{\y+2+0.35/\h};
	\draw \punkt{1.5-0.35/\w}{\y+2-0.35/\h} rectangle \punkt{1.5+0.35/\w}{\y+2+0.35/\h};
	\node at \punkt{1.5}{\y+2} {$\mathscr{F}_3$};
}

\foreach \y in {0,1,2}{
	\draw[->,shorten >= 0.4cm] \punkt{1}{2*\y} -- \punkt{0.5}{1.5*\y+1};
	\draw[->,shorten >= 0.4cm] \punkt{1}{2*\y+1} -- \punkt{0.5}{1.5*\y+1};
	\draw[->,shorten >= 0.1cm] \punkt{0.5}{1.5*\y+1} -- \punkt{0}{\y};
	\draw[->,shorten >= 0.1cm] \punkt{0.5}{1.5*\y+1} -- \punkt{0}{\y+3};
	\fill[color=white] \punkt{0.5-0.4/\w}{1.5*\y+1-0.4/\h} rectangle \punkt{0.5+0.4/\w}{1.5*\y+1+0.4/\h};
	\draw \punkt{0.5-0.4/\w}{1.5*\y+1-0.4/\h} rectangle \punkt{0.5+0.4/\w}{1.5*\y+1+0.4/\h};
	\node at \punkt{0.5}{1.5*\y+1} {$\mathscr{F}_2$};
}

\faktor{2.5}{3}{\omega}
\faktor{4}{5}{\omega^2}

\node at \punkt{0}{-0.2} [above] {$\hat{f}$};
\node at \punkt{1}{-0.2} [above] {$F(3,2,\omega)f$};
\node at \punkt{2}{-0.2} [above] {$f$};

\knoten

\end{tikzpicture}
\end{document}

