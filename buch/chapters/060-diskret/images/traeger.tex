%
% traeger.tex -- Traeger als Lokalisierungsmass
%
% (c) 2021 Prof Dr Andreas Müller, OST Ostschweizer Fachhochschule
%
\documentclass[tikz]{standalone}
\usepackage{amsmath}
\usepackage{times}
\usepackage{txfonts}
\usepackage{pgfplots}
\usepackage{csvsimple}
\usetikzlibrary{arrows,intersections,math}
\begin{document}
\def\skala{1}
\definecolor{darkgreen}{rgb}{0,0.6,0}
\begin{tikzpicture}[>=latex,thick,scale=\skala]

\fill[color=blue!20] (6,0) rectangle (11,6);

\draw[color=red] (0,6) --(13,6);
\node[color=red] at (3,6) [above] {$\|u\|_\infty$};

\draw[->] (-0.1,0) -- (13.5,0) coordinate[label={$x$}];
\draw[->] (0,-0.1) -- (0,6.5) coordinate[label={right:$y$}];

\node at (8,0) [below] {$T=\operatorname{supp}(u)$};

\draw[color=blue,line width=1.4pt] (6,0) -- (6,6) -- (11,6) -- (11,0);

\fill[color=darkgreen!40,opacity=0.5] 
	plot[domain=6:8,samples=100] ({\x},{3*(1-cos((\x-6)*(360/4)))})
	--
	plot[domain=8:11,samples=100] ({\x},{3*(1-cos((\x-5)*(360/6)))})
	-- cycle;

\draw[color=darkgreen,line width=1.4pt]
	(0,0)
	-- plot[domain=6:8,samples=100] ({\x},{3*(1-cos((\x-6)*(360/4)))})
	-- plot[domain=8:11,samples=100] ({\x},{3*(1-cos((\x-5)*(360/6)))})
	-- (13,0);

\node[color=darkgreen] at (8,2) {$\|u\|_1$};


\node[color=blue] at (6,3) [above,rotate=90] {$\|u\|_\infty\cdot\chi_T$};

\node[color=blue] at (11,6) [below left] {$\|u\|_\infty\cdot |T|$};

\end{tikzpicture}
\end{document}

