%
% 1-endlich.tex
%
% (c) 2023 Prof Dr Andreas Müller, OST Ostschweizer Fachhochschule
%
\section{Endliche Gruppen
\label{buch:diskret:section:endlich}}
\kopfrechts{Endliche Gruppen}
Diskrete Fourier-Analysis handelt von der Fourier-Theorie auf
endlichen abelschen Gruppen.
Dieser Abschnitt treibt die Theorie der endlich erzeugten Gruppen
weit genug, damit wir alle endlichen abelschen Gruppen beschreiben
können.

%
% Endlich erzeugt Gruppen
%
\subsection{Endlich erzeugte Gruppen
\label{buch:diskret:endlich:subsetion:endlicherzeugt}}
Endliche Gruppen enthalten nur endlich viele Elemente, die Gruppenoperationen
können daher leicht durch eine Multiplikations oder Invertierungstabelle
vollständig beschrieben werden.
In diesem Abschnitt soll gezeigt werden, dass sogar noch viel weniger
nötig ist, weil solche Gruppen aus einer kleineren Menge von Elementen
Erzeugt werden können.

\begin{definition}
Sei $F\subset G$ eine Teilmenge einer Gruppe $G$.
Dann heisst die kleinste Untergruppe $\langle F\rangle$ von $G$,
der $F$ enthält, die von $F$ erzeugte Gruppe.
\end{definition}

\begin{beispiel}
Sei $\mathbb{R}$ die additive Gruppe der reellen Zahlen und
\[
F
=
\biggl\{\frac{1}{p^k}
\;
\bigg|
\;
\text{$p$ ist Primzahl, $k\in\mathbb{N}$}\biggr\}
\]
eine Teilmenge.
Die erzeugte Untergruppe enthält sicher alle Brüche der Form
\[
\frac{a}{p^k},\quad a\in\mathbb{Z},k\in\mathbb{N},\text{$p$ prim}.
\]
Da aber auch Summen von Brüchen der Form
\[
\frac{a}{p^k} + \frac{b}{q^l}
=
\frac{aq^l+bp^k}{p^kq^l}
\]
enthalten, die erzeugte Gruppe muss daher auch alle Brüche mit
Nennern erhalten, die Produkte von beliebigen Primzahlpotenzen sind.
Da jeder Nenner als Primzahlprodukt geschrieben werden kann, folgt,
dass die von $F$ erzeugte Gruppe alle rationalen Zahlen enthalten
muss.
Da aber in $F$ nur rationale Zahlen vorhanden sind, folgt
$\langle F\rangle = \mathbb{Q}$.
\end{beispiel}

\begin{satz}
Die erzeugte Gruppe der Teilmenge $F$ der Gruppe $G$ ist
\[
\langle F\rangle
=
\bigcap_{F\subset H\subset G \atop\text{$H$ Gruppe}} L
\]
\end{satz}

\begin{definition}
Eine Gruppe $G$ heisst {\em endlich erzeugt}, wenn es eine endliche Menge
$F\subset G$ gibt derart, dass $G=\langle F\rangle$.
\end{definition}

%
% Zyklische Gruppen
%
\subsection{Zyklische Gruppen
\label{buch:diskret:endlich:subsection:zyklisch}}

%
% Produkte
%
\subsection{Produkte
\label{buch:diskret:endlich:subsection:produkte}}

\begin{satz}
Die Gruppen $\mathbb{Z}/p^k\mathbb{Z}$ und $\mathbb{Z}/p^l\mathbb{Z}$
sind nicht isomorph, wenn $k\ne l$.
\end{satz}

%
% Struktursatz
%
\subsection{Der Struktursatz
\label{buch:diskret:endlich:subsection:}}



\begin{satz}
\label{buch:diskret:endlich:struktursatz}
Ist $G$ eine endliche erzeugte abelsche Gruppe, dann gibt es eine
Zahl $d\in\mathbb{N}$ und eindeutig bestimmte Primzahlpotenzen $d_i$
derart, dass
\[
G
\cong
\mathbb{Z}^d
\oplus
\bigoplus_{i=1}^n
\mathbb{Z}/d_i\mathbb{Z}.
\]
Die Ordnung der Gruppe $G$ ist $|G|=\prod_{i=1}^n d_i$.
\end{satz}

%
% Die duale Gruppe
%
\subsection{Die duale Gruppe
\label{buch:diskret:endlich:subsection:dual}}
Für die Fourier-Theorie für Funktionen mit einer endlich erzeugten
Gruppe $G$ muss die duale Gruppe ermittelt werden.
Aus dem Struktursatz~\ref{buch:diskret:endlich:struktursatz}
für endlich erzeugte Gruppen folgt, dass ein Homomorphismus
$G\to\mathbb{C}$ durch die Werte auf den erzeugenden Elementen
jedes einzelnen Summanden eindeutig bestimmt ist.
Daher gilt der folgende Satz:

\begin{satz}
Ist $G$ eine endlich erzeugte Gruppe, dann ist
$\hat{G}\cong G$.
\end{satz}

\begin{proof}[Beweis]
Sei
\[
G
=
\bigoplus_{i=1}^k \mathbb{Z}/d_i\mathbb{Z}
\]
die Zerlegung von $G$ gemäss dem Struktursatz
\ref{buch:diskret:endlich:struktursatz}.
\end{proof}



