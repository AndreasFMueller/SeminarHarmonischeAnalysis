%
% inhalt.tex -- Inhaltsübersicht
%
% (c) 2023 Prof Dr Andreas Müller, OST Ostschweizer Fachhochschule
%
\section{Inhalt
\label{buch:einleitung:section:inhalt}}
Der erste Teil des Buches gibt einen wie folgt in Kapitel gegliederten
Überblick über die allgemeine Theorie der harmonischen Analysis.

\begin{enumerate}
\item {\bf Skalarprodukte:}
In diesem Kapitel werden die Eigenschaften eines Skalarproduktes 
axiomatisch formuliert und dich wichtigsten Eigenschaftn daraus
abgleitet.
Ebenso wird der Begriff des Hilbert-Raumes eingeführt, der natürlichen
Bühne für harmonische Anaylsis.
\item {\bf Orthogonale Funktionen:}
Im zweiten Kapitel werden interessante Familien von orthogonalen
Funktionen konstruiert und es wird gezeigt, wie sich orthogonale
Funktionenfamilien im Kontext von Differentialgleichungen ganz natürlich
ergeben.
\item {\bf Gruppen:}
Die Struktur einer Gruppe ist die Basis für die Konstruktion der Faltung,
die selbst wieder eine grosse Zahl von Anwendungen hat.
In diesem Kapitel wird gezeigt, wie aus vielen bekannten und wichtigen
Gruppen Faltungsoperationen und interessante Funktionenfamilien
konstruiert werden können.
\item {\bf Operatoren:}
Der Laplace-Operator, der in Kapitel~\ref{buch:chapter:orthofunkt}
im Zusammenhang mit partiellen Differentialgleichungen wiederholt
aufgetaucht ist, ist nur ein Beispiel für einen Differentialoperator,
der besondere Eigenschaften im Zusammenhang mit einer Gruppe hat.
In diesem Kapitel werden weitere Beispiel solcher Operatoren gezeigt.
\item {\bf Radon-Transformation:}
Die Fourier-Transformation in $n$ Dimensionen kann zerlegt werden in
eine eindimensionale Fourier-Transformation und die Radon-Transformation.
Dies ist 
Letztere tritt auf natürliche Art direkt in Anwedungen auf.
\item {\bf Diskrete harmonische Analysis:}
Viele Anwendungen können nicht in geschlossener Form gelöst werden
und sind daher auf numerische Methoden angewiesen.
Diese wiederum funktionieren nur in einer diskreten Approximation.
Im Falle der Fourier-Transformation muss die Diskretisation so gestaltet
werden, dass sich möglichst viele der bei der Konstruktion der Funktionenbasis
wesentlichen Eigenschaften retten lassen.
Ist das Definitionsgebiet eine Gruppe, lässt sich ziemlich allgemeine
Art ein besonders schneller Algorithmus konstruieren.
Diese schnelle Fourier-Transformation funktioniert auch für Funktionen
mit Werten in einem endlichen Körper.
\item {\bf Nichtkommutative harmonische Analysis:}
Die Radon-Transformation lässt sich auf nichtkommutative Gruppen
übertragen.
Die Ideen der Faltung und der Lösung technischer Probleme mit Hilfe
der Fourier-Transformation lassen sich auch auf nichtkommutative
Gruppen übertragen.
\end{enumerate}

Die harmonische Analysis hat eine Vielfalt von interessanten
Anwendungen, die aus den verschiedensten Aspekten der Theorie
Nutzen ziehen .
Beispiele für die solche Anwendungen werden in den Kapiteln
des zweiten Teils vorgestellt.

