%
% randrand.tex -- show correlation between two random series
%
% (c) 2019 Prof Dr Andreas Müller, Hochschule Rapperswil
%
\documentclass[tikz]{standalone}
\usepackage{amsmath}
\usepackage{times}
\usepackage{txfonts}
\usepackage{pgfplots}
\usepackage{csvsimple}
\usetikzlibrary{arrows,intersections,math}
\begin{document}
\begin{tikzpicture}[>=latex,scale=0.95]

\input{randpoints.tex}

\draw[->,line width=0.7pt] (-0.1,0)--(9.7,0) coordinate[label={$t$}];
\draw[->,line width=0.7pt] (0,-2.2)--(0,2.3)
	coordinate[label={right:${\color{red}x},{\color{blue}y}$}];

\definecolor{pink}{rgb}{0.8,0.2,1}

\def\punkt#1#2#3{
	\draw[line width=1pt,color=pink] ({#1},{#3})--({#1},{#2});
	\fill[color=red] ({#1},{#2}) circle[radius=0.05];
	\fill[color=blue] ({#1},{#3}) circle[radius=0.05];
}

\graphA

\graphB
\punkteB

\begin{scope}[xshift = 12cm]

\fill[color=gray!20] (0,0) rectangle (2.1,-2.1);
\fill[color=gray!20] (0,0) rectangle (-2.1,2.1);

\node[color=gray!50,scale=6] at (-1.05,1.05) {$-$};
\node[color=gray!50,scale=6] at (1.05,-1.05) {$-$};
\node[color=gray!50,scale=6] at (-1.05,-1.05) {$+$};
\node[color=gray!50,scale=6] at (1.05,1.05) {$+$};

\draw[->,line width=0.7pt,color=red]
	(-2.1,0)--(2.3,0) coordinate[label={$x$}];
\draw[->,line width=0.7pt,color=blue]
	(0,-2.1)--(0,2.3) coordinate[label={right:$y$}];

\def\punkt#1#2#3{
	\fill[color=pink] ({#2},{#3}) circle[radius=0.05];
}

\punkteB

\end{scope}


\end{tikzpicture}
\end{document}

