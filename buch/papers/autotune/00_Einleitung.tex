%
% 00_Einleitung.tex
%
% (c) 2023 Florian Baumgartner, OST Ostschweizer Fachhochschule
%
% !TEX root = ../../buch.tex
% !TEX encoding = UTF-8
%
\section{Einleitung\label{autotune:section:teil0}}
\rhead{Einleitung}
Auto-Tune ist eine revolutionäre Technologie, die in der Musikindustrie zur Tonhöhenkorrektur von Gesang und instrumentalen Musikstücken verwendet wird.
Erfunden wurde Auto-Tune im Jahr 1997 von Dr. Andy Hildebrand, einem Erdölgeophysiker,
der seine Kenntnisse der digitaler Signalverarbeitung zur Erkundung von Erdölvorkommen nutzte,
um ein Werkzeug zur Korrektur musikalischer Tonhöhen zu schaffen.
Seine Erfindung, die unter dem Namen Auto-Tune bekannt wurde, hatte drastische Veränderungen in der Musikindustrie zu folge.

Auto-Tune analysiert die Tonhöhen einer Aufnahme und passt sie so an, dass sie mit den für die Musik relevanten Tönen in Einklang gebracht werden,
in der Regel den Tönen einer diatonischen Skala.
Dies kann verwendet werden, um falsche Töne zu korrigieren oder um künstlerische Effekte zu erzeugen, indem die Stimme stark moduliert wird.

Seit seiner Einführung hat sich Auto-Tune zu einem unverzichtbaren Werkzeug in der Musikproduktion entwickelt,
das in einer Vielzahl von Genres eingesetzt wird.
Es hat nicht nur die Art und Weise verändert, wie Musik aufgenommen und produziert wird,
sondern auch die Klanglandschaft der zeitgenössischen Musik geprägt.

Trotz der weitreichenden Anwendung und des Erfolgs von Auto-Tune bleibt die zugrunde liegende Technologie ein faszinierendes Thema der Signalverarbeitung.
Insbesondere die Beziehung zwischen Auto-Tune und der harmonischen Analysis ist ein reichhaltiges Forschungsfeld,
das immer noch eine Fülle von Entdeckungen und Entwicklungen zu bieten hat.


\subsection{Blockschaltbild
\label{autotune:subsection:blockschaltbild}}
Das Blockschaltbild in Abbildung \ref{autotune:fig:blockschaltbild} zeigt die grundlegende Funktionsweise von Auto-Tune.
% TODO: Blockschaltbild einfügen
