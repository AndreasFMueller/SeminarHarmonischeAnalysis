%
% 01_Pitch_Tonsystem_und_Stimmung.tex
%
% (c) 2023 Florian Baumgartner, OST Ostschweizer Fachhochschule
%
% !TEX root = ../../buch.tex
% !TEX encoding = UTF-8
%
\section{Tonsystem und Stimmung
\label{autotune:section:tonsystemUndStimmung}}
\rhead{Tonsystem und Stimmung}
In der westlichen Musik wird das Tonsystem in 12 (Halb)-Töne pro Oktave unterteilt (chromatische Skala).
Dabei gibt es verschiedene Möglichkeiten wie diese zueinander gestimmt werden, also wie sich die Anordnung der Töne unterscheiden.

Als Kammerton, auch Normalstimmton genannt, wird der Ton bezeichnet auf den sich alle Instrumente einstellen.
International hat sich der Kammerton A4 mit einer Frequenz von 440\;Hz durchgesetzt.
Jedoch gibt Länder welche andere Frequenzen verwenden, so ist in der Schweiz der Kammerton A4 mit 442\;Hz üblich.
Im Rahmen dieser Arbeit wird der Kammerton A4 definiert als 
\begin{equation}
    f_{A4}
    =
    440\;\text{Hz}.
\end{equation}

\subsection{Gleichstufige Stimmung
\label{autotune:subsection:gleichstuffigeStimmung}}
Die gleichstufige Stimmung ist das meist verbreitetet Stimmungssystem in der westlichen Musik, insbesondere bei Tasteninstrumenten.
Es ist ein System, bei dem die Oktave in zwölf Halbtöne aufgeteilt ist,
wobei das Frequenzverhältnis zwischen jedem Paar aufeinanderfolgender Töne gleich ist.
Dieses Verhältnis ist die zwölfte Wurzel aus zwei ($\approx 1.05946$). Somit kann die Frequenz eines Tons $f_n$ berechnet werden als

\begin{equation}
    f_n
    =
    \sqrt[12]{2^n} \cdot f_{A4}
    \quad \text{mit} \quad
    n \in \mathbb{Z}.
\end{equation}
Der Vorteil der gleichstufigen Stimmung ist, dass sie in allen Tonarten gleich funktioniert, was bedeutet,
dass ein Musikstück in jede Tonart transponiert werden kann, ohne dass sich die harmonischen Beziehungen zwischen den Tönen ändern.
Ein Nachteil ist jedoch, dass die Intervalle (ausser Oktaven) nicht völlig rein sind, das heisst,
sie weichen leicht von den einfachen ganzzahligen Frequenzverhältnissen ab, die in der reinen Stimmung verwendet werden.

Im Anwendungsbereich von Auto-Tune wird praktisch ausschliesslich die gleichstufige Stimmung verwendet.
Dies hat den einfachen Grund, dass in den Musik-Genres in welchen Auto-Tune verbreitet ist (Pop, Hip-Hop, Elektronische Musik, etc.), die gleichstufige Stimmung üblich ist.
Somit lässt sich elegant eine Tabelle aus Referenzfrequenzen von gleichstufigen Tönen erstellen.
Als Korrekturfrequenz wird dann die Frequenz des nächstgelegenen gleichstufigen Tons aus der Tabelle verwendet.


\subsection{Reine Stimmung
\label{autotune:subsection:reineStimmung}}
Die reine Stimmung, auch bekannt als natürliche oder harmonische Stimmung, ist ein System,
das auf den natürlichen Harmonischen basiert, die in den Schwingungsmustern physikalischer Objekte wie Saiten oder Luftsäulen gefunden werden.
In diesem System werden die Frequenzen der Noten so gewählt, dass sie einfache ganzzahlige Verhältnisse zueinander haben.
Zum Beispiel beträgt das Verhältnis der Frequenzen zweier Töne in einer reinen Quinte 3:2, und in einer reinen Quarte 4:3.
Dies führt zu besonders harmonisch klingenden Intervallen ohne stark bemerkbare Schwebungen.
Allerdings bringt die reine Stimmung einige Probleme mit sich.
Während sie innerhalb einer einzelnen Tonart sehr gut funktioniert, führt sie zu Unstimmigkeiten,
wenn man versucht, zwischen verschiedenen Tonarten zu wechseln.
Deshalb ist die reine Stimmung in der Praxis eher auf Streich- und Blasinstrumente beschränkt,
bei denen die Musiker die Intonation anpassen können.

In der Tabelle \ref{autotune:table:stimmung} werden die Frequenzen der Töne mit reiner Stimmung verglichen mit der gleichstufigen Stimmung.

\begin{table}[htb]
    \begin{tabular}{ccrlrlll}\toprule
    \begin{tabular}[c]{@{}c@{}}$Tonname$\\ $(gleich)$\end{tabular}   & \begin{tabular}[c]{@{}c@{}}$Tonname$\\ $(rein)$\end{tabular}   & $f_{gleich}$                    & $(Hz)$             & $f_{rein}$                    & $(Hz)$                 & $\Delta{f}\;(Hz)$ & $Abweichung$  \\\midrule
    a                                                                & a                                                              & $\sqrt[12]{2^{ 0}}\cdot f_{A4}$ & $=\textbf{440.00}$ & $\sfrac{1}{1}  \cdot f_{A4}$  & $=\textbf{440.00}$     & $0.00$            & $ 0.00\;\%$   \\
    ais/b                                                            & b                                                              & $\sqrt[12]{2^{ 1}}\cdot f_{A4}$ & $\approx 469.16$   & $\sfrac{16}{15}\cdot f_{A4}$  & $= 469.\overline{33}$  & $3.17$            & $11.73\;\%$   \\
    h                                                                & h                                                              & $\sqrt[12]{2^{ 2}}\cdot f_{A4}$ & $\approx 493.88$   & $\sfrac{9}{8}  \cdot f_{A4}$  & $= 495.00$             & $1.12$            & $ 1.96\;\%$   \\
    c                                                                & c                                                              & $\sqrt[12]{2^{ 3}}\cdot f_{A4}$ & $\approx 523.25$   & $\sfrac{6}{5}  \cdot f_{A4}$  & $= 528.00$             & $4.75$            & $ 5.21\;\%$   \\
    cis/des                                                          & des                                                            & $\sqrt[12]{2^{ 4}}\cdot f_{A4}$ & $\approx 554.37$   & $\sfrac{5}{4}  \cdot f_{A4}$  & $= 550.60$             & $4.37$            & $ 3.42\;\%$   \\
    d                                                                & d                                                              & $\sqrt[12]{2^{ 5}}\cdot f_{A4}$ & $\approx 587.33$   & $\sfrac{4}{3}  \cdot f_{A4}$  & $= 586.\overline{66}$  & $0.66$            & $ 0.39\;\%$   \\
    dis/es                                                           & es                                                             & $\sqrt[12]{2^{ 6}}\cdot f_{A4}$ & $\approx 622.25$   & $\sfrac{45}{32}\cdot f_{A4}$  & $= 618.75$             & $3.50$            & $ 1.63\;\%$   \\
    e                                                                & e                                                              & $\sqrt[12]{2^{ 7}}\cdot f_{A4}$ & $\approx 659.26$   & $\sfrac{3}{2}  \cdot f_{A4}$  & $= 660.00$             & $0.74$            & $ 0.28\;\%$   \\
    f                                                                & f                                                              & $\sqrt[12]{2^{ 8}}\cdot f_{A4}$ & $\approx 698.46$   & $\sfrac{8}{5}  \cdot f_{A4}$  & $= 704.00$             & $5.54$            & $ 1.71\;\%$   \\
    fis/ges                                                          & fis                                                            & $\sqrt[12]{2^{ 9}}\cdot f_{A4}$ & $\approx 739.99$   & $\sfrac{5}{3}  \cdot f_{A4}$  & $= 733.\overline{33}$  & $6.66$            & $ 1.74\;\%$   \\
    g                                                                & g                                                              & $\sqrt[12]{2^{10}}\cdot f_{A4}$ & $\approx 783.99$   & $\sfrac{9}{5}  \cdot f_{A4}$  & $= 792.00$             & $8.01$            & $ 1.76\;\%$   \\
    gis/as                                                           & as                                                             & $\sqrt[12]{2^{12}}\cdot f_{A4}$ & $\approx 830.61$   & $\sfrac{15}{8} \cdot f_{A4}$  & $= 825.00$             & $5.61$            & $ 1.07\;\%$   \\
    a                                                                & a                                                              & $\sqrt[12]{2^{12}}\cdot f_{A4}$ & $=880.00$          & $\sfrac{2}{1}  \cdot f_{A4}$  & $= 880.00$             & $0.00$            & $ 0.00\;\%$   \\\bottomrule
    \end{tabular}
    \caption{Vergleich von Frequenzen der Töne mit reiner und gleichstufiger Stimmung.}
    \label{autotune:table:stimmung}
\end{table}


\subsection{Schwebungen
\label{autotune:subsection:schwebungen}}
Es zeigt sich, dass teilweise grosse Abweichungen zwischen den Tönen der reiner und gleichstufiger Stimmung bestehen.
In der gleichstufigen Stimmung kommt es beim Spielen von Kadenzen (z.B. Akkorden) zu Schwebungen.
Dies kann gut anhand eines Beispiels, der a-e Quinte, in einem A-Dur gestimmten Tonsystem gezeigt werden. 
Dabei teilen sich beim gleichstufig- und reingestimmten Intervall beide den Grundton A4 = 440 Hz.
Die Schwebung entsteht nun beim Zweitton E4 und berechnet sich als 

\begin{equation}
    f_\text{Schwebung}
    =
    \left| \sqrt[12]{2^{7}} \cdot f_{A4} - \frac{3}{2} \cdot f_{A4} \right|
    =
    f_{A4} \left| \sqrt[12]{2^{7}} - \frac{3}{2} \right|
    \approx
    0.744\;\text{Hz}. 
\end{equation}
