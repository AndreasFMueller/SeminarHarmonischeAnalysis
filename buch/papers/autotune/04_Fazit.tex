%
% 04_Fazit.tex
%
% (c) 2023 Florian Baumgartner, OST Ostschweizer Fachhochschule
%
% !TEX root = ../../buch.tex
% !TEX encoding = UTF-8
%
\section{Fazit
\label{autotune:section:fazit}}
\rhead{Fazit}
Die mathematischen Prinzipien hinter Auto-Tune, insbesondere die Fourier-Transformation und die Phasenrekonstruktion,
sind Schlüsselelemente für die Genauigkeit und Qualität des endgültigen Ausgangssignals.
Durch fortlaufende Verbesserungen und Anpassungen dieser Algorithmen ist es möglich, immer natürlicher klingende Korrekturen zu erzielen.

Insgesamt zeigt das Thema Auto-Tune eindrucksvoll, wie mathematische Konzepte wie die der harmonischen Analysis direkt in praktischen Anwendungen wie der Musikproduktion eingesetzt werden können.
Die kontinuierliche Weiterentwicklung dieser Techniken bietet nicht nur Möglichkeiten zur Verbesserung der künstlerischen Ausdrucksformen,
sondern liefert auch wertvolle Einblicke in die Komplexität und Vielseitigkeit der digitalen Signalverarbeitung.
