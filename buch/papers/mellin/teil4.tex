%
% teil4.tex
%
% (c) 2023 Vincent Haufe, Hochschule Rapperswil
%
% !TEX root = ../../buch.tex
% !TEX encoding = UTF-8

\section{Lösen des Registrierungsproblems
\label{mellin:section:teil4}}
\rhead{Teil 4}
Im ursprünglichen Problem, das uns über die letzten Abschnitte auf die 
Mellin-Transformation geführt hat, ging es ja darum, den Skalierungsfaktor 
$s$ zu finden, um den sich die beiden sonst identischen Funktionen 
$f(x)$ und $g(x)$ unterscheiden, um diese durch das Kompensieren mit dem 
Faktor $s^{-1}$ zur Deckung zu bringen.
Dies hat uns auf eine Art multiplikative Version der Kreuzkorrelation 
in Gleichung \eqref{mellin:ksigma} geführt, jedoch ist diese, wie schon 
dort erwähnt wurde, eben nicht ganz vollständig. 

Nun, da uns die Mellin-Transformation zur Verfügung steht, kennen wir 
deren Faltungs- und Korrelationsintegral.
Die Korrelation \eqref{mellin:kreuzkorrelation*} entspricht dabei bis 
auf das zusätzliche Mass $x^{-1}$ dem Integral aus Gleichung 
\eqref{mellin:ksigma}. 
Dieser Faktor lässt die Extremalstellen der Korrelation jedoch unverändert 
und findet somit den gesuchten Skalierungsfaktor $s$ unseres 
Registrierungsproblems gleichermassen.
Das nun korrekte Korrelationsintegral lässt sich über das Faltungstheorem 
der Integraltransformationen als Rücktransformation des Produktes der 
beiden Mellin-Transformierten der zu korrelierenden Funktionen beschreiben, 
völlig analog zum Vorgehen aus Abschnitt \ref{mellin:section:teil1}, 
das im ersten Beispiel auf die Lösung \eqref{mellin:x0ft} geführt hat, 
mit der einzigen Ausnahme, dass {\em spiegeln} im Kontext der Gruppe 
$(\mathbb{R^+},\cdot)$ die multiplikative Inversion des Arguments bedeutet:
\begin{align*}
    (f \star g)(\sigma ) 
    &= \int_\mathbb{R^+} 
    f(x) \cdot g(\sigma ^{-1} \cdot x)\,\frac{dx}{x} \\ \\
    &= \int_\mathbb{R^+} 
    f(x) \cdot g((\sigma \cdot x^{-1})^{-1})\,\frac{dx}{x} 
    = (f \ast \check{g})(\sigma)
    .
\end{align*}
Somit lautet die Lösung des Registrierungsproblems mathematisch ausgedrückt
\begin{equation}
    s 
    = \argmax\limits_{\sigma \in \mathbb{R^+}}
    \mathcal{M}^{-1}(\hat{f}(u) \cdot \overline{\hat{g}(u)})(\sigma)
    .
    \label{mellin:smt}
\end{equation}
Welche identisch ist zur Lösung für $x_0$ des ersten Registrierungsproblems, 
nur dass anstatt Fourier hier die Mellin-Transformation zum Einsatz kommt.

\subsection{Quantitative Bestimmung des Gewinns der Transformationsmethode
\label{mellin:subsection:gewinn}}
Das erste Registrierungsproblem mit den zueinander verschobenen Funktionen 
lässt sich sehr einfach in Python mit dem Modul {\em numpy} lösen. 
Dabei werden die beiden Funktionen zuerst diskretisiert indem die Funktionswerte 
für gleichmässig verteilte, diskrete $x$-Werte in ein Array gepackt werden.
Will man diese beiden Datenarrays nun für verschiedene Verschiebungswerte 
korrelieren, stellt {\em numpy} dafür die Funktion {\em convolve(f,g)} zur 
Verfügung. 
Durch Spiegeln der zweiten Funktion, was für das Array eine Umkehrung der 
Indizierung bedeutet, berechnet die Funktion {\em convolve()} also die 
gewünschte Kreuzkorrelation in Form eines weiteren Arrays aus, dessen 
Maximum gesucht ist und sehr einfach über Sortieralgorithmen 
gefunden werden kann. 
Die Funktion {\em convolve()} ist dabei im Hintergrund gemäss der diskreten 
Faltungsformel
\[
    (a \ast v)_n = \sum_{m = -\infty}^{\infty} a_m v_{n-m}
\]
in hoch optimiertem C-Code implementiert.
Für die in diesem Kapitel behandelte Transformationsmethode zur Berechnung der 
Korrelation sind natürlich ebenfalls Funktionen verfügbar. 
Das Modul {\em scipy.signal} bietet die Funktion {\em fftconvolve(f,g)}, die die 
übergebenen Datenarrays zuerst über den FFT-Algorithmus 
transformiert, sie anschliessend elementweise zusammenmultipliziert und dieses 
resultierende Array mit dem inversen FFT-Algorithmus wieder zurücktransformiert.

Mithilfe der Funktionen {\em perf\_counter()} des Moduls {\em time}, lassen sich beide 
Wege quantitativ anhand der für die Berechnung benötigte Rechenzeit miteinander 
vergleichen.
Es gilt allerdings anzumerken, dass die Resultate möglicherweise aufgrund der Natur 
von Python in ihrer Präzision und Wiederholbarkeit eingeschränkt und damit mit 
Vorbehalt zu geniessen sind.

Die Resultate sind dabei stark von den Arraygrösse und somit vom Grad der Auflösung 
der Diskretisierung abhängig.
Die Abbildung \ref{fig:mellin:zeiten} vergleicht die resultierenden Rechenzeiten 
der beiden Methoden für vier verschiedene Auflösungen.
\begin{figure}
    \centering
    \includegraphics[width=\textwidth]{papers/mellin/images/zeiten.pdf}
    \caption{Vgl. der Rechenzeiten verschiedener Auflösungen (linear/logarithmisch)}
    \label{fig:mellin:zeiten}
\end{figure}
Wie man den Resultaten entnehmen kann, steigt die Rechenzeit der direkten Korrelation
ab einer Auflösung von etwa 1000 Funktionswerten im gegebenen Intervall 
$x\in \left[-5,5\right]$ explosionsartig an, während die Rechenzeit der Korrelation 
über die FFT nur etwa linear mit der Auflösung steigt.
Hier zeigt sich also der in Abschnitt \ref{mellin:section:teil1} erwähnte enorme 
Rechenaufwand der direkten Korrelationsmethode.
Für hohe Auflösungen, die eine Anwendung meistens erfordert, ist die 
Transformationsmethode also unerlässlich.

Um auch das zweite Registrierungsproblem rechnerisch lösen zu können, brauchen wir also 
eine diskrete Implementierung der Mellin-Transformation.
Diese könnte man wieder über die Gelfand-Theorie herleiten, als Transformation auf der 
Gruppe $(\mathbb{N}/n\mathbb{N},\cdot)$.
% https://www.researchgate.net/publication/224736063_Discrete_Mellin_transform_for_signal_analysis
Die Relationsgleichungen aus dem vorangehenden Abschnitt erlauben es uns aber, wegen der 
isomorphen Eigenschaft der Exponentialfunktion, erneut die FFT zu verwenden, und somit nicht explizit 
eine diskrete Mellin-Transformation aufzustellen.
Das grosse Problem dabei ist aber, dass die Modifikationen, die die Fourier-Trans\-for\-ma\-tion
in die Mellin-Transformation überführen, im Argument der Funktion durchgeführt werden 
müssen.
Bei den Datenarrays sind aber nur die Funktionswerte gegeben, und nicht die Funktion 
selbst, was es unmöglich macht die neuen Funktionswerte direkt aus den gegeben
Funktionswerten zu bestimmen.
Indirekt kann man das Problem lösen, indem man die Funktionswerte geeignet interpoliert, 
und die Verknüpfungen, die aus der Fourier-Transformation die Mellin-Transformation 
machen, auf das für die Interpolation verwendete Modell anwendet.

Die Aufgabe besteht also darin, aus gegebenen, Datenpunkten $P_k(x,f_k(x))$ für 
$k \in \mathbb{N}$ die entsprechenden Datenpunkte der Funktion $P_k(x,f_k(e^x))$ zu erhalten.
Dies kann elegant mithilfe eine Spline Interpolation erreicht werden, welche die gegebenen 
Datenpunkte über einzelne Polynome 3. Grades verbindet.
Einerseits könnte man nun über diese Spline Interpolation $f_{spl}(x)$ der Funktion $f(x)$ nun die 
benötigte Verknüpfung $f_{spl}(x) \circ exp$ erreichen und an geeigneten Stellen die Funktionswerte 
$f_{spl_k}(e^x)$ berechnen.
Wenn man allerdings anstelle der Datenpunkte $P_k(x,f_k(x))$ zwischen den Datenpunkten 
$P_k(ln(x),f_k(x))$ interpoliert, kann man damit die Funktion $f_k(e^x)$ auch direkt interpolieren, 
und anschliessen an für die FFT relevante Stellen auslesen.
Damit wäre die erste Verknüpfung gemäss\ref{mellin:relationsgleichungen} erreicht, und die Daten 
können dem FFT Algorithmus übergeben werden. 
Nach der Multiplikation der beiden Funktionen im Frequenzraum und aschliessender Rücktransformation 
über die inverse FFT, lassen sich die resultierenden Daten nach demselben Konzept noch einmal 
verarbeiten um die gesuchte Korrelationsfunktion und dessen Maximum zu erhalten.

Bei der konkreten Ausführung des gesamten Prozesses müssen allerdings noch einige grundlegende Fragen 
geklärt werden, wie:
\begin{enumerate}
    \item Welche Stellen der Interpolation von $f_k(e^x)$ müssen ausgewählt werden, um den Prinzipien der Integraltransformationen treu zu bleiben?
    \item Wie hängt das Rückgabearray der FFT mit der neuen Ortsvariable $\sigma$ zusammen und wie muss diese dimensioniert werden?
    \item Welche Stellen der Interpolation von $f_k(ln(x))$ sind relevant für die Korrelationsfunktion?
\end{enumerate}

Diese spannenden Fragen gehen damit in den Bereich der Numerik über und dessen Beantwortung würde den 
Rahmen dieses Kapitels über die Mellin-Transformation leider sprengen, weswegen es mit diesem offen
Ausblick belassen werden soll.





% Interpolation von f(e^t) über Splines -> auslesen an für FFT relevante Stellen 
% -> Interpolation von f(ln(t)) über Splines -> auslesen an für Korrelation relevante Stellen ODER 
%    direkt Maximum bestimmen wenn Form uninteressant, da Verknüpfung mit ln() die Extremalstellen nicht verändert


% Fast Mellin-Transform? Subst e^x in FFT? ---> nicht möglich wenn nur Datenpunkte bekannt (y-Werte)
% das Überführen der FFT in eine MFT funktioniert nur wenn die Funktionen bekannt sind (Interpolation?)


% relevante Stellen der Spline Interpolation für FFT?
% Rückgabearray der FFT? Wie hängt dies mit $\sigma$ zusammen?
% relevante Stellen der Spline Interpolation für Korrelation?


