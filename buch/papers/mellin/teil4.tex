%
% teil4.tex
%
% (c) 2023 Vincent Haufe, Hochschule Rapperswil
%
% !TEX root = ../../buch.tex
% !TEX encoding = UTF-8

\section{Lösen des Registrierungsproblems mithilfe der $\mathcal{MT}$
\label{mellin:section:teil4}}
\rhead{Teil 4}
Im ursprünglichen Problem, das uns über die letzten Abschnitte auf die Mellin-Transformation geführt hat, ging es ja darum, den 
Skalierungsfaktor $s$ zu finden, um den sich die beiden sonst identischen Funktionen $f(x)$ und $g(x)$ unterscheiden, um diese 
durch das Kompensieren mit dem Faktor $s^{-1}$ zur Deckung zu bringen.
Dies hat uns auf eine Art multiplikative Version der Kreuzkorrelation in Gleichung \eqref{mellin:ksigma} geführt, jedoch ist diese, 
wie schon dort erwähnt wurde, eben nicht ganz vollständig. 

Nun, da uns die Mellin-Transformation zur Verfügung steht, kennen wir deren Faltungs- und Korrelationsintegral.
Die Korrelation \eqref{mellin:kreuzkorrelation*} entspricht dabei bis auf das zusäzliche Mass $x^{-1}$ dem Integral aus Gleichung 
\eqref{mellin:ksigma}. 
Dieser Faktor lässt die Extremalstellen der Korrelation jedoch unverändert und findet somit den gesuchten Skalierungsfaktor $s$ 
unseres Registrierungsproblems gleichermassen.
Das nun korrekte Korrelationsintegral lässt sich über das Faltungstheorem der Integraltransformationen als Rücktransformation des Produktes 
der beiden Mellin-Transformierten der zu korrelierenden Funktionen beschreiben, völlig analog zum Vorgehen aus Abschnitt 
\ref{mellin:section:teil1}, das im ersten Beispiel auf die Lösung \eqref{mellin:x0ft} geführt hat, mit der einzigen Ausnahme, dass 
\emph{spiegeln} im Kontext der Gruppe $(\mathbb{R^+},\cdot)$ die multiplikative Inversion des Arguments bedeutet.

\begin{align*}
    (f \star g)(\sigma ) &= \int_\mathbb{R^+} f(x) \cdot g(\sigma ^{-1} \cdot x)\,\frac{dx}{x} \\ \\
    &= \int_\mathbb{R^+} f(x) \cdot g((\sigma \cdot x^{-1})^{-1})\,\frac{dx}{x} = (f \ast \check{g})(\sigma)
\end{align*}
Somit lautet die Lösung des Registrierungsproblems mathematisch ausgedrückt
\begin{equation}
    s = \argmax\limits_{s \in \mathbb{R^+}}{\mathcal{M}^{-1}(\hat{f}(u) \cdot \overline{\hat{g}(u)})}
    \label{mellin:smt}
\end{equation}
Welche identisch ist zur Lösung für $x_0$ des ersten Registrierungsproblems, nur dass anstatt Fourier hier die Mellin-Transformation zum 
Einsatz kommt.

\subsection{Quantitative Bestimmung des Gewinns der Transformationsmethode
\label{mellin:subsection:gewinn}}
% Python Vgl Rechenzeiten
% einfache Funktionen finden
% Fast Mellin-Transform? Subst e^x in FFT?


