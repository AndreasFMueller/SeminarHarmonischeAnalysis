%
% teil2.tex
%
% (c) 2023 Vincent Haufe, Hochschule Rapperswil
%
% !TEX root = ../../buch.tex
% !TEX encoding = UTF-8

\section{Konstruktion der Mellin-Transformation
\label{mellin:section:teil2}}
\rhead{Mellin-Transformation}
Wie schon bemerkt, sind wir bei diesem Problem mit der multiplikativen Gruppe der positiven reellen Zahlen $(\mathbb{R^+},\cdot)$ 
konfrontiert \footnote{Eine ausführliche Beschreibung dessen, was in diesem Abschnitt folgt, findet sich im Kaptitel 
\ref{buch:chapter:gruppen} diese Buches}.
\begin{definition}
    Eine Gruppe setzt sich aus einer \textbf{Menge} und einer \textbf{Verknüpfung} zusammen. Elemente aus der Menge können über die 
    Verknüpfung miteinander verrechnet werden, müssen aber stets wieder Elemente der Menge bilden.
\end{definition}
Diese Information ist der Startpunkt auf der Suche nach einer Integraltransformation die unser Problem lösen kann.
Die Gelfandtheorie ist die verallgemeinerte Theorie der Fourieranalyse und beschreibt die notwendigen Bausteine einer solchen.
Zum einen sind \textbf{Analysefunktionen} zu bestimmen, welche die Gruppenoperation in die Multiplikation überführt und gewisse 
Orthogonalitätsbedingungen erfüllen. 
Des Weiteren muss das \textbf{Skalarprodukt} auf die multiplikative Gruppe angepasst werden, mit dem die Integraltransformation durchgeführt 
werden soll.
Aus Gruppe, Analysefunktionen und Skalarprodukt folgt dann \footnote{für kommutative Lie-Gruppen} die entsprechende Transformation.
% Diese Bedingungen mögen dem Ingenieur auf den ersten Blick etwas ominös und sehr abstrahiert erscheinen, was aber auch der im Sinne 
% einer verallgemeinerten Theorie ist.
% Im Folgenden soll konkret gezeigt werden, wie sich also mithilfe dieser Theorie



% Prinzipien:

% -Gruppe 
% -Analysefunktionen sind Homomorphismen und orthogonal zueinander: h (Eigenfunktionen (Differential)operator?)
% -Skalarprodukt (Integral)
% -Analyse <h,f> = Gf(h)
% -Faltung: Integral über ganze Gruppe mit Gruppenoperation
% -Faltungsformel: Aus Faltung wird Produkt
% -zulässige h bilden duale Gruppe -> Plancherel


\subsection{Analysefunktionen der Mellin-Transformation
\label{mellin:subsection:analysefunktionen}}

Unser Ziel ist es, die für unser Registrierungsproblem neu definierte Kreuzkorrelation mithilfe einer Integraltransformation in 
einfache Multiplikation der transformierten Funktionen zu verwandeln.
Dafür braucht es spezielle Funktionen, sogenannte Homomorphismen, welche die Gruppenoperation in die Multiplikation überführen 
können. 
\begin{satz}
    \label{buch:papers:mellin:teil2:satz:hom}
    Eine (kommutative?) Integraltransformation enthält das Faltungstheorem.
    Dafür braucht es Funktionen, welche die Gruppenoperation in eine \textbf{Multiplikation} überführt. 
    \[
        f(x \diamond y) = f(x) \cdot f(y),
    \]
    wobei $\diamond$ der Gruppenoperation enstpricht.
\end{satz}
Dies gilt sowohl für die Gruppenoperationen der Gruppe im Ortsraum als auch für die Gruppenoperation der dualen Gruppe im Bildraum der Transformation.
Mit dem Konzept der dualen Gruppe werden wir uns am Ende des Abschnitts nochmals genauer befassen, fürs Erste konzentrieren wir uns auf 
die multiplikative Gruppe in $\mathbb{R^+}$, welche relevant für unsere skalierten Funktionen ist, also im Ortsraum.
\medskip

Bei der Fouriertransformation, dessen Gruppenoperation die Additon ist, müssen die Analysefunktionen also die Gleichung 
\begin{equation}
    f(x + y) = f(x) \cdot f(y)
    \label{mellin:hom1}
\end{equation}
erfüllen. 
Die Funktion, die diese Eigenschaft besitzt ist natürlich die Exponentialfunktionen, denn 
\begin{equation}
    e^{x + y} = e^x \cdot e^y,
    \label{mellin:exp}
\end{equation}
und somit die bekannte Analysefunktion der Fouriertransformation.
In der Fortsetzung bedeutet das für die neue Transformation, dass dessen Analysefunktionen die Gleichung
\begin{equation}
    f(x \cdot y) = f(x) \cdot f(y)
    \label{mellin:hom2}
\end{equation}
erfüllen müssen.
Erfüllt wird die Gleichung durch jede Potenzfunktion $h(x) = x^{-z}$ für beliebige $z \in \mathbb{C}$
\footnote{das Minus vor dem $z$ ist bezüglich der Homomorphismuseigenschaft unbedeutend, muss aber aus demselben Grund wie 
bei der Fouriertransformation ($e^{-j\omega t}$) enthalten sein, auf den hier aber nicht eingegangen werden soll}
\begin{equation}
    h(x \cdot y) =  (x \cdot y)^{z} = x^{z} \cdot y^{z},
\end{equation}
% denn Potenzen lassen sich bekanntlich beliebig zusammenfassen so wie
% \begin{equation}
%     (a \cdot b)^2 = a^2 \cdot b^2 \left(\frac{a}{b}\right)^2  = \frac{a^2}{b^2} 
% \end{equation}
Unsere Analysefunktionen müssen also vom Typ Potenzfunktion sein.
% Die komplexe Exponentialfunktion $e^{j\omega_l t}$ ist orthogonal zu $e^{j\omega_k t}$, sofern $\omega_l \neq \omega_k$, da diese 
% immer ein Skalarprodukt gleich Null haben wie sich leicht zeigen lässt
% \begin{proof}
%     \begin{align*}
%         \langle h_l,h_k \rangle &= \int_\mathbb{R} e^{j\omega_l t} \cdot e^{-j\omega_k t}\,dt \\
%         &= \int_\mathbb{R} e^{j(\omega_l - \omega_l) t}\,dt \\
%         &= \int_\mathbb{R} e^{j\omega_l-k t}\,dt = 0 \vert \omega_l \neq \omega_k 
%     \end{align*}
%     Da die komplexe Exponentialfunktion periodisch ist.
% \end{proof}
% Bei den komplexen Exponentialfunktionen der Fouriertransformation ist dies auch intuitiv, da dahinter die Sinus- und 
% Cosinusschwingungen unterschiedlicher Frequenzen stehen, in die eine Funktion zerlegt werden kann.


\subsection{Skalarprodukt der Mellin-Transformation
\label{mellin:subsection:skalarprodukt}}
Die Entdeckung der Fourierreihen brachte die Notwendigkeit mit sich, den Riemannschen Integralbegriff
\footnote{Siehe auch Kapitel \ref{buch:skalarprodukt:section:funktionenraeume}} zu überdenken und zu präzisieren.
Dafür wurde vor allem die aufkommende Masstheorie angewendet.
Relevant ist dabei nun, dass wir für unser Integral und somit Skalarprodukt mit welchem wir die Transformation durchführen wollen, 
mit einem \emph{Mass}\footnote{in diesem Kontext genauer: ein Haar-Mass} ausstatten. 
Damit gelingt es uns, das Integral \emph{invariant bezüglich der Gruppenoperation} zu machen, was für den Erfolg der Fouriertheorie 
zwingend nötig ist. 
\begin{satz}
    \label{buch:papers:mellin:teil2:satz:int}
    Das Integral, mit dem eine Integraltransformation durchgeführt wird, muss \textbf{invariant bezüglich der Gruppenoperation} 
    der Gruppe der zu transformierenden Funktionen sein.
    \[
        \int_\mathbb{G} f(x)\,dx = \int_\mathbb{G} f(x \diamond x_0)\,dx
    \]
\end{satz}
Invariant bezüglich der Gruppenoperation bedeutet, dass sich das Integral über den gesamten Definitionsbereich der Gruppe nicht 
ändert, wenn das Argument der zu integrierenden Funktion ``verschoben'' wird. 
Das gewöhnliche Integral der Fouriertransformation erfüllt diese Eigenschaft 
\begin{equation}
    \int_\mathbb{R} f(x)\,dx = \int_\mathbb{R} f(x + x_0)\,dx
\end{equation}
da hier ja von $-\infty$ bis $+\infty$ integriert wird und somit eine nur endlich verschobene Funktion stets vollständig abmisst.
Verschoben meint in diesem Sinne aber Verschiebung mit der Gruppenoperation, was bei Fourier also ein ``+'' war, wird bei unserer 
neuen Transformation zu einem ``$\cdot$'', das heisst zu einer Streckung des Arguments. 
Es wird schnell klar, dass das herkömmliche Integral eben \emph{nicht} invariant bezüglich dieser Streckung ist und allgemein gilt 
\begin{equation}
    \int_\mathbb{R^+} f(x)\,dx \neq \int_\mathbb{R^+} f(s \cdot x)\,dx
\end{equation}
da, einfach gesagt, eine Streckung der x-Achse die Funktion und somit deren Flächeninhalt auseinderzieht.
Es braucht nun also eine Anpassung, um diese Veränderung des Flächeninhalts zu kompensieren.
Das kann überraschend einfach mithilfe einer Division durch die Variable des Arguments $x$ erreicht werden, was auch sehr einfach gezeigt 
werden kann.
\begin{proof}[Beweis]
    Zu beweisen ist die folgende Gleichung:
    \[
        \int_\mathbb{R^+} f(s \cdot x)\,\frac{dx}{x} = \int_\mathbb{R^+} f(x)\,\frac{dx}{x}
    \]
    Eine Substitution von $s \cdot x$ mit $u$
    \[
    % \left[
    \begin{aligned}
        u &= sx \\
        x &= \frac{u}{s} \\
        dx &= \frac{1}{s} du
    \end{aligned}
    % \right]
    \]
    führt auf
    \[
        \int_\mathbb{R^+} f(sx) \cdot x^{-1}\,dx = \int_\mathbb{R^+} f(u) \cdot \frac{s}{u} \frac{1}{s}\,du 
        = \int_\mathbb{R^+} f(u)\,\frac{du}{u}
    \]
    Was beweist, dass sich der Faktor $s$ durch das zusätzliche Mass $x^{-1}$ herauskürzt. 
\end{proof}
Eine Intuition für diesen Umstand liefert auch ein Beispiel.
\begin{beispiel}
Man betrachte die Funktion 
\[
f(x) 
= 
\begin{cases}
    x^3 &\qquad 0\leq x\leq 1\\
    0 &\qquad \text{sonst}
\end{cases}
\]
% to do
\end{beispiel}
Das für die Multiplikation als Gruppenoperation modifizierte Integral ist somit
\begin{equation}
    \int_\mathbb{R^+} f(x)\,\frac{dx}{x},
\end{equation}
und das Skalarprodukt mit dem die Transformation erfolgt
\begin{equation}
    \langle f,g \rangle = \int_\mathbb{R^+} \overline{f(x)} \cdot g(x) \,\frac{dx}{x}.
\end{equation}
% Are you sure?


\subsection{Korrelation und Faltung auf $\mathbb{R^+}$
\label{mellin:subsection:faltung}}
Wir haben nun alle essenziellen Bausteine beisammen um die gesuchte Transformation zu formulieren: grundlegende Gruppe, 
Analysefunktionen und Skalarprodukt.
Das Rezept dafür lautet nun wie folgt. Es wird das für die Gruppe neu definierte Skalarprodukt der Analysefunktion  $h(x)$
mit der zu transformierenden Funktion $f(x)$ gebildet.
\begin{equation}
    \langle h,f \rangle = \int\limits_{0}^{\infty} x^{z} f(x) \,\frac{\mathrm{d}x}{x}
\end{equation}
eine kleine Umformung führt zu einer kompakten Formel der \emph{Mellin-Transformation}
\begin{equation}
    \mathcal{M}\{f \}(z) = \int\limits_{0}^{\infty} x^{z-1} f(x) \,\mathrm{d}x  \qquad\text{für $z \in \mathbb{C}$}
    \label{mellin:mellin}
\end{equation}
Dies ist die allgemeinste Form der Mellin-Transformation. Wenn man aber die komplexe Zahl $z = \delta + ju$ auf ihren Imaginärteil beschränkt
$\operatorname{Re}(z) = \delta = 0$ erhält man die Formel
\begin{equation}
    \mathcal{M}\{f \}(u) = \int\limits_{0}^{\infty} x^{ju-1} f(x) \,\mathrm{d}x  \qquad\text{für $u \in \mathbb{R}$},
    \label{mellin:mellinu}
\end{equation}
die sich noch als nützlich und intuitiv erweisen wird.

Ebenfalls lassen sich die Korrelation sowie die Faltung über die Gruppenoperation und das modifizierte Skalarprodukt bestimmen:
\begin{equation}
    (f \star g)(\sigma ) = \int_\mathbb{R^+} f(x) \cdot g(\sigma ^{-1} \cdot x)\,\frac{dx}{x}
    \label{mellin:kreuzkorrelation*}
\end{equation}
\begin{equation}
    (f \ast g)(x) = \int_\mathbb{R^+} f(y) \cdot g(x \cdot y^{-1})\,\frac{dy}{y} 
\end{equation}
Damit erhalten wir von der Gelfandtheorie zum krönenden Abschluss das Faltungstheorem
\begin{equation}
    \mathcal{M}\{f \ast g\} = \mathcal{M}\{f\} \cdot \mathcal{M}\{g\}.
\end{equation}

\subsection{Rücktransformation
\label{mellin:subsection:ruecktransformation}}
Die Mellin-Transformation wurde nun aus den Regeln der Gelfandtheorie konstruiert und findet in den Gleichungen 
\eqref{mellin:mellin} \& \eqref{mellin:mellinu} ihre definitive Form.
% \footnote{obschon je nach Literatur und Geschmack auch noch Konstanten $\frac{1}{\sqrt{2\pi j}}$, $\frac{1}{2\pi j}$ vorkommen können.}
Jedoch haben wir damit eigentlich erst die Hälfte des Bildes, schliesslich wollen wir aus dem Bildraum der Transformation, 
nennen wir ihn \textbf{Mellin-Raum}, wieder in den Ortsraum zurücktransformieren. 
Bei Fourier ging dies mit dem fast gleichen Integral wie das der Hintransformation, mit Ausnahme der Skalierungskonstante $\frac{1}{2\pi}$
und dem Kehren des Vorzeichens vor der Integrationsvariable.
\[
\begin{aligned}
    \hat{f}(\omega) &= \int\limits_{-\infty}^{\infty} e^{-i\omega{}t} f(t) \,\mathrm{d}t \\
    f(x) &= \frac{1}{2\pi} \int\limits_{-\infty}^{\infty} e^{j\omega t} \hat{f}(\omega) \,\mathrm{d}\omega
\end{aligned}
\]
Tatsächlich beruht diese Symmetrie der Hin- und Rücktransformation auf der zur Fouriertransformation gehörenden \emph{dualen} Gruppe.
\begin{definition}
    Die duale Gruppe einer Integraltransformation ist ...
\end{definition}
% Wichtig ist zu unterscheiden, dass sich der Satz \ref{buch:papers:mellin:teil2:satz:hom} auf das Faltungstheorem bezieht, und 
% \emph{nicht} auf die Gruppenoperation des dualen Raums. <-- this probably gugus aber verwechslungsgefahr, Erklärungsbedarf?
Untersucht man also, was die duale Gruppe der Fouriertransformation ist, also die Gruppe des Frequenzraums, kommt man auf die 
Gruppe $(\mathbb{R},+)$, welche diesselbe ist wie die des Zeitraums!
Aus diesem Grund sind die Transformationsintegrale diesselben, nämlich das Skalarprodukt der Gruppe $(\mathbb{R},+)$.
Was also vielleicht vorher als selbstverständliche Symmetrie angesehen wurde, ist in Tat und Wahrheit nicht immer so.
Will man nun die duale Gruppe der Mellin-Transformation bestimmen kommt es auf die Definition an. 

Die duale Gruppe der Mellin-Transformation aus Gleichung \eqref{mellin:mellinu} ist ebenfalls, wie die der Fouriertransformation, die Gruppe 
$(\mathbb{R},+)$.
Das liegt auch daran, dass bei der Rücktransformation die Analysefunktion nicht mehr über $x$, sondern neu über die Variable des 
Frequenzraums $u$ integriert wird. 
Durch diesen feinen Unterschied, wandelt sich die Natur der Analysefunktion von einer Potenzfunktion in Richtung der Hintransformation zu 
einer Exponentialfunktion in Richtung der Rücktransformation.
Wie in der Fouriertheorie, erfolgt die Zerlegung einer Funktion in komplexe Exponentialfunktion, oder eben komplexe Schwingungen.
Der kleine Unterschied ist dabei nur, dass die Variable $x$ der Funktion als Basis eben jener Exponentialfunktionen vorkommt, und nicht 
im Exponent.
% Was heisst das?
Da wir nun verstehen, dass die duale Gruppe der Mellin-Transformation die Gruppe $(\mathbb{R},+)$ ist, können wir auf gleiche Weise wie 
die Hintransformation konstruiert wurde, auch die Rücktransformation konstruieren.
Dieses Mal müssen wir aber nichts Neues herleiten, denn die Analysefunktionen müssen diesselben sein, und das Skalarprodukt auf der 
Gruppe $(\mathbb{R},+)$ ist dasselbe wie bei der Fouriertransformation, das normale Skalarprodukt aus \eqref{mellin:skalaprodukt}. 
Das Zusammensetzen dieser Bausteine sowie das Hinzumultiplizieren einer Skalierungskonstante liefert die Inversionsformel der 
Mellin-Transformation
\begin{equation}
    f(x) = \frac{1}{2\pi} \int\limits_{-\infty}^{\infty} x^{-ju} \hat{f}(u) \,\mathrm{d}u
    \label{mellin:mellininvu}
\end{equation}

Gehen wir der Definition der Mellin-Transformation aus \eqref{mellin:mellin} nach, und lassen einen Realteil $\delta \neq 0$ für die Variable $z$ zu, 
was einer zusätzlichen Dämpfung oder Verstärkung der komplexen Exponentialfunktionen entspricht, ist die Menge der dualen Gruppe nun die gesamten 
komplexen Zahlen $\mathbb{C}$.
Weil dadurch die Variable des Mellin-Raums komplex ist, ist diese gewissermassen zweidimensional hinsichtlich des Real- und Imaginärteils. 
Da bezüglich der Konvergenz des Transformationsintegrals nicht alle Werte von $\delta$ zulässig sind, bedeutet das, dass die Mellin-Transformation 
für komplexe $z$ in einem Streifen der komplexen Zahlenebene existiert, auch \emph{Streifen der Analyzität} 
\footnote{aus dem Englischen: strip of analyticity oder SOA} gennant.
% pgfplot SOA und Integrationspfad entlang vertikaler linie
Dies erinnert stark an den Laplaceraum, dessen Variable ebenfalls komplex ist und meist in einer Halbebene der komplexen Zahlenebene ab einem 
gewissen Wert $\delta_0$ für den Realteil existiert.
Bei der Rücktransformation muss allerdings der Realteil $\delta \in SOA$ fixiert werden. 
Integriert wird daraufhin über die Parallele der imaginäre Achse an der Stelle von $\delta$. 

\begin{equation}
    f(x) = \frac{1}{2\pi j} \int\limits_{\delta -j\infty}^{\delta +j\infty} x^{-z} \hat{f}(z) \,\mathrm{d}z
    \label{mellin:mellininv}
\end{equation}
Somit vervollständigt sich das Bild der Mellin-Transformation.




% Zum Abschluss dieses Abschnitts soll noch ein BLick auf die Rücktransformation der Mellin-Transformation geworfen werden.
% Diese deckt sich grösstenteils
% \begin{equation}
%     f(x) = \frac{1}{2\pi j} \int\limits_{-\infty}^{\infty} x^{-z} \hat{f}(z) \,\mathrm{d}z
%     \label{mellin:mellininv}
% \end{equation}
% Etwas anders als bei der Inversionsformel der Fouriertransformation ist, dass das sich Integral für Hin- und Rücktransformation leicht 
% voneinander unterscheidet. 
% Der Grund dafür liegt bei einem recht wichtigen , die duale Gruppe.
% Dies liegt daran, dass bei der Fouriertransformation beide Variablen, die des Ortsraum und des Frequenzraumes, beide nebeneinander in 
% der Exponentialfunktion stehen, während bei der Mellin-Transformation in der Analysefunktion die Variable des Ortsraums die Basis der 
% Potenz bildet, die Variable des Bildraums den Exponent.
% Das führt zu mehreren Asymmetrien der Transformationsrichtungen.
% Die Grenzen der Rücktransformation gehen von $-\infty \rightarrow \infty$, da hier über $z$, die nicht an die Menge $\mathbb{R^+}$ 
% gebunden ist, integriert wird. 
% Aus dem selben Grund fällt das Mass der Hintransformation $x^{-1}$ weg.





% Analysefunktionen zur Bestimmung der dualen Gruppe: Fourier & Mellin

% Bem. Konvergenz und Scope
% Beispiel Skaleninvarianz
% Orthogonalität ausführen (Charaktere?)
% Integralbegriff und Masstheorie?

