%
% teil6.tex
%
% (c) 2023 Vincent Haufe, Hochschule Rapperswil
%
% !TEX root = ../../buch.tex
% !TEX encoding = UTF-8

\section{Schlusswort
\label{mellin:section:teil6}}
\rhead{Teil 6}
In diesem Kapitel wurde die Mellin-Transformation als Beispiel 
einer Variante der Fourier-Trans\-for\-ma\-tion untersucht.
Dabei ist herausgekommen, dass eine Funktion je nach verwendeten 
Analysefunktionen auf verschiedene Arten in ein Frequenzspektrum 
zerlegt werden kann. 
Diese Varianten der Fourier-Transformation erlauben es, 
Problemstellungen für andere mathematische Gruppen nach dem selben 
Prinzip der harmonischen Analyse anzugehen.
In diesem Kapitel ging es dabei um das Problem der multiplikativen 
Korrelation, wirklich spannend wird diese Theorie aber bei der 
Anwendung auf speziellere, auch nicht kommutativen Gruppen, wo es 
noch einiges an Potenzial der harmonischen Analyse auszuschöpfen gibt.