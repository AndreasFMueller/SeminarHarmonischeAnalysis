%
% teil0.tex
%
% (c) 2023 Vincent Haufe, Hochschule Rapperswil
%
% !TEX root = ../../buch.tex
% !TEX encoding = UTF-8

\section{Einleitung\label{mellin:section:teil0}}
\rhead{Teil 0}

Das folgende Kapitel beschäftigt sich mit einer Integraltransformation,
die vom finnischen Mathematiker Hjalmar Mellin gegen Ende des 19.
Jahrhunderts erstmals vorgestellt wurde, die Mellin-Transformation.
Allem voran gilt es zu bemerken, dass in diesem Kapitel lediglich eine
kurze Vorstellung der Mellin-Transformation gegeben wird, ohne dabei zu
tief in die mathematischen Details, Bedingungen oder rigorose Beweise 
einzugehen.
Vielmehr soll es demonstrieren, dass sich hinter der Fourier-Theorie eben eine
viel allgemeinere Theorie der Integraltransformationen verbirgt, 
die Gelfand-Transformation von Kapitel \ref{buch:gruppen:section:gelfand}, 
und sich damit Integraltransformationen für viele, meist speziellere 
Anwendungen und Problemstellungen finden lassen.
% Additionally, it will deepen our understanding of our dear and holy fourier-transform and may lead to a deeper appreciation for it.
Zudem wird es nebenbei das Verständnis der Fourier-Transformation vertiefen
und vielleicht sogar zu einer grösseren Wertschätzung derselben führen.
Das Kapitel bedingt grundlegende Kenntnisse der Integraltransformationen.


% Im Folgenden soll für ein Registrierungsproblem mithilfe elementarer Gruppen- und Gelfandtheorie die entsprechende 
% Integraltransformation hergeleitet und auf einige Eigenschaften dieser Mellin-Transformation, insbesondere der Rücktransformation,
% eingegangen werden.
% Ein weiterer Abschnitt beschäftigt sich damit, die Parallelen zur wohlbekannten Fourier-Transformation zu ziehen.
% Danach soll das ursprünglich gestellte Registrierungsproblem mit der gefundenen Transformation gelöst und mit der traditionellen
% Methode verglichen wird.
% Ein letzter Abschnitt behandelt eine Methode der Integralauflösung mithilfe der Mellin-Transformation, welche im Bereich der
% Antennentechnik bei Berechnungen eingesetzt werden kann.







