%
% teil0.tex
%
% (c) 2023 Vincent Haufe, Hochschule Rapperswil
%
% !TEX root = ../../buch.tex
% !TEX encoding = UTF-8

\section{Einleitung\label{mellin:section:teil0}}
\rhead{Teil 0}

Das folgende Kapitel beschäftigt sich mit einer Integraltransformation, die vom finnischen Mathematiker Hjalmar 
Mellin am Ende des 19. Jahrhunderts erstmals vorgestellt wurde, die Mellin-Transformation\footnote{\cite{mellin:mellin-wikipedia}}.
Allem voran gilt es zu bemerken, dass in diesem Kapitel lediglich eine kurze Vorstellung der Mellin-Transformation 
gegeben wird, ohne dabei zu tief in die unzähligen mathematischen Details, Bedingungen oder rigorose Beweise einzugehen.
% ,die eine konkrete Umsetzung meist erfordern würde. 
Mehr soll es demonstrieren, dass sich hinter der Fourier-Theorie eben eine viel allgemeinere Theorie der Integraltransformationen 
verbirgt (namentlich die Gelfand-Transformation)\footnote{Siehe auch Kapitel \ref{buch:gruppen:section:gelfand}}, und sich damit 
Integraltransformationen für viele, meist speziellere Anwendungen und Problemstellungen finden lassen.
Das Kapitel bedingt grundlegende Kenntnisse der Integraltransformationen.


% Im Folgenden soll für ein Registrierungsproblem mithilfe elementarer Gruppen- und Gelfandtheorie die entsprechende 
% Integraltransformation hergeleitet und auf einige Eigenschaften dieser Mellin-Transformation, insbesondere der Rücktransformation,
% eingegangen werden.
% Ein weiterer Abschnitt beschäftigt sich damit, die Parallelen zur wohlbekannten Fouriertransformation zu ziehen.
% Danach soll das ursprünglich gestellte Registrierungsproblem mit der gefundenen Transformation gelöst und mit der traditionellen
% Methode verglichen wird.
% Ein letzter Abschnitt behandelt eine Methode der Integralauflösung mithilfe der Mellin-Transformation, welche im Bereich der
% Antennentechnik bei Berechnungen eingesetzt werden kann.



% Hierbei soll nicht zu sehr auf die komplexeren mathematischen Grundlagen eingegangen werden, welche die Mellin-Transformation
% in der Anwendung erfordern und den Rahmen dieses Kapitels sprengen würde (Für den interessierten Leser: KOMA).
% Es geht mehr um das generelle Konzept der Ausweitung der Fouriertheorie auf andere Gruppen anhand dieses konkreten Beispiels
% der Mellin-Transformation und darum die Verbindung und Parallelen zur wohlbekannten Fouriertransformation aufzuzeigen.


% \section{Einleitung}
%     \subsection{Zielsetzung}
%     -Vereinfachte Demonstration des Prozesses, eine Integraltransformationen nach Gelfandtheorie zu konstruieren.
%     -Lösen eines Registrierungsproblems mit der Mellin-Transformation
%     -Vorstellen einer Methode zur analytische(re)n Lösung von technischen Integralen, die die Mellin-Transformation beinhaltet.


% \section{Motivation}
% \subsection{Registrierungsproblem}

% \section{Konstruktion der Mellin-Transformation gemäss Gelfand-Theorie}
% Gelfand
%     \subsection{Verträglichkeit des Integrals}
%     Integralbegriff
%     Skaleninvarianz
%     Haar-Mass
%     \subsection{Analysefunktion}
%     Homomorphismus
%     Orthogonalität
%     \subsection{Definieren der Faltung/Korrelation}
%     \subsection{Rücktransformation}
%     Gamma-Funktion
%     Mellin-Barnes
%     \subsection{Bem. Konvergenz und Scope}

% \section{Vergleich mit der Fouriertransformation}
% Substitutionsmethode bei Mellin (weil so ähnlich)
% Eigenschaften
% Laplace

% \section{Lösen des Registrierungsproblems mithilfe der $\mathcal{MT}$}
% Python Korrelation vs Mellin -> vgl. Rechenzeiten

% \section{weitere Anwendungen des Faltungstheorems auf R+}
% Antennentechnik / Elektromagnetismus
% Anwendung auch ohne KOMA, Verständnis nützlich zum Erkennen wann nützlich

% \section{Conclusion}
% Was wurde gezeigt
% Limitations
% Anwendungsscope der Mellin-Transformation
% Wissen zur Anwendung
% Weiterführende Literatur






