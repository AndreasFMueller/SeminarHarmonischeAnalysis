%
% teil5.tex
%
% (c) 2023 Vincent Haufe, Hochschule Rapperswil
%
% !TEX root = ../../buch.tex
% !TEX encoding = UTF-8

\section{weitere Anwendungen des Faltungstheorems auf $\mathbb{R^+}$
\label{mellin:section:teil5}}
\rhead{Teil 5}
Sed ut perspiciatis unde omnis iste natus error sit voluptatem
accusantium doloremque laudantium, totam rem aperiam, eaque ipsa
quae ab illo inventore veritatis et quasi architecto beatae vitae
dicta sunt explicabo. Nemo enim ipsam voluptatem quia voluptas sit
aspernatur aut odit aut fugit, sed quia consequuntur magni dolores
eos qui ratione voluptatem sequi nesciunt. Neque porro quisquam
est, qui dolorem ipsum quia dolor sit amet, consectetur, adipisci
velit, sed quia non numquam eius modi tempora incidunt ut labore
et dolore magnam aliquam quaerat voluptatem. Ut enim ad minima
veniam, quis nostrum exercitationem ullam corporis suscipit laboriosam,
nisi ut aliquid ex ea commodi consequatur? Quis autem vel eum iure
reprehenderit qui in ea voluptate velit esse quam nihil molestiae
consequatur, vel illum qui dolorem eum fugiat quo voluptas nulla
pariatur?

\subsection{De finibus bonorum et malorum
\label{mellin:subsection:malorum}}
At vero eos et accusamus et iusto odio dignissimos ducimus qui
blanditiis praesentium voluptatum deleniti atque corrupti quos
dolores et quas molestias excepturi sint occaecati cupiditate non
provident, similique sunt in culpa qui officia deserunt mollitia
animi, id est laborum et dolorum fuga. Et harum quidem rerum facilis
est et expedita distinctio. Nam libero tempore, cum soluta nobis
est eligendi optio cumque nihil impedit quo minus id quod maxime
placeat facere possimus, omnis voluptas assumenda est, omnis dolor
repellendus. Temporibus autem quibusdam et aut officiis debitis aut
rerum necessitatibus saepe eveniet ut et voluptates repudiandae
sint et molestiae non recusandae. Itaque earum rerum hic tenetur a
sapiente delectus, ut aut reiciendis voluptatibus maiores alias
consequatur aut perferendis doloribus asperiores repellat.


\section{Conclusion
\label{mellin:section:teil6}}
\rhead{Teil 6}

% Einstieg, Anwendung führt in KomA (oder symbolic integration routines, software), weitere Analyse in Gelfandtheorie (the Beyond)
%     Gamma-Funktion
%     Mellin-Barnes
% kleine Horizonterweiterung            