%
% teil3.tex
%
% (c) 2023 Vincent Haufe, Hochschule Rapperswil
%
% !TEX root = ../../buch.tex
% !TEX encoding = UTF-8
%
\section{Vergleich mit der Fouriertransformation
\label{mellin:section:teil3}}
\rhead{Teil 3}
Im vorangehenden Abschnitt wurde die Mellin-Transformation aus den Regeln der Gelfandtheorie konstruiert. 
Dabei wurde auch immer wieder auf die bekannte Fouriertransformation verwiesen und deren Parallelen gezogen, um die doch sehr abstrahierte 
Theorie etwas zu bodigen.
Das hat dabei in diesem Fall besonders gut funktioniert, da die Fourier- und Mellin-Transformation nämlich nicht nur beide eine generische 
Gelfandtransformation sind, sondern auch untereinander enger verwandt sind als prinzipiell nötig wäre.
Der folgende Abschnitt erkundet eine Alternative, wie man auf die Mellin-Transformation auf sehr einfache Weise auch direkt vom Integral 
der Fouriertransformation hätte kommen können.

\subsection{Von Fourier zu Mellin
\label{mellin:subsection:foumel}}
Was folgt ist eine einzelne Rechnung.
Wir starten mit dem bekannten Integral der Fouriertransformation
\begin{equation}
    \mathcal{F}\{f \}(\omega) = \int\limits_{-\infty}^{\infty} e^{-i\omega{}t} f(t) \,\mathrm{d}t
    \label{mellin:fourier}
\end{equation}
Jetzt gilt es, zwei Substitution durchzuführen
\[
    \left[
    \begin{aligned}
        -j\omega &= z
    \end{aligned}
    \right]
\]
und
\[
    \left[
    \begin{aligned}
        t &= \ln x \\
        dt &= \frac{1}{x} dx
    \end{aligned}
    \right]
\]
Eingesetzt in das Fourier-Integral ergibt sich
\[
    \int e^{z \ln x} \cdot f(\ln x) \,\frac{dx}{x} = \int e^{\ln x^z} \cdot f(\ln x) \,\frac{dx}{x}
    = \int x^{z} \cdot f(\ln x) \,\frac{dx}{x} = \int x^{z-1} \cdot f(\ln x) \,\mathrm{d}x
\]
Nun gilt es bei der Substitution der Integrationsvariablen $t$ noch die neuen Grenzen zu berechnen
\[
    \begin{aligned}
        &e^{t} =& x \\
        &e^{-\infty} &\rightarrow 0 \\
        &e^{\infty} &\rightarrow \infty 
    \end{aligned}
\]
Dies führt auf das Integral
\[
    \int\limits_{0}^{\infty} x^{z-1} f(\ln x) \,\mathrm{d}x
\]
Das nun bekannt vorkommen sollte, denn es entspricht exakt der Mellin-Transformation!

Die Fouriertransformation einer Funktion ist also dasselbe wie die Mellin-Transformation derselben Funktion, welche aber
mit dem natürlichen Logarithmus logarithmiert wurde, oder andersherumn, nimmt man das Argument einer Funktion 
in den Exponent von $e$ $f(e^x)$ und fouriertransformiert diese, ergibt dies die Mellin-Transformation der Funktion $f(x)$.
Dies ergibt zusätzlich Sinn, da die Exponentialfunktion eine Funktion von $\mathbb{R} \mapsto \mathbb{R^+}$ ist, ein 
Gruppenhomomorphismus.
Das ist eine erstaunliche Erkenntnis und lässt ein paar einfache Relationen zur Fourier- und Laplace-Transformation 
formulieren.
\begin{align*}
    \mathcal{M}\{f(x)\}(z) &= \mathcal{F}\{f (e^{x}\}(jz) \\ \\
    \mathcal{M}\{f(x)\}(z) &= \mathcal{L}\{f (e^{-x}\}(-z) + \mathcal{L}\{f (e^{x}\}(z) 
\end{align*}
Wie die Hintransformation, die in diesem Abschnitt hergeleitet wurde, kann auch die Rücktransformation der Mellin-Transformation gegen 
Ende des letzten Abschnitts aus dieser Relation einfach hergeleitet werden, auch ohne Kenntnis der zugrundeliegenden Gruppentheorie.

In Theorie könnte die Mellin-Transformation also alle Aufgaben der in der modernen Welt allgegenwärtigen Fouriertransformation 
übernehmen. 
Joseph Fourier hätte die Wärmeleitgleichung damals ebenso mit der Mellin-Transformation lösen können und ein Grossteil der modernen 
Elektrotechnik könnte darauf basieren. 
In der Praxis wäre diese aber bedeutend weniger handlich und intuitiv als die Fouriertransformation, deren Gleichungen etwas symmetrischer
daherkommen. %und bei der man eine sehr genaue, physikalische Vorstellung des Frequenzraumes und deren Wirkung hat. -really?

Auch überrascht nun gar nicht mehr, dass eigentlich alle Eigenschaften der Fouriertransformation übersetzt werden können, 
was vorher im Kontext der abstrakten Gelfandtheorie vielleicht noch etwas undurchsichtig erschienen ist.
