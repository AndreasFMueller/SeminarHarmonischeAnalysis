%
% teil3.tex
%
% (c) 2023 Vincent Haufe, Hochschule Rapperswil
%
% !TEX root = ../../buch.tex
% !TEX encoding = UTF-8
%
\section{Vergleich mit der Fouriertransformation
\label{mellin:section:teil3}}
\rhead{Teil 3}
Im vorangehenden Abschnitt wurde die Mellin-Transformation aus den 
Regeln der Gelfandtheorie konstruiert. 
Dabei wurde auch immer wieder auf die bekannte Fouriertransformation 
verwiesen und deren Parallelen gezogen, um die doch sehr abstrahierte 
Theorie etwas zu bodigen.
Das hat dabei in diesem Fall besonders gut funktioniert, da die Fourier- 
und Mellin-Transformation nämlich nicht nur beide eine generische 
Gelfandtransformation sind, sondern auch untereinander enger verwandt 
sind als prinzipiell nötig wäre.
Der folgende Abschnitt erkundet eine Alternative, wie man auf die 
Mellin-Transformation auf sehr einfache Weise auch direkt vom Integral 
der Fouriertransformation hätte kommen können.

\subsection{Von Fourier zu Mellin
\label{mellin:subsection:foumel}}
Was folgt ist eine einzelne Rechnung.
Wir starten mit dem bekannten Integral der Fouriertransformation
\begin{equation}
    \mathcal{F}\{f \}(\omega) = 
    \int\limits_{-\infty}^{\infty} e^{-j\omega{}t} f(t) \,\mathrm{d}t
    \label{mellin:fourier}
\end{equation}
Jetzt gilt es, zwei Substitution durchzuführen
\[
    \begin{aligned}
        -j\omega &= z
    \end{aligned}
\]
und
\[
    \begin{aligned}
        t &= \ln x \\
        \mathrm{d}t &= \frac{1}{x} \mathrm{d}x
    \end{aligned}
\]
Eingesetzt in das Fourier-Integral ergibt sich
\[
    \begin{aligned}
    &\int e^{z \ln x} \cdot f(\ln x) \,\frac{\mathrm{d}x}{x} \\
    = &\int e^{\ln x^z} \cdot f(\ln x) \,\frac{\mathrm{d}x}{x} \\
    = &\int x^{z} \cdot f(\ln x) \,\frac{\mathrm{d}x}{x} \\
    = &\int x^{z-1} \cdot f(\ln x) \,\mathrm{d}x \\
    \end{aligned}
\]
Nun gilt es bei der Substitution der Integrationsvariablen $t$ noch die 
neuen Grenzen zu berechnen
\[
    \begin{aligned}
        &e^{t} &= x \\
        &e^{-\infty} &\rightarrow 0 \\
        &e^{\infty} &\rightarrow \infty 
    \end{aligned}
\]
Dies führt auf das Integral
\[
    \int\limits_{0}^{\infty} x^{z-1} f(\ln x) \,\mathrm{d}x
\]
Das nun bekannt vorkommen sollte, denn es entspricht exakt der 
Mellin-Transformation!
\medskip

Die Fouriertransformation einer Funktion ist also dasselbe wie die 
Mellin-Transformation derselben Funktion, welche aber mit dem natürlichen 
Logarithmus logarithmiert wurde, oder andersherumn, nimmt man das 
Argument einer Funktion in den Exponent von $e \mapsto e^x$ und 
fouriertransformiert diese, ergibt dies die Mellin-Transformation 
der Funktion $f(x)$.
Dies rührt daher, da die Exponentialfunktion eine Funktion von 
$\mathbb{R} \mapsto \mathbb{R^+}$ ist.
Das ist eine erstaunliche Erkenntnis und lässt ein paar einfache 
Relationen zur Fourier- und Laplace-Transformation formulieren.
\begin{align*}
    \mathcal{M}\left\{f(x)\right\}(z) 
    &= \mathcal{F}\left\{f (e^{t})\right\}(jz) \\ \\
    \mathcal{M}\left\{f(x)\right\}(z) 
    &= \mathcal{L}\left\{f (e^{-t})\right\}(-z) 
    + \mathcal{L}\left\{f (e^{t})\right\}(z) 
\end{align*}
und
\begin{align*}
    \mathcal{M}^{-1}\left\{f(z)\right\}(x) 
    &= \mathcal{F}^{-1}\left\{f (jz)\right\}(e^t) \\ \\
    \mathcal{M}^{-1}\left\{f(z)\right\}(x) 
    &= \mathcal{L}^{-1}\left\{f (-z)\right\}(e^{-t}) 
    + \mathcal{L}^{-1}\left\{f (z)\right\}(e^{t}) 
\end{align*}
Aus diesen Relationen können also Hin- und Rücktransformationsformeln 
der Mellin-Transformation einfachst aus den Formeln der Fourier- oder 
Laplacetransformation hergeleitet werden, ganz ohne Kenntnis der 
zugrundeliegenden Gruppen- beziehungweise Gelfandtheorie.
Auch überrascht nun gar nicht mehr, dass eigentlich alle Eigenschaften 
der Fouriertransformation übersetzt werden können, was vorher im Kontext 
der abstrakten Gelfandtheorie vielleicht noch etwas undurchsichtig 
erschienen ist.

In der Theorie könnte die Mellin-Transformation also alle Aufgaben der 
in der modernen Welt allgegenwärtigen Fouriertransformation übernehmen. 
Joseph Fourier hätte die Wärmeleitgleichung damals ebenso mit der 
Mellin-Transformation lösen können und ein Grossteil der modernen 
Elektrotechnik könnte darauf basieren. 
Aus diesem Vergleich offenbart sich aber auch die Schönheit der 
Fouriertransformation. 
Durch die Symmetrie der Gruppen und dadurch der Transformationsgleichungen 
kommt diese extrem handlich und intuitiv daher und ist der 
Mellin-Transformation in dieser Hinsicht wohl doch überlegen.
Die Vorstellung der Fouriertransformation, als das Aus- und 
Einrollen eines Seiles in einer Seiltrommel ist in diesem Kontext 
einzigartig.
% to be elaborated



