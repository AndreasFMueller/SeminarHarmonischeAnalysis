%
% teil3.tex
%
% (c) 2023 Vincent Haufe, Hochschule Rapperswil
%
% !TEX root = ../../buch.tex
% !TEX encoding = UTF-8
%
\section{Vergleich mit der Fourier-Transformation
\label{mellin:section:teil3}}
\rhead{Teil 3}
Im vorangehenden Abschnitt wurde die Mellin-Transformation aus den 
Regeln der Gelfand-Theorie konstruiert. 
Dabei wurde auch immer wieder auf die bekannte Fourier-Transformation 
verwiesen und deren Parallelen gezogen, um die doch sehr abstrakte 
Theorie etwas zu bodigen.
Das hat dabei in diesem Fall besonders gut funktioniert, da die 
Fourier- und Mellin-Transformation nämlich nicht nur beide eine 
generische Gelfand-Transformation sind, sondern auch untereinander 
enger verwandt sind als prinzipiell nötig wäre.
Der folgende Abschnitt erkundet eine Alternative, wie man auf die 
Mellin-Transformation auf sehr einfache Weise auch direkt vom 
Integral der Fourier-Transformation hätte kommen können.

\subsection{Von Fourier zu Mellin
\label{mellin:subsection:foumel}}
Was folgt ist eine direkte Rechnung.
Wir starten mit dem bekannten Integral der Fourier-Trans\-for\-ma\-tion
\begin{equation}
    \mathcal{F}\{f \}(\omega) 
    = \int_{-\infty}^{\infty} e^{-j\omega{}t} f(t) \,{d}t
    .
    \label{mellin:fourier}
\end{equation}
Jetzt gilt es, zwei Substitution durchzuführen, nämlich
\begin{align*}
    -j\omega &= z
\end{align*}
und
\begin{align*}
    t &= \ln x \\
    {d}t &= \frac{1}{x}\,{d}x
    .
\end{align*}
Eingesetzt in das Fourier-Integral ergibt sich
\begin{align*}
    \int e^{z \ln x} \cdot f(\ln x) \,\frac{{d}x}{x}
    = &\int e^{\ln x^z} \cdot f(\ln x) \,\frac{{d}x}{x} \\
    = &\int x^{z} \cdot f(\ln x) \,\frac{{d}x}{x} \\
    = &\int x^{z-1} \cdot f(\ln x) \,{d}x
    .
\end{align*}
Nun gilt es bei der Substitution der Integrationsvariablen $t$ 
noch die neuen Grenzen zu bestimmen.
\begin{align*}
    e^{t} &= x \\
    e^{-\infty} &\rightarrow 0 \\
    e^{\infty} &\rightarrow \infty 
\end{align*}
Dies führt auf das Integral
\[
    \int_{e^{-\infty}}^{e^{\infty}} x^{z-1} f(\ln x) \,{d}x 
    = \int_{0}^{\infty} x^{z-1} f(\ln x) \,{d}x
    ,
\]
das nun bekannt vorkommen sollte, denn es entspricht exakt der 
Mellin-Transformation!
\medskip

Die Fourier-Transformation einer Funktion ist also dasselbe wie die 
Mellin-Transformation derselben Funktion, welche aber mit dem 
natürlichen Logarithmus logarithmiert wurde, oder andersherum, 
nimmt man das Argument einer Funktion in den Exponent von 
$e \mapsto e^x$ und fourier-transformiert diese, ergibt dies die 
Mellin-Transformation der Funktion $f(x)$.
Dies rührt daher, da die Exponentialfunktion eine Funktion von 
$\mathbb{R} \mapsto \mathbb{R^+}$ ist.
Etwas algebraischer lässt sich die Exponentialfunktion als 
Isomorphismus der additiven Gruppe auf die multiplikative Gruppe 
\[
exp : (\mathbb{R},+) \to (\mathbb{R^+},\cdot):x \mapsto e^x
\]
verstehen. 
Da sich die beiden Gruppen also über die Exponentialfunktion 
ineinander überführen lassen, sind sie eigentlich beide nur eine Seite 
der gleichen Münze, nämlich eine eindimensionale, kommutative, nicht 
kompakte Lie-Gruppe, wovon es nur eine einzige gibt.
\medskip

Das ist eine erstaunliche Erkenntnis und lässt ein paar einfache 
Relationen zur Fourier- und Laplace-Transformation formulieren:
\begin{align*}
    \mathcal{M}\left(f(x)\right)(z) 
    &= \mathcal{F}\left(f (e^{t})\right)(jz), \\
    \mathcal{M}\left(f(x)\right)(z) 
    &= \mathcal{L}\left(f (e^{-t})\right)(-z) 
    + \mathcal{L}\left(f (e^{t})\right)(z) 
    .
\end{align*}
Diese etwas ineinander verschachtelte Darstellung lässt sich auch 
als Folge von Verknüpfungen betrachten, also das in Reihe Schalten 
von verschiedenen Funktionen, wobei die Funktion auf der Linken den 
Ausgang der Funktion auf der Rechten als Eingang nimmt.
Dabei sei die Exponentialfunktion $\exp(t) = e^t$, die Multiplikation 
des Arguments mit $j$ die Funktion $m_j(x) = jx$ und die 
Multiplikation des Arguments mit $-1$ die Funktion $m_{-1}(x) = -x$.
So ergibt sich folgende Darstellung:
\begin{align*}
    \mathcal{M}(f) 
    &= \mathcal{F}(f\circ\exp)\circ m_j, \\
    \mathcal{M}(f) 
    &= \mathcal{L}(f\circ\exp\circ m_{-1})\circ m_{-1}
    + \mathcal{L}(f\circ\exp)
    .
    \label{mellin:relationsgleichungen}
\end{align*}
Die Darstellung über die Verknüpfungen fungiert dabei als Rezept, 
das genau die Schritte beschreibt, die man ausführen muss um von der 
Fourier- oder Laplace-Transformation direkt auf die 
Mellin-Transformation zu kommen. 
Das gleiche Spiel kann auch umgekehrt, für die Rücktransformationen 
gemacht werden:
\begin{align*}
    \mathcal{M}^{-1}(\hat{f}) 
    &= \mathcal{F}^{-1}(\hat{f}\circ m_{-j})\circ\ln, \\
    \mathcal{M}^{-1}(\hat{f}) 
    &= \mathcal{L}(\hat{f}\circ m_{-1})\circ\ln\circ m_{-1}
    + \mathcal{L}(\hat{f})\circ\ln
    .
\end{align*}
% \begin{align*}
%     \mathcal{M}^{-1}\left(f(z)\right)(x) 
%     &= \mathcal{F}^{-1}\left(f (jz)\right)(e^t) \\ \\
%     \mathcal{M}^{-1}\left(f(z)\right)(x) 
%     &= \mathcal{L}^{-1}\left(f (-z)\right)(e^{-t}) 
%     + \mathcal{L}^{-1}\left(f (z)\right)(e^{t}) 
% \end{align*}
Aus diesen Relationen können also Hin- und Rücktransformationsformeln 
der Mellin-Trans\-for\-ma\-tion einfachst aus den Formeln der 
Fourier- oder Laplace-Transformation hergeleitet werden, ganz ohne 
Kenntnis der zugrundeliegenden Gruppen- beziehungweise Gelfand-Theorie.
Auch überrascht nun gar nicht mehr, dass eigentlich alle Eigenschaften 
der Fourier-Transformation übersetzt werden können, was vorher im 
Kontext der abstrakten Gelfand-Theorie vielleicht noch etwas 
undurchsichtig erschienen ist.

In der Theorie könnte die Mellin-Transformation also alle Aufgaben 
der in der modernen Welt allgegenwärtigen Fourier-Transformation 
übernehmen. 
Joseph Fourier hätte die Wärmeleitgleichung damals ebenso mit der 
Mellin-Transformation lösen können und ein Grossteil der modernen 
Elektrotechnik könnte darauf basieren. 
Aus diesem Vergleich offenbart sich aber auch die Schönheit der 
Fourier-Transformation. 
Durch die Symmetrie der Gruppen und dadurch der 
Transformationsgleichungen, kommt diese extrem handlich und intuitiv 
daher und ist der Mellin-Transformation in dieser Hinsicht 
wohl doch überlegen.
Nichtsdestotrotz hat die Mellin-Transformation durchaus ihre 
Berechtigung für Probleme, die spezifisch auf der multiplikativen 
Gruppe vorkommen.
% Die Vorstellung der Fourier-Transformation, als das Aus- und 
% Einrollen eines Seiles in einer Seiltrommel ist in diesem Kontext 
% einzigartig.
% to be elaborated



