%
% teil1.tex -- Beispiel-File für das Paper
%
% (c) 2020 Prof Dr Andreas Müller, Hochschule Rapperswil
%
% !TEX root = ../../buch.tex
% !TEX encoding = UTF-8
%
\section{Diskrete Kosinus-Transformation
\label{jpeg:section:dct}}
\rhead{dct}
Die Fouriertransformation und auch die diskrete Variante davon erzeugen ein komplexwertiges Spektrum, auch wenn das ursprüngliche Signal nur aus reellen Werten besteht.
Da weder die Sinus- noch Kosinusfunktion alleine ausreichen, um ein vollständiges Basisfunktionssystem bilden zu können.
Ein reellwertiges Signal bildet im Frequenzspektrum jeweils konjugiert komplexe Paare, ist also redundant, es muss nur die Hälfte der Werte berechnet werden.
Die diskrete Kosinus Transformation ermöglicht genau das, sie bildet ein Reihe aus Kosinus Basisfunktionen und ist beschränkt auf reellwertige Signale, genau was wir benötigen.
Es existiert analog dazu die diskrete Sinus Transformation, die aus Sinus Basisfunktionen besteht.

Für die Berechnung der Koeffizienten wird die Diskrete Kosinustransformation (DCT-II oder DCT) mit folgender Formel\footnote{Herleitung für DCT-II\url{https://jberner.info/data/BSc_Thesis_Berner.pdf}} benötigt 
\begin{equation}
    b_k
    =
    \sum \limits_{n=0}^{N-1} a_n 
    \cos\left(
        \frac{k\pi}{N}\left(n + \frac{1}{2} \right) 
    \right),
    \qquad k = 0,\dots,N-1
\label{jpeg:equationdct2}
\end{equation}
wobei \(b_k\) das Transformierte und \(a_n\) das Einganssignal ist.
Um von der transformierten Form in die ursprüngliche zu gelangen braucht man die inverse Kosinustransformation (DCT-III oder inverse DCT)
\begin{equation}
    a_n
    =
    \frac{b_0}{2} +
    \sum \limits_{k=1}^{N-1} b_k 
    \cos\left(
    \frac{k\pi}{N}\left(n + \frac{1}{2} \right) 
    \right),
    \qquad k = 0,\dots,N-1.
    \label{jpeg:equationdct3}
\end{equation}
Dabei lässt sich mit diesen Formeln nur die eindimensionale Transformation durchführen.
Bei Bilden handelt es sich aber um zweidimensionale Signale

\subsection{Zweidimensionale Kosinus Transformation
\label{jpeg:subsection:dctdim2}}
Für die zweidimensionale Variante der DCT wird das Bild zunächst zeilenweise betrachtet. Jede dieser Zeilen sind wiederum eindimensionale Signale und transformierbar.
Werden auf jeder der Zeilen die Transformation angewandt, entsteht ein neues Bild.
Dieses wird nun in Spalten betrachtet und ebenfalls transformiert.
Das resultierende Bild davon ist die vollständige Transformation.
Analog dazu lässt sich die inverse Kosinus Transformation beschreiben.

Die Gleichung \eqref{jpeg:equationdct2} zweidimensional lässt sich
\begin{equation}
    b_{k,l}
    =
    \sum \limits_{n=0}^{N-1} 
    \sum \limits_{m=0}^{N-1} a_{n,m} 
    \cos\left(
    \frac{k\pi}{N}\left(n + \frac{1}{2} \right) 
    \right)
    \cos\left(
    \frac{l\pi}{N}\left(m + \frac{1}{2} \right) 
    \right)
    \label{jpeg:equationdct2dim2}
\end{equation}
und die inverse Transformation  
\begin{align*}
    a_{n,m}
    =
    \frac{b_{0,0}}{4} &+
    \sum \limits_{k=1}^{N-1} 
    \frac{b_{k,0}}{2} 
    \cos\left(
    \frac{k\pi}{N}\left(n + \frac{1}{2} \right) 
    \right) \\ &+
    \sum \limits_{l=1}^{N-1} 
    \frac{b_{0,l}}{2} 
    \cos\left(
    \frac{l\pi}{N}\left(m + \frac{1}{2} \right) 
    \right) \\ &+
    \sum \limits_{k=1}^{N-1} 
    \sum \limits_{l=1}^{N-1} b_{k,l} 
    \cos\left(
    \frac{k\pi}{N}\left(n + \frac{1}{2} \right) 
    \right)
    \cos\left(
    \frac{l\pi}{N}\left(m + \frac{1}{2} \right) 
    \right)
    \label{jpeg:equationdct3dim2}
\end{align*}
beschreiben.

\begin{figure}
    \centering
    \includegraphics[width=60mm]{papers/jpeg/pictures/dctjpeg.pdf}
    \caption{Koeffizienten der zweidimensionalen DCT mit \(N=8\)
        \label{jpeg:fig:dctkoeff}}
\end{figure}

In Abschnitt \ref{jpeg:fig:dctkoeff}  zeigt eine zweidimensionale DCT mit in jeder Dimension jeweils Acht Koeffizienten.
Oben rechts ist der DC Anteil und in horizontaler Richtung sind die Frequenzanteile aufsteigend.
Gleiches gilt für die vertikale Richtung und die Linearkombination aus beiden.
Diese wird so für die Transformation von den Pixelblöcken in der Bildverarbeitung mittels JPEG verfahren verwendet.

