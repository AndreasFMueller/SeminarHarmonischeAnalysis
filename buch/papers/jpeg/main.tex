%
% main.tex -- Paper zum Thema <jpeg>
%
% (c) 2020 Autor, OST Ostschweizer Fachhochschule
%
% !TEX root = ../../buch.tex
% !TEX encoding = UTF-8
%
\chapter{Bildkompression\label{chapter:jpeg}}
\kopflinks{Bildkompression Methoden}
\begin{refsection}
\chapterauthor{Jakob Gierer}

Ein paar Hinweise für die korrekte Formatierung des Textes
\begin{itemize}
\item
Absätze werden gebildet, indem man eine Leerzeile einfügt.
Die Verwendung von \verb+\\+ ist nur in Tabellen und Arrays gestattet.
\item
Die explizite Platzierung von Bildern ist nicht erlaubt, entsprechende
Optionen werden gelöscht. 
Verwenden Sie Labels und Verweise, um auf Bilder hinzuweisen.
\item
Beginnen Sie jeden Satz auf einer neuen Zeile. 
Damit ermöglichen Sie dem Versionsverwaltungssysteme, Änderungen
in verschiedenen Sätzen von verschiedenen Autoren ohne Konflikt 
anzuwenden.
\item 
Bilden Sie auch für Formeln kurze Zeilen, einerseits der besseren
Übersicht wegen, aber auch um GIT die Arbeit zu erleichtern.
\end{itemize}

%
% einleitung.tex -- Beispiel-File für die Einleitung
%
% (c) 2023 Dmitry Grigoriev, OST Ostschweizer Fachhochschule
%
% !TEX root = ../../buch.tex
% !TEX encoding = UTF-8
%
\section{Einleitung\label{spektral:section:Einleitung}}
\rhead{Einleitung}

Spektrale Methoden spielen eine wichtige Rolle in der Meteorologie, insbesondere in der Analyse von atmosphärischen Phänomenen und Daten.
Diese Methoden basieren auf der Fourier-Transformation, die es ermöglicht, Signale in den Frequenzbereich zu transformieren und somit ihre spektrale Zusammensetzung zu analysieren.
In der Meteorologie werden spektrale Methoden auf verschiedene Arten angewendet:

\begin{itemize}
\item
\textbf{Spektrale Analyse von Zeitreihen:} Meteorologische Daten, wie Temperatur, Druck, Windgeschwindigkeit usw., werden oft als Zeitreihen erfasst.
Die Fourier-Transformation ermöglicht die Aufschlüsselung dieser Zeitreihen in verschiedene Frequenzkomponenten.
Dies kann hilfreich sein, um periodische Muster oder Trends in den Daten zu identifizieren. 
Zum Beispiel können saisonale Schwankungen in den Temperaturdaten erkannt werden.
\item
\textbf{Wellenanalyse:} Spektrale Methoden werden verwendet, um Wellenphänomene in der Atmosphäre zu analysieren, wie zum Beispiel atmosphärische Schwingungen, Schallwellen, Gezeitenwellen und Rossby-Wellen.
Die Identifizierung von Wellenmustern und -frequenzen hilft bei der Vorhersage von Wetter- und Klimaereignissen.
\item
\textbf{Numerische Wettervorhersage (NWP):} In der numerischen Wettervorhersage werden spektrale Methoden verwendet, um die atmosphärischen Zustandsänderungen im Laufe der Zeit zu simulieren.
Dies geschieht durch die Diskretisierung der atmosphärischen Gleichungen im spektralen Raum und ihre Lösung mithilfe von numerischen Verfahren.
Diese Modelle helfen bei der Vorhersage von Wetterereignissen über verschiedene Zeitskalen.
\item
\textbf{Klimaforschung:} In der Klimaforschung werden spektrale Methoden verwendet, um langfristige klimatische Veränderungen zu analysieren.
\end{itemize}

Diese Anwendungen verdeutlichen, wie spektrale Methoden in der Meteorologie zur Analyse, Modellierung und Vorhersage von atmosphärischen Phänomenen und Prozessen eingesetzt werden können.

Wir werden uns auf die Wellenanalyse und die Numerische Wettervorhersage konzientrieren.
%
% teil1.tex -- Beispiel-File für das Paper
%
% (c) 2020 Prof Dr Andreas Müller, Hochschule Rapperswil
%
% !TEX root = ../../buch.tex
% !TEX encoding = UTF-8
%
\section{Diskrete Kosinus-Transformation
\label{jpeg:section:dct}}
\rhead{dct}
Die Fouriertransformation und auch die diskrete (DFT) Variante davon erzeugen ein komplexwertiges Spektrum, auch wenn das ursprüngliche Signal nur aus reellen Werten besteht.
Da weder die Sinus- noch Kosinusfunktion alleine ausreichen, um ein vollständiges Basisfunktionssystem bilden zu können.
Ein reellwertiges Signal bildet im Frequenzspektrum jeweils konjugiert komplexe Paare, ist also redundant, es muss nur die Hälfte der Werte berechnet werden.
Die diskrete Kosinus Transformation ermöglicht genau das, sie bildet ein Reihe aus Kosinus Basisfunktionen und ist beschränkt auf reellwertige Signale, genau was wir benötigen.

Da stellt sich die Frage, wie es möglich ist, dass die diskrete Kosinus Transformation ohne die Sinus Komponenten auskommt die bei der DFT unentbehrlich sind.
Das wird erreicht, indem das Intervall in der die Transformation durchgeführt wird, gegenüber der DFT gerade symmetrisch verdoppelt wird.
Es existiert analog dazu die diskrete Sinus Transformation, die aus Sinus Basisfunktionen besteht.

Für die Berechnung der Koeffizienten wird die Diskrete Kosinustransformation (DCT-II oder DCT) mit folgender Formel\footnote{Herleitung für DCT-II\url{https://jberner.info/data/BSc_Thesis_Berner.pdf}} benötigt 
\begin{equation}
    b_k
    =
    \sum \limits_{n=0}^{N-1} a_n 
    \cos\left(
        \frac{k\pi}{N}\left(n + \frac{1}{2} \right) 
    \right),
    \qquad k = 0,\dots,N-1
\label{jpeg:equationdct2}
\end{equation}
wobei \(b_k\) das Transformierte und \(a_n\) das Einganssignal ist.
Um von der transformierten Form in die ursprüngliche zu gelangen braucht man die inverse Kosinustransformation (DCT-III oder inverse DCT)
\begin{equation}
    a_n
    =
    \frac{b_0}{2} +
    \sum \limits_{k=1}^{N-1} b_k 
    \cos\left(
    \frac{k\pi}{N}\left(n + \frac{1}{2} \right) 
    \right),
    \qquad k = 0,\dots,N-1.
    \label{jpeg:equationdct3}
\end{equation}
Dabei lässt sich mit diesen Formeln nur die eindimensionale Transformation durchführen.
Bei Bilden handelt es sich aber um zweidimensionale Signale

\subsection{Zweidimensionale Kosinus Transformation
\label{jpeg:subsection:dctdim2}}
Für die zweidimensionale Variante der DCT wird das Bild zunächst zeilenweise betrachtet. Jede dieser Zeilen sind wiederum eindimensionale Signale und transformierbar.
Werden auf jeder der Zeilen die Transformation angewandt, entsteht ein neues Bild.
Dieses wird nun in Spalten betrachtet und ebenfalls transformiert.
Das resultierende Bild davon ist die vollständige Transformation.
Analog dazu lässt sich die inverse Kosinus Transformation beschreiben.

Die Gleichung \eqref{jpeg:equationdct2} zweidimensional lässt sich
\begin{equation}
    b_{k,l}
    =
    \sum \limits_{n=0}^{N-1} 
    \sum \limits_{m=0}^{N-1} a_{n,m} 
    \cos\left(
    \frac{k\pi}{N}\left(n + \frac{1}{2} \right) 
    \right)
    \cos\left(
    \frac{l\pi}{N}\left(m + \frac{1}{2} \right) 
    \right)
    \label{jpeg:equationdct2dim2}
\end{equation}
und die inverse Transformation  
\begin{align*}
    a_{n,m}
    =
    \frac{b_{0,0}}{4} &+
    \sum \limits_{k=1}^{N-1} 
    \frac{b_{k,0}}{2} 
    \cos\left(
    \frac{k\pi}{N}\left(n + \frac{1}{2} \right) 
    \right) \\ &+
    \sum \limits_{l=1}^{N-1} 
    \frac{b_{0,l}}{2} 
    \cos\left(
    \frac{l\pi}{N}\left(m + \frac{1}{2} \right) 
    \right) \\ &+
    \sum \limits_{k=1}^{N-1} 
    \sum \limits_{l=1}^{N-1} b_{k,l} 
    \cos\left(
    \frac{k\pi}{N}\left(n + \frac{1}{2} \right) 
    \right)
    \cos\left(
    \frac{l\pi}{N}\left(m + \frac{1}{2} \right) 
    \right)
    \label{jpeg:equationdct3dim2}
\end{align*}
beschreiben.

\begin{figure}
    \centering
    \includegraphics[width=55mm]{papers/jpeg/pictures/dctjpeg.pdf}
    \caption{Koeffizienten der zweidimensionalen DCT mit \(N=8\)
        \label{jpeg:fig:dctkoeff}}
\end{figure}

Abbildung \ref{jpeg:fig:dctkoeff}  zeigt eine zweidimensionale DCT mit in jeder Dimension jeweils Acht Koeffizienten.
Oben rechts ist der DC Anteil und in horizontaler Richtung sind die Frequenzanteile aufsteigend.
Gleiches gilt für die vertikale Richtung und die Linearkombination aus beiden.
Diese wird so für die Transformation von den Pixelblöcken in der Bildverarbeitung mittels JPEG verfahren verwendet.
Möchte man das Bild berechnen muss jeweils das Element aus der Koeffizientenmatrix mit der entsprechenden Funktion multipliziert und aufaddiert werden.


%
% teil2.tex -- Beispiel-File für teil2 
%
% (c) 2020 Prof Dr Andreas Müller, Hochschule Rapperswil
%
% !TEX root = ../../buch.tex
% !TEX encoding = UTF-8
%
\section{Teil 2 
\label{jpeg:section:teil2}}
\rhead{Teil 2}
Sed ut perspiciatis unde omnis iste natus error sit voluptatem
accusantium doloremque laudantium, totam rem aperiam, eaque ipsa
quae ab illo inventore veritatis et quasi architecto beatae vitae
dicta sunt explicabo. Nemo enim ipsam voluptatem quia voluptas sit
aspernatur aut odit aut fugit, sed quia consequuntur magni dolores
eos qui ratione voluptatem sequi nesciunt. Neque porro quisquam
est, qui dolorem ipsum quia dolor sit amet, consectetur, adipisci
velit, sed quia non numquam eius modi tempora incidunt ut labore
et dolore magnam aliquam quaerat voluptatem. Ut enim ad minima
veniam, quis nostrum exercitationem ullam corporis suscipit laboriosam,
nisi ut aliquid ex ea commodi consequatur? Quis autem vel eum iure
reprehenderit qui in ea voluptate velit esse quam nihil molestiae
consequatur, vel illum qui dolorem eum fugiat quo voluptas nulla
pariatur?

\subsection{De finibus bonorum et malorum
\label{jpeg:subsection:bonorum}}
At vero eos et accusamus et iusto odio dignissimos ducimus qui
blanditiis praesentium voluptatum deleniti atque corrupti quos
dolores et quas molestias excepturi sint occaecati cupiditate non
provident, similique sunt in culpa qui officia deserunt mollitia
animi, id est laborum et dolorum fuga. Et harum quidem rerum facilis
est et expedita distinctio. Nam libero tempore, cum soluta nobis
est eligendi optio cumque nihil impedit quo minus id quod maxime
placeat facere possimus, omnis voluptas assumenda est, omnis dolor
repellendus. Temporibus autem quibusdam et aut officiis debitis aut
rerum necessitatibus saepe eveniet ut et voluptates repudiandae
sint et molestiae non recusandae. Itaque earum rerum hic tenetur a
sapiente delectus, ut aut reiciendis voluptatibus maiores alias
consequatur aut perferendis doloribus asperiores repellat.



%
% teil3.tex
%
% (c) 2023 Vincent Haufe, Hochschule Rapperswil
%
% !TEX root = ../../buch.tex
% !TEX encoding = UTF-8
%
\section{Vergleich mit der Fouriertransformation
\label{mellin:section:teil3}}
\rhead{Teil 3}
Im vorangehenden Abschnitt wurde die Mellin-Transformation aus den 
Regeln der Gelfandtheorie konstruiert. 
Dabei wurde auch immer wieder auf die bekannte Fouriertransformation 
verwiesen und deren Parallelen gezogen, um die doch sehr abstrahierte 
Theorie etwas zu bodigen.
Das hat dabei in diesem Fall besonders gut funktioniert, da die Fourier- 
und Mellin-Transformation nämlich nicht nur beide eine generische 
Gelfandtransformation sind, sondern auch untereinander enger verwandt 
sind als prinzipiell nötig wäre.
Der folgende Abschnitt erkundet eine Alternative, wie man auf die 
Mellin-Transformation auf sehr einfache Weise auch direkt vom Integral 
der Fouriertransformation hätte kommen können.

\subsection{Von Fourier zu Mellin
\label{mellin:subsection:foumel}}
Was folgt ist eine einzelne Rechnung.
Wir starten mit dem bekannten Integral der Fouriertransformation
\begin{equation}
    \mathcal{F}\{f \}(\omega) = 
    \int\limits_{-\infty}^{\infty} e^{-j\omega{}t} f(t) \,\mathrm{d}t
    \label{mellin:fourier}
\end{equation}
Jetzt gilt es, zwei Substitution durchzuführen
\begin{align*}
    -j\omega &= z
\end{align*}
und
\begin{align*}
    t &= \ln x \\
    \mathrm{d}t &= \frac{1}{x} \mathrm{d}x
\end{align*}
Eingesetzt in das Fourier-Integral ergibt sich
\begin{align*}
    \int e^{z \ln x} \cdot f(\ln x) \,\frac{\mathrm{d}x}{x}
    = &\int e^{\ln x^z} \cdot f(\ln x) \,\frac{\mathrm{d}x}{x} \\
    = &\int x^{z} \cdot f(\ln x) \,\frac{\mathrm{d}x}{x} \\
    = &\int x^{z-1} \cdot f(\ln x) \,\mathrm{d}x \\
\end{align*}
Nun gilt es bei der Substitution der Integrationsvariablen $t$ noch die 
neuen Grenzen zu berechnen
\begin{align*}
    e^{t} &= x \\
    e^{-\infty} &\rightarrow 0 \\
    e^{\infty} &\rightarrow \infty 
\end{align*}
Dies führt auf das Integral
\[
    \int\limits_{0}^{\infty} x^{z-1} f(\ln x) \,\mathrm{d}x
\]
Das nun bekannt vorkommen sollte, denn es entspricht exakt der 
Mellin-Transformation!
\medskip

Die Fouriertransformation einer Funktion ist also dasselbe wie die 
Mellin-Transformation derselben Funktion, welche aber mit dem natürlichen 
Logarithmus logarithmiert wurde, oder andersherumn, nimmt man das 
Argument einer Funktion in den Exponent von $e \mapsto e^x$ und 
fouriertransformiert diese, ergibt dies die Mellin-Transformation 
der Funktion $f(x)$.
Dies rührt daher, da die Exponentialfunktion eine Funktion von 
$\mathbb{R} \mapsto \mathbb{R^+}$ ist.
Das ist eine erstaunliche Erkenntnis und lässt ein paar einfache 
Relationen zur Fourier- und Laplace-Transformation formulieren.
\begin{align*}
    \mathcal{M}\left\{f(x)\right\}(z) 
    &= \mathcal{F}\left\{f (e^{t})\right\}(jz) \\ \\
    \mathcal{M}\left\{f(x)\right\}(z) 
    &= \mathcal{L}\left\{f (e^{-t})\right\}(-z) 
    + \mathcal{L}\left\{f (e^{t})\right\}(z) 
\end{align*}
und
\begin{align*}
    \mathcal{M}^{-1}\left\{f(z)\right\}(x) 
    &= \mathcal{F}^{-1}\left\{f (jz)\right\}(e^t) \\ \\
    \mathcal{M}^{-1}\left\{f(z)\right\}(x) 
    &= \mathcal{L}^{-1}\left\{f (-z)\right\}(e^{-t}) 
    + \mathcal{L}^{-1}\left\{f (z)\right\}(e^{t}) 
\end{align*}
Aus diesen Relationen können also Hin- und Rücktransformationsformeln 
der Mellin-Transformation einfachst aus den Formeln der Fourier- oder 
Laplacetransformation hergeleitet werden, ganz ohne Kenntnis der 
zugrundeliegenden Gruppen- beziehungweise Gelfandtheorie.
Auch überrascht nun gar nicht mehr, dass eigentlich alle Eigenschaften 
der Fouriertransformation übersetzt werden können, was vorher im Kontext 
der abstrakten Gelfandtheorie vielleicht noch etwas undurchsichtig 
erschienen ist.

In der Theorie könnte die Mellin-Transformation also alle Aufgaben der 
in der modernen Welt allgegenwärtigen Fouriertransformation übernehmen. 
Joseph Fourier hätte die Wärmeleitgleichung damals ebenso mit der 
Mellin-Transformation lösen können und ein Grossteil der modernen 
Elektrotechnik könnte darauf basieren. 
Aus diesem Vergleich offenbart sich aber auch die Schönheit der 
Fouriertransformation. 
Durch die Symmetrie der Gruppen und dadurch der Transformationsgleichungen 
kommt diese extrem handlich und intuitiv daher und ist der 
Mellin-Transformation in dieser Hinsicht wohl doch überlegen.
Die Vorstellung der Fouriertransformation, als das Aus- und 
Einrollen eines Seiles in einer Seiltrommel ist in diesem Kontext 
einzigartig.
% to be elaborated





\printbibliography[heading=subbibliography]
\end{refsection}
