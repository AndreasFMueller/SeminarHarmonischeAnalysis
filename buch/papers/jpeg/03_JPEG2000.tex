%
% 03_JPEG2000.tex -- Beispiel-File für teil2 
%
% (c) 2020 Prof Dr Andreas Müller, Hochschule Rapperswil
%
% !TEX root = ../../buch.tex
% !TEX encoding = UTF-8
%
\section{JPEG-2000 
\label{jpeg:section:jpeg2000}}
\rhead{JPEG-2000}
JPEG-2000 ist die Weiterentwicklung und Qualitative Verbesserung des JPEG-Standards.
Der JPEG-Standard konnte sich schnell verbreiten, weil er lizenzfrei zur Verfügung steht und eine entsprechend effiziente Software bereitgestellt wurde.
Wie Abschnitt \ref{jpeg:subsection:probleme} andeutet hat die JPEG Kompression seine Grenzen.
JPEG konnte schon verlustlos und verlustbehaftet Bilder komprimieren, nutzt dafür aber verschiedene Algorithmen.
Im Abschnitt \ref{jpeg:section:kompjpeg} ist nur der verlustbehaftete beschrieben.

Bei der Weiterentwicklung wurden folgende Anforderungen definiert:
JPEG-2000 sollt bei einer höheren Kompressionsrate eine bessere Bildqualität haben als JPEG.
Der verlustfreie und verlustbehaftete Modus sollten den gleichen Algorithmus nutzen und man wollte eine progressive Übertragung.
Das bedeutet, Bilder werden mit zunehmender Qualität aufgebaut, man sieht zuerst das Grundgerüst und mit der Zeit werden weitere Details hinzugefügt.
Ausserdem sollte es möglich sein, um eine schnellere Übertragung zu erreichen, nur einen Ausschnitt aus dem Bild fein aufzulösen.
Zusätzliche neue Funktionen:
\begin{itemize}
    \item Region of interest(ROI): ermöglicht es Anwendern Bildbereiche verschiedenen stark zu komprimieren.
    \item Der Aufbau des Datenstroms ermöglicht das leichtere Erkennen von Übertragungsfehlern.
    \item Bilder dekodieren, ohne das gesamte Bild zwischenspeichern zu müssen (sequentieller Bildaufbau).
\end{itemize}
Der JPEG-2000-Standard ist in 13 Unterstandards aufgegliedert, in diverse Ergänzungen und Erweiterungen für unterschiedliche Anwendungsgebiete. 


\subsection{Anwendungsgebiete
\label{jpeg:subsection:anwendungsgebiete}}
JPEG-2000 konnte sich noch nicht richtig durchsetzten, da es sehr rechenintensiv ist und einige Algorithmen geschützt sind.
Zudem wird es nur in wenigen kommerziellen Programmen unterstützt.
Dafür findet JPEG-2000 Anwendung im professionellen Bereich:
\begin{itemize}
    \item In der Medizinaltechnik, Speicherung von Röntgenbildern, CT- und MRI-Scans.
    Festgelegt durch den Digital Imaging and Communications in Medicine (DICOM) Standard.
    Dort kann man z.B. nur den Teil des Bildes übertragen, der von Interesse ist um Übertragungszeiten zu verringern.
    \item Die Digital Cinema Initiative (DCI) nutzt das Motion-JPEG, ein Teil von JPEG-2000, um Kinofilme digital zu Speichern und den Kinos zur Verfügung zu stellen.
    \item Speicherung von Passbildern auf den deutschen Reisepässen.
    \item Das schweizer Bundesarchiv legt Bilder in TIFF und JPEG-2000 ab.
    \item Live HDTV wird mit JPEG-2000 komprimierten Bildern übertragen.
    \item Die Bilder der Marsrover werden in einem JPEG-2000 verwandten Format übermittelt.
    \item Satelliten-Technik für zivile und militärische Zwecke.   
\end{itemize}


\subsection{Prinzip
\label{jpeg:subsection:prinzip}}
Die JPEG-2000 Komprimierung funktioniert ähnlich wie der ältere Standard JPEG.
Bilder werden zunächst einer Vorverarbeitung unterzogen, das ist das Aufteilen in Bildbereiche von höherem Interesse (ROIs).
Die Farbraum Umrechnung ist nicht zwingend, da JPEG-2000 auf verschiedene Farbmodelle (RGB, CMYK, YCbCr und weitere) anwendbar ist.
Die vorbereiteten Bilder werden nun mit der 2D-Wavelet-Transformation umgewandelt.
JPEG-2000 verwendet die Cohen-Daubechies-Feauveau-Wavelets (CDF), 5/3 bei verlustfreier und 9/7 bei verlustbehafteter Kompression.
Darauf folgt die Entropiekodierung, welche die komprimierten Daten liefert. 

\subsection{Verbesserungen gegenüber JPEG
\label{jpeg:subsection:verbesserungen}}
JPEG-2000 bietet einige Vorteile gegenüber dem älteren JPEG-Standard.
Die Farbtiefe wurde erhöht, es stehen nun bis zu 38 Bits pro Farbkanal zur Verfügung für z.B. HDR Bilder.
Es können grössere Bilder komprimiert werden als bei JPEG.
Es wurde zusätzlich Raum für die Metadaten geschaffen und einen Alphakanal für die Transparenz.
 
Um die Blockartefakte zu vermeiden wird das Bild in gössere Teilbilder unterteilt und nicht in kleine \(8\times8\) Pixelblöcke.
Da die DCT nicht zurechtkommt mit grösseren Blöcken als \(8\times8\), wurde auf die Wavelet-Transformation (WT) gewechselt.
Die WT ist in Zweierpotenzen leicht zu skalieren und lässt sich somit auf beliebig grosse Zweierpotenz-Blöcke anwenden.

\subsubsection{Nachteile von JPEG-2000
\label{jpeg:subsubsection:nachteil}}
Wie in Abschnitt \ref{jpeg:subsection:anwendungsgebiete} schon beschrieben, ist auch der JPEG-2000-Standard nicht perfekt.
JPEG-2000 Bilder mit hoher Kompression, tendieren zu Unschärfe und Schatten bei starken Kontrasten. 
JPEG-2000-Standard ist sehr rechenintensiv und hat einen hohen Arbeitsspeicherbedarf. 
Die Wavelettransformation lässt sich bisher noch nicht in Hardware umsetzen, kann also nicht auf einem Microchip implementiert werden.
Der neue Standard hat keine so weite Verbreitung wie JPEG und ist mit dem älteren Standard inkompatibel.
Das JPEG-2000 Komitee garantiert keine mögliche Lizenzansprüche Dritter.
