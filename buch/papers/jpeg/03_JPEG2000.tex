%
% 03_JPEG2000.tex -- Beispiel-File für teil2 
%
% (c) 2020 Prof Dr Andreas Müller, Hochschule Rapperswil
%
% !TEX root = ../../buch.tex
% !TEX encoding = UTF-8
%
\section{JPEG-2000 
\label{jpeg:section:jpeg2000}}
\rhead{JPEG-2000}
JPEG-2000 ist die Weiterentwicklung des JPEG-Standards.
Der JPEG-Standard konnte sich schnell verbreiten, weil er lizenzfrei zur Verfügung steht und eine entsprechend effiziente Software bereitgestellt wurde.
Wie Abschnitt \ref{jpeg:subsection:probleme} angedeutet hatte die JPEG Kompression eine Grenze.
Diese sollte mit der Weiterentwicklung behoben werden. 
JPEG kann Bilder verlustlos und verlustbehaftet Komprimieren, nutzt dafür aber verschiedene Algorithmen im Abschnitt \ref{jpeg:section:kompjpeg} ist nur der verlustbehaftete beschrieben.

JPEG-2000 bietet einige Vorteile gegenüber dem älteren JPEG-Standard.
Um ein paar zu nennen, es stehen bis zu 38 Bits pro Farbkanal zur Verfügung (HDR Bilder komprimierbar), Bilder die grösser als 64000x64000 Pixel komprimierbar, höhere Fehlerresilienz, auf verschiedenen Farbmodelle anwendbar (RGB, CMYK, YCbCr und weitere), Raum für Metadaten und zusätzlicher Alphakanal für die Transparenz.


Dies führte zur Anforderungsliste an JPEG-2000:
\begin{itemize}
    \item JPEG-2000 soll bei höherer Kompressionsrate eine bessere Bildqualität haben als JPEG.
    \item Verlustfreie und verlustbehaftete Kompression soll mit demselben Algorithmus funktionieren.
    \item Progressive Übertragung ermöglicht, dass Bilder mit zunehmender Qualität geladen werden.
\end{itemize}
Zusätzlich erhielt JPEG-2000 neue Funktionen:
\begin{itemize}
    \item Region of interest(ROI): ermöglicht es Anwender Bildbereiche mit verschiedenen stark zu komprimieren.
    \item Der Aufbau des Datenstroms ermöglicht das leichtere erkennen von Übertragungsfehler.
    \item Bilder dekodieren, ohne das gesamte Bild zwischenspeichern zu müssen (Sequentieller Bildaufbau).
\end{itemize}
Der JPEG-2000-Standard ist in 13 Unterstandards aufgegliedert, in diverse Ergänzungen und Erweiterungen für unterschiedliche Anwendungsgebiete. 

\subsection{Anwendungsgebiet
\label{jpeg:subsection:anwendungsgebiet}}
JPEG-2000 konnte sich noch nicht 