%
% teil3.tex -- Beispiel-File für Teil 3
%
% (c) 2020 Prof Dr Andreas Müller, Hochschule Rapperswil
%
% !TEX root = ../../buch.tex
% !TEX encoding = UTF-8
%
\section{JPEG Kompression
\label{jpeg:section:kompjpeg}}
\rhead{JPEG Kompression}
Um Bilder mit der DCT transformieren zu können, ist eine Vorverarbeitung nötig. 
in der Vorverarbeitung wird die Farbraumumrechnung um eine bessere Kompression zu erreichen.
Zudem wird das ganze Bild in \(8x8\) Pixelblöcke unterteilt, da die zweidimensionale DCT damit arbeitet.

\subsection{Farbraumumrechnung
\label{jpeg:subsection:farbraumumrechnung}}
Ein weit verbreiteter Farbmodell ist das Rot-Grün-Blau-Modell (RGB), dabei wird eine Pixelfarbe mit einem Rot-, Grün- und Blauwert additive erzeugt.

Beim JPEG-Standart wechselt man die Basis, in der die Farben dargestellt werden und benutzt statt RGB das YCbCr-Modell.
In diesem Modell werden die Helligkeit (Luminaz Y), sowie die Farbigkeit (Chominanz).
\(C_b\) von Grau in Richtung Blau/Gelb und \(C_r\) von Grau nach Rot/Türkis.

Die Transformation wird gemacht, weil die Augen von Menschen Empfindlicher sind auf Helligkeitsunterschiede, als auf Farbunterschiede.
Damit lassen sich, nach einer DCT, die Koeffizienten aus den Cb und Cr teilen stärker Quantisieren als die der Y teile \ref{}.
Der beschriebene Frabraum wechsel lässt sich
\begin{equation}
    \begin{pmatrix}
        Y\\
        C_b\\
        C_r\\
    \end{pmatrix}
    \thickapprox
    \begin{pmatrix}
        0\\
        128\\
        128\\
    \end{pmatrix}
    +
    \begin{pmatrix}
        0.299 & 0.587 & 0.114\\
        -0.168736 & -0.331264 & 0.5\\
        0.5 & -0.418688 & -0.081312\\
    \end{pmatrix}
    \cdot
    \begin{pmatrix}
        R\\
        G\\
        B\\
    \end{pmatrix}
    \label{jpeg:equationfrab}
\end{equation}
berechnen.
Durch begrenzte Rechengenauigkeit und Rundungsfehler entstehen hier Datenverluste.

\subsection{Tiefpassfilter und Unterabtastung
\label{jpeg:subsection:tiefpass}}
In \ref{jpeg:subsection:farbraumumrechnung} wurde beschrieben, dass die Farbauflösung der \(C_b\) und \(C_r\) für Menschen deutlich geringer ist wie in \ref{} ersichtlich.
Dazu werden sie Tiefpass gefiltert, zudem üblicherweise vertikal und horizontal um den Faktor 2 Unterabgetastet, was einer 4 fachen Datenreduktion entspricht.

\subsection{Tiling
\label{jpeg:subsection:tiling}}
Beim Tiling wird das Bild in jeweils \(8x8\) Pixel grosse Blöcke unterteilt.
Da normalerweise die Seitenlängen eines Bildes sich nicht durch 8 teilen lässt, werden die Restlichen Zeilen bzw. Spalten jeweils aufgefüllt.
Wie das auffüllen geschieht ist im JPEG-Standard nicht festgelegt, es wird jedoch die letzte Pixel-Zeile oder -Spalte jeweils sooft zu wiederholen bis es aufgeht.
Wie in \ref{label} ersichtlich.
Auf den einzelnen Blöcken wird nun die DCT angewendet.


