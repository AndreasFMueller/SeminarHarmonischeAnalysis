%
% 00_Einleitung.tex
%
% (c) 2020 Prof Dr Andreas Müller, Hochschule Rapperswil
%
% !TEX root = ../../buch.tex
% !TEX encoding = UTF-8
%
\section{Einleitung\label{jpeg:section:einleitung}}
\kopfrechts{Einleitung}
\begin{figure}
    \centering
    \includegraphics[width=\linewidth]{papers/jpeg/pictures/kompressionsschema.pdf}
    \caption{Vorgehen bei der Kompression und Dekompression von JPEG
        \label{jpeg:fig:kompressionsschema}}
\end{figure}
Daten können besonders effizient übertragen oder gespeichert werden, wenn sie zuerst komprimiert werden.
Das kann in vollständig rekonstruierbarer Form oder mit Verlusten geschehen.

Verlustloses Komprimieren gelingt, indem man einen Kodieralgorithmus auf die Daten anwendet und die redundanten Informationen wie in einem Wörterbuch speichert und jeweils darauf referenziert. Beim verlustbehafteten Verfahren werden z.~B.~bei Bildern die Daten, die für Menschen nicht sichtbar sind, entfernt um den Inhalt zu reduzieren.

Es gibt inzwischen einige Methoden um Bilder zu komprimieren.
In diesem Kapitel soll es um die beiden Algorithmen von der Joint Photographic Experts Group (JPEG) gehen.
\index{Joint Photographic Experts Group}%
\index{JPEG}%


Bilder werden zuerst mit einer Vorverarbeitung auf die Transformation vorbereitet.
Das sind Farb\-raum\-trans\-formationen und Einteilung in kleinere Bereiche, dieser Schritt heisst Tiling.
An\-schlies\-send wird die Basistransformation durchgeführt und die Koeffizienten mit einer Tabelle quantisiert und ganzzahlig gerundet.
Die aus der Quantisierung entstandenen Werte werden entropiekodiert, meistens wird das Prinzip von Huffman verwendet.
Die verwendete Methode wird im File abgelegt.
Zudem werden die Tabellen der unterschiedlichen Quantisierungen und Kodierungen mit abgespeichert (s. Abb. \ref{jpeg:fig:kompressionsschema}).
Für die Dekompression werden diese Schritte rückwärts angewandt, wobei die verloren gegangenen Informationen bei der Quantisierung nicht wieder hergestellt werden können. 

Grundlegend werden die Bilder mit Hilfe verschiedener Methoden in Basisfunktionen aufgeteilt, zu denen jeweils die Koeffizienten berechnet werden.
Damit ist es möglich, die Koeffizienten in einem späteren Schritt zu manipulieren und Daten einzusparen. 
Bei JPEG ist das die discrete cosine transform (DCT) und bei JPEG-2000 die discrete wavelet transform (DWT).
\index{JPEG-2000}%
\index{discrete wavelet transform}%
\index{dwt}%

Das Kapitel basiert auf dem Inhalt des Papers \textit{Bilddatenkompression mit JPEG und JPEG2000} \cite{jpeg:laurahochstrat}.
