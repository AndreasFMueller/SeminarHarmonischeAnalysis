%
% truncation.tex -- triangular truncation
%
% (c) 2021 Prof Dr Andreas Müller, OST Ostschweizer Fachhochschule
%
\documentclass[tikz]{standalone}
\usepackage{times}
\usepackage{amsmath}
\usepackage{txfonts}
\usepackage{pgfplots}
\usepackage{csvsimple}

\usetikzlibrary{arrows,intersections,math}

\begin{document}
\def\skala{1}
\begin{tikzpicture}[>=latex,thick,scale=\skala]

\def\h{0.8}

\fill[color=gray!40,rounded corners=0.4cm] (0,{-0.707*\h})
	-- ({-3.5*\h-0.707*\h},{3.5*\h})
	-- ({3.5*\h+0.707*\h},{3.5*\h})
	-- cycle;

\draw[->] ({-4.5*\h},0) -- ({4.5*\h},0) coordinate[label={\large $n$}];
\draw[->] (0,{-0.5*\h}) -- (0,{5.5*\h}) coordinate[label={right:\large $m$}];

\draw[color=red,dashed] ({-4.5*\h},{3*\h}) -- ({4.5*\h},{3*\h});
\node[color=red] at ({-4.5*\h},{3*\h}) [left] {\large $N=3$};

\foreach \n in {0,...,4}{
	\foreach \m in {-\n,...,\n}{
		\fill ({\m*\h},{\n*\h}) circle[radius=0.06];
	}
}

\end{tikzpicture}
\end{document}

