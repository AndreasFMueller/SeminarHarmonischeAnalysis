%
% Spektrale Modelle.tex -- Beispiel-File für teil2 
%
% (c) 2023 Dmitry Grigoriev, OST Ostschweizer Fachhochschule
%
% !TEX root = ../../buch.tex
% !TEX encoding = UTF-8
%
\section{Spektrale Modelle 
\label{spektral:section:spektralemodelle}}
\rhead{Spektrale Modelle}

Wir werden die Idee der Spektralemethode auf einem Beispiel anschauen. Dafür nehmen wir die Wellengleichung.

\begin{equation}
 \dfrac{\partial^2U}{\partial{t^2}} = \alpha^2\Delta{U}.
\label{spektral:equation7}
\end{equation}
Durch eine sehr aufwändige Rechnung, kann die Wellengleichung in den Kugelkoordinaten folgendermaßen geschrieben werden (siehe \cite[Anhang B]{spektral:QuantenmechanikMathemathischesSeminar})

\begin{equation}
 \frac{1}{r^2}\frac{\partial}{\partial{r}}\left(r^2\frac{\partial{U}}{\partial{r}}\right) + \frac{1}{r^2}\left(\frac{1}{\sin(\phi)}\frac{\partial}{\partial{\theta}}\left(\sin(\theta)\frac{\partial{U}}{\partial{\theta}}\right) + \frac{1}{\sin^2(\theta)}\frac{\partial^2{U}}{\partial{\phi^2}}\right) = \frac{1}{\alpha^2}\frac{\partial^2{U}}{\partial{t^2}}.
\label{spektral:equation8}
\end{equation}
Diese Gleichung ist sehr schwer zu lösen und bei der Lösung werden uns die spektrale Modelle helfen.

\subsection{Kugelflächenfunktionen
\label{spektral:subsection:kugelflaechenfunktionen}}

Damit wir das spektrale Modell für die Gleichung \eqref{spektral:equation7} entwicklen können, werden wir ein vollständigen und orthogonaler Satz von Basisfunktionen benötigt.
Dafür muss die Eigenwertgleichung des Laplace-Operators \refeq{spektral:equation8}

\begin{equation}
 \Delta{V} = -\lambda{V(x)}.
\label{spektral:equation9}
\end{equation}
Die vollständige Lösung ist in \cite[Seite 62]{spektral:NichtkommutativeBildvearbeitung} gezeigt.
Für uns ist wichtig, dass wir die Basisfunktionen finden und durch diese Funktionen - Kugelflächenfunktionen (siehe \cite{spektral:MathsemSpezfunk} und \cite{spektral:NichtkommutativeBildvearbeitung}) - unsere gesuchte Funktion aus der Gleichung \eqref{spektral:equation7} darstellen können

\begin{equation}
u(\phi,\theta,t) = \sum_{m=-\infty}^{\infty}\sum_{n=-|m|}^{\infty}a_n^m(t)Y_n^m(\phi, \theta).
\label{spektral:equation10}
\end{equation}

\subsection{Wellengleichung
\label{spektral:subsection:wellengleichung}}
Jetzt sind wir bereit die Wellengleichung mit der spektrale Methode zu lösen.

Für die Vereinfachung nehmen wir an, dass in der Gleichung \eqref{spektral:equation7} $\alpha=1$ ist und der Kugelradius auch $r=1$ ist.
Dann wird der Laplace-Operator für die Kugelflächenfunktion so aussehen:
\begin{equation}
 \frac{\partial}{\partial{r}}\left(r^2\frac{\partial{Y_n^m}}{\partial{r}}\right) + \frac{1}{\sin(\phi)}\frac{\partial}{\partial{\theta}}\left(\sin(\theta)\frac{\partial{Y_n^m}}{\partial{\theta}}\right) + \frac{1}{\sin^2(\theta)}\frac{\partial^2{Y_n^m}}{\partial{\phi^2}} = 0.
\label{spektral:equation11}
\end{equation}
Wenn wir die Gleichung \eqref{spektral:equation10} für die Gelichung \eqref{spektral:equation7} anwenden, dann bekommen wir
\begin{equation}
\Delta{U} = \sum_{m=-\infty}^{\infty}\sum_{n=-|m|}^{\infty}a_n^m(t)\Delta{Y_n^m(\phi, \theta)}.
\label{spektral:equation12}
\end{equation}
Wir müssen jetzt die $\Delta{Y_n^m(\phi, \theta)}$ finden.
Das können wir aus den Laplace-Operator in sphärischen Koordinaten ausrechnen.
Im \cite[Seite 656]{spektral:DynamicOfTheAtmosphere} wurde nachgewiesen, dass 
\begin{equation}
\frac{\partial}{\partial{r}}\left(r^2\frac{\partial{Y_n^m}}{\partial{r}}\right) = n(n+1)Y_n^m(\phi, \theta)
\label{spektral:equation13}
\end{equation}
ist.
Jetzt wenden wir die Gleichung \eqref{spektral:equation13} für die linke Seite der Gleichung \eqref{spektral:equation11} an und bekommen
\begin{equation}
n(n+1)Y_n^m(\phi, \theta) + \frac{1}{\sin(\phi)}\frac{\partial}{\partial{\theta}}\left(\sin(\theta)\frac{\partial{Y_n^m}}{\partial{\theta}}\right) + \frac{1}{\sin^2(\theta)}\frac{\partial^2{Y_n^m}}{\partial{\phi^2}} = 0
\label{spektral:equation14}
\end{equation}
und
\begin{equation}
\frac{1}{\sin(\phi)}\frac{\partial}{\partial{\theta}}\left(\sin(\theta)\frac{\partial{Y_n^m}}{\partial{\theta}}\right) + \frac{1}{\sin^2(\theta)}\frac{\partial^2{Y_n^m}}{\partial{\phi^2}} = -n(n+1)Y_n^m(\phi, \theta).
\label{spektral:equation15}
\end{equation}
Die linke Seite der Gleichung \eqref{spektral:equation15} ist ein horizontal Laplace-Operator in den sphärischen Koordinaten.
Somit bekommen wir
\begin{equation}
\Delta{Y_n^m(\phi, \theta)} = -n(n+1)Y_n^m(\phi, \theta).
\label{spektral:equation16}
\end{equation}
Laplace-Operator der Kugelfklächenfunktion ist proportional der Funktion selbst. Das ist sehr wichtige Eigenschaft, die wir gleich anwenden werden.
Aus der Gleichung \eqref{spektral:equation12} bekommen wir:
\begin{equation}
\Delta{U} = -\sum_{m=-\infty}^{\infty}\sum_{n=-|m|}^{\infty}a_n^m(t)n(n+1)Y_n^m(\phi, \theta)
\label{spektral:equation17}
\end{equation}
und wenn wir die Gleichung \eqref{spektral:equation10} auf die linke Seite der Gleichung \eqref{spektral:equation7} anwenden, bekommen wir
\begin{equation}
\sum_{m=-\infty}^{\infty}\sum_{n=-|m|}^{\infty}\ddot{a}_n^m(t)Y_n^m(\phi, \theta) = -\sum_{m=-\infty}^{\infty}\sum_{n=-|m|}^{\infty}a_n^m(t)n(n+1)Y_n^m(\phi, \theta).
\label{spektral:equation18}
\end{equation}
Wir sehen, dass auf den beiden Seiten der Gleichung die Kugelflächenfunktion $Y_n^m(\phi, \theta)$ vorkommt und die Gleichung \eqref{spektral:equation18} kann umgeschrieben werden
\begin{equation}
\ddot{a}_n^m(t) = -n(n+1)a_n^m(t).
\label{spektral:equation19}
\end{equation}
D.h. wir bekommen das System der Differenzialgleichungen zweiten Grades.
Die Lösung dieser Differenzialgleichung ist
\begin{equation}
a_n^m(t) = a_n^m(0)e^{i\sqrt{n(n+1)}t}
\label{spektral:equation20}
\end{equation}
und die gesuchte Funktion
\begin{equation}
u(\phi, \theta, t) = \sum_{m=-\infty}^{\infty}\sum_{n=-|m|}^{\infty}a_n^m(0)e^{i\sqrt{n(n+1)}t}Y_n^m(\phi, \theta).
\label{spektral:equation21}
\end{equation}

\subsection{Ergebnis
\label{spektral:subsection:ergebnis}}

Als Ergebnis bekommen wir die Gleichung, die im Vergleich zu Differenzialgleichung \eqref{spektral:equation8}, viel einfache zu lösen ist. Das Vorteil wurde bereits im Kapitel \ref{spektral:subsection:vorteile} erwähnt. Jetzt haben wir das bestätigt.