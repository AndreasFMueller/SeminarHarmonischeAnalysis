%
% Spektrale Modelle.tex -- Beispiel-File für teil2 
%
% (c) 2023 Dmitri Grigoriev, OST Ostschweizer Fachhochschule
%
% !TEX root = ../../buch.tex
% !TEX encoding = UTF-8
%
\section{Spektrale Modelle 
\label{spektral:section:spektralemodelle}}
\rhead{Spektrale Modelle}

Wir werden die Idee der Spektralemethode auf einem Beispiel anschauen. Dafür nehmen wir die Wellengleichung.

\begin{equation}
 \dfrac{\partial^2U}{\partial{t^2}} = \alpha^2\Delta U
\label{spektral:equation7}
\end{equation}

Durch eine sehr aufwändige Rechnung, kann die Wellengleichung in den Kugelkoordinaten folgenderßen geschrieben werden: (siehe [5] Anhang B)

\begin{equation}
 \frac{1}{r^2}\frac{\partial}{\partial{r}}\left[r^2\frac{\partial{U}}{\partial{r}}\right] + \frac{1}{r^2}\left[\frac{1}{\sin(\phi)}\frac{\partial}{\partial{\theta}}\left[\sin(\theta)\frac{\partial{U}}{\partial{\theta}}\right] + \frac{1}{\sin^2(\theta)}\frac{\partial^2{U}}{\partial{\phi^2}}\right] = \frac{1}{\alpha^2}\frac{\partial^2{U}}{\partial{t^2}}
\label{spektral:equation8}
\end{equation} 