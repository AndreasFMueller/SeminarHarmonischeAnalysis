%
% Spektrale Modelle.tex -- Beispiel-File für teil2 
%
% (c) 2023 Dmitry Grigoriev, OST Ostschweizer Fachhochschule
%
% !TEX root = ../../buch.tex
% !TEX encoding = UTF-8
%
\section{Spektrale Modelle 
\label{spektral:section:spektralemodelle}}
\rhead{Spektrale Modelle}

Wir werden die Idee der Spektralmethode auf einem Beispiel anschauen. Dafür nehmen wir die Wellengleichung

\begin{equation}
 \dfrac{\partial^2U}{\partial{t^2}} = \alpha^2\Delta{U}.
\label{spektral:equation7}
\end{equation}
Durch eine sehr aufwändige Rechnung, kann die Wellengleichung in den Kugelkoordinaten folgendermaßen geschrieben werden (siehe \cite[Anhang B]{spektral:QuantenmechanikMathemathischesSeminar}):

\begin{equation}
 \frac{1}{r^2}\frac{\partial}{\partial{r}}\left(r^2\frac{\partial{U}}{\partial{r}}\right) + \frac{1}{r^2}\left(\frac{1}{\sin\theta}\frac{\partial}{\partial{\theta}}\left(\sin\theta\frac{\partial{U}}{\partial{\theta}}\right) + \frac{1}{\sin^2\theta}\frac{\partial^2{U}}{\partial{\phi^2}}\right) = \frac{1}{\alpha^2}\frac{\partial^2{U}}{\partial{t^2}}.
\label{spektral:equation8}
\end{equation}
Diese Gleichung ist sehr schwer zu lösen und bei der Lösung werden uns die spektrale Modelle helfen.

\subsection{Kugelflächenfunktionen
\label{spektral:subsection:kugelflaechenfunktionen}}

Damit wir das spektrale Modell für die Gleichung \eqref{spektral:equation7} entwicklen können, werden wir ein vollständigen und orthogonaler Satz von Basisfunktionen benötigt.
Dafür muss die Eigenwertgleichung des Laplace-Operators \eqref{spektral:equation8}

\begin{equation}
 \Delta{V} = -\lambda{V(x)}
\label{spektral:equation9}
\end{equation}
gelöst werden.
Die vollständige Lösung ist in \cite[Seite 62]{spektral:NichtkommutativeBildvearbeitung} gezeigt.
Für uns ist wichtig, dass wir die Basisfunktionen finden und durch diese Funktionen --- Kugelflächenfunktionen (siehe \cite{spektral:MathsemSpezfunk} und \cite{spektral:NichtkommutativeBildvearbeitung}) --- unsere gesuchte Funktion aus der Gleichung \eqref{spektral:equation7} als

\begin{equation}
u(\phi,\theta,t) = \sum_{m=0}^{\infty}\sum_{n=-m}^{m}a_n^m(t)Y_n^m(\phi, \theta)
\label{spektral:equation10}
\end{equation}
darstellen können.

\subsection{Wellengleichung
\label{spektral:subsection:wellengleichung}}
Jetzt sind wir bereit die Wellengleichung mit der spektrale Methode zu lösen.
Für die Vereinfachung nehmen wir an, dass in der Gleichung \eqref{spektral:equation7} $\alpha=1$ ist und der Kugelradius auch $r=1$ ist.
Dann wird die Wellengleichung so aussehen:
\begin{equation}
\frac{1}{\sin\theta}\frac{\partial}{\partial{\theta}}\left(\sin\theta\frac{\partial{U}}{\partial{\theta}}\right) + \frac{1}{\sin^2\theta}\frac{\partial^2{U}}{\partial{\phi^2}} = \frac{\partial^2{U}}{\partial{t^2}}.
\label{spektral:equation11}
\end{equation}
Wenn wir die Gleichung \eqref{spektral:equation10} für die Gleichung \eqref{spektral:equation7} anwenden, dann bekommen wir
\begin{equation}
\Delta{U} = \sum_{m=0}^{\infty}\sum_{n=-m}^{m}a_n^m(t)\Delta{Y_n^m(\phi, \theta)}.
\label{spektral:equation12}
\end{equation}
Wir müssen jetzt die $\Delta{Y_n^m(\phi, \theta)}$ finden.
Das können wir aus den Laplace-Operator in sphärischen Koordinaten ausrechnen. Wir wissen, dass der Laplace-Operator selbstadjungiert ist, seine Eigenfunktionen orthogonal sind und der Eigenwert $-l(l+1)$ ist (siehe Satz~\ref{buch:orthofkt:pde:satz:kugel}).
Diese Information erlaubt uns die harmonisches Analysis durchzuführen.
Gemäß dem Satz~\ref{buch:orthofkt:pde:satz:kugel} bekommen wir
\begin{equation}
\Delta{Y_n^m(\phi, \theta)} = -l(l+1)Y_n^m(\phi, \theta).
\label{spektral:equation16}
\end{equation}
Laplace-Operator der Kugelflächenfunktion ist proportional der Funktion selbst \cite[Seite 663]{spektral:DynamicOfTheAtmosphere}. Das ist sehr wichtige Eigenschaft, die wir gleich anwenden werden.
Aus der Gleichung \eqref{spektral:equation12} bekommen wir:
\begin{equation}
\Delta{U} = -\sum_{m=0}^{\infty}\sum_{n=-m}^{m}a_n^m(t)l(l+1)Y_n^m(\phi, \theta)
\label{spektral:equation17}
\end{equation}
und wenn wir die Gleichung \eqref{spektral:equation10} auf die linke Seite der Gleichung \eqref{spektral:equation7} anwenden, bekommen wir
\begin{equation}
\sum_{m=0}^{\infty}\sum_{n=-m}^{m}\ddot{a}_n^m(t)Y_n^m(\phi, \theta) = -\sum_{m=0}^{\infty}\sum_{n=-m}^{m}a_n^m(t)l(l+1)Y_n^m(\phi, \theta).
\label{spektral:equation18}
\end{equation}
Wir stellen fest, dass auf den beiden Seiten der Gleichung die Kugelflächenfunktion $Y_n^m(\phi, \theta)$ vorkommt, und die Gleichung \eqref{spektral:equation18} kann in 
\begin{equation}
\ddot{a}_n^m(t) = -l(l+1)a_n^m(t)
\label{spektral:equation19}
\end{equation}
umgeschrieben werden.
Dadurch bekommen wir ein System der Differenzialgleichungen zweiter Ordnung, das folgende Lösungen hat:
\begin{equation}
a_n^m(t) = a_n^m(0)e^{i\sqrt{l(l+1)}t}.
\label{spektral:equation20}
\end{equation}
Somit die gesuchte Funktion ist
\begin{equation}
u(\phi, \theta, t) = \sum_{m=0}^{\infty}\sum_{n=-m}^{m}a_n^m(0)e^{i\sqrt{l(l+1)}t}Y_n^m(\phi, \theta).
\label{spektral:equation21}
\end{equation}

\subsection{Ergebnis
\label{spektral:subsection:ergebnis}}

Als Ergebnis bekommen wir die Gleichung, die im Vergleich zu Differenzialgleichung \eqref{spektral:equation8} wesentlich einfacher zu lösen ist.
Dieser Vorteil wurde bereits im Abschnitt \ref{spektral:subsection:vorteile} erwähnt und wurde nun bestätigt.