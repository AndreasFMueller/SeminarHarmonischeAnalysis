%
% einleitung.tex -- Beispiel-File für die Einleitung
%
% (c) 2023 Dmitry Grigoriev, OST Ostschweizer Fachhochschule
%
% !TEX root = ../../buch.tex
% !TEX encoding = UTF-8
%
\section{Einleitung\label{spektral:section:Einleitung}}
\rhead{Einleitung}

Spektrale Methoden spielen eine wichtige Rolle in der Meteorologie, insbesondere in der Analyse von atmosphärischen Phänomenen und Daten.
Diese Methoden basieren auf der Fourier-Trans\-for\-ma\-tion, die es ermöglicht, Signale in den Frequenzbereich zu transformieren und somit ihre spektrale Zusammensetzung zu analysieren.
In der Meteorologie werden spektrale Methoden auf verschiedene Arten angewendet:

\begin{itemize}
\item
\textbf{Spektrale Analyse von Zeitreihen:} Meteorologische Daten, wie Temperatur, Druck, Windgeschwindigkeit usw., werden oft als Zeitreihen erfasst.
Die Fourier-Transformation ermöglicht die Aufschlüsselung dieser Zeitreihen in verschiedene Frequenzkomponenten.
Dies kann hilfreich sein, um periodische Muster oder Trends in den Daten zu identifizieren. 
Zum Beispiel können saisonale Schwankungen in den Temperaturdaten erkannt werden.
\item
\textbf{Wellenanalyse:} Spektrale Methoden werden verwendet, um Wellenphänomene in der Atmosphäre zu analysieren, wie zum Beispiel atmosphärische Schwingungen, Schallwellen, Gezeitenwellen und Rossby-Wellen.
Die Identifizierung von Wellenmustern und -frequenzen hilft bei der Vorhersage von Wetter- und Klimaereignissen.
\item
\textbf{Numerische Wettervorhersage (NWP):} In der numerischen Wettervorhersage werden spektrale Methoden verwendet, um die atmosphärischen Zustandsänderungen im Laufe der Zeit zu simulieren.
Dies geschieht durch die Diskretisierung der atmosphärischen Gleichungen im spektralen Raum und ihre Lösung mithilfe von numerischen Verfahren.
Diese Modelle helfen bei der Vorhersage von Wetterereignissen über verschiedene Zeitskalen.
\item
\textbf{Klimaforschung:} In der Klimaforschung werden spektrale Methoden verwendet, um langfristige klimatische Veränderungen zu analysieren.
\end{itemize}

Diese Anwendungen verdeutlichen, wie spektrale Methoden in der Meteorologie zur Analyse, Modellierung und Vorhersage von atmosphärischen Phänomenen und Prozessen eingesetzt werden können.

Wir werden uns auf die Wellenanalyse und die Numerische Wettervorhersage konzentrieren.