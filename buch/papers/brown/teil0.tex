%
% einleitung.tex -- Beispiel-File für die Einleitung
%
% (c) 2020 Prof Dr Andreas Müller, Hochschule Rapperswil
%
% !TEX root = ../../buch.tex
% !TEX encoding = UTF-8
%
\section{Brownische Bewegung\label{brown:section:teil0}}
\rhead{Brownische Bewegung}

Als der Schottische Botaniker Robert Brown im Jahr 1827 in sein Mikroskop schaut, beobachtet er kleine Partikel von Pflanzenpollen in einer Flüssigkeit. Er bemerkt, dass sich die Teilchen scheinbar zufällig bewegen, obwohl scheinbar keine Kräfte auf die Teilchen einwirken. Eine Erklärung hatte Robert Brown zu diesem Zeitpunkt für das Verhalten noch nicht und notierte dass sich die Teilchen scheinbar zufällig und unregelmässig bewegen.

Es dauerte fast ein Jahrhundert, bis Albert EInstein diese Beobachtung auf ein solides theoretisches Fundament stellte und nutzbar machte. Er schloss darauf, dass dir unregelmässige Bewegung auf Kolisionen mit umgebenden Mölekühlen zurückzuführen ist, welche ständig in Bewegung sind. In seiner THeorei beschreibt er...

Die 



Text
\cite{brown:bibtex}.


