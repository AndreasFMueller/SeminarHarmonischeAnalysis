%
% teil0.tex -- Einleitung
%
% (c) 2023 Lukas Reitemeier, OST Ostschweizer Fachhochschule
%
% !TEX root = ../../buch.tex
% !TEX encoding = UTF-8
%

Folgende Seiten beschäftigen sich mit gegensätzlichen Konzepten zur harmonischen Analysis. Durch das Beleuchten konträrer Eigenschaften und dem Verschmelzen der Themen, entsteht hoffentlich ein wertvoller Beitrag zum Buch. Dieses Kapitel handelt von der brownschen Bewegung und stochastischen Differenzialgleichungen (SDGL), zwei eng verbundene Themen, die interdisziplinär in vielen verschiedenen Gebieten zur Anwendung kommen.

Die Entdeckungsgeschichte widerspiegelt diese Interdisziplinarität. Der Botaniker Robert Brown entdeckte im 19. Jahrhundert, dass in Flüssigkeit suspendierte Pollenteilchen unter seinem Mikroskop scheinbar zufällige Bewegungen ausführten. Er gilt als Entdecker der nach ihm benannten brownschen Bewegung. Der französische Mathematiker Luis Bachelier, der für seine Arbeit \textit{Théorie de la Spéculation} in der Wahrscheinlichkeitstheorie und Finanzmathematik bekannt ist, hat versucht, die Brownsche Bewegung erstmals mathematisch zu modellieren, analog zu den Fluktuationen von Börsenkursen. Später trug der Physiker Albert Einstein mit seiner Arbeit \textit{Untersuchungen über die Theorie der Brownschen Bewegung} maßgeblich zum theoretischen Verständnis dieses Phänomens bei. Dies wiederum diente dem Mathematiker Norbert Wiener als Grundlage für die Entwicklung des Wiener-Prozesses, der ein zentrales Konzept in der Wahrscheinlichkeitstheorie und SDGLs darstellt. So kann der Wiener-Prozess auch dazu genutzt werden, um sogenanntes weisses Rauschen zu modellieren .

Dank all diesen Entdeckungen ist es heute möglich Finanzmathematik zu betreiben, Populationsmodelle zu erstellen und den Prozess der Diffusion mathematisch zu beschreiben. %Viele dieser Systeme können sensibel auf Störeinflüsse stochastischer Natur reagieren. Auf diesen Umstand und wie man damit umgehen kann, wird am Ende dieses Kapitels eingegangen.