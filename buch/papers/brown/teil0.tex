%
% teil0.tex -- Einleitung
%
% (c) 2023 Lukas Reitemeier, OST Ostschweizer Fachhochschule
%
% !TEX root = ../../buch.tex
% !TEX encoding = UTF-8
%

\chapter{Brownsche Bewegung und Stochastische Differenzialgleichungen\label{chapter:brown}}
\kopflinks{Brownische Bewegung und stochastische Differenzialgleichungen}
% Kurzform des Kapitel-Titels
\begin{refsection}
\chapterauthor{Lukas Reitemeier}

Folgende Seiten beschäftigen sich mit gegensätzlichen Konzepten zur harmonischen Analysis, wie zufälligen Verläufen und Störungen. Durch das Beleuchten konträrer Eigenschaften und dem Verschmelzen beider Themen, entsteht hoffentlich ein wertvoller Beitrag zum Buch. Dieses Kapitel handelt von der brownschen Bewegung und stochastischen Differenzialgleichungen, zwei eng verbundene Themen, welche interdisziplinär in vielen verschiedenen Gebieten zur Anwendung kommen.

Die Entdeckungsgeschichte widerspiegelt diese Interdisziplinarität. Der Botaniker Robert Brown entdeckte im 19. Jahrhundert, dass in Flüssigkeit suspendierte Pollenteilchen unter seinem Mikroskop scheinbar zufällige Bewegungen ausführten. Der französische Mathematiker Luis Bachelier, der für seine Arbeit \textit{Théorie de la Spéculation} in der Wahrscheinlichkeitstheorie und Finanzmathematik bekannt ist, hat die erste mathematische Modellierung einer Brownschen Bewegung versucht, analog zu Fluktuationen von Börsenkursen. Später trug der Physiker Albert Einstein mit seiner Arbeit \textit{Untersuchungen über die Theorie der Brownschen Bewegung} maßgeblich zum theoretischen Verständnis dieses Phänomens bei. Dies wiederum diente dem Mathematiker Norbert Wiener als Grundlage für die Entwicklung des Wiener-Prozesses, welcher ein zentrales Konzept in der Wahrscheinlichkeitstheorie darstellt. Bei stochastischen Differenzialgleichungen (SDGL) wird dieser Wiener-Prozess genutzt, um sogenanntes weisses Rauschen zu modeliern.

Dank all diesen Entdeckungen ist es heute möglich Finanzmathematik zu betreiben, realitätsfremdere Populationsmodelle zu erstellen und den Prozess der Diffusion mathematisch zu beschreiben. Viele dieser Systeme können sensibel auf Störeinflüsse stochastischer Natur reagieren. Auf diesen Umstand und wie man damit umgehen kann, wird am Ende dieses Kapitels eingegangen.