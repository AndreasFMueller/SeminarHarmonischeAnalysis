%
% teil0.tex -- Einleitung
%
% (c) 2023 Lukas Reitemeier, OST Ostschweizer Fachhochschule
%
% !TEX root = ../../buch.tex
% !TEX encoding = UTF-8
%

\chapter{Brownische Bewegung und stochastische Differenzialgleichungen\label{chapter:brown}}
\kopflinks{Brownische Bewegung und stochastische Differenzialgleichungen}
% Kurzform des Kapitel-Titels
\begin{refsection}
\chapterauthor{Lukas Reitemeier}

Es gibt natürlich kein Gegenteil zur Harmonischen Analysis, jedoch werden sich folgende Seiten mit gegensätzlichen Konzepten beschäftigen, konträre Eigenschaften beleuchten und hoffentlich einen Beitrag zum Verständnis des Buchthemas leisten.


Dieses Kapitel befasst sich mit der Brown'schen Bewegung und stochastischen Differenzialgleichungen - zwei eng verbundene Themen, welche interdisziplinär in vielen verschiedenen Gebieten zur Anwendung kommen.

Die Entdeckungsgeschichte widerspiegelt diese Interdisziplinarität. Der Botaniker Robert Brown entdeckte im 19. Jahrhundert, dass in Flüssigkeit suspendierte Pollenteilchen unter seinem Mikroskop scheinbar zufällige Bewegungen ausführten. Der französische Mathematiker Luis Bachelier, der für seine Arbeit \textit{Théorie de la Spéculation} in der Wahrscheinlichkeitstheorie und Finanzmathematik bekannt ist, hat die erste mathematische Modellierung einer Brownischen Bewegung versucht, analog zu Fluktuationen von Börsenkursen. Später trug der Physiker Albert Einstein mit seiner Arbeit \textit{Investigaions on the Theory of the Brwonian Movement} maßgeblich zum theoretischen Verständnis dieses beobachteten Phänomens bei. Dies wiederum diente dem Mathematiker Norbert Wiener als Grundlage für die Entwicklung des Wiener Prozesses, welcher ein zentrales Konzept in der Wahrscheinlichkeitstheorie darstellt. Bei stochastischen Differenzialgleichungen (SDGL) wird dieser Wienerprozess genutzt, um sogenanntes "white noise" bei der Modellbildung zu berücksichtigen. 

Dank all diesen Entdeckungen ist es heute möglich Finanzmathematik zu betreiben, bessere Populationsmodelle zu erstellen und den Prozess der Diffusion mathematisch zu beschreiben. Viele dieser Systeme können sensibel auf Störeinflüsse stochastischer Natur reagieren. Auf diesen  Umstand und wie man damit umgehen kann, darauf wird am Ende dieses Kapitels noch eingegangen.