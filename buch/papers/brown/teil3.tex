%
% teil3.tex -- Stochastische Differenzialgleichungen
%
% (c) 2023 Lukas Reitemeier, OST Ostschweizer Fachhochschule
%
% !TEX root = ../../buch.tex
% !TEX encoding = UTF-8
%

\section{Stochastische Differentialgleichungen\label{brown:SDGL}}
\rhead{Stochastische DGL}

Formal kann eine SDGL folgendermassen notiert werden: 
\begin{equation}
	\label{brown:SDGL:whiteNoise}
	\dot{X}(t) = a \ X(t) + B \ X(t) \ \xi(t) \quad (t>0) \ .
\end{equation}
In diesem Fall ist die Störung durch $ m $-dimensionales weisses Rauschen $ \xi(t) $ modelliert. $ B $ ist die sogenannte Dispersionsmatrix mit der Dimension $ n $ $\times $ $ m $. Sie beschreibt, wie sich die Störung auf die unterschiedlichen inneren Zustände des Systems auswirkt. Dabei beschreibt die Diagonale der Matrix, wie stark sich jede Dimension des Rausches auf die entsprechenden inneren Zustände auswirkt. Die Bereiche links und rechts der Diagonalen der Matrix beschreiben die Beeinflussung der Dimensionen untereinander. $ a $ ist der Drift-Vektor und beschreibt die erwartete Veränderung für jeden inneren Zustand des Systems. Alles in allem beschreiben
\begin{align*}
	a = 
	\begin{pmatrix}
		a_{1} \\
		\vdots \\
		a_{m}\\ 
	\end{pmatrix}
	\ und \
	B = 
	\begin{pmatrix}
		B_{11} & \dots & B_{1m} \\
		& \vdots & \\
		B_{n1} & \dots & B_{nm} 
	\end{pmatrix}
\end{align*}
die Dynamik des Systems. Da weisses Rauschen $ \xi(t) $ formal als die Ableitung vom Wiener-Prozess $ \frac{dW(t)}{dt} $ modelliert werden kann, ergibt sich die Form
\begin{equation}
	\frac{dX(t)}{dt} = aX(t) + BX(t) \ \frac{dW(t)}{dt} \quad (t>0) \ .
\end{equation}
Nun sollte noch mit $ dt $  multipliziert werden, da die Gleichung sonst impliziert, dass man weisses Rauschen weiter differenzieren kann. Diese Form
\begin{equation}
	dX(t) = a \ X(t) \ dt + B \ X(t) \ dW(t) \quad (t>0)
\end{equation}
nennt sich auch ``Itō-Form'' und ist vielfach ein erster Schritt im weiteren Umgang mit SDGLs.
Weisses Rauschen $ \frac{dW(t)}{dt} $ ist per Definition nicht sinnvoll weiter differenzierbar. Dies ist den Eigenschaften des Wiener-Prozesses geschuldet, wie in Abschnitt \ref{brown:Rauschen:RandomWalkWiener} dargelegt.



\subsection{Euler-Maruyama-Methode\label{brown:Simulation}}
\rhead{Simulation} %Kurz-Titel der Section

Die Euler-Maruyama-Methode wird oft zur numerischen Simulation von stochastischen Differentialgleichungen (SDGL) verwendet und basiert auf der bekannten Euler-Methode zur Lösung von gewöhnlichen Differentialgleichungen. Die Idee ist es, die SDGL in diskrete Zeitschritte zu zerlegen und den deterministischen, wie auch den stochastischen Anteil, separat zu behandeln. Die Methode hat zwar gewisse Einschränkungen bezüglich ihrer Genauigkeit und Stabilität, ist aber dennoch mitunter aufgrund ihrer Einfachheit weit verbreitet \cite{Bayram2018}. So wurden auch alle in diesem Kapitel gezeigten Simulationen mit dieser Methode umgesetzt. Um eine SDGL mittels der Euler-Maruyama-Methode zu simulieren, geht man wie folgt vor:

%Gegeben sei eine SDGL der folgenden Form:

%\begin{equation}
%	\mathrm{d}X(t) = a(X(t), t) \mathrm{d}t + b(X(t), t) \mathrm{d}W(t),
%\end{equation}

%wobei $ a(X(t), t) $ den deterministische Anteil darstellt, $ b(X(t), t) $ der stochastische Anteil ist und $ W(t) $ ein Wiener-Prozess ist, welcher das Rauschen repräsentiert. Die Methode beginnt mit einer Anfangsbedingung $X(0) = X_0$.

\begin{enumerate}
	\item Man wählt eine Schrittweite $ \Delta t > 0 $ und teilt das Zeitintervall $ [0, T] $ in $ N $ gleich grosse Teilintervalle der Länge $ \Delta t $.
	% : $ t_0 = 0, t_1 = \Delta t, \dots, t_i = i\Delta t, \dots, t_N = T $.
	\item Für jeden Zeitschritt $ i $ von $ 0 $ bis $ N-1 $ werden die Werte der Funktion $ X(t) $ an den diskreten Zeitpunkten $ t_i $ iterativ gemäss
	\begin{equation}
		X(t_{i+1}) = X(t_i) + a \ X(t_i) \ \Delta t + B \ X(t_i) \ \sqrt{\Delta t} \cdot Z_i \ ,
	\end{equation}
	berechnet, wobei die $ Z_i $ unabhängige standardnormalverteilte Zufallsvariablen sind.
\end{enumerate}

Vereinfacht kann die nummerische Berechnung des nächsten Simulationswertes
\begin{equation}
	X_{n+1} = X_n + f(X_n) \ \Delta t + g(X_n) \ \Delta W_n \ .
\end{equation}
als Funktion der des aktuellen Wertes  $ X_n $ ausgedrückt werden. Die Funktion $ f(X_n) $ beschreibt wieder den deterministischen Teil der SDGL, also die Drift-Komponente, welche Information enthält. Das Rauschen, wird mit $ g(X_n) $ charakterisiert, wobei $ \Delta W_n $ als numerische Implementation des Wiener-Prozesses den Zufall bei jedem Berechnungsschritt Schritt einbringt.



\subsection{Die Itō-Formel\label{brown:ito}}
\rhead{Itō} %Kurz-Titel der Section

Itō Kiyoshi war ein japanischer Mathematiker, der seine Karriere der Stochastik widmete und heute als Begründer der stochastischen Analysis erachtet werden kann. So legte er auch einen Grossteil des Fundaments, auf dem der Umgang mit SDGLs beruht. Nach ihm ist auch die schon verwendete \textit{Itō'sche Form} einer SDGL benannt, bei welcher die Gleichung mit dem Nenner des Differentialquotienten multipliziert wird. Diese Form eignet sich als Ausgangslage für viele seiner Konzepte, bietet jedoch auch den Vorteil, dass keine vermeintliche Differenzierbarkeit impliziert wird.
Eines der wichtigsten Werkzeuge im Umgang mit SDGLs, ist die sogenannte \textit{Itō-Formel}, auch \textit{Lemma von Itō} genannt. Sie stellt das Äquivalent zur klassischen Kettenregel bzw. der Substitutionsregel dar. Hier ein Beispiel einer SDGL für einen Prozess $ X(t) $ in \textit{Itō-Form}:

%Dieses wird auch im Abschnitt XXXX für das Black-Scholes-Merton Modell verwendet. Auch nach ihm benannt ist die "ito-Form", in welcher die Gleichungen notiert sind XXX.%

\begin{equation}
	dX(t) = a(X(t)) \ dt + B(X(t)) \ dW(t) \quad (t>0) \ .
\end{equation}
Angenommen, man hat eine Funktion $ f(X,t) $, dann kann die Änderung von $ f $ in Bezug auf $ X $ und $ t $ wie folgt geschrieben werden
%Die Funktion f(X,t) und die Itō-Formel sind folgendermassen gegeben:
\begin{equation}
	df = \frac{\partial f}{\partial t} \ dt + \ \frac{\partial f}{\partial X} \ dX + \frac{1}{2} \frac{\partial^2 f} {\partial X^2} \ (dX)^2 \ .	
\end{equation}
Die Itō-Formel erlaubt es uns solche Differentialgleichungen für Funktionen von stochastischen Prozessen abzuleiten. Das Einsetzen der SDGL in die Itō-Formel ergibt
\begin{equation}
	df = \frac{\partial f}{\partial t} \ dt + \frac{\partial f}{\partial X} (a \ dt + B \ dW) + \frac{1}{2} \frac{\partial^2 f}{\partial X^2} (B^2 \ dt) \ .
\end{equation}
Die Itō-Formel kann als eine stochastische Version der Taylor-Entwicklung angesehen werden. $ \frac{\partial f}{\partial t} \ dt $ beschreibt, wie sich die Funktion $ f $ allein durch den Verlauf der Zeit ändert, unabhängig von anderen Faktoren. $ \frac{\partial f}{\partial X} (a \ dt) $ ist der \textit{Drift-Term}, der den deterministischen Teil darstellt und der Term $  \frac{\partial f}{\partial X} (B \ dW) $ stellt den stochastischen \textit{Diffusions-Term} dar. Zusammen beschreiben die zwei Terme den Einfluss der ersten Ableitung und der Term $ \frac{1}{2} \frac{\partial^2 f}{\partial X^2} (B^2 \ dt) $ den Einfluss zweiter Ordnung. Dass aus $ (dX)^2 $ eingesetzt $ (B^2 \ dt) $ wird, ist folgendermassen zu erklären:
Eigentlich würde eingesetzt sich $ (dX)^2 = a^2 dt^2 + 2a B \ dt \ dW + B^2 dW^2 $ ergeben. Da $ dt^2 $ und $ dt \ dW $ im stochastischen Kontext vernachlässigbar klein sind (tendieren viel schneller gegen Null als $ dt $), ist der einzig relevante Beitrag zu $ (dX)^2 $ der Term $ B^2 \ dW^2 $. Und aufgrund der Eigenschaft des Wiener-Prozesses, dass $ dW^2 = dt $ ergibt, wird der dieser Ausdruck zu $ ()B^2 \ dt) $.

%  $ dt $ als letzter der Gleichung ergibt sich aus der Eigenschaft des Wiener-Prozess, dass die Änderung $ dW $ im Quadrat $ (dW)^2 = dt $ entspricht. 

%Die Funktion $ a(X,t) $ ist der \textit{Drift-Term}, der den deterministischen Teil darstellt, und $ B(X,t) $ der stochastische \textit{Diffusions-Term} der SDGL.

\subsection{Fazit\label{brown:fazit}}

Um den Bogen zum Anfang wieder zu schliessen, der Zufall und insbesondere Rauschen weissen keinerlei Periodizität oder harmonische Eigenschaften auf. Genau deshalb lässt sich zufälliges Rauschen gut mittels der harmonischen Analysis untersuchen und charakterisieren. Dies ermöglicht in diversen technischen Bereichen den Umgang mit zufälligen Störungen, das Eliminieren von Rauschen und es lassen sich durch SDGLs solche zufälligen Störungen modellieren. Dies führt zu einem wahrheitsgetreueren Abbild der Realität. Denn der Zufall beeinflusst in einem gewissen Masse praktisch jedes System --- hoffentlich lieferte dieses Kapitel einen hilfreichen und interessanten Überblick im Umgang damit.
