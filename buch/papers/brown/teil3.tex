%
% teil3.tex -- Stochastische Differenzialgleichungen
%
% (c) 2023 Lukas Reitemeier, OST Ostschweizer Fachhochschule
%
% !TEX root = ../../buch.tex
% !TEX encoding = UTF-8
%

\section{Stochastische Differenzialgleichungen\label{brown:SDGL}}
\rhead{Stochastische DGL}

Formal kann eine SDGL folgendermassen notiert werden: 

\begin{equation}
	\label{brown:SDGL:whiteNoise}
	\dot{X}(t) = a(X(t)) + b(X(t))\xi(t) \quad (t>0)
\end{equation}

In diesem Fall ist die Störung durch $ m $-dimensionales weisses Rauschen $ \xi(t) $ modelliert. $ B $ ist die sogenannte Dispersionsmatrix mit der Dimension $ n $ x $ m $. Sie beschreibt, wie sich die Störung auf die unterschiedlichen inneren Zustände des Systems auswirkt. Dabei beschreibt die Diagonale der Matrix wie stark sich jede Dimension des Rausches auf die entsprechenden inneren Zustände auswirkt. Die Dreieckbereiche der Matrix beschreiben die Beeinflussung der Dimensionen untereinander. $ a $ ist der Drift-Vektor und beschreibt die erwartete Veränderung für jeden inneren Zustand des Systems. Alles in allem beschreiben $ a $ und $ B $ die Dynamik des Systems. 

\begin{align*}
	a = 
	\begin{pmatrix}
		a_{1} \\
		\vdots \\
		a_{m}\\ 
	\end{pmatrix}
	, \quad
	B = 
	\begin{pmatrix}
		B_{11} & \dots & B_{1m} \\
		& \vdots & \\
		B_{n1} & \dots & B_{nm} 
	\end{pmatrix}
\end{align*}


Da weisses Rauschen $ \xi(t) $ formal als die Ableitung vom Wiener-Prozess $ \frac{dW(t)}{dt} $ modelliert werden kann, ergibt sich folgende Form:

\begin{equation}
	\frac{dX(t)}{dt} = a(X(t)) + B(X(t)) \frac{dW(t)}{dt} \quad (t>0)
\end{equation}

Nun sollte noch mit $ dt $  multipliziert werden, da die Gleichung sonst impliziert, dass man weisses Rauschen weiter differenzieren kann. Diese Form nennt sich auch ``Itō-Form'' und ist vielfach ein erster Schritt im weiteren Umgang mit SDGLs.

\begin{equation}
	dX(t) = a(X(t)) dt + B(X(t)) dW(t) \quad (t>0)
\end{equation}

Weisses Rauschen $ \frac{dW(t)}{dt} $ ist per Definition nicht sinnvoll weiter differenzierbar. Dies ist den Eigenschaften des Wiener-Prozesses geschuldet. Wie in \ref{brown:Rauschen:RandomWalkWiener} dargelegt, entspricht die Differenzierung im Frequenzbereich einer Multiplikation mit der Frequenz $ jwt $. Differenziert man nun einen sonst schon konstantes gleichmäßig verteiltes Frequenzspektrum, vergrößert man nur die Amplituden, was sich im Zeitbereich durch ein zunehmend unstetigeres Signal äussert.



\subsection{Euler-Maruyama-Methode\label{brown:Simulation}}
\rhead{Simulation} %Kurz-Titel der Section

Die Eueler-Meruyama-Methode wird oft zur nummerischen Simulation von stochastischen Differentialgleichungen (SDGLs) verwendet und basiert auf der bekannten Euler-Methode zur Lösung von gewöhnlichen Differentialgleichungen. Die Idee ist es, die SDGL in diskrete Zeitschritte zu zerlegen und den deterministischen und stochastischen Anteil separat zu behandeln. Die Methode hat zwar gewisse Einschränkungen bezüglich ihrer Genauigkeit und Stabilität, ist aber dennoch mitunter auf Grund ihrer Einfachheit weit verbreitet. So wurden auch alle in diesem Kapitel gezeigten Simulationen mit dieser Methode umgesetzt. 


%Gegeben sei eine SDGL der folgenden Form:

%\begin{equation}
%	\mathrm{d}X(t) = a(X(t), t) \mathrm{d}t + b(X(t), t) \mathrm{d}W(t),
%\end{equation}

%wobei $ a(X(t), t) $ den deterministische Anteil darstellt, $ b(X(t), t) $ der stochastische Anteil ist und $ W(t) $ ein Wiener-Prozess ist, welcher das Rauschen repräsentiert. Die Methode beginnt mit einer Anfangsbedingung $X(0) = X_0$.

Um die SDGL mittels der Euler-Maruyama-Methode zu simulieren, geht man wie folgt vor:

\begin{enumerate}
	\item Man wählt eine Schrittweite $ \Delta t > 0 $ und teilt das Zeitintervall $ [0, T] $ in $ N $ gleich große Teilintervalle der Länge $ \Delta t$ : $ t_0 = 0, t_1 = \Delta t, \dots, t_i = i\Delta t, \dots, t_N = T $.
	\item Für jeden Zeitschritt $ i $ von $ 0 $ bis $ N-1 $ werden die Werte der Funktion $ X(t) $ an den diskreten Zeitpunkten $ t_i $ berechnet,
	\begin{equation}
		X(t_{i+1}) = X(t_i) + a(X(t_i), t_i) \Delta t + b(X(t_i), t_i) \sqrt{\Delta t} \cdot Z_i,
	\end{equation}
	wobei $ Z_i $ unabhängige standardnormalverteilte Zufallsvariablen sind.
	\item Diese Berechnungen führt man iterativ für alle Zeitschritte durch.
\end{enumerate}

\begin{equation}
	X_{n+1} = X_n + f(X_n,t_n) \Delta t + g(X_n,t_n) \Delta W_n
\end{equation}

Die Funktion $ f(X_n,t_n) $ beschreibt den deterministischen Teil der SDGL, auch Drift genannt. Ohne diesen Anteil, würde der Erwartungswert zu jedem Zeitpunkt dem Startwert entsprechen. Dies ist auch der Teil, welcher Information enthält und durch harmonische Analysis und anderen Methoden, untersucht werden kann. Das Rauschen, welches hier mit $ g(X_n,t_n) $ beschrieben ist, enthält keine Information und kann lediglich bezüglich Intensität charakterisiert werden. $ \Delta W $ beschreibt die Geschwindigkeit, mit welcher der stochastische Prozess ablaufen soll und $ W_n $  beschreibt den Wiener-Prozess.



\subsection{Die Itō-Formel\label{brown:ito}}
\rhead{Ito} %Kurz-Titel der Section

Itō Kiyoshi war ein japanischer Mathematiker, der seine Kariere der Stochastik widmete und heute als Begründer der stochastischen Analysis erachtet werden kann. So legte er auch einen Grossteil des Fundaments auf dem der Umgang mit SDGLs beruht. 
Nach ihm ist auch die schon verwendete \textit{Itō'sche Form} einer SDGL benannt, bei welcher die Gleichung mit dem Nenner des Differenzialquotienten multipliziert wird. Diese Form eignet sich als Ausgangslage für viele seiner Konzepte, bietet jedoch auch den Vorteil, dass keine vermeintliche Differenzierbarkeit impliziert wird.
Eines der wichtigsten Werkzeuge, um mit SDGLs umzugehen, ist die sogenannte \textit{Itō-Formel}, auch \textit{Lemma von Itō} genannt. Dies stellt das Äquivalent zur Kettenregel bzw. der Substitutionsregel dar. %Dieses wird auch im Abschnitt XXXX für das Black-Scholes-Merton Modell verwendet. Auch nach ihm benannt ist die "ito-Form", in welcher die Gleichungen notiert sind XXX.%


Hier ein Beispiel einer SDGL für einen Prozess X(t) in \textit{Itō-Form}:

\begin{equation}
	dX = a(X,t) dt + B(X,t) dW
\end{equation}

Angenommen, man h eine Funktion f(X,t), wobei X die Lösung der obigen SDGL darstellt, dann kann die Änderung von $ f $ in Bezug auf $ X $ und $ t $ wie folgt geschrieben werden:
%Die Funktion f(X,t) und die Itō-Formel sind folgendermassen gegeben:

\begin{equation}
	df = \frac{\partial f}{\partial t} dt + \frac{\partial f}{\partial X} dX + \frac{1}{2} \frac{\partial^2 f}{\partial X^2} (dX)^2	
\end{equation}

Die Itō-Formel erlaubt es uns, solche Differentialgleichungen für Funktionen von stochastischen Prozessen abzuleiten. Das Einsetzen der SDG in das Ito-Lemma ergibt:

\begin{equation}
	df = \frac{\partial f}{\partial t} dt + \frac{\partial f}{\partial X} (a dt + B dW) + \frac{1}{2} \frac{\partial^2 f}{\partial X^2} (B^2 dt)
\end{equation}

Die Funktion $ a(X,t) $ ist der \textit{Drift-Term}, der den deterministischen Teil darstellt und $ B(X,t) $ den \textit{Diffusions-Term}, also den stochastischen Teil der SDGL.

\subsection{Fazit\label{brown:fazit}}

Um den Bogen zum Anfang wieder zu schliessen, der Zufall und insbesondere Rauschen, weist keinerlei Periodizität oder harmonische Eigenschaften auf. Genau desshalb lässt es sich gut mittels der harmonischen Analysis untersuchen und charakterisieren. Dies ermöglicht in diversen technischen Bereichen den Umgang mit zufällige Störungen, das eliminieren von Rauschen. 
Mittels stochastischen Differenzialgleichungen können solche zufällige Störungen beim modellieren von Systemen beachtet werden, wodurch ein wahrheitsgetreueres Abbild der Realität entsteht. Denn der der Zufall beeinflusst in einem gewissen Masse praktisch jedes System --- hoffentlich lieferte dieses Kapitel einen hilfreichen interessanten Überblick im Umgang damit.
