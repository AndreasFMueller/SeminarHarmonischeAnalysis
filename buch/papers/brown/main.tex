%
% main.tex -- Paper zum Thema <brown>
%
% (c) 2020 Lukas Reitemeier, OST Ostschweizer Fachhochschule
%
% !TEX root = ../../buch.tex
% !TEX encoding = UTF-8
%
\chapter{Brownische Bewegung und stochastische Differenzialgleichungen\label{chapter:brown}}
\kopflinks{Brownische Bewegung und stochastische Differenzialgleichungen}
% Kurzform des Kapitel-Titels
\begin{refsection}
\chapterauthor{Lukas Reitemeier}


Dieses Kapitel befasst sich mit der Brown'schen Bewegung und stochastischen Differenzialgleichungen - zwei eng verbundene Themen welche interdisziplinär in vielen verschiedenen Gebieten zur Anwendung kommen.

Die Entdeckungsgeschichte widerspiegelt diese Interdisziplinaritarität. Der Botaniker Robert Brown entdeckte im 19. Jahrhundert, dass in Flüssigkeit suspendierte Pollenteilchen unter seinem Mikroskop scheinbar zufällige Bewegungen ausführen. Der französische Mathematiker Luis Bachelier, der für seine Arbeit \glqq \textit{Théorie de la spéculation}\glqq{} in der Wahrscheinlichkeitstheorie und Finanzmathematik bekannt ist, hat die erste mathematische Modellierung einer Brownischen Bewegung versucht, analog zu Fluktuationen von Börsenkursen. Später trug der Physiker Albert Einstein mit seiner Arbeit \glqq \textit{Investigaions on the Theory of the Brwonian Movement}\glqq{} maßgeblich zum theoretischen Verständnis dieses beobachteten Phänomens bei. Dies wiederum diente dem Mathematiker Norbert Wiener als Grundlage für die Entwicklung des Wiener Prozesses, der ein zentrales Konzept in der Wahrscheinlichkeitstheorie darstellt. Stochastische Differenzialgleichungen (SDE) basieren genau auf diesem Wienerprozess, um sogenannten "White Noise"  bei der Modellbildung zu berücksichtigen. So ist es heute möglich Finanzmathematik zu betreiben, bessere Populationsmodelle zu erstellen, Diffusion und Wärmeaustausch zu erklären.


Um dem Thema des Buch gerecht zu werden und alle in diesem Kapitel behandelten Themen in einer Anwendung zu vereinen, wird noch das Black-Scholes-Merton-Modell eingeführt. Dieses wurde in den 1970 Jahren entwickelt und gilt als bedeutender Meilenstein in der Finanzmathematik.

% ToDo: In Kontext bringen
% Die harmonische Analysis....

%
% einleitung.tex -- Beispiel-File für die Einleitung
%
% (c) 2020 Prof Dr Andreas Müller, Hochschule Rapperswil
%
% !TEX root = ../../buch.tex
% !TEX encoding = UTF-8
%
\section{Brownische Bewegung\label{brown:section:teil0}}
\rhead{Brownische Bewegung}

Als der Schottische Botaniker Robert Brown im Jahr 1827 in sein Mikroskop schaut, beobachtet er kleine Partikel von Pflanzenpollen in einer Flüssigkeit. Er bemerkt, dass sich die Teilchen scheinbar zufällig bewegen, obwohl scheinbar keine Kräfte auf die Teilchen einwirken. Eine Erklärung hatte Robert Brown zu diesem Zeitpunkt für das Verhalten noch nicht und notierte dass sich die Teilchen scheinbar zufällig und unregelmässig bewegen.

Es dauerte fast ein Jahrhundert, bis Albert EInstein diese Beobachtung auf ein solides theoretisches Fundament stellte und nutzbar machte. Er schloss darauf, dass dir unregelmässige Bewegung auf Kolisionen mit umgebenden Mölekühlen zurückzuführen ist, welche ständig in Bewegung sind. In seiner THeorei beschreibt er...

Die 



Text
\cite{brown:bibtex}.



%
% teil1.tex -- Brownische Bewegung
%
% (c) 2023 Lukas Reitemeier, OST Ostschweizer Fachhochschule
%
% !TEX root = ../../buch.tex
% !TEX encoding = UTF-8
%

\section{Brownsche Bewegung\label{brown:BrownBewegung}}
\rhead{Brownsche Bewegung}

Als der Schottische Botaniker Robert Brown im Jahr 1827 in sein Mikroskop schaut, beobachtet er kleine Pollen-Partikel in einer Flüssigkeit. Er bemerkt, dass sich die Teilchen scheinbar zufällig bewegen, obwohl keine Kräfte auf die Teilchen einwirken. Eine Erklärung hatte Robert Brown zu diesem Zeitpunkt für das Verhalten nicht.


Es dauerte fast ein Jahrhundert, bis Albert Einstein diese Beobachtung auf ein solides theoretisches Fundament stellte und so nutzbar machte. Er schloss darauf, dass die unregelmäßige Bewegung auf Kollisionen mit umgebenden Molekülen zurückzuführen ist, welche ständig in Bewegung sind. Albert Einstein entwickelte ein mathematische Theorie, mit welcher die beobachtete Bewegung durch stochastische Zusammenstöße von Molekülen erklärt werden kann. Weiter konnte er einen Zusammenhang zwischen den Messungen der Bewegung, der Bolzmann-Konstante und der  Avogadro-Zahl herstellen\footnote{Die Avogadro-Zahl ($N_\mathrm{A}$) ist die Anzahl Teilchen (Atome oder Moleküle) welche in einem Mol einer Substanz enthalten sind. Die Boltzmann-Konstante ($k$) ist eine physikalische Konstante, die den Zusammenhang zwischen der thermischen Energie und der Temperatur eines Systems herstellt. Beide Konstanten sind miteinander verbunden durch die Gleichung $k = \frac{R}{N_\mathrm{A}}$, wobei $R$ die allgemeine Gaskonstante ist.}.


Dies hatte auch Auswirkungen auf das Verständnis von Materie, da dies einen direkten empirischen Beleg für die Existenz von Atomen und Moleküle lieferte --- die Atomhypothese wurde bestätigt und sogar beobachtbar. Diese besagt, dass Materie aus diskreten unteilbaren Einheiten besteht, sprich einzelnen Atomen oder ganzen Molekülen. Man muss dazu sagen, dass es schon zuvor Indizien für deren Existenz gab, doch es fehlte am entscheidenden experimentellen Beweis.


Die \textit{Einstein-Smoluchowski-Gleichung} ist eine zentrale Gleichung seiner Arbeit, welche die mittlere quadratische Verschiebung (MSD: \textit{Mean square displacement}) beschreibt
\begin{equation}
	\mathrm{MSD} = 2nDt
\end{equation}
in Beziehung zur Diffusionskonstanten $ D $ , der Anzahl Raumdimensionen $ n $ und der Zeit $ t $.

Die Avogadro-Zahl $ A $ kann indirekt über die \textit{Stokes-Einstein-Gleichung}, anhand von Messungen berechnet werden. Die Diffusionskonstante $ D $ kann dabei mittels des Radius der suspendierten Teilchen, der Boltzmann-Konstante  $ k $, der Temperatur $ T $ und der Viskosität $ \eta $ des umgebenden Mediums, berechnet werden:
\begin{equation}
	D = \frac{kT}{6\pi\eta r}
\end{equation}
$ r $ beschreibt dabei den Radius des suspendierten Teilchens. Die Boltzmann-Konstante $ k $ kann auch durch die allgemeine Gaskonstante $ R $ a und die Avogadro-Zahl $ \mathrm{A} $ ausgedrückt werden: $ k = R/\mathrm{A} $

So ergibt sich folgender Zusammenhang:
\begin{equation}
	\mathrm{A} = \frac{R T}{D (6 \pi \eta r)}
\end{equation}

So ermöglichen die beiden Gleichungen einen experimentellen Ansatz zur Bestimmung der Avogadro-Zahl und ferner die quantitative Analyse einer brownschen Bewegung.


Eine solche brownsche Bewegung kann relativ einfach mittels der Euler-Maruyama-Methode \ref{brown:Simulation} numerisch simuliert werden. Zugrunde liegt dabei der Wiener-Prozess \ref{brown:Rauschen:RandomWalkWiener}, als stochastischer Prozess. Werden Änderungsraten in X- und Y-Richtung von diesem abhängig gemacht und die Simulationsschritte mittels Linien verbunden, ergibt sich das Bild einer typischen brownschen Bewegung, wie dies in der Abbildung \ref{brown:2Dbrownian} zu sehen ist. Führt man eine solche Simulation in einer Dimension durch und implementiert eine durchschnittlich erwartete Änderungsrate, so ergibt sich ein Verlauf der stark an einen Börsenkurs erinnert, wie es in der Abbildung \ref{brown:1Dbrownian} dargestellt ist. Man kann vielleicht schon erahnen, dass die zugrundeliegende Mathematik auch für die Finanzindustrie von grossem Nutzen ist.

\begin{figure}
	\centering
	\begin{minipage}{0.45\textwidth}
		\centering
		\includegraphics[width=\linewidth]{papers/brown/images/simulierter-boersenkurs.png}
		\caption{Simulierter Börsenkurs}
		\label{brown:1Dbrownian}
	\end{minipage}
	\hspace{0.05\linewidth}
	\begin{minipage}{0.45\textwidth}
		\centering
		\includegraphics[width=\linewidth]{papers/brown/images/simulierte-bronsche-bewegung.png}
		\caption{Simulierte brownsche Bewegung}
		\label{brown:2Dbrownian}
	\end{minipage}
\end{figure}



%
% teil2.tex
%
% (c) 2023 Vincent Haufe, Hochschule Rapperswil
%
% !TEX root = ../../buch.tex
% !TEX encoding = UTF-8

\section{Konstruktion der Mellin-Transformation
\label{mellin:section:teil2}}
\rhead{Mellin-Transformation}
Wie schon bemerkt, sind wir bei diesem Problem mit der multiplikativen 
Gruppe der positiven reellen Zahlen $(\mathbb{R^+},\cdot)$ konfrontiert
\footnote{Eine ausführliche Beschreibung dessen, was in diesem Abschnitt 
folgt, findet sich im Kapitel \ref{buch:chapter:gruppen} diese Buches}.
\begin{definition}
    Eine Gruppe setzt sich aus einer {\em Menge} und einer 
    {\em Verknüpfung} zusammen. 
    Elemente aus der Menge können über die Verknüpfung miteinander 
    verrechnet werden, müssen aber stets wieder Elemente der Menge bilden.
\end{definition}
Diese Information ist der Startpunkt auf der Suche nach einer 
Integraltransformation die unser Problem lösen kann.
Die Gelfand-Theorie ist die verallgemeinerte Theorie der Fourieranalyse und 
beschreibt die notwendigen Bausteine einer solchen.
Zum einen sind {\em Analysefunktionen} zu bestimmen, welche die 
Gruppenoperation in die Multiplikation überführt und gewisse 
Orthogonalitätsbedingungen erfüllen. 
Des Weiteren muss das {\em Skalarprodukt} auf die multiplikative Gruppe 
angepasst werden, mit dem die Integraltransformation durchgeführt werden 
soll.
Aus Gruppe, Analysefunktionen und Skalarprodukt folgt dann
für kommutative (Lie-)Gruppen die entsprechende Transformation.


% Prinzipien:

% -Gruppe 
% -Analysefunktionen sind Homomorphismen und orthogonal zueinander: h (Eigenfunktionen (Differential)operator?)
% -Skalarprodukt (Integral)
% -Analyse <h,f> = Gf(h)
% -Faltung: Integral über ganze Gruppe mit Gruppenoperation
% -Faltungsformel: Aus Faltung wird Produkt
% -zulässige h bilden duale Gruppe


\subsection{Analysefunktionen der Mellin-Transformation
\label{mellin:subsection:analysefunktionen}}

Unser Ziel ist es, die für unser Registrierungsproblem neu definierte 
Kreuzkorrelation mithilfe einer Integraltransformation in eine 
einfache Multiplikation der transformierten Funktionen zu verwandeln.
Dafür braucht es spezielle Funktionen, sogenannte Homomorphismen, welche 
die Gruppenoperation in die Multiplikation überführen können, und dabei 
die Faltungsformel der Integraltransformationen garantieren.
Dem Warum wollen wir an dieser Stelle nicht weiter nachgehen.
% \begin{definition}
%     Homomorphismus nur für Faltungsformel and I don't understand
% \end{definition}
\begin{satz}
    Die Analysefunktion einer Gelfand-Transformation ist ein 
    Homomorphismus und führt die Gruppenoperation in eine 
    Multiplikation über
    \[
        f(x \diamond y) 
        = f(x) \cdot f(y)
        ,
    \]
    wobei $\diamond$ der Gruppenoperation enstpricht.
\end{satz}
Dies gilt sowohl für die Gruppenoperationen der Gruppe im Ortsraum als auch 
für die Gruppenoperation der dualen Gruppe im Bildraum der Transformation.
Mit dem Konzept der dualen Gruppe werden wir uns am Ende des Abschnitts 
nochmals genauer befassen, fürs Erste konzentrieren wir uns auf die 
multiplikative Gruppe in $\mathbb{R^+}$, welche relevant für unsere 
skalierten Funktionen ist, also im Ortsraum.
\medskip

Bei der Fourier-Transformation, dessen Gruppenoperation die Additon ist, 
müssen die Analysefunktionen also die Gleichung 
\begin{equation}
    f(x + y) 
    = f(x) \cdot f(y)
    \label{mellin:hom1}
\end{equation}
erfüllen. 
Die Funktion, die diese Eigenschaft besitzt ist natürlich die 
Exponentialfunktionen, denn 
\begin{equation}
    e^{x + y} 
    = e^x \cdot e^y
    ,
    \label{mellin:exp}
\end{equation}
und somit der Typ der bekannten Analysefunktion der Fourier-Transformation.
In der Fortsetzung bedeutet das für die neue Transformation, dass dessen 
Analysefunktionen die Gleichung
\begin{equation}
    f(x \cdot y) 
    = f(x) \cdot f(y)
    \label{mellin:hom2}
\end{equation}
erfüllen müssen.
Erfüllt wird die Gleichung durch jede Potenzfunktion $h(x) = x^{z}$ für 
beliebige $z \in \mathbb{C}$ 
\begin{equation}
    h(x \cdot y) 
    = (x \cdot y)^{z} = x^{z} \cdot y^{z}
    ,
\end{equation}
% denn Potenzen lassen sich bekanntlich beliebig zusammenfassen so wie
% \begin{equation}
%     (a \cdot b)^2 = a^2 \cdot b^2 \left(\frac{a}{b}\right)^2  = \frac{a^2}{b^2} 
% \end{equation}
Unsere Analysefunktionen müssen also vom Typ Potenzfunktion sein.

Zudem muss sichergestellt sein, dass die Integrale der Transformation konvergieren.
Dafür dürfen die Analysefunktionen über den Integrationsbereich gesehen nicht zu 
stark anwachsen, oder noch besser, einen konstanten Betrag aufweisen. 
Bei den Analysefunktionen der Fourier-Transformation wird deshalb ein rein 
imaginärer Exponent verwendet, nämlich $j\omega t$, was gemäss der Eulerschen 
Definition einer komplexen Schwingung mit konstantem Betrag $= 1$ entspricht.
Damit das Integral der Mellin-Transformation konvergiert, muss jedoch dafür 
das verwendete Skalarprodukt selbst modifiziert werden.
% Die komplexe Exponentialfunktion $e^{j\omega_l t}$ ist orthogonal zu $e^{j\omega_k t}$, sofern $\omega_l \neq \omega_k$, da diese 
% immer ein Skalarprodukt gleich Null haben wie sich leicht zeigen lässt
% \begin{proof}
%     \begin{align*}
%         \langle h_l,h_k \rangle &= \int_\mathbb{R} e^{j\omega_l t} \cdot e^{-j\omega_k t}\,dt \\
%         &= \int_\mathbb{R} e^{j(\omega_l - \omega_l) t}\,dt \\
%         &= \int_\mathbb{R} e^{j\omega_l-k t}\,dt = 0 \vert \omega_l \neq \omega_k 
%     \end{align*}
%     Da die komplexe Exponentialfunktion periodisch ist.
% \end{proof}
% Bei den komplexen Exponentialfunktionen der Fourier-Transformation ist dies auch intuitiv, da dahinter die Sinus- und 
% Cosinusschwingungen unterschiedlicher Frequenzen stehen, in die eine Funktion zerlegt werden kann.


\subsection{Skalarprodukt der Mellin-Transformation
\label{mellin:subsection:skalarprodukt}}
Die Entdeckung der Fourierreihen brachte die Notwendigkeit mit sich, den 
Riemannschen Integralbegriff\footnote{Siehe auch Kapitel 
\ref{buch:skalarprodukt:section:funktionenraeume}} zu überdenken und zu 
präzisieren.
Dafür wurde vor allem die aufkommende Masstheorie angewendet.
Relevant ist dabei nun, dass wir für unser Integral und somit Skalarprodukt 
mit welchem wir die Transformation durchführen wollen, 
mit einem {\em Mass}\footnote{in diesem Kontext genauer: ein Haar-Mass} 
ausstatten. 
Damit gelingt es uns, das Integral 
invariant bezüglich der Gruppenoperation zu machen, was für den 
Erfolg der Gelfand-Theorie zwingend nötig ist, denn es gilt:

\begin{definition}
    Das Integral, mit dem eine Integraltransformation durchgeführt wird, 
    muss invariant bezüglich der Gruppenoperation der Gruppe 
    der zu transformierenden Funktionen sein.
    \[
        \int_\mathbb{G} f(x)\,{d}x 
        = \int_\mathbb{G} f(x \diamond x_0)\,{d}x
    \]
\end{definition}


Invariant bezüglich der Gruppenoperation bedeutet, dass sich das Integral 
über den gesamten Definitionsbereich der Gruppe nicht ändert, wenn das 
Argument der zu integrierenden Funktion ``verschoben'' wird. 
Das gewöhnliche Integral der Fourier-Transformation erfüllt diese 
Eigenschaft 
\begin{equation}
    \int_\mathbb{R} f(x)\,{d}x 
    = \int_\mathbb{R} f(x + x_0)\,{d}x
\end{equation}
da hier ja von $-\infty$ bis $+\infty$ integriert wird und somit eine nur 
endlich verschobene Funktion stets vollständig abmisst.
Verschoben meint in diesem Sinne aber Verschiebung mit der Gruppenoperation, 
was bei Fourier also ein ``+'' war, wird bei unserer neuen Transformation 
zu einem ``$\cdot$'', das heisst zu einer Streckung des Arguments. 
Es wird schnell klar, dass das herkömmliche Integral eben {\em nicht} 
invariant bezüglich dieser Streckung ist und allgemein gilt 
\begin{equation}
    \int_\mathbb{R^+} 
    f(x)\,{d}x \neq \int_\mathbb{R^+} f(s \cdot x)\,{d}x
    \label{mellin:ungl}
\end{equation}
da, einfach gesagt, eine Streckung der $x$-Achse die Funktion und somit deren 
Flächeninhalt auseinderzieht.
Es braucht nun also eine Anpassung, um diese Veränderung des Flächeninhalts 
zu kompensieren.
Das kann überraschend einfach mithilfe einer Division durch die Variable 
des Arguments $x$ erreicht werden, was auch sehr einfach gezeigt werden 
kann.
\begin{proof}[Beweis]
    Zu beweisen ist die folgende Gleichung:
    \[
        \int_\mathbb{R^+} f(s \cdot x)\,\frac{{d}x}{x} 
        = \int_\mathbb{R^+} f(x)\,\frac{{d}x}{x}
    \]
    Eine Substitution von $s \cdot x$ mit $u$
    \begin{align*}
        u &= sx \\
        x &= \frac{u}{s} \\
        {d}x &= \frac{1}{s} {d}u
    \end{align*}
    führt auf
    \[
        \int_\mathbb{R^+} f(sx) \cdot x^{-1}\,{d}x 
        = \int_\mathbb{R^+} f(u) \cdot \frac{s}{u} \frac{1}{s}\,{d}u
        = \int_\mathbb{R^+} f(u)\,\frac{{d}u}{u}
        = \int_\mathbb{R^+} f(x)\,\frac{{d}x}{x}
    \]
    Was beweist, dass sich der Faktor $s$ durch das zusätzliche Mass 
    $x^{-1}$ herauskürzt.
\end{proof}
Eine Intuition für diesen Umstand liefert auch ein Beispiel.
\begin{beispiel}
Man betrachte die Funktion 
\[
f(x) 
= 
\begin{cases}
    x^3 &\qquad 0\leq x\leq 1\\
    0 &\qquad \text{sonst}
\end{cases}
\]
Die Ungleichung\eqref{mellin:ungl} lässt sich für das normale Integral 
leicht bestätigen:
\[
    \int_\mathbb{R^+} f(x)\,dx
    = \int_0^1 x^3\,dx
    = \frac{x^4}{4} \Big|_0^1
    = \frac{1}{4}
\]
\[
    \int_\mathbb{R^+} f(sx)\,dx
    = \int_0^2 \frac{x}{2}^3\,dx
    = \frac{x^4}{8} \Big|_0^2
    = \frac{1}{8}\frac{16}{4}
    = \frac{1}{2}
\]
Wird hingegen das Integral mit dem zusätzlichem Mass $x^{-1}$ verwendet 
ergibt sich
\[
    \int_\mathbb{R^+} f(x)\,\frac{{d}x}{x} 
    = \int_0^1 x^2\,dx
    = \frac{x^3}{3} \Big|_0^1
    = \frac{1}{3}
\]
\[
    \int_\mathbb{R^+} f(sx)\,\frac{{d}x}{x} 
    = \int_0^2 \frac{x}{2}^3\,\frac{{d}x}{x}
    = \int_0^2 \frac{1}{8} x^2\,dx
    = \frac{1}{8}\frac{x^3}{3} \Big|_0^2
    = \frac{1}{\bcancel{8}}\frac{\bcancel{8}}{3}
    = \frac{1}{3}
\]
Der zusätzliche Faktor kürzt sich also durch die Auswertung der ebenso 
gestreckten Grenzen wieder heraus, weil diese in die gleiche Potenz von 
$x$ eingesetzt werden, was dank dem Mass $x^{-1}$ der Fall war.
\end{beispiel}
Das für die Multiplikation als Gruppenoperation modifizierte Integral ist 
somit
\begin{equation}
    \int_\mathbb{R^+} f(x)\,\frac{{d}x}{x}
    ,
\end{equation}
und das Skalarprodukt mit dem die Transformation erfolgt
\begin{equation}
    \langle f,g \rangle 
    = \int_\mathbb{R^+} \overline{f(x)} \cdot g(x) \,\frac{{d}x}{x}
    .
\end{equation}

\subsection{Korrelation und Faltung auf $\mathbb{R^+}$
\label{mellin:subsection:faltung}}
Wir haben nun alle essenziellen Bausteine beisammen um die gesuchte 
Transformation zu formulieren: grundlegende Gruppe, Analysefunktionen 
und Skalarprodukt.
Das Rezept dafür lautet nun wie folgt. Es wird das für die Gruppe neu 
definierte Skalarprodukt der Analysefunktion $h(x)$ mit der zu 
transformierenden Funktion $f(x)$ gebildet.
\begin{equation}
    \langle h,f \rangle 
    = \int\limits_{0}^{\infty} x^{z} f(x) \,\frac{{d}x}{x}
\end{equation}
eine kleine Umformung führt zu einer kompakten Formel der 
{\em Mellin-Transformation}
\begin{equation}
    \mathcal{M}(f)(z) 
    = \int\limits_{0}^{\infty} x^{z-1} f(x) \,{d}x 
    \qquad\text{für $z \in \mathbb{C}$}
    \label{mellin:mellin}
\end{equation}
Dies ist die allgemeinste Form der Mellin-Transformation. 
Wenn man aber die komplexe Zahl $z = \delta + ju$ auf ihren Imaginärteil 
beschränkt $\operatorname{Re}(z) = \delta = 0$ erhält man die Formel
\begin{equation}
    \mathcal{M}(f)(u) 
    = \int\limits_{0}^{\infty} x^{ju-1} f(x) \,{d}x 
    \qquad\text{für $u \in \mathbb{R}$}
    ,
    \label{mellin:mellinu}
\end{equation}
die sich noch als nützlich und intuitiv erweisen wird.

Ebenfalls lassen sich die Korrelation sowie die Faltung über die 
Gruppenoperation und das modifizierte Skalarprodukt bestimmen:
\begin{equation}
    (f \star g)(\sigma ) 
    = \int_\mathbb{R^+} 
    f(x) \cdot g(\sigma ^{-1} \cdot x)\,\frac{{d}x}{x}
    \label{mellin:kreuzkorrelation*}
\end{equation}
\begin{equation}
    (f \ast g)(x) 
    = \int_\mathbb{R^+} 
    f(y) \cdot g(x \cdot y^{-1})\,\frac{{d}y}{y} 
\end{equation}
Damit erhalten wir von der Gelfand-Theorie zum krönenden Abschluss das 
Faltungstheorem
\begin{equation}
    \mathcal{M}\left(f \ast g\right)
    = \mathcal{M}\left(f\right) \cdot \mathcal{M}\left(g\right)
    .
\end{equation}

\subsection{Rücktransformation
\label{mellin:subsection:ruecktransformation}}
Die Mellin-Transformation wurde nun aus den Regeln der Gelfandtheorie 
konstruiert und findet in den Gleichungen \eqref{mellin:mellin} \& 
\eqref{mellin:mellinu} ihre definitive Form.
% \footnote{obschon je nach Literatur und Geschmack auch noch Konstanten $\frac{1}{\sqrt{2\pi j}}$, $\frac{1}{2\pi j}$ vorkommen können.}
Jedoch haben wir damit eigentlich erst die Hälfte des Bildes, schliesslich 
wollen wir aus dem Bildraum der Transformation, nennen wir ihn 
{\em Mellin-Raum}, wieder in den Ortsraum zurücktransformieren. 
Bei Fourier ging dies mit dem fast gleichen Integral wie das der 
Hintransformation, mit Ausnahme der Skalierungskonstante $\frac{1}{2\pi}$
und dem Kehren des Vorzeichens vor der Integrationsvariable.
\begin{align*}
    \hat{f}(\omega) &= \int\limits_{-\infty}^{\infty} 
    e^{-j\omega{}t} f(t) \,{d}t \\
    f(x) &= \frac{1}{2\pi} \int\limits_{-\infty}^{\infty} 
    e^{j\omega t} \hat{f}(\omega) \,{d}\omega
\end{align*}
Tatsächlich beruht diese Symmetrie der Hin- und Rücktransformation auf der 
zur Fourier-Transformation gehörenden {\em dualen} Gruppe.
\subsubsection{Die duale Gruppe einer Transformation}
Die duale Gruppe einer Integraltransformation charakterisiert den 
(dualen) Raum von denjenigen Funktionen, mit denen sich die 
ursprüngliche Funktion wieder zusammensetzen lässt.
Die Funktionen, aus denen sich die ursprüngliche Funktion zusammenbauen 
lässt, beziehungsweise in die sie sich zerlegen lässt, 
sind immer komplexe Exponentialfunktion und damit Frequenzen gemäss 
der Eulerschen Definition.
Wie wir schon festgestellt haben, ist die Gruppenoperation von solchen 
Exponentialfunktionen immer die Addition \eqref{mellin:exp}.
Der zweite Bestandteil der Gruppe, die Menge, ist dabei von der Variable 
abhängig, welche die Menge dieser Exponentialfunktionen charakterisiert.
Bei den Fourierreihen beispielsweise war diese Variable eine ganze
Zahl $k \in \mathbb{Z}$.
Das bedeutete, dass für jedes $k \in \mathbb{Z}$ eine solche 
Exponentialfunktion $e^{jkx}$ sowie ein Koeffizient existiert haben, aus 
welchen sich die ursprüngliche, dort $2\pi$- periodische Funktion wieder 
zusammensetzen liess.
Die Menge der dualen Gruppe war hier also die der ganzen Zahlen 
$\mathbb{Z}$. 
Bei der Fourier-Transformation ist die Variable $\omega \in \mathbb{R}$, 
somit ist auch die Menge der dualen Gruppe $\mathbb{R}$.

Wir erinnern uns, dass die Fourier-Transformation für Funktionen auf der 
Gruppe $(\mathbb{R},+)$ definiert ist. 
Somit tritt bei der Fourier-Transformation der seltene Fall auf, dass die 
Gruppe der Transformation diesselbe ist, wie die duale Gruppe!
Aus diesem Grund sind die Transformationsintegrale diesselben, nämlich das 
normale Skalarprodukt der Gruppe $(\mathbb{R},+)$.
Was also vielleicht vorher als selbstverständliche Symmetrie angesehen 
wurde, ist in Tat und Wahrheit nicht immer so.
Will man nun die duale Gruppe der Mellin-Transformation bestimmen kommt 
es auf die Definition an. 

\subsubsection{Die duale Gruppe der Mellin-Transformation}
Die duale Gruppe der Mellin-Transformation aus Gleichung 
\eqref{mellin:mellinu} ist ebenfalls, wie die der Fourier-Transformation, 
die Gruppe $(\mathbb{R},+)$.
Die Analysefunktion, die vorher noch Potenzfunktionen waren, wandeln nun 
bei der Rücktransformation ihre Natur zu ebenjenigen Exponentialfunktion
der dualen Gruppe, da diese in dieser Richtung jetzt neu durch die Variable 
$u \in \mathbb{R}$ charakterisiert werden, über die ja auch integriert wird.
Wie in der Fourier-Theorie, erfolgt die Zerlegung einer Funktion also in 
komplexe Exponentialfunktion, oder eben komplexe Schwingungen, die mit 
einem Koeffizienten $\hat{f}(u)$ multipliziert werden, der aus der 
Hintransformation gewonnen wurde.
Der kleine Unterschied ist dabei nur, dass die Variable $x$ der Funktion 
als Basis eben jener Exponentialfunktionen vorkommt, und nicht im Exponent, 
was der Tatsache beipflichtet, dass die Gewinnung der Koeffizientenfunktion 
$\hat{f}(u)$, über Potenzfunktionen erfolgt ist.

Da wir nun verstehen, dass die duale Gruppe der Mellin-Transformation nach 
Definition \eqref{mellin:mellinu} die Gruppe $(\mathbb{R},+)$ ist, können 
wir auf gleiche Weise wie die Hintransformation konstruiert wurde, auch 
die Rücktransformation konstruieren.
Dieses Mal müssen wir aber nichts Neues herleiten, denn die 
Analysefunktionen bleiben diesselben, und das Skalarprodukt auf der Gruppe 
$(\mathbb{R},+)$ ist dasselbe wie bei der Fourier-Transformation, das 
normale Skalarprodukt aus \eqref{mellin:skalaprodukt}. 
Das Zusammensetzen dieser Bausteine sowie das Hinzumultiplizieren einer 
Skalierungskonstante liefert die {\em Inversionsformel der Mellin-Transformation}
\begin{equation}
    f(x) = 
    \frac{1}{2\pi} \int\limits_{-\infty}^{\infty} 
    x^{-ju} \hat{f}(u) \,{d}u
    \label{mellin:mellininvu}
\end{equation}

Gehen wir der Definition der Mellin-Transformation aus 
\eqref{mellin:mellin} nach, und lassen einen Realteil $\delta \neq 0$ für 
die Variable $z$ zu, was einer zusätzlichen Dämpfung oder Verstärkung der 
komplexen Exponentialfunktionen entspricht, ist die Menge der dualen 
Gruppe nun die gesamten komplexen Zahlen $\mathbb{C}$.
Weil dadurch die Variable des Mellin-Raums komplex ist, ist diese 
gewissermassen zweidimensional hinsichtlich des Real- und Imaginärteils. 
Weil durch die zusätzliche Dämpfung oder Verstärkung der komplexen 
Exponentialfunktionen, deren Betrag nicht länger $1$ ist, existiert die 
Transformation nur für bestimmte Werte von $\delta$, für welche das 
Transformationsintegral noch konvergiert.
Da bezüglich der Konvergenz des Transformationsintegrals nur bestimmte 
Werte von $\delta$ zulässig sind, bedeutet das, dass die 
Mellin-Transformation für komplexe $z$ in einem Streifen der komplexen 
Zahlenebene existiert, auch {\em Streifen der Analyzität}
\footnote{aus dem Englischen: strip of analyticity oder SOA} genannt.
\begin{figure}
    \centering
    \begin{tikzpicture}[scale=0.70,
        arrowdeco/.style={
            postaction={decorate},
            decoration={
                markings,
                mark=between positions 0 and 1 step #1 with {\arrow{latex}}
            }
        }
      ]
    \draw[->] (-1,0) -- (9,0) coordinate[label={right:$Re$}];
    \draw[->] (0,-4) -- (0,4) coordinate[label={$Im$}];
    \node at (7,4) [right, color=purple] {$\displaystyle z \text{-Ebene}$};
    \draw[<-,color=black,line width=1.0pt] (1.5,-2) -- (3.5,-2);
    \node at (3.35,-2) [right, color=black] {$\displaystyle \text{SOA}$};
    \draw[->,color=black,line width=1.0pt] (4.5,-2) -- (5.5,-2);
    \draw[arrowdeco=0.2, color=purple,line width=1.4pt] (3,-4) -- (3,4);
    \draw[color=black,line width=0.5pt] (1.5,-4) -- (1.5,4);
    \draw[color=black,line width=0.5pt] (5.5,-4) -- (5.5,4);
    \node at (3,-0.25) [right, color=black] {$\displaystyle \delta$};
    \end{tikzpicture}
    \caption{Integrationspfad der Rücktransformation in der $z$-Ebene
    \label{fig:mellin:z}}
\end{figure}
Dies erinnert stark an den Laplace-Raum, dessen Variable ebenfalls komplex 
ist und meist in einer Halbebene der komplexen Zahlenebene ab einem 
gewissen Wert $\delta_0$ für den Realteil existiert.
Bei der Rücktransformation muss allerdings der Realteil 
$\delta \in \text{SOA}$ fixiert werden. 
Integriert wird daraufhin über die Parallele der imaginäre Achse an der 
Stelle von $\delta$, wie dargestellt in Abbildung \ref{fig:mellin:z}.
Die Inversionsformel hier lautet also

\begin{equation}
    f(x) = 
    \frac{1}{2\pi j} 
    \int\limits_{\delta -j\infty}^{\delta +j\infty} 
    x^{-z} \hat{f}(z) \,{d}z
    \label{mellin:mellininv}
\end{equation}
Somit vervollständigt sich das Bild der Mellin-Transformation und fügt 
sich ein in bereits bekannte Integraltransformationen, wie Fourierreihen, 
die Fourier- und Laplacetransformationen sowie deren diskrete Varianten, 
die DFT und $z$-Transformation.




%
% teil3.tex -- Beispiel-File für Teil 3
%
% (c) 2020 Prof Dr Andreas Müller, Hochschule Rapperswil
%
% !TEX root = ../../buch.tex
% !TEX encoding = UTF-8
%
\section{Rekonstruktion eines CT-Scans
\label{ct:section:rekonstruktion}}
\rhead{Rekonstruktion eines CT-Scans}
Im letzten Abschnitt über die Rekonstruktion von CT-Bildern wird untersucht, wie Röntgendaten in klare und aussagekräftige Bilder umwandelt werden. Dabei werden die zwei wichtigen Theoreme das Zentralschnitt-Theorem und die gefilterte Rückprojektion nochmals angewendet. In diesem Abschnitt wird ein praktisches Beispiel für die Rekonstruktion von CT-Bildern verwendet, um diese Theoreme in der Praxis zu veranschaulichen. 

\subsection{Das Beispiel
\label{ct:subsection:malorum}}
Als einfaches Beispielbild werden zwei Kreise in der Ebene verwendet, siehe Abbildung \ref{ct:img_example}. So bleibt auch das radontransformierte Bild übersichtlich.

\begin{figure}
	\centering
	\includegraphics[width=0.5\linewidth]{papers/ct/images/img_recon.png}
	\caption{Das Beispielbild
		\label{ct:img_example}}
\end{figure}

Die Rückprojektion in der Praxis verwendet nur eine endliche Anzahl von Werten. Dies führt dazu, dass die Rückprojektion nicht mit einem Integral berechnet werden kann, sondern mit einer Summe approximiert werden muss. 
Die kontinuierliche Variable $\theta$ wird in der diskreten Umgebung der Rückprojektion mit der diskreten Variable $\theta_i$, die sich auf die Winkel ${k\pi/N: 0\le k \le N-1}$ beschränkt, ersetzt. Diese Diskretisierung wird mit 
\begin{equation}\label{ct:discreteBP}
		\mathscr{B}_Dh(x, y) = \big(\dfrac{1}{N}\big)\sum_{k=0}^{N-1} h(x\cos(k\pi/N)+Y\sin(k\pi/N), k\pi/N)
\end{equation}
beschrieben, wobei $k\pi/N$ mit $\theta_i$ zur Übersichtlichkeit ersetzt werden kann und somit resultiert
\begin{equation}\label{ct:discreteBP}
	\mathscr{B}_Dh(x, y) = \big(\dfrac{1}{N}\big)\sum_{k=0}^{N-1} h(x\cos(\theta_i)+y\sin(\theta_i), \theta_i).
\end{equation}

Das Bild, welches vom Scanner erhalten wird, wird in \ref{ct:img_radon} gezeigt.
\begin{figure}
	\centering
	\includegraphics[width=0.5\linewidth]{papers/ct/images/radon.png}
	\caption{Die Radon-Transformation des Beispielbildes.
		\label{ct:img_radon}}
\end{figure}

Auf das Bild vom Scanner wird zunächst die Rückprojektion angewendet. Dies führt dazu, dass das Bild eine \glqq geglättete\grqq{} Version des Beispielbildes wird. Dies wird in der Abbildung \ref{ct:img_bp} gezeigt.
\begin{figure}
	\centering
	\includegraphics[width=0.5\linewidth]{papers/ct/images/bp.png}
	\caption{Die Rücktransformation des Beispielbildes.
		\label{ct:img_bp}}
\end{figure}

Damit ein ungeglättetes Bild resultiert, wird die gefilterte Rückprojektion angewendet. Dabei wird als erstes, Abbildung \ref{ct:img_radon} Fourier-transformiert. Danach wird die Filterung mit $|S|$ durchgeführt und die inverse Fourier-Transformation wird gebildet. Schliesslich wird noch die Rückprojektion durchgeführt, was zur gefilterten Rückprojektion führt. 

\begin{figure}
	\centering
	\includegraphics[width=0.5\linewidth]{papers/ct/images/fbp.png}
	\caption{Die gefilterte Rücktransformation des Beispielbildes.
		\label{ct:img_fbp}}
\end{figure}

Im Bild kann man noch gewisse Artefakte sehen, die sich durch Linien im Bild zeigen. Diese Artefakte kommen daher, dass die Winkelauflösung begrenzt ist. In der Abbildung \ref{ct:img_resol} werden noch zwei Rekonstruktionen gezeigt, die einerseits eine tiefere Winkelauflösung und andererseits eine noch höhere Auflösung darstellen.

\begin{figure}
	\centering
	\begin{minipage}{0.45\linewidth}
		\includegraphics[width=\linewidth]{papers/ct/images/fbp_low.png}
	\end{minipage}
	\begin{minipage}{0.45\linewidth}
		\includegraphics[width=\linewidth]{papers/ct/images/fbp_high.png}
	\end{minipage}
	\caption{Die gefilterte Rücktransformation mit 10° (links) und 1° (rechts) Winkelauflösungen.
	\label{ct:img_resol}}
\end{figure}












\printbibliography[heading=subbibliography]
\end{refsection}
