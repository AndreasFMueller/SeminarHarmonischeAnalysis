%
% teil2.tex -- Beispiel-File für teil2 
%
% (c) 2020 Prof Dr Andreas Müller, Hochschule Rapperswil
%
% !TEX root = ../../buch.tex
% !TEX encoding = UTF-8
%
\section{Black-Scholes-Merton-Gleichung\label{brown:BlackScholes}}
\rhead{Black-Scholes-Merton}

Die Black-Scholes-Merton-Gleichung wurde von Robert Merton, Myron Scholes und Fischer Blacl entwickelt. Es dient zur Bewertung von Optionen. Es vereint beinahe schon wunderschön SDGLs zur Modellbildung, die diskrete Fourriertransformation zur Auswertung und den Wiener Prozess zur Modellbildung.

Damit das die Black-Scholes-Merton-Gleichung zur Anwendung kommen kann, müssen folgende Grundannahmen stimmen:
%https://finanzderivate.info/optionen/bewertung-von-optionen/das-black-scholes-modell-zur-optionsbewertung/
\begin{enumerate}
	\item Die Option ist Europäisch: 
	\item Es gibt keine Dividenden (Fussnote) oder sonstige Cashflows während der Laufzeit.
	\item Es gibt keine Transaktionskosten
	\item Die Erträge der Basiswerte sind normalverteilt.
	\item Der risikolose Zins ist bekannt und über die Laufzeit der Options hinweg bekannt.
	\item Die Volatilität (Schwankung des Preises) des Basiswertes ist bekannt und über die Laufzeit der Option hinweg konstant.
\end{enumerate}


\subsection{Theorie\label{brown:BlackScholes:theorie}}



\subsection{Modellbildung\label{rown:BlackScholes:modell}}
\ref{brown:section:teil2}.

Text
