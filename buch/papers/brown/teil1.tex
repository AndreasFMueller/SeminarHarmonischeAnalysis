%
% teil1.tex -- Beispiel-File für das Paper
%
% (c) 2020 Prof Dr Andreas Müller, Hochschule Rapperswil
%
% !TEX root = ../../buch.tex
% !TEX encoding = UTF-8
%
\section{Stochastische Differenzialgleichungen\label{brown:section:teil1}}
\rhead{Problemstellung}

Wird eine DGL (Differenzialgleichtung) hergeleitet, kann das Verhalten eines Systems unter Berücksichtigung der Anfangsbedingungen vorhergesagt werden. Vom Anfangswert aus /entwickelt sich/folgt/ die Funktion dem Vektorfeld. Die resultierende Funktion ist stetig und "smooth". 
//


Bild
//


In vielen Bereichen entspricht  ein solch "ideales" System jedoch nicht der Relaität. Kleine Einflüsse durch Störfunktionen und "Zufälle" können das Resultat massgeblich beeinflussen. 
Bild
//


Um diesem Umstand gerecht zu werden, kann man die Möglickeit von Zufälligen Störungen beim Aufstellen des Modells miteinbeziehen - et. voilà, man hat eine stochastische Differenzialgelichung.

Formal kann dies folgendermassen notiert werden, 

\begin{equation}
\dot{X}(t) = b(X(t)) + B(X(t))\xi(t) \quad (t>0)
\end{equation}\

wobei B eine n x m Matrize mit Faktoren ist, welche beschreiben, wie sich Störungen auf die unterschiedlichen ..... des Systems auswirken.

m ist die Dimension der Störfunktion. Alos m dimensionales "white noise".






\begin{equation}
	\int_a^b x^2\, dx
	=
	\left[ \frac13 x^3 \right]_a^b
	=
	\frac{b^3-a^3}3.
	\label{brown:equation1}
\end{equation}

Text

\subsection{De finibus bonorum et malorum
\label{brown:subsection:finibus}}

Text \eqref{brown:equation1}.

Text
\ref{brown:section:teil2}.

Text
\ref{brown:section:teil3}.



