%
% teil1.tex -- Brownische Bewegung
%
% (c) 2023 Lukas Reitemeier, OST Ostschweizer Fachhochschule
%
% !TEX root = ../../buch.tex
% !TEX encoding = UTF-8
%

\section{Brownische Bewegung\label{brown:BrownBewegung}}
\rhead{Brownische Bewegung}

Als der Schottische Botaniker Robert Brown im Jahr 1827 in sein Mikroskop schaut, beobachtet er kleine Pollen-Partikel in einer Flüssigkeit. Er bemerkt, dass sich die Teilchen scheinbar zufällig bewegen, obwohl keine Kräfte auf die Teilchen einwirken. Eine Erklärung hatte Robert Brown zu diesem Zeitpunkt für das Verhalten noch nicht.


Es dauerte fast ein Jahrhundert, bis Albert Einstein diese Beobachtung auf ein solides theoretisches Fundament stellte und so nutzbar machte. Er schloss darauf, dass die unregelmäßige Bewegung auf Kollisionen mit umgebenden Molekülen zurückzuführen ist, welche ständig in Bewegung sind. Albert Einstein entwickelte eine mathematische Theorie, mit welcher die beobachtete Bewegung durch stochastischen Zusammenstöße von Molekülen erklärt werden kann. Weiter konnte er auch einen Zusammenhang zwischen den Messungen der Bewegung, der Bolzmankonstante und der  Avogadrozahl herstellen.\footnote{Die Avogadrozahl ($N_\mathrm{A}$) ist die Anzahl Teilchen (Atome oder Moleküle) welche in einem Mol einer Substanz enthalten sind. Die Boltzmannkonstante ($k$) ist eine physikalische Konstante, die den Zusammenhang zwischen der thermischen Energie und der Temperatur eines Systems herstellt. Beide Konstanten sind miteinander verbunden durch die Gleichung $k = \frac{R}{N_\mathrm{A}}$, wobei $R$ die allgemeine Gaskonstante ist.}


Dies hatte auch Auswirkungen auf das Verständnis von Materie, da dies einen direkten empirisch Beleg für die Existenz von Atomen und Moleküle lieferte. Die sogenannte Atomhypothese besagt, dass Materie aus diskreten unteilbaren Einheiten besteht, sprich einzelnen Atomen oder ganzen Molekülen. 
Man muss dazu sagen, dass es schon zuvor Indizien für deren Existenz gab, doch es fehlte am entscheidenden experimentellen Beweis.
Die \textit{Einstein-Smoluchowski-Gleichung} ist eine zentrale Gleichung seiner Arbeit, welche die mittlere quadratische Verschiebung (MSD: \textit{Mean square displacement}) beschreibt
\begin{equation}
	\mathrm{MSD} = 2nDt
\end{equation}
in Beziehung zur Diffusionskonstanten $ D $ , der Anzahl Raumdimensionen $ n $ und der Zeit $ t $.

Die Avogadrozahl $ \mathrm{A} $ kann indirekt über die \textit{Stokes-Einstein-Gleichung}, anhand von Messungen berechnet werden. Die Diffusionskonstante $ D $ kann dabei mittels des Radius des suspendierten Teilchens, der Boltzmann-Konstante  $ k $, der Temperatur $ T $ und der Viskosität $ \eta $ des umgebenden Mediums, beschrieben werden:
\begin{equation}
	D = \frac{kT}{6\pi\eta r}
\end{equation}
$ r $ beschreibt dabei den Radius des suspendierten Teilchens. Die Boltzmannkonstante $ k $ kann auch durch die allgemeine Gaskonstante $ R $ a und die Avogadrozahl $ \mathrm{A} $ ausgedrückt werden: $ k = R/\mathrm{A} $

So ergibt sich folgender Zusammenhang:
\begin{equation}
	\mathrm{A} = \frac{R T}{D (6 \pi \eta r)}
\end{equation}

So ermöglichen die beiden Gleichungen einen experimentellen Ansatz zur Bestimmung der Avogadrozahl und ferner die quantitative Analyse der Brownischen Bewegung.


