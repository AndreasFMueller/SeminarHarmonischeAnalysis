%
% uebersicht.tex -- Uebersicht ueber die Seminar-Arbeiten
%
% (c) 2022 Prof Dr Andreas Mueller, OST Ostschweizer Fachhochschule
%
\chapter*{Übersicht}
\fancyhead[RE]{}
\fancyhead[LO]{Übersicht}
\label{buch:uebersicht}
Im zweiten Teil kommen die Teilnehmer des Seminars selbst zu Wort.
Die im ersten Teil dargelegten mathematischen Methoden und
grundlegenden Modelle werden dabei verfeinert, verallgemeinert
und auch numerisch überprüft.

Harmonische Analysis kann verwendet werden, um Tonsignale zu analysieren
und zu modifizieren.
{\em Alain Keller} erklärt im Paper Sonogramm
\index{Keller, Alain}%
\index{Sonogramm}%
(Kapitel~\ref{chapter:sonogramm}), wie man damit eine
visuelle Darstellung eines Musikstücks bekommen kann.
{\em Florian Baumgartner} erweitert diese Ideen dann auf eine
\index{Baumgartner, Florian}%
Methode zur Korrektur einzelner Töne.
Solche Systeme sind unter dem Namen Auto-Tune (Kapitel~\ref{chapter:autotune})
\index{Auto-Tune}%
bekannt und spielen
in der modernen Musikproduktion eine wichtige Rolle.

Harmonische Analysis kann auch verwendet werden, um verschiedene
Arten von Signalen zu unterscheiden.
Zum Beispiel kann man sich die Frage stellen, was überhaupt
``Rauschen'' ist und wie man es mit Hilfe der Frequenzanalyse
charakterisieren kann.
{\em Lukas Reitemeier} beschreibt in seinem Paper
\index{Reitemeier, Lukas}%
über die Brownsche Bewegung (Kapitel~\ref{chapter:brown})
\index{Brownsche Bewegung}%
ein Modell für Rauschen zeigt auch das Frequenzspektrum.

Die Fourier-Transformation kann auch auf Bilder angewendet werden,
was natürlich mit sehr viel grösserem Rechenaufwand als für eine
eindimensionales Signal möglich ist.
{\em Marco Niederberger} und {\em Yanick Schoch} zeigen in ihrem Paper
(Kapitel~\ref{chapter:opt}), 
\index{Schoch, Yanick}%
\index{Niederberger, Marco}%
wie das Beugungsphänomen auf eine optische Realisierung der
Fourier-Transformation führt.
Damit wird es möglich, die Fourier-Transformation ohne Rechnung in
besonders kurzer Zeit rein optisch durchzuführen, was die Autoren
in einem Versuch ebenfalls vorführen.

Die Fourier-Transformation kann auch gebraucht werden, um Bilder
zu komprimieren.
Wie das im Falle des JPEG-Formates funktioniert erklärt
\index{JPEG}%
{\em Jakob Gierer} in Kapitel~\ref{chapter:jpeg}.
\index{Gierer, Jakob}%
Er deutet auch an, wie JPEG-2000 die gleichen Ideen, allerdings mit
Wavelets statt harmonischen Schwingungen, für fortgeschrittene 
Komprimierung mit weniger Artefakten verwendet.
Mehr Information zur Wavelet-Transformation und insbesondere
zur Realisierung derselben mit dem Computer stellt {\em Stefan Richle}
\index{Richle, Stefan}%
in Kapitel~\ref{chapter:wavelets} dar.
\index{Wavelets}%

Ein Computer-Tomograph kann Querschnitt-Bilder eines Körpers
herstellen.
\index{Computer-Tomograph}%
Diese Fähigkeit macht ihn zu einem der bedeutendsten Bildgebenden
Verfahren in der Medizin.
{\em Nathan Hoffman} erklärt in seiner Arbeit (Kapitel~\ref{chapter:ct}),
\index{Hoffman, Nathan}%
wie ein Bild aus der
Radon-Transformation wiederhergestellt werden kann und zeigt auch
\index{Radon-Transformation}%
an Simulationen wie das praktisch funktioniert, wenngleich natürlich
je nach Auflösung der Messung mehr oder weniger störende Artefakte bleiben.

Die Gelfand-Theorie zeigt, das die wohlbekannte harmonische
\index{Gelfand-Theorie}%
Analysis mit harmonischen Schwingungen nur ein Fall von vielen
möglichen solchen harmonischen Analysetheorien ist.
{\em Vincent Haufe} beschreibt in Kapitel~\ref{chapter:mellin},
\index{Haufe, Vincent}%
wie man aus einer Aufgabe
zur Bestimmung der zeitlichen Skalierung eines Signals eine
neue Integraltransformation, die Mellin-Transformation, ableiten 
\index{Mellin-Transformation}%
kann.
Seine Darstellung weist auch auf Anwendungen in der Antennentheorie
hin.

Eine der ersten pratischen Anwendungen der harmonischen Analysis in
fast industriellem Massstab war die Analyse der Gezeiten durch
William Thomson.
Mit der Unterstützung seines Bruders konstruierte er eine Maschine zur
Bestimmung der Fourier-Koeffizienten aus langjährigen Aufzeichnungen
von Tidenständen.
Mit einer weiteren Maschine konnte er daraus Vorhersagen künftiger
Tidenstände für die Schiffahrt oder für die Planung von Küstenbauwerken
ableiten.
{\em David Bättig} erzählt die Geschichte dieser frühen aber besonders
wichtigen Anwendung in Kapiteln~\ref{chapter:gezeiten}.
\index{Bättig, David}%
\index{Gezeiten}%

Gezeiten sind nicht das einzige Phänomen auf der Erde, bei dem die
Verwendung der harmonischen Analysis einen Vorteil bei der Lösungsfindung
bringen kann.
{\em Dmitry Grigoriev} erklärt in Kapitel~\ref{chapter:spektral},
\index{Grigoriev, Dmitry}%
wie die spektralen Methoden funktionieren, die seit den achtziger Jahren
in der numerischen Wetterprognose verwendet werden und zu einer
merklichen Verbesserung der Prognosegenauigkeit geführt haben.

Über längere Zeiträume haben astronomische Phänomene einen periodischen
Einfluss auf die Parameter der Erdbahn.
Diese schlagen sich nieder in periodischen Veränderungen im Klima.
{\em Jan Langenegger} analysiert berechnete Bahndaten über mehrere
\index{Langenegger, Jan}%
Millionen Jahre mit einfachsten Mitteln und findet die Periodiztäten,
die Milankovi\'c vor über 100 Jahren bereits entdeckt hat.
Er untersucht in Kapitel~\ref{chapter:milankovic} auch, welchen Einfluss
auf die aktuelle Klimadiskussion diese Zyklen haben können.

Machine Learning und AI sind zur Zeit vor allem wegen ChatGPT und anderen
spektakulären Anwendungen in aller Munde.
\index{ChatGPT}%
Die dahinterstehende Technologie der künstlichen neuronalen Netzwerke
verspricht, für jedes beliebige Problem einen ``Approximator'' zu finden,
der darauf trainiert werden kann, jedes beliebige Problem zu lösen.
Es liegt daher nahe zu fragen, wie gut diese Technologie die
Fourier-Transformation lernen kann.
{\em Dominik Geschwind} beantwortet diese Frage in seiner Arbeit
\index{Gschwind, Dominik}%
(Kapitel~\ref{chapter:ml}).
Er untersucht auch die Frage nach der Genauigkeit und Performance mit der
AI die Fourier-Transformation berechnen kann, mit ernüchternden Resultaten.







