%
% 0_lineare_regression.tex
%
% (c) 2023 Dominik Gschwind, Hochschule Rapperswil
%
% !TEX root = ../../buch.tex
% !TEX encoding = UTF-8
%

\section{Lineare Regression\label{ml:regression}}
\rhead{Lineare Regression}

\emph{Regression} ist ein statistisches Analyseverfahren mit dem Ziel, Beziehungen zwischen
Daten zu modellieren. Hat man zum Beispiel den Aktienkurs einer Firma von den letzten zwanzig
Tagen, kann man mit Regression den Preis am nächsten Tag statistisch vorhersagen.
So ein Modell hat also das Ziel, die Vorhersage so gut wie möglich aus den gegebenen Daten zu approximieren.

Die einfachste Form der Regression ist die \emph{lineare Regression}, bei der alle Abhängigkeiten linear
sind.

\subsubsection{Ein lineares Modell}
\begin{itemize}
    \item $y$ soll die abhängige Variable des vorherzusagenden Wertes sein.
    \item $x_1, \dots, x_i$ sind unabhängigen Variablen die einen Beitrag zur Vorhersage
    $y$ leisten können, genannt \emph{Features}.
    \item $\theta_0, \theta_1, \dots, \theta_i$ sind die Koeffizienten der Features $x_i$,
    sie bestimmen wie wichtig das entsprechenden Feature ist.
\end{itemize}

Gegeben sind eine sehr grosse Anzahl $m$ Datenpunkte wobei ein Datenpunkt $j$ immer aus
den Feature-Werten $\{ x_0, x_1, \dots, x_n \}_j$ und des vorherzusagenden Wertes $y_j$ besteht.
Am besten ist $m \gg n$. Daten die $y$-Werte enthalten, werden auch \emph{labeled data} genannt,
im Gegensatz zu \emph{unlabeled data} wo nur die Feature-Werte bekannt sind.

Man kann nun das lineare Modell mit der Gleichung
\begin{equation}
y = \theta_0 + \theta_1 x_1 + \theta_2 x_2 + \dots + \theta_n x_n
\label{ml:regression:modell}
\end{equation}
aufstellen. Führt man die Variable $x_0 := 1$ ein, kann die Gleichung
\ref{ml:regression:modell} einfacher als
\begin{equation}
y = \sum_{i = 0}^{n} \theta_i x_i = \vec \theta \cdot \vec x =: h(\vec\theta, \vec x)
\label{ml:regression:hypothesis}
\end{equation}
geschrieben werden. Diese Beziehung wird auch \emph{hypothesis function} genannt und mit
$h$ bezeichnet.

Unser Ziel ist jetzt die Koeffizienten $\theta_i$ aus den Daten zu lernen. Hierfür
benötigen wir ein Kriterium, das uns die Distanz der Vorhersage $y_p = h(\vec x)$ zum tatsächlichen
Wert $y$ des Datenpunkts angibt. Es wird \emph{cost function} oder
\emph{loss function} genannt. Eine praktische Cost-Function ist der quadratische Fehler
\begin{equation}
J(\vec x, \vec \theta) = (y - y_p)^2.
\label{ml:regression:cost:sqerr}
\end{equation}

$J(\vec \theta)$ soll also minimiert werden, damit die Vorhersage $y_p$ möglichst nahe an den
tatsächlichen Wert $y$ kommt. Zwei allgemeine Methoden existieren um dieses
Minimierungsproblem zu lösen.

\subsection{Minimierung mit Matrix-Inversion}

Dieses Verfahren ist auch bekannt als ``Least-Squares für überbestimmte
Gleichungssysteme''. Es verwendet die Matrix-Inversion um mit \emph{einem} Schritt an die
Lösung zu kommen.

Zuerst wird das lineare Modell aus Gl. \eqref{ml:regression:hypothesis} mit allen
$m$ Datenpunkten in Matrix-Form als
\begin{equation}
    \begin{pmatrix}
        x_{00}& \cdots& x_{0i}& \cdots& x_{0n}\\
        \vdots& \vdots& \vdots& \vdots& \vdots\\
        x_{j0}& \cdots& x_{ji}& \cdots& x_{jn}\\
        \vdots& \vdots& \vdots& \vdots& \vdots\\
        x_{m0}& \cdots& x_{mi}& \cdots& x_{mn}
    \end{pmatrix}
    \cdot
    \begin{pmatrix}
        \theta_0\\ \theta_1 \\ \vdots\\ \theta_n
    \end{pmatrix}
    = \begin{pmatrix}
        y_{0,p}\\ \vdots \\ y_{j,p} \\ \vdots \\ y_{m,p}
    \end{pmatrix}
    \quad \iff \quad
    \mathbf{X} \vec \theta = \vec y_p
    \label{ml:regression:lstsq:modell}
\end{equation}
und die Cost-Function $J$ in Matrix-Form als
\begin{align}
    J(\vec \theta) &= \sum_{j=0}^m \left(y_j - (\theta_0 x_{j0} + \theta_1 x_{j1} + \cdots + \theta_n x_{jn}) \right)^2
    = \sum_{j=0}^{m} \left( y_j - \vec \theta \cdot \vec x_j \right)^2 \nonumber\\
    &= \left(\vec y - \mathbf{X} \vec \theta\right) \cdot \left(\vec y - \mathbf{X} \vec \theta\right)
    = \left(\vec y - \mathbf{X} \vec \theta\right)^\mathrm{T} \left(\vec y - \mathbf{X} \vec \theta\right)
    \label{ml:regression:lstsq:cost}
\end{align}
formuliert. In Gl. \eqref{ml:regression:lstsq:cost} wurde die Summation über die
einzelnen quadratischen Fehler $(y_j - y_{j,p})^2$ zuerst als Skalarprodukt und
anschliessend als Matrizen-Produkt (Zeilen-Vektor mal Spalten-Vektor) geschrieben.

Um das Minimum zu finden wird der Gradient von $J(\vec \theta)$ 
\begin{align}
    \operatorname{grad} J(\vec \theta) &=
    \begin{pmatrix} \partial J / \partial \theta_0 \\ \vdots \\ \partial J / \partial \theta_n \end{pmatrix}
    = \sum_{j = 0}^{m} 2 (y_j - \vec \theta \cdot \vec x_j) \cdot
    \begin{pmatrix} - x_{j0}\\ \vdots \\ - x_{jn} \end{pmatrix} \nonumber\\
    &= \sum_{j = 0}^{m} 2 \left( \vec \theta \cdot \vec x_j - y_j \right) \cdot \vec x_j
    = \sum_{j = 0}^{m} 2 \vec \theta \cdot \vec x_j\, \vec x_j - 2 y_j \vec x_j \nonumber\\
    &= 2 \mathbf{X}^\mathrm{T} \mathbf{X}\, \vec \theta - 2 \mathbf{X}\, \vec y
    \label{ml:regression:lstsq:grad}
\end{align}
auf den Null-Vektor $\vec 0$ gesetzt
und schliesslich nach $\vec \theta$ aufgelöst:
\begin{align}
    &2\mathbf{X}^\mathrm{T} \mathbf{X}\, \vec \theta - 2 \mathbf{X}\, \vec y \overset{!}= \vec 0 \quad\Leftrightarrow\quad
    \mathbf{X}^\mathrm{T} \mathbf{X}\, \vec \theta = \mathbf{X}\, \vec y \nonumber\\
    &\implies \vec \theta = \left( \mathbf{X}^\mathrm{T} \mathbf{X} \right)^{-1} \mathbf{X} \, \vec y.
    \label{ml:regression:lstsq:formel}
\end{align}

\eqref{ml:regression:lstsq:formel} gibt die Formel an um aus dem überbestimmten
Gleichungssystem von \eqref{ml:regression:lstsq:modell} direkt mittels Matrix-Inversion
die Koeffizienten $\vec \theta$ exakt zu berechnen.

\bigskip
Aber nicht alle solche Matrizen sind invertierbar, sind Zeilen oder Spalten linear
abhängig oder gibt es mehr Features als Datenpunkte, kann $\mathbf{X}^\mathrm{T}\mathbf{X}$
nicht invertiert werden.

Dazu kommt dass die Matrizen-Inversion (gleich wie das Matrizen-Produkt) mit dem besten
Algorithmus, etwa $\mathcal{O}(n^{\log_2 7})$, sehr aufwendig ist.\footnote{Mehr dazu ist
in \cite{ml:computational-complexity-math-op} und
\cite{ml:computational-complexity-matrix-mult} zu finden.} Die Inversion  einer (oder
Multiplikation zweier) $1000 \times 1000$-Matrix braucht also rund $1000^{\log_2 7}
\approx 260$ Millionen Multiplikationen.

Wir haben uns hier auf die exakte Lösung der linearen Regression beschränkt mit der
simplen Cost-Function aus \eqref{ml:regression:cost:sqerr}. Später wollen wir
aber auch eine Lösung eines höchst nichtlinearen Modells mit vielleicht sehr viel
komplexerer Cost-Function berechnen können. Mit diesem Ansatz existiert aber nur
sehr selten eine Lösung in geschlossener Form (als Formel).

Mit einem geschickten Algorithmus, der die Koeffizienten $\theta_i$ numerisch
approximiert können die oben genannten Problemen weitgehend umgangen werden. Wir möchten
diesen als nächstes anschauen.

\subsection{Minimierung mit Gradient-Descent \label{chapter:ml:regression:gd}}

Betrachtet man die Cost-Function $J(\vec \theta)$ aus \eqref{ml:regression:cost:sqerr}
ergibt sich deren partielle Ableitung nach $\theta_i$ als
\begin{equation}
    \frac{\partial J}{\partial \theta_i} = 2 (\underbrace{\operatorname{h}(\theta_0, \cdots, \theta_n, x_0, \cdots, x_n)}_{y_p} - y) \cdot x_i
    = 2 (h(\vec \theta, \vec x) - y) \cdot x_i.
\end{equation}
Es ist wieder der Gradient, ähnlich zu \eqref{ml:regression:lstsq:grad},
\begin{align}
    \operatorname{grad} J(\vec \theta)
    = 2 (\operatorname{h}(\vec\theta, \vec x) - y) \cdot \begin{pmatrix} x_0\\ \vdots \\ x_n \end{pmatrix}
    = 2 (\operatorname{h}(\vec\theta, \vec x) - y) \cdot \vec x.
\end{align}

Eine zentrale Eigenschaften vom Gradienten ist, dass er relativ zum Auswertungspunkt immer
in die Richtung des steilsten Anstiegs zeigt. Läuft man also immer entgegengesetzt des
Gradienten und passt die $\theta_i$ entsprechend an, gelangt man nach einer gewissen
Anzahl Iterationen zu einem lokalen Minimum. Dieses Verfahren wird \emph{gradient descent}
oder deutsch \emph{Gradientenverfahren} genannt.

Konkret kann die Idee als
\begin{align}
    \vec \theta_{k+1} &= \vec \theta_{\sf k}
        - \alpha \cdot \frac{1}{2m}\sum_{j=1}^{m} \operatorname{grad} J(\vec \theta_k, \vec x_j) \nonumber\\
    &= \vec \theta_k
        - \alpha \frac{1}{m} \sum_{j=1}^{m} (\operatorname{h}(\vec \theta_k, \vec x_j) - y_j)\cdot \vec x_j
    \label{ml:regression:gd:update}
\end{align}
geschrieben werden. Der Gradient wird als Durchschnitt über alle
Datenpunkte berechnet. Mit dem zusätzlichen Parameter $\alpha > 0$ kann die Geschwindigkeit der
Lösungsfindung eingestellt werden, er wird \emph{learning rate} genannt.

Als Startwert $\vec\theta_0$ kann beim linearen Modell irgendein Wert verwendet werden, meistens wird dieser jedoch
auf den Nullvektor $\vec 0$ gesetzt.

$J(\vec \theta)$ ist eine quadratische Funktion von $\theta$ mit einem einzigen lokalen
Minimum, %TODO: vielleicht Herleitung
dieses Minimum ist also auch gerade das globale Minimum. Man kann sich mit einem linearen
Modell sicher sein, je näher der Fehler $J$ bei $0$ ist, desto näher sind die berechneten
$\theta_i$ bei den tatsächlichen $\theta_i$.

Ob oder wie schnell die Berechnung zur Lösung konvergiert ist aber von $\alpha$ abhängig.
Ein grosses $\alpha$ gibt eine schnelle Konvergenz aber auch ein grosses Potential zum
Überschiessen oder sogar zur Divergenz, bei einem kleinen $\alpha$ dauert die Lösungssuche
aber umso länger. Bei Modellen mit grossen Anzahl $n$ Features und für nichtlineare
Modelle gibt es leider keine gute oder überhaupt keine Methode um den besten Wert von
$\alpha$ zu finden. Er muss empirisch ermittelt werden, wobei am besten Werte im Bereich
$0.001$ bis $1$ auszuprobieren sind. \cite[S. 51]{ml:introduction-to-ml}

\subsubsection{Stochastischer Gradient-Descent}

Der bis jetzt beschriebene Gradient-Descent-Algorithmus berechnet einen Schritt immer über
\emph{alle} Daten, was viel Rechenleistung benötigt und schlecht parallelisiert werden
kann.

Nimmt man für jeden Berechnungsschritt eine zufällig ausgewählte, kleinere Stickprobe von
$b \ll m$ Datenpunkten, kann die Performance stark verbessert werden. $b$ wird
\emph{batch size} genannt. Die Aktualisierungsgleichung \eqref{ml:regression:gd:update} wird

\begin{equation}
  \Delta \vec\theta_{k+1} = - \alpha \frac{1}{2b} \sum_{j=1}^{b} \operatorname{grad} J(\vec
  \theta_k, \vec x_j).
\end{equation}

In Fällen mit mehreren lokalen Minima kann stochastischer gegenüber dem totalen Gradient-Descent manchmal
verhindern, in diese lokalen Minima zu fallen. Je kleiner die Learning-Rate $\alpha$ gewählt
wird, desto besser wird totaler Gradient-Descent approximiert. 
\cite[S. 93ff]{ml:ml-tom-mitchell}

\subsection{Normalisierung}

Bei der Normalisierung werden alle Feature-Werte auf einen ähnlichen Bereich skaliert.
\begin{equation}
    \hat x_{ij} = \frac{2x_{ij} - (\max x_i + \min x_i)}{\max x_i - \min x_i}
\end{equation}
für ein Feature $i$ mit Daten $j$ skaliert die Feature-Werte auf den Bereich zwischen $-1$
und $1$. Es sind auch andere Methoden möglich.

Sind alle Werte in der gleichen Grössenordnung, konvergiert der
Gradient-Descent-Algorithmus viel schneller.
