%
% 0_einleitung.tex
%
% (c) 2023 Dominik Gschwind, Hochschule Rapperswil
%
% !TEX root = ../../buch.tex
% !TEX encoding = UTF-8
%
\section{Einleitung\label{ml:einleitung}}
\rhead{Einleitung}

Wir wollen untersuchen ob ein künstliches neuronales Netzwerk in der Lage ist die Zerlegung beliebiger
Daten in Fourier-Koeffizienten zu lernen. Dazu wollen wir die algorithmische
Komplexität untersuchen und mit der schnellen Fouriertransformation vergleichen.

\paragraph{Was ist ein künstliches neuronales Netzwerk?} \emph{Künstliche neuronale
Netzwerke}\footnote{Kürzer auch nur als neuronales Netz bezeichnet mit der deutschen
Abkürzung KNN.} (englisch \emph{artificial neural networks}, ANN\footnote{Es wird in
diesem Paper nur der englische Begriff und dessen Abkürzung verwendet.}) werden vor allem im Gebiet der
künstlichen Intelligenz und \emph{data science} eingesetzt. Ihre Struktur und Funktionsweise ist den
biologischen neuronalen Netzwerken zum Beispiel im Gehirn nachempfunden. Prinzipiell sind
sie universelle Approximatoren\footnote{Ein Beweis dafür ist in
\cite{ml:universala-approximator-theorem} zu finden.}, das heisst sie können beliebige
Sachverhalte beliebig genau approximieren (natürlich abhängig von Struktur, Rechenaufwand,
vorhandenen Daten und weiterem).

Diese Eigenschaft ist vor allem bei künstlicher Intelligenz sehr wichtig,
da nur schon die Definition von ``Intelligenz'' noch immer ein offenes Problem in der
Philosophie ist. Andere, traditionelle Methoden wie zum Beispiel \emph{expert systems} können
nur begrenzt und mit viel Aufwand Intelligenz darstellen.

ANNs werde aber auch vor allem dort eingesetzt, wo es keine oder nur sehr wenig erforschte
Theorie und Lösungsmethoden für eine Problemstellung gibt, oder auch Schwierigkeit und
Komplexität des Problems, dessen Lösung verunmöglichen. Einige Beispiele sind
Bilderkennung, Klassifizierung, Spracherkennung oder maschinelle Übersetzung.

\medskip
In den nächsten zwei Abschnitten wollen wir diesen universellen Approximator konstruieren.
Wir fangen mit einem linearen Modell an, welches uns schliesslich zum allgemein
nichtlinearen \emph{fully connected feedforward} ANN bringt.
