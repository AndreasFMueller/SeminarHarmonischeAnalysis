%
% tikztemplate.tex -- template for standalon tikz images
%
% (c) 2021 Prof Dr Andreas Müller, OST Ostschweizer Fachhochschule
%
\documentclass[tikz]{standalone}
\usepackage{amsmath}
\usepackage{times}
\usepackage{txfonts}
\usepackage{pgfplots}
\usepackage{csvsimple}
\usepackage{ifthen}
\usetikzlibrary{arrows,intersections,math}
\begin{document}
\def\skala{1.5}
\newboolean{draft}
\setboolean{draft}{false}


% Input layer neurons'number
\newcommand{\inputnum}{4} 
% Hidden layer neurons'number
\newcommand{\hiddennum}{4}  
% Output layer neurons'number
\newcommand{\outputnum}{4} 

\tikzset{%
  every neuron/.style={
    circle,
    draw,
    minimum size=7 mm
  },
  neuron missing/.style={
    draw=none, 
    scale=3,
    text height=0.333cm,
    yshift=0.5 mm,
    execute at begin node=\color{black}$\vdots$
  },
}

% \begin{tikzpicture}[x=1.5cm, y=1.5cm, >=stealth]
\begin{tikzpicture}[>=latex,thick,scale=\skala]

\foreach \m/\l [count=\y] in {1,2,missing,3}
  \node [every neuron/.try, neuron \m/.try] (input-\m) at (0,2.5-\y) {};

\foreach \m [count=\y] in {1,2,missing,3}
  \node [every neuron/.try, neuron \m/.try ] (hidden-\m) at (2,2.5-\y) {};
\foreach \m [count=\y] in {1,2,missing,3}
  \node [every neuron/.try, neuron \m/.try ] (hidden1-\m) at (3.5,2.5-\y) {};

\foreach \m [count=\y] in {1,2,missing,3}
  \node [every neuron/.try, neuron \m/.try ] (output-\m) at (5.5,2.5-\y) {};

\foreach \l [count=\i] in {1,2,n}{
   \draw [<-] (input-\i) -- ++(-0.5,0);
   \node [] at (input-\i) {\small$x_\l$};
}

\foreach \l [count=\i] in {1,2,n}
  \node [] at (hidden-\i) {\small$h_\l^{(1)}$};
\foreach \l [count=\i] in {1,2,n}
  \node [] at (hidden1-\i) {\small$h_\l^{(2)}$};

\foreach \l [count=\i] in {1,2,n}{
  \draw [->] (output-\i) -- ++(0.5,0);
  \node [] at (output-\i) {$y_\l$};
}

\foreach \i in {1,...,3}
  \foreach \j in {1,...,3}
    \draw [->] (input-\i) -- (hidden-\j) node (conn-ih-\i\j) {};

\foreach \i in {1,...,3}
  \foreach \j in {1,...,3}
    \draw [->] (hidden-\i) -- (hidden1-\j) node (conn-hh-\i\j) {};

\foreach \i in {1,...,3}
  \foreach \j in {1,...,3}
    \draw [->] (hidden1-\i) -- (output-\j) node (conn-ho-\i\j) {};

\node [align=center,above,xshift=-1.5 cm] at (conn-ih-11) {${\boldsymbol \thetaup}^{(1)}$};
\node [align=center,above,xshift=-1 cm] at (conn-hh-11) {${\boldsymbol \thetaup}^{(2)}$};
\node [align=center,above,xshift=-1.5 cm] at (conn-ho-11) {${\boldsymbol \thetaup}^{(3)}$};

\node [align=center, above, yshift=0.5 cm] at (input-1) {Input layer};
\node [align=center, above, yshift=0.5 cm,xshift=1 cm] at (hidden-1) {Hidden layers};
\node [align=center, above, yshift=0.5 cm] at (output-1) {Output layer};

\end{tikzpicture}
\end{document}
