%
% tikztemplate.tex -- template for standalon tikz images
%
% (c) 2021 Prof Dr Andreas Müller, OST Ostschweizer Fachhochschule
%
\documentclass[tikz]{standalone}
\usepackage{amsmath}
\usepackage{times}
\usepackage{txfonts}
\usepackage{pgfplots}
\usepackage{csvsimple}
\usepackage{ifthen}
\usetikzlibrary{arrows,intersections,math}
\begin{document}
\def\skala{1.5}
\newboolean{draft}
\setboolean{draft}{false}

\tikzset{%
  every neuron/.style={
    circle,
    draw,
    minimum size=7 mm
  },
  neuron missing/.style={
    draw=none, 
    scale=3,
    text height=0.333cm,
    yshift=0.14 mm,
    execute at begin node=\color{black}$\vdots$
  },
}

\begin{tikzpicture}[>=latex,thick,scale=\skala]

\foreach \m/\l [count=\y] in {1,2,3, missing,4}
  \node [every neuron/.try, neuron \m/.try] (input-\m) at (0,2.5-\y*0.75) {};

\foreach \m [count=\y] in {1,2,3,missing,4}
  \node [every neuron/.try, neuron \m/.try ] (output-\m) at (2,2.5-\y*0.75) {};

\foreach \l [count=\i] in {0,1,n,{N-1}}{
   \draw [<-] (input-\i) -- ++(-0.5,0);
   \node [] at (input-\i) {\small$x_{\l}$};
}

\foreach \l [count=\i] in {0,1,k,{N-1}}{
  \draw [->] (output-\i) -- ++(0.5,0);
  \node [] at (output-\i) {$a_{\l}$};
}

\foreach \i in {1,...,4}
  \foreach \j in {1,...,4}
    \draw [->] (input-\i) -- (output-\j) node (conn-io-\i\j) {};
\node [align=center,above,xshift=-1.5 cm] at (conn-io-11) {$\frac{2}{N}\cos(\omega_k n)$};
\end{tikzpicture}
\end{document}