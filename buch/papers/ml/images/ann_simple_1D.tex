%
% tikztemplate.tex -- template for standalon tikz images
%
% (c) 2021 Prof Dr Andreas Müller, OST Ostschweizer Fachhochschule
%
\documentclass[tikz]{standalone}
\usepackage{amsmath}
\usepackage{times}
\usepackage{txfonts}
\usepackage{pgfplots}
\usepackage{csvsimple}
\usepackage{ifthen}
\usetikzlibrary{arrows,intersections,math}
\begin{document}
\def\skala{1.5}
\newboolean{draft}
\setboolean{draft}{false}

\tikzset{%
  every neuron/.style={
    circle,
    draw,
    minimum size=7 mm
  },
  neuron missing/.style={
    draw=none, 
    scale=3,
    text height=0.333cm,
    yshift=0.5 mm,
    execute at begin node=\color{black}$\vdots$
  },
}

% \begin{tikzpicture}[x=1.5cm, y=1.5cm, >=stealth]
\begin{tikzpicture}[>=latex,thick,scale=\skala]

\node [] (input-1) at (0,2.5-0.75) {};
\node [every neuron/.try] (hidden-1) at (1.5,2.5-0.75) {};
\node [every neuron/.try] (hidden-2) at (3,2.5-0.75) {};
\node [every neuron/.try] (hidden-3) at (4.5,2.5-0.75) {};
\node [] (hidden-ell) at (5,2.5-0.75) {$\cdots$};
\node [every neuron/.try] (hidden-4) at (5.5,2.5-0.75) {};
\node [every neuron/.try] (output-1) at (7,2.5-0.75) {};

\node [xshift=-1mm] at (input-1) {\small$x$};
\draw [->] (output-1) -- ++(0.5,0) node (conn-oup) {};
\draw [<-] (hidden-1) -- ++(0,-0.5) node [yshift=-2mm] {$b_1$};
\draw [<-] (hidden-2) -- ++(0,-0.5) node [yshift=-2mm] {$b_2$};
\draw [<-] (hidden-3) -- ++(0,-0.5) node [yshift=-2mm] {$b_3$};
\draw [<-] (hidden-4) -- ++(0,-0.5) node [yshift=-2mm] {$b_{n-1}$};
\draw [<-] (output-1) -- ++(0,-0.5) node [yshift=-2mm] {$b_n$};
\node [] at (output-1) {\small$l_n$};
\node [] at (hidden-4) {\small$l_{n-1}$};
\node [] at (hidden-3) {\small$l_{3}$};
\node [] at (hidden-2) {\small$l_{2}$};
\node [] at (hidden-1) {\small$l_{1}$};

\draw [->] (input-1) -- (hidden-1) node (conn-ih) {};
\draw [->] (hidden-1) -- (hidden-2) node (conn-hh1) {};
\draw [->] (hidden-2) -- (hidden-3) node (conn-hh2) {};
\draw [->] (hidden-4) -- (output-1) node (conn-ho) {};
\node [xshift=2 mm] at (conn-oup) {\small $y$};

\node [align=center,above,xshift=-1.25 cm] at (conn-ih) {$\theta_1$};
\node [align=center,above,xshift=-1.25 cm] at (conn-hh1) {$\theta_2$};
\node [align=center,above,xshift=-1.25 cm] at (conn-hh2) {$\theta_3$};
\node [align=center,above,xshift=-1.25 cm] at (conn-ho) {$\theta_n$};

\end{tikzpicture}
\end{document}
