\section{Diskrete Fouriertransformation mit ANN\label{ml:dft-with-ann}}
\rhead{DFT mit ANN}

Die diskrete Fouriertransformation einer Zahlenreihe $x_n \in \{ x_0, x_1, \cdots, x_{N-1}\}$
mit $N$ Elementen kann mit
\begin{equation}
    c_k = \frac{1}{N} \sum_{n=0}^{N-1} x_n e^{\normalsize -jk\frac{2\pi}{N}n}
\end{equation}
berechnet werden, oder in vektorschreibweise
    \begin{equation}
    c_k = \begin{pmatrix}
        x_0\\
        x_1\\
        \vdots\\
        x_{N-1}
    \end{pmatrix} \cdot
    \frac{1}{N} \begin{pmatrix}
        e^{-j \omega_k 0} \\
        e^{-j \omega_k 1} \\
        \vdots \\
        e^{-j \omega_k (N-1)} \\
    \end{pmatrix}
    \qquad \textsf{mit}\qquad \omega_k = \frac{2\pi k}{N}.
    \label{ml:dft-with-ann:dft:vector}
\end{equation}
Die einzelnen $x_n$ werden mit einem komplexen Koeffizienten gewichtet und anschliessend
summiert. In der Vektorschreibweise wird dieser Effekt mit dem Skalarprodukt erzielt. Man
erkennt, dass dies ein linearer Zusammenhang ist.

Es kann sehr einfach ein entsprechendes neuronales Netzwerk konstruiert werden, das \eqref{ml:dft-with-ann:dft:vector} abbildet: