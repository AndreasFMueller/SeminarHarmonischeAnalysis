%
% einleitung.tex -- Beispiel-File für die Einleitung
%
% (c) 2020 Prof Dr Andreas Müller, Hochschule Rapperswil
%
% !TEX root = ../../buch.tex
% !TEX encoding = UTF-8
%
\section{Vorwort\label{wavelets:section:teil0}}
\rhead{Teil 0}
Das Ziel dieser Arbeit war kein klares Ziel zu haben und zu schauen, wo hin es einen trägt. Natürlich war es notwendig sich einen Rahmen zu stecken, d.~h.~ eine grobe Wunschvorstellung festzulegen. Aber tatsächlich war der Weg in dieser Arbeit das Ziel.
Die Wunschvorstellung war ein tieferes Verständnis über die Wavelettransformation zu erhalten. Hier muss angemerkt sein, dass es grundsätzlich zwei Oberbegriffe im Zusammenhang mit Wavelets gibt.
Es kann einerseits eine diskrete und andererseits eine kontinuierliche Wavelettransformation gemacht werden.

\begin{itemize}
	\item   Die diskrete Wavelet Transformation (DWT) arbeitet auf Basis einer Filterbank, dient also dazu Signale direkt zu Filtern z.B. rauschbelastete Signale zu Entrauschen.
	Die Funktionsweise kann man einfach als eine wiederholte Aufsplittung in einen Tief- und Hochpass gefilterten Signalteil beschreiben. Wobei der Tiefpass jeweils den Durchschnitt über die Filterbreite und der Hochpass die Änderung über die gefilterte Signalbreite ausgibt (man spricht oft auch von der Detaillierung). Für eine Entrauschung wird also eine Filterbank aufgebaut die gerade eine so hohe Detaillierung besitzt, dass die Rauschanteile eliminiert werden (Abbildung \ref{wavelet:fig:1_Four-Level-Wavelet-Decomposition}).

 	\item Die kontinuierliche Wavelettransformation (CWT) ist die Transformation des Zeitsignals in den Frequenzbereich. Der Ablauf ist relativ analog zur diskreten Fouriertransformation (DFT), wo die Unterschiede liegen und was die Vor- sowie die Nachteile der CWT gegenüber der schnellen Fouriertransformation (FFT) sind, ist eines der Ziele dieser Arbeit. 
\end{itemize}

\begin{figure}
	\centering
	\includegraphics[width=0.75\textwidth]{papers/wavelets/images/1_Four-Level-Wavelet-Decomposition.png}
	\caption{\cite{Haider.2015} Funktionsweise der diskreten Wavelettransformation (DWT).}
	\label{wavelet:fig:1_Four-Level-Wavelet-Decomposition}
\end{figure}

Die Untersuchung beschränkt sich aus zeitlichen Gründen auf die CWT. Natürlich könnte man sich nun fertigen Tools bedienen und eine Vielzahl von Spielereien damit durchführen. Das Ziel ist aber nicht breit gefächert alle möglichen Tools anzuwenden, vielmehr soll Verständnis für eine angewandte Transformation aufgebaut werden. Aus diesem Grund wird versucht, selbst ein lauffähiges CWT-Programm zur Signalanalyse zu schreiben. 