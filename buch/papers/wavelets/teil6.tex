%
% teil6.tex -- Beispiel-File für Teil 6
%
% (c) 2020 Prof Dr Andreas Müller, Hochschule Rapperswil
%
% !TEX root = ../../buch.tex
% !TEX encoding = UTF-8
%
\section{Fazit zur CWT mit dem Morlet-Wavelet
\label{wavelets:section:teil6}}
\kopfrechts{Fazit zur CWT mit dem Morlet-Wavelet}

Das Ziel, ein lauffähiges CWT-Programm zu schreiben und damit einen
Vergleich zwischen FFT und CWT machen zu können, wurde erreicht.
Ein grosser Teil der Arbeit wurde in die Rechenzeitminimierung
investiert.
Der Grund lag darin, dass bereits bei Abtastraten im Bereich von
$2^{10}$ sehr hohe Rechenzeiten entstanden.

Durch den zeitlichen Aufwand zur Lösung des Rechenzeitproblems
konnten keine weiterführenden Untersuchungen an verschiedenen
Signaltypen mehr gemacht werden.
Als positiver Nebeneffekt, wurde das Verständnis für das Morlet-Wavelet
verbessert und auch ein effektives Gefühl für den mathematischen
Mehraufwand durch die zusätzliche freie Variable erzeugt.
Zudem war es fast schon sinnbildlich, dass der Einsatz des direkten
Skalarproduktes des Wavelet- und Signal-Arrays eine deutliche
Beschleunigung des Programms mit sich brachte.

Abschliessend kann folgendes statuiert werden:

\begin{itemize}
	\item Die CWT ist ein Analyse-Tool und die Umkehrung ist kaum relevant.
	\item Die DWT ist ein Signalverabeitungstool, die Umkehrung
	ist von zentraler Bedeutung.
\end{itemize}
