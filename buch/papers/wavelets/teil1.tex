%
% teil1.tex -- Beispiel-File für das Paper
%
% (c) 2020 Prof Dr Andreas Müller, Hochschule Rapperswil
%
% !TEX root = ../../buch.tex
% !TEX encoding = UTF-8
%
\section{Mathematische Grundlagen
\label{wavelets:section:teil1}}
\kopfrechts{Problemstellung}

\subsection{Diskrete Fourier-Transformation (DFT)
\label{wavelets:subsection:DFT}}
Um die CWT richtig nachvollziehen zu können, lohnt es sich, sich
vorab mit dem Modell der diskreten Fourier-Transformation (DFT)
\cite{wavelets:Hayes.1996}

\begin{equation}
	X(k)=\frac{1}{N}\sum_{n=0}^{N-1}x(n)\cdot e^{-j\frac{2\pi}{N}\cdot k\cdot n}
	\label{wavelets:equation1}
\end{equation}
auseinanderzusetzen.
Hierbei beschreibt $X(k)$  das Fourier-transformierte Signal in
Abhängigkeit von der Frequenz $k$.
Die Funktionsweise lässt sich aus der Gleichung direkt ableiten.
Ein Signal $x(n)$ wird an jeder diskreten Stelle $x(n)$, wobei $n$
die ganze Zahlen von $0$ bis $N-1$ durchläuft, mit der Exponentialfunktion
\[e^{-j\frac{2\pi}{N}\cdot k\cdot n}\] verrechnet.

Die Exponentialfunktion kann auch mit

\begin{equation}
	\cos \left( \frac{2\pi}{N}kn \right) -j\cdot \sin \left( \frac{2\pi}{N}kn \right)
	\label{wavelets:equation2}
\end{equation} 
beschrieben werden.
Das Minuszeichen bestimmt nur den Umlaufsinn des komplexen
Einheitsvektors im Einheitskreis.
Es wird also ein Signal an jeder Stelle $x(n)$ mit der komplexen
Exponentialfunktion für die betrachtete Frequenz $k$ verrechnet und
das über alle Punkte $x(n)$ für $n=0,1,2,...N$.
Die resultierende Summe wird dann über die Anzahl verrechneter
Punkte geteilt, was dem arithmetischen Mittelwert entspricht.
\begin{figure}
	\centering
	\includegraphics[width=0.49\textwidth]{papers/wavelets/images/2_DFT1.png}
	\includegraphics[width=0.49\textwidth]{papers/wavelets/images/2_DFT2.png}
	\caption{\cite{wavelets:AndreasMuller.2019} Darstellung zur Funktionsweise der DFT.
In der linken Hälfte mit einer hohen Deckung zwischen Signal (blau)
und der gesuchten Sinusfrequenz $k$ (rot).
In der rechten Hälfte ein zufallsverteiltes Signal mit schlechter
Überdeckung.
Die Deckung wird nebenan in einem $+$ und $-$ Verrechnungsplot, der
aus vier Quadranten besteht, ausgegeben.}
	\label{wavelet:fig:2_DFT1&2}
\end{figure}%
\begin{figure}
	\centering
	\includegraphics[width=0.75\textwidth]{papers/wavelets/images/3_AmplitudengangExtraktionDFT.png}
	\caption{\cite{wavelets:QingkaiKong} Erzeugung des
	Amplitudenganges bei der DFT.
	Extraktion der vorhandenen Sinusanteile (blau) aus dem
	Zeitsignal (rot).}
	\label{wavelet:fig:AmplitudengangExtraktionDFT}
\end{figure}%

Anstelle der Funktion \eqref{wavelets:equation2} kann man sich im
2D vereinfacht die Verrechnung des Wertes $x(n)$ an der Position
$n$ mit dem $\sin(\frac{2\pi}{N}kn+\phi)$ vorstellen, wobei $\phi$
die Phasenverschiebung des Sinus beschreibt.
Das $\phi$  wird in der Formulierung \eqref{wavelets:equation2} mit
dem Cosinusanteil miteinbezogen.
Wenn nun das Signal den Sinus mit der Frequenz $k$ enthält wie in
Abbildung \ref{wavelet:fig:2_DFT1&2}, kommt es zu einer Überdeckung
und entsprechend hoher Aufsummierung.
Das kann in der linken Hälfte der Abbildung \ref{wavelet:fig:2_DFT1&2}
an den Verrechnungspunkten, die alle in den positiven Quadranten
liegen, nachvollzogen werden.
In diesem Fall kommt es zu einem Amplitudenausschlag auf der
entsprechenden Frequenz.
In Abbildung \ref{wavelet:fig:AmplitudengangExtraktionDFT} ist das
schematisch für ein Signal (rot) und die enthaltenen Sinusanteile
(blau) dargestellt.
Die Frequenzdomäne entspricht dabei dem Bildbereich des transformierten
Zeitsignals (rot).
Im Gegensatz dazu werden andere Frequenzanteile oder zufallsverteilte
Signalpunkte durch den ständigen Wechsel von plus und minus in der
Summe gegen Null gehen, was in der rechten Hälfte von Abbildung
\ref{wavelet:fig:2_DFT1&2} eher der Fall ist.
Die Verrechnungspunkte liegen nun verstreuter in den plus und minus
Quadranten.
Mit diesem Prinzip kann die DFT die Frequenzanteile ermitteln
(Abbildung \ref{wavelet:fig:2_DFT1&2} und
\ref{wavelet:fig:AmplitudengangExtraktionDFT}).

An der Stelle soll noch der die Verbindung zwischen Frequenzspektrum
(Autospektrum) und Frequenzgang (englisch Frequency Response Function
FRF) hergestellt werden.

\begin{itemize}
	\item Beim Autospektrum handelt es sich lediglich um die
	Fourier-Transformierte eines Signales, meistens des
	Ausgangssignals.
	\item Der Frequenzgang ist die Übertragungsfunktion zwischen
	Eingang und Ausgang.
	Dabei wird der Quotient aus der
	Fourier-Transformierten des Ausgang $Y$ über dem ebenfalls
	transformierten Eingang $X$,
	\[
	\frac{\text{FFT}(Y)}{\text{FFT}(X)} = \text{FRF},
	\]
	gebildet.
	Dieses Verhältnis von Antwort zu Erregung lässt eine
	eindeutige Bestimmung der Resonanzen und des Phasenverlaufes
	zu.
\end{itemize}

\begin{figure}
	\centering
	\includegraphics[width=\textwidth]{papers/wavelets/images/4_FRF_iso.png}
	\caption{3D-Darstellung eines Frequenzganges aus einem
	gemessenen Eingang- (Shakersignal) sowie Antwortsignal
	(Piezo-Sensor an einer Struktur).
	Die Achsen in vertikaler
	und in Tiefenrichtung beschreiben die über den Eingang (Kraft)
	normierten Beschleunigungen $\frac{m/s^2}{N}$.
	Auf der horizontalen Achse ist die Frequenz aufgetragen.
	Die Tiefenrichtung steht für die reelle Ebene (Realteil des
	komplexen Frequenzganges) und die Vertikale für die imaginäre
	Ebene (Imaginärteil des komplexen Frequenzganges).}
	\label{wavelet:fig:FRF_iso}
\end{figure}

Die 3D-Abbildung\ref{wavelet:fig:FRF_iso} zeigt einen FRF und die
Projektionen in seine Bestandteile.

\begin{itemize}
	\item Der Realteil entspricht dem auf die reelle Ebene
	abgebildete Projektion des FRF. Der Realteil geht in der
	Resonanz auf Null, d.~h.~alle Energie liegt im Imaginärteil.
	Ausserdem kann die Resonanz am vorhanden Phasensprung im
	Realteil abgelesen werden.
	\item Der Imaginärteil des FRF ist analog die Projektion
	auf die imaginäre Ebene , hier lassen sich eindeutig die
	Resonanzfrequenzen auslesen.
	Das System ist in Resonanz,
	zur Phase der Anregung, um 90° phasenverschoben.
	\item Die Ortskurve ist die auf den Einheitskreis abgebildete
	Projektion des FRF.
	Anhand der Ortskurve kann der Verlauf
	der Phase nachvollzogen werden.
\end{itemize}

\subsection{Kontinuierliche Wavelettransformation für das Morlet-Wavelet
	\label{wavelets:subsection:CWT}}
Die stetige Wavelettransformation (CWT) eines Signals $f(t)$ ist
\cite{wavelets:H.Burkhardt.2020}
\begin{equation}
	W_{a,b}=\langle f \; , \; \psi_{a,b} \rangle = \frac{1}{\sqrt{a}}\int_{-\infty}^{\infty} f(t)\cdot\psi\left(\frac{t-b}{a}\right) dt,
	\label{wavelets:equation3}
\end{equation}
wobei
\[
\psi_{a,b}(t)=\psi\left(\frac{t-b}{a}\right)
\]
definiert wird und die Funktion $\psi(t)$ wird das {\em Mutterwavelet}
\index{Mutterwavelet}
genannt. Das Mutterwavelet ist, analog zum Sinus in der DFT, die
Ausgangsfunktion mit der das Signal im Skalarprodukt verrechnet
wird.
Was sofort auffällt ist, dass man nun zwei unabhängige Variablen
besitzt $a$, $b$.
Zum einen ist das die Frequenzskalierung $a$, so
genannt weil mit diesem Faktor das Mutterwavelet gedehnt oder
komprimiert wird, man somit die Frequenz des Wavelets festgelegt.
Zum anderen die Verschiebung auf der Zeitachse $b$, also wo das
Wavelet sich zeitlich befindet (Abbildung
\ref{wavelet:fig:5_WaveletKompUndShift}).

\begin{figure}
	\centering
	\includegraphics[width=\textwidth]{papers/wavelets/images/5_WaveletKompUndShift.png}
	\caption{Darstellung der Wavelet Komprimierung a und der Verschiebung b}
	\label{wavelet:fig:5_WaveletKompUndShift}
\end{figure}%

Hierbei handelt es sich um einen Kernpunkt der Anwendung der
Wavelettransformation, ihre grosse Stärke ist nämlich die zeitlich
exakte Auflösung. Um der Wavelettransformation die gleiche Maske
wie der DFT zu geben, kann man
\begin{equation}
	W_{a,b}=\sum_{a=f_0}^{a_n}\sum_{b=0}^{b_m}\frac{1}{N(a)}\sum_{n=0}^{N-1} x(n)\cdot\psi\left(\frac{n-b}{a}\right)
	\label{wavelets:equation4}
\end{equation}
als Summenformel nachvollziehen.
Die Formel deutet an, dass der Berechnungsaufwand deutlich ansteigen wird,
sofern mit hoher Frequenz und Zeitauflösung gerechnet wird.

Die Relation zwischen der Auflösung von Frequenz und Zeit ist ein
Kernthema in der Signalanalyse.
Bei der DFT kann entweder die Zeit oder die Frequenz genau aufgelöst
werden.
Beides zusammen ist an sich nicht möglich, weil bedingt durch die
Unschärferelation (Kapitel~\ref{buch:diskret:section:unschaerfe})
\index{Unschärferelation}%
sind Zeit- und Frequenzauflösung ein komplementäres Paar, wodurch
nicht beides gleichzeitig exakt bestimmbar ist.
Anders ausgedrückt, die beiden Auflösungen bedingen
eine gegensätzliche Periodenlänge.
Ein schmales Fenster, welches die Periodenbreite bestimmt, ergibt
eine hohe Zeit- aber schlechte Frequenzauflösung.
Ein breites Fenster führt demnach gerade zum Gegenteil.

Dieses konträre Verhalten kann relativ einfach mit folgender
Überlegung nachvollzogen werden.
%\begin{itemize}
%\item
Um eine Frequenzauflösung von 0.1\,Hz zu erhalten, braucht
es mit $f=1⁄T$ eine Zeitperiode von 10\,s, was einer sehr guten
Frequenzauflösung entspricht, handkehrum bekommt man auch immer nur
für die Periode von 10 Sekunden eine zeitliche Information. Wenn
bei dieser Auflösung bestimmt werden soll, ob es nach 1, 4 oder 10
Sekunden zum Event kam, ist das nicht bestimmbar.
%\end{itemize}
 
Man kann mit gewissen Kniffen, wie beispielsweise mit Hilfe von
Signalüberlappung oder Zero-Padding, dem entgegen wirken.
\index{Zero-Padding}%
\begin{itemize}
	\item Bei der Signalüberlappung wird das Auswertungsfenster
	mit einer Überdeckung verschoben. Dabei ist $2/3$ ein
	gängiger Wert für diese Überdeckung. Es muss erwähnt sein,
	dass es dabei mehr um die Problematik des Informationsverlustes
	als um die bessere zeitliche Auflösung geht.
	\item Beim Zero-Padding wird die Periodenbreite künstlich
	mit Nullen verbreitert.
	Man kann sich das mit einer Doppelfensterung vorstellen,
	das innere Fenster dient der hohen zeitlichen Auflösung und
	das äussere Fenster der besseren Frequenzauflösung und alles
	was zwischen dem inneren und äusseren Fenster liegt wird
	mit Nullen befüllt.
	Dadurch kann sowohl die Zeit als auch die Frequenzauflösung erhöht
	werden.
	Durch die künstliche Verbreiterung geht aber die
	Amplitudengenauigkeit verloren.
\end{itemize}
Die Unschärfeproblematik in (Abbildungen \ref{wavelet:fig:FFTAufloesung}
und \ref{wavelet:fig:AufloesungZeitVsFrequenz}) bleibt aber bestehen.

\begin{figure}
	\centering
	\includegraphics[width=0.4\textwidth]{papers/wavelets/images/6-1_FFTAufloesung.png}
	\caption{Schematische Darstellung
	der Unschärfeproblematik, entweder wird die Zeit oder die
	Frequenzgenauigkeit verfeinert, dargestellt anhand der beiden
	Achsrichtungen.
	Aus \cite{wavelets:J.Mayer.2002}.
}
	\label{wavelet:fig:FFTAufloesung}
\end{figure}

\begin{figure}
	\centering
	\includegraphics[width=\textwidth]{papers/wavelets/images/6-2_AufloesungZeitVsFrequenz.png}
	\caption{Unschärfeproblematik dargestellt an der FFT mit
	einem Hanning-Fenster. Im obersten Plot pro Gruppe ist
\index{Hanning-Fenster}%
	jeweils das Zeitsignal das mit dem darunter liegenden Fenster
	verrechnet wird. Im untersten Plot der Gruppe ist der
	zugehörige Amplitudengang abgebildet.
	Mit grün hinterlegt ist die Ausgangsauswertung mit einer
	Fensterbreite von 1s, was einer Frequenzauflösung von 1\,Hz
	entspricht.
	Der rote Weg beschreibt die Möglichkeit der Fensterüberlappung,
	die Fensterbreite bleibt dabei 1s.
	Der blaue Weg stellt die Auswertung mit einem schmaleren
	Fenster der Breite $1/5$\,s dar (5\,Hz Auflösung).
	Die zeitliche Auflösung kann dadurch deutlich verbessert werden.
	Am Amplitudengang ist aber die schlechtere Frequenzauflösung
	an den breiter werdenden Peaks zu erkennen.}
	\label{wavelet:fig:AufloesungZeitVsFrequenz}
\end{figure}

Das Wavelet hingegen besitzt die Eigenschaft, dass sowohl zeitlich
als auch auf der Frequenzachse zugleich eine hohe Auflösung erzielt
werden kann.
Es besitzt sowohl in der Skalierung als auch in der Verschiebung
einen freien Parameter.
Wie bereits erwähnt geht das aber Hand in Hand mit einem hohen
Berechnungsaufwand, sofern beides hoch aufgelöst werden soll.
Ausserdem unterliegt das Wavelet durch seine Gewichtung, im Falle
des Morlet-Wavelets der Gauss-Gewichtung,
einer gewissen Verschmierung bei der Transformation ins Frequenzspektrum.
Die Bandbreite im Amplitudengang ist wesentlich grösser als bei der
FFT, weil Frequenzen nahe der zentralen Frequenz stärker betont
werden.
Die zeitliche Auflösung und die Detektion von zeitlichen Veränderungen
durch das Wavelet ist hingegen fantastisch (schematisch in Abbildung
\ref{wavelet:fig:CWTAufloesungRadix2} und anschaulich in Abbildung
\ref{wavelet:fig:ErsteAnwendung}).

\begin{figure}
	\centering
	\includegraphics[width=0.4\textwidth]{papers/wavelets/images/6-3_CWTAufloesungRadix2.png}
	\caption{Schematische Darstellung der Auflösungsmöglichkeit
	bei der Wavelettransformation \cite{wavelets:J.Mayer.2002}.
	Zeit und Frequenz können nach freiem belieben aufgelöst
	werden (separate Parameter), im der Darstellung verdeutlicht
	an einer Radix 2 Filterbank. Die Idee dahinter, das bei
	tiefen Frequenzen bedingt durch die Wellenlänge eine
	schlechtere Zeitauflösung genügt und erst bei kurzen
	Wellenlängen (hoher Frquenz) eine höhere zeitliche Auflösung
	benötigt wird.}
	\label{wavelet:fig:CWTAufloesungRadix2}
\end{figure}

Das Mutterwavelet $\psi\left(\frac{t-b}{a}\right)$ kann ganz
verschiedene Funktionen besitzen, dargestellt in Abbildung
\ref{wavelet:fig:MutterwavletTypen}, welche aber zwei Vorgaben
einhalten müssen:

\begin{figure}
	\centering
	\includegraphics[width=\textwidth]{papers/wavelets/images/6-4_MutterwavletTypen.png}
	\caption{\cite{wavelets:Baleanu.2015} Beispiele von
	verschiedenen Mutterwavelettypen.}
	\label{wavelet:fig:MutterwavletTypen}
\end{figure}

\begin{itemize}
	\item Der Mittelwert des Wavelets muss Null sein, ansonsten
	würde das Signal bei der Verrechnung durch das Wavelet
	gewichtet.
	\item Das Wavelet muss eine endliche Energie $||\psi||_2<\infty$
	besitzen. Als Konsequenz aus der Cauchy-Schwarz-Ungleichung
	wird dann auch
	\begin{equation}
		\int_{-\infty}^{\infty} f(t)\cdot\psi\left(\frac{t-b}{a}\right) dt
		\label{wavelets:equation5}
	\end{equation}
	einen endlichen Wert besitzen, sofern $||f||_2<\infty$ ist.
	Im Gegensatz dazu hat beispielsweise ein reiner Sinus bei
	dieser Integration einen unendlichen Wert. In dieser Bedingung
	versteckt sich gerade auch noch eine weitere Eigenschaft,
	welche die Wavelettransformation gegenüber der herkömmlichen
	Fourier-Transformation besitzt. Die Gewichtung wird vom
	Wavelet selbst und nicht über die Verrechnung mit einer
	Fensterfunktion getätigt.
\end{itemize}



%\subsection{De finibus bonorum et malorum
%\label{wavelets:subsection:finibus}}
%blabla \eqref{wavelets:equation1}.

%Et harum quidem rerum facilis est et expedita distinctio
%\ref{wavelets:section:teil2}.
%test
%\ref{wavelets:section:teil3}.



