%
% teil5.tex -- Beispiel-File für Teil 5
%
% (c) 2020 Prof Dr Andreas Müller, Hochschule Rapperswil
%
% !TEX root = ../../buch.tex
% !TEX encoding = UTF-8
%
\section{Rücktransformation (ICWT)
	\label{wavelets:section:teil5}}
\rhead{Teil 5}

In diesem kurzen Abschnitt wird die Rücktransformation vom Frequenz- in den Zeitbereich betrachtet. Dabei wird die erhaltene $a_n \times b_m$ Matrix (aus der CWT), welche die Transformationsparameter im Bildbereich beinhaltet in den Zeitbereich rücktransformiert. Dabei sollte in dieser Betrachtung wieder das Ausgangssignal aus der Abbildung \ref{wavelet:fig:FFTnoiseFree} zurück gewonnen werden.
Das erhaltene Resultat aus der ICWT Abbildung \ref{wavelet:fig:ICWT}, entpricht aber nicht wirklich dem eingangs transformierten Signal, weil:

\begin{itemize}
	\item die CWT ein verschmiertes Spektrum erzeugt.
	\item das Morlet Wavelet einem Gauss gewichteten Cosinus entspricht und damit kein scharfkantiges Sinus- oder Cosinussignal reproduziert werden kann.
\end{itemize}

Um letztere These genauer zu untersuchen, wurden zwei Wege eingeschlagen. Zum einen die Rücktransformation des Cosinus über dasselbe Gaussfenster mit dem schon das Wavelet generiert wurde und zum anderen mit einem Rechteckfenster, in der Hoffnung das dieses ein reines Sinussignal besser reproduzieren kann.

Abbildung \ref{wavelet:fig:ICWT} zeigt aber, dass die Problematik in erster Linie beim Verschmieren der Frequenzausgabe liegt. Die CWT detektiert die Frequenz nicht gleich scharf wie die FFT und dadurch werden Frequenzen nahe der Hauptfrequenz ebenfalls mit einer nennenswerten Amplitude im Frequenzspektrum angegeben. Bei Rücktransformation werden diese Anteile mit einbezogen und verursachen eine Überlagerung (analog zur Abbildung \ref{wavelet:fig:AmplitudengangExtraktionDFT} unten). Diese Störung verhindert Hauptsächlich die Möglichkeit mit der CWT eine sinnvolle Rücktransformation zu erhalten. Aus Zeit gründen konnten hier auch keine weiteren Versuche für eine Optimierung gestartet werden. In gemachten ICWT wurde die Mittlung noch vernachlässigt, wodurch die Amplitudenwerte durch die Aufsummierung der einzelnen spektralen Anteile stark überhöht sind.

\begin{figure}
	\centering
	\includegraphics[width=0.75\textwidth]{papers/wavelets/images/19-1_ICWT.png}
	\caption{Rücktransformation des CWT transformierten Testsignal aus Abbildung \ref{wavelet:fig:FFTnoiseFree}. Einerseits mit dem gaussschen Fenster welches schon für die Erzeugung des Wavelets genutzt wurde. Andererseits mit einem Rechteckfenster. D.h. die durch die CWT erhaltene Transformationsmatrix $a_n \times b_m$, wird vom Bild in den Zeitbereich zurück transferiert. Die beiden Fenster stehen hierbei sinnbildlich dafür, ob eine reine Cosinuswelle (Rechteckfenster) oder ein Wavelet (Gaussfenster) mit den Bildmatrixparametern bestückt wird.
	Es zeigt sich, dass beide Varianten aufgrund der Verschmierung durch die CWT, keine sinnvolle Rücktransformationen in den Zeitbereich ermöglichen.}
	\label{wavelet:fig:ICWT}
\end{figure}