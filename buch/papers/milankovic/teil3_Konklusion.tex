%
% teil3.tex -- Beispiel-File für Teil 3
%
% (c) 2020 Prof Dr Andreas Müller, Hochschule Rapperswil
%
% !TEX root = ../../buch.tex
% !TEX encoding = UTF-8
%
\section{Konklusion
\label{milankovic:section:Konklusion}}

Den Untersuchungen folgend ist es so, dass die Temperatur auf der Erde tatsächlich periodischen Schwankungen unterliegt.
Die von Milankovic untersuchten astronomischen Bewegungen haben dabei zusammen mit weiteren Phänomenen einen Einfluss auf das Langzeitverhalten des Klimas.
Aufgrund der imensen Zeiträume in der sich die Zyklen abspielen, sind jegliche rasch oder plötzlich auftretende Phänomene durch andere Phänomene zu erklären und nicht auf die Milankovic-Zyklen zurückzuführen. 

Im Gegensatz dazu ist der viel diskutierte menschliche Einfluss auf einer anderen, viel kurzweiligeren, Zeitskala zu betrachten.
Aussergewöhnlich fällt dabei das anhaltende Temperaturhoch auf, das sich an den Ausstoss von Treibhausgasen der Menschheit koppelt.
Die Zyklen sind also für Temperaturänderungen in der Atmosphäre verantwortlich, jedoch nicht für plötzliche Ausreisser, sondern für jahrtausende anhaltende Schwankungen.

