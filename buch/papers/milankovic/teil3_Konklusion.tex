%
% teil3.tex -- Beispiel-File für Teil 3
%
% (c) 2020 Prof Dr Andreas Müller, Hochschule Rapperswil
%
% !TEX root = ../../buch.tex
% !TEX encoding = UTF-8
%
\section{Schlussfolgerung
\label{milankovic:section:Schlussfolgerung}}

Den Untersuchungen folgend erkennt man, dass die Temperatur auf der Erde tatsächlich periodischen Schwankungen unterliegt.
Die von Milankovic untersuchten astronomischen Bewegungen haben dabei zusammen mit weiteren Phänomenen einen Einfluss auf das Langzeitverhalten des Klimas.
Aufgrund der imensen Zeiträume in der sich die Zyklen abspielen, sind jegliche rasch oder plötzlich auftretende Veränderungen durch andere Phänomene zu erklären und nicht auf die Milankovic-Zyklen zurückzuführen. 
Die Zyklen sind also für Temperaturänderungen in der Atmosphäre verantwortlich, jedoch nicht für plötzliche Ausreisser, sondern für jahrtausende anhaltende Schwankungen.

Im Gegensatz dazu spielt sich der viel diskutierte menschliche Einfluss auf einer anderen, bedeutend kürzeren Zeitskala ab.
Dieser lässt sich besser auf die aktuell auftretenden Phänomene applizieren.
Aussergewöhnlich fällt dabei das anhaltende Temperaturhoch auf, das sich an den Ausstoss von Treibhausgasen der Menschheit koppelt.
Es liegt also auch hier nahe, dass eine Kausalität besteht.

Abschliessend lässt sich sagen, dass das Klima auf der Erde in stetigem Wandel ist und durch verschiedene Phasen läuft.
Die Milankovic-Zyklen beeinflussen diese Prozesse Mal mehr und weniger stark.
Nahezu schlagartige Veränderungen können über die beschriebenen Zyklen aber nicht stattfinden, da die Perioden dafür zu gross sind.

jl
