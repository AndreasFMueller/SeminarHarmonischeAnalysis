%
% teil1.tex -- Beispiel-File für das Paper
%
% (c) 2020 Prof Dr Andreas Müller, Hochschule Rapperswil
%
% !TEX root = ../../buch.tex
% !TEX encoding = UTF-8
%
\section{Vorstellung der Milankovic-Zyklen
\label{milankovic:section:VorstellungMIlankovic-Zyklen}}

\subsection{Die Bewegungen der Erde
\label{milankovic:subsection:BewegungenderErde}}
Unser Heimatplanet, die Erde, bewegt sich auf einem komplexen Pfad durch die Raumzeit.
Dabei sind die Bahnen, auf welchen wir uns bewegen, selbst ständigen Veränderungen ausgesetzt.
Stetige, langsame Änderungen der Bahnparameter führen zu geringfügigen Variationen der elliptischen Umlaufbahn.
Bei der Betrachtung der Bewegungen ist es naheliegend, dass unser Leben besonders von der Rotation um die eigene Achse sowie der Rotation um die Sonne beeinflusst wird.
Wir kennen diese Zyklen im Allgemeinen als Tage und Jahre.
Zeiträume, die wir uns gut vorstellen können.
Doch damit ist die Bewegung nicht komplett beschrieben.
Mit der Exzentrizität, der Präzession und der Obliquität beschreiben die Milankovic-Zyklen drei weitere zyklische, der Mehrheit eher unbekannte, Parameter, die wesentliche Einflüsse auf das Klima des Erdballs haben.
Deren Entstehung und Wirkung auf das Klima der Erde werden nachstehend erläutert.

\begin{figure}
	\centering
	\includegraphics[width=\linewidth]{papers/milankovic/pictures/01_MilankovicZyklenÜbersicht.pdf}
	\caption{Übersicht der Milankovic-Zyklen
		\label{picture Übersicht Milankovic-Zyklen}}
\end{figure}

\subsection{Exzentrizität
\label{milankovic:subsection:Exzentrizität}}
Mit der Exzentrizität wird die Elliptizität der Umlaufbahn der Erde um die Sonne beschrieben.
Diese ist nicht kreisförmig, sondern leicht oval --- elliptisch.
Je elliptischer die Umlaufbahn wird, desto grösser ist die Exzentrizität.
Eine Exzentrizität von null entspricht einem perfekten Kreis.
Verantwortlich für diese Abweichungen sind die anderen Planeten im Sonnensystem, vorwiegend die Gasriesen Jupiter und Saturn.

In einem Intervall von rund 100'000 Jahren schwankt die Form zwischen einer geringen Exzentrizität, mit dem Minimalwert 0.0006, und einer leicht elliptischen Bahn, mit dem Maximalwert bei 0.058.
Alle 400'000 Jahre überlagern sich bestimmte Bewegungen was in diesen Zeiten zu extremeren Werten führt.
Aktuell beträgt die Exzentrizität 0.0167, mit abnehmender Tendenz.
Dabei variiert die Entfernung zur Sonne zwischen dem sonnennächsten Punkt, dem Perihelion, und dem sonnenfernsten Punkt, dem Aphelion, um rund 3.4\%, was absolut etwa 5.1 Millionen Kilometern entspricht.
Durch die unterschiedlichen Entfernungen schwankt aktuell die Intensität der Sonneneinstrahlung um rund 7\% zwischen den Maximalwerten.
Die Intensitätsschwankungen können dabei zwischen 2\%, bei der nahezu kreisförmigen Umlaufbahn, und 23\%, im Falle der grössten Exzentrizität, liegen.

Daraus erkennt man, dass mit der Zunahme der Exzentrizität die Unterschiede der eintreffenden Sonnenenergie zwischen den beiden Extrema, Perihelion und Aphelion, ebenfalls steigen.
Es nehmen also die Wetterextreme zu.
Zurzeit ist der sonnennächste Punkt Anfang Januar und somit im Nordwinter, wo das Aphelion, Anfangs Juli ist.
Dann wenn auf der Nordhalbkugel Sommer ist.
Diese Stellung sorgt für gemässigtere Jahreszeiten. 

\subsection{Obliquität
\label{milankovic:subsection:Obliquität}}
Unter der Obliquität versteht man die sich ändernde Erdachsenneigung.
Die Erdachsenneigung ist als Winkel zwischen der Erdachse, welche durch den Massenmittelpunkt und die Rotation der Erde bestimmt ist, und der Normalen zur Ekliptikebene, auch Erdbahnebene, definiert.

In einem Zyklus von rund 41'000 Jahren pendelt der Winkel zwischen 22.1 und 24.5 Grad.
Aktuell liegt die Erdachsenneigung mit 23.43 Grad im mittleren Bereich.
Die Tendenz ist abnehmend und wird in rund 8’000 Jahren ein Minimum erreichen.
Ist die Erdachse wenig geneigt, bleibt die Intensität der Sonneneinstrahlung an den Polen konstanter.
Bei grösserer Neigung steigen die Extreme und Änderungen des Klimas an.
Der Sommer wird wärmer, die Winter kälter, je weiter man sich in Richtung der Pole bewegt.
Kälte- oder Eiszeiten folgen stets auf Phasen mit kleiner Erdachsenneigung.
Besonders weil im Äquatorbereich stets viel Meerwasser verdunstet und sich an den Polen Schnee und Eis bildet, das im Sommer weniger abschmilzt, weil die Sonneneinstrahlung nur gering ist.
So können sich an den Polen Kontinentaleisschilde bilden, die aufgrund des Albedo-Effekts die Abkühlung weiter verstärken.

Doch nicht nur klimatisch, sondern auch astronomisch ist der Einfluss spürbar.
Denn auch die Entfernung der Erdachse zum Polarstern obliegt Schwankungen, sodass der aktuelle Polarstern in rund 8 Jahrtausend nicht mehr Richtung Norden zeigt.

\subsection{Präzession
\label{milankovic:subsection:Präzession}}
Unter dem Begriff der Präzession werden zwei unterschiedliche Bewegungen verstanden.
Denn einerseits rotiert die Erdachse um eine Vertikale und andererseits findet eine Präzession der Absiden statt. 

\begin{itemize}
\item
Erdachse
\item
Apsidenpräzession
\end{itemize}

Die «Taumelbewegung» der Erdachse um die Normalen zur Ekliptikebene,  wiederholt sich in einem Zyklus von rund 23'000 Jahren.
Sie wird als massgebende Präzession verstanden.
Demzufolge wird die Erdachse in etwa 11'000 Jahren relativ zum heutigen Stand in inverser Lage sein, was zur Verschiebung der Jahreszeiten führt.
Sommer wird zum Winter und umgekehrt.
Ausgelöst wird dies besonders durch die Kraft, welche der Mond und die Sonne auf den Äquatorwulst ausüben.

Die Apsidenpräzession wird auch Periheldrehung genannt.
Dabei dreht sich die gesamte Erdumlaufbahn, bei gleichbleibender Form und Lage der Ebene, um die Sonne. Eine Änderung der Form wird dabei der Exzentrizät
\ref{milankovic:subsection:Exzentrizität}
zugeschrieben und nicht der Präzession.
Der Zyklus einer Apsidenpräzession beträgt rund 112'000 Jahre.

Für die Veränderung der globalen Temperatur ist die Präzession, wie auch die Obliquität, alleine nur wenig wirksam, weil sich die gegensätzlichen Wirkungen auf der Nord- und Südhemisphäre aufheben.
Verädert sich die Lage der Erdachse im Zusammenhang mit der Exzentrizität, werden markante Unterschiede bemerkbar.
Wenn beispielsweise der Nordsommer auf den Perihelion trifft, wie dies zu Beginn des Holozäns der Fall war, kann so das Abschmelzen grosser eiszeitlicher Gletschermassen begünstigt werden.

\subsection{Aktuelle Lage
\label{milankovic:subsection:AktuelleLage}}
Bei der Betrachtung der Intensität der Sonneneinstrahlung unter Berücksichtigung der Milankovic-Zyklen befinden wir uns aktuell in einer Übergangsphase in eine Kaltzeit.
Es müsste tendenziell eher kälter werden.
Jedoch sind die Messperioden lange und die Temperaturunterschiede nur klein.
In einem Jahrtausend sinkt die Temperatur nur um wenige Grade.
Würde dieser Trend ungestört anhalten, folgt in etwa 10'000 Jahren eine neue Glazialperiode.
Ob dieses Ereignis wie berechnet eintreten wird oder die aktuelle Warmphase länger anhält, hängt von unterschiedlichen, antropogen mehr oder weniger beinflussbaren, Faktoren ab. Besonders die Veränderung der Atmosphäre durch den Eintrag von Treibhausgasen und Aerosolen beeinflusst den fortschreitenden Klimaverlauf.
Diese können durch die Menschheit, aber auch auf natürliche Faktoren auftreten.

jl

