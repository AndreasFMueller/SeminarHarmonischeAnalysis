%
% einleitung.tex -- Beispiel-File für die Einleitung
%
% (c) 2020 Prof Dr Andreas Müller, Hochschule Rapperswil
%
% !TEX root = ../../buch.tex
% !TEX encoding = UTF-8
%
\section{Einleitung\label{milankovic:section:teil0}}
\rhead{Einleitung}

teil 0
Die Debatte um den Klimawandel ist in vollem Gange.
Manifestiert halten die Parteien an ihren Argumenten fest.
Immer wieder wird darauf hingewiesen, dass es schon immer warme und kalte Perioden in der Geschichte der Erde gab.
Demzufolge gehe kein Risiko von einer klimatischen Erwärmung aus.
Wer sich mit den Temperaturverläufen der Erdatmosphäre in den letzten Jahrtausenden auseinandersetzt, trifft unweigerlich auf den Begriff der Milankovic-Zyklen.

Die vom serbischen Mathematiker anfangs des zwanzigsten Jahrhunderts entwickelten Erkenntnisse über den Einfluss der Erdbewegung auf das Klima wurden 1924 in einem Kapitel eines Werkes über «Klimate der geologischen Vorzeit» veröffentlicht.
Daraufhin begann die grossflächige Verbreitung der Erkenntnisse in der Wissenschaft. Erst durch die Publikation in Papers renommierter Mathematiker gelang der Durchbruch und zur allgemeinen Anerkennung seiner Forschung.


