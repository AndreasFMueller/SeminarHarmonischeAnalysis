%
% teil3.tex -- Beispiel-File für Teil 3
%
% (c) 2020 Prof Dr Andreas Müller, Hochschule Rapperswil
%
% !TEX root = ../../buch.tex
% !TEX encoding = UTF-8
%
\section{Konklusion
\label{milankovic:section:Konklusion}}

Den Untersuchungen folgend ist es so, dass die Temperatur auf der Erde tatsächlich periodischen Schwankungen unterliegt.
Die von Milankovic untersuchten astronomischen Bewegungen haben dabei zusammen mit weiteren Phänomenen einen Einfluss auf das Langzeitverhalten des Klimas.

Im Gegensatz dazu ist der viel diskutierte menschliche Einfluss auf einer anderen Zeitskala zu betrachten.
Als aussergewöhnlich fällt dabei das anhaltende Temperaturhoch auf, das sich an den Ausstoss von Treibhausgasen der Menschheit koppelt. 



\subsection{De finibus bonorum et malorum
\label{milankovic:subsection:malorum}}
At vero eos et accusamus et iusto odio dignissimos ducimus qui
blanditiis praesentium voluptatum deleniti atque corrupti quos
dolores et quas molestias excepturi sint occaecati cupiditate non
provident, similique sunt in culpa qui officia deserunt mollitia
animi, id est laborum et dolorum fuga. Et harum quidem rerum facilis
est et expedita distinctio. Nam libero tempore, cum soluta nobis
est eligendi optio cumque nihil impedit quo minus id quod maxime
placeat facere possimus, omnis voluptas assumenda est, omnis dolor
repellendus. Temporibus autem quibusdam et aut officiis debitis aut
rerum necessitatibus saepe eveniet ut et voluptates repudiandae
sint et molestiae non recusandae. Itaque earum rerum hic tenetur a
sapiente delectus, ut aut reiciendis voluptatibus maiores alias
consequatur aut perferendis doloribus asperiores repellat.

	\subsection{De finibus bonorum et gleichung

Sed ut perspiciatis unde omnis iste natus error sit voluptatem
accusantium doloremque laudantium, totam rem aperiam, eaque ipsa
quae ab illo inventore veritatis et quasi architecto beatae vitae
dicta sunt explicabo.
Nemo enim ipsam voluptatem quia voluptas sit aspernatur aut odit
aut fugit, sed quia consequuntur magni dolores eos qui ratione
voluptatem sequi nesciunt

\begin{equation}
	\int_a^b x^2\, dx
	=
	\left[ \frac13 x^3 \right]_a^b
	=
	\frac{b^3-a^3}3.
	\label{milankovic:equation1}
\end{equation}
Neque porro quisquam est, qui dolorem ipsum quia dolor sit amet,
consectetur, adipisci velit, sed quia non numquam eius modi tempora
incidunt ut labore et dolore magnam aliquam quaerat voluptatem.

Ut enim ad minima veniam, quis nostrum exercitationem ullam corporis
suscipit laboriosam, nisi ut aliquid ex ea commodi consequatur?
Quis autem vel eum iure reprehenderit qui in ea voluptate velit
esse quam nihil molestiae consequatur, vel illum qui dolorem eum
fugiat quo voluptas nulla pariatur?

\ref{milankovic:section:teil2}.
Nam libero tempore, cum soluta nobis est eligendi optio cumque nihil
impedit quo minus id quod maxime placeat facere possimus, omnis
voluptas assumenda est, omnis dolor repellendus
\ref{milankovic:section:teil3}.
Temporibus autem quibusdam et aut officiis debitis aut rerum
necessitatibus saepe eveniet ut et voluptates repudiandae sint et
molestiae non recusandae.
Itaque earum rerum hic tenetur a sapiente delectus, ut aut reiciendis
voluptatibus maiores alias consequatur aut perferendis doloribus
asperiores repellat.

Ein paar Hinweise für die korrekte Formatierung des Textes und den Kurztests.

\begin{itemize}
	\item
	Absätze werden gebildet, indem man eine Leerzeile einfügt.
	Die Verwendung von \verb+\\+ ist nur in Tabellen und Arrays gestattet.
	\item
	Die explizite Platzierung von Bildern ist nicht erlaubt, entsprechende
	Optionen werden gelöscht. 
	Verwenden Sie Labels und Verweise, um auf Bilder hinzuweisen.
	\item
	Beginnen Sie jeden Satz auf einer neuen Zeile. 
	Damit ermöglichen Sie dem Versionsverwaltungssysteme, Änderungen
	in verschiedenen Sätzen von verschiedenen Autoren ohne Konflikt 
	anzuwenden.
	\item 
	Bilden Sie auch für Formeln kurze Zeilen, einerseits der besseren
	Übersicht wegen, aber auch um GIT die Arbeit zu erleichtern.
\end{itemize}
