%
% exzentrizitaet.tex -- Exzentrizitaet
%
% (c) 2021 Prof Dr Andreas Müller, OST Ostschweizer Fachhochschule
%
\documentclass[tikz]{standalone}
\usepackage{amsmath}
\usepackage{times}
\usepackage{txfonts}
\usepackage{pgfplots}
\usepackage{csvsimple}
\usetikzlibrary{arrows,intersections,math}
\begin{document}
\def\skala{1}
\begin{tikzpicture}[>=latex,thick,scale=\skala]
\def\a{3}
\def\ellipse#1{
	\pgfmathparse{(#1)*\a}
	\xdef\e{\pgfmathresult}
	\pgfmathparse{\a*sqrt(1-(#1)*(#1))}
	\xdef\b{\pgfmathresult}
	\draw (-\e,0) circle[x radius=\a, y radius=\b];
}

\begin{scope}

\fill[color=red] (0,0) circle[radius=0.2];

\ellipse{0}
\ellipse{0.2}
\ellipse{0.4}
\ellipse{0.6}
\ellipse{0.8}
\ellipse{0.9}

\node at (0,\a) [above] {$e=0$};
\node at (-0.9,{0.97*\a}) [above left] {$e=0.2$};
\node at (-2.3,{0.83*\a}) [above left] {$e=0.4$};
\node at (-3.5,{0.63*\a}) [above left] {$e=0.6$};
\node at (-4.5,{0.39*\a}) [above left] {$e=0.8$};
\node at (-5.6,0) [left] {$e=0.9$};

\end{scope}

\end{tikzpicture}
\end{document}

