%
% strahlung.tex -- template for standalon tikz images
%
% (c) 2021 Prof Dr Andreas Müller, OST Ostschweizer Fachhochschule
%
\documentclass[tikz]{standalone}
\usepackage{amsmath}
\usepackage{times}
\usepackage{txfonts}
\usepackage{pgfplots}
\usepackage{csvsimple}
\usetikzlibrary{arrows,intersections,math}
\begin{document}
\def\skala{1}
\begin{tikzpicture}[>=latex,thick,scale=\skala]

\def\expfad{ (0,0)
	-- ({0.000*\dx},{0.000*\dy})
	-- ({0.010*\dx},{0.000*\dy})
	-- ({0.020*\dx},{0.000*\dy})
	-- ({0.030*\dx},{0.000*\dy})
	-- ({0.040*\dx},{0.001*\dy})
	-- ({0.050*\dx},{0.001*\dy})
	-- ({0.060*\dx},{0.002*\dy})
	-- ({0.070*\dx},{0.002*\dy})
	-- ({0.080*\dx},{0.003*\dy})
	-- ({0.090*\dx},{0.004*\dy})
	-- ({0.100*\dx},{0.005*\dy})
	-- ({0.110*\dx},{0.006*\dy})
	-- ({0.120*\dx},{0.007*\dy})
	-- ({0.130*\dx},{0.009*\dy})
	-- ({0.140*\dx},{0.010*\dy})
	-- ({0.150*\dx},{0.011*\dy})
	-- ({0.160*\dx},{0.013*\dy})
	-- ({0.170*\dx},{0.015*\dy})
	-- ({0.180*\dx},{0.017*\dy})
	-- ({0.190*\dx},{0.019*\dy})
	-- ({0.200*\dx},{0.021*\dy})
	-- ({0.210*\dx},{0.023*\dy})
	-- ({0.220*\dx},{0.025*\dy})
	-- ({0.230*\dx},{0.028*\dy})
	-- ({0.240*\dx},{0.030*\dy})
	-- ({0.250*\dx},{0.033*\dy})
	-- ({0.260*\dx},{0.036*\dy})
	-- ({0.270*\dx},{0.039*\dy})
	-- ({0.280*\dx},{0.042*\dy})
	-- ({0.290*\dx},{0.045*\dy})
	-- ({0.300*\dx},{0.048*\dy})
	-- ({0.310*\dx},{0.052*\dy})
	-- ({0.320*\dx},{0.056*\dy})
	-- ({0.330*\dx},{0.059*\dy})
	-- ({0.340*\dx},{0.063*\dy})
	-- ({0.350*\dx},{0.068*\dy})
	-- ({0.360*\dx},{0.072*\dy})
	-- ({0.370*\dx},{0.076*\dy})
	-- ({0.380*\dx},{0.081*\dy})
	-- ({0.390*\dx},{0.086*\dy})
	-- ({0.400*\dx},{0.091*\dy})
	-- ({0.410*\dx},{0.096*\dy})
	-- ({0.420*\dx},{0.102*\dy})
	-- ({0.430*\dx},{0.108*\dy})
	-- ({0.440*\dx},{0.114*\dy})
	-- ({0.450*\dx},{0.120*\dy})
	-- ({0.460*\dx},{0.126*\dy})
	-- ({0.470*\dx},{0.133*\dy})
	-- ({0.480*\dx},{0.140*\dy})
	-- ({0.490*\dx},{0.147*\dy})
	-- ({0.500*\dx},{0.155*\dy})
	-- ({0.510*\dx},{0.163*\dy})
	-- ({0.520*\dx},{0.171*\dy})
	-- ({0.530*\dx},{0.179*\dy})
	-- ({0.540*\dx},{0.188*\dy})
	-- ({0.550*\dx},{0.197*\dy})
	-- ({0.560*\dx},{0.207*\dy})
	-- ({0.570*\dx},{0.217*\dy})
	-- ({0.580*\dx},{0.228*\dy})
	-- ({0.590*\dx},{0.239*\dy})
	-- ({0.600*\dx},{0.250*\dy})
	-- ({0.610*\dx},{0.262*\dy})
	-- ({0.620*\dx},{0.275*\dy})
	-- ({0.630*\dx},{0.288*\dy})
	-- ({0.640*\dx},{0.302*\dy})
	-- ({0.650*\dx},{0.316*\dy})
	-- ({0.660*\dx},{0.331*\dy})
	-- ({0.670*\dx},{0.347*\dy})
	-- ({0.680*\dx},{0.364*\dy})
	-- ({0.690*\dx},{0.382*\dy})
	-- ({0.700*\dx},{0.400*\dy})
	-- ({0.710*\dx},{0.420*\dy})
	-- ({0.720*\dx},{0.441*\dy})
	-- ({0.730*\dx},{0.463*\dy})
	-- ({0.740*\dx},{0.487*\dy})
	-- ({0.750*\dx},{0.512*\dy})
	-- ({0.760*\dx},{0.539*\dy})
	-- ({0.770*\dx},{0.568*\dy})
	-- ({0.780*\dx},{0.598*\dy})
	-- ({0.790*\dx},{0.631*\dy})
	-- ({0.800*\dx},{0.667*\dy})
	-- ({0.810*\dx},{0.706*\dy})
	-- ({0.820*\dx},{0.748*\dy})
	-- ({0.830*\dx},{0.795*\dy})
	-- ({0.840*\dx},{0.847*\dy})
	-- ({0.850*\dx},{0.905*\dy})
	-- ({0.860*\dx},{0.972*\dy})
	-- ({0.870*\dx},{1.050*\dy})
	-- ({0.880*\dx},{1.145*\dy})
	-- ({0.890*\dx},{1.264*\dy})
	-- ({0.900*\dx},{1.419*\dy})
	-- ({0.910*\dx},{1.628*\dy})
	-- ({0.920*\dx},{1.926*\dy})
	-- ({0.930*\dx},{2.369*\dy})
	-- ({0.940*\dx},{3.067*\dy})
}


\def\dx{10}
\def\dy{2}

\draw[->] (-0.1,0) -- (10.3,0) coordinate[label={$e$}];
\draw[->] (0,-0.1) -- (0,6.3) coordinate[label={right:$\Delta E\,[\%]$}];
\foreach \e in {1,...,9}{
	\draw ({0.1*\e*\dx},-0.05) -- ({0.1*\e*\dx},0.05);
	\node at ({0.1*\e*\dx},-0.05) [below] {$0.\e\mathstrut$};
}
\draw ({\dx},-0.05) -- ({\dx},0.05);
\node at ({\dx},-0.05) [below] {$1\mathstrut$};
\node at (0,-0.05) [below] {$0\mathstrut$};

\foreach \E in {50,100,150,200,250,300}{
	\draw (-0.05,{0.01*\E*\dy}) -- (0.05,{0.01*\E*\dy});
	\node at (-0.05,{0.01*\E*\dy}) [left] {$\E\mathstrut$};
}

\draw[line width=0.2pt] (\dx,0) -- (\dx,{3*\dy});

\begin{scope}
\clip (0,0) rectangle (\dx,3*\dy);
\draw[color=red,line width=1.2pt] \expfad;
\end{scope}

\end{tikzpicture}
\end{document}

