%
% einleitung.tex -- Beispiel-File für die Einleitung
%
% (c) 2020 Prof Dr Andreas Müller, Hochschule Rapperswil
%
% !TEX root = ../../buch.tex
% !TEX encoding = UTF-8
%
\section{Geschichtliche Hintergründe\label{gezeiten:section:Geschichtliche Hintergründe}}
\rhead{Geschichtliche Hintergründe}
Für die Seefahrt ist die Vorhersage der Gezeiten von besonderer Bedeutung.
Denn wer die Gezeiten nicht kannte, strandete mit dem Schiff unerwartet vor dem angepeilten Hafen.
Ebenso war die Gezeitenvorhersage wichtig für das Bauingenieurwesen, dort spielten jedoch die Höhen der Flut eine Rolle, damit man wusste mit welcher Höhe ein Schutzdamm gebaut werden muss.

Bereits im 17. Jahrhundert sind die ersten Tabellen für die Hafenstädte entstanden,
in welchen jeweils die Uhrzeiten mit den dazugehörigen Wasserständen notiert wurden.
Die \glqq{tide tables}\grqq, welche im Vereinigten Königreich 1664 von John Flamesteed
herausgegeben wurden, waren nahezu 150 Jahre im Einsatz.
Sir John Lubbock entwickelte Tafeln, 
in welchem er die Gezeiten den Einflüssen von Mondphase, und den Umlaufbahnen von
Sonne und Mond zuschrieb. Dieses Verfahren lieferte zuverlässige Werte für die
die Nordsee, jedoch war es beschränkt auf Seegebiete, in welchen je zwei Hoch- und Niedrigwasser innerhalb von 24 Stunden und 50 Minuten auftraten. 24 Stunden und 50 Minuten ist die durchschnittliche Länge eines Mondtages.

1868 revolutionierte Lord Kelvin, damals noch William Thomson, die Vorhersage der Gezeiten, in dem er die Kurve des Wasserstandes in die einzelnen Gezeitenkurven der einzelnen astronomischen Ursachen zerlegte.
Schon bei diesem Prozess machte sich Lord Kelbin das Leben einfacher, in dem er eine Boje benutze um den Wasserstand zu messen und direkt auf eine Papierrolle zu übertragen.
Er zerlegte den Verlauf der Gezeiten somit in harmonische Schwingungen, welche anhand von Perioden und Amplituden aufkonstruiert werden können.
Die Perioden resultierten aus den Bewegungsabläufen von Erde, Mond oder Sonne.
Die Phasen und Amplituden resultierten aus Mittelwerten von Beobachtungen von Lord Kelvin über eine längere Zeit.




