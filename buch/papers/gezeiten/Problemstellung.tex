%
% teil1.tex -- Beispiel-File für das Paper
%
% (c) 2020 Prof Dr Andreas Müller, Hochschule Rapperswil
%
% !TEX root = ../../buch.tex
% !TEX encoding = UTF-8
%
\section{Problemstellung
\label{gezeiten:section:teil1}}
\rhead{Problemstellung}
Kelvin wollte herausfinden wie sich seine gemessenen Wasserstände mit den Umlaufbahnen von Sonne und Mond herleiten liessen.
Der Schlüssel dazu fand er bei Joseph Fourier welcher im Jahre 1822 die Fourier-Reihen untersuchte, und die Fourier-Analyse erfand.
Viele Wissenschaftler waren dazumals skeptisch gegenüber der Theorie von Joseph Fourier, Kelvin jedoch war schon in jungen Jahren begeistert von ihm.
So schrieb er mit 17 Jahren ein Verteidigung zu gunsten von Fourier, welche anschliessend veröffentlicht wurde.
Die Gezeitenkurve, respektive die Kurve der Wasserstände zu entschlüsseln mit Hilfe der Fouriertheorie, war an und für sich ganz einfach.
Jedoch war der Rechenaufwand enorm.
Denn die Fourier-Analyse funktioniert folgendermassen.
Die vorhandene Kurve wird in möglichst kleine Zeitintervalle aufgeteilt, je kleiner die Intervalle sind, desto genauer wird am Ende das Ergebnis.
Bei jedem Intervall, wird dann der Gezeitenstand mit der Sinuswelle, mit der gewünschten Frequenz, multipliziert.
Die Summe dieser Rechtecke, wird durch die Gesamtzeit der verwendeten Intervalle dividiert.
Das Ergebnis dieser Division ergibt einen einzelnen Koeffizienten, genauer gesagt die Amplitude dvon der Sinuswelle mit der gewählten Frequenz.
Dieser Vorgang muss ebenfalls mit der Kosinusfunktion mit der selben Frequenz wiederholt werden.
Kelvin merkte dabei, dass er für eine genaue Vorhersage der Gezeiten neun verschiedene Frequenzkomponenten benötigte.
Dieser lange Rechenvorgang, ist jedoch nur das halbe Problem, denn sobald die Phasen und Amplituden der Sinusfunktionen bekannt sind, müssen diese auch noch addiert werden um zum Schluss den Wasserstand vorhersagen zu können.
Ist dies erreicht, ist jedoch lediglich die Gezeit für einen Standort berechnet.
Für jeden weiteren Standort, muss dieser umständliche und für Fehler anfällige Vorgang wiederholt werden.

Lord Kelvin hat Jahre damit verbracht, diese Rechenvorgänge zu analysieren und die Gezeiten vorherzusagen.
Dann aber kam Lord Kelvin die Idee, eine Maschine zu entwickeln, welche ihm diese mathematischen Vorgänge abnimmt. In Kelvins Worten: \glqq{Gehirne durch Messing ersetzen.}\grqq




