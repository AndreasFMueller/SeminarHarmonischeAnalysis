%
% sy.tex -- scotch yoke mechanismus
%
% (c) 2021 Prof Dr Andreas Müller, OST Ostschweizer Fachhochschule
%
\documentclass[tikz]{standalone}
\usepackage{amsmath}
\usepackage{times}
\usepackage{txfonts}
\usepackage{pgfplots}
\usepackage{csvsimple}
\usetikzlibrary{arrows,intersections,math,calc}
\definecolor{darkgreen}{rgb}{0,0.6,0}
\begin{document}
\def\skala{1}
\def\r{0.6}
\def\o{0.8}
\pgfmathsetseed{4719}
\pgfmathparse{0.5*random()} \xdef\azero{\pgfmathresult}
\pgfmathparse{0.5*random()}  \xdef\aone{\pgfmathresult}
\pgfmathparse{0.5*random()} \xdef\atwo{\pgfmathresult}
\pgfmathparse{0.5*random()}  \xdef\athree{\pgfmathresult}
\pgfmathparse{0.5*random()} \xdef\afour{\pgfmathresult}
\pgfmathparse{0.5*random()}  \xdef\afive{\pgfmathresult}
\pgfmathparse{0.5*random()} \xdef\asix{\pgfmathresult}
\pgfmathparse{0.5*random()}  \xdef\aseven{\pgfmathresult}
\pgfmathparse{360*random()} \xdef\phizero{\pgfmathresult}
\pgfmathparse{360*random()} \xdef\phione{\pgfmathresult}
\pgfmathparse{360*random()} \xdef\phitwo{\pgfmathresult}
\pgfmathparse{360*random()} \xdef\phithree{\pgfmathresult}
\pgfmathparse{360*random()} \xdef\phifour{\pgfmathresult}
\pgfmathparse{360*random()} \xdef\phifive{\pgfmathresult}
\pgfmathparse{360*random()} \xdef\phisix{\pgfmathresult}
\pgfmathparse{360*random()} \xdef\phiseven{\pgfmathresult}
\def\rad#1{
	\pgfmathparse{random()*60}
	\xdef\aph{\pgfmathresult}
	\fill[color=gray!60] #1 circle[radius={0.90*\r}];
	\foreach \a in {0,60,...,330}{
		\fill[color=white]
			($#1+({\a+\aph}:{0.6*\r})$) circle[radius={0.18*\r}];
		\draw[color=gray,line width=2pt]
			#1 -- +({\a+\aph+30}:{0.90*\r});
	}
	\draw[color=gray,line width=2pt] #1 circle[radius={0.91*\r}];
	\fill[color=gray] #1 circle[radius={0.18*\r}];
}
\def\arad#1#2#3{
	\fill[color=gray!20] #1 circle[radius={0.8*\r}];
	\fill[color=gray!40] #1 -- ++(0:#2) arc (0:{#3+90}:#2) -- cycle;
	\draw[color=black,line width=0.3pt] #1 circle[radius={#2}];
	\draw[color=black,line width=0.3pt] #1 -- +({#3+90}:#2);
	\fill[color=white] #1 circle[radius={0.04}];
	\fill[color=black] ($#1+({#3+90}:#2)$) circle[radius=0.05];
}
\def\yoke#1#2#3#4{
	\draw[color=darkgreen,line width=1.5pt] ($#1+(0,{#2*cos(#3)+0.1})$) -- ($#1+(0,{#2*cos(#3)+3+#4})$);
	\fill[color=darkgreen] ($#1+(0,{#2*cos(#3)+3+#4})$) circle[radius=0.05];
	\draw[color=darkgreen,rounded corners=2.0pt,line width=1.5pt]
		($#1+(-0.5,{#2*cos(#3)-0.08})$)
		rectangle
		($#1+(0.5,{#2*cos(#3)+0.08})$);
	
		
}
\def\in#1{
	\node at ({(2*#1-1)*\r},{-3.5}) [right,rotate=-90]
		{$a_{#1}\cos(v_{#1}t+\phi_{#1})$};
}

\begin{tikzpicture}[>=latex,thick,scale=\skala,
	declare function={
		g(\x,\frequency,\amplitude,\phase) = \amplitude*cos(\frequency*\x+\phase);
		f(\x) = \azero*cos(\x+\phizero)+\aone*cos(2*\x+\phione)+\atwo*cos(3*\x+\phitwo)+\athree*cos(4*\x+\phithree)+\afour*cos(5*\x+\phifour)+\afive*cos(6*\x+\phifive)+\asix*cos(7*\x+\phisix)+\aseven*cos(8*\x+\phiseven);
	},
]

\rad{(\r,{g(0,1,\azero,\phizero)-\o})}
\rad{({3*\r},{g(0,2,\aone,\phione)+\o})}
\rad{({5*\r},{g(0,3,\atwo,\phitwo)-\o})}
\rad{({7*\r},{g(0,4,\athree,\phithree)+\o})}
\rad{({9*\r},{g(0,5,\afour,\phifour)-\o})}
\rad{({11*\r},{g(0,6,\afive,\phifive)+\o})}
\rad{({13*\r},{g(0,7,\asix,\phisix)-\o})}
\rad{({15*\r},{g(0,8,\aseven,\phiseven)+\o})}

\fill[color=blue] (0,1.45) -- ++(0.1,0.1) -- ++(0,0.2) -- ++(-0.2,0)
	-- ++(0,-0.2) -- cycle;
\draw[color=blue,line width=1.5pt] (0,1.5)
	-- (0,{g(0,1,\azero,\phizero)-\o}) arc (180:360:\r)
	-- ({2*\r},{g(0,2,\aone,\phione)+\o}) arc (180:0:\r)
	-- ({4*\r},{g(0,3,\atwo,\phitwo)-\o}) arc (180:360:\r)
	-- ({6*\r},{g(0,4,\athree,\phithree)+\o}) arc (180:0:\r)
	-- ({8*\r},{g(0,5,\afour,\phifour)-\o}) arc (180:360:\r)
	-- ({10*\r},{g(0,6,\afive,\phifive)+\o}) arc (180:0:\r)
	-- ({12*\r},{g(0,7,\asix,\phisix)-\o}) arc (180:360:\r)
	-- ({14*\r},{g(0,8,\aseven,\phiseven)+\o}) arc (180:0:\r)
	-- ({16*\r},{f(0)-0.2});
	;

\arad{({1*\r},-3)}{\azero}{\phizero}
\yoke{({1*\r},-3)}{\azero}{\phizero}{-\o}

\arad{({3*\r},-3)}{\aone}{\phione}
\yoke{({3*\r},-3)}{\aone}{\phione}{\o}
\arad{({5*\r},-3)}{\atwo}{\phitwo}
\yoke{({5*\r},-3)}{\atwo}{\phitwo}{-\o}
\arad{({7*\r},-3)}{\athree}{\phithree}
\yoke{({7*\r},-3)}{\athree}{\phithree}{\o}
\arad{({9*\r},-3)}{\afour}{\phifour}
\yoke{({9*\r},-3)}{\afour}{\phifour}{-\o}
\arad{({11*\r},-3)}{\afive}{\phifive}
\yoke{({11*\r},-3)}{\afive}{\phifive}{\o}
\arad{({13*\r},-3)}{\asix}{\phisix}
\yoke{({13*\r},-3)}{\asix}{\phisix}{-\o}
\arad{({15*\r},-3)}{\aseven}{\phiseven}
\yoke{({15*\r},-3)}{\aseven}{\phiseven}{\o}

\begin{scope}[xshift=10.2cm]
	\begin{scope}
	\clip (0,-2) rectangle (3,2);
	\draw[color=red,line width=1.4pt] plot[domain=0:180,samples=100]
		({\x/60},{f(\x)});
	\end{scope}
	\draw[->] (0,-2) -- (0,2) coordinate[label={right:$R(t)$}];
	\draw[->] (0,0) -- (3.2,0) coordinate[label={$t$}];
	\fill[color=red] (0,{f(0)}) -- ++(-0.15,0.05) -- ++(-0.6,0)
		-- ++(0,-0.1) -- ++(0.6,0) -- cycle;
	%\draw[color=red,line width=3pt] (-0.8,{f(0)}) -- (-0.1,{f(0)});
\end{scope}

\in{1}
\in{2}
\in{3}
\in{4}
\in{5}
\in{6}
\in{7}
\in{8}

\end{tikzpicture}
\end{document}

