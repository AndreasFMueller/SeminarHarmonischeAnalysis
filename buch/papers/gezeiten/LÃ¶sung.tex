%
% Lösung.tex -- Beispiel-File für Lösung
%
% (c) 2020 Prof Dr Andreas Müller, Hochschule Rapperswil
%
% !TEX root = ../../buch.tex
% !TEX encoding = UTF-8
%
\section{Lösung
	\label{gezeiten:section:Lösung}}
\rhead{Teil 3}
Die Lösung des Problems besteht aus zwei Teilen, der erste Teil besteht darin die Kurve des gemessenen Wasserstandes in Sinuswellen aufzuteilen, analog der Fourieranalyse.
Im zweiten Teil müssen lediglich die zehn Sinuswellen miteinander addiert werden, um die Gezeiten vorherzusagen.
Die im Prozess erste Maschine entwickelte Lord Kelvin erst einige Jahre später als die Maschine, welche für den zweiten Schritt benötigt wird.

\subsection{Gezeitenkurve in Sinuswellen aufteilen
	\label{gezeiten:subsection:malorum}}
Wie bereits beschrieben musste Lord Kelvin eine Maschine erfinden, welche eine Gezeitenkurve in ihre Komponentfrequenzen zerlegt.
Diese Maschine entwickelte Lord Kelvin jedoch nicht alleine, sondern er zog sich die Hilfe seines älteren Bruders James Thompson bei.
Zusammen entwickelten sie die den mechanischen Integrator.
Die Funktionsweise wird nachfolgend beschrieben, um die funktionweise besser zu verstehen, finden sie eine Skizze zum mechanischen Integrator.
Für diese Maschine benötigt er eine Scheibe, eine Welle, eine Kugel, ein Ausgabepapier für den Integral und natürlich die bestehende Wasserstandskurve.
Die Kugel wird dafür in der horizontalen Achse auf die Scheibe gelegt.
Je nach Position der Kugel, bewegt sich die kurve, langsam, schnell oder sogar rückwärts.
Die Bewegung der Kugel, wird über die Welle weitergegeben an das Ausgabepapier.
Bewegt sich die Kugel auf der Scheibe schnell, ist die Funktion des Integrals in diesem Moment steiler.
Bewegt sich die Kugel auf der Scheibe langsam, sprich befindet sich diese nahe am Zentrum der Scheibe, entsteht ein flaches Integral.
Wenn sich die Kugel exakt im Zentrum befindet, entsteht auf dem Ausgabepapier ein Wendepunkt.
Um die Position der Kugel zu bestimmen, wird mit einem Stift die Wasserstandskurve nachgefahren, wobei der Stift mit der Kugel verbunden ist.
Die vorhandene Kurve ist dabei orthogonal zum ganzen Rest der Maschine.
Das Papier mit der vorhanden Kurve, bewegt sich also von unten nach oben.

\begin{figure}
	\centering
	\includegraphics[width=\linewidth]{"papers/gezeiten/Skizze Integralmaschine"}
	\caption{Skizze mechanischer Integrator}
	\label{fig:skizze-integralmaschine}
\end{figure}

Für diesen Prozess, der Fourieranalyse, muss jedoch die  Wasserstandskurve als erstes mit der Sinuswelle mit der entsprechenden Frequenz mulitpliziert werden.
Um die Multiplikation in den Prozess zu integrieren, wird die Scheibe mit der Frequenz hin- und her-gedreht.
Somit ist die Ausgabekurve das Integral, der Mulitplikation von der Sinuswelle mit der entsprechenden Funktion und der Wasserstandskuve, und somit schon das gewünschte Resultat.
Um den Koeffizienten zu erhalten, muss lediglich das entstandene Integral durch die Gesamtzeit geteilt werden.

\subsection{Sinuswellen addieren}

Um nun die erhaltenen Sinuswellen miteinander zu addieren, zog Lord Kelvin wiederum Hilfe bei.
Er wusste von einem Gerät mit dem Namen Scotch-Yoke.
Dieser Scotch Yoke konnte eine Sinusbewegung erzeugen, sprich aus einer Kreisbewegung wird eine Kurve.
Seine Herausforderung war es jedoch, mindestens zehn Sinuskurven miteinander zu addieren.
Den Input für die Lösung bekam er einem Freund Mr. Tower, welcher er zufälligerweise im Zug traf und im das Problem schilderte.
Towers schlug im vor, eine Kette, welche um mehrer Rollen läuft, zu verwenden wie Wheatstone in seinem alphabetischen Telegrapheninstrument.
Er skizzierte so auf der Zugfahrt noch seine Gezeitenvorhersage Maschine.
Und zwar befestigte er an jedem Scotch-Yoke ein Rad.
Dort wo Whatstone eine Kette verwendete, verwendete Lord Kelvin einen Draht, welcher am Anfang befestigt ist.
Und danch durch alle Rädchen der Scotch-Yokes hindurchläuft.
Am Ende des Drahts befestigte er ein Gewicht und einen Stift.
So konnte er alle Beträge der Sinuswellen mit den unterschiedlichen Frequenzen auf einmal addieren, und erhielt so die Kurve für die Vorhersage der Gezeiten.

\begin{figure}
	\centering
	\includegraphics[width=\linewidth]{"papers/gezeiten/Skizze Integralmaschine"}
	\caption{}
	\label{fig:skizze-maschine}
\end{figure}

Wenn man nun alle relativen Beträge der unterschiedlichen Frequeznkörper kennt, kann mit dieser Maschine ziemlich einfach die Gezeitenvorhersage gemacht werden.
Dies war ein grosser Sprung der Technik.
Wenn nun Lord Kelvin oder auch jemand anders, vier Stunden den Hebel der Maschine betätigte, wurde die Gezeitenvorhersage für ein ganzes Jahr mechanisch gerechnet.
Nun konnte mit einer Scotch-Yoke-Riemenmaschine, viel einfach die Gezeitenvohersage gemacht werden.

Diese Präzisionsmaschinen wurden bis in die 1960er Jahre verwendet, und fanden auch im zweiten Weltkrieg grossen Anklang.
Später wurden die Maschinen mit Frequenzkomponenten erweitert.

