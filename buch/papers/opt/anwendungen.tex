%
% anwendungen.tex -- Paper zum Thema Optische Fouriertransformation <opt>
%
% (c) 2023 Marco Niederberger, Yanick Schoch; OST Ostschweizer Fachhochschule
%
% !TEX root = ../../buch.tex
% !TEX encoding = UTF-8
%
\section{Anwendungen 
\label{opt:section:anwendungen}}
\rhead{Praktische Anwendungen}


\subsection{Mustererkennung}
Die schnelle Erstellung einer Fouriertransformation kann für die Erkennung von vordefinierten Mustern verwendet werden.
\opttodo{Add skizze}
Dabei wird das zu untersuchende Muster auf die Bildebene der ersten Linse gelegt \opttodo{Ref zu Bild}.
In der Fourierebene wird dann die bekannte Fouriertransformation des gesuchten Musters montiert.
Diese Maske ist bei den gewünschten Frequenzanteilen durchlässig, an allen anderen Orten lichtundurchlässig.
Mittels einer einfachen Helligkeitsmessung mithilfe einer Photodiode kann anschliessend auf das gesuchte Muster geschlossen werden.
\cite{opt:YT:PatternRecognition}
Die Geschwindigkeit hierbei ist nicht mehr durch eine elektronische Schaltung gegeben.
Einzig limitierend ist die Geschwindigkeit, mit der das Licht durch den Versuchsaufbau gelangt plus die Anstiegszeit der Photodiode.
Abgeschätzt mit einer Distanz von 20 cm und einer Anstiegszeit von 100 ps liegt die totale Zeit pro Erkennung bei rund 900 ps.
Dies entspricht einer Frequenz im Bereich von 1 GHz, mit welcher Muster erkennt werden können.
\opttodo{Add sketch with character A and B after fouriermask}

\subsection{Diffractive deep neural network}
Beim obenstehenden Beispiel wurde nur mit einer einzelnen Blende gearbeitet, um das gesuchte Muster zu erkennen.
In \cite{opt:Lin.2018} wurde dieser Ansatz erweitert und mit mehreren Ebenen erfolgreich getestet.
Der Grundsatz bleibt gleich; koheräntes Licht wird von einer ersten Input Ebene gebrochen und anschliessend an fünf weiteren Ebenen.
Dies entspricht im Ansatz den verschiedenen Verknüpften Neutronen eines neuronalen Netzwerkes.
Nach den Beugungsebenen wird das Licht durch mehrere Detektoren erkannt.
Xing et al. konnten somit erfolgreich handschriftliche Zahlen detektieren.
Mittels fünf aufeinanderfolgenden Ebenen und zehn Detektoren gelang es ihnen, mehr als 90\% der Schriften korrekt zuzuordnen.

\subsection{Aktuelle Forschung im Bereich Photonics}
\opttodo{Add infos from OST - Photonics}

\subsection{Fazit}
Die obenstehenden Beispiele zeigen, was mit optischer Fouriertransformation möglich ist.
Mittels Verzicht auf die elektronische Berechnung der Fouriertransformation kann viel Zeit und Rechenleistung 
eingespart werden, was diese Art der Rechnung vor allem im Bereich der neuronalen Netzwerke interessant macht.
