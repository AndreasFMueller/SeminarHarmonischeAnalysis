%
% filterarten.tex -- Bild zum Thema Optische Fouriertransformation <opt>
%
% (c) 2023 Marco Niederberger, Yanick Schoch; OST Ostschweizer Fachhochschule
%

\documentclass[tikz]{standalone}
\def\skala{0.25}

%% Create a filter with a custom image on top
\newcommand{\filterBackground}[1]{%
    \begin{tikzpicture}[>=latex,thick,scale=\skala]
        \draw[draw=none] circle (7); %Make drawing symmetric

        #1 % Custom per each filter
        
        % x and y axis
        \draw[->, dashed] (-5,0)--(5,0) node[right]{$\omega_x$};
        \draw[->, dashed] (0,-5)--(0,5) node[right]{$\omega_y$};

    \end{tikzpicture}
}

\begin{document}
    % Lowpass
    \filterBackground{
        \draw[draw=none, fill=lightgray] (-4.2,-4.2) rectangle (4.2,4.2) (0,0) circle (2);
        \draw[draw] circle (2);
    }

    % Highpass
    \filterBackground{
        \draw[draw,fill=lightgray,] circle (2);
        \node[label={45:{$\omega_c$}}, circle, fill, inner sep=1pt] at (2, 0) {};
    }

    % Bandpass
    \filterBackground{
        \draw[draw=none, fill = lightgray] (-4.2,-4.2) rectangle (4.2,4.2) (0,0) circle (2);
        \draw[draw] circle (2);
        \draw[draw, fill = lightgray] circle (1);
    }

    % Bandstop
    \filterBackground{
        \path [draw,fill=lightgray, even odd rule] circle (2) circle (1);
        \node[label={45:{$\omega_c$}}, circle, fill, inner sep=1pt] at (2, 0) {};
    }
\end{document}
