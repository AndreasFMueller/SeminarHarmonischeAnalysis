%
% jwst_sechseck.tex -- Bild zum Thema Optische Fourier-Transformation <opt>
%
% (c) 2023 Marco Niederberger, Yanick Schoch; OST Ostschweizer Fachhochschule
%

\documentclass[tikz]{standalone}
\usepackage{times}
\usepackage{txfonts}
\def\skala{1}

%
% opt_common.tex -- Commands and color definition for the paper <opt>
%
% (c) 2023 Marco Niederberger, Yanick Schoch; OST Ostschweizer Fachhochschule
%

%%% NEW COMMANDS %%%

% Lense (x, height, curvature)
\newcommand{\lense}[3]{
    \def\curvature{0.2}
    \path[fill=glass, draw=black, line width = 0.6, opacity=0.8] (#1,-#2) .. controls (#1 - #3,0) .. (#1,#2) .. controls (#1 + #3,0) .. (#1,-#2);
}

% Dimension arrow (xStart, xEnd, yHeight, text)
\newcommand{\optMeasurement}[4]{%
    \draw[<->] (#1, #3)--(#2, #3) node[above,midway] {#4};
}

% Annotated point
\newcommand{\point}[3]{
    \draw[fill=black] (#1) circle (1pt) node[#3] {#2};
}

%%% COLORS %%%

% Define Color
\definecolor{glass}{cmyk}{0.2,0,0,0}
\colorlet{optBlue}{blue!70!black}
\colorlet{optRed}{red!90!black}

%%% STYLES %%%

% Laser rays
\tikzset{red ray/.style = {optRed, line width = 0.6}}
\tikzset{ray arrow/.style = {red ray, postaction=decorate,decoration={markings,mark=at position 0.52 with \arrow{stealth}}}}


\begin{document}

\definecolor{jwst}{rgb}{241,207,0}
\tikzstyle{spikes}=[dashed, optBlue]
\tikzstyle{stud}=[black, line width=4]

\def\size{3}
\def\offset{0.6}
\def\mirror{0.3}
\def\angleMarking{0.8}

\begin{tikzpicture}[>=latex,thick,scale=\skala]
    \draw[draw=none] circle (\size + \offset); %Make drawing symmetric

    \draw[fill=lightgray] (0:\size)--(60:\size)--(120:\size)--(180:\size)--(240:\size)--(300:\size)--(360:\size);

    \draw[spikes] (30:\size + \offset)--(210:\size + \offset);
    \draw[spikes] (90:\size + \offset)--(270:\size + \offset);
    \draw[spikes] (150:\size + \offset)--(330:\size + \offset);

\end{tikzpicture}

\begin{tikzpicture}[>=latex,thick,scale=\skala]
    \draw[draw=none] circle (\size + \offset); %Make drawing symmetric
   
    \draw[fill=lightgray] (0:\size)--(60:\size)--(120:\size)--(180:\size)--(240:\size)--(300:\size)--(360:\size);
    \draw[fill=black, opacity=1] (0:\mirror)--(60:\mirror)--(120:\mirror)--(180:\mirror)--(240:\mirror)--(300:\mirror)--(360:\mirror);

    \draw[stud] (0:0)--(0, {cos(30)*\size});
    \draw[stud] (0:0)--(240:\size);
    \draw[stud] (0:0)--(300:\size);

    \draw[spikes] (30:\size + \offset)--(210:\size + \offset);
    \draw[spikes] (150:\size + \offset)--(330:\size + \offset);
    
    \draw[spikes, optRed] (0:\size + \offset)--(180:\size + \offset);

    \draw[black] (150:\angleMarking) arc (150:240:\angleMarking);
    \draw[black] (300:\angleMarking) arc (300:390:\angleMarking);

    \node[circle,fill,inner sep=1.5pt] at (195:0.6*\angleMarking) {};
    \node[circle,fill,inner sep=1.5pt] at (345:0.6*\angleMarking) {};

\end{tikzpicture}

\end{document}
