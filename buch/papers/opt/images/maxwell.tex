%
% maxwell.tex -- Bild zum Thema Optische Fourier-Transformation <opt>
%
% (c) 2023 Marco Niederberger, Yanick Schoch; OST Ostschweizer Fachhochschule
%

\documentclass[tikz]{standalone}
\usepackage{times}
\usepackage{txfonts}
\usepackage{tikz,tikz-3dplot}
\usepackage{pgfplots}
\usetikzlibrary{calc}

% \pgfplotsset{compat=1.16}
\def\skala{1}

%
% opt_common.tex -- Commands and color definition for the paper <opt>
%
% (c) 2023 Marco Niederberger, Yanick Schoch; OST Ostschweizer Fachhochschule
%

%%% NEW COMMANDS %%%

% Lense (x, height, curvature)
\newcommand{\lense}[3]{
    \def\curvature{0.2}
    \path[fill=glass, draw=black, line width = 0.6, opacity=0.8] (#1,-#2) .. controls (#1 - #3,0) .. (#1,#2) .. controls (#1 + #3,0) .. (#1,-#2);
}

% Dimension arrow (xStart, xEnd, yHeight, text)
\newcommand{\optMeasurement}[4]{%
    \draw[<->] (#1, #3)--(#2, #3) node[above,midway] {#4};
}

% Annotated point
\newcommand{\point}[3]{
    \draw[fill=black] (#1) circle (1pt) node[#3] {#2};
}

%%% COLORS %%%

% Define Color
\definecolor{glass}{cmyk}{0.2,0,0,0}
\colorlet{optBlue}{blue!70!black}
\colorlet{optRed}{red!90!black}

%%% STYLES %%%

% Laser rays
\tikzset{red ray/.style = {optRed, line width = 0.6}}
\tikzset{ray arrow/.style = {red ray, postaction=decorate,decoration={markings,mark=at position 0.52 with \arrow{stealth}}}}


% \usetikzlibrary{arrows,intersections,math}
\usetikzlibrary{decorations.markings, } %calc

\begin{document}
\tdplotsetmaincoords{70}{125}

\begin{tikzpicture}[>=latex,thick,scale=\skala,tdplot_main_coords]
    % Die Breite des generierte PDFs berechnet sich aus b = cos(125 - 90) * (3.5 + 3.5) = 57.34mm
    \draw[draw=none](0,-3.5,0)--(0,3.5,0);

    % Koordinaten
    \draw[->, thin] (-2,0,0) -- (3,0,0) node[above]{$x$};
    \draw[->, thin] (0,-2,0) -- (0,3,0) node[above]{$y$};
    \draw[->, thin] (0,0,-2.8) -- (0,0,2.8) node[left]{$z$};

    % Elektrisches Feld
    \foreach \z in {-1, 0, 1} {
        \foreach \angleTheta in {0, 45, 90, 135, 180, -135, -90, -45} {
            \tdplotsetcoord{P}{1.4}{90}{\angleTheta}
            \draw[->, thin, optGreen] (0,0,\z) -- ($(P)+(0,0,\z)$);
        }
    }

    \node[optGreen] at (1,-1.4,1){$\vec{E}$};

    % Ladung
    \draw[line width=2, gray] (0,0,-2) -- (0,0,2);

    % Zylinder
    \path (1,0,0);
    \pgfgetlastxy{\cylxx}{\cylxy}
    \path (0,1,0);
    \pgfgetlastxy{\cylyx}{\cylyy}
    \path (0,0,1);
    \pgfgetlastxy{\cylzx}{\cylzy}
    \pgfmathsetmacro{\cylt}{(\cylzy * \cylyx - \cylzx * \cylyy)/ (\cylzy * \cylxx - \cylzx * \cylxy)}
    \pgfmathsetmacro{\ang}{atan(\cylt)}
    \pgfmathsetmacro{\ct}{1/sqrt(1 + (\cylt)^2)}
    \pgfmathsetmacro{\st}{\cylt * \ct}
    \fill[optBlue,opacity=0.3] (\ct,\st,2) -- ++(0,0,-4) arc[start angle=\ang,delta angle=180,radius=1] -- ++(0,0,4) arc[start angle=\ang+180,delta angle=-180,radius=1];
    \draw (0,0,2) circle[radius=1];
    \draw (\ct,\st,2) -- ++(0,0,-4);
    \draw (-\ct,-\st,2) -- ++(0,0,-4);

    \foreach \height in {-1, 0, 1} {
        \draw[opacity=0.5, thin] (\ct,\st,\height) arc[start angle=\ang,delta angle=180,radius=1];
        \draw[dashed, opacity=0.5, thin] (\ct,\st,\height) arc[start angle=\ang,delta angle=-180,radius=1];
    }

    \draw (\ct,\st,-2) arc[start angle=\ang,delta angle=180,radius=1];
    \draw[dashed] (\ct,\st,-2) arc[start angle=\ang,delta angle=-180,radius=1];

    % Messungen
    \draw[->] (0,0,2) -- (0,1,2) node[anchor=south, midway]{$r$};
    \draw[<->] (0,2,-2) -- (0,2,2) node[anchor=south west, midway]{$a$};
    \draw[-,thin, dashed] (0,1,2)--(0,2.3,2);
    \draw[-,thin, dashed] (0,0,-2)--(0,2.3,-2);
\end{tikzpicture}
\end{document}

