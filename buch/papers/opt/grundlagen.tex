%
% grundlagen.tex -- Paper zum Thema Optische Fouriertransformation <opt>
%
% (c) 2023 Marco Niederberger, Yanick Schoch; OST Ostschweizer Fachhochschule
%
% !TEX root = ../../buch.tex
% !TEX encoding = UTF-8
%
\section{Grundlagen\label{opt:section:grundlagen}}
\rhead{Von der Beugung zu Fourier}

\subsection{Grundlagen Wellentheorie}
Beugung kurz erklärt

\subsubsection{Fressnel / Frauhofner}
Was ist der Unterschied zwischen den beiden Approximationen?

\subsection{Herleitung}
Betrachten wir nun wie in TODO gezeigt eine unendlich dünne zylindrische Lichtquelle, welche sich axial in beide Richtingen unendlich weit erstreckt.
Allerdings muss die Quelle nicht zwingend als Lichtquelle betrachtet werden, sondern kann als eine generelle elektromagentische Quelle angesehen werden.
Dadurch kann durch anwenden der ersten Maxwellschen Gleichung
\begin{equation}
\oint_{S=\partial V} \varepsilon\vec{E}\, d\vec{S}
=
\int_{V}\rho\, dV
\end{equation}
die Elektrischefeldstärke $\vec{E}$ an jedem beliebigen Punkt in abhängigkeit des radialen Abstandes $r$ berechnet werden.
Angewendet auf die gegebene Geometrie des Zylindermantels lässt sich diese Gleichung mittels $dS = r d\varphi d\l$ als
\begin{align}
\int_{0}^{a}\int_{0}^{2\pi} \varepsilon Er\, d\varphi d\l
&=
Q
\\
2\pi r\varepsilon aE
&=
Q
\end{align}
schreiben.
Die Deckflächen konnten aufgrund der infiniten Länge des Zylinders vernachlässigt werden.
Nach der Elektrischenfeldstärke umgeformt lautet die Gleichung
\begin{equation}
E(r)
=
\frac{Q}{2\pi\varepsilon a} \cdot \frac{1}{r}
=
C \cdot \frac{1}{r}
\end{equation}
, wobei der konstante Anteil als $C$ zusammengefasst werden kann.
