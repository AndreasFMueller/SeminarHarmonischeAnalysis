%
% grundlagen.tex -- Paper zum Thema Optische Fouriertransformation <opt>
%
% (c) 2023 Marco Niederberger, Yanick Schoch; OST Ostschweizer Fachhochschule
%
% !TEX root = ../../buch.tex
% !TEX encoding = UTF-8
%
\section{Grundlagen\label{opt:section:grundlagen}}
\rhead{Von der Beugung zu Fourier}

\subsection{Grundlagen Wellentheorie}
Beugung kurz erklärt

\subsubsection{Fressnel / Frauhofner}
Was ist der Unterschied zwischen den beiden Approximationen?

\subsection{Herleitung}
Betrachten wir nun, wie in TODO gezeigt, eine unendlich dünne linienförmige Lichtquelle, welche sich axial in beide Richtingen unendlich weit erstreckt.
Verallgemeinert kann es sich bei dieser Art von Quelle um eine beliebige elektromagentische Quelle handeln.
Somit kann durch anwenden der ersten Maxwellschen Gleichung
\begin{equation}
\oint_{S=\partial V} \varepsilon\vec{E} \cdot\, d\vec{S}
=
\int_{V}\rho\, dV
\end{equation}
die elektrische Feldstärke $\vec{E}$ an jedem beliebigen Punkt in Abhängigkeit des radialen Abstandes $r$ berechnet werden.
Angewendet auf die gegebene Geometrie des Zylindermantels lässt sich diese Gleichung mittels $dS = r d\varphi d\l$ als
\begin{align}
\int_{0}^{a}\int_{0}^{2\pi} \varepsilon E\cdot 1 \cdot r\, d\varphi d\l
&=
Q
\\
2\pi ra\varepsilon E
&=
Q
\end{align}
schreiben.
Die Deckflächen können aufgrund der infiniten Länge des Zylinders vernachlässigt werden.
Zu beachten sei zudem, dass die normierten vektoriellen Grössen $\hat{E}$ und $\hat{S}$ parallel verlaufen und sich ihr Skalarprodukt dementsprechend zu 1 vereinfacht.
Nach der elektrischen Feldstärke umgeformt lautet die Gleichung
\begin{equation}
E(r)
=
\frac{Q}{2\pi\varepsilon a} \cdot \frac{1}{r}
=
C \cdot \frac{1}{r}
\end{equation}
, wobei der konstante Anteil als $C$ zusammengefasst wurde.
Angenommen eine planare elektromagentische Welle, siehe TODO, treffe nun auf eine Blende mit einer unendlich langen und $d$ breiten Öffnung.
Wie aus dem Physik Unterricht bereits bekannt ist, wird diese Welle am Spalt gebogen und sie sich hinter der Blende in Abhängigkeit von $d$ mehr oder weniger kreisförmig ausbreitet.
Zurückzuführen ist dieses Verhalten auf das Prinzip von Huygens TODO.
Dieses Verhalten der kreisförmigen Ausbreitung kann mittels der zuvor betrachteten Linienquellen modelliert werden.
Infinit viele dieser Linienquellen seien nun nebeneinander entlang der Öffnungsbreite $d$ angereiht.
Die Auswirkung dieser Quellen kann nun an jedem beliebigen Punkt hinter der Blende durch superponierung der einzelenen Quelleneinflüsse berechnet werden.

Ein Schirm werde nun im Abstand $x_p$ hinter der Blende angebracht, an welchem die elektrische Feldstärke auf verschiedenen Höhen $y_p$ gemessen werden soll.
Siehe TODO. Aus der geometrischen Anordnung geht hervor, dass
\begin{equation}
R
=
\sqrt{x_p^2 + y_p^2}
\label{opt:equation:R}
\end{equation}
und
\begin{equation}
r
=
\sqrt{x_p^2 + (y_p^2-y)}
=
\sqrt{x_p^2 + y_p^2 + y^2 - 2y_py}
=
R \sqrt{1 + \frac{y^2}{R^2} - \frac{2y_py}{R^2}}
\label{opt:equation:r}
\end{equation}
betragen.
Im letzten Schritt konnte Gleichung~\ref{opt:equation:R} eingesetzt und $R$ ausgeklammert werden.
Wie in TODO ersichtlich sein wird, ist dieser Ausdruck aber noch zu komplex um geschickt weiterrechnen zu können.
Eine Vereinfachung, die getroffen werden kann, ist, dass der Ausdruck $\frac{y^2}{R^2}$ für grosse $R$ stärker gegen 0 konvergiert als $\frac{2y_py}{R^2}$ und dementsprechend gleich 0 gesetzt werden darf.
Durch anwenden der Binominalexpansion
\begin{equation}
(1 + \varepsilon)^n
\approx
1 + n\varepsilon
\end{equation}
, für die $\varepsilon \ll 1$ gelten muss, ist es möglich den Wurzelausdruck noch weiter zu vereinfachen.
Hierbei entspricht
\begin{equation}
\varepsilon
=
\frac{2y_py}{R^2}
.
\end{equation}
Somit kann die Bedingung für $\varepsilon$ eingehalten werden, da
\begin{equation}
2y_py
\ll
R^2
\end{equation}
und dadurch
\begin{equation}
\varepsilon
=
\frac{2y_py}{R^2}
\ll
1
\end{equation}
gegeben ist.
Gleichung~\ref{opt:equation:r} vereinfacht sich demnach näherungsweise zu
\begin{equation}
r
=
R \sqrt{1 + \frac{y^2}{R^2} - \frac{2y_py}{R^2}}
\approx
R \left(1 - \frac{2y_py}{R^2}\right)^\frac{1}{2}
\approx
R \left(1 - \frac{y_py}{R^2}\right)
.
\end{equation}

