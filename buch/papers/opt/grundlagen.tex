%
% grundlagen.tex -- Paper zum Thema Optische Fouriertransformation <opt>
%
% (c) 2023 Marco Niederberger, Yanick Schoch; OST Ostschweizer Fachhochschule
%
% !TEX root = ../../buch.tex
% !TEX encoding = UTF-8
%
\section{Grundlagen\label{opt:section:grundlagen}}
\rhead{Von der Beugung zu Fourier}

\subsection{Grundlagen Wellentheorie}
Beugung kurz erklärt

\subsubsection{Fressnel / Frauhofner}
Was ist der Unterschied zwischen den beiden Approximationen?

\subsection{Herleitung}
Betrachten wir nun wie in TODO gezeigt eine unendlich dünne zylindrische Lichtquelle, welche sich axial in beide Richtingen unendlich weit erstreckt.
Allerdings muss die Quelle nicht zwingend als Lichtquelle betrachtet werden, sondern kann als eine generelle elektromagentische Quelle angesehen werden.
Somit kann durch anwenden der ersten Maxwellschen Gleichung
\begin{equation}
\oint_{S=\partial V} \varepsilon\vec{E} \cdot\, d\vec{S}
=
\int_{V}\rho\, dV
\end{equation}
die elektrische Feldstärke $\vec{E}$ an jedem beliebigen Punkt in abhängigkeit des radialen Abstandes $r$ berechnet werden.
Angewendet auf die gegebene Geometrie des Zylindermantels lässt sich diese Gleichung mittels $dS = r d\varphi d\l$ als
\begin{align}
\int_{0}^{a}\int_{0}^{2\pi} \varepsilon E\cdot 1 \cdot r\, d\varphi d\l
&=
Q
\\
2\pi ra\varepsilon E
&=
Q
\end{align}
schreiben.
Die Deckflächen konnten aufgrund der infiniten Länge des Zylinders vernachlässigt werden.
Zu beachten sei zudem, dass die normierten vektoriellen Grössen $\hat{E}$ und $\hat{S}$ parallel sind und sich ihr Skalarprodukt dementsprechend zu 1 vereinfacht.
Nach der elektrischen Feldstärke umgeformt lautet die Gleichung
\begin{equation}
E(r)
=
\frac{Q}{2\pi\varepsilon a} \cdot \frac{1}{r}
=
C \cdot \frac{1}{r}
\end{equation}
, wobei der konstante Anteil als $C$ zusammengefasst werden kann.

Eine planare elektromagentische Welle, siehe TODO, treffe auf eine Blende mit einer unendlich langen und $d$ breiten Öffnung.
Diese Welle wird anhand dem Prinzip der Beugung abgelenkt.
Modelliert kann dieses Verhalten durch die, wie zuvor betrachteten, unendlich langen und dünnen elektromagentischen Quellen werden.
Die Auswirkung dieser Quellen, verteilt über die Öffnungsbreite $d$, kann nun an jedem beliebigen Punkt hinter der Blende durch superponierung der einzelenen Quellen berechnet werden.

Ein Schirm werde nun im Abstand $x_p$ hinter der Blende angebracht, an welchem die elektrische Feldstärke auf verschiedenen Höhen $y_p$ gemessen werden soll.
Siehe TODO.
