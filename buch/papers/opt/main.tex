%
% main.tex -- Paper zum Thema Optische Fouriertransformation <opt>
%
% (c) 2023 Marco Niederberger, Yanick Schoch; OST Ostschweizer Fachhochschule
%
% !TEX root = ../../buch.tex
% !TEX encoding = UTF-8
%

% TODO markings
\newcommand{\opttodo}[1] {\textbf{\textcolor{red}{TODO: #1}}}
\newcommand{\optrh}{\opttodo{REF HERE}}

\chapter{Optische Fouriertransformation\label{chapter:opt}}
\kopflinks{Optische Fouriertransformation}
\begin{refsection}
\chapterauthor{Marco Niederberger, Yanick Schoch}

Hier entsteht ein Paper zum Thema \emph{Optische Fouriertransformation}. Behandelt werden die Grundlagen der
Wellentheorie, ein konkretes Experiment und mögliche Anwendungen. Ausserdem wird im
Kapitel \ref{opt:section:versuch} auf das Filterdesign eingegangen.

%
% grundlagen.tex -- Paper zum Thema Optische Fouriertransformation <opt>
%
% (c) 2023 Marco Niederberger, Yanick Schoch; OST Ostschweizer Fachhochschule
%
% !TEX root = ../../buch.tex
% !TEX encoding = UTF-8
%
\section{Grundlagen\label{opt:section:grundlagen}}
\rhead{Von der Beugung zu Fourier}

\subsection{Grundlagen Wellentheorie}
Beugung kurz erklärt

\subsubsection{Fressnel / Frauhofner}
Was ist der Unterschied zwischen den beiden Approximationen?

\subsection{Herleitung}
Betrachten wir nun wie in TODO gezeigt eine unendlich dünne zylindrische Lichtquelle, welche sich axial in beide Richtingen unendlich weit erstreckt.
Allerdings muss die Quelle nicht zwingend als Lichtquelle betrachtet werden, sondern kann als eine generelle elektromagentische Quelle angesehen werden.
Dadurch kann durch anwenden der ersten Maxwellschen Gleichung
\begin{equation}
\oint_{S=\partial V} \varepsilon\vec{E}\, d\vec{S}
=
\int_{V}\rho\, dV
\end{equation}
die Elektrischefeldstärke $\vec{E}$ an jedem beliebigen Punkt in abhängigkeit des radialen Abstandes $r$ berechnet werden.
Angewendet auf die gegebene Geometrie des Zylindermantels lässt sich diese Gleichung mittels $dS = r d\varphi d\l$ als
\begin{align}
\int_{0}^{a}\int_{0}^{2\pi} \varepsilon Er\, d\varphi d\l
&=
Q
\\
2\pi r\varepsilon aE
&=
Q
\end{align}
schreiben.
Die Deckflächen konnten aufgrund der infiniten Länge des Zylinders vernachlässigt werden.
Nach der Elektrischenfeldstärke umgeformt lautet die Gleichung
\begin{equation}
E(r)
=
\frac{Q}{2\pi\varepsilon a} \cdot \frac{1}{r}
=
C \cdot \frac{1}{r}
\end{equation}
, wobei der konstante Anteil als $C$ zusammengefasst werden kann.

%
% versuch.tex -- Paper zum Thema Optische Fouriertransformation <opt>
%
% (c) 2023 Marco Niederberger, Yanick Schoch; OST Ostschweizer Fachhochschule
%
% !TEX root = ../../buch.tex
% !TEX encoding = UTF-8
%
\section{Versuch
\label{opt:section:versuch}}
\rhead{Praktische Erprobung}

\subsection{Beschreibung Versuchsaufbau}

\subsection{Simulation der erwarteten Ergebnissen}

\subsection{Durchführung der Ergebnissen}

\subsection{Vergleich der Simulation}

\subsection{Wieso mit Linsen? -> Positionsunabhängigkeit?}

%
% anwendungen.tex -- Paper zum Thema Optische Fouriertransformation <opt>
%
% (c) 2023 Marco Niederberger, Yanick Schoch; OST Ostschweizer Fachhochschule
%
% !TEX root = ../../buch.tex
% !TEX encoding = UTF-8
%
\section{Anwendungen
  \label{opt:section:anwendungen}}
\rhead{Praktische Anwendungen}

\subsection{Mustererkennung}
Die schnelle Erstellung einer Fouriertransformation kann für die Erkennung von vordefinierten Mustern verwendet werden.
Dabei wird das zu untersuchende Muster auf die Bildebene der ersten Linse gelegt. 
In Abbildung \ref{opt:fig:4fAufbau} als \emph{Originalbild} bezeichnet.
In der Fourierebene ist anschliessend die bekannte Fouriertransformation ersichtlich.
Abbildung \ref{opt:fig:patternYT} zeigt drei verschiedene solche Transformationen.
Mittels einer Maskierung der jeweiligen Transformationen und einer anschliessenden Helligkeitsmessung kann auf das gesuchte Muster geschlossen werden.
In \cite{opt:YT:PatternRecognition} wird dies anhand eines Versuches mit den Buchstaben A und B visualisiert.
Die Geschwindigkeit hierbei ist nicht mehr durch eine elektronische Schaltung gegeben.
Einzig limitierend ist die Geschwindigkeit, mit der das Licht durch den Versuchsaufbau gelangt plus die Anstiegszeit der Photodiode.
Abgeschätzt mit einer Distanz von 20 cm und einer Anstiegszeit von 100 ps liegt die totale Zeit pro Erkennung bei rund 900 ps.
Dies entspricht einer Frequenz im Bereich von 1 GHz, mit welcher Muster erkennt werden können.

\begin{figure}
    \centering
    \includegraphics[width=0.6\textwidth]{papers/opt/images/pattern_YT.png}
    \label{opt:fig:patternYT}
    \caption{Frequenzspektrum der verschiedenen Muster. 
    Mittels einer Maskierung und einer Helligkeitsmessung kann das entsprechende Muster detektiert werden.}
\end{figure}

\subsection{Diffractive deep neural network}
Beim obenstehenden Beispiel wurde nur mit einer einzelnen Blende gearbeitet, um das gesuchte Muster zu erkennen.
In \cite{opt:Lin.2018} wurde dieser Ansatz durch Xing et al. erweitert und mit mehreren Ebenen erfolgreich getestet.
Abbildung \ref{opt:fig:handwriting}a zeigt den schematische Aufbau mit fünf Ebenen.
Der Grundsatz bleibt gleich; koheräntes Licht wird von einer ersten Input Ebene gebogen und anschliessend an fünf weiteren Ebenen.
Dies entspricht im Ansatz den verschiedenen Verknüpften Neutronen eines neuronalen Netzwerkes.
Nach den Beugungsebenen wird das Licht durch mehrere Detektoren erkannt.
Xing et al. konnten somit erfolgreich handschriftliche Zahlen detektieren.
Mittels fünf aufeinanderfolgenden Ebenen und zehn Detektoren gelang es ihnen, mehr als 90\% der Schriften korrekt zuzuordnen.

\begin{figure}
    \centering
    \includegraphics[width=\textwidth]{papers/opt/images/handwriting.pdf}
    \caption{Abbildung a) zeigt den Aufbau, wie er in \cite{opt:Lin.2018} verwendet wurde, um handschriftliche Ziffern zu erkennen.
    Abbildung b) und c) zeigen den Eingang sowie das Bild auf dem Detektor des Systems}
    \label{opt:fig:handwriting}
\end{figure}

\subsection{James-Webb-Weltraumteleskop}
Eine weiteres Beispiel der Beugung befindet sich aktuell im Weltraum.
Auf den Bildern des James-Webb-Weltraumteleskop (JWST) sind jeweils sechs (beziehungsweise drei durchgehende) helle Strahlen ersichtlich.
Diese werden durch die Geometrie des Spiegels erzeugt.

Das Teleskop besteht aus einem sechseckigen Hauptspiegel mit einem Loch sowie einem vorgelagerten Spiegel.
Dieser wiederum ist mit drei Stützen mit der Struktur des JWST verbunden.
Dabei wurden zwei Stützen so platziert, dass deren Strahlen deckungsgleich mit denjenigen des Hauptspiegels sind.
Die dritte Stütze ist senkrecht montiert und somit nicht mit einem Strahl des Hauptspiegels deckungsgleich.
Dies ist auf den Bildern als vierter Strahl zu sehen.
Zusammen ergeben diese beiden Effekte in Abbildung \ref{opt:fig:jwst} die charakteristischen Strahlen, welche auf
den veröffentlichten Bildern des JWST zu sehen sind.

Im Gegensatz dazu sind beim Hubble Teleskop vier (beziehungsweise zwei durchgehende) Strahlen ersichtlich.
Dies kommt davon, dass bei diesem Teleskop der runde Spiegel mittels vier Stützen mit der restlichen Struktur verbunden ist.

\begin{figure}
    \centering

    \subfigure{
        \includegraphics[page=1, width=0.3\linewidth]{papers/opt/images/jwst_sechseck.pdf}
    }
    \hfill
    \subfigure{
        \includegraphics[page=2, width=0.3\linewidth]{papers/opt/images/jwst_sechseck.pdf}
    }
    \hfill
    \subfigure{
        \includegraphics[width=0.3\linewidth]{papers/opt/images/jamesWebb_cropped_publicDomain.png}
    }
    \caption{Links die Strahlen der sechs Aussenkanten, mittig der drei Stützen sowie rechts eine Aufnahme der NASA (Public Domain)
        des James-Webb-Weltraumteleskop mit den charakteristischen Strahlen.}
    \label{opt:fig:jwst}
\end{figure}

%
% filter.tex -- Paper zum Thema Optische Fouriertransformation <opt>
%
% (c) 2023 Marco Niederberger, Yanick Schoch; OST Ostschweizer Fachhochschule
%
% !TEX root = ../../buch.tex
% !TEX encoding = UTF-8
%
\section{Filter
\label{opt:section:filter}}
\rhead{Filterdesign}

Auch bei der optischen Fouriertransformation kann das Signal anschliessend in der Fourierebene bearbeitet werden.
Im optischen entspricht dies einer Blende, welche zwischen den beiden Linsen platziert wird. 
(siehe Abbildung \ref{opt:fig:4fAufbau})
Typische Filter in der Elektrotechnik sind Tiefpass, Hochpass, Bandpass und Bandsperre.
Deren Realisierung in der Optik ist in Abbildung \ref{opt:fig:filterarten} ersichtlich.

Im Nullpunkt der Filterebene sind die tiefen Frequenzen (DC-Anteil) und entlang der Achsen die höheren Frequenzen zu erkennen.
Somit entspricht eine lichtundurchlässige Fläche mit einem Loch in der Mitte einem Tiefpass und eine lichtundurchlässige Scheibe einem Hochpass.
Nach diesem Prinzip lassen sich auch die anderen typischen Filter realisieren.

\subsection{Vergleich mit Filter 1. / 2. Ordnung}

\subsection{Cutoff Frequenz, Abfall nach $\omega_c$}

\subsection{Wo ist welche Frequenz $Hz <=> mm$ auf der Fourierebene}
Evt Formel in Abhängigkeit von der Wellenlänge, Distanz und Brennweite

% todo: Move to \begin{subfigure} with the package subcaption
\begin{figure}
    \centering

    \subfigure{
        \includegraphics[width=0.23\linewidth]{papers/opt/images/tiefpass.pdf}
        % \caption{Tiefpass}
    }
    \hfill
    \subfigure{
        \includegraphics[width=0.23\linewidth]{papers/opt/images/hochpass.pdf}
        % \caption{Hochpass}
    }
    \hfill
    \subfigure{
        \includegraphics[width=0.23\linewidth]{papers/opt/images/bandpass.pdf}
        % \caption{Bandpass}
    }
    \hfill
    \subfigure{
        \includegraphics[width=0.23\linewidth]{papers/opt/images/bandsperre.pdf}
        % \caption{Bandsperre}
    }
    \caption{Realisierung typischer Filter als optische Systeme;
        wobei grau lichtundurchlässig und weiss lichtdurchlässig bedeutet.}
    \label{opt:fig:filterarten}
\end{figure}


\printbibliography[heading=subbibliography]
\end{refsection}
