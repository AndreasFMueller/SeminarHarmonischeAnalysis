%
% teil2.tex -- Beispiel-File für teil2 
%
% (c) 2020 Prof Dr Andreas Müller, Hochschule Rapperswil
%
% !TEX root = ../../buch.tex
% !TEX encoding = UTF-8
%
\section{Audioanalyse
\label{sonogramm:section:teil2}}
\rhead{Audioanalyse}
In diesem Abschnitt werden wir mit Hilfe des Sonogramms ein Musikstücks
analysieren und versuchen die gespielten Noten zu rekonstruieren.

Wir nehmen dafür einen fünf Sekunden langen Abschnitt vom Lied
Dern Kala von Khruangbin.
Um uns die Arbeit zu vereinfachen, verwenden wir ein Audiosignal, in
dem nur die Gitarre zu hören ist.
Abbildung \ref{sonogramm:dernTime} zeigt das Audiosignal in der Zeit.
Man kann die Stellen, and welchen Töne gespielt wurden, gut erkennen.
Um nun die Tonhöhen zu erkennen brauchen wir das Sonogram.
Wie das Tonsystem aufgebaut ist, wird in Abschnitt \ref{autotune:section:tonsystemUndStimmung}
genauer beschrieben.
\begin{figure}
    \centering
    \includegraphics{papers/sonogramm/images/audioTimeDern.pdf}
    \caption{Audio Spur für die Tonanalyse.
    \label{sonogramm:dernTime}
    }
\end{figure}

Eine grundsätzliche Entscheidung, die wir treffen müssen
ist die Zeitliche Auflösung, die wir haben möchten, welche 
dann auch unsere Frequenzauflösung beeinflusst.
Für unser Stück haben wir den Vorteil, dass jeweils nur einzelne
Töne gespielt werden, wir müssen also nicht zwei Töne zur selben 
Zeit erkennen.
Die benötigte Zeitauflösung können wir anhand der Geschwindigkeit des Lieds abschätzen.
Diese beträgt 83 Schläge pro Minute mit einem 4/4 Takt, also jeweils alle 0.69 Sekunden ein viertel Takt.
Möchten wir sechzehntel Noten erkennen darf die STFT über maximal 0.17 Sekunden
angewendet werden.
Damit die sechzehntel Noten nicht nur in einem Segment zu sehen sind, teilen
wir die 0.17 Sekunden nochmals durch zwei.
Das liefert uns bei einer Abtastfrequenz von 44100 Hz eine ungefähre Segmentlänge von 
3800 Samples. 
Nach einem Bisschen Ausprobieren ergaben sich gute Resultate bei einer Segmentlänge
von 4096 Samples.
Um die Leckfrequenzen zu unterdrücken verwenden wir ein Hamming Fenster.
Somit resultiert das Sonogramm in Abbildung \ref{sonogramm:dernSono}.

Die gespielten Töne können wir nun anhand der Frequenzen 
der Peaks in den Segmenten errechnen.
Abbildung \ref{sonogramm:dernSonoPeaks} zeigt die gefundenen Frequenzen, und die
jeweiligen interessanten Abschnitte.
Die Frequenzen müssen wir nun noch in die Töne umrechnen, wobei wir jeweils auf den
nächsten Ton runden.
Wir machen die Annahme, dass der gespielte der Ton derjenige mit der höchsten Amplitude im Spektrum 
ist. 
Tabelle \ref{sonogramm:tabDern} zeigt den Vergleich mit den tatsächlich gespielten Tönen.
Diese lassen sich in diesem Fall tatsächlich gut ermitteln.
Sogar nur kurz gespielte Noten, wie das B4 im Abschnitt 10 
wurden korrekt erkannt.
Was wir vernachlässigt haben, ist der Rhythmus der gespielten Noten.
Um diesen herauszufinden, bräuchten wir jeweils die exakten Zeitpunkte, wo die Töne
beginnen und enden, was weitere Analysen erfordern würde.

\begin{figure}
    \centering
    \includegraphics{papers/sonogramm/images/dernSono1.pdf}
    \caption{Sonogram des Audiosignalabschnitts.
    \label{sonogramm:dernSono}
    }
\end{figure}

\begin{figure}
    \centering
    \includegraphics{papers/sonogramm/images/dernNoten.pdf}
    \caption{Gefundene Frequenzen in rot eingezeichnet. Bei gewissen 
    Segmenten werden Peaks auch bei der doppelten Frequenz gefunden.
    Dies sind die Obertöne der gespielten Gitarrensaiten.
    \label{sonogramm:dernSonoPeaks}
    }
\end{figure}
\begin{table}
    \begin{center}
    \begin{tabular}{c | c c  } 
     \hline
     Abschnitt & Gespielter Ton & Gefundener Ton \\ [0.5ex] 
     \hline
     1 & E3 & E3 \\ 
     2 & F\musSharp3 & F\musSharp3 \\
     3 & A3 & A3 \\
     4 & B3 & B3 \\
     5 & A3 & A3 \\
     6 & D4 & D4 \\
     7 & B3  & B3 \\
     8 & E4 slide F\musSharp4  & E4 - F4 - F\musSharp4 \\
     9 & B4  & B4 \\
     10 & A4 - B4 - A4   & A4 - B4 - A4 \\
     11 & F\musSharp4 slide E4 & F\musSharp4 - F4 - E4 \\
     12 & E4 - D3 - B3 - A3 & E4 - D3 - B3 - A3 \\
     \hline
    \end{tabular}
\end{center}
\caption{\label{sonogramm:tabDern} Vergleich der gespielten und gefundenen Töne im Analysierten Abschnitt von Dern Kala.}
\end{table}
\subsection{Weiter Anwendungen}
Das Sonogramm wird in vielen Bereichen der Audioanalyse eingesetzt.
So basierten zum Beispiel frühere Spracherkennung und Sprachsynthese Methoden 
aus Sonogramme.
Nicht nur gesprochene Wörter, sondern auch Tiere wie Vögel können mit Hilfe
von Sonogrammen von ihrem Gesang bestimmt werden.
Dabei wird das Sonogramm als Input eines Neuronalen Netzwerks verwendet.
Es ermöglicht mit Neuronalen Netzwerken, welche für Bilder gemacht sind
Audiosignale zu analysieren.