%
% teil1.tex -- Beispiel-File für das Paper
%
% (c) 2020 Prof Dr Andreas Müller, Hochschule Rapperswil
%
% !TEX root = ../../buch.tex
% !TEX encoding = UTF-8
%
\section{Unschärferelation
\label{sonogramm:section:teil1}}
\rhead{Unschärferelation}
Wie bereits in der Einleitung des Kapitels angedeutet, sind die zusätzlichen
zeitlichen Informationen nicht gratis. 
Mit dem Aufteilen des ursprünglichen Signals in kurze Fenster bekommen wir eine
verbesserte zeitliche Auflösung, wir verlieren jedoch an Frequenzauflösung.
% In der Theorie ist die bestmögliche Frequenzauflösung 
% \begin{equation}
%     \Delta f = \frac{1}{T}
% \end{equation}
% wobei $T$ die zeitliche Länge der Fensterfunktion ist.
Zusätzlich werden durch das abrupte Abschneiden des Signals mit dem Rechteckfenster
zusätzliche Frequenzen hinzugefügt. 
Diese Effekte hängen mit der Unschärferelation zusammen, welche in Abschnitt
\ref{buch:diskret:section:unschaerfe} genauer beschrieben ist.
%Dieser Effekt ist auch als "Frequency Leakage" bekannt.
In diesem Abschnitt werden wir diese Effekte genauer untersuchen.
\subsection{Auswirkung der Fensterfunktion}
Der Faltungssatz beschreibt 
\begin{align}
    f(t) * g(t)& \xrightarrow{\mathscr{F}} F(f)G(f),\\
    f(t) g(t)&\xrightarrow{\mathscr{F}}F(f) * G(f).
\end{align}
Das bedeutet, dass die Verwendung von Fensterfunktionen das Frequenzspektrum
des Signals mit dem Frequenzspektrum der Fensterfunktion gefaltet wird.
Die Fouriertransformation eines diskreten Rechteckfensters ist 
\begin{equation}
    W_r(f) = \frac{\sin(2 \pi f  L_w / (2 f_s))}{\sin(2 \pi f / (2 f_s))} e^{-i 2 \pi f (L_w-1)/(2f_s)}
\end{equation}
mit $f_s$ als Abtastfrequenz \cite{sonogramm:digsig}.
% \begin{figure}
%     \centering
%     \includegraphics{papers/sonogramm/images/rect_time.pdf}
%     \caption{Rechtecksfenster mit $L_w = 20$.
%     \label{sonogramm:recttime}
%     }
% \end{figure}
\begin{figure}
    \centering
    \includegraphics{papers/sonogramm/images/rect_freq.pdf}
    \caption{Frequenzspektrum eines ein Sekunden langes Rechteckfensters mit Abtastfrequenz von 20 Hz. Die Nulldurchgänge sind jeweils bei ganzzahligen Vielfachen
    von $1/T$.
    \label{sonogramm:rectfreq}
    }
\end{figure}
In Abbildung \ref{sonogramm:rectfreq} sind zwei solche Frequenzspektren mit unterschiedlichen $L_w$ abgebildet.
Ideal wäre ein einzelner Peak bei 0, dafür müsste jedoch das Fenster eine Länge von $\infty$ haben.

\subsubsection{Leck-Effekt}
Durch die Sprünge bei $n = 0$ und $n = L-1$ im Zeitbereich kommen zusätzliche Frequenzen dazu, der sogenannte
Leck-Effekt.
Diese unerwünschten Frequenzen werden durch die Faltung im Frequenzbereich zu den eigentlichen Frequenzen 
geschoben.
Abbildung \ref{sonogramm:leakageDemo} zeigt ein solches Spektrum eines Eintonsignals.
Man sieht wie die Leckfrequenzen, vor allem beim kürzeren Fenster,
sich über das ganze Spektrum verbreiten.
Schwächere Frequenzen können dabei im Spektrum von Leckfrequenzen von dominanten Frequenzen
übertönt werden.
Die schwächeren Frequenzen sind dann entweder nicht erkennbar, oder es entsteht ein Peak 
bei einer falschen Frequenz. 
Abbildung \ref{sonogramm:leakageDemo2} zeigt das Spektrum von 
\begin{equation}
    f(t) = \sin(3.45\cdot 2\pi t) + 0.3  \cos(5.7\cdot  2\pi t)
\label{sonogramm:eq:sigLeck}
\end{equation}
mit 20 Hz über eine und fünf Sekunden gesampled.
Die Frequenz des Kosinus lässt sich mit einer Sekunde Messzeit nicht korrekt bestimmen, 
da die Leck-Frequenzen vom Sinus zu dominant sind.

\subsubsection{Frequenzauflösung}
Die Breite des Hauptpeaks limitiert die Frequenzauflösung.
Bei einem Rechteckfenster ist dieser $2f_s/L_w$ Hz breit.
Sind zwei Frequenzen genügend nahe bei einander, überlappen sich die Hauptpeaks, und können
nicht mehr unterschieden werden. 
Zwei unterschiedliche Frequenzen können dadurch nur genau bestimmt werden wenn
\begin{equation}
    \Delta f \geq \frac{1}{T_w},
\end{equation}
was beim Rechteckfenster gleich der halben Hauptpeakbreite ist.
Die Annahme $\Delta f = 1/T_w$ ist jedoch mit Vorsicht zu gebrauchen, da $\Delta f$ je nach 
Phase der beiden Harmonischen grösser sein muss.
Abbildung \ref{sonogramm:freqdiffdemo} zeigt als Beispiel das Signal
\begin{equation}
    s(t) = \sin(2\pi 3.45 t ) + \sin (s\pi 4.45 t + \varphi)
    \label{sonogramm:eq:twoHarmPhi}
\end{equation}
mit $T_w = 1$ s, was eine Frequenzauflösung von maximal 1 Hz ergibt.
Je nach Phase des zweiten Sinus, können die Frequenzen unterschiedlich genau, oder sogar überhaupt nicht
bestimmt werden.
Wieso das so ist, lässt sich in der Zeit zeigen.
Werden zwei Sinus mit einem kleinen $\Delta f$ addiert entsteht eine sogenannte Schwebung,
was im Fall von \eqref{sonogramm:eq:twoHarmPhi}
\begin{equation}
    s(t) = \sin(2\pi 3.45 t ) + \sin (2\pi 4.45 t) = 2 \sin\left(2 \pi \frac{3.45 + 4.45}{2}t\right)
    \cos\left(2 \pi  \frac{3.45 - 4.45}{2} t\right),
\label{sonogramm:eq:twoHarm}
\end{equation}
ergibt.
Abbildung \ref{sonogramm:twoHarmTime} zeigt \eqref{sonogramm:eq:twoHarm} über vier Sekunden.
Die Phase $\varphi$ bewirkt eine Verschiebung von \eqref{sonogramm:eq:twoHarm} in der Zeit, 
was bedeutet, dass abhängig von $\varphi$ ein anderer Abschnitt des Signales gefenstert wird.
Im Fall von $\varphi = \pi$ ist das gefensterte Segment genau zwischen zwei Nulldurchgänge
vom Kosinus.
Dies ist Äquivalent zu einem Sinus mit einer Frequenz von 3.95 Hz, welcher mit einem 
Sinusfenster multipliziert wurde. 
Ein Sinusfenster der Länge T ist ein 2T-periodischer Sinus, welcher nach $T$ abgeschnitten wird.
Die Informationen über die eigentlichen Frequenzen müssen also in den Phasensprüngen bei den 
Nulldurchgänen des Kosinus sein
Geht $\varphi$ von $\pi$ gegen 0 kommt der Phasensprung ins Segment und die eigentlichen Frequenzen
können immer besser unterschieden werden.

Man sollte daher mit der theoretischen Frequenzauflösung nicht ans Limit gehen, wenn man
zuverlässige Resultate erzielen möchte.

\begin{figure}
    \centering
    \includegraphics{papers/sonogramm/images/RectWinHarmEx.pdf}
    \caption{Frequenzspektrum Sinus mit Frequenz 3.45 Hz, welcher mit einem T langen 
    Rechteckfenster zugeschnitten wurde.
    \label{sonogramm:leakageDemo}
    }
\end{figure}

\begin{figure}
    \centering
    \includegraphics{papers/sonogramm/images/twohamrrect.pdf}
    \caption{Frequenzspektrum des Signals \eqref{sonogramm:eq:sigLeck}.
    Durch die Leck-Frequenzen kann bei einer kurzen Messzeit die Frequenz des Kosinus 
    nicht richtig bestimmt werden.
    \label{sonogramm:leakageDemo2}
    }
\end{figure}

\begin{figure}
    \centering
    \includegraphics{papers/sonogramm/images/twoharmphasedifff.pdf}
    \caption{Frequenzspektren von \eqref{sonogramm:eq:twoHarmPhi} in abhängigkeit von $\varphi$.
    Je nach Phasenunterschied der beiden Sinuse, lassen sich die Frequenzen nicht mehr unterscheiden.
    \label{sonogramm:freqdiffdemo}
    }
\end{figure}

\begin{figure}
    \centering
    \includegraphics{papers/sonogramm/images/twoharmTime.pdf}
    \caption{Funktion in \eqref{sonogramm:eq:twoHarm} über 4 Sekunden in blau, und der resultierende
    umhüllende Kosinus in rot.
    Bei den Nulldurchgängen hat der multiplizierte Sinus der Frequenz 3.95 Hz Phasensprünge.
    \label{sonogramm:twoHarmTime}
    }
\end{figure}

\section{Fensterfunktionen}
\label{sonogramm:section:windows}
Bis jetzt haben wir uns auf das Rechteckfenster beschränkt.
Dieses eignet sich zwar gut, um die theoretischen Grundlagen zu zeigen,
es ist aber nicht immer das beste.

Wie im vorherigen Abschnitt aufgezeigt, sind die Frequenzauflösung und 
Leck-Frequenzen abhängig von der Fensterfunktion.
Durch die abrupten Sprünge im Rechteckfenster sind die Leck-Frequenzen 
über ein weites Frequenzband sehr prominent.

Um dies zu verbessern wäre es besser eine Fensterfunktion zu haben,
welche bei $n = 0$ einen Wert nahe bei 0 hat,
bis zur Mitte des Fensters auf 1 steigt um dann bis $n = L_w -1$ wieder Richtung 0 zu gehen. 
Im Frequenzspektrum soll diese Fensterfunktion wie das Rechteckfenster einen 
Peak bei $f = 0$ hat, aber mit $f \rightarrow \infty$ schneller
gegen 0 geht.
Ein Mass wie schnell das Frequenzspektrum einer Funktion gegen 0 geht, liefert
die $n$-mal steige Differenzierbarkeit der Funktion.
Ist eine Funktion $n$-mal stetig differenzierbar gilt im Frequenzspektrum
\begin{equation}
    x^{(n)}(t) \xrightarrow{\mathscr{F}} (i f)^n X(f).
\end{equation}
Da $(i f)^n X(f)$ Rücktransfomrierbar ist, gilt
\begin{equation}
    |(i f)^n X(f)| < \infty \quad \forall -\infty \leq f \leq \infty,
\end{equation}
was zur Folge hat, dass $X(f)$ bei grösser werdenden $f$ mit mindestens $1/f^n$ gegen 
0 gehen muss.

\subsection{Gauss-Fenster}
Eine bekannte $\infty$-mal stetig differenzierbare Funktion ist die gausssche Glockenkurve
\begin{equation}
    f(t) = e^{-\frac{t^2}{\alpha}},
\end{equation}
wobei $\alpha$ ein frei wählbarer Parameter ist, welcher die Breite der Glocke 
bestimmt.
Damit wir sehen, wie sich ein Gauss-Fenster auf das Signal auswirkt, brauchen wir
die Fourier-Transformation einer gaussschen Glocke.
Diese wird in Satz \ref{buch:gruppen:fourier:satz:gaussfourier} berechnet und ist mit
\begin{equation}
    W_g(f) =
    \frac{\sqrt[2]{\alpha \pi}}{2} e^{-\alpha \pi^2 f^2}
\end{equation}
wieder eine gausssche Glocke, bei welcher der Parameter $\alpha$ invertiert vorkommt.
Das bedeutet, dass eine in der Zeit schmale gausssche Glocke ein breites Frequenzspektrum hat und umgekehrt eine
breite gausssche Glocke in der Zeit ein schmales Frequenzspektrum hat.
Das deckt sich mit unseren Beobachtung bezüglich dem Zusammenhang der 
Frequenzauflösung und der Länge eines Fensters.

In Abbildung \ref{sonogramm:gausstime} sind zwei Gauss-Fenster mit unterschiedlichen 
$\alpha$ abgebildet.
Da das Fenster einer STFT eine endliche Länge hat, müssen wir die gaussschen Glocken 
mit einem Rechteckfenster zuschneiden. 
Somit wird es wieder Leckfrequenzen geben, die aber je nach $\alpha$ viel kleiner sind
als beim Rechteckfenster, wie in Abbildung \ref{sonogramm:gaussfreq} zu sehen ist.
Ausserdem sehen wir, dass die Hauptpeaks der Gauss-Fenster breiter sind als der 
Hauptpeak des Rechteckfensters.
Allgemein gilt das Rechteckfenster als optimal bezüglich der Frequenzauflösung.
Wir bezahlen also die Minimierung der Leckfrequenzen mit einer kleineren Frequenzauflösung. 
Mit einem passend gewählten $\alpha$ kann man somit die Fensterfunktion auf eine spezifische
Anwendung optimieren.

\begin{figure}
    \centering
    \includegraphics{papers/sonogramm/images/gauss_time.pdf}
    \caption{Gauss Fenster mit $L_w = 20$ mit unterschiedlichen $\alpha$.
    \label{sonogramm:gausstime}
    }
\end{figure}

\begin{figure}
    \centering
    \includegraphics{papers/sonogramm/images/gauss_freq.pdf}
    \caption{Frequenzspektrum der Gauss-Fenster von Abbildung \ref{sonogramm:gausstime}
    mit einer Abtastfrequenz von 20 Hz. Zum Vergleich ist in schwarz das Spektrum eines Rechteckfenster 
    von der selben Länge.
    von $1/T$.
    \label{sonogramm:gaussfreq}
    }
\end{figure}

Nehmen wir das Signal \eqref{sonogramm:eq:sigLeck} und wenden statt dem Rechteckfenster
ein Gauss-Fenster an, resultiert das Frequenzspektrum in Abbildung \ref{sonogramm:twoHarmGauss}.
Der Kosinus lässt sich somit beim kürzeren Signal genauer bestimmen.
$\alpha$ musste dabei sorgfältig gewählt werden, da bei einem zu kleinen $\alpha$ die 
Frequenzauflösung zu klein gewesen wäre, und bei einem zu grossen $\alpha$ die Leckfrequenzen
noch immer dominiert hätten.

\begin{figure}
    \centering
    \includegraphics{papers/sonogramm/images/twoHarmGauss.pdf}
    \caption{Frequenzspektrum des Signals \eqref{sonogramm:eq:sigLeck} mit
    einem Gauss-Fenster. Das Spektrum ist bei den Frequenzen, welche nicht im Signal vorkommen
    nun beinahe bei 0.
    \label{sonogramm:twoHarmGauss}
    }
\end{figure}

\subsection{Andere Fensterfunktionen}
Das Gauss-Fenster ist eine von vielen Fensterfunktionen. 
So gibt es zum Beispiel eine ganze Gruppe von Fensterfunktionen, die mit
\begin{equation}
    w(n) = \sum_{k=0}^{K} (-1)^k a_k \cos\left(\frac{2 \pi k n}{N}\right), \quad 0 \leq n < N
\end{equation}
beschrieben werden.
Zu diesen gehört auch das Hammingfenster mit $k = 1$, $a_0 \approx 0.54$ und $a_1 \approx 0.46$.

Eine beste Fensterfunktion gibt es nicht, denn jede hat ihre Vor- und Nachteile, welche
je nach Anwendung anders gewichtet werden.
Für interessierte Leser bietet \cite{sonogramm:wikiWin} eine 
gute Übersicht von Fensterfunktionen.

