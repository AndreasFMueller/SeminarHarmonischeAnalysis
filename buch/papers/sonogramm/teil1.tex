%
% teil1.tex -- Beispiel-File für das Paper
%
% (c) 2020 Prof Dr Andreas Müller, Hochschule Rapperswil
%
% !TEX root = ../../buch.tex
% !TEX encoding = UTF-8
%
\section{Unschärferelation
\label{sonogramm:section:teil1}}
\rhead{Unschärferelation}
Wie bereits in der Einleitung des Kapitels angedeutet sind die zusätzlichen
Zeitlichen Informationen nicht gratis. 
Mit dem aufteilen des Ursprünglichen Signals in kurze Fenster bekommen wir eine
verbesserte zeitliche Auflösung, wir verlieren jedoch an Frequenzauflösung.
In der Theorie ist die bestmögliche Frequenzauflösung 
\begin{equation}
    \delta f = \frac{1}{T}
\end{equation}
wobei $T$ die zeitliche Länge der Fensterfunktion ist.

Zusätzlich werden durch das abrupte abschneiden des Signals mit dem Rechteckfenster
zusätzliche Frequenzen hinzugefügt. 
Diese Effekte hängen mit der Unschärferelation zusammen, welche in Abschnitt
\ref{buch:diskret:section:unschaerfe} genauer beschrieben ist.
%Dieser Effekt ist auch als "Frequency Leakage" bekannt.
In diesem Abschnitt werden wir diese Effekte genauer untersuchen.
\subsection{Auswirkung der Fensterfunktion}
Der Faltungssatz beschreibt 
\begin{align}
    f(t) * g(t)& \xrightarrow{\mathscr{F}} F(f)G(f),\\
    f(t) g(t)&\xrightarrow{\mathscr{F}}F(f) * G(f).
\end{align}
Das bedeutet, das die Verwendung von Fensterfunktionen das Frequenzspektrum
des Signals mit dem Frequenzspektrum der Fensterfunktion gefaltet wird.
Die Fouriertransformation eines Rechteckfensters ist 
\begin{equation}
    W_r(f) = \frac{\sin(2 \pi f  L_w / (2 f_s))}{\sin(2 \pi f / (2 f_s))} e^{-i 2 \pi f (L_w-1)/(2f_s)}
\end{equation}
mit $f_s$ als Abtastfrequenz.

% \begin{figure}
%     \centering
%     \includegraphics{papers/sonogramm/images/rect_time.pdf}
%     \caption{Rechtecksfenster mit $L_w = 20$.
%     \label{sonogramm:recttime}
%     }
% \end{figure}

\begin{figure}
    \centering
    \includegraphics{papers/sonogramm/images/rect_freq.pdf}
    \caption{Frequenzspektrum eines Rechtecksfenster Abtastfrequenz von 20 Hz. Die Nulldurchgänge sind jeweils bei ganzzahligen Vielfachen
    von $1/T$.
    \label{sonogramm:rectfreq}
    }
\end{figure}
In Abbildung \ref{sonogramm:rectfreq} sind zwei solche Frequenzspektren mit unterschiedlichen $L_w$ abgebildet.

Ideal wäre ein einzelner Peak bei 0, dafür müsste jedoch das Fenster eine Länge von $\infty$ haben.
Durch die Sprünge bei $n = 0$ und $n = L-1$ im Zeitbereich kommen zusätzliche Frequenzen dazu, der sogenannte
Leck-Effekt.
Diese unerwünschten Frequenzen werden durch die Faltung im Frequenzbereich zu den eigentlichen Frequenzen 
geschoben.
Abbildung \ref{sonogramm:leakageDemo} zeigt ein solches Spektrum eines Eintonsignals.
Man sieht dabei sehr gut wie die Leck-Frequenzen, vor allem beim kürzeren Signal, 
sich über das ganze Spektrum verbreiten.
Schwächere Frequenzen können dabei im Spektrum von Leck-Frequenzen von dominanten Frequenzen
übertönt werden.
Die schwächeren Frequenzen sind dann entweder nicht erkennbar, oder verfälscht, wie in 
Abbildung \ref{sonogramm:leakageDemo2}


\begin{figure}
    \centering
    \includegraphics{papers/sonogramm/images/RectWinHarmEx.pdf}
    \caption{TODO
    \label{sonogramm:leakageDemo}
    }
\end{figure}

\begin{figure}
    \centering
    \includegraphics{papers/sonogramm/images/twohamrrect.pdf}
    \caption{TODO
    \label{sonogramm:leakageDemo2}
    }
\end{figure}



% \begin{figure}
%     \centering
%     \subfigure[Rechtecksfenster mit $L_w = 20$ und T = 1 Sekunde. \label{sonogramm:recttime}]
%       {\includegraphics[width=.45\linewidth]{papers/sonogramm/images/rect_time.pdf}}
%     \qquad
%     \subfigure[Frequenzspektrum des Rechtecksfenster von Abbildung \ref{sonogramm:recttime}
%          mit einer Abtastfrequenz von 20 Hz. \label{sonogramm:rectfreq}]
%       {\includegraphics[
%         width=.45\linewidth
%         % scale = 1.2,
%         % trim = 0 40 0 0, clip,
%       ]{papers/sonogramm/images/rect_freq.pdf}}
%     \caption{
%         Frequenzanalyse eines Rechteckfensters. In (b)
%         sieht man die Nulldurchgänge bei vielfachen von $1/T$.
%     }
% \end{figure}
% \begin{figure}
%     \centering
%     \begin{subfigure}
%         \centering
%         \includegraphics[width=1\linewidth]{papers/sonogramm/images/rect_time.pdf}
%         \caption{Rechtecksfenster mit $L_w = 20$ und T = 1 Sekunde.}
%         \label{sonogramm:recttime}
%     \end{subfigure}
%     \hfill
%     \begin{subfigure}
%         \centering
%         \includegraphics[width=1\linewidth]{papers/sonogramm/images/rect_freq.pdf}
%         \caption{Frequenzspektrum des Rechtecksfenster von Abbildung \ref{sonogramm:recttime}
%         mit einer Abtastfrequenz von 20 Hz.}
%         \label{sonogramm:rectfreq}
%     \end{subfigure}
%     \label{fig:three graphs}
%        \caption{Frequenzanalyse eines Rechteckfensters. In Abbildung \ref*{sonogramm:rectfreq}
%        sieht man die Nulldurchgänge bei vielfachen von $1/T$.}

% \end{figure}



\begin{figure}
    \centering
    \includegraphics{papers/sonogramm/images/gauss_time.pdf}
    \caption{Gauss mit $L_w = 20$.
    \label{sonogramm:gausstime}
    }
\end{figure}

\begin{figure}
    \centering
    \includegraphics{papers/sonogramm/images/gauss_freq.pdf}
    \caption{Frequenzspektrum des Rechtecksfenster von Abbildung \ref{sonogramm:recttime}
    mit einer Abtastfrequenz von 20 Hz. Die Nulldurchgänge sind jeweils bei ganzzahligen Vielfachen
    von $1/T$.
    \label{sonogramm:gaussfreq}
    }
\end{figure}
