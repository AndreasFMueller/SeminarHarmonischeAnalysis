%
% teil1.tex -- Beispiel-File für das Paper
%
% (c) 2020 Prof Dr Andreas Müller, Hochschule Rapperswil
%
% !TEX root = ../../buch.tex
% !TEX encoding = UTF-8
%
\section{Parameter
\label{sonogramm:section:teil1}}
\rhead{Parameter}

\subsection{Fensterlänge}
Die Maximale Frequenzauflösung, welche man mit der diskreten Fouriertransformatrin DFT erreichen
kann, ist 
\begin{equation}
    \Delta f = \frac{1}{T}
\end{equation}
wobei $T$ die Messzeit des Signales ist.
Diese Relation ist wahrscheinlich die grösste Limitierung des Sonogramms.
Mit einer kurzen Fensterlänge erreicht man zwar eine grössere zeitliche Auflösung,
man verliert jedoch an präzision bei den Frequenzen.

TODO Grafik


\subsection{Rechnerische Auflösung}
TODO ALles

\subsection{Fensterfunktion}
TODO Alles

\subsection{Überlappung}
TODO Alles


