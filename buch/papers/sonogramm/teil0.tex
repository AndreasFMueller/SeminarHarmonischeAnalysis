%
% einleitung.tex -- Beispiel-File für die Einleitung
%
% (c) 2020 Prof Dr Andreas Müller, Hochschule Rapperswil
%
% !TEX root = ../../buch.tex
% !TEX encoding = UTF-8
%
\section{Short Time Fourier Transform STFT\label{sonogramm:section:teil0}}
\rhead{Teil 0}
Mit der Short-Time Fourier Transform (STFT) kann man einen Teil der fehlenden zeitlichen Informationen zurückholen.
Die Idee ist die Fourier-Transformation nicht auf das ganze Signal anzuwenden, sondern dieses zuerst in kleinere
Abschnitte zu unterteilen, und diese dann zu transformieren.

Um das Signal zu unterteilen, werden sogenannte Fensterfunktionen $w(t)$ verwendet, mit welchen das Signal
\index{Fensterfunktion}%
$x(t)$ multipliziert wird.
Die simpelste Fensterfunktion ist das Rechteckfenster
\index{Rechteckfenster}%
\begin{equation}
    w_r(t) = 
        \begin{cases}
        1,& \text{wenn } 0 \le t < T\\
        0, & \text{sonst, }
        \end{cases}
\end{equation}
welches in diesem Kapitel verwendet wird.
Andere Fensterfunktionen und deren Effekte werden im Abschnitt \ref{sonogramm:section:windows} vorgestellt.
Verwenden wir das Rechteckfenster für das Signal $x(t)$, resultiert 
\begin{equation}
    x(t) w_r(t) = 
    \begin{cases}
    x(t),& \text{wenn } 0 \le t < T\\
    0, & \text{sonst.}
    \end{cases}
\end{equation}
Verschieben wir $w_r(t)$ in der Zeit, können wir mit 
\begin{equation}
    x(t) w_r(t-\tau) = 
    \begin{cases}
        x(t),& \text{wenn } \tau \le t < \tau + T\\
        0, & \text{sonst}
    \end{cases}
\end{equation}
ein beliebiges Zeitfenster von $x(t)$ extrahieren.

So lässt ich nun die Fourier-Transformation von einem $T$ langen Abschnitt
von $x$ zum Zeitpunkt $\tau$ mit
\begin{equation}
    \mathscr{F}(x)(\omega, \tau) = \int_{\tau}^{\tau+T} x(t) w_r(t - \tau) e^{-i \omega t} dt
\end{equation}
beschreiben.

Für diskrete Signale müssen wir auch die Fensterfunktion diskretisieren.
Die Funktionen sind nun nicht mehr abhängig von der Zeit $t$ sondern vom 
Index $n \in \mathbb{N}$.
Das Rechteckfenster der Länge $L_w$ definieren wir mit
\begin{equation}
    w_r(n) = 
    \begin{cases}
    1,& \text{wenn } 0 \le n < L_w-1\\
    0, & \text{sonst.}
    \end{cases}
\end{equation}
Die Fensterfunktion ist nun auch diskret und statt
der Fourier-Transformation wird die diskrete Fourier-Transformation (DFT)
\begin{equation}
    X(k, m) = \sum_{n = m}^{m + L - 1} x(n) w_r(n-m) e^{ -i \omega_k n}
\end{equation}
verwendet.
\subsection{Spektrogramm}

Das Spektrogramm, oder bei akustischen Signalen auch Sonogramm genannt, ist eine visuelle
\index{Spektrogramm}%
\index{Sonogramm}%
Darstellung der STFTs eines Signals. 
Dabei werden die Magnituden der einzelnen STFTs chronologisch
als Spalten eines Bildes dargestellt.
Dabei gehen Phaseninformationen verloren, was eine genaue Rekonstruktion
des Signales aus dem Spektrogramm verunmöglicht.
Es gibt auch die Möglichkeit, die Phaseninformationen in einem anderen Kanal im 
Bild zu decodieren, und somit das Spektrogramm invertierbar zu machen.
So kann man zum Beispiel ein Bild im HSV Farbraum erzeugen, wobei
\index{HSV Farbraum}%
der H-Kanal die Phase und der V-Kanal die Amplitude darstellt.

Ein schönes Beispiel ergibt sich bei einem Spektrogramm eines frequenzmodulierten 
Signals.
Bei der Frequenzmodulierung \cite{sonogramm:NAT} ist die momentane Frequenz des Trägersignales 
\begin{equation}
    x_{\text{FM}}(t) = A \cos\left( \omega_c t + k_f \int_{-\infty}^{t} m(\tau) d\tau\right)
\end{equation}
abhängig von der Amplitude des Nachrichtensignals $m(t)$.
Stellt man ein frequenzmoduliertes Signal in einem Spektrogramm dar,
sieht man $m(t)$ auf der Höhe von $\omega_c$, wie in Abbildung \ref{sonogramm:fmsono} mit 
\begin{align}
    m(t) = 1.5 \frac{\sin(23 \pi (t-0.5))}{23 \pi (t-0.5)}
\end{align}
und $\omega_c =2000\cdot 2\pi$ rad/s dargestellt ist.
\begin{figure}
    \centering
    \includegraphics{papers/sonogramm/images/fm_sono_windows.pdf}
    \caption{Spektrogramm eines mit einem Sinus cardinalis frequenzmodulierten Sinus.
\index{sinus cardinalis}%
    \label{sonogramm:fmsono}
    }
\end{figure}
