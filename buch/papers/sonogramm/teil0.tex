%
% einleitung.tex -- Beispiel-File für die Einleitung
%
% (c) 2020 Prof Dr Andreas Müller, Hochschule Rapperswil
%
% !TEX root = ../../buch.tex
% !TEX encoding = UTF-8
%
\section{Short Time Fourier Trasform STFT\label{sonogramm:section:teil0}}
\rhead{Teil 0}
Mit der Short-Time Fourier Transform kann man einen Teil der fehlenden zeitlichen Informartionen zurükhohlen.
Die Idee ist die Fouriertransfromation nicht auf das ganze Signal anzuwenden, sondern dieses zuerst in kleinere
Abschnitte zu unterteilen, und diese dann zu transformieren.

Um das Signal zu unterteilen werden sogenannte Fensterfunktionen $w(t)$ verwendet, mit welchen das Signal
$s(t)$ multipliziert wird.
Die simpelste Fensterfunktion ist das Rechteckfenster
\begin{equation}
    w_r(t) = 
        \begin{cases}
        1,& \text{wenn } 0 \le t < M\\
        0, & \text{sonst, }
        \end{cases}
\end{equation}
welches in diesem Kapitel verwendet wird.
Andere Fensterfunktionen und deren Effekte werden im Kapitel \dots vorgestellt.
Verwenden wir das Rechteckfenster für das Signal $s(t)$ resultiert 
\begin{equation}
    s(t) w_r(t) = 
    \begin{cases}
    s(t),& \text{wenn } 0 \le t < M\\
    0, & \text{sonst.}
    \end{cases}
\end{equation}
Verschieben wir $w_r(t)$ in der Zeit können wir ein Beliebiges Zeitfenster von $s(t)$
extrahieren